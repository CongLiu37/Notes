\documentclass[11pt]{article}

% \usepackage[UTF8]{ctex} % for Chinese 

\usepackage{setspace}
\usepackage[colorlinks,linkcolor=blue,anchorcolor=red,citecolor=black]{hyperref}
\usepackage{lineno}
\usepackage{booktabs}
\usepackage{graphicx}
\usepackage{float}
\usepackage{floatrow}
\usepackage{subfigure}
\usepackage{caption}
\usepackage{subcaption}
\usepackage{geometry}
\usepackage{multirow}
\usepackage{longtable}
\usepackage{lscape}
\usepackage{booktabs}
\usepackage{natbib}
\usepackage{natbibspacing}
\usepackage[toc,page]{appendix}
\usepackage{makecell}
\usepackage{amsfonts}
 \usepackage{amsmath}

\title{The melanization response in insect immunity}
\author{}
\date{}

\linespread{1.5}
\geometry{left=2cm,right=2cm,top=2cm,bottom=2cm}

\begin{document}
  \maketitle

  \linenumbers
Melanization is an immune response triggered locally in response to cuticle injury or systemically following microbial invasion. 
It is characterized by synthesis of melanin and cross-linking with molecules on microbial surfaces, resulting in killing of invaders. 
Melanization is also linked with coagulation system: coagulation initiates clotting process and melanization contributes to hardening clots (Eleftherianos and Revenis, 2011). 
Besides, it is essential for cuticle sclerotization or tanning that leads to hardening of exoskeleton by cross-linking cuticular proteins by quinones (Andersen, 2010). 

\newline

Phenoloxidase (PO) is a key enzyme in melanin synthesis. 
It mediates the oxidation of tyrosine to dihydroxyphenylalanine, and the oxidation of dihydroxyphenylalanine and dopamine to respective quinones, precursors of melanin (Vavricka \textit{et al.}, 2020). 
PO is produced as prophenoloxidase (PPO), which is converted to active PO by a clip domain serine proteinase (CLIP). 
CLIPs are specific to invertebrates and act in cascades to modulate coagulation, melanization and activation of Toll pathway that activates antimicrobial peptides (AMPs) synthesis. 
CLIPs lack one or more of the three residues (His, Asp, Ser) that form catalytic triad are non-catalytic, or called clip-domain containing serine proteinase homologs (cSPHs). 
The rest catalytic CLIPs are known as clip-domain containing serine proteinase (cSP). 

\newline

The most upstream proteinase that has been characterized in PPO activation cascades is a modular serine proteinase (ModSp) that lacks clip domain but contains other domain for interactions (Buchon \textit{et al.}, 2009; Ji \textit{et al.}, 2004; Roh \textit{et al.}, 2009; Takahashi \textit{et al.}, 2015). 
ModSps are often autoactivaed and lead to proteolytic cleavage and activation of a CLIPC, which activates a CLIPB that functions as PPO-activating proteinase (Kanost and Jiang, 2015). 
CLIP cascades controling PPO activation is regulated by serpins, a family of serine proteinase inhibitors. 

\section{Melanin biosynthesis pathways in insects}
Insect melanogenesis is initiated by hydroxylation of phenylalanine by phenylalanine 4-monooxygenase (PAH), which forms rate-limiting substrate tyrosine (Futahashi and Fujiwara, 2005; Gorman \textit{et al.}, 2007). 
The tyrosinase-like POs catalyses oxidation of tyrosine into dihydroxyphenylalanine (Dopa), and oxidation of Dopa into dopaquinone. 
With thiol compounds, dopaquinone is converted to cysteinyl and glutathionyl conjugates that mediate synthesis of cutaneous redish pigment phoemelanin. 
Without thiol compounds, dopaquinone undergoes spontaneous cyclization into dopachrome, which in turn is decarboxylated by dopachrome conversion enzyme to generate 5,6-dihyroxyindole (DHI). 
Following PO-meidated DHI oxidation, indole quinones polymerize and give rise to heteropolymer eumelanin. 
DHI-eumelanin can also be derived from dopamine produced early on decarboxylation of dopa by dopa decarboxylase (DDC). 

\section{PPO activation in model insects}
\subsection{\textit{Drosophila melanogaster}}
The infection-induced melanization in \textit{Drosophila melanogaster} requires two CLIPs: MP1 and MP2. 
The proteinase cascade for PPO activation includes MP1 and MP2, while its upstream pattern recognition receptors (PRRs) remain unclear (Tang \textit{et al.}, 2008; An \textit{et al.}, 2013). 
However, PRRs including PGRP-LE (Takehana \textit{et al.}, 2002) and GNBP3 (Matskevich \textit{et al.}, 2010) are involved in melanization without linking to MP1-MP2 module. 
Additionally, another CLIP called Hayan is a key activator of PPO in systemic wound responses (Nam \textit{et al.}, 2012). 

\subsubsection{\textit{Manduca sexta}}
Beta-glucan recognition proteins betaGRP1 and betaGRP2 trigger PPO activation (Jiang \textit{et al.}, 2004; Ma and Kanost, 2000). 
Binding of betaGRP2 recruits ModSp HP14, which is autoactivated (Wang and Jiang, 2006) and cleaves cSP proHP21 into active HP21. 
HP21 cleaves PPO-activating proteinase-2 zymogen (PAP-2) into active PAP-2, the terminal cSP in the cascade that processes PPO into PO (Wang and Jiang, 2007). 
Additionally, HP21 also cleaves PAP-3 (Gorman \textit{et al.}, 2007), which activates PPO directly (Jiang \textit{et al.}, 1998; Jiang \textit{et al.}, 2003; Jiang \textit{et al.}, 2003). 
PAP-1 is also a direct activator of PPO, but is regulated by a pathway different from HP14-HP21, but requires HP6 (Ann \textit{et al.}, 2009). 
Two cSPHs, SPH1 and SPH2, seem to be required as cofactors for PPO cleavage (Gupta \textit{et al.}, 2005; Yu \textit{et al.}, 2003). 
HP6 also controls Toll pathway by cleaving HP8 (An \textit{et al.}, 2009).

\newline

PPO cascade is subject to a positive feedback. 
PAP-1 activates HP6, hence increases PAP-1 activation (Wang and Jiang, 2008). 
PAP-3 cleaves PPO as well as SPH1, SPH2, PAP-3, and thus leading to a positive feedback loop (Wang \textit{et al.}, 2014). 
Besides, PAP-3 is targeted by several serpins including serpin 1J (Jiang \textit{et al.}, 2003), serpin-3 (Christen \textit{et al.}, 2012), serpin-6 (Wang and Jiang, 2004) and serpin-7 (Suwanchaichinda \textit{et al.}, 2013). 
Serpin-4 and -5 are also involved in regulation of PPO cascade upstream of PAPs (Tong and Kanost, 2005). 
Serpin-4 inhibits HP21, HP6 and HP1 (Tong \textit{et al.}, 2005). 
Serpin-5 inhibits HP6 and HP1 (An and Kanost, 2010). 

\subsection{\textit{Tenebrio molitor}}
In \textit{Tenebrio molitor}, PGRP-SA and GNBP1 act as upstream PRRs of PPO cascade (Park \textit{et al.}, 2006). 
They recruit an autoactivated ModSp, which cleaves downstream cSP called SAE (Kim \textit{et al.}, 2008). 
SPE activates Toll pathway and PPO, and process a precursor of cSPH1 (Kan \textit{et al.}, 2008). 
cSPH1 ligand PO to microbial surface (Zhang \textit{et al.}, 2003). 
PPO cascade is inhibited by serpin 40, serpin 55, serpin 48 (Jiang \textit{et al.}, 2009), and a melanization-inhibiting protein (MIP) inhibits melanization (Zhao \textit{et al.}, 2005).

\subsection{\textit{Anopheles gambiae}}
Complement-like thioester-containing protein 1 (TEP1) promotes melanization (Povelones \textit{et al.}, 2013) and its downstream includes CLIPA8, a cSPH cleaved during melanization response (Volz \textit{et al.}, 2006; Schnitger \textit{et al.}, 2007). 
CLIPA2 is another cSPH that inhibits melanization by controling TEP1 (Volz \textit{et al.}, 2006; Kamareddine \textit{et al.}, 2016; Yassine \textit{et al.}, 2014). 
SPCLIP1 activates TEP1 as cSPH (Povelones \textit{et al.}, 2013). 
Other cSPHs required for melanization include CLIPB17, CLIPB8, CLIPB3 and CLIPB4 (Volz \textit{et al.}, 2006). 

\newline

Serpin 2 inhibits PPO cascade by targeting several cSPs (Michel \textit{et al.}, 2005). 
One of its targets is CLIPB9, which is predicted as a PAP (An \textit{et al.}, 2011).

\subsection{\textit{Aedes aegypti}}
Tissue melanization requires two cSPs, IMP1 and CLIPB8, and is inhibited by serpin-2 (Zou \textit{et al.}, 2010). 
Hemolymph melanization requires two cSPs, IMD1 and IMD2, and is inhibited by serpin-1 (Zou \textit{et al.}, 2010). 
Additionally, modular serine protease CLSP2 also inhibits hemolymph PPO (Wang \textit{et al.}, 2015).

\section{Melanization and AMP synthesis}
There is extensive crosstalk between PO cascade and other humoral immune pathways, especially the Toll pathway. 
In \textit{Drosophila melanogaster}, this link is through Spn27A (serpin 27A). 
Toll activation requires depletion of Spn27A from Hemolymph, which activates PO cascade (De Gregorio \textit{et al.}, 2002; Ligoxygakis \textit{et al.}, 2002). 
Spn27A inhibits PO cascade by binding to MP2 (An \textit{et al.}, 2013). 
Additionally, PO cascade can be triggered by fungal receptor GNBP3 in a Toll-independent manner (Matskevich \textit{et al.}, 2010). 
In \textit{Anopheles gambiae}, upregulation of Toll leads to increased melanization, which is partially due to increased expression of TEP1 (Frolet \textit{et al.}, 2006). 
Besides, the Imd/Rel2 pathway triggered by PGRP-LC inhibits melanization, which is partially due to activation of CLIPA2 that inhibits TEP1 (Frolet \textit{et al.}, 2006; Meister \textit{et al.}, 2005). 
In \textit{Aedes aegypti}, Toll pathway activates melanization by controling expression of two cSPs (IMP1 and IMP2) and several PPO genes (Zou \textit{et al.},2010).

\newline

PO cascade and Toll can be controlled by common upstream signals. 
In \textit{Tenebrio molitor}, SPE cleaves PPO and cSPH1 (Kan \textit{et al.}, 2008), as well as Spz that activates Toll (Kim \textit{et al.}, 2008). 
In \textit{Manduca sexta}, HP6 activates cleaves Spz, resulting in Toll activation; and activates PPO by cleaving proPAP1 (Ann \textit{et al.}, 2009). 
In \textit{Bombyx mori}, serpin-5 regulates both Toll and PPO (Li \textit{et al.}, 2016). 
In \textit{Drosophila}, PGRP-LE activates Imd pathway and PPO cascade (Takehana \textit{et al.}, 2002; Takehana \textit{et al.}, 2004).

\end{document}