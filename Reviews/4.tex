\documentclass[11pt]{article}

% \usepackage[UTF8]{ctex} % for Chinese 

\usepackage{setspace}
\usepackage[colorlinks,linkcolor=blue,anchorcolor=red,citecolor=black]{hyperref}
\usepackage{lineno}
\usepackage{booktabs}
\usepackage{graphicx}
\usepackage{float}
\usepackage{floatrow}
\usepackage{subfigure}
\usepackage{caption}
\usepackage{subcaption}
\usepackage{geometry}
\usepackage{multirow}
\usepackage{longtable}
\usepackage{lscape}
\usepackage{booktabs}
\usepackage{natbib}
\usepackage{natbibspacing}
\usepackage[toc,page]{appendix}
\usepackage{makecell}
\usepackage{amsfonts}
 \usepackage{amsmath}

\title{Phagocytosis in insect immunity}
\author{}
\date{}

\linespread{1.5}
\geometry{left=2cm,right=2cm,top=2cm,bottom=2cm}

\begin{document}
  \maketitle

  \linenumbers
\section{Abbreviation}
SC: scavenger receptor.\newline
EGF: epidermal growth factor.\newline
PGRP: peptidoglycan recognition receptor.\newline
PGN: peptidoglycan.\newline
AMP: antimicrobial peptide.\newline
Dscam: Down syndrome adhesion molecular.\newline
IgSF: immunoglobulin superfamily.\newline
TEP: thioester-containing protein.\newline
Mcr: macroglobulin complement related.\newline
GEF: guanine nucleotide exchange factor.\newline
ESCRT: endosomal sorting complex required for transport.\newline 
VPS-C: vacuolar protein sorting-C.\newline


\section{Phagocytic receptors in insects}
Phagocytosis is initiated when phagocytic cell surface receptors recognize their ligands and trigger the engulfment of targets into phagosome. 
Phagocytic receptors can recognize targets directly, or recognize opsonins coating targets. 
Additionally, due to the diversity of targets for phagocytosis, there is overlap and redundancy in receptor-ligand specificities to facilitate recognition. 
It also provides evolutionary advantage as it allows recognition of pathogens that have developed mechanisms to evade detection by a particular receptor. 

\subsection{Scavenger receptors}
Scavenger receptors (SRs) are a family of structurally diversified transmembrane proteins, subdivided into nine classes (Class A-I). 
They exhibit broad ligand specificity, including both altered self and molecular patterns from invaders. 

\newline

Croquemort in \textit{Drosophila} is homolog of mammalian CD36 (France \textit{et al.}, 1996). 
It mediates phagocytosis of apoptotic cells and participates immunity against bacteria (France \textit{et al.}, 1999; Stuart \textit{et al.}, 2005). 

\newline

Class C SRs are unique to insects, with four members in \textit{Drosophila}: SR-CI, -CII, -CIII and -CIV. 
SR-CI recognizes bacteria, mediating their phagocytosis (Ramet \textit{et al.}, 2001; Ulvila \textit{et al.}, 2006).  

\newline

\textit{Drosophila} Peste is a class B SRs and homolog of mammalian CD36. 
It is involved in phagocytosis of bacteria (Philips \textit{et al.}, 2005; Agaisse \textit{et al.}, 2005).

\subsection{Nimrod receptors}
Nimrod family is characterized by epidermal growth factor (EGF)-like repeats called NIM repeats (Kurucz \textit{et al.}, 2007). 
It is divided into three groups: 
(1) draper-type, including \textit{Drosophila} nimrod A and draper; 
(2) nimrod-B type, including \textit{Drosophila} nimrod B 1-5; 
(3) nimrod-C type, including \textit{Drosophila} nimrod C 1-4 and eater. 

\newline

\textit{Drosophila} eater protein mediates phagocytosis of bacteria as a pattern recognition receptor (Ramet \textit{et al.}, 2002; Kocks \textit{et al.}, 2005; Chung and Kocks, 2011).

\newline

Nimrod C1 (NimC1) is located on hemocyte plasm membrane and bound to bacteria for phagocytosis (Kurucz \textit{et al.}, 2007).

\newline

\textit{Drosophila} draper is identified as phagocytosis receptor (Freeman \textit{et al.}, 2003).
It is involved in phagocytosis of apoptotic cells (Manaka \textit{et al.}, 2004; Kuraishi \textit{et al.}, 2009; Tung \textit{et al.}, 2013) and bacteria (Cuttell \textit{et al.}, 2008; Hashimoto \textit{et al.}, 2009).

\subsection{Peptidoglycan-recognition receptors}
Peptidoglycan-recognition receptors (PGRPs) bind to peptidoglycan (PGN), a polymer restricted to bacterial cell wall. 
There are 13 PGRP genes in \textit{Drosophila}. 
They are upstream of Toll and IMD signaling pathways that regulate the expression of antimicrobial peptides (AMPs) and other effectors. 

\newline

In \textit{Drosophila} PGRPs, there are six long (L) form proteins, four of which are located at plasm membrane (Werner \textit{et al.}, 2000). 
The remaining seven short (S) from proteins predicted to be secreted (Werner \textit{et al.}, 2000). 
Members of non-catalytic group (PGRP-SA, -SD, -LA, -LC, -LD, -LE, -LF) serve as pattern recognition receptors. 
They lack the critical cysteine residue in the enzymatic pocket of PGRP domian and are unable to degrade PGN (Mellroth \textit{et al.}, 2003). 
Catalytic PGRPs (PGRP-SC1, -SC2, -LB, -SB1, -SB2) posses amidase activity and degrade PGN (Zaidman-Remy \textit{et al.}, 2011).

\newline

PGRP-SC1a is a receptor for bacteria (Garver \textit{et al.}, 2006). 
Its catalytic activity is required for mediating phagocytosis (Koundakjian \textit{et al.}, 2004). 

\newline

PGRP-SA is a pattern recognition receptor with dual roles in \textit{Drosophila} humoral and cellular immunity. 
It activates Toll pathway and thus up-regulates drosomycin, an AMP (Michel \textit{et al.}, 2001). 
PGRP-SA is also important for phagocytosis of Gram-negative bacteria (Garver \textit{et al.}, 2006).

\newline

PGRP-LC is membrane-bound and mediates phagocytosis of Gram-negative but not Gram-positive bacteria (Ramet \textit{et al.}, 2002). 
It is also the major upstream receptor of IMD pathway (Ramet \textit{et al.}, 2002; Choe \textit{et al.}, 2002; Gottar \textit{et al.}, 2002).

\subsection{Integrins}
Integrin functions as a heterodimer of two transmembrane subunits, $\alpha$ and $\beta$ integrin. 
In \textit{Drosophila}, there are 5 genes coding $\alpha$ integrin, and 2 coding $\beta$ integrin (Brown \textit{et al.}, 2000). 

\newline

Integrin heterodimer $\alpha$PS3 and $\beta\nu$ is a receptor for bacteria and apoptotic cells (Nagaosa \textit{et al.}, 2011; Nonaka \textit{et al.}, 2013; Shiratsuchi \textit{et al.}, 2012).

\subsection{Down syndrome adhesion molecular (Dscam)}
Down syndrome adhesion molecular (Dscam) is a immunoglobulin superfamily (IgSF) in \textit{Drosophila}. 
There are four Dscam-like genes and \textit{Dscam1} is the most extensively characterized (Armitage \textit{et al.}, 2012). 
\textit{Dscam1} is arranged into clusters of variable exons (exon 4, 6, 9, 17) that are flanked by constant exons. 
Via alternative splicing, large isoform repertoires are generated for recognition of diverse ligands (Schmucker \textit{et al.}, 2000). 
\textit{Dscam1} expresses in immune competent tissues of \textit{Drosophila} and acts as phagocytosis receptor (Watson \textit{et al.}, 2005). 

\subsection{Opsonin in insect phagocytosis}
Opsonization is the process by which humoral molecules bind to pathogens and promotes phagocytosis. 
In mammals, antibodies and complement factors act as opsonins. 
Activated complement factors form covalent binding pathogens or altered self, and mark them for phagocytosis. 

\newline

Insect thioester-containing proteins (TEPs) share sequence similarity with vertebrate complement factor. 
In \textit{Drosophila}, there are six TEPs (TEPI-VI). 
The present of signal peptide indicates they are secreted proteins. 
TEPV does not seem to be expressed (Lagueux \textit{et al.}, 2000). 
TEPI-IV are closely related to mammalian complement factors as they share a CGEQ motif critical for the formation of thioester bonds with targets. 
\textit{TEPVI}, also called \textit{macroglobulin complement related} (\textit{Mcr}), lacks the critical cysteine residue in the thioester-binding site (Stroschein-Stevenson \textit{et al.}, 2006). 

\section{Regulation of signaling during phagocytosis}
Signaling from bound phagocytic receptors triggers coordinated rearrangement of the actin cytoskeleton. 
GTPase of Ras superfamily, including Rho-GTPase Cdc42, Rac1 and Rac2, are recruited to the plasma membrane. 
They are activated by binding with GTP, which is facilitated by guanine nucleotide exchange factors (GEFs); 
and inhibited by hydrolysis of GTP by guanine nucleotide disassociation inhibitors. 

\newline

\textit{Drosophila} \textit{Zir} is a Rho-GEF that interacts with Cdc42 and Rac2 to mediate larval phagocytosis (Sampson \textit{et al.}, 2012). 
Rac2 activates WAVE. 
WAVE then activates Arp 2/3 complex, which stimulates actin nucleation, the initial step for the formation of new filament structure. 
Cdc42 activates WAS(p), which activates Arp 2/3 complex. 
Cdc 42, Rac1, Rac2 and Arp 2/3 complex are all involved in phagocytosis (Agaisse \textit{et al.}, 2005; Philips \textit{et al.}, 2005; Stroschein-Stevenson \textit{et al.}, 2006; Stuart \textit{et al.}, 2005).

\section{Phagocytosome maturation}
The process of internalization of targets forms a membrane-bound vesicle, the phagosome, which contains targets for degradation. 
Phagosome formation is followed by a series of ordered fission/fusion events with components of endosomal pathway. 
This process, termed as phagocytosome maturation, produces a highly acidic and hydrolytic phagolysosome designed to destroy the targets. 
Phagocytosome maturation involves interactions with early endosomes, recycling endosomes, late endosomes and lysosomes. 
Involved proteins include Rab GTPase, phosphatidylinositol 3-kinase, vacuolar hydrion-ATPase, endosomal sorting complex required for transport (ESCRT) and vacuolar protein sorting-C (VPS-C) complex. 

\newline

Phagosome fuse with early endosome quickly (Mayorga \textit{et al.}, 1991). 
GTP ase Dynamin recruits Rab5 to newly formed phagosome (Bucci \textit{et al.}, 1992; Kinchen \textit{et al.}, 2008). 
Rab5 recruits effectors to early endosomal/phagosomal membrane, including early endosome antigen 1 (EEA1), SNARE proteins required for membrane fusion, Vps34 and Vps15 (also called p150, regulatory subunit of Vps34). 

\newline

Vps15 is a serine-threonine kinase recruiting Vps34 to early phagosome. 
Vps34 is a class III phosphatidylinositol 3-kinase (PI3-kinase) generating phosphatidylinositol-3-phosphate (PI3P) on early phagosome membrane (Vieira \textit{et al.}, 2001). 
PI3P interacts with Fab1, YOTB, Vac1 and EEA1 via their conserved FYVE domain. 
In \textit{Drosophila}, PI3-kinase 59F (Pi3K59F) is homolog of mammalian Vps34 and functions in cellular immune responses (Qin \textit{et al.}, 2008; Qin \textit{et al.}, 2011). 
Rebenosyn-5, \textit{Drosophila} homolg of EEA1, contains a FYVE domain that binds to PI3P and Rab5 on the phagosome surface, and is required for fusion of early endosomes and phagosomes (Morrison \textit{et al.}, 2008; Simonsen \textit{et al.}, 1998). 

\newline

Vaciolar hydrion-ATPase (V-ATPase) comples presents on phagosome membrane and is required for acidification of phagosomal lumen (Beyenbach and Wieczorek, 2006). 
In \textit{Drosophila}, 8 subunits of V-ATPase are important for phagocytosis (Cheng \textit{et al.}, 2005).

\newline

During phagosome maturation, multivesicular bodies (MVBs) appear within the phagosome by inward budding and scission of phagosome membrane. 
Transmembrane proteins that are destined for degradation are ubquitinated and sorted into MVBs (Lee \textit{et al.}, 2000). 

\newline

After MVB formation, phagosome transitions to late stage, characterized by acidic lumen and several molecules including lysosomal-associated membrane proteins (LAMPs) and hydrolase. 
LAMPs, \textit{e.g.} \textit{Drosophila} Lamp1 (also called CG3305), are required for the last step of phagosome maturation, the fussion of phagosome with lysosome (Huynth \textit{et al.}, 2007; Peltan \textit{et al.}, 2012).

\newline

Additional V-ATPase are acquired by late phagosomes, and the vesicles also acquire Rab GTPase Rab7, a marker of late phagosome (Desjardins \textit{et al.}, 1994). 
Rab7 recruits effectors such as Rab-interacting lysosomal protein, faciliating the movement of phagosome (Harrison \textit{et al.}, 2003; Jordens \textit{et al.}, 2001). 

\newline

VPS-C complexes interact with SNAREs and Rabs during phagosome maturation. 
There are two VPS-C complexes: CORVET and HOPS. 
CORVET interacts with Rab5-GTP and promotes early endosome/phagosome fussion. 
HOPS interacts with Rab7-GTP on late endosomes/MVBs and promotes fussion with lysosomes. 
CORVET and HOPS are composed of four shared class C subunits (Vps11, Vps16, Vps18 and Vps33) and two Rab-specific subunits. 
In \textit{Drosophila}, Vps33 and Vps16 have two homologs: car and Vps33B, Vps16A and Vps16B (Li and Blissard \textit{et al.}, 2015; Pulipparacharuvil \textit{et al.}, 2005). 
Vps16A and Vps16B are predicted to associate with HOPS compleses (Pulipparacharuvil \textit{et al.}, 2005). 
Vps16A is required for fussion of autophagosomes with lysosomes (Takats \textit{et al.}, 2015). 
Vps16B mediates phagosome to lysosome fussion (Akbar \textit{et al.}, 2011).

\newline

The final step of phagosome maturation is the formation of phagolysosome (pH about 4.5). 
Phagolysosomes are equiped with host factors that impede microbial growth while attacking and degrading pathogens. 
Cofactors of bacterial housekeeping enzymes, such as Fe$^{2+}$, Zn$^{2+}$ and Mn$^{2+}$, are removed from phagolysosome lumen by sequesteration by lactoferrin and removing by membrane-bound protein NRAMP. 
Reactive oxygen (ROS) and nitrogen (RNS) attack bactera. 
ROS is generated by membrane-bound NOX2 NADPH oxidase, which transfers electrons from cytosolic NADPH to molecular oxygen, and releases O$_2^{-}$ to phagolysosome lumen. 
Superoxide dismutase converts O$_2^{-}$ into H$_2$O$_2$, which can be converted into ROS like hypochlorous acid and chloramines. 
RNS is generated by iNOS, the enzyme catalyses the formation of nitric oxide on cytoplasmic side of phagolysosome. 
Nitric oxide dissfuses into phagolysosome lumen, where it encounters ROS and is converted into various RNS that are highly toxic to bacteria. 
Phagosomes are also equiped with onther bactericidal elements: AMPs, peptidase, lipase and hydrolyase.

\end{document}