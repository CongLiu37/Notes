\documentclass[11pt]{article}

% \usepackage[UTF8]{ctex} % for Chinese 

\usepackage{setspace}
\usepackage[colorlinks,linkcolor=blue,anchorcolor=red,citecolor=black]{hyperref}
\usepackage{lineno}
\usepackage{booktabs}
\usepackage{graphicx}
\usepackage{float}
\usepackage{floatrow}
\usepackage{subfigure}
\usepackage{caption}
\usepackage{subcaption}
\usepackage{geometry}
\usepackage{multirow}
\usepackage{longtable}
\usepackage{lscape}
\usepackage{booktabs}
\usepackage{natbib}
\usepackage{natbibspacing}
\usepackage[toc,page]{appendix}
\usepackage{makecell}
\usepackage{amsfonts}
 \usepackage{amsmath}

\title{Insect immunology and hematopoiesis}
\author{}
\date{}

\linespread{1.5}
\geometry{left=2cm,right=2cm,top=2cm,bottom=2cm}

\bibliographystyle{plain}
\bibliographystyle{unsrtnat}
\bibliographystyle{plainnat}
\bibliographystyle{dinat}
\bibliographystyle{abbrvnat}
\bibliographystyle{rusnat}
\bibliographystyle{ksfh_nat}
\setcitestyle{round}

\begin{document}
\begin{sloppypar}
  \maketitle

  \linenumbers

The most encompassing physical barrier of insects is the cuticle. 
This chitinous, hydrophobic material forms the exoskeleton, and also lines foregut, hindgut and tracheal system. 
Pathogens enter body through cuticle via wound or enzymatic digestion. 
Ingestion is another routine for pathogen entrance. 

\par

Multiple insect cells and tissues are involved in immunity. 
Hemocytes are the primary immune cells. 
They circulate with hemolymph (circulating hemocytes) or attach to tissues (sessile hemocytes). 
These cells drive cellular and humoral immunity. 
Fat body is composed of loosely associated cells that are rich in lipids and glycogen, lines the integument of hemocoel. 
It functions in energy storage and synthesis of vitellogenin precursors that are required for egg production. 
Fat body also produces antimicrobial peptide. 
Midgut mainly functions in digestion and nutrition absorption. 
It produces nitric oxide synthesis and other lytic effectors killing pathogens. 
Salivary glands are primarily involved in feeding and usually located in the anterior of thorax. 
It is involved in immunity.

\par

\section{Pattern recognition receptors (PRRs)}
Immune responses are initiated by recognition of pathogen-associated molecular patterns (PAMPs) by pattern recognition receptors (PRRs). 
Among PRR families are 
\newline
% ../OrthoFinder/OrthoFinder/Results_Aug07/Phylogenetic_Hierarchical_Orthogroups/N0.tsv
(1) PGRP: peptidoglycan recognition protein, characterized by peptidoglycan-binding domain; \newline
(2) Ig: immunoglobulin domain proteins; \newline
(3) FREP: fibrinogen-related protein, or fibrinogen domain immunolectin (FBN), contain fibrinogen-like domain; \newline
(4) TEP: thioester-containing protein; \newline
(5) betaGRP: beta-1,3-glucan recognition proteins, or Gram-negative bacterial-binding protein (GNBP); \newline
(6) Galectin: bind specifically to beta-galactoside sugars; \newline
(7) CTL: C-type lectin; \newline
(8) LRR: leucin-rich repeat containing protein; \newline
(9) DSCAM: down syndrome cell adhesion molecule; \newline
(10) Nimrod: include eater and draper in \textit{Drosophila melanogaster}; \newline
(11) ML: MD-2-like protein, or Niemann-Pick type C-2 protein, involved in recognizing lipopolysaccharide; \newline
(12) SR: scavenger receptor, include croquemort and peste in \textit{Drosophila melanogaster}; \newline
(13) Integrin.

\section{Toll signaling}
Toll pathway functions in both development and immunity. 
In immunity, Toll signaling is effective in combating Gram-positive bacteria, fungi and viruses.
Toll pathway includes
\newline
(1) SPZ: Spatzle/spaetzle, extracellular cytokine; \newline
(2) Toll: or toll-like receptor (TLR). \newline 
(3) MyD88: myeloid differentiation primary response 88; \newline
(4) Tube: or interleukin-1 receptor-associated kinase 4 (IRAK4); \newline
(5) Pelle: orthologous to human interleukin 1 receptor associated kinase 1 (IRAK1); \newline
(6) Dorsal; \newline 
(7) Dif: Dorsal-related immune factor; \newline
(8) Cactus: orthologous to human NF-kappaB inhibitor alpha (NFKBIA).

\par

In Toll signaling, SPZ is activated by cleavage. 
SPZ binds to cellular receptor Toll. 
Toll recruits MyD88, Tube and Pelle. 
Pelle acts as serine/threonine-protein kinase, phosphorylating Cactus. 
Thus, NF-kappaB transcription factor Dorsal and Dif are released from Cactus and relocated to nuclear.

\section{Imd Signaling}
Imd signaling is effective in combating Gram-negative bacteria and viruses. 
Players in Imd signaling include: 
\newline
(1) Imd: immune deficiency, or AGAP004959-PA/-PB in \textit{Anopheles gambiae} str. PEST, or receptor-interacting serine/threonine-protein kinase 1-like, or LOC5572865 in \textit{Aedes aegypti}; \newline
(2) TAK1: transforming growth factor (TGF)-beta activated kinase 1, orthologous to human mitogen-activated protein kinase kinase kinase 7 (MAP3K7); \newline
(3) Tab: TAK1-associated binding protein, or MAP3K7 binding protein; \newline
(4) IKKgamma: inhibitor of NF-kappaB (IkappaB) kinase subunit gamma, or Kenny in \textit{Drosophila melanogaster}, or NF-kappaB essential modulator (NEMO), or optineurin; \newline
(5) IKKbeta: IkappaB kinase subunit beta; \newline
(6) Fadd: fas-associated death domain, or Fas associated via death domain; \newline
(7) Dredd: death-related ced3/Nedd2-like caspase, or caspase 8, orthologous to human caspase 10; 
\newline
(8) Relish: or nuclear factor NF-kappa-B p110 subunit; \newline
(9) Diap: death-associated inhibitor of apoptosis, or inhibitor of apoptosis (Iap); \newline
(10) Effete: or ubiquitin-conjugating enzyme E2-17 kDa; \newline
(11) UEV1A: ubiquitin-conjugating enzyme variant 1A, or ubiquitin conjugating enzyme E2 variant 2; \newline
(12) Bendless: or ubiquitin-conjugating enzyme E2 N, or ubiquitin-conjugating enzyme 13 (UBC13); \newline
(13) Caspar: or fas-associated factor 1 (FAF1); \newline
(14) Hemipterous: or dual specific mitogen activated protein kinase kinase 7 (MAP2K7), or MAP2K7; \newline
(15) Basket: or stress-activated protein kinase JNK; \newline
(16) Jra: Jun-related antigen, or transcription factor AP-1, or transcription factor jun-D-like; \newline
(17) Kayak. 

\par

In Imd signaling, PRRs activate cascade composed of Imd, Fadd and Dredd. 
Dredd activates transcription factor Relish. 
Another line for Relish activation includes Imd, Tab, TAK1, IKKgamma and IKKbeta. 
Both signals require ubiquitination mediated by a protein complex including Diap, Effete, UEV1A, Bendless. 
Caspar inhibits Dredd-mediated activation of relish. 
TAK1 also activates JNK (Jun amino-terminal kinase) signaling, including Hemipterous, Basket, Jra and transcription factor Kayak. 

\section{JAK/STAT signaling}
JAK/STAT signaling functions in development and immunity. 
In immunity, it activates antimicrobial genes like nitric oxide synthase and functions in antibacterial and antiviral responses. 
JAK/STAT signaling includes 
\newline
(1) Unpaired \newline
(2) Domeless. \newline
(3) Hopscotch: orthologous to several human genes including JAK1 (Janus kinase 1) and JAK3 (Janus kinase 3). \newline
(4) Stat: signal-transducer and activator of transcription protein. \newline
(5) Socs: suppressor of cytokine signaling; \newline
(6) Pias: protein inhibitor of activated Stat as E3 SUMO-protein ligase, or suppressor of variegation 2-10 (Su(var)2-10) in \textit{Drosophila melanogaster}. 

\par

In JAK/STAT signaling, extracellular protein Unpaired activates membrane protein Domeless. 
Domeless activates Hospotch and Stat. 
Stat relocates to nuclear, acting as transcription factor. 
Socs and Pias inhibit JAK/STAT signaling. 

\section*{Phagocytosis}
Phagocytosis is a rapid progress conducted by hemocytes. 
PRRs that have been shown to be involved in phagocytosis include TEPs, Nimrods, DSCAMs, beta-integrins and PGRPs. 
The intracelular signaling in phagocytosis remains poorly understood. 
In mosquitoes, 
\newline
(1) CED2: cell death abnormality 2; \newline
(2) CED5; \newline
(3) CED6 \newline 
are involved in signaling regulate internalization of bacteria (Moita \textit{et al.}, 2005). 

\section*{Melanization}
Melanization is an enzymatic process involved in cuticle hardening, egg chorion tanning, wound healing and immunity and is mainly conducted by hemocytes. 
In immunity, melanization functions in killing bacteria, fungi, protozoa parasites, nematode worms and parasitoid wasps. 
It is manifested as a darkened proteinaceous capsule that surrounds pathogens, and kills pathogens via oxidative damage or starvation. 
Players in melanization include:
\newline
(1) PAH: phenylalanine hydroxylase, or phenylalanine 4-monooxygenase, or Henna in \textit{Drosophila melanogaster}; \newline
(2) PO: phenoloxidase, or phenol oxidase, formed via cleavage of prophenoloxidase (PPO); \newline
(3) DCE: dopachrome conversion enzyme or dopachrome decarboxylase/tautomerase, known as yellow in \textit{Drosophila melanogaster}; \newline
(4) DDC: dopa decarboxylase, aromatic-L-amino-acid decarboxylase (AADC or AAAD), tryptophan decarboxylase or 5-hydroxytryptophan decarboxylase, decarboxylates dopa into dopamine, which is oixidized into dopaminequinone by PO, and further converts into dopaminechrome non-enzymatically, and further into DHI non-enzymatically. \newline 
(5) ModSp: modular serine protease that lacks clip domain but contains other domain for interactions; \newline
(6) cSP: clip domain-containing serine protease, includes \textit{Drosophila melanogaster} 
  snake, easter, serine protease 7 (SP7), serine protease immune response integrator (spirit), persephone, spatzle-processing enzyme (SPE), Gram-positive specific serine protease (grass), melanization protease 1 (MP1), hayan, Ser7, lethal (2) k05911, 
activated by ModSp cleavage and activates PO by cleavage. \newline
(7) serpin: serine protease inhibitors. 

\par

In synthesis of melanine, PAH hydroxylates phenylalanine to tyrosine. 
PO oxidizes tyrosine into dihydroxyphenylalanine (Dopa), and further into dopaquinone. 
Dopaquinone is oxidazed into dopachrome non-enzymatically. 
DCE decarboxylates dopachrome into 5,6-dihyroxyindole (DHI). 
Another way from Dopa to DHI is: 
DDC decarboxylates dopa into dopamine, which is oixidized into dopaminequinone by PO. 
Dopaminequinone is further converted into dopaminechrome non-enzymatically, and further into DHI non-enzymatically. 
Finally, following PO-meidated DHI oxidation, indole-5,6-quinones polymerize and give rise to heteropolymer eumelanin. 
PO activity is controlled by ModSP, cSP and Serpin.


\section*{Encapsulation}
Encapsulation is a cellular immune response against pathogens that are too large to be phagocytosed. 
In encapsulation, hemocytes attach to form a capsule surrounding pathogens. 
The capsule may be melanized. 
In Lepidoptera, hemocyte adhesion is dependent on binding of integrin to specific sites defined by Arg-Gly-Asp (RGD) sequence. 

\section*{Nodulation}
Nodulation is an immune response in which hemocyte adhere to large aggregrates of bacteria and form layers, usually followed by melanization. 
Underlying molecular mechanism of nodulation remains poorly understood, but it relies on eicosanoid-based signaling and extracellular matrix-like protein Noduler. 

\section*{Lysis}
Lysis of pathogens is resulted from disruption of cellular membrane by immune effectors including 
\newline
(1) AMP: antimicrobial peptide, small secreted peptide including apisimin, attacin, cecropin, defensin, diptericin, drosocin, drosomycin, gambicin, gloverin, holitricin, jelleine, lebocin, melittin, metchnikowin, moricin, persulcatusin, ponericin, pyrrhocoricin, sapecin; \newline
(2) Lysozymes: or muramidase, or N-acetylmuramide glycanhydrolase, hydrolyze beta-1,4-glycosidic linkage between N-acetylumuramic and N-acetylglucosamine of peptidoglycan; \newline
(3) Transferrin: binds to Fe; \newline
(4) Chitinase: degrades chitin and is involved in antifungal responses.

\par

Reactive species are effect in lysis. 
Synthesis of reactive species include 
\newline
(1) DUOX: dual oxidase, generates hydrogen peroxide; \newline
(2) NOX: NADPH oxidase, generates hydrogen peroxide; \newline
(2) NOS: nitric oxide synthase, generates nitric oxide; \newline
(3) SOD: superoxide dismutase, catalyzes the dismutation (or partitioning) of the superoxide radical into ordinary molecular oxygen and hydrogen peroxide; \newline
(4) peroxidase: also peroxide reductase, peroxiredoxin, break up peroxides.  

\section*{RNA interference (RNAi)}
In RNA interference (RNAi) pathways, small RNA (sRNA) associates with Argonaute protein, forming RNA induced silencing complex (RISC). 
RISC recognizes targets by complementary bases, and silences targets in an Argonaute-mediated manner. 
RNAi functions in antiviral responses, gene expression regulation and anti-transponson responses. 
In insects, there are three RNAi pathways: micro-RNA (miRNA), small-interfering-RNA (siRNA) and piwi-interacting-RNA (piRNA). 

\par

miRNA pathway is mainly involved in gene expression regulation. 
Players in miRNA pathway include: 
\newline
(1) Drosha; \newline
(2) Pasha: partner of Dosha, or microprocessor complex subunit DGCR8. \newline
(3) Dicer 1: endoribonuclease; \newline
(4) Loquacious: or interferon-inducible doube-stranded RNA-dependent protein kinase activator A homolog, or TARBP2. \newline
(5) Argonaute 1. \newline
miRNA originates from nuclear genome, and is processed by nuclear protein Dorsha and Pasha. 
Matured miRNA relocates to cytoplasm, and is further processed by Dicer 1 and Loquacious. 
Then fully-matured miRNA is loaded to Argonaute 1.

\par 

siRNA pathway is involved in defenses against viral dsRNA and transposonal elements. 
\newline
(6) Dicer 2; \newline
(7) R2D2: or double-stranded RNA-binding protein Staufen homolog; \newline
(8) Argonaute 2. \newline
Viral dsRNA is processed by Dicer 2 and R2D2, forming siRNA. 
siRNA is loaded into Argonaute 2. 
In anti-transposonal elements, dsRNA is processed by Dicer 2 and Loquacious. 

\par

piRNA pathway is involved in defenses against transposonal element in germline. 
\newline
(9) Zucchini: or mitochondrial cardiolipin hydrolase; \newline
(10) Piwi: P-element induced wimpy testis, or Argonaute 3, or Aubergine, or Piwi-like protein Siwi; \newline
Transposon transcripts is processed by Zucchini, forming piRNA. 
piRNA is loaded into Piwi. 

\section*{Autophagy}
Autophagy is a process of degradation of intracellular materials, and is involved in elimation of intracellular bacteria and viruses. 
In \textit{Drosophila}, autophagy defenses against vesicular stomatitis virus and Rift Valley fever virus, but enhances infection of Sindbis virus. 
Major players in autophagy include: 
\newline
(1) PI3K: phosphatidylinositol 3-kinase, or phosphoinositide 3-kinase; \newline
(2) AKT: or RAC serine/threonine-protein kinase; \newline
(3) TOR: target of rapamycin, protein kinase. \newline
(4) Atg1: autophagy-related (Atg) 1, or unc-51 like autophagy activating kinase (ULK), or unc-51, a serine/threonine protein kinase; \newline
(5) Atg13: serine/threonine protein kinase regulatory subunit; \newline
(6) Atg14: or Beclin 1-associated autophagy-related key regulator; \newline
(7) Vps15: vacuolar protein sorting (Vps) 15, or phosphoinositide 3-kinase regulatory subunit 4; \newline
(8) Vps34: phosphatidylinositol 3-kinase 59F, or phosphatidylinositol 3-kinase catalytic subunit type 3; \newline
(9) Atg5; \newline
(10) Atg12; \newline
(11) Atg8: or gamma-aminobutyric acid receptor-associated protein (GABARAP). 

\par

In immunity, autophagy initiates with PI3K-AKT signaling, inactivating TOR. 
TOR inactivation activates protein complex containing Atg1 and Atg13, which leads to nucleation of autophagosomal membrane via a complex containing Atg14, Vps15 and Vps34. 
Then autophagosome is elongated, dependent on Atg5, Atg12 and Atg8.

\section*{Apoptosis}
Apoptosis is a form of programmed cell death that often functions in antiviral responses. 
Key players include: 
\newline
(1) Dronc: death regulator Nedd2-like caspase, or Nedd2-like caspase (Nc); \newline
(2) Dark: death-associated APAF1-related killer, or apoptotic protease-activating factor 1 (APAF1); \newline
(3) Drice: death related ICE-like caspase; \newline
(4) DCP1: death caspase-1. 

\par

Dronc and Dark form a protein comples, and Dronc activates downstream caspase including Drice and DCP1.

\end{sloppypar}
\end{document}