\documentclass[11pt]{article}

% \usepackage[UTF8]{ctex} % for Chinese 

\usepackage{setspace}
\usepackage[colorlinks,linkcolor=blue,anchorcolor=red,citecolor=black]{hyperref}
\usepackage{lineno}
\usepackage{booktabs}
\usepackage{graphicx}
\usepackage{float}
\usepackage{floatrow}
\usepackage{subfigure}
\usepackage{caption}
\usepackage{subcaption}
\usepackage{geometry}
\usepackage{multirow}
\usepackage{longtable}
\usepackage{lscape}
\usepackage{booktabs}
\usepackage{natbibspacing}
\usepackage[toc,page]{appendix}
\usepackage{makecell}
\usepackage{amsfonts}
 \usepackage{amsmath}

\usepackage[backend=bibtex,style=authoryear,sorting=nyt,maxnames=1]{biblatex}
\bibliography{} # Reference bib

\title{Abstracts_Reviews}
\author{}
\date{}

\linespread{1.5}
\geometry{left=2cm,right=2cm,top=2cm,bottom=2cm}

\setlength\bibitemsep{0pt}
\setcitestyle{braket}

\begin{document}
\begin{sloppypar}
  \maketitle

  \linenumbers
\textbf{1. Schmidt \textit{et al.}, 2008, Insect and veterbrate immunity: key similarities versus differences.} 
\par
\textbf{2. Strand, 2008, Insect hemocytes and their role in immunity.} 
\par
\textbf{3. Waterhouse \textit{et al.}, 2020, Characterization of insect immune systems from genomic data.}
\par
\textbf{4. Nazario-Toole \textit{et al.}, 2017, Phagocytosis in insect immunity.}
\par
\textbf{5. Nakhleh \textit{et al.}, 2017, The melanization response in insect immunity.}
\par
\textbf{6. Hillyer, 2016, Insect immunology and hematopoiesis.}
\par
\textbf{7. Gerardo\textit{et al.}, 2020, Evolution of animal immunity in the light of beneficial symbioses.}

\par

\textbf{Garcia \textit{et al.}, 2014, The symbiont side of symbiosis: do microbes really benefit?} \newline
It has been presumed that microbial symbionts receive host-derived nutrients or a competition-free environment with reduced predation, but there have been few empirical tests, or even critical assessments, of these assumptions. 
Evaluation of these hypotheses based on available evidence indicates reduced competition and predation are not universal benefits for symbionts. 
Some symbionts do receive nutrients from their host, but this has not always been linked to a corresponding increase in symbiont fitness.
\par
\textbf{Palmer \textit{et al.}, 2015, Comparative genomics reveals the origin and diversity of Arthropod immune system.} \newline
Immune gene families are searched in predicted peptide sets of arthropod species, including 
insect \textit{Drosophila melanogaster}, 
crustacean \textit{Daphnia pulex} (water flea), 
myriapod \textit{Strigamia maritima} (coastal centipede) and 
five chelicerates: \textit{Mesobuthus martensii} (Chinese scorpion), 
                   \textit{Parasteatoda tepidariorum} (house spider),
                   \textit{Ixodes scapularis} (deer tick),
                   \textit{Metaseiulus occidentalis} (western orchard predatory mite),
                   \textit{Tetranychus urticae} (red spider mite). 
Arthropod Toll-like receptors (TLRs) are a dynamically evolving gene family that includes relatives of vertebrate TLRs. 
The Toll signaling pathway is conserved across arthropods. 
The IMD signaling pathway is highly reduced in chelicerates. 
The JAK/STAT signaling pathway is highly conserved. 
$\beta$-1,3 glucan recognition proteins ($\beta$GRPs) have been lost in chelicerates. 
Arthropod TEPs include relatives of vertebrate C3 complement factors and proteins lacking the thioester motif. 
Gene duplication generates diversity in the immune receptor Down syndrome cell adhesion molecule (Dscam). 
\par
\textbf{Viljakainen, 2015, Evolutionary genetics of insect innate immunity.} \newline
Toll and Imd signaling pathways are well conserved across insects. 
Antimicrobial peptides (AMPs) are the most labile component of insect immunity showing rapid gene birth-death dynamics and lineage-specific gene families. 
Immune genes and especially recognition genes are frequently targets of positive selection driven by host-pathogen arms races. 
Homology-based annotation is useful but to some extent restricted approach to find immune-related genes in a newly sequenced genome. 
Novel immune genes have been found in many insects and should be looked for in future research.
\par
\textbf{Boehm, 2012 Evolution of vertebrate immunity.} \newline
Could it be possible then that an immune system employing structurally diversified antigen receptors facilitated increased species-richness in autochthonous microbial communities, for example, in the intestine? 
The selective advantage of increasing antigen receptor diversity with respect to the species-richness of microbiomes is illustrated by the role of secreted antibodies, such as IgA in mammals, in the maintenance of microbial homeostasis on mucosal surfaces; defective structural diversification of secreted antibodies is associated with dysbiosis, which is characterized by generally lower species diversity and an ‘unhealthy’ composition of the microbiome. 
Autoimmunity can be a price for the evolution of adaptive immunity. 
\par
\textbf{McFall-Ngai, 2007, Care for the community.} \newline
A memory-based immune system may have evolved in vertebrates because of the need to recognize and manage complex communities of beneficial microbes. 
Invertebrates are no less challenged by the microbial world than vertebrates, nor are they less able to remain healthy by entirely relying on innate immunity. 
Invertebrates often harbor much less diversified symbiont communities compared with vertebrates. 
There are three possible strategies for management of symbionts in invertebrates: 
maintain symbionts intracellularly; 
build physical barrier between host tissue and symbionts; 
express a high number of specific recognition components of immate immunity. 
\par
\textbf{Hoang & King, 2022, Symbiont-mediated immune priming in animals through an evolutionary lens.} \newline
While research on symbiont-mediated immune priming (SMIP) has focused on ecological impacts and agriculturally important organisms, the evolutionary implications of SMIP are less clear. 
Here, we review recent advances made in elucidating the ecological and molecular mechanisms by which SMIP occurs. 
We draw on current works to discuss the potential for this phenomenon to drive host, parasite, and symbiont evolution. 
We also suggest approaches that can be used to address questions regarding the impact of immune priming on host-microbe dynamics and population structures. 
Finally, due to the transient nature of some symbionts involved in SMIP, we discuss what it means to be a protective symbiont from ecological and evolutionary perspectives and how such interactions can affect long-term persistence of the symbiosis. 
\par
\textbf{Sharp & Hoster, 2022, Host control and the evolution of cooperation in host microbiomes.} \newline
It is often suggested that the mutual benefits of host-microbe relationships can alone explain cooperative evolution. 
Here, we evaluate this hypothesis with evolutionary modelling. 
Our model predicts that mutual benefits are insufficient to drive cooperation in systems like the human microbiome, because of competition between symbionts. 
However, cooperation can emerge if hosts can exert control over symbionts, so long as there are constraints that limit symbiont counter evolution. 
We test our model with genomic data of two bacterial traits monitored by animal immune systems. 
In both cases, bacteria have evolved as predicted under host control, tending to lose flagella and maintain butyrate production when host-associated. 
Moreover, an analysis of bacteria that retain flagella supports the evolution of host control, via toll-like receptor 5, which limits symbiont counter evolution. 
Our work puts host control mechanisms, including the immune system, at the centre of microbiome evolution. 
\par
\textbf{Costello \textit{et al.}, 2012, The application of ecological theory toward an understanding of the human microbiome.} \newline
Review of three core scenarios of human microbiome assembly: 
development in infants, representing assembly in previously unoccupied habitats; 
recovery from antibiotics, representing assembly after disturbance; 
and invasion by pathogens, representing assembly in the context of invasive species. 
\par
\textbf{Hansen & Moran, 2013, The impact of microbial symbionts on host plant utilization by herbivorous insects.} \newline
Herbivory, defined as feeding on live plant tissues, is characteristic of highly successful and diverse groups of insects and represents an evolutionarily derived mode of feeding. Plants present various nutritional and defensive barriers against herbivory; nevertheless, insects have evolved a diverse array of mechanisms that enable them to feed and develop on live plant tissues. For decades, it has been suggested that insect-associated microbes may facilitate host plant use, and new molecular methodologies offer the possibility to elucidate such roles. Based on genomic data, specialized feeding on phloem and xylem sap is highly dependent on nutrient provisioning by intracellular symbionts, as exemplified by Buchnera in aphids, although it is unclear whether such symbionts play a substantive role in host plant specificity of their hosts. Microorganisms present in the gut or outside the insect body could provide more functions including digestion of plant polymers and detoxification of plant-produced toxins. However, the extent of such contributions to insect herbivory remains unclear. We propose that the potential functions of microbial symbionts in facilitating or restricting the use of host plants are constrained by their location (intracellular, gut or environmental), and by the fidelity of their associations with insect host lineages. Studies in the next decade, using molecular methods from environmental microbiology and genomics, will provide a more comprehensive picture of the role of microbial symbionts in insect herbivory.
\par
\textbf{Zaidman-Rémy \textit{et al.}, 2018, What can a weevil teach a fly, and reciprocally? Interaction of host immune systems with endosymbionts in \textit{Glossina} and \textit{Sitophilus}} \newline

\end{sloppypar}
\end{document}