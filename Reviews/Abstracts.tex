\documentclass[11pt]{article}

% \usepackage[UTF8]{ctex} % for Chinese 

\usepackage{setspace}
\usepackage[colorlinks,linkcolor=blue,anchorcolor=red,citecolor=black]{hyperref}
\usepackage{lineno}
\usepackage{booktabs}
\usepackage{graphicx}
\usepackage{float}
\usepackage{floatrow}
\usepackage{subfigure}
\usepackage{caption}
\usepackage{subcaption}
\usepackage{geometry}
\usepackage{multirow}
\usepackage{longtable}
\usepackage{lscape}
\usepackage{booktabs}
\usepackage{natbibspacing}
\usepackage[toc,page]{appendix}
\usepackage{makecell}
\usepackage{amsfonts}
 \usepackage{amsmath}

\usepackage[backend=bibtex,style=authoryear,sorting=nyt,maxnames=1]{biblatex}
\bibliography{} # Reference bib

\title{Abstracts_Reviews}
\author{}
\date{}

\linespread{1.5}
\geometry{left=2cm,right=2cm,top=2cm,bottom=2cm}

\setlength\bibitemsep{0pt}
\setcitestyle{braket}

\begin{document}
\begin{sloppypar}
\maketitle
\linenumbers

\textbf{Garcia \textit{et al.}, 2014, The symbiont side of symbiosis: do microbes really benefit?} \newline
It has been presumed that microbial symbionts receive host-derived nutrients or a competition-free environment with reduced predation, but there have been few empirical tests, or even critical assessments, of these assumptions. 
Evaluation of these hypotheses based on available evidence indicates reduced competition and predation are not universal benefits for symbionts. 
Some symbionts do receive nutrients from their host, but this has not always been linked to a corresponding increase in symbiont fitness.

      

  
\end{sloppypar}
\end{document}