\documentclass[11pt]{article}

\usepackage{setspace}
\usepackage[colorlinks,linkcolor=blue,anchorcolor=red,citecolor=black]{hyperref}
\usepackage{lineno}
\usepackage{booktabs}
\usepackage{graphicx}
\usepackage{float}
\usepackage{floatrow}
\usepackage{subfigure}
\usepackage{caption}
\usepackage{subcaption}
\usepackage{geometry}
\usepackage{multirow}
\usepackage{longtable}
\usepackage{lscape}
\usepackage{booktabs}
\usepackage{natbib}
\usepackage{natbibspacing}
\usepackage[toc,page]{appendix}
\usepackage{makecell}

\title{Stoicism}
\date{}

\linespread{1.5}
\geometry{left=2cm,right=2cm,top=2cm,bottom=2cm}

\begin{document}

  \maketitle

  \linenumbers

Stoicism is another school in the Ethical Movement, found by Zeno (336 BC-264 BC) in Athens at around 300 BC. 
It goes against with the mechanism and hedonism of Epicurus, but understands the universe as a teleological system of reason and order. 
Stoicism had many followers in Greece and Rome, and kept its existence far into Christian time.

\newline

The goal of Stoics is to find rational bases for ethics. 
They agree with Epicurus that we cannot understand the meaning of truth unless we have a criterion of truth and a theory of the universe, that is, unless we study logic and metaphysics. 
But these are just tools to achieve the fruit of ethics.

\section{Logic}
Stoic logic shows similarity with that of Aristotle, but denies inherent knowledge in human reason. 
All knowledge is gained through sense perceptions. 
There is no innate knowledge as Plato and Aristotle proposed. 
The soul are at birth an empty tablet, which receives impressions of things through senses. 
When impressions are combined, concepts and ideas are formed. 
This process is conducted by human reason, which classifies particular impressions by similarity and forming universal concepts and judgments. 
Human reason has such an ability because it is identical with the world-reason, and thus, has the ability to know the order of the universe.

\newline

Hence, all our knowledge is based on perceptions. 
In order to be true, an impression must be accompanied with the conviction that there is an object corresponding to the impression. 
To make the impression reliable, one should be conscious of the observation, make sure that his sense organs function normally, and verify his impression through repetitive observation of himself and others. 
This conviction or consciousness ensures the distinction and reliability of perceptions, which is the basis of knowledge.

\newline

As raw materials, impressions are filtered by the conviction and are organized into concepts and ideas, and they can be further advanced by deduction. 
Judgments concerning concepts are made, and they shall be further deduced into propositions by logical necessity. 
Knowledge is a comprehensive body of true propositions, in which one is deduced from another by logical necessity. 
Deduction is another important way of reaching truth. 
It frees us from being limited by things around us, but leads our cognition to the remote in time and place. 
Thus, Stoics attach importance on formal logic, concerning syllogism and categories, but they seldom make original contributions beyond Aristotle.

\section{Metaphysics}
The existence of things are based on two principles. 
One principle acts, moves and forms, corresponding to the form in Aristotelian metaphysics. 
Another one is acted, moved and formed, corresponding to the matter in Aristotelian metaphysics. 
Forms and matter can be distinguished in thought, but they are always united in real objects. 
However, Stoics have different views on the nature of the principles. 
To Aristotle, only matter is material, while forms are abstract entities of no corporeality. 
But Stoics reject immaterial existence and declare that both forms and matter are corporeal. 
In Stoic metaphysics, everything is material, even including human souls and God. 
Particular thing has different qualities and distinguishes itself from others because of the material form permeating it.

\newline

Forms are active and intelligent. 
They constitute one all-pervasive form. 
It is the divine fire, the \textit{logos}, the ultimate principle of the universe, the rational and active soul of the world. 
The world-soul must be one, because the universe controlled by it is a unity, and because all its parts are in harmony. 
It is the giver of life and the causer of movement. 
It is intelligent, purposeful and good, and thus organizes the universe into a beautiful, well-ordered and perfect whole. 
This world-soul is the Stoic God.

\newline

The God is a single godhead of two aspects, one pervades the universe and forms everything, while the other one retains its original shape as pure form. 
God, the perfect and blessed being, is the father of all things and has prevision and will. 
He is a lover of human, cares for everything, rewards the good and punishes the evil. 
The God of Stoicism is similar with that of theism, but they are not completely identical. 
Stoic concept of God is both of theism and of pantheism. 
It is theism for God is conferred with personalities to certain extent, while it is of pantheism for God presents in all objects as forms.

\section{Cosmology}
Stoics offer a description of the evolution of universe from the original divine fire. 
From fire arise air, water and earth. 
Natural objects are the combinations of the four elements, and fire, as the active principle and God of pantheism, always permeates into lower elements. 
Fire can be differentiated into forms of different objects in accordance with the degree of purity. 
In inorganic world, it is blind causality; 
in plants, it is blind but purposive natural force; 
in animals, it is purposive impulse guided by ideas; 
and in human, it is purposive and rational consciousness.

\newline

The universe is a perfect sphere floating in empty space, held together and organized by the God. 
It arose from fire, and will return to it, and will arise again. 
The recurrence of universe has no end. 
Each circle of the origin and decay of world is completely the same, for it always follows the will of God. 
Everything is absolutely determined. 
Man may choose to assent to the fate or not, but he must obey it, for it is the will of the God, which nothing can go against with. 
God makes the universe a beautiful and perfect whole, and determines the proper place and purpose of everything.

\newline

How about evil? 
If the universe is perfect, determined by the inviolable dominance of the God, how could there be evil and ugly things? 
Stoics adopt two solutions to this question. 
The first one is to deny the existence of evil, arguing that what we call evils are only relative. 
They contribute to the beauty and perfection of the world as a whole. 
The second one is to regard evils as necessary means of the realization of the perfect universe. 
The so-called evils are needed by the manifestation of the will of God. 
Without them, the universe cannot be perfect.

\section{Psychology}
Man is composed of body and material soul. 
The human soul is a part of the world-soul, and thus, is the controlling part, and confers human reason, or the ability to learn the will of the God. 
Man is free when he is in accordance with his reason, for his will is identical with that of the God. 
He knows his place and purpose in the universe. 
His wishes are also the wishes of the God, and thus, can always be realized.

\section{Ethics}
Consistent with the metaphysics and psychology, in Stoic ethics, the meaning of life is interpreted as to act in harmony with the purpose of the universe, to fit one’s own place and finish his purpose, to be consistent with the order and perfection of the universe. 
That is, to subordinate to the determination of God. 
For a human being, this means to act in conformity with his reason consciously, intelligently and voluntarily. 
This is because human reason is essencially a part of the world-soul. 
It reveals one’s duties in the universe and guides him to contribute to the perfection of the world. 
To follow human reason means to live a life of virtue. 
This is the only way to live a happy and meaningful life.

\newline

A virtuous act is consciously directed by human reason. 
To act unconsciously can never be genius virtue. 
Virtue is dependent on human reason. 
For a human being, he may choose to subordinate to reason or not, and there is no middle ground. 
He is either be a wise man or a fool. 
In this sense, where one virtue is, all others are. 
So virtue is one.

\section{Politics}
Man has two inherent impulses: 
the impulse of self-protection and the impulse of protecting his species. 
The latter are the start of human constitutions, whose necessity is further reinforced by reason. 
It teaches us that we are all members of a society of rational beings, and everyone has his place and duties in it, and is supposed to work together for the perfection of the universe. 
All men are related, all are brothers, children of the same God, for the same world-soul speaks in everyone, and we are all under the same universal order. 
We are citizens of the whole universe instead of merely certain state or country, and we all obey the law of the universe, work for its order, beauty and perfection through participating in political affairs.
  
\end{document}