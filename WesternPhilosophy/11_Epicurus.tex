\documentclass[11pt]{article}

\usepackage{setspace}
\usepackage[colorlinks,linkcolor=blue,anchorcolor=red,citecolor=black]{hyperref}
\usepackage{lineno}
\usepackage{booktabs}
\usepackage{graphicx}
\usepackage{float}
\usepackage{floatrow}
\usepackage{subfigure}
\usepackage{caption}
\usepackage{subcaption}
\usepackage{geometry}
\usepackage{multirow}
\usepackage{longtable}
\usepackage{lscape}
\usepackage{booktabs}
\usepackage{natbib}
\usepackage{natbibspacing}
\usepackage[toc,page]{appendix}
\usepackage{makecell}

\title{Epicurus}
\date{}

\linespread{1.5}
\geometry{left=2cm,right=2cm,top=2cm,bottom=2cm}

\begin{document}

  \maketitle

  \linenumbers
  
The last period of Greek philosophy is a process of decline. 
Wars among Greek states broken frequently and left the Greek world for the Philip of Macedon to conquer in 338 BC. 
Alexander the Great made a splendid kingdom which was broken by a series of wars soon after his death in 323 BC. 
The history of Greece after Alexander the Great is accompanied with conflicts among different political forces, such as Macedon, Ptolemaic Dynasty of Egypt, Seleucid Dynasty, Carthage, Rome, and traditional states of Greece like Athens and Sparta. 
Finally, in 146 BC, Greece became a Roman province.

\newline

Under such chaos of the society, ethical problems become especially important and the construction of comprehensive systems with multiple aspects is given up. 
The old institutions are breaking down, accompanied with the demoralization of life. 
Pessimism gains its influence and it becomes a urgency to find a place for the weary soul to rest. 
What is the meaning of life? 
How shall it be organized? 
What is the thing and action of worth? 
What one shall strive for? 
It is a problem of value and different answers are given.

\newline

Such concerns lead to the Ethical Movement of Greek philosophy. 
Ethics is the main research area of philosophers, although logic and metaphysics still have their places. 
To Greek philosophers, the meaning of life is something need to be thought through. 
One need to think to see what is good and of worth in life. 
To do this, he needs, firstly, to think correctly, which leads to the necessity of logic, or theory of knowledge; 
secondly, to understand the substance for him to think, which leads to the necessity of metaphysics, or theory of being. 
But still, ethics is the first concern. 
Here we shall start study of this movement with Epicurus (341 BC-270 BC).

\newline

The object of philosophy, according to Epicurus, is to enable human to live a happy life. 
All sciences that do not contribute to the goal is valueless. 
Music, geometry, arithmetic and astronomy are all meaningless. 
Logic and metaphysics are useful in the realization of the goal. 
They are of value only as tools used in ethical investigations.

\section{Logic}
The task of Epicurean logic is to develop the criterion of truth, to find out what a proposition should be composed of to be true. 
This must be based on sense perceptions. 
What we sense is always true. 
What we see, feel, smell, taste and hear is real and true. 
Without sensations, we can not have any knowledge. 
Senses are the rigorous copies of things, but they may be interpreted wrongly, resulting in illusions. 
Such mistakes can be introduced by disturbances of sense organs or interference of senses on their way to the organs. 
These can be avoid by repeating the observation or appealing to experience of others.

\newline

Through observation, things of similarity can be classified into the same class and we may give this class a name or marker to build a general idea of certain objects. 
But this general idea itself is not real being. 
It is a result of human intelligence and is not of entity or substance. 
Namely, there are concrete desks, but no abstract \textit{idea} or forms of desk proposed by Plato and Aristotle.

\newline

In addition to sensations and general ideas, human mind also forms opinions and hypotheses. 
We imagine, and propose hypotheses. 
In order to make them true, these hypotheses need to be verified by sensations, or at least indicated by sensations and do not contradict with them.

\newline

Overall, sensation is the criterion of truth. 
General ideas or conceptions are invented, not discovered, and hypotheses should be verified or indicated by sensations to be true. 
The Epicurean logic is consistent with modern scientific methods in many aspects.

\section{Metaphysics}
The metaphysics of Epicurus is essencially a restatement of atomism. 
The universe is composed of permanent and physically undivisible atoms, which differ from each other in quantitative terms like weight, shape and size, and infinite empty space for atoms to move in. 
The origin, decay and transformation of things are resulted from the combination, separation and rearrangement of atoms. 
The motion of atoms are inherent. 
All atoms are of such a nature to move vertically downward, but they have the capacity to deviate spontaneously from the straight routine, or there would be nothing but a rain of atoms.

\newline

Living beings follows the same principle of the physical world. 
Organisms arose from earth. 
At the beginning monsters were produced, but they were succeeded by others who adapted the environment better. 
The heavenly bodies are explained in the same mechanistic principles and are not endowed with souls or spirits. 
Also, there are gods, but they are not something for human to fear or admire. 
In fact, they do not interfere the world at all, only live peacefully, free from care and trouble.

\section{Psychology}
Only living beings have souls. 
The soul is composed of atoms, diffusing through the whole body and responsible for all sensations, which are explained naturally, following the teaching of atomists. 
There is a rational part of the soul in the breast, whose will the rest of the soul obeys. 
The soul is mortal. 
When the body dissolves, the soul dissolves into atoms of which it is composed of, and loses its powers. 
So consciousness ceases with the end of life, and death, thus, is nothing to fear. 
Or in the worlds of Lucretius, an Epicurean poet of Rome:

\textit{Death therefore to us is nothing, concerns us not a jot, since the nature of the mind is proved to be mortal.}

\section{Ethics}
Pleasure is the goal at which we all aim, and thus, is the highest good. 
Every pleasure is good and every pain is bad. 
But this does not mean that all pleasures are of the same value. 
We should be prudent in the choice of pleasure. 
Some pleasures are followed by pains or loss of pleasure, while some pains are followed by happiness. 
Besides, pleasures differ in intensity, which is measured by how much it frees us from pain. 
Mental pleasures are more intense than physical pleasures and mental pains are worse than physical pains for the same reason. 
It is wisdom to decide what pleasures to follow and what pains to avoid. 
To make such decisions prudently, we should choose an intellect life, to understand the natural causes of things, to study philosophy.

\newline

So the aim of life is to obtain pleasure, and to do this, we should be prudent, and thus, we are moral. 
Epicurus believes that to gain genuine pleasure, we must exercise virtues like wisdom, courage, temperance and justice. 
Our prudence anticipates the marvellous pleasure that brought by virtues and thus, we should be moral for the sake of our own happiness. 
Hence, although a hedonist, Epicurus achieves one of the most noble and elevated ethical positions of all times. 
The philosopher himself prefers poetry and sciences and the life of virtue, so does his followers. 
The start point of Epicureanism is seeking pleasure, which means it is essencially a doctrine of hedonism, but it finally leads to a refined and moderate life. 
However, Epicurean philosophy is often misinterpreted as a justification of indulgence and debauchery.

\textit{Social Philosophy}
The social and political philosophy of Epicurus is consistent with his mechanism and hedonism. 
Individuals joint together for self protection. 
Laws and institutions owe their origin to utility and are necessary only for the sake of security, and thus, are merely conventional. 
We follow these conventions because it is to our own advantage to follow conventions. 
Nothing is inherently good or evil, only the consequences are good or evil. 
Breaking laws and morality is bad, only because of the punishments from authorities and the fear of being punished. 
Certain rules of action have been found necessary whenever men live together by experience. 
They are universal laws prevailing in all societies. 
But meanwhile, each society has its special conditions, and thus, has its special conventions.

\end{document}