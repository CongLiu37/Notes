\documentclass[11pt]{article}

\usepackage{setspace}
\usepackage[colorlinks,linkcolor=blue,anchorcolor=red,citecolor=black]{hyperref}
\usepackage{lineno}
\usepackage{booktabs}
\usepackage{graphicx}
\usepackage{float}
\usepackage{floatrow}
\usepackage{subfigure}
\usepackage{caption}
\usepackage{subcaption}
\usepackage{geometry}
\usepackage{multirow}
\usepackage{longtable}
\usepackage{lscape}
\usepackage{booktabs}
\usepackage{natbib}
\usepackage{natbibspacing}
\usepackage[toc,page]{appendix}
\usepackage{makecell}

\title{Aristotle}
\date{}

\linespread{1.5}
\geometry{left=2cm,right=2cm,top=2cm,bottom=2cm}

\begin{document}

  \maketitle

  \linenumbers
  
Plato has left the first comprehensive philosophical system we could find, with the existence of contradictions and inconsistencies. 
The main problem is the relationship of forms and objects. 
Plato degrades the world of experience as appearances and exalts forms as real and permanent, and bridges them by an unprecisely defined conception of “matter”, which causes difficulties. 
This dualism, separation and hierarchy can be found in nearly all aspects of Platonic philosophy and often becomes trouble-maker. 
Besides, the mythical elements referred by Plato must be reconsidered. 
Turning to mythology and popular religions for help is an indication of ignorance. 
The task of reconstructing the system falls to Aristotle (384 BC-322 BC), a pupil of Plato with independent mind.

\newline

In Aristotle’s system, the dualism between forms and matter is retained, but he rejects the transcendence of forms and the fake of matter. 
Forms and matter are not separated, but combined and result in objects, which move and change under the control of forms. 
Thus, the world of experience is not illusion, but a real world of forms and matter. 
Hence, Aristotle’s theories keep a close relationship with experience, which prepare basis for the rise of sciences. 

\section{Philosophy and the Sciences}
Aristotle accepts the teleological theory of Plato that the universe is an ideal world, a harmony system of eternal forms. 
These are the essences and causes and purposes of things. 
Forms make things what they are and control their moving and changing. 
However, forms are not detached from the world of experience, but a part of it. 
The sensible world is not illusion, but a reality for us to study and understand. 
Thus, to study forms, we must start from world of material and our knowledge, hence, is based on experience. 
Starting from experience and the particular, we rise to principles and the general. 
To Aristotle, knowledge is not about an ideal world of transcendence, but our sensible world. 
It is not merely a pile of facts, but to investigate their reasons or causes, and why they cannot be otherwise than what they are. 
To gain genuine knowledge, we should devote ourselves to the world of experience, to observe the world and investigate it. 
Here, what we call sciences today gain their positions in the system of knowledge for the first time.

\newline

Thus, Aristotle drags knowledge to experience, make it become the study of sensible world. 
To rise from facts to principles of generality, there must be certain methods. 
So there is logic, the science of the way to gain genuine knowledge, which is used in all sciences. 
Here the word “science” means knowledge of any kind, originating from the Latin word \textit{scire}, which means “know”. 
By logic, we rise from particular objects to general principles, and develop theories. 
Thus, there are theoretical sciences, which includes mathematics, physics, biology, psychology and first philosophy or metaphysics. 
Besides, knowledge should also include the concern of human self, of human behaviour and institutions. 
This part of knowledge is as patterns or guides of human conduct and needs to be put into practice to make sense. 
Thus, there are practical sciences, namely ethics and politics. 
Furthermore, there are sciences about the creation of beauty, the productive sciences. 
Aristotle’s \textit{Poetics} is an investigation of this sphere, which is now called aesthetics.

\newline

Aristotle’s classification of sciences is of great historical meaning. 
Before him, all knowledge is called “philosophy” vaguely as the love of wisdom in all its forms, and we artificially break up philosophical systems of prior times for the convenience of study. 
We are so accustomed to thinking of different subjects that we ignore the fact that this system of classification is artificial and is the result of human intelligence. 
The classification of knowledge is so influencial that for centuries, it boosts the development of different sciences and it is still indicated in the curriculums of schools and colleges now, although there is a tendency of development of multidisciplinary subjects. 
We can justly study Aristotle’s system following his classification, but we may omit mathematics, for Aristotle makes no original contribution in it, and most details of his natural sciences, for they are already antiquated. 

\section{Logic}
The creation of logic is one of the most important achievements of Aristotle. 
If Aristotle had no contributions in other fields, he would still become one of the greatest scholars in history by the construction of a detailed and delicate system of logic. 
It is true that certain basis of logic has been prepared by pioneer philosophers, like dialectic of Zeno, the arguments of Sophists, the Socratic method of conceptions and the dialectic of Plato, but Aristotle is the first philosopher who deals with valid forms of reasoning and make it a special discipline; 
and thus becomes the founder of logic, which keeps its influence until today.

\newline

To Aristotle, logic is the science of the methods of gaining genuine knowledge. 
It is the very first science that one should familiarize with before proceeding to the study of other sciences, since it is a tool to be used in the scientific investigations of every sphere of knowledge. 
The principles of logical thinking are necessary instruments for avoiding mistakes and finding truths.  

\newline

The theme of logic is the analysis of the form and content of thought, of the process by which genuine knowledge is reached, and of the way of thinking correctly. 
The final aim of logic is to discover knowledge, which is characterized by strict necessity, or in his own words, “something which cannot be other than it is.” 
A scientific demonstration of certain knowledge is composed of two parts: 
firstly it is the case and secondly it cannot be otherwise. 
In this process, one must rise from the particular to the universal, from conditions to causes. 
This is the process of inference, which is composed of propositions. 
By figuring out the relations of propositions and arranging them in proper orders, there are demonstrations, which lead to truth.

\newline

Hence, demonstrations, the process of elaborating the derivative propositions from original truths, become a concern of logic. 
Aristotle’s demonstrations take the form of a syllogism or series of syllogisms, which is proposed as basic form of all demonstrations and deductions. 
A syllogism is composed of three propositions: 
the major premise, 
the minor premise and a conclusion deduced from premises. 
Here is a famous example of syllogism that is often quoted for certain reasons:

\newline

\textit{Major premise: All men are mortal.
\newline
Minor premise: Socrates is a man.
\newline
Conclusion: Socrates is mortal.}

\newline

For Aristotle, syllogism is the basic form that all demonstrations of knowledge of certainty can be reduced to. 
A valid demonstration is composed of valid syllogisms, in each of which the premises must be correct and the conclusion must be depended on its premises. 
Premises need to be proved to be correct, namely they must ground themselves on other premises. 
This process cannot go forever. 
It finally ends with self-evident propositions, or axioms, the principles of absolute certainty, but cannot be proved and do not require proofs. 
Or in Aristotle’s own words, "A basic truth is one which has no propositions prior to it".

\newline

Thus, the whole system of knowledge is based on axioms and constructed by deduction of syllogism. 
Then what is the origin of axioms? 
Here Aristotle intermingles idealistic doctrines and scientific spirit. 
It is idealistic, for basic truths are inherent in human reason, the highest part of the soul. 
It is scientific, for basic truths must be brought into consciousness by experience. 
So the process of cognition is that, axioms hidden in human reason are explicated and exemplified by experience, then deductive knowledge would be possible following the law of syllogism. 
Experience is necessary in cognition, but it must ground itself on a prior basis, the human reason, or it would be a yield of probability. 
Hence, both induction and deduction have their own functions, and empiricism and rationalism are united.

\section{Theory of Categories}
Logic concerns the way of gaining genuine knowledge of certainty, namely the way of thinking correctly. 
Obviously, thinking makes sense only when it is of something. 
We must thinking of certain being. 
So the transition from logic to metaphysics is inevitable, and theory of categories is the bridge.

\newline

Categories are fundamental and indivisible units of thinking, and the basic feature of being. 
They are the predicates that must be used in thinking of something. 
It is impossible to think about certain reality or existence without putting it under certain categories. 
Aristotle gives a list of ten categories: 
substance (what it is), 
quality (how it is constituted), 
quantity (how large it is), 
relation (how related), 
space (where it is), 
time (when it is), 
position (what postures it assumes), 
state (the condition it is in), 
activity (what it does), 
passivity (what it suffers).

\newline

A little bit explanation shall be made about the first category, substance, for it is the key concept in Aristotelian metaphysics. 
It is prior to other categories, or in his own words, a substance is “that which is neither predicable of a subject nor present in a subject”. 
This abstruse expression means that a substance is independent of other categories, while the latter must depend on the substance. 

\section{Metaphysics}
Aristotle defines first philosophy, or metaphysics as a science which investigates the nature of being. 
To Plato, it is the universal forms, the \textit{Ideas} of permanence and transcendence, which he separates above the sensible world. 
But Aristotle rejects this account of substance, and provides seven arguments: 
(1) \textit{Ideas} are merely abstractions and cannot account for concrete objects; 
(2) they are eternal, and thus are unable to explain the change of concrete things; 
(3) they are posterior to particular things, and thus cannot explain them; 
(4) \textit{Ideas} are unnecessary reduplication of objects instead of their explanations; 
(5) by saying individual is the copy of certain \textit{Idea}, our understanding of the individual is not increased; 
(6) for the relation between certain individual and its \textit{Idea}, there shall be an ideal relation, the \textit{Idea} of the relation, and thus it becomes an argument of circular; 
(7) the separation of \textit{Ideas} and objects goes against the unity of particulars. 
Criticisms (1)-(4) point out that \textit{Ideas} are not adequate to explain objects, while (5)-(7) criticizing the relation between ideas and objects.

\newline

So what is the nature of being? 
Aristotle’s answer is substance. 
A substance is an individual object, composed of matter and form. 
To Aristotle, forms cannot exist independently as Plato proposed, and matter itself is not adequate for the existence of objects since it is disordered and purposeless. 
Forms and matter must combine together to compose objects. 
Form is responsible for the universal qualities shared by things of the same type, while matter confers uniqueness and particularity of certain object. 
Form and matter are two inseparable aspects of a substance and are both real.

\newline

  Both forms and matter are eternal. They are two basic principle of substances. However, matter may combine with different forms. Matter combines with a series of forms, one after another, and thus transformations take place. This also means that form is the essence of substances. It is the form that confers the substance essencial qualities of the class to which it belongs and makes what it is. Overall, a single substance is composed of two inseparable elements, matter and form. They are equally real, but the latter is more essencial.
  Transformation takes place when matter combine with different forms one after another. So transformation is qualitative. When something transforms into other things of different type, there is only qualitative change. In this process, substance develops itself into different stages. The earlier stages are potentialities and the latter ones are actualities. This relation is relative. For example, an acorn is the potentiality of an oak, which is the actuality of the acorn. But the oak can also be the potentiality of an oak table. So transformation is such a process in which the potentiality becomes actuality, and in which the form of a substance is exchanged into another.
  One more thing, obviously the process of transformation follows certain rules or patterns. This is especially true for the life cycles of species. What determines the directions of transformations? Aristotle’s answer is form. Form knows it is in the substance and causes it to transform to realize an end, the purpose of form. It drives and controls the development of the substance, and determines its final stage. This will be further explained based on Aristotelian four causes. 
  Aristotle use the term cause to designate any condition required by the occurrence of something. There are four kinds of causes. The first one is the material cause, which is the relatively crude and undifferentiated stuff, of which the substance in question is composed. The second one is the formal cause. It is the form which certain substance is going to embrace. The third one is the moving cause, which drives the thing in question to come into being. It is the active agent which produces things as its effect. The fourth one is the end or purpose. It is that for the sake of which a thing is made.
  Everything can be explained by all four causes, no matter whether it is natural or artificial, and this is especially obvious for artworks. Taking a sculpture of stone for example, its material cause is the stone going to be carved; formal cause, the design of the sculpture desired by the artist; moving cause, the endeavor of the sculptor and his tools; purpose, the complete sculpture which is identically same with the design of the artist. Based on the four causes, a stone is turned into a statue. But to natural processes, things are a little different. The material cause of the substance in question corresponds to its matter, while the other three are attributed to its form. Form is the purposive force that drives the process of change(moving cause) and it foresees and desires its final production(formal cause). When the process is completed, its production is dependent on its form(purpose). So form knows its purpose and desires it, and is the motive power of motion. Hence there are only two essencial causes of change, of occurrence and perish of substances, form and matter. Since they are eternal, change is also eternal, following unchangeable patterns.
  Based on previous argument, it seems that form completely controls the change or motion of substance. Form is perfect, so things of the same class should be perfect and completely the same, which is obviously fake. How should this be explained? Aristotle attributes the imperfection of objects to matter. Matter has its own power. It is not a compliant slave, but offers resistance to the form, resulting in the differences of things which belong to one type.
  Overall, individual objects or substances are true being. A substance is composed of two eternal and inseparable elements: form and matter. The former determines what it is and confers it universal qualities of the type to which it belongs, while the latter is responsible for uniqueness. Matter can combine with different forms and thus, transformation between different substances takes place and it is qualitative. This process is also purposive. It is directed and driven by the form, which knows and desires its purpose and provides motive power. But matter provides resistance, which makes the results of transformation imperfect, different from what form proposes. 

Theology
  The causes of motion or change are forms and matter. Since they are both eternal, motion is eternal. For a given motion, it is produced by another motion. This motion again is produced by a third motion. This regress cannot be continued forever and it should be ended at the first cause of all series of motion, the first mover or the God. God is unmoved and permanent, for if it was of motion, it would be moved or changed by something else, and this goes against with the definition of God. By regress, any motion must be traced back to the process that the unmoved God makes something move. Hence God induces all motion without being moved. This is done because God is the final purpose of everything that occurs; he is the highest good that everything desires. So God is the unifying and directing principle of the world, the goal towards which all things strive, and the purpose of every substance.
  God is unmoved, thus it is pure form, for where there is matter, there is motion and change. He is the exception to the doctrine that form and matter are inseparable. He is the final goal of complete actuality and has no potentiality.
  The purpose of human is reason or thinking, the highest part of the soul, which is ascribed to God. Thus, God is pure reason. Then, what does God think of? To God, the only thing that is worthy to think of is himself. So God is thought-thinking-thought. He is the one who thinks about himself, and his thinking is a thinking on thinking. He is both the subject and the object of knowledge. God’s thinking is different from that of human. Human’s thinking is deductive, from premise to conclusion. But to God, it is entirely intuitive. He knows all at once by a single flat of insight. He contemplates the essence of everything and completely understand all things at once. As pure reason, God has no emotions; he has no pain or passion and is supremely happy. He is everything that a philosopher longs to be.
  Overall, Aristotle’s metaphysics finally rises itself to the study of the first mover or God, and thus is bridged to theology through teleology. To Aristotle, every substance has its purpose. Motion takes place because the objects desire to do so. So the universe is dynamic and purposive, driving by its goal, instead of being passive and mechanical, driving by probability. Besides, he emphasizes change in qualitative level by the combination of matter with a series of eternal forms. Aristotle is confident with his teleological doctrines and put them in the study of natural sciences. He declares that something is impossible because it goes against with his metaphysics. 

Physics
  Aristotelian physics, the science of bodies and motion, is characterized by its antagonism to mechanistic theories. He rejects explain the universe in quantitative term by motion and arrangement of atoms as atomists has done, but understands it qualitatively and more dynamically, as if it is an organism, consisting with his teleological metaphysics.
  Space is defined as the limit between a surrounded and surrounding body. The existence of space is dependent on body. Where there is no body, there is no space. Thus, the empty space proposed by atomists is denied. Furthermore, whatever is not be bounded by other body, is not in space. So infinite space bounded by nothing is impossible. Thus, the world is finite; and it does not move as a whole, but its parts suffers change. God, the first mover is not in the world, for it does not move. So he is outside the world of bodies, and is not in space. 
  By the term motion, Aristotle designates change of all sorts, defining it as “the realization of the possible” under his framework of teleological metaphysics. There are in total four kinds of motion: substantial(origin and decay); quantitative(change in the size of a body by addition and subtraction); qualitative(transformation of one thing into another); local(change of place). Aristotle understands nature qualitatively. There are absolute qualitative change in objects. Elements can transform from one to another, and their mixture produces substances of new qualities.
  The universe is eternal. In its center there is earth. Around it, in concentric layers, are water, air, fire, celestial sphere composed of ether, the sun, the moon, the outermost sphere of fixed stars. The God encompasses outside the fixed stars and causes its motion, which results in the motion in other spheres one after another.

Biology
  Aristotle’s biology, like his physics, rejects quantitative, mechanical and casual conceptions of nature, but subordinates to qualitative, dynamic and teleological explanations. Form is dynamic and purposive force who desires its end. It is also the soul of organism. The body is an instrument while its soul is the user, moving it and determining its structure. Body and soul constitute an indivisible unity, but soul is the prior, controlling and guiding principle of the whole. That is, the realization of the whole is prior to its parts. To understand organisms, parts must be related to the whole.
  Where there is life, there is soul. The souls constitute a hierarchy, from plant soul whose function is of nutrition, growth and reproduction; to animal soul, which governs senses, instincts, emotions and lusts; to the highest reason soul, which is responsible for thinking and other additional functions. Three kinds of souls correspond to different organisms, from the lowest plants, to animals, and finally, human. Soul must combine with its body and no soul could exist alone. The union of soul and its body is specific, i.e. a human soul cannot dwell on the body of a horse and versa vice.

Psychology
  Human is the final goal of nature. His soul resembles the plant soul in vital functions and the animal soul in the faculties of perceptions, but is distinguished by the possession of reason. Sense perception is a change produced in the soul by perceived things, mediated by sense organs. Different senses perceive qualities of things and meet in the heart, the organ of common sense, who combines qualities furnished by special senses and provides a total picture of the thing perceived. The common sense also has the power of memory and associative thinking.
  Except functions of growth and perception, human soul also possesses the power of thinking the universal and necessary essences of things, the faculties of gaining conceptions. That is the reason. We apprehend sensible objects by perception, and grasp conceptions, the essences of things, by reason. Reason is, potentially, whatever it can conceive or think, and actually, conceptions.
  Aristotle further divides reason as two parts: the passive and the creative. In passive reason concepts are potential, and the creative reason brings them real or actual. The passive and creative reason is a reflection of the dualism of form and matter in psychology. Passive reason is the matter on which creative reason acts, resulting in conceptions, the direct cognition of essences of things. So in reasoning process, conceptual thoughts hide in passive reason, and are  found and actualized by creative reason.
  Perceptions are connected with body and perish with it. Passive reason, since it is operates semi-manufactured concepts, is also perishable. Creative reason, however, exists prior to perceptions and passive reason, and is absolutely immaterial, immortal and imperishable. It is not bound to a body, but a spark of divine mind. To some extent, it can be identified with the God, the final purpose.

Ethics
  Aristotelian ethics is based on his metaphysics and psychology, attempting to give a definite answer to the question of highest good. Every human action has its end or purpose, which can be means to a higher purpose. Keep this deduction and finally, we must end at a final purpose, an ultimate principle of good for whose sake all other good is valuable and to be sought. It is the highest good, the first principle of human action, the meaning and purpose of life.
  So what is the highest good? For certain substance, its highest good is moving towards its purpose and realizing it. And its purpose is the realization of its specific nature, which distinguishes itself from other things. So, the highest good is the actualization of the peculiar essences of things. For plants, the highest good is growth; for animals, it is perception; then, what is the highest good for human? The highest good of man is not merely the existence of body or sensuous feeling, but a life of reason, which distinguishes man from non-living things, plants and animals. Hence, the highest good for human is the complete and habitual exercise of reason, the essence of human.
  Reason is in the soul of human. However, the soul also has irrational parts. In order to realize the highest good, reason should cooperate the other parts of the soul. Different parts of the soul must act correctly as a whole and the body must function properly. Besides, environmental conditions are necessary to the realization of the highest good. Neither a child or slave can attain the ethical goal. Poverty, sickness, misfortune, ect., may interfere the attainment.
  So, a virtuous soul is well ordered with a correct relationship between its reason and other irrational parts. In this condition, reason is the dominant role of human and provides ground for virtue. The perfect action of reason constitutes intellectual efficiency, or dianoetic virtue, the virtue of wisdom, insight and knowledge. The perfect action of irrational soul, the emotional-impulsive functions constitutes moral virtue, like courage, temperance, ect. For all actions, there is a moral virtue. These virtues constitute a rational attitude toward real life, toward danger, anger, fear, desire, ect. This attitude is seeking the mean of two extremes, the excess and the deficiency. For example, courage is the mean between cowardice and foolhardiness, and modesty is between bashfulness and shamelessness. The doctrine of mean is not universal. To certain feelings and actions, whether they are good or bad is dependent on quantity. There are still others that are totally bad, like envy, theft and murder. 
  What is good or bad is different to different individuals and under different circumstances. The determination of goodness or badness is essencially conducted by reason, and practically by the right-minded man, whose soul is well ordered and virtuous. He is the measure and standard of things. He finds truth of every case and judges everything correctly. 
  There are other two points about the practice of virtue. Morality does not consist in a single action, but in the expression of a stable character of will. Furthermore, it is voluntary action. It is conscious and freely chosen. Aristotle concludes his ideas in the following definition:
  Virtue is a disposition, or habit, involving deliberate decision or choice, consisting in a mean that is relative to ourselves, the meaning being determined by reason, or as a prudent man would determine it.
  Hence, the highest good for man is actualization of reason, that is, self-realization. It is not selfish individualism, but altruistic spirit. Self-realization is done when a man loves and gratifies his supreme part, the true self, that is, his rational soul. This requires being moved by a motive of nobleness, and promoting interests of others and serving his country. In Nicomackean Ethics, he wrote:
  The virtuous man will act often in the interest of his friends and of his country, and if need be, will even die for them. He will surrender money, honour, and all the goods for which the world contends, reserving only nobleness for himself, as he would rather enjoy an intense pleasure for a short time than a moderate pleasure long, and would rather live one year nobly than many nears indifferently, and would rather perform one noble and lofty action than many poor actions.
  So, to reach the ethical goal, one needs to do good to others, since man is a social being and is disposed to live together. The virtue implying to social relations is justice, for it promotes interests of another, whether he is a ruler or an ordinary people. The only difference between justice and virtue is that, justice is considered under the relations among people, while virtue designates a state of man.
  Pleasure is the necessary and immediate consequence of virtuous activity, but is not its purpose. It is something added, and is the symbol of the completion of action of virtue. The more pleasant, the more perfect the activity would be. Pleasure are bound with life, since pleasure is impossible without activity, and activity is perfected by pleasure. True pleasure is determined by virtuous man. His judgement on what is pleasant is the only reliable standard. People who have never tasted true pleasure may mistakenly make it the satisfaction of bodily desires, but on ethical problems, man of virtue is the only measure.
  The most pleasure activity is speculation, whose form is contemplation. It is the practice of reason, the immortal, supreme and divine part of human. Reason is the highest part of human being, and following it is the most virtuous, and thus, the most pleasant. 
  Hence, our discussion focuses on theory of ethics. The purpose or highest good of life is the realization of reason, which distinguishes human from other things. The dominance of reason confers human virtue, and thus, he becomes the fair measure of things. He determines meanings of things correctly and behaves virtuously. His action is altruistic, accompanied with true pleasure. But to Aristotle, learning these ethical theories is not enough to guarantee morality, which is against to Socrates’ belief that one cannot become evil as long as he knows the true meaning of virtue. In addition to knowledge of virtue, we meed to endeavor to possess and exercise it. Knowledge of virtue may be enough to stimulate certain talents, but for the majority people, more is needed. Moral action is fostered by a moral environment. To guarantee morality with highest possibility, one should receive a right inclination to virtue in his early days, and this requires a virtuous society, which is based on virtuous laws and government. Laws and government are required to lead the state to virtue, for most people are motivated by necessity and the fear of punishment rather than by reason and nobleness.
  Thus, the task of human institution is to elevate its people to virtue, to reach the goal of human life. This is done by providing a social environment of morality and employing punishment and other legal devices. Hence, Aristotelian ethics is crowned with politics, the science of human institutions. In fact, ethics and politics are never divorced in Aristotle’s system. The moral goal of life is promoted in political means.

Politics
  Man is a social being, who can realize his final goal only in society and the state. The state originated from families and small communities, but is prior in terms of significance and worth, consisting with the doctrine that the whole is prior to its parts. The purpose of society is to produce citizens of virtue and of pleasure, while to individual, the realization of ethical goal requires participation of social life. 
  The constitution of the state must be adapted to the character and requirement of people. It would be just for the state to confer its people equal right in so far as they are equal, and to confer unequal right in so far as they are unequal. Here “equal” means the same in all aspects, including personal capacity, qualifications, birth, freedom, ect. Since citizens are different in these aspects, they should be treated differently in accordance with this inequality. So, as for public affairs of government, participators need to be qualified, and this is based on standards including capacity, education and social position. This is an aristocracy, the best form of government. 

Summary
  In all possible standards that could be used to judge a scholar, Aristotle deserves the title of “master of those who know”. His breadth of learning, originality and influence give him a special position in the history of thoughts. Aristotelian system embraces nearly all aspects of knowledge, reaches most of them with his genius and arranges them into a comprehensive system, which is so huge that merely its existence is amazing enough. In the history of any subject, Aristotle is always the absolutely necessary chapter. For centuries, nearly all advances of knowledge started with the examination of Aristotelian doctrines. He is definitely the peak of Greek philosophy.
\end{document}