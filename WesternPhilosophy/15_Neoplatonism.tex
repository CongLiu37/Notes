\documentclass[11pt]{article}

\usepackage{setspace}
\usepackage[colorlinks,linkcolor=blue,anchorcolor=red,citecolor=black]{hyperref}
\usepackage{lineno}
\usepackage{booktabs}
\usepackage{graphicx}
\usepackage{float}
\usepackage{floatrow}
\usepackage{subfigure}
\usepackage{caption}
\usepackage{subcaption}
\usepackage{geometry}
\usepackage{multirow}
\usepackage{longtable}
\usepackage{lscape}
\usepackage{booktabs}
\usepackage{natbib}
\usepackage{natbibspacing}
\usepackage[toc,page]{appendix}
\usepackage{makecell}

\title{Neoplatonism}
\date{}

\linespread{1.5}
\geometry{left=2cm,right=2cm,top=2cm,bottom=2cm}

\begin{document}

  \maketitle

  \linenumbers

Jewish-Greek philosophy attempts to absorb Greek speculations into the teachings of Moses and the Prophets, while neoplatonism represents an opposed direction by constructing a religious philosophy on the basis of Greek wisdom, especially that of Plato. 
System of Plato becomes the framework for a religious world-view, which utilizes whatever seems valuable in teachings of other Greek philosophers. 
God is conceived as the source and goal of everything; 
from him all things origin, and to it all things return. 
He is the alpha and omega. 
He is the source of value, and thus, communication with God or absorption in God, is the real goal of all our striving. 
Among neoplatonic scholars, we shall focus on Plotinus (204-269), one of the most representative figure of this school and the last master of antiquity. 
This would be enough for a comprehensive understanding of the teachings of neoplatonism.

\section{God the Source of All Being}
God is the absolute One, the source of all existence, the first cause from which everything emanates. 
His transcendence is such that whatever we say of him merely limits him. 
We cannot define him by saying what God is, but only say what God is not. 
Silence contains much more wisdom than any words in understanding his transcendence. 
He cannot be conceived as being, for being is thinkable, which implies the subjective thinker and the objective things to be thought, and thus, a limitation. 
He is prior to all being and beyond all being. 
He cannot be conceived as thinking, because this implies something for God to think of, which again introduces the subjective and the objective and robs the absolute transcendence of God. 
Even the statement that God thinks of himself is unacceptable, 
for it divides God, the absolute One into the objective and the subjective. 

\newline

The world origins from God, but this does not mean that God created world, for creation implies the thinking or will of God. 
The universe is an emanation of God, an inevitable overflow of his infinite power. 
He is the source of spring from which streams flow without exhausting the infinite source. 
He is the sun from which light radiate without loss to the sun. 
God is of infinite and absolute power, and the universe is merely his effect, which cannot impose any change or influence on him.

\section{Stages of Being}
The farther we are from God, the source of perfection, the closer we are down to the imperfect. 
Emanation is the process of fall from the perfect to the imperfect, from the one to the plural and diversified, from the eternal to the changeable. 
There are three stages in emanation: pure thought, soul and matter.

\newline

In the first stage, the One is divided into thought and \textit{ideas}. 
Here the term \textit{idea} is Platonic. 
Ideas are different from each other, but form a unified system. 
Thought thinks of \textit{ideas}, but they are not separated in time and space. 
So he thinks of himself. 
Thus, Thought is subjective and \textit{ideas} are objective. 
The thinking process of thought is not discursive, from promises to conclusions, but intuitive, contemplating all ideas as a whole and at once. 
This world of thought and ideas is spaceless and timeless. 
It is an eternal and perfect pattern or mold of the sensible world. 
For each particular object, there is a perfect idea.

\newline

The second stage is soul emanated from pure thought. 
It is active, intelligent, and has the power of pure thought, though incompletely. 
It has two aspects, the world-soul and the nature. 
The world-soul is closer to pure thought, acts as it and contemplates the pure \textit{ideas}. 
From world-soul nature emanates. 
It impels to bring order into matter and has desire.

\newline

In the third stage, matter emanates from soul. 
Matter has neither form, quality, power nor unity. 
It is the farthest from God, the opposite of God and the principle of evil. 
It is necessary substance composing things of sensible world but we cannot form any image of it. 
Upon matter, soul acts and fashions it into sensible things, imperfect copies of ideas. 
The imperfection and chaos of the world of sense are attributed to matter, while it still shows beauty and order to some extent, which is the effect of soul, and harks back to pure thought and God.

\section{Human Soul}
The hierarchy of being leads to ethical theories of Plotinus. 
Human soul is a part of the soul of the second stage, with a phase towards God and a phase towards matter. 
It has contemplated the universe of pure thought and ideas before falling into matter and combines with a body in the sensible world, which is a consequence of its nature. 
It is one’s freedom to choose the dominant part of his soul, but obviously, the phase towards God is attached with value and is considered as the goal of life, while the other one is the irrational, the animal, the seat of sensuality and the source of sin. 
If the soul fails move towards God, it will steeps in bodily life, attach after death to another body of human, animal or plant, in accordance with the degree of its guilty.

\newline

Hence, the final goal of life is the union of soul and God. 
To do this, ordinary virtues and speculations are not enough. 
The soul must get rid of all sensuality and free itself from the combination of body. 
That is, asceticism. 
But what we can finally gain through these efforts is merely contemplation to the \textit{ideas}, which is near to God. 
The highest good, the union with God, cannot be reached through striving of this exalted kind. 
It needs a state of ecstasy, a mysterious and religious state, in which the soul transcends itself and loses itself in God. 
This is the religious and mysterious end of life.

\end{document}