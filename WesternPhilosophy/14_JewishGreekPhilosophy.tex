\documentclass[11pt]{article}

\usepackage{setspace}
\usepackage[colorlinks,linkcolor=blue,anchorcolor=red,citecolor=black]{hyperref}
\usepackage{lineno}
\usepackage{booktabs}
\usepackage{graphicx}
\usepackage{float}
\usepackage{floatrow}
\usepackage{subfigure}
\usepackage{caption}
\usepackage{subcaption}
\usepackage{geometry}
\usepackage{multirow}
\usepackage{longtable}
\usepackage{lscape}
\usepackage{booktabs}
\usepackage{natbib}
\usepackage{natbibspacing}
\usepackage[toc,page]{appendix}
\usepackage{makecell}

\title{Jewish-Greek Philosophy}
\date{}

\linespread{1.5}
\geometry{left=2cm,right=2cm,top=2cm,bottom=2cm}

\begin{document}

  \maketitle

  \linenumbers

Through the conquering of Alexander the Great, oriental and Greek civilization started to interact with each other, and finally, led to the rise of a philosophy strongly related with religious mysticism. 
Greek speculations and Hebrew religion combined together, resulting in the Religious Movement. 
Philosophers of this movement agree the concept of God as a transcendent being, the idea of revealed and mystical knowledge of God, asceticism, the dualism of God and the world, and the belief in intermediary beings like angles and demons. 
Their theories are different in the proportion of Greek and Jewish element. 
In Jewish-Greek philosophy, orientalism predominates and in neoplatonism, Greek thought is stronger.

\section{Aristobulus}
The first trace of the union of Greek and Jewish ideas can be found in a treatise by Aristobulus (~ 150 BC). 
He tries to show the harmony between the teachings of \textit{Old Testament} and Greek philosophers, and asserts that Greeks got their wisdom from Jewish Scriptures. 
Scriptures are interpreted as allegories, corresponding to Greek speculations, in order to get rid of anthropomorphism. 
He conceives God as a transcendent being, the divine power of dominance, who is only visible to pure intelligence, and thus, mortals can never behold him.

\section{Philo}
The Jewish-Greek philosophy peaks in the system of Philo (30 BC- 50 AD). 
In his view, Judaism is the sum-total of human wisdom. 
Jewish Scriptures are interpreted as allegories to show that the teachings of Greek philosophers and the Prophets are the same. 
For example, Adam stands for spirit or mind, Eve for sensuality and Jacob for asceticism.

\newline

The fundamental concept of Philo’s system is God, an being of absolute transcendence, the greatest good, the ineffable one, who is above human intelligence. 
We can know he is, but we cannot know what he is. 
We are certain about the existence of God through our reason, but are not qualified to define him. 
He is the source of everything, the absolute power, the absolute goodness, the absolute perfection, and pure intelligence.

\newline

God is too exalted to be in contact with impure matter, so there should be intermediate powers to exhibit the order of God to the universe. 
Here Philo refers to conceptions in both Judaism and Greek philosophy. 
He sometimes describes these powers as angles and demons, the messengers or servants of God, but in other times, they are understood as properties or thoughts of God, or as parts of the universal reason. 
All such powers Philo combined together as the \textit{Logos}, the first-born son of God, the image of God, the second God. 
The concept of \textit{Logos} proposed by Philo can be conceived as the world-soul in Stoicism. 
We know the existence of \textit{Logos} through the \textit{logos} in ourselves.

\newline

The wisdom and power and goodness of God are shown through \textit{Logos}, and it needs material to act upon. 
Hence, quality-less matter is introduced as another principle. 
God fashioned the universe from the chaotic matter by \textit{Logos}. 
Time and space were created when the universe was created. 
The world had a beginning in time, but no end. 
Since \textit{Logos} is good and perfect, the evils and defects of the world are attributed to matter.

\newline

Man is the most important piece of creations, composed of the soul and the body. 
The ruling part of the soul is constituted of desire, courage and reason (\textit{logos}), while other parts of the soul and the body belong to the world of matter. 
The pure reason is the chief essence of human and is added to the soul from above. 
Combination of soul and body is a fall, for body is the source of evil. 
So man is inherently predisposed to be evil, and he should deliver the true himself, the reason, from his body, the evil principle, and he should eradicate all his passions and sensuality by theoretical contemplation, that is, asceticism. 
But we cannot do this by ourselves, because we are too weak. 
We need divine help. 
God must illustrate us. 
Only then can we apprehend God, plunge ourselves to the pure source of being, and be freed from evil. 
Hence, Philo combines asceticism and mysticism, which becomes a common pattern in the philosophy of Middle Age.
  
\end{document}