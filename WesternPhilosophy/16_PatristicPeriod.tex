\documentclass[11pt]{article}

\usepackage{setspace}
\usepackage[colorlinks,linkcolor=blue,anchorcolor=red,citecolor=black]{hyperref}
\usepackage{lineno}
\usepackage{booktabs}
\usepackage{graphicx}
\usepackage{float}
\usepackage{floatrow}
\usepackage{subfigure}
\usepackage{caption}
\usepackage{subcaption}
\usepackage{geometry}
\usepackage{multirow}
\usepackage{longtable}
\usepackage{lscape}
\usepackage{booktabs}
\usepackage{natbib}
\usepackage{natbibspacing}
\usepackage[toc,page]{appendix}
\usepackage{makecell}

\title{Patristic Period}
\date{}

\linespread{1.5}
\geometry{left=2cm,right=2cm,top=2cm,bottom=2cm}

\begin{document}

  \maketitle

  \linenumbers
The period of ancient philosophy officially ends at 529, when Justinian closed the school at Athens. 
The intellectual life of next era is dominated by Christianity, whose history traces back to first century and begins with patristic period.

\newline

The philosophy of patristic period is treated as a prelude to medieval philosophy proper. 
It results the fusion of early Christian religion with Hellenistic philosophy. 
On the narrowest interpretation, patristic period ranges from the time of Christ to the death of St. Augustine in 430. 
Interpreted most widely, it also includes further development of Christian dogmas until the Council of Trullo in 692.

\section{Early Theology}
At the beginning of Christian era, the new religion is compelled to define its doctrines, to defend them, to construct a theology declaring its attitude towards prevailing Jewish religion and Hellenistic philosophy. 
Jewish-Greek philosophy serves as the best adapted system for this immediate purpose. 
Allegorical explanation of \textit{Old Testament} and doctrine of logos forms the centre of the new religion. 
Judaism is put under the light of Hellenistic philosophy to reconcile thoughts of Greek metaphysicians and Jewish teachers. 

\newline

The beginning of Christian theology is found in writings of St. Paul. 
The person Jesus Christ is exalted as the incarnate and unique Son of God, and is interpreted in terms of philosophic concepts originated from Hellenistic world. 
He is identified with God's power and wisdom. 
He is the Logos. 
He pre-existed as the archetypal man, but was created by God.
Thus, religious and philosophic elements begin to be welded together with emphasis on religious part. 
The Logos is interpreted as a person, the Son of a living Father, not a cold philosophical abstraction.

\section{Gnostics}
It is not surprising that in the growth of the union of Greek philosophy and Jewish religion, some thinkers sought to interpret this new religion with the light of philosophic thinking, to rationalize it, to transform faith into knowledge. 
This work was done by Gnostics in second century. 
They speculate upon faith to provide a Christian philosophy, a harmony of faith and knowledge. 

\newline

Gnosticism is an embryonic scholasticism, though close to mythology it is. 
They assert that their secret or esoteric doctrines have been transmitted by Jesus Christ to his followers who are able to receive them. 
Christianity is a divine doctrine, while Judaism is a corrupted paganism. 
The Judaism God is false, opposite to the true God. 
Christ, one of the highest spirit, entered a human body in order to free spirits imprisoned in matter by Judaism God. 
Those able to comprehend true teaching of Christ become gnostics and eventually escape from their material bondage via asceticism. 
Those who fail to understand teaching of Christ perish with their matter bodies. 
The material world is a result of fall towards matter the evil. 

\newline
Gnostics fail to provide a philosophy of Christianity, and are in conflict with prevailing concepts of Jesus' teaching by reputing \textit{Old Testament} and conceive Jesus as a man whose body is used by a heavenly spirit. 
At the same time, gnostics exercise considerable influence on the new religion, and provide an impetus to philosophical formulation of Christian theology.

\section{Apologists}

\end{document}