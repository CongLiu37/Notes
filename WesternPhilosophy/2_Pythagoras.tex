\documentclass[11pt]{article}

\usepackage{setspace}
\usepackage[colorlinks,linkcolor=blue,anchorcolor=red,citecolor=black]{hyperref}
\usepackage{lineno}
\usepackage{booktabs}
\usepackage{graphicx}
\usepackage{float}
\usepackage{floatrow}
\usepackage{subfigure}
\usepackage{caption}
\usepackage{subcaption}
\usepackage{geometry}
\usepackage{multirow}
\usepackage{longtable}
\usepackage{lscape}
\usepackage{booktabs}
\usepackage{natbib}
\usepackage{natbibspacing}
\usepackage[toc,page]{appendix}
\usepackage{makecell}

\title{Pythagoras}
\date{}

\linespread{1.5}
\geometry{left=2cm,right=2cm,top=2cm,bottom=2cm}

\begin{document}

  \maketitle
  
  \newpage

  \linenumbers
  
Milesian philosophers are interested in the problem of substance. 
They wonder the essence of things: what is the substance, or substances, of which everything is composed of? 
They regard it as something concrete, either determinate like water or air, or indeterminate like the \textit{Infinite} of Anaximander. 
Compared with substance, Pythagoras (?-500 BC) and his school pay more attention on the relations and forms of things. 
As mathematicians, they are interested in quantitative relations and try to speculate the uniformity and regularity of the world, attempting to explain it by making an entity of numbers and making it the first principle.

\section{Number Theory}
Pythagoras was impressed by the fact that measure, order, proportion recur and they could be expressed by numbers. 
So he reasoned that numbers must lie in the basis of these relations, and if there is no number, such relations will not exist. 
For example, one apple plus one apple equals two apples and one duck plus one duck equals two ducks. 
The reason why they are true is that one plus one equals two. 
If there is no such a relation between one and two, one apple plus one apple will not equal two apples and one duck plus one duck will not equal two ducks. 
So numbers are the essence of relations.

\newline

Numbers are also essence of objects. 
An object must have its spatial form, which is composed of spots, lines, and surfaces. 
All these elements can be expressed by numbers. 
So spatial forms are represented by numbers. 
On the other hand, numbers are independent from spatial forms. 
So numbers are the essence of forms. 
Moreover, there is no such an object without any spatial form, while spatial forms could exist without reference to any object. 
So forms are the cause of objects. 
Hence, numbers are concluded as the essence of objects. 
In other words, since there are numbers, there are spatial forms. 
Since there are spatial forms, there are objects. 
It shall be emphasized that the numbers and forms Pythagoras talked about are entities. 
Numbers are material causes of objects. 

\newline

Since numbers are the first principle of things, what is true of numbers must be also true of things. 
Numbers are classified into odd and even. 
Odd numbers cannot be divided by two while even numbers can be divided by two. 
So the former are limited while the later are unlimited. 
Such dualism in numbers also appears in the world. 
So nature itself is a union of opposites. 
Pythagoras also gave a list of ten opposites:

\begin{center}
    \textit{Limited} and \textit{Unlimited}
    \newline
    \textit{Odd} and \textit{Even}
    \newline
    \textit{One} and \textit{Many}
    \newline
    \textit{Right} and \textit{Left}
    \newline
    \textit{Male} and \textit{Female}
    \newline
    \textit{Rest} and \textit{Motion}
    \newline
    \textit{Straight} and \textit{Crooked}
    \newline
    \textit{Light} and \textit{Darkness}
    \newline
    \textit{Good} and \textit{Bad}
    \newline
    \textit{Square} and \textit{Retangle}
\end{center}


\section{Astronomy}
The dualism of limited and unlimited also appears in the astronomy of Pythagoras, as the opposition of stellar system, which is relatively fixed and uniform, and terrestrial region, which lacks the order of the other one. 
The universe is spherical and in its centre is the central fire, around which all planets revolve. 
The fixed stars are in the highest arch of heaven. 
Below them, in turn, are Saturn, Jupiter, Mars, Mercury, Venus, the sun, the moon and the earth. 
Since ten is the perfect number, there must be ten heavenly bodies. 
However, we have just mentioned nine planets: Saturn, Jupiter, Mars, Mercury, Venus, the sun, the moon, the earth and the central fire. 
To solve this riddle, a counter-earth between the earth and the central fire is introduced. 
The earth and the counter-earth revolve around the central fire in such a way that the earth always turns the same face to the counter-earth and the central fire. 
For this reason, we, living on the opposite face of the earth, cannot see the central fire and the counter-earth, but the light of central fire reflected by the sun.

\section{Ethics}
Pythagoras’ ethics is also based on his number theory. 
Non-corporeal things are also expressed by numbers. 
For example, love and friendship are expressed by the number eight, because love and friendship are harmony, and octave is harmony, too. 

\newline

In a famous simile, life is likened to a public game and people are divided into three categories: 
the vendors who are only interested in profits; 
the players who aim at honour and praise; 
the observers who are seeking wisdom.

\end{document}