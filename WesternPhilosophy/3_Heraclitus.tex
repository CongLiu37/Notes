\documentclass[11pt]{article}

\usepackage{setspace}
\usepackage[colorlinks,linkcolor=blue,anchorcolor=red,citecolor=black]{hyperref}
\usepackage{lineno}
\usepackage{booktabs}
\usepackage{graphicx}
\usepackage{float}
\usepackage{floatrow}
\usepackage{subfigure}
\usepackage{caption}
\usepackage{subcaption}
\usepackage{geometry}
\usepackage{multirow}
\usepackage{longtable}
\usepackage{lscape}
\usepackage{booktabs}
\usepackage{natbib}
\usepackage{natbibspacing}
\usepackage[toc,page]{appendix}
\usepackage{makecell}

\title{Heraclitus}
\date{}

\linespread{1.5}
\geometry{left=2cm,right=2cm,top=2cm,bottom=2cm}

\begin{document}

  \maketitle

  \linenumbers

Philosophers of Milesian and Pythagorean schools focused on the problem of substance and assumed that nothing could absolutely originate or be lost, \textit{i.e.} nothing could originate from nothing or turn to nothing. 
Besides, they also implicated that substance, or substances, could transform into other objects of different qualities. 
None of them thought about change itself. 
It would be natural that the following philosophers start to think about problem of change and put it the centre of their theories. 
That is what Heraclitus (535 BC-475 BC) and philosophers of Eleatic school did. 
In their theories, change is explicit and is speculated as an important problem. 
The former, holds the view that all things are in a state of ceaseless change, while the latter, on the contrary, insists that change is impossible. 
This essay talks about Heraclitus' philosophy.

\section{Fire and Universal Flux}
To emphasize the importance of change, Heraclitus chooses the most mobile element he knows, fire, as the first principle of objects. 
Fire is undergoing continual qualitative transformation. 
From it all things originate and to it all things return, as what he wrote:

\textit{All things are exchanged for fire, and fire for all things; as wares are exchanged for gold and gold for wares.}

\newline

Things seem to be permanent because what they lose in one way they gain in another. 
Or in other words, things reach a balance between loss and gain, and we do not perceive such change, which makes things seem to be permanent.

\section{Opposites and Their Union}
Everything is changing into its opposite and everything, therefore, is a union of opposite qualities. 
Such opposites result in harmony of objects. 
So harmony is the union of opposites. 
Ultimately, every opposition in objects will be reconciled. 
Without oppositions, everything will not exist and return to fire, in which oppositions are reproduced. 
Thus, everything originates again.

\newline

The two aspects of opposite are the same, for one could transform into another. 
Hot and cold, justice and injustice, the long and the short, the light and the dark, the good and the bad, life and death, are essencially the same. 
So, all things are good, fair and just as they are, because they are in harmony resulted from the oppositions in them, also because all the opposite qualities in them are the same and thus, are equally good. 
We, human, mistakenly suppose some to be just and others unjust, or some are better than others. 

\section{The Law of Reason}
Change is not haphazard, but in accordance with “fixed measure”. 
Namely, it is governed by law. 
This is one order of things neither any one of the gods nor of men has made, but it always is, was, and ever shall be, an ever-living fire, kindling according to fixed measure and extinguished according to fixed measure. 
In the midst of all change, it is the only thing that persists, the law that underlies all movement, change and opposition. 
It is the rational reason in things, the \textit{logos}. 
It is something higher than the objective world and plays a role of model to everything. 
It is the essencial order followed by all change. 
We cannot make sure that whether the \textit{logos} of Heraclitus is impersonal rationality or personal intelligence. 
Following scholars interpreted it in both two ways and the latter seemed to be more of influence. 

\section{Psychology and Ethics}
The controlling element in a man is the soul, which is composed of fire. 
So the driest and warmest soul is the best. 
The soul is akin to divine reasoning. 
So man must subordinate himself to the universal reason. 
To be ethical is to live a rational life, to obey the dictates of reason. 
Morality means respect for law, self-discipline and control of the passions. 
To be moral is to govern oneself by rational principles. 
Sense knowledge is inferior to reason. 
Perception does not reveal to us the hidden truth, which is discernible to reason.

\end{document}