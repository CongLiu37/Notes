\documentclass[11pt]{article}

\usepackage{setspace}
\usepackage[colorlinks,linkcolor=blue,anchorcolor=red,citecolor=black]{hyperref}
\usepackage{lineno}
\usepackage{booktabs}
\usepackage{graphicx}
\usepackage{float}
\usepackage{floatrow}
\usepackage{subfigure}
\usepackage{caption}
\usepackage{subcaption}
\usepackage{geometry}
\usepackage{multirow}
\usepackage{longtable}
\usepackage{lscape}
\usepackage{booktabs}
\usepackage{natbib}
\usepackage{natbibspacing}
\usepackage[toc,page]{appendix}
\usepackage{makecell}

\title{Milesian School}
\date{}

\linespread{1.5}
\geometry{left=2cm,right=2cm,top=2cm,bottom=2cm}

\begin{document}

  \maketitle
  
  \newpage

  \linenumbers
  
The first school of philosophy in Greek rose in Miletus, one of the most flourished cities in Ionian. 
Its importance lies in the fact that Milesian philosophers put philosophical questions squarely and answered them without reference to mythical being.

\section{Thales}
Thales (624 BC-548 BC) is the first philosopher of Greece and is known as one of the Seven Wise Men of Greece, with his saying “Water is best”. 
Since Thales himself has no works remained, our knowledge about him is limited to secondary sources and mostly conjectural. 
According to Aristotle, Thales’ philosophical views can be concluded to three points: 
(1)All things are full filled with gods; 
(2)The earth is a disc floating on water; 
(3)Water is the material cause of all objects.
  
\section{Anaximander}
It is said that Anaximander (611 BC-546 BC) was Thales’ pupil. 
His advance compared with Thales lies in a tendency towards abstraction and endeavours to explain the process of becoming. 
Here is his argument about the problem of substance: 
The essence of objects can not be any concrete substance, for these materials, like water, must be self-explained. 
So concrete elements must not be the material cause of things. 
The essence of things is an eternal, imperishable substance, which he calls the \textit{Infinite}. 
It is an infinite, intermediate and qualitatively undifferentiated matter. 
It should be noted that the view of the \textit{Infinite}, although shows a tendency to abstraction, is still focus on material cause. 
From the \textit{Infinite} all things are separated and to it all things return.

\newline

Different matters are separated from the Infinite due to its eternal motion. 
First it is hot, then cold, which is surrounded by hot. 
The cold becomes moisture, then air, after being heated by the hot. 
The air expands and breaks the sphere of hot. 
As a result, the hot streams and forms heavenly bodies surrounded by air. 
The are forced to move around the cyclindrical earth, which is the central of the universe. 
The most remote heavenly body is the sun, then the moon, then the stars and planets. 
As for living beings on earth, they arise from moisture. 
Some of them come out of the water upon land and finally adapt themselves to new environment. 
Man, like other creatures, derives through such process.

\newline

Anaximander also proposed theory about the recurrence of world. 
Things are originated from the \textit{Infinite} by separation, \textit{i.e.} the origin of things is a process by which the \textit{Infinite} is robbed, so it is an injustice. 
As a result, everything must finally return to the \textit{Infinite}, following the order of justice. 
Things are the same to the whole universe. 
It originates from the the \textit{Infinite} and finally returns to it, then the universe is formed again. 
Thus, there is an eternal and cyclical recurrence of the universe.
  
\section{Anaximenes}
The last philosopher of Milesian school is Anaximenes (588 BC-524 BC), who holds the view that air is the essence of objects and things arise by the process of rarefaction and condensation. 
When air is rarefied, it becomes fire; when condensed, air becomes, in turn, wind, cloud, water, earth, and stone. 
Thus, changes are explained by motion of air and qualitative differences among elements are reduced to quantitative term, \textit{i.e.} the degree of rarefaction or condensation. 
This is the first example of reducing qualitative differences to quantitative term. 

\end{document}