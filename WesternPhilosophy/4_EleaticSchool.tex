\documentclass[11pt]{article}

\usepackage{setspace}
\usepackage[colorlinks,linkcolor=blue,anchorcolor=red,citecolor=black]{hyperref}
\usepackage{lineno}
\usepackage{booktabs}
\usepackage{graphicx}
\usepackage{float}
\usepackage{floatrow}
\usepackage{subfigure}
\usepackage{caption}
\usepackage{subcaption}
\usepackage{geometry}
\usepackage{multirow}
\usepackage{longtable}
\usepackage{lscape}
\usepackage{booktabs}
\usepackage{natbib}
\usepackage{natbibspacing}
\usepackage[toc,page]{appendix}
\usepackage{makecell}

\title{Eleatic School}
\date{}

\linespread{1.5}
\geometry{left=2cm,right=2cm,top=2cm,bottom=2cm}

\begin{document}

  \maketitle
  
  \newpage

  \linenumbers

Eleatic school takes its name from Elea, a city in southern Italy, the hometown of Parmenides. 
Philosophers of this school think that change is impossible and unthinkable. 
Although the found of the school is attributed to Parmenides (515 BC-?), the thesis of permanence could be seen in works of Xenophanes (570 BC-478 BC).

\section{Xenophanes}
Nowadays, Xenophanes’ argument about God is often referred as criticism to superstitions. 
He wrote: 

\textit{Yes, and if oxen or lions had hands, and could paint with their hands and produce works of art as men do, horses would paint the forms of the gods like horses and oxen like oxen. Each would represent with bodies according to the form of each. ······So the Ethiopians make there gods black and snub-nosed; the Thracians give theirs red hair and blue eyes.}

But his real aim is to show the unity of God, while mortals think that the gods are born as they are, and have perceptions like theirs, and voice and form. 

\newline

God is one, eternal and unlike mortals in both mind and form. 
He sees all over, thinks all over, hears all over and governs all things by the thought of his mind. 
He is unlimited because nothing is beside him, while he is limited because he is a perfect form, not a formless infinite. 
He is immovable as a whole, because motion is in contrast with his unity, but there is motion or change in his parts. 
God is the eternal principle of the world. 
In other words, God is the world. 
However, Xenophanes never solved the contrary between the permanent God, who is the world, and the sensible world fulfilled with changes.

\newline

Xenophanes invites an eternal God, who is the first principle of the world and governs everything. 
However, he is interested in reducing God to the level of the forces of nature instead of explaining everything as the will of God and do offer some scientific theories. 
For example, from evidence of shells and imprints of sea products in stone, he infers that at one time the earth was mingled with the sea, but in the course of time it rose above the sea.
  
\section{Parmenides}
Parmenides challenges Heraclitus’ teach that nothing is permanent. 
If change is possible, it means that things are and are not. 
That is impossible, for a thing cannot both be and not be. 
So change is impossible. 
There is also another line of argument: 
If being has become, it must has become from being or become from non-being. 
The latter means that being has become from nothing, which is impossible. 
As for the former, it means that being has become from itself, which is equivalent to saying that being is identical with itself, and thus has always been.

\newline

Since change is impossible, everything remains what it is. 
Hence, there can be only one eternal, underived and unchangeable being, because if the being is plural, motion, the change of positions is inevitable. 
Since being is one, it must be continuous and indivisible because there is not anything in it but being. 
There is no break in being, for if there is, the break itself would become being, going against the doctrine that being is one. 
So again being is continuous. 
Furthermore, being must be immovable, for there is no empty space, non-being, for it to move in. 
It is also because of the doctrine that change and motion is impossible. 
Besides, thought and being are one, for what cannot be thought, cannot be; and what cannot be, cannot be thought.

\newline

The last problem: 
since change is inconceivable and being is one and permanent by reasoning, how are we supposed to understand the phenomena of change and motion observed by our senses? 
The answer is: 
our senses are cheating us. 
The world of sense is just an illusion and we should believe in our reasoning firmly.

\newline
Parmenides might be a victim of the linguistic fallacy as pointed by Betrand Russell. 
Parmenides' first line of argument for the impossibility of change says: 
"If change is possible, it means that things are and are not. That is impossible, for a thing cannot both be and not be. So change is impossible." 
However, in the flux of change, things first are and then are not. 
As for the second line of argument, it says: 
"If being has become, it must has become from being or become from non-being. The latter means that being has become from nothing, which is impossible. As for the former, it means that being has become from itself, which is equivalent to saying that being is identical with itself, and thus has always been." 
Here, it is pre-assumed that there could be only being, instead of beings.

\newline

Perhaps the main lesson we could learn from Parmenides is a negative one. 
We must be careful with the language we use when reasoning and deducting, for it may cause confusions and mistakes. 

\section{Zeno}
Zeno (490 BC-430 BC) is famous for his four paradoxes led by the possibility of motion. 
The first one is that one can never reach a goal from a start point because he must passing through the infinite number of points. 
The second one is about Achilles of great speed and the tortoise. 
The intelligent runner cannot reach a tortoise because in any interval of time, when he is moving from his start point to the tortoise’s start point, the tortoise has moved a certain distance. 
The third is known as the paradox of the moving arrow. 
When an arrow is flying to its target, at any given moment, it is in a definite position in space. 
Namely, it is at rest. 
As a result, it can never reach its goal. 
It is implicated in the first three paradoxes that time and space are discrete instances and points respectively, instead of being continuous as we conceive today. 
The last one paradox is about relative motion. 
The observation of motion could be different, depending on whether it is observed from a position at rest or moving at different speeds. 

\newline

Zeno also proofed that plurality is impossible because it leads to contradictions in a more general way compared with his paradoxes. 
Here is his refutation of plurality: 
If the whole being is plural, it is made up of many parts. 
It would lead to such a contradiction. 
The whole being is infinitely small, because any totality of being can be divided to infinitely small parts, and the sum of infinitely small parts can only be infinitely small. 
On the other hand, the whole being is infinitely large, because any totality can always be added with other parts to make it larger. 
In conclusion, the being is one.
  
\end{document}