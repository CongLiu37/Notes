\documentclass[11pt]{article} 

\usepackage{setspace}
\usepackage[colorlinks,linkcolor=blue,anchorcolor=red,citecolor=black]{hyperref}
\usepackage{lineno}
\usepackage{booktabs}
\usepackage{graphicx}
\usepackage{float}
\usepackage{floatrow}
\usepackage{subfigure}
\usepackage{caption}
\usepackage{subcaption}
\usepackage{geometry}
\usepackage{multirow}
\usepackage{longtable}
\usepackage{lscape}
\usepackage{booktabs}
\usepackage{natbib}
\usepackage{natbibspacing}
\usepackage[toc,page]{appendix}
\usepackage{makecell}

\title{Skepticism}
\date{}

\linespread{1.5}
\geometry{left=2cm,right=2cm,top=2cm,bottom=2cm}

\begin{document}

  \maketitle

  \linenumbers
  
The last school in Ethical Movement is skepticism preached by Pyrrho (365 BC-270 BC) of Elis. 
Consistent with Epicureanism and Stoicism, skepticism concerns with ethics, but it goes towards a different direction. 
Instead of basing ethics on logic and metaphysics, skepticism declares that we cannot know any truth and gives up cognition. 
This is the skeptic way of seeking the peace of mind.

\newline

The basic doctrine of skepticism is that we cannot know the nature of things. 
Our senses tell us what things appear to us instead of what they are in themselves. 
We cannot know to what extent that our sensations are consistent with objects, because we never get out of our sensations. 
Our thoughts are not reliable, for we cannot find any criterion to tell the true from the false. 
Epicureanism makes sensations the criterion, but our senses may deceive us. 
Stoicism make sensations carrying convictions the criterion, but perceptions with nothing to correspond to can be as clear, distinct and evident as the true ones.

\newline

Hence, we cannot know anything with certainty. 
So we should suspend judgment. 
What objects appear to us is sufficient for practical purposes, and we do not need, and cannot, conduct further investigations. 
Thus, we are saved from unhappiness by ceasing striving for knowledge of certainty. 
Such an attitude of making no comment and judgment brings us the peace of mind. 
Overall, skepticism advises the abandonment of all knowledge.

\end{document}