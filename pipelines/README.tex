\documentclass[11pt]{article}

% \usepackage[UTF8]{ctex} % for Chinese 

\usepackage{setspace}
\usepackage[colorlinks,linkcolor=blue,anchorcolor=red,citecolor=black]{hyperref}
\usepackage{lineno}
\usepackage{booktabs}
\usepackage{graphicx}
\usepackage{float}
\usepackage{floatrow}
\usepackage{subfigure}
\usepackage{caption}
\usepackage{subcaption}
\usepackage{geometry}
\usepackage{multirow}
\usepackage{longtable}
\usepackage{lscape}
\usepackage{booktabs}
\usepackage{natbibspacing}
\usepackage[toc,page]{appendix}
\usepackage{makecell}
\usepackage{amsfonts}
 \usepackage{amsmath}
\usepackage[utf8]{inputenc}
\usepackage{amssymb}
\usepackage{amsthm}
\usepackage{enumerate}
\usepackage{comment}

\usepackage[backend=bibtex,style=authoryear,sorting=nyt,maxnames=1]{biblatex}
\bibliography{README} % Reference bib

\title{Pipelines}
\author{}
\date{}

\linespread{1.5}
\geometry{left=2cm,right=2cm,top=2cm,bottom=2cm}

\setlength\bibitemsep{0pt}

\begin{document}
\begin{sloppypar}
  \maketitle

  \linenumbers
\section{Pipeline for genome decontamination (DeCon)}
\subsection{Introduction}
Pipeline DeCon is designed to retrieve genomic sequences of target \textbf{phylum} from metagenomic assembly of paired next generation sequencing (NGS) reads. 
First, NGS reads are mapped to assembly by minimap2 \parencite{li2018minimap2}, generating BAM file. 
Second, SprayNPray \parencite{garber2022spraynpray} is used to compute coverage, GC content and coding density of each contigs. 
Third, all contigs are searched against non-redundant (nr) database by DIAMOND \parencite{buchfink2015fast} and assigned to phyla by MEGAN \parencite{huson2007megan}. 
Forth, contigs below 400 base pair (bp) are removed. 
Then a decision tree classifier is trained, taking coverage, GC content and coding density as training features and phylum assignment as target value. 
This classifier is used to compute phylum assignment of contigs that DIAMOND and MEGAN failed to compute assignments. 
Fifth, contigs assigned to the target phylum are retrieved. 
QUAST \parencite{gurevich2013quast} and BUSCO \parencite{simao2015busco} are used to evaluate retrieved genome. 
Distributions of contig coverage and GC content of retrieved genome are plotted. 
\subsection{Dependencies}
\textbf{Softwares} \newline
R \newline
Python \newline
minimap2 \newline
SAMtools \newline
SprayNPray \newline
DIAMOND \newline
MEGAN (blast2rma & rma2info scripts) \newline
seqkit \newline
QUAST \newline
BUSCO \newline
\par
\textbf{Databases} \newline
DIAMOND database (nr) \newline
MEGAN database \newline
BUSCO database \newline
\par
\textbf{Python modules} \newline
numpy \newline
pandas \newline
scikit-learn \newline
\par
\textbf{R packages} \newline
reticulate \newline
stringr \newline
ggplot2 \newline
ggExtra \newline
\par
\subsection{Usage}
Modify configuration file (templated as DeCon\_conf.R), and run \newline
Rscript \textif{path}/DeCon\_pipeline.R \textif{path}/DeCon\_main.R \textif{path}/DeCon\_main.py \textif{path}/DeCon\_conf.R

\section{Pipeline for calling protein-coding genes from genome (ProtGeneCall)}
\subsection{Introduction}
Pipeline ProtGeneCall is designed to call protein-coding genes from genome, combining protein-genome alignments, transcriptome-genome alignments and \textit{ab initio} gene predictions. 
First, repeat elements are identified by RepeatModeler \parencite{smit2015repeatmodeler} and masked by RepeatMasker \parencite{smit2015repeatmasker}. 
Masked genome is used for downstream analysis. 
Second, proteins of closely related species are mapped to the masked genome by miniprot \parencite{li2023protein}. 
Third, paired RNA-sequencing (RNA-seq) reads are mapped to masked genome by Hisat2 \parencite{kim2019graph}. 
Forth, transcriptome-genome alignments are computed by StringTie \parencite{pertea2015stringtie}. 
Fifth, gene structures are predicted from transcriptome-genome alignments by TransDecoder \parencite{haas2016transdecoder}, combining searching against UniRef and PfamA databases. 
Sixth, AUGUSTUS \parencite{stanke2003gene} is trained with gene structures from TransDecoder to compute gene predictions. 
Seventh, BRAKER \parencite{hoff2019whole} is trained with RNA-seq mapping to call genes. 
Eighth, GALBA \parencite{hoff2019whole} is trained with proteins of closely related species. 
Ninth, protein-genome alignments from miniprot, transcript-genome alignments from Hisat2-StringTie, and \textit{ab initio} gene predictions from TransDecoder, AUGUSTUS, BRAKER and GALBA are integrated into consensus gene structures by EvidenceModeler \parencite{haas2008automated}. 
Tenth, genes supported by only one \textit{ab initio} predictor and lack protein/RNA-seq evidence are removed. 
Eleventh, PASA \parencite{haas2008automated} is run twice to update filtered gene structures from EvidenceModeler. 
Twelfth, genes with in-frame stop codons are removed and the predicted peptide set is evaluated by BUSCO \parencite{simao2015busco}.
\subsection{Dependencies}
\textbf{Softwares} \newline
R \newline
Python \newline
RepeatModeler \newline
RepeatMasker \newline
miniprot \newline
Hisat2 \newline
SAMtools \newline
StringTie \newline
TransDecoder \newline
HMMER \newline
DIAMOND \newline
AGAT \newline
AUGUSTUS \newline
BLAST+ \newline
GALBA \newline
BRAKER \newline
EvidenceModeler \newline
BUSCO \newline
gffread \newline
seqkit \newline
MAKER \newline
\par
\textbf{Databases} \newline
DIAMOND database (UniRef) \newline
Pfam-A \newline
\par
\textbf{External scrpts} \newline
cufflinks\_gtf\_to\_alignment\_gff3.pl from EvidenceModeler \newline
augustus\_GFF3\_to\_EVM\_GFF3.pl from EvidenceModeler \newline
gth2gtf.pl from AUGUSTUS \newline
computeFlankingRegion.pl from AUGUSTUS \newline
gff2gbSmallDNA.pl from AUGUSTUS \newline
gtf2aa.pl from AUGUSTUS \newline
simplifyFastaHeaders.pl from AUGUSTUS \newline
aa2nonred.pl from AUGUSTUS \newline
filterGenesIn.pl from AUGUSTUS \newline
autoAug.pl from AUGUSTUS \newline
evm\_evidence.py in this GitHub \newline
\par
\textbf{R packages} \newline
stringr \newline
parallel \newline
\par
\subsection{Usage}
Modify configuration file (templated as ProtGeneCall\_conf.R), and run \newline
Rscript \textif{path}/ProtGeneCall\_pipeline.R \textif{path}/ProtGeneCall\_main.R \textif{path}/ProtGeneCall\_conf.R

\section{Pipeline for calling repeat elements from genome (RepCall)}
\subsection{Introduction}
Pipeline RepCall is designed to call repeat elements genes from genome. 
First, miniature inverted-repeat transposable elements (MITE) are called by MITE-Hunter \parencite{han2010mite}. 
Second, long terminal repeats (LTRs) are identified by incorporating LTR\_FINDER\_parallel \parencite{ou2019ltr_finder_parallel}, LTRharvest \parencite{ellinghaus2008ltrharvest} and LTR\_retriever \parencite{ou2018ltr_retriever}. 
Third, identified MITEs and LTRs are masked by RepeatMasker \parencite{smit2015repeatmasker}. 
Forth, RepeatModeler \parencite{smit2015repeatmodeler} is used to further identify repeats in the masked genome. 
Fifth, the locations of MITEs, LTRs and repeats from RepeatModeler are identified by RepeatMasker and all repeats are incorporated into a consensus library. 
\subsection{Dependencies}
\textbf{Softwares} \newline
R \newline
seqkit \newline
MITE-Hunter \newline
LTR\_FINDER\_parallel \newline
LTRharvest \newline
LTR\_retriever \newline
RepeatMasker \newline
RepeatModeler \newline
\par
\subsection{Usage}
Modify configuration file (templated as RepCall\_conf.R), and run \newline
Rscript \textif{path}/RepCall\_pipeline.R \textif{path}/RepCall\_main.R \textif{path}/RepCall\_conf.R

\section{Pipeline for calling non-coding RNA (ncRNAcall)}
\subsection{Introduction}
Pipeline ncRNAcall is designed to call non-coding RNA (ncRNA) from genome. 
First, transfer RNA (tRNA) is identified by tRNAscan-SE \parencite{lowe1997trnascan}. 
Second, microRNA is called by miRNAture \parencite{velandia2021mirnature}. 
Third, target genes of microRNA are identified by searching microRNA against annotated three prime untranslated regions (3'UTR) by miRanda \parencite{enright2003microrna}. 
Forth, Infernal \parencite{nawrocki2013infernal} searches against Rfam \parencite{kalvari2021rfam} database to call other non-coding RNA, \textit{e.g.} ribosomal RNA (rRNA) and small nuclear RNA (snRNA). 
Fifth, all results are incorporated together. 
\subsection{Dependencies}
\textbf{Softwares} \newline
R \newline
tRNAscan-SE \newline
biocode \newline
miRNAture \newline
miRanda \newline
bedtools \newline
seqkit \newline
Infernal \newline
\par
\textbf{Databases} \newline
miRNAture database \newline
Rfam database \newline
\par
\textbf{R packages}
parallel \newline
stringr \newline
\par
\subsection{Usage}
Modify configuration file (templated as ncRNAcall\_conf.R), and run \newline
Rscript \textif{path}/ncRNAcall\_pipeline.R \textif{path}/ncRNAcall\_main.R \textif{path}/ncRNAcall\_conf.R

\printbibliography
\end{sloppypar}
\end{document}