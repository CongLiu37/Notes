\documentclass[11pt]{article}

% \usepackage[UTF8]{ctex} % for Chinese 

\usepackage{setspace}
\usepackage[colorlinks,linkcolor=blue,anchorcolor=red,citecolor=black]{hyperref}
\usepackage{lineno}
\usepackage{booktabs}
\usepackage{graphicx}
\usepackage{float}
\usepackage{floatrow}
\usepackage{subfigure}
\usepackage{caption}
\usepackage{subcaption}
\usepackage{geometry}
\usepackage{multirow}
\usepackage{longtable}
\usepackage{lscape}
\usepackage{booktabs}
\usepackage{natbib}
\usepackage{natbibspacing}
\usepackage[toc,page]{appendix}
\usepackage{makecell}
\usepackage{amsfonts}
 \usepackage{amsmath}

\title{Insect immunology and hematopoiesis}
\author{}
\date{}

\linespread{1.5}
\geometry{left=2cm,right=2cm,top=2cm,bottom=2cm}

\bibliographystyle{plain}
\bibliographystyle{unsrtnat}
\bibliographystyle{plainnat}
\bibliographystyle{dinat}
\bibliographystyle{abbrvnat}
\bibliographystyle{rusnat}
\bibliographystyle{ksfh_nat}
\setcitestyle{round}

\begin{document}
\begin{sloppypar}
  \maketitle

  \linenumbers

The most encompassing physical barrier of insects is the cuticle. 
This chitinous, hydrophobic material forms the exoskeleton, and also lines foregut, hindgut and tracheal system. 
Pathogens enter body through cuticle via wound or enzymatic digestion. 
Ingestion is another routine for pathogen entrance. 

\par

Multiple insect cells and tissues are involved in immunity. 
Hemocytes are the primary immune cells. 
They circulate with hemolymph (circulating hemocytes) or attach to tissues (sessile hemocytes). 
These cells drive cellular and humoral immunity. 
Fat body is composed of loosely associated cells that are rich in lipids and glycogen, lines the integument of hemocoel. 
It functions in energy storage and synthesis of vitellogenin precursors that are required for egg production. 
Fat body also produces antimicrobial peptide. 
Midgut mainly functions in digestion and nutrition absorption. 
It produces nitric oxide synthesis and other lytic effectors killing pathogens. 
Salivary glands are primarily involved in feeding and usually located in the anterior of thorax. 
It is involved in immunity.

\par

\section{Pattern recognition receptors (PRRs)}
Immune responses are initiated by recognition of pathogen-associated molecular patterns (PAMPs) by pattern recognition receptors (PRRs). 
Among PRR families are 
\newline
(1) PGRPs: peptidoglycan recognition proteins; \newline
(2) immunoglobulin domain proteins; \newline
(3) FREPs: fibrinogen-related proteins, also known as fibrinogen domain immunolectins (FBNs); \newline
(4) TEPs: thioester-containing proteins; \newline
(5) betaGRP: beta-1,3-recognition proteins, also known as Gram-negative bacterial-binding proteins ( GNBPs); \newline
(6) galectins: bind specifically to beta-galactoside sugars; \newline
(7) CTLs: C-type lectins; \newline
(8) leucin-rich repeat (LRR) containing proteins; \newline
(9) DSCAMs: down syndrome cell adhesion molecules, include draper and eater in \textit{Drosophila melanogaster}; \newline
(10) Nimrod proteins; \newline
(11) MLs: MD-2-like proteins, also known as Niemann-pick type C-2 proteins, possess myeloid-differentiation-2-related lipid-recognition domains involved in recognizing lipopolysaccharide; \newline
(12) SRs: scavenger receptors, include croquemort and peste in \textit{Drosophila melanogaster}; \newline
(13) integrins.

\section{Toll signaling}
Toll pathway functions in both development and immunity. 
In immunity, Toll signaling is initiated when PRR activates 
\newline
(1) SPZ: Spatzle/spaetzle, extracellular cytokine; \newline
SPZ binds cellular receptor 
\newline
(2) TLR: toll-like receptors, also known as Toll. \newline 
TLR activates downstream cascade including 
\newline
(3) MyD88: myeloid differentiation primary response 88; \newline
(4) Tube; \newline
(5) Pelle, orthologous to several human genes including interleukin 1 receptor associated kinase 1 (IRAK1); \newline
(6) Dorsal, orthologous to several human genes including RELA (RELA proto-oncogene, NF-kappaB subunit) and RELB (RELB proto-oncogene, NF-kappaB subunit); \newline 
(7) Dif:Dorsal-related immune factor, orthologous to several human genes including RELA (RELA proto-oncogene, NF-kB subunit) and RELB (RELB proto-oncogene, NF-kB subunit). \newline
The inhibitor of Toll signaling is 
\newline
(8) Cactus orthologous to several human genes including NF-kappaB inhibitor alpha (NFKBIA); \newline
Toll signaling is effective in combating Gram-positive bacteria, fungi and viruses. 

\section{Imd Signaling}
Imd signaling is activated by membrane receptor PGRP-LC, followed by intracellular signaling including 
\newline
(1) Imd: immune deficiency; \newline
(2) TAK1: transforming growth factor (TGF)-beta activated kinase 1, orthologous to human mitogen-activated protein kinase kinase kinase 7 (MAP3K7); \newline
(3) IKKgamma: inhibitor of NF-kappaB (IkappaB) kinase gamma, also known as Kenny in \textit{Drosophila melanogaster}, orthologous to human IKKgamma and optineurin; \newline
(4) IKKbeta: IkappaB kinase beta; \newline
(5) Fadd: fas-associated death domain; \newline
(6) Dredd: death-related ced3/Nedd2-like caspase, orthologous to several human genes including caspase 10; \newline
and finally activates NF-kappaB transcription factor 
\newline
(7) Relish, orthologous to several human genes including NFKB2 (nuclear factor kappa B subunit 2). \newline
The inhibitor of Imd signaling is 
\newline
(8) Caspar, orthologous to human fas-associated factor 1 (FAF1). \newline
Imd signaling is effective in combating Gram-negative bacteria and viruses. 

\section{JAK/STAT signaling}
JAK/STAT signaling functions in development and immunity. 
In immunity, JAK/STAT signaling begins with extracellular cytokine 
\newline
(1) Unpaired \newline
that activates 
\newline
(2) Domeless. \newline
Domeless is phosphorylated by 
\newline
(3) Hopscotch: orthologous to several human genes including JAK1 (Janus kinase 1) and JAK3 (Janus kinase 3). \newline
Hopscotch activates transcription factor activity of 
\newline
(4) Stat: signal-transducer and activator of transcription protein. \newline
Inhibitors of JAK/STAT signaling are 
\newline
(5) Socs: suppressor of cytokine signaling; \newline
(6) Pias: protein inhibitor of activated Stat as E3 SUMO-protein ligase, known as suppressor of variegation 2-10 (Su(var)2-10) in \textit{Drosophila melanogaster}. \newline
JAK/STAT signaling activates antimicrobial genes like nitric oxide synthase and functions in antibacterial and antiviral responses. 

\section*{Phagocytosis}
Phagocytosis is a rapid progress conducted by hemocytes. 
PRRs that have been shown to be involved in phagocytosis include TEPs, Nimrods, DSCAMs, beta-integrins and PGRPs. 
The intracelular signaling in phagocytosis remains poorly understood. 
In mosquitoes, 
\newline
(1) CED2: cell death abnormality 2; \newline
(2) CED5; \newline
(3) CED6 \newline 
are involved in signaling regulate internalization of bacteria (Moita \textit{et al.}, 2005). 

\section*{Melanization}
Melanization is an enzymatic process involved in cuticle hardening, egg chorion tanning, wound healing and immunity and is mainly conducted by hemocytes. 
In immunity, melanization functions in killing bacteria, fungi, protozoa parasites, nematode worms and parasitoid wasps. 
It is manifested as a darkened proteinaceous capsule that surrounds pathogens, and kills pathogens via oxidative damage or starvation. 
Melanin synthesis pathway includes: 
\newline
(1) PAH: phenylalanine hydroxylase, also known as phenylalanine 4-monooxygenase, or Henna in \textit{Drosophila melanogaster}, hydroxylates phenylalanine to tyrosine; \newline
(2) PO: phenoloxidase, formed via cleavage of prophenoloxidase (PPO), oxidizes tyrosine into dihydroxyphenylalanine (Dopa), and further into dopaquinone, and further into dopachrome non-enzymatically; \newline
(3) DCE: dopachrome conversion enzyme or dopachrome decarboxylase/tautomerase, known as yellow in \textit{Drosophila melanogaster}, decarboxylates dopachrome into 5,6-dihyroxyindole (DHI). \newline
Another line from Dopa to DHI is 
\newline
(4) DDC: dopa decarboxylase, aromatic L-amino acid decarboxylase (AADC or AAAD), tryptophan decarboxylase or 5-hydroxytryptophan decarboxylase, decarboxylates dopa into dopamine, which is oixidized into dopaminequinone by PO, and further converts into dopaminechrome non-enzymatically, and further into DHI non-enzymatically. \newline
Following PO-meidated DHI oxidation, indole-5,6-quinones polymerize and give rise to heteropolymer eumelanin. 
PO activity is tightly controlled. 
After PRR activation, PO is activated by a serine protease cascade including: 
\newline
(5) ModSp: modular serine protease that lacks clip domain but contains other domain for interactions; \newline
(6) cSP: clip domain-containing serine protease, includes \textit{Drosophila melanogaster} 
  snake, easter, serine protease 7 (SP7), serine protease immune response integrator (spirit), persephone, spatzle-processing enzyme (SPE), Gram-positive specific serine protease (grass), melanization protease 1 (MP1), hayan, Ser7, lethal (2) k05911, 
activated by ModSp cleavage and activates PO by cleavage. \newline
The inhibitor of PO is 
\newline
(7) serpin: serine protease inhibitors. \newline

\section*{Encapsulation}
Encapsulation is a cellular immune response against pathogens that are too large to be phagocytosed. 
In encapsulation, hemocytes attach to form a capsule surrounding pathogens. 
The capsule may be melanized. 
In Lepidoptera, hemocyte adhesion is dependent on binding of integrin to specific sites defined by Arg-Gly-Asp (RGD) sequence. 

\section*{Nodulation}
Nodulation is an immune response in which hemocyte adhere to large aggregrates of bacteria and form layers, usually followed by melanization. 
Underlying molecular mechanism of nodulation remains poorly understood, but it relies on eicosanoid-based signaling and extracellular matrix-like protein Noduler. 

\section*{Lysis}
Lysis of pathogens is resulted from disruption of cellular membrane by immune effectors including antimicrobial peptides (AMPs). 
AMPs are small secreted peptides, including 
\newline
(1) apisimin, attacin, cecropin, defensin, diptericin, drosocin, drosomycin, gambicin, gloverin, holitricin, jelleine, lebocin, melittin, metchnikowin, moricin, persulcatusin, ponericin, pyrrhocoricin, sapecin. 

\par

(2) Lysozymes \newline
are another family of proteins mediating lysis. 
Lysozymes hydrolyze beta-1,4-glycosidic linkage between N-acetylumuramic and N-acetylglucosamine of peptidoglycan, and therefore, mainly function in antibacterial responses. 
They are usually present in low, constitutive levels, and are transcriptionally up-regulated in immune responses. 

\par

Reactive species are effect in lysis. 
Synthesis of reactive species include 
\newline
(1) DUOX: dual oxidase, generates hydrogen peroxide; \newline
(2) NOX: NADPH oxidase, generates hydrogen peroxide; \newline
(2) NOS: nitric oxide synthase, generates nitric oxide; \newline
(3) SOD: superoxide dismutase, catalyzes the dismutation (or partitioning) of the superoxide radical into ordinary molecular oxygen and hydrogen peroxide; \newline
(4) peroxidase: also peroxide reductase, break up peroxides.  

\section*{RNA interference (RNAi)}
In RNA interference (RNAi) pathways, small RNA (sRNA) associates with Argonaute protein, forming RNA induced silencing comples (RISC). 
RISC recognizes targets by complementary bases, and silences targets in an Argonaute-mediated manner. 
In insects, there are three RNAi pathways: micro-RNA (miRNA), small-interfering-RNA (siRNA) and piwi-interacting-RNA (piRNA). 

\par

miRNA pathway is mainly involved in gene expression regulation. 
miRNA originates from nuclear genome, and is processed by nuclear protein 
\newline
(1) Dorsha; \newline
(2) Pasha: partner of Dosha. \newline
Matured miRNA relocates to cytoplasm, and is further processed by 
\newline
(3) Dicer 1; \newline
(4) Loquacious. \newline
Fully matured miRNA is loaded into 
\newline
(5) Argonaute 1. 

\par 

siRNA pathway is involved in defenses against viral dsRNA and transposonal elements. 
In antiviral responses, viral dsRNA is processed by 
\newline
(6) Dicer 2; \newline
(7) R2D2; \newline
forming siRNA, which is loaded into 
(8) Argonaute 2. 
In anti-transposonal elements, 
dsRNA is processed by Dicer 2 and Loquacious. 

\par

piRNA pathway is involved in defenses against transposonal element in germline. 
Primary piRNA is generated by cleavage of transposon transcripts by 
\newline
(9) zucchini. \newline
Matured piRNA is loaded into Argonaute proteins of PIWI sub-clade, \textit{i.e.} 
\newline
(10) Argonaute 3; \newline
(11) Aubergine; \newline
(12) Piwi: P-element induced wimpy testis. 

\section*{Autophagy}
Autophagy is a process of degradation of intracellular materials, and is involved in elimation of intracellular bacteria and viruses. 
The upstream signal in autophagy includes 
\newline
(1) PI3K: phosphatidylinositol 3-kinase; \newline
(2) AKT1; \newline
(3) TOR: target of rapamycin. \newline
This leads to activation of a complex containing 
\newline
(4) Atg1: autophagy-related (Atg) 1, protein kinase; \newline
(5) Atg13: phosphoprotein. 
\newline
Then autophagosome membrane is nucleated via a complex containing 
\newline
(6) Atg14; \newline
(7) Vps15: vacuolar protein sorting (Vps) 15; \newline
(8) Vps34: PI3K59F. \newline
Then autophagosome is enlongated, dependent on 
\newline
(9) Atg5; \newline
(10) Atg12; \newline
and conjugates phosphatidylethanolamine to 
\newline
(11) Atg8. \newline
In \textit{Drosophila}, autophagy defenses against vesicular stomatitis virus and Rift Valley fever virus, but enhances infection of Sindbis virus. 

\section*{Apoptosis}
Apoptosis is a form of programmed cell death. 
At molecular level, a complex containing 
\newline
(1) Dronc: death regulator Nedd2-like caspase; \newline
(2) Dark: death-associated APAF1-related killer. \newline
is formed. Dronc activates downstream caspase including 
\newline
(3) Drice: death related ICE-like caspase; \newline
(4) DCP1: death caspase-1. \newline
Other factors influencing apoptosis pathway including 
\newline
(5) Diap: death-associated inhibitor of apoptosis.
Apoposis often plays a role in antiviral responses.
\end{sloppypar}
\end{document}