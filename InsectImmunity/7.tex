\documentclass[11pt]{article}

% \usepackage[UTF8]{ctex} % for Chinese 

\usepackage{setspace}
\usepackage[colorlinks,linkcolor=blue,anchorcolor=red,citecolor=black]{hyperref}
\usepackage{lineno}
\usepackage{booktabs}
\usepackage{graphicx}
\usepackage{float}
\usepackage{floatrow}
\usepackage{subfigure}
\usepackage{caption}
\usepackage{subcaption}
\usepackage{geometry}
\usepackage{multirow}
\usepackage{longtable}
\usepackage{lscape}
\usepackage{booktabs}
\usepackage{natbib}
\usepackage{natbibspacing}
\usepackage[toc,page]{appendix}
\usepackage{makecell}
\usepackage{amsfonts}
 \usepackage{amsmath}

\title{Immune-related genes and pathways in disease-vector mosquitoes}
\author{}
\date{}

\linespread{1.5}
\geometry{left=2cm,right=2cm,top=2cm,bottom=2cm}

\begin{document}
  \maketitle

  \linenumbers
\textbf{Immunity-related genes:} 
285 \textit{Drosophila melanogaster} (Dm), 338 \textit{Anopheles gambiae} (Ag), and 353 \textit{Aedes aegypti} (Aa) genes from 31 gene families and functional groups implicated in classical innate immunity or defense functions such as apoptosis and response to oxidative stress. 

\newline

\textbf{Orthology groups (whole genome):} 
4951 orthologous trios (1:1:1 orthologs in the three species) and 886 mosquito-specific orthologous pairs (absent from Dm). 

\newline

\textbf{Orthology groups (immune-related):} 
91 trios and 57 pairs, plus a combined total of 589 paralogous genes in the three species. 

\newline

\textbf{Immune-related orthology trios are more divergent than that of whole genome:}
Phylogenetic distances of genes in each trio is measured by amino acid substitutions. 
With Dm as reference, immune-related trios of Ag and Am are more divergent (on average) compared with trios of whole genome, and several Ag immunity genes are considerably more divergent than their Aa orthologs.

\newline

\textbf{Large variation exists in different immune families in their proportions of orthologous trios,
mosquito-specific pairs and species-specific genes:}
(1) Predominantly trio orthologs: 
apoptosis inhibitors (IAPs), 
oxidative defense enzymes [
    superoxide dismutases (SODs), 
    glutathione peroxidases (GPXs), 
    thioredoxin peroxidases (TPXs), 
    heme-containing peroxidases (HPXs)],
class A and B scavenger receptors (SCRs). 
\newline
(2) Rarely trio orthologs: 
immune effectors, including three antimicrobial peptides.
\newline
(3) Intermediately: 
C-type lectins.

\newline

\textbf{Strong divergent evolution of immune recognition genes:}
Fruit fly and mosquito recognition proteins mostly form distinct clades within each gene family.

\end{document}