\documentclass[11pt]{article}

% \usepackage[UTF8]{ctex} % for Chinese 

\usepackage{setspace}
\usepackage[colorlinks,linkcolor=blue,anchorcolor=red,citecolor=black]{hyperref}
\usepackage{lineno}
\usepackage{booktabs}
\usepackage{graphicx}
\usepackage{float}
\usepackage{floatrow}
\usepackage{subfigure}
\usepackage{caption}
\usepackage{subcaption}
\usepackage{geometry}
\usepackage{multirow}
\usepackage{longtable}
\usepackage{lscape}
\usepackage{booktabs}
\usepackage{natbib}
\usepackage{natbibspacing}
\usepackage[toc,page]{appendix}
\usepackage{makecell}
\usepackage{amsfonts}
 \usepackage{amsmath}

\title{Abstracts}
\author{}
\date{}

\linespread{1.5}
\geometry{left=2cm,right=2cm,top=2cm,bottom=2cm}

\begin{document}
\begin{sloppypar}
  \maketitle

  \linenumbers

\textbf{Bosco-Drayon \textit{et al.}, 2012, Peptidoglycan sensing by the receptor PGRP-LE in the \textit{Drosophila} gut induces immune responses to infectious bacteria and tolerance to microbiota.} \newline
In \textit{Drosophila}, peptidoglycan recognition protein (PGRP)-LE senses peptidoglycan, and induces NF-kappaB dependent responses to infectious bacteria, but also tolerance to symbionts via up-regulation of pirk and PGRP-LB, which inhibits IMD signaling. 
Loss of PGRP-LE-mediated detection of bacteria in the gut results in systemic immune activation, which can be rescued by overexpressing PGRP-LB in the gut.

\par

\textbf{Wang \textit{et al.}, 2009, Interactions between mutualist \textit{Wigglesworthia} and tsetse peptidoglycan recognition protein (PGRP-LB) influence trypanosome transmission.} \newline
Tsetse flies have coevolved with mutualistic endosymbiont \textit{Wigglesworthia glossinidiae}. 
A tsetse peptidoglycan recognition protein (PGRP-LB) is crucial for symbiotic tolerance and trypanosome infection processes. 
Tsetse \textit{pgrp-lb} is expressed in the \textit{Wigglesworthia}-harboring organ (bacteriome) in the midgut, and its level of expression correlates with symbiont numbers. 
Adult tsetse cured of \textit{Wigglesworthia} infections have significantly lower \textit{pgrp-lb} levels than corresponding normal adults. 
RNA interference (RNAi)-mediated depletion of \textit{pgrp-lb} results in the activation of the immune deficiency (IMD) signaling pathway and leads to the synthesis of antimicrobial peptides (AMPs), which decrease \textit{Wigglesworthia} density. 
Depletion of \textit{pgrp-lb} also increases the host's susceptibility to trypanosome infections. 
Finally, parasitized adults have significantly lower \textitP{pgrp-lb} levels than flies, which have successfully eliminated trypanosome infections. 
When both PGRP-LB and IMD immunity pathway functions are blocked, flies become unusually susceptible to parasitism. 
Based on the presence of conserved amidase domains, tsetse PGRP-LB may scavenge the peptidoglycan (PGN) released by \textit{Wigglesworthia} and prevent the activation of symbiont-damaging host immune responses. 
In addition, tsetse PGRP-LB may have an anti-protozoal activity that confers parasite resistance. 

\par

\textbf{Maire \textit{et al.}, 2018, An IMD-like pathway mediates both endosymbiont control and host immunity in the cereal weevil \textit{Sitophilus spp.}.} \newline
In the cereal weevil \textit{Sitophilus spp.}, which houses \textit{Sodalis pierantonius}, endosymbionts are secluded in specialized host cells, the bacteriocytes that group together as an organ, the bacteriome. 
At standard conditions, the bacteriome highly expresses the coleoptericin A (colA) antimicrobial peptide (AMP), which was shown to prevent endosymbiont escape from the bacteriocytes. 
However, following the insect systemic infection by pathogens, the bacteriome upregulates a cocktail of AMP encoding genes, including colA. 
The regulations that allow these contrasted immune responses remain unknown. 
Here, evidence shows that an IMD-like pathway is conserved in two sibling species of cereal weevils, \textit{Sitophilus oryzae} and \textit{Sitophilus zeamais}. 
RNA interference (RNAi) experiments showed that \textit{imd} and \textit{relish} genes are essential for 
(i) colA expression in the bacteriome under standard conditions, 
(ii) AMP up-regulation in the bacteriome following a systemic immune challenge, 
(iii) AMP systemic induction following an immune challenge. 
Histological analyses also showed that \textit{relish} inhibition by RNAi resulted in endosymbiont escape from the bacteriome, strengthening the involvement of an IMD-like pathway in endosymbiont control. 
It is concluded that \textit{Sitophilus}’ IMD-like pathway mediates both the bacteriome immune program involved in endosymbiont seclusion within the bacteriocytes and the systemic and local immune responses to exogenous challenges. 
This work provides an example of how a conserved immune pathway, initially described as essential in pathogen clearance, also functions in the control of mutualistic associations.

\par

\textbf{Zaidman-Rémy \textit{et al.}, 2018, What can a weevil teach a fly, and reciprocally? Interaction of host immune systems with endosymbionts in \textit{Glossina} and \textit{Sitophilus}} \newline


\end{sloppypar}
\end{document}