\documentclass[11pt]{article}

% \usepackage[UTF8]{ctex} % for Chinese 

\usepackage{setspace}
\usepackage[colorlinks,linkcolor=blue,anchorcolor=red,citecolor=black]{hyperref}
\usepackage{lineno}
\usepackage{booktabs}
\usepackage{graphicx}
\usepackage{float}
\usepackage{floatrow}
\usepackage{subfigure}
\usepackage{caption}
\usepackage{subcaption}
\usepackage{geometry}
\usepackage{multirow}
\usepackage{longtable}
\usepackage{lscape}
\usepackage{booktabs}
\usepackage{natbib}
\usepackage{natbibspacing}
\usepackage[toc,page]{appendix}
\usepackage{makecell}
\usepackage{amsfonts}
 \usepackage{amsmath}

\title{Abstracts}
\author{}
\date{}

\linespread{1.5}
\geometry{left=2cm,right=2cm,top=2cm,bottom=2cm}

\begin{document}
\begin{sloppypar}
  \maketitle

  \linenumbers

\textbf{1. Bosco-Drayon \textit{et al.}, 2012, Peptidoglycan sensing by the receptor PGRP-LE in the \textit{Drosophila} gut induces immune responses to infectious bacteria nad tolerance to microbiota.} \newline
In \textit{Drosophila}, peptidoglycan recognition protein (PGRP)-LE senses peptidoglycan, and induces NF-kappaB dependent responses to infectious bacteria, but also tolerance to symbionts via up-regulation of pirk and PGRP-LB, which inhibits IMD signaling.

\par

\textbf{2. }

\end{sloppypar}
\end{document}