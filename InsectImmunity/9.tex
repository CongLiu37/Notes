\documentclass[11pt]{article}

% \usepackage[UTF8]{ctex} % for Chinese 

\usepackage{setspace}
\usepackage[colorlinks,linkcolor=blue,anchorcolor=red,citecolor=black]{hyperref}
\usepackage{lineno}
\usepackage{booktabs}
\usepackage{graphicx}
\usepackage{float}
\usepackage{floatrow}
\usepackage{subfigure}
\usepackage{caption}
\usepackage{subcaption}
\usepackage{geometry}
\usepackage{multirow}
\usepackage{longtable}
\usepackage{lscape}
\usepackage{booktabs}
\usepackage{natbib}
\usepackage{natbibspacing}
\usepackage[toc,page]{appendix}
\usepackage{makecell}
\usepackage{amsfonts}
 \usepackage{amsmath}

\title{Immunity and other defenses in pea aphids, \textit{Acyrthosiphon pisum}}
\author{}
\date{}

\linespread{1.5}
\geometry{left=2cm,right=2cm,top=2cm,bottom=2cm}

\begin{document}
  \maketitle

  \linenumbers
Pea aphids appear to be missing genes present in insect genomes characterized to date and thought critical for recognition, signaling and killing of microbes. 
In line with results of gene annotation, experimental analyses designed to characterize immune response through the isolation of RNA transcripts and proteins from immune-challenged pea aphids uncovered few immune-related products. 
Gene expression studies, however, indicated some expression of immune and stress-related genes.

\newline

In the fruit fly \textit{Drosophila melanogaster}, recognition of an invasive microbe leads to signal production via four pathways (Toll, IMD, JNK, and JAK/STAT) (Boutros \textit{et al.}, 2002). 
Each pathway is activated in response to particular pathogens (Dionne \textit{et al.}, 2008). 
Signaling triggers the production of multitude effectors, including, most notably, antimicrobial peptides (AMPs). 
In insect genomes annotated to date, these pathways appear well conserved, with most of the key components found across flies (\textit{Drosophila spp.}) (Sackton \textit{et al.}, 2007), mosquitoes (\textit{Aedes aegypti}, \textit{Anopheles gambiae}) (Waterhouse \textit{et al.}, 2007; Christophides \textit{et al.}, 2002), bees (\textit{Apis mellifera}) (Evans \textit{et al.}, 2006) and beetles (\textit{Tribolium castaneum}) (Zou \textit{et al.}, 2007).

\newline

The cellular component of pea aphids’ innate immune response may also be different to that seen in other insects. 
While many insects encapsulate parasitoid wasp larvae, smothering them to death with hemocytes, aphids appear not to have this layer of protection (Bensadia \textit{et al.}, 2006; Carver \textit{et al.}, 1988). 
Aphids, however, appear to recruit some hemocytes to parasitoid eggs, suggesting that cellular immunity may play an alternative, though possibly more limited, role (Bensadia \textit{et al.}, 2006).

\newline

There is evidence that pea aphid has some defense systems common to other arthropods, \textit{e.g.}, the Toll and JAK/STAT signaling pathways, HSPs, ProPO. 
However, several of the genes thought central to arthropod innate immunity are missing in pea aphid, including PGRPs, the IMD signaling pathway, defensins, c-type lysozymes. 

\newline

The failure of finding aphid homologs to many insect immune genes can be resulted from large evolutionary distance between pea aphid and taxa used as reference (divided ~100 million years ago). 
However, similar homology-search based method successfully detected immune-related genes in even more divergent insects. 
Another explanation for lack of immune genes is that pea aphid mount an alternative but equal immunity. 
However, functional analysis, together with Altincicek \textit{et al.}, 2008, found little evidence for an alternative response to \textit{E. coli} infection. 

\newline

Altincicek \textit{et al.}, 2008 proposed three hypotheses on the ecological success of pea aphid with the possibility of lacking a strong immunity. 
First, aphids feeds on plant sap which is often sterile, leading to reduced risk for encountering pathogens. 
However, aphids are capable of acquiring pathogenic bacteria from the surface of their host plants’ leaves (Stavrinides \textit{et al.}, 2009), and aphids become host to a diverse assemblage of bacteria and fungi under stressful conditions (Nakabachi \textit{et al.}, 2003). 
Furthermore, \textit{Sitophilus} weevils, which when challenged with \textit{E. coli} significantly up-regulate immune genes (Anselme \textit{et al.}, 2008), spend their entire larval and nymph stages within sterile cereal grains, indicating that a sterile diet is not likely to explain the absence of antibacterial defenses in aphids. 
Second, aphid symbionts may provide protection against pathogens, \textit{e.g.} pea aphid has been reported to be protected against fungal pathogens by the facultative symbiotic Gram-negative bacterium \textit{Regiella insecticola} (Scarborough et al., 2005) and also against the parasitoid wasp \textit{Aphidius ervi} by the facultative symbiotic Gram-negative bacterium \textit{Hamiltonella defensa} (Oliver et al., 2005). 
This seems plausible regards to the cost of immune gene expression versus the benefit of protection by the secondary endosymbionts. 
However, it does not explain how the secondary endosymbionts (as Gram-negative bacteria), often present in aphid hemolymph, are themselves perceived and controlled by the aphid immune system.
Third, aphids may invest in terminal reproduction in response to an immune challenge, rather than in a costly immune response, as Altincicek \textit{et al.}, 2008 found increased viviparous offspring production upon wounding. 
Such an increase has been found in many invertebrates including \textit{Biomphalaria} snails (Minchella \textit{et al.}, 1981; Minchella \textit{et al.}, 1985), \textit{Acheta} crickets (Adamo \textit{et al.}, 1999), \textit{Daphnia} waterfleas (Chadwick \textit{et al.}, 2005), and \textit{Drosophila} flies (Polak \textit{et al.}, 1998). 
Even without immune challenge, these insects also tends to invest most resources towards rapid, early onset reproduction (r-selection), and such organisms may specifically invest less in costly immune responses (Zuk \textit{et al.}, 2002; Miller \textit{et al.}, 2007). 
However, this may not sufficient for explaining weak immunity of pea aphids, as r-selected taxa such as \textit{Drosophila} still mount complex immune responses. 
Furthermore, aphids do not increase their reproductive effort in the face of all immune challenges: 
fungal infection reduces the number of offspring pea aphid produce within 24 hours of inoculation (Baverstock \textit{et al.}, 2006), and response to stabbing with bacteria seems to be specific to the aphid genotype and to the location of the stab.

\end{document}