\documentclass[11pt]{article}

\usepackage[UTF8]{ctex} % for Chinese 

\usepackage{setspace}
\usepackage[colorlinks,linkcolor=blue,anchorcolor=red,citecolor=black]{hyperref}
\usepackage{lineno}
\usepackage{booktabs}
\usepackage{graphicx}
\usepackage{float}
\usepackage{floatrow}
\usepackage{subfigure}
\usepackage{caption}
\usepackage{subcaption}
\usepackage{geometry}
\usepackage{multirow}
\usepackage{longtable}
\usepackage{lscape}
\usepackage{booktabs}
\usepackage{natbibspacing}
\usepackage[toc,page]{appendix}
\usepackage{makecell}
\usepackage{amsfonts}
 \usepackage{amsmath}
\usepackage[utf8]{inputenc}
\usepackage{amssymb}
\usepackage{amsthm}
\usepackage{enumerate}
\usepackage{comment}

\usepackage[backend=bibtex,style=authoryear,sorting=nyt,maxnames=1]{biblatex}
\bibliography{} % Reference bib

\title{Phenotypic evolution}
\author{}
\date{}

\linespread{1.5}
\geometry{left=2cm,right=2cm,top=2cm,bottom=2cm}

\setlength\bibitemsep{0pt}

\begin{document}
\begin{sloppypar}
  \maketitle

  \linenumbers
Quantitative trait values of individuals in a population often follow a normal distribution. 
Its mean value evolves when allele frequencies at the loci changes. 
Phenotypic variance within a population results from both genetic and environmental causes. 

\section{Fitness functions}
Fitness functions quantify the relationship between trait value and fitness, and thus describe selection on quantitative traits. 
Directional selection favours either an increase or a decrease in a trait's mean. 
The strength of directional selection is measured by selection gradient, the derivative of fitness function. 
Stabilizing selection favours individuals whose trait values are near the population's mean, and reduces phenotypic variance. 
Disruptive selection favours individuals with trait values close to the minimium and the maximun, but discards trait close to the mean. 
The fitness function and the trait's distribution together determine whether selection is directional, stablizing or disruptive. 
Regards to multiple traits, selection can favour particular combinations of traits, which is called correlational selection. 

\section{Evolution by directional selection}

  
\end{sloppypar}
\end{document}
