\documentclass[11pt]{article}

\usepackage[UTF8]{ctex} % for Chinese 

\usepackage{setspace}
\usepackage[colorlinks,linkcolor=blue,anchorcolor=red,citecolor=black]{hyperref}
\usepackage{lineno}
\usepackage{booktabs}
\usepackage{graphicx}
\usepackage{float}
\usepackage{floatrow}
\usepackage{subfigure}
\usepackage{caption}
\usepackage{subcaption}
\usepackage{geometry}
\usepackage{multirow}
\usepackage{longtable}
\usepackage{lscape}
\usepackage{booktabs}
\usepackage{natbibspacing}
\usepackage[toc,page]{appendix}
\usepackage{makecell}
\usepackage{amsfonts}
 \usepackage{amsmath}
\usepackage[utf8]{inputenc}
\usepackage{amssymb}
\usepackage{amsthm}
\usepackage{enumerate}
\usepackage{comment}

\usepackage[backend=bibtex,style=authoryear,sorting=nyt,maxnames=1]{biblatex}
\bibliography{} % Reference bib

\title{Phenotypic evolution}
\author{}
\date{}

\linespread{1.5}
\geometry{left=2cm,right=2cm,top=2cm,bottom=2cm}

\setlength\bibitemsep{0pt}

\begin{document}
\begin{sloppypar}
  \maketitle

  \linenumbers
Quantitative trait values of individuals in a population often follow a normal distribution. 
Its mean value evolves when allele frequencies at the loci changes. 
Phenotypic variance within a population results from both genetic and environmental causes. 

\section{Fitness functions}
Fitness functions quantify the relationship between trait value and fitness, and thus describe selection on quantitative traits. 
Directional selection favours either an increase or a decrease in a trait's mean. 
The strength of directional selection is measured by selection gradient, the derivative of fitness function. 
Stabilizing selection favours individuals whose trait values are near the population's mean, and reduces phenotypic variance. 
Disruptive selection favours individuals with trait values close to the minimium and the maximun, but discards trait close to the mean. 
The fitness function and the trait's distribution together determine whether selection is directional, stablizing or disruptive. 
Regards to multiple traits, selection can favour particular combinations of traits, which is called correlational selection. 

\section{Evolution by directional selection}
The evolutionary change in the mean of a trait from a single generation equals the product of two quantities: the strength of directional selection, and the amount of genetic variation. 
Let $\bar{z}$ be the mean of a trait at the start of a generation. 
Selection acts on the trait, and the survivors breed to produce the new generation. 
Let $\bar{z}'$ be the mean of the trait at the start of the new generation. 
The change of the mean is given by breeder's equation: 
\begin{equation}
  \Delta \bar{z} = \bar{z}' - \bar{z} = h^2 S
\end{equation}
$h^2$ is the trait's heritability, ranging from 0 to 1. 
When $h^2=0$, the trait is not heritable. 
When $h^2=1$, the offspring is exactly identical with the parents. 
$S$ is the amount of change in the mean of the trait caused by selection within a generation. 
That is, $S$ equals the difference between the mean of the population after selection. 
The strength of directional selection is given by selection gradient 
\begin{equation}
  \beta = \frac{S}{P}
\end{equation}
where $P$ is the phenotypic variance. 

\par

Another version of breeder's equation is 
\begin{equation}
  \Delta \bar{z} = G \beta
\end{equation}
$G$ is additive genetic variance, the part of the phenotypic variation that is caused by genetic variation, defined as 
\begin{equation}
  G = h^2 P
\end{equation}
where $P$ is the phenotypic variance of the trait. 
$G$ is often smaller than $P$ ($h^2 <1$). 
The rest of the phenotypic variance is contributed by 
(1) nongenetic factors (environmental variance); and 
(2) genetic variation that is not additive, caused by dominance and epistasis. 

\par

To predict the direction and strength of directional selection, we can measure the trait of parents and offsprings. 
The trait's heritability $h^2$ is estimated by the slope of regression line between trait of offsprings (y) and trait of parents (x). 
These trait values also give estimation on the phenotypic variance $P$. 
Then we have additive genetic variance $G=h^2P$. 
Selection gradient $\beta$ is estimated by the slope of regression line between relative fitness ($y$) and trait ($x$). 
Finally, the change of mean trait $\Delta \bar{z} = G \beta$. 
The implication here is that we can predict the outcome of genetic evolution without knowing anything about the genes that affect the trait. 
Neither number of genes that affect the trait, nor the population size, affects the evolution of a quantitative trait, at least in the short term. 

\section{Dominance and epistasis}
Dominance variance results when the phenotype of heterozygotes is not intermediate between the phenotypes of the homozygotes. 
Epistatic variance results from the interaction of alleles at different loci. 
These variances do not contribute to evolutionary changes. 
For the great majority of traits, the additive genetic variance is much larger than the dominance variance and the epistatic variance. 

\section{Adaptation from standing genetic variation versus new mutations}
In the change of environmental conditions, many traits will evolve rapidly using genetic variation that already exists, which is called standing genetic variation. 
Other traits do not have genetic variation now and cannot evolve until new mutations favoured by the new environment appear. 
In some cases the new mutations take place quickly, but in other cases the critical mutations may not appear for long periods of time. 

\section{Adaptation-extinction race}
Imagine a species intially at a fitness peak of a quantitative trait. 
The environment changes, favouring a new value for the trait and causing the mortality rate to exceed the birth rate. 
Then the population will decline to extinction, unless the trait is heritable, so the population can evolve towards the fitness peak of new environment. 
Thus, there is a race between adaptation and extinction. 

\par

Several factors influence the outcome of adpatation-extinction race. 
High standing genetic variation allows the population to adapt quickly. 
Large initial population size helps survival since the population size must decline a long way before extinction, and many mutations enter the population in each generation. 
Some species can buffer themselves from the environmental change by adjusting to new conditions physiologically. 

\section{Artificial selection}
Artificial selection on model species has revealed several key conclusions that are likely to be general for all organisms. 
Almost all traits evolve when selected. 
Selection can cause a trait to evolve far beyond its original range of variation. 
Large populations evolve faster and farther than small populations. 
Strong selection on one trait often have negative side effects on other traits. 

\section{Correlated traits}
Traits are correlated. 
Consider two traits, the evolutionary change in trait 1 caused by one generation of selection is 
\begin{equation}
  \Delta \bar{z}_1 = G_1 \beta_1 + G_{1,2} \beta_2
\end{equation}
$G_1$ is the additive genetic variance of trait 1, and $\beta_1$ is the selection gradient of trait 1. 
$G_{1,2}$ is the genetic covariance between trait 1 and trait 2, and $\beta_2$ is the selection gradient of trait 2. 
A genetic covariance of 0 means the two traits are inherited independently. 
A positive genetic covariance means that individuals that are large in one trait will tend to have offsprings that are large in both traits. 
A negative genetic covariance means the opposite: individuals that are large in one trait will tend to have offsprings that are large in that trait, but small in the second trait. 

\par

Therefore, one trait can evolve in two ways. 
One is direct response to selection, meaning that the trait is evolving because of selection acting on it. 
Another one is indirect response, meaning that the trait is evolving because of its correlation with another trait on which selection acts. 

\section{Evolutionary constraints and trade-offs}
Traits that lack standing genetic variation cannot respond to directional selection, and so they have an evolutionary constraint that can prevent them from changing. 
Evolutionary trade-off occurs when increasing fitness in one way decreases it in another. 
Genetic correlation can cause evolutionary constraints. 
When two traits are closely correlated, only specific combinations of trait values exist in the population, and there is little variation on the combinations of trait values. 

\section{Causes of genetic correlations}
Genetic correlations have two sources. 
The first is pleiotropy, which occurs when a single locus affects more than one trait. 
Another source is linkage disequilibrium, the nonrandom association between alleles at different loci. 

\section{Phenotypic plasticity}
Phenotypic plasticity occurs when an individual's phenotype changes in response to the environment it experiences. 
It is seen in a wide range of traits including gene expression, development, morphology, physiology and behaviour. 
Phenotypic plasticity can be adaptive or non-adaptive. 

\par

Phenotypic plasticity can be visualized with reaction norm plotting phenotypes against environmental conditions. 
Reaction norm can differ among genotypes. 
Genetic variation in a reaction norm is referred to as genotype-environment interaction. 

\section{Quantitative trait loci (QTL)}
Quantitative trait loci are genomic regions that affect a quantitative trait. 
They range in size from a single nucleotide to a chromosomal segment. 
Identification of QTL is based on the correlation between genotypes and traits: scanning the genome for loci that have large allele frequency differences among individuals with divergent phenotypes. 


\end{sloppypar}
\end{document}
