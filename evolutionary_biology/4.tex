\documentclass[11pt]{article}

%\usepackage[UTF8]{ctex} % for Chinese 

\usepackage{setspace}
\usepackage[colorlinks,linkcolor=blue,anchorcolor=red,citecolor=black]{hyperref}
\usepackage{lineno}
\usepackage{booktabs}
\usepackage{graphicx}
\usepackage{float}
\usepackage{floatrow}
\usepackage{subfigure}
\usepackage{caption}
\usepackage{subcaption}
\usepackage{geometry}
\usepackage{multirow}
\usepackage{longtable}
\usepackage{lscape}
\usepackage{booktabs}
\usepackage{natbibspacing}
\usepackage[toc,page]{appendix}
\usepackage{makecell}
\usepackage{amsfonts}
 \usepackage{amsmath}
\usepackage[utf8]{inputenc}
\usepackage{amssymb}
\usepackage{amsthm}
\usepackage{enumerate}
\usepackage{comment}

\usepackage[backend=bibtex,style=authoryear,sorting=nyt,maxnames=1]{biblatex}
%\bibliography{} % Reference bib

\title{Mutation and variation}
\author{}
\date{}

\linespread{1.5}
\geometry{left=2cm,right=2cm,top=2cm,bottom=2cm}

\setlength\bibitemsep{0pt}

\begin{document}
\begin{sloppypar}
  \maketitle

  \linenumbers
The variation among individuals of a species are different phenotypes. 
Natural selection acts on phenotypes, but that process only results in evolution if at least some of the variation in phenotypes is transmitted between generations. 
Inheritance is the result of genotypes encoded by DNA. 
The basic unit of genetic inheritance is a locus, a section of chromosome. 
The DNA sequence at a given locus often varies among different individuals, which is called polymorphism. 
The different variants at a locus are alleles. 

\section{Segregation and Hardy-Weinberg equilibrium}
Segregation is the selection of one of the two copies of a locus when a gamete is made during meiosis. 
Consider a population that is of 
(1) infinitely large population size; 
(2) no natural selection; 
(3) no mutation; 
(4) no migration; 
(5) random mating; 
and consider a locus with two alleles $A$ and $a$ with frequencies $p$ and $1-p$ respectively. 
It is easy to show that: 
(1) the frequency of $A$ is always $p$ over generations; and 
(2) the distribution of genotype frequency reaches Hardy-Weinberg proportions (the frequencies of $AA$, $Aa$ and $aa$ are $p^2$, $2p(1-p)$ and $(1-p)^2$, respectively) after one generation, and no longer changes over following generations. 

\par

If a population is not in Hardy-Weinberg equilibrium, something is happening. 
For example, the beta-hemoglobin locus of human has two alleles $A$ and $S$, and its genotype frequency distribution observed in Africa populations are significantly different from Hardy-Weinberg equilibrium: the observed heterozygote frequency is higher. 
This difference results from survival: $AA$ and $SS$ do not survive as well as $AS$. 

\section{Recombination}
Recombination is the process that combines in a gamete a gene copy at one locus that was inherited from the mother with a gene copy at a second locus that was inherited from the father. 
It happens between loci on the same chromosome by crossing over, which joins together a piece of a chromosome inherited from the mother with a piece inherited from the father. 

\par

The recombination rate $r$ is the probability that recombination occurs between a given pair of loci. 
If the two loci are on different chromosome, there is $r=0.5$, which is the maximum possible value for recombination rate. 
At the other extreme, the recombination rate for loci on the same chromosome is extremely low. 
When an allele at one locus is found together with an allele at a scond locus in a population more often than expected by chance, the loci are in linkage equilibrium. 
Consider one locus with alleles $A/a$ and a second locus with alleles $B/b$, and $p_A$, $p_B$ are the frequencies of gametes with $A$ and $B$, respectively. 
The frequency of gamete $AB$, by chance, is $p_Ap_B$, and let the observed frequency of gamete $AB$ be $p_AB$. 
Linkage equilibrium can be measured by $D=p_{AB}-p_Ap_B$. 
Given recombination $r$, generation $t$, and assume both loci are in Hardy-Weinberg equilibrium, $p_{AB}(t+1)=(1-r)p_{AB}(t)+rp_Ap_B$ and $D(t+1)=(1-r)D(t)$, and linkage disequilibrium approaches to linkage equilibrium ($D=0$) over generations.
Natural selection that prefers some combinations of alleles or mixing of populations with different allele frequencies can maintain linkage disequilibrium. 

\section{Mutation}
Mutation, the imperfections in DNA replication, is an inevitable consequence of the Second Law of Thermodynamics, and is the ultimate source of variation. 
Mutations come in a variety of forms. 

\par

Point mutations occur when a single DNA base is changed from one to another. 
Point mutations at genomic regions that coding proteins are synonymous if ther protein sequence is not changed, or nonsynonymous if the protein sequence is changed. 
For example, a nonsynonymous point mutation in human beta-hemoglobin results the $S$ allele. 
Point mutations in non-coding regions might affect organisms by altering gene expression. 

\par

Structural mutations occur on more than one DNA bases and have several classes. 
Deletions occur when a segment of a chromosome is dropped. 
For example, cystic fibrosis of human is caused by a deletion of three bases in a sodium channel genes. 
Insertions occur when a segment is added to a chromosome, either from elsewhere on the genome. 
For example, Huntington's disease is caused by multiple insertions of three bases (CAG) in \textit{huntingtin} gene. 
Duplications occur when extra copies of a segment are inserted into the genome. 
Duplication events of one gene give rise to gene family, whose members might evolve new functions. 
Invertions occur when a chromosome breaks at two places and the middle segment is reinserted in the reverse orientation. 
Reciprocal translocations occur when two nonhomologous chromosomes exchange segments with each other. 
Fusions occur when two nonhomologous chromosomes are joined. 
Fissions occur when one chromosome breaks into two. 
Fussiona and fissions are responsible for dynamics of haploid chromosome number, which can be as large as 630 in fern \textit{Ophioglossum reticulatum} and 16,000 in ciliate (\textit{Oxytricha trifallax}). 
Finally, there are whole genome duplications. 

\section{Mutation rate}
Mutation rate is the probability that an offspring carries a new mutation. 
For human, roughly 1 out of 1e+8 DNA bases carries a new mutation, namely mutation rate is approximately 1e-8 per base per generation. 
RNA viruses have mutation rates as high as 1e-5 to 1e-3 per base per generation.  
For cellular organisms, mutation rate is related with genome size: organisms with large genome tend to have high mutation rate. 

\par

The concept of mutation rate also applies to gene or genome. 
A gene/genome carries a mutation if there is one base mutates, so the mutation rate per gene/genome is given by mutation rate per base times gene/genome length. 
Mutation rates of protein-coding gene in eukaryotes are generally 1e-5 to 1e-7. 

\section{Effects of mutations}
Mutations virtually affect all aspects of organisms and they show two general features. 
The first one is pleiotropy: a single mutation affects multiple traits. 
Virtually all mutations with phenotypic effects show pleiotropy: genetic changes that alter one aspect of an organism invariably have side effects on other aspects. 
Second, there are much more deleterious mutations (harmful to survival or reproduction) than beneficial mutations. 

\section{Is mutation random?}
Whether mutations are random depends on the meaning of the world "random". 

\par

Mutations are not random since not all mutationa are equally likely. 
The mutation rate differs substantially among different regions of the genome. 
For a single DNA base, transition mutations (between A and G, and between C and T) are twice as many possible transversion mutations (between A and C, between G and T, between A and T, and between G and C). 

\par

On the other senes, beneficial mutations take place randomly, indenpendent from environmental conditions. 
Environmental conditions do not increase the frequency of mutations that are beneficial under these conditions. 

\section{Nongenetic inheritance}
The majority of inherited changes involve alternations in DNA/RNA sequences. 
However, other mechanisms can also contribute to inheritance. 

\par

Epigenetic inheritance is caused by inherited changes to chromosomes that do not alter DNA sequence. 
Such changes affect phenotype by altering gene expression. 
Epigenetic changes occur in several mechanisms, including methylation of DNA bases (A and C) and modifications on histones that bind to DNA. 
Most epgenetic changes are not stable and dissipate after a few generations. 
Therefore, they can be important in short terms, but do not make major contributions to long-term evolutionary changes. 

\par

Maternal effects occur when the genotype/phenotype of the mother directly influences the phenotype of her offspring. 
For example, the direction of coiling in the snail \textit{Lymnaea peregra} is determined by the genotype of an individual's mother instead of its own genotype. 
Materal effects contribute to the resemblance between mothers and their offsprings, but are only transimtted across a few generations and therefore, do not contribute much to long-term evolution. 

\par

Cultural inheritance occur when traits are transmitted by behaviour and learning. 
It apparently plays a key role in human society: language, religion, and dietary preference, are strongly influenced by cultural inheritance. 
An important difference between cultural inheritance and other forms of inheritance is that traits can be transmitted between unrelated individuals. 

\end{sloppypar}
\end{document}
