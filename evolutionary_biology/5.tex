\documentclass[11pt]{article}

%\usepackage[UTF8]{ctex} % for Chinese 

\usepackage{setspace}
\usepackage[colorlinks,linkcolor=blue,anchorcolor=red,citecolor=black]{hyperref}
\usepackage{lineno}
\usepackage{booktabs}
\usepackage{graphicx}
\usepackage{float}
\usepackage{floatrow}
\usepackage{subfigure}
\usepackage{caption}
\usepackage{subcaption}
\usepackage{geometry}
\usepackage{multirow}
\usepackage{longtable}
\usepackage{lscape}
\usepackage{booktabs}
\usepackage{natbibspacing}
\usepackage[toc,page]{appendix}
\usepackage{makecell}
\usepackage{amsfonts}
 \usepackage{amsmath}
\usepackage[utf8]{inputenc}
\usepackage{amssymb}
\usepackage{amsthm}
\usepackage{enumerate}
\usepackage{comment}

\usepackage[backend=bibtex,style=authoryear,sorting=nyt,maxnames=1]{biblatex}
%\bibliography{} % Reference bib

\title{The genetic theory of natural selection}
\author{}
\date{}

\linespread{1.5}
\geometry{left=2cm,right=2cm,top=2cm,bottom=2cm}

\setlength\bibitemsep{0pt}

\begin{document}
\begin{sloppypar}
  \maketitle

  \linenumbers

\section{Fitness}
An individual's absolute fitness is the number of offspring produced over its lifetime. 
It is often useful to consider the average fitness of individuals of specific genotype/phenotype, and it is useful to simplify  absolute fitness (denoted by $W$) as the product of two components: the probability that the individual survives to maturity, and the expected number of offspring if the individual does survive. 

\par

The strength of selection is determined by fitness differences. 
Therefore, it is convienient to work with relative fitness, which is the absolute fitness divided by a fitness of reference. 

\section{Positive selection: the spread of beneficial mutations and the elimination of variation}
Whenever one allele has higher fitness than others, natural selection will favour its spread through the population, which is called positive selection. 
Consider a single locus with two alleles $A$ and $a$ under Hardy-Weiberg equilibrium and the allele frequency of $a$ is $p_0$. 
Assume $a$ increases fitness: for each $a$ copy that an individual carries, its fitness increases by a proportion of $s$. 
The relative fitness for $AA$, $Aa$ and $aa$, are $1$, $1+s$ and $1+2s$, respectively. 
The genotype of $AA$, $Aa$ and $aa$ in the first generation is $\frac{(1-p_0)^2}{\bar{w}}$, $\frac{2p_0(1-p_0)(1+s)}{\bar{w}}$ and $\frac{p_0^2(1+2s)}{\bar{w}}$, respectively, where $\bar{w}=(1-p_0)^2+2p_0(1-p_0)(1+s)+p_0^2(1+2s)$ is the mean fitness of the 0th population. 
The allele frequency of $a$ in the first generation is $p_1=\frac{p_0[1+s(1+p_0)]}{\bar{w}}$, and $\Delta p=p_1-p_0=\frac{sp_0(1-p_0)}{\bar{w}} \approx sp_0(1-p_0)$ since $s$ is often very small (say, below 0.1). 
The key implications here are: 
(1) the rate of evolution caused by selection is proportional to the strength of selection measured by $s$, which is called selection coefficient; 
(2) the rate of evolution caused by selection is proportional to $p(1-p)$, which represents the genetic variation at the locus. 
(3) the beneficial allele $a$ will spread through the population and fixed, which is called selective sweep. 
As a rough estimation, a beneficial allele will increase frequency from 0.1 to 0.9 in about $\frac{4}{s}$ generations. 

\par

Dominance of beneficial allele speeds its spread driven by natural selection, as heterozygotes share the full fitness benefit that the homozygotes have, especially when it is rare. 
However, when the beneficial allele is common, almost all individuals have high fitness and there is very little variation in fitness, and selection has little power to increase the frequency of beneficial allele further. 

\section{Evolutionary side effect}
Genetic correlations occur when two traits tend to be inherited together and cause evolutionary side effect: one phenotype can be preferred by natural selection because of its correlation with another beneficial phenotype, while itself can be neutral or even deleterious. 
Therefore, an allele that increase fitness through its effect on one trait can decrease fitness due to its effect on another trait, a phenomenon called evolutionary trade-off. 
Also, an allele that has no effect on fitness can also spread by natural selection by association with a beneficial allele of a second locus, which is called hitchhiking. 
In hitchhiking, a beneficial mutation first appear, and it is in perfect linkage equilibrium with all other alleles on its chromosome. 
As the mutation spreads, recombination breaks down the linkage disequilibrium. 
However, on the chromosome, there are more recombinations between distant sites than neighboring sites. 
So sites close to the selected locus do not have a chance to recombine before the mutation is fixed. 
As a result, these sites carry the same alleles that were on the original chromosome where the mutation appeared. 
Variation at these sites is eliminated. 

\section{Balancing selection: the maintainance of variation}
Balancing selection maintains genetic variation within a population. 
One form of balancing selection is overdominance, which occur when the heterozygote have higher fitness than both homozygotes. 
Consider a locus with two alleles $A$ and $S$, and the relative fitness of $AA$, $AS$ and $SS$ are $w_{AA}<1$, $1$ and $w_{aa}<1$, respectively, and assume that natural selection is the only cause of evolution. 
The population approaches to polymorphic equilibrium, with the frequency of $S$ given by $p=\frac{1-w_{AA}}{2-w_{AA}-w_{SS}}$. 

\par

Another form of balancing selection is frequency-dependent selection, which occur when the fitnesses of alleles change depending on their own frequencies. 
In some cases, an allele gets a fitness advantage when it is rare, which is called negative frequency dependence. 

\par

A third form of balancing selection occur when different genotypes specialize on different ecological niches. 
In such cases, each genotype is partly shielded from competition with other genotypes and has its own ecological capacity. 

\section{Selection with historical contigency}
Positive selection leads to the fixation of beneficial allele, while balancing selection drives the population evolve towards a constant equilibrium allele frequency no matter where it begins. 
In selection with historical contigency, variation is eliminated, but which allele spreads to fixation depends on the initial allele frequency. 

\par

One form of historical contigency is underdominance, which occur when the heterozygote has lower fitness than homozygotes. 
The scenario is: if the initial frequency of an allele is below a threshold, almost all of its copies are in low fitness heterozygotes and selection drives it out of the population. 
The threshold is determined by the relatibe fitness of the two homozygotes. 

\par

Another form is positive frequency-dependent selection, which favours alleles that have high frequency. 
As a result, it eliminates genetic variation within population. 

\section{Mean fitness of population ($\bar{w}$)}
The mean fitness of a population evolves as the allele frequencies changes. 
By natural selection, mean fitness of a population increases, and the increase per generation depends on genetic variation of the population (fundamental theorem of natural selection). 

\par

A complementary perspective on the evolution of fitness is adaptive landscape, in which mean fitness $\bar{w}$ is plotted against allele frequency $p$. 
By selection, the population evolve uphill on the landscape, and the allele frequency changes at rate $\Delta p=\frac{1}{2}p(1-p) \frac{d \ln \bar{w} }{d p}$. 

\section{Purifying selection: removal of deleterious mutations}
Deleterious mutations are much more common than beneficial ones. 
Selection that acts to remove deleterious mutations from a population is called purifying selection. 

\par

Purifying selection often fails to eliminate deleterious mutations, as they are being continually introduced. 
This flowof new mutations into the population is offset by natural selection that acts to eliminate them. 
This situation is called mutation-selection balance. 
Let the mutated-allele free homozygotes have relative fitness 1, and the heterozygotes have relative fitness $1-s$, and the homozygotes with two mutated alleles have relative fitness $1-2s$. 
The probability that a copy of normal allele mutates to a deleterious allele in a given generation is $\mu$. 
When the input of the deleterious allele by mutation balances its removal by selection, the deleterious mutation reaches an equilibrium frequency of $\hat{p} \approx \frac{\mu}{s} (\mu << s)$. 

\par

Mutation load is the proportion by which the mean fitness of the population is reduced by deleterious mutations compared with a hypothetical population without mutations. 
For a single locus, mutation load is given by $L=2\frac{\mu}{s}(1-\frac{\mu}{s})s+(\frac{\mu}{s})^2 2s = 2\mu$. 

\par

Mutation rates at individual loci are often very small, but there are many loci in a genome. 
Let $U$ be the average number of new deleterious mutations that are added to the genome each generation. 
Assume that mutations have independent effects on fitness, the mutation load is given by $L=1-e^{-U}$.


  
\end{sloppypar}
\end{document}
