\documentclass[11pt]{article}

\usepackage[UTF8]{ctex} % for Chinese 

\usepackage{setspace}
\usepackage[colorlinks,linkcolor=blue,anchorcolor=red,citecolor=black]{hyperref}
\usepackage{lineno}
\usepackage{booktabs}
\usepackage{graphicx}
\usepackage{float}
\usepackage{floatrow}
\usepackage{subfigure}
\usepackage{caption}
\usepackage{subcaption}
\usepackage{geometry}
\usepackage{multirow}
\usepackage{longtable}
\usepackage{lscape}
\usepackage{booktabs}
\usepackage{natbibspacing}
\usepackage[toc,page]{appendix}
\usepackage{makecell}
\usepackage{amsfonts}
 \usepackage{amsmath}
\usepackage[utf8]{inputenc}
\usepackage{amssymb}
\usepackage{amsthm}
\usepackage{enumerate}
\usepackage{comment}

\usepackage[backend=bibtex,style=authoryear,sorting=nyt,maxnames=1]{biblatex}
% \bibliography{} % Reference bib

\title{Natural selection and adaption}
\author{}
\date{}

\linespread{1.5}
\geometry{left=2cm,right=2cm,top=2cm,bottom=2cm}

\setlength\bibitemsep{0pt}

\begin{document}
\begin{sloppypar}
  \maketitle

  \linenumbers
Adaptation is a characteristic that enhances the survival or reproduction of organisms that bear it. 
It can be very rapid. 
For example, soapberry bugs (\textit{Jadera haematoloma}) in North America have adapted to new host plants introduced in the last 50 years. 
Bacteria have evolved resistance to antibiotics rapidly. 
Similarly, resistance to pesticides has evolved in hundreds of insect species, and many weed species have evolved resistance to herbicides within 10-20 years. 
In some plants, metal-tolerant populations have evolved in area where soils have been contaminated by mine works less than 100 years ago. 
Commercially overexploitated fish species, such as Atlantic cod (\textit{Gadus morhua}), often evolve towards earlier sexual maturation at smaller size. 
These evolutionary changes can be so rapid because populations in altered environments can experience stringent natural selection, and because they contain genetic variation in many characteristics. 

\section{Natural selection}
Natural selection is any consistent difference in fitness among different classes of biological entities. 
Fitness can be viewed as the number of offsprings an individual is expected to leave in the next generation. 
If natural selection occurs, there must be a correlation between an individual's phenotype and its fitness, and variation in the phenotype is correlated between parents and their offsprings. 

\par

The environmental factors that impose natural selection on a species are greatly influenced by the characteristics of the species itself. 
Organisms “screen off” some aspects of their environment, which may then cease to exert natural selection. 
Many species of ants, rodents, and other animals have become so reliant on chemical signals that they have become blind, because natural selection for sight has become reduced or even negative: well-developed eyes may be disadvantageous if they conflict with other important functions. 
Likewise, humans have lost functions of many olfactory receptors, having become so much more reliant on vision than smell. 

\section{Levels of selection}
Natural selection occurs in any biological entities, such as genes, cell types, individuals, populations and species. 
It is nothing more than reproductive success, which does not necessarily result in adaptation or any improvement, and natural selection at different levels can be opposite. 

\par

Natural selection in the level of gene and individual do not necessarily consistent. 
One example is the success of selfish genetic elements, which have high rate of reproduction but provide no advantange to the individual, and can even be harmful. 
They are preferred by natural selection in the level of gene, but not individuals. 
For example, transposons transmitted at a higher rate than the rest of the genome and may be detrimental, or at least no advantageous. 
Another example of selfish genetic elements is the \textit{t} locus of mouse (\textit{Mus musculus}). 
In male heterzygous \textit{Tt}, \textit{t} allele tends to kill gametes carrying \textit{T} and as a result, over 90\% of sperms carry \textit{t}. 
However, homozygous \textit{tt} embryos either die or are sterile. 
Despite this disadvantage, \textit{t} allele reaches a high frequency in many populations. 
Beyond selfish genetic elements, another example illustrate the difference between natural selection in the level of gene and individual is the altruistic traits evolving under kin selection, a form of selection in which alleles differ in fitness by influencing the effect of their bearers on the reproductive success of individuals (kin) who carry the same allele by common descent. 
For example, in eusocial insects, workers and queens are highly similar in genotype, so any behaviour of the former that is of advantage to the latter promote the reproductive success of their genotypes, though such behaviour can be suicidal for the workers. 
Another example is the parental care in mammals. 

\par

The relationship between natural selection in the level of individual and population remains controversial. 
Specifically, is it possible that a trait evolve that benefits the population at a cost to the individual? 
An altruistic trait can not evolve if it reduces the fitness of an individual than bears it, as such a genotype would decline in frequency simply because it generates fewer offsprings than others. 
Conversely, if a population were to consist of altruistic genotypes, a selfish mutant would have an advantage over others and increase to fix. 
So it seems that altruistic trait that benefit the whole population at a cost of the individuals bearing it seems impossible. 
However, group selection provides a conceivable scinerio for the evolution of altruistic traits by letting populations made up of selfish genotypes under high rate of extinction. 
Whether such selection at population level is effective enough to overcome selection at individual level remains controversial, since the number of populations are much lower than the number of individuals, and the turnover rate of population is lower than the turnover rate of individual. 

\par

Selection among groups of organisms is called species selection when the groups involved are species and there is a correlation between some characteristic and the rate of speciation or extinction. 
It does not affect adaptations, but the disparity, the diversity of biological characteristics. 
It is correlated with some characteristic and the rate of speciation/extinction. 
The consequence of species selection is that the proportion of species that have one character rather than another change over time. 
One example is the prevalence of sexual species. 
Many groups of animals and plants give rise to asexual lineages, which are often young with close genetic similarity to sexual relatives. 
It implies that asexual forms have a higher rate of extinction than sexual forms. 

\section{Nature of adaptations}

\end{sloppypar}
\end{document}
