\documentclass[11pt]{article}

% \usepackage[UTF8]{ctex} % for Chinese 

\usepackage{setspace}
\usepackage[colorlinks,linkcolor=blue,anchorcolor=red,citecolor=black]{hyperref}
\usepackage{lineno}
\usepackage{booktabs}
\usepackage{graphicx}
\usepackage{float}
\usepackage{floatrow}
\usepackage{subfigure}
\usepackage{caption}
\usepackage{subcaption}
\usepackage{geometry}
\usepackage{multirow}
\usepackage{longtable}
\usepackage{lscape}
\usepackage{booktabs}
\usepackage{natbibspacing}
\usepackage[toc,page]{appendix}
\usepackage{makecell}
\usepackage{amsfonts}
 \usepackage{amsmath}
\usepackage[utf8]{inputenc}
\usepackage{amssymb}
\usepackage{amsthm}
\usepackage{enumerate}
\usepackage{comment}

\usepackage[backend=bibtex,style=authoryear,sorting=nyt,maxnames=1]{biblatex}
% \bibliography{} % Reference bib

\title{Evolution of evolutionary biology}
\author{}
\date{}

\linespread{1.5}
\geometry{left=2cm,right=2cm,top=2cm,bottom=2cm}

\setlength\bibitemsep{0pt}

\begin{document}
\begin{sloppypar}
  \maketitle

  \linenumbers
Evolutionary biology aims at understanding the diversity of living things and their characteristics: what has been the history that produced this diversity, and what have been the causes of this history? 
It aims to develop broad principles and document common patterns of evolution, to arrive at general principles that apply to diverse organisms and diverse characteristics. 
It complements studies of the proximate causes (immediate, mechanical causes) of biological phenomena (\textit{e.g.} biochemistry, cell biology and neurobiology), and analyzes the ultimate causes of these phenomena: their historical causes. 

\section{Before Darwin}
The prevailing view on living things was that species have fixed properties. 
This notion was largely originated from Plato and Aristotle. 
Later, Christians concluded the blief of special creation: each species had been created individually by God in the same form it has today. 
Theologians argued that God is complete, His creation must be good. 
Since order is superior to disorder, living things created by God must follow a hierarchical order, a gradation from non-living things, to less animate organisms like plants, and to animate forms of life like animals. 
Human, being both physical and spiritual in nature, forms the link between animals and angles, and is superior to all other things. 
This "Great Chain of Being" is permanent and unchanging, since it is a perfect creation of the God. 
Any modification would imply its imperfection. 
Under the notion of special creation, scientific researches were justified as a way to reveal the chain of being so that people can appreciate the triumph of God. 

\par

The notion of "Great Chain of Being" was largely challenged since Enlightment. 
Newton's explanations of physical phenomena adopts reason instead of God's intelligence as the major basis of authority. 
Theories of the origin of stars and planets were deveopled. 
Evidence that Earth had undergone profound changes was accumulating, and it had been populated by many ancient creatures now extinct. 
The geologiest James Hutton and Charles Lyell expounded the principle of uniformitarianism: the same processes operated in the past as in the present, and that the data of geology should therefore be explained by causes that we can observe now. 
These achievements greatly challenged the view that the world is permanent and perfect, revealing a world of ceaseless change. 
It is natural that scholars started to suggest that living things had arisen by natural causes instead of wisdom of God. 

\par

Chevalier de Lamarck proposed the most significat pre-Darwin evolutionary hypothesis. 
Lamark hypothesed that living things originated spontaneously from non-living things independently, and progress to higher hierarchy in the chain of being, driven by a "nervous fluid". 
Species originated at different time, so the observed hierarchy of species is because they differ in age. 
Lamarck also argued that species differ from each other because that have different needs. 
To meet their needs, species use some organs more often than others, which drives the divergence of species. 
Such acquired traits are inheritable, making species differ from each other over the course of generations. 
In the most famous example of Lamarck's theory, giraffes must have stretched their necks to reach leaves above them, and so their necks were lengthed. 
The long necks were inheritable, and over the course of generations, their necks become longer and longer. 

\section{Darwin}
Charles Robert Darwin's theory of biological evolution is one of the most revolutionary ideas in the history of thought. 
It contains mainly two ideas. 
The first one is descent with modification. 
It holds that all species have descended from one or a few common ancestors instead of originating from non-living matters many times independently as Lamarck proposed. 
Species that diverge from a common ancestor are at first very similar, and differ from each other over long time span. 
The second one is natural selection. 
It holds that the frequency of a variant from increases within a population over generations, if the variant from provides certain advantage in surviving and reproducing. 
It proposes that evolution takes place at population level instead of individual level as Lamarck proposed. 

\par

Darwin's theory of evolution includes five distinct components: 
(1) "Evolution as such" is the proposition that the characteristics of organisms change over time; 
(2) "Common descent" holds that all species diverged from common ancestors and that species can be protrayed as one great family tree representing ancestry; 
(3) "Gradualism" is the proposition that differences between species have evolved by small steps through intermediate forms; 
(4) "Populational change" is the hypothesis that evolution occurs by changes in the frequencies of variant individuals within a population; 
(5) "Natural selection" accounts for adaptions: the spread of features with good functions fitting to the environment. 

\par

A huge gap in Darwin's theory is the emergence of inheritable variants within populations. 
According to prevailing belief of blending inheritance that traits of offsprings intermediate between that of parents, populations should be largely homogeneity and there should be no variation. 
Darwin never filled this gap. 

\section{Evolutionary biology after Darwin}
The idea of descent with modification and natural selection provided a new framework for speculating the history of life. 
A great deal of information on evolution was revealed from advances in paleonotology, comparative morphology and comparative embryology. 
But the consensus did not extend to the idea that natural selection is the cause of evolution, and several alternative theories were proposed. 

\par

Neo-Lamarckism is based on the idea that modifications acquired during the lifetime of an individual are inheritable. 
However, extensive experiments provided no evidence that specific mutations can be induced by environmental conditions under which they would be advantageous. 
In a famous experiment, August Weismann cut off the tails of mice for many generations and showed that this modification has no effect on the tail length of their offsprings. 

\par

Theories of orthogenesis hold that variation that arises is directed towards fixed goals, and a species evolves towards a predetermined direction by some internal driving forces. 
This direction is not necessarily adaptive and can lead to extinction. 
One example is the extinct Irish elk (\textit{Megaloceros giganteus}) with huge antlers. 

\par

Mutationist theories were inspired by advances in genetics: discretely different new phenotypes can arise from mutations. 
They proposed that such mutations lead to arise of new species, and natural selection is not necessary. 

\section{The evolutionary synthesis}
These anti-Darwin theories were refuted in 1930s and 1940s. 
The consensus forged on evolutionary biology is known as neo-Darwinism, or the evolutionary synthesis, with the chief principle that adaptive evolution is caused by natural selection acting on particulate genetic variation. 
Mutation, gene flow, natural selection and genetic drift are the major causes of evolution within species (microevolution) and between species (macroevolution). 
Main principles of evolutionary synthesis are: 
(1) an individual's phenotype is dependent on its genotype and environmental effects; 
(2) acquired characteristics are not inherited; 
(3) hereditary variations are based on particlesof genetic materials, and it holds for both discrete and continuous variations; 
(4) genetic variations arises by random mutations, which do not arise in response to need; 
(5) evolution is the changes of a population instead of an individual; 
(6) changes in allele frequencies can be random or nonrandom; 
(7) natural selection can account for both slight and great differences among species; 
(8) natural selection can alter populations beyond the original range of variation; 
(9) populations usually have considerable genetic variations; 
(10) the differences between species evolve by accumulation of small genetic differences over many generations; 
(11) species are groups of inbreeding or potentially inbreeding individuals that do not exchange genes with other such groups; 
(12) speciation usually occurs by the genetic differentiation of geographically isolated populations; 
(13) higher taxa arise by the sequential accumulation of small differences; 
(14) all organisms form a great Tree of Life. 

\section{After the synthesis}
Since the evolutionary synthesis, numerous research has tested and extended its principles. 
Investigations in DNA sequences gave rise to the neutral theory of molecular evolution, which argues that most of the evolution of DNA sequences occurs by genetic drift instead of natural selection. 
Evolutionary developmental biology is devoted to understanding how the evolution of developmental processes underlies the evolution of morphology. 
Sequencing technology boosted evolutionary genomics concerning with variation and evolution of multiple genes or entire genomes. 
Evolutionary theory has expanded into areas such as ecology, animal behavior, and reproductive biology.


  
\end{sloppypar}
\end{document}
