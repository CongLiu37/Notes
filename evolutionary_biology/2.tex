\documentclass[11pt]{article}

%\usepackage[UTF8]{ctex} % for Chinese 

\usepackage{setspace}
\usepackage[colorlinks,linkcolor=blue,anchorcolor=red,citecolor=black]{hyperref}
\usepackage{lineno}
\usepackage{booktabs}
\usepackage{graphicx}
\usepackage{float}
\usepackage{floatrow}
\usepackage{subfigure}
\usepackage{caption}
\usepackage{subcaption}
\usepackage{geometry}
\usepackage{multirow}
\usepackage{longtable}
\usepackage{lscape}
\usepackage{booktabs}
\usepackage{natbibspacing}
\usepackage[toc,page]{appendix}
\usepackage{makecell}
\usepackage{amsfonts}
 \usepackage{amsmath}
\usepackage[utf8]{inputenc}
\usepackage{amssymb}
\usepackage{amsthm}
\usepackage{enumerate}
\usepackage{comment}

\usepackage[backend=bibtex,style=authoryear,sorting=nyt,maxnames=1]{biblatex}
%\bibliography{} % Reference bib

\title{Tree of life}
\author{}
\date{}

\linespread{1.5}
\geometry{left=2cm,right=2cm,top=2cm,bottom=2cm}

\setlength\bibitemsep{0pt}

\begin{document}
\begin{sloppypar}
  \maketitle

  \linenumbers
In a tree of life, or phylogenetic tree, closely adjacent twigs represent living species derived only recently from their common ancestors. 
Twigs on more distant branches represent species derived from more ancient common ancestors.

\section{The great tree of life}
All orginams have descended from a single ancestor that lived 4-3.7 billion years gao. 
The first cellular organisms were prokaryotes that evolved into two groups: Bacteria and Archaea. 
One lineage of Archaea associated with a bacterium which evolved into mitochondrion, giving rise to Eukaryotes. 
One early clade of Eukaryotes became green algae, which acquired symbiotic photosynthetic cyanobacteria that evolved into chloroplasts. 
Green algae gave rise to the true plants. 
Another clade of Eukaryotes gave rise to fungi and animals. 
The third clade gave rise to brown algae and other protists. 
Complex multicullar organisms evolved several times. 
These groups include true plants, brown algae, some fungi, and animals. 

\par

The diversity of species within different groups is uneven. 
Bony fishes, with about 33,000 species, are the most diverse group of vertebrates. 
All the other vertebrates sum to about 30,000 species. 
Beetles are the most diverse group of animals, with about 350,000 species. 
Among flowering plants, the sunflowers (about 23,000 species) and orchids (about 19,500) species are diverse groups. 

\section{Phylogenetic trees}
A phylogeny is the history of the events by which taxa have successively arisen from common ancestors. 
It contains four processes: 
anagenesis or the evolutionary change of features within a single lineage; 
cladogenesis or branching of a lineage into two or more descendant lineages; 
extinction; and 
reticulation, or two lineages merge or form a hybrid descendant. 

\par

Different regions of genomes often have evolutionary history different from the whole genomes. 
The topology of a gene tree is often incongruent from the species tree for several reasons. 
Species can arise from hybridization between two different ancestors. 
Genes can horizontally transfer between different taxa. 
Genes can also be duplicated or lost. 
Incomplete lineage sorting also contributes to the incongruence between gene trees and the species tree, as two descendant populations might only inherit different part of genetic variations that appeared in its ancestral population. 

\section{Phylogenetic insights into evolutionary history}
One important use of phylogeny is to reconstruct evolutionary changes of characteristics by mapping characteristic states/values on the phylogenetic tree and inferring the state/values of each ancestral nodes. 
This approach allows speculations on which branch of the species tree changes in characteristics occured, and the direction of such changes. 

\par

Another use of phylogeny is estimating time of divergence, with molecular clock and fossil evidence. 
Theory of molecular clock proposes that the proportion of base pairs that differ between homologous DNA sequences is related with time of divergence. 
The fossils have been used to calibrate the rate of evolution, which differs among the positions in codons, among different genes, and among different taxa.  

\section{Patterns of evolution}
Most features of organisms have been modified from pre-existing features, instead of arise \textit{de novo} from nothing. 
During the process of evolution, characteristics originated from a common ancestor (homologous) can divergence from each other in terms of morphology or function, and features that share no homology can also converge to similar morphology or function. 
The most common criteria for homology of anatomical charaters are correspondence of position relative to other parts of the body and the correspondence of structures. 
For example, the wings of birds, bats, and pterosaur are highly modified forelimbs, and share homology to forelimbs of human, seal and horse, even though they are distinct in morphology and functions. 

\par

Rates of character evolution differ. 
Some characteristics are retained with little change over long periods among the many descendants of an ancestor, while others can be very different even between closely related taxa. 
For example, five-toed limb is conserved in all amphibians, while body size of mamals vary at many folds. 
Evolution of different characters at different rates within a lineage is called mosaic evolution. 

\par

Evolution is often gradual. 
Although many higher taxa that diverged in the distant past are not bridged by intermediate forms, either among living species or in the fossil records, graduations among closely related living species are common. 
For example, the length and shape of the bill differ greatly among sandpiper species, but the most extreme forms are bridged by intermediate forms. 

\par

Homoplasy, the independent evolution of a character or character state in different taxa including convergent evolution, parallel evolution and evolutionary reversal, is common. 
Convergent evolution is that similar characteristics independently evolved in distantly related taxa. 
For example, the eyes of vertebrates and squids were evolved independently. 
Such convergent characteristics are often adaptions by different lineages to similar environment conditions. 
For example, a long, thin beak has evolved independently in at least six lineages of nectar-feeding birds. 
Such a beak enables these birds to reach nectar in the bottom of the long tubular flowers in which they often feed. 
Likewise, long tubular flowers have evolved independently in many lineages of bird-pollinated plants. 
Parallel evolution describes independent evolution of a character state with similar genetic and developmental basis. 
For example, mutations on gene \textit{Pitx1} is the basis of independent loss of the pelvic girdle and fins in many populations of three-spined stickleback. 
Evolutionary reversal describes the return from a derived character state to a more ancestral state. 
For example, winged insects evolved from wingless ancestors, but many insect lineages lost wings in the course of subsequent evolution. 
It was long assumed that complex characters are unlikely to be regained once they were lost (Dollo's law). 
One exception is the aquatic larval stage of salamanders, which was lost in the subfamily Plethodontinae, but regained in one lineage of this subfamily (\textit{Desmognathus}). 

\par

Numerous related lineages can diverge from each other in a relatively short time, and in most cases, these lineages become modified for different ways of life. 
This is known as evolutionary radiation or adaptive radiation. 
For example, finches in Galapagos Island descended from a single ancestor from South America, differ in the morphology of bills, which provides adaption to diverse diets.

\end{sloppypar}
\end{document}
