\documentclass[11pt]{article}

\usepackage[UTF8]{ctex} % for Chinese 

\usepackage{setspace}
\usepackage[colorlinks,linkcolor=blue,anchorcolor=red,citecolor=black]{hyperref}
\usepackage{lineno}
\usepackage{booktabs}
\usepackage{graphicx}
\usepackage{float}
\usepackage{floatrow}
\usepackage{subfigure}
\usepackage{caption}
\usepackage{subcaption}
\usepackage{geometry}
\usepackage{multirow}
\usepackage{longtable}
\usepackage{lscape}
\usepackage{booktabs}
\usepackage{natbibspacing}
\usepackage[toc,page]{appendix}
\usepackage{makecell}
\usepackage{amsfonts}
 \usepackage{amsmath}
\usepackage[utf8]{inputenc}
\usepackage{amssymb}
\usepackage{amsthm}
\usepackage{enumerate}
\usepackage{comment}

\usepackage[backend=bibtex,style=authoryear,sorting=nyt,maxnames=1]{biblatex}
\bibliography{} % Reference bib

\title{Genetic drift}
\author{}
\date{}

\linespread{1.5}
\geometry{left=2cm,right=2cm,top=2cm,bottom=2cm}

\setlength\bibitemsep{0pt}

\begin{document}
\begin{sloppypar}
  \maketitle

  \linenumbers
Genetic drift is evolution resulted from chance events of survival, reproduction and inheritance. 
Drift has five fundamental features. 
First, drift is unbiased: allele frequency can go up or down, and no specific allele is favoured. 
Second, random fluctuations in allele frequency are larger in smaller populations. 
Third, drift causes genetic variation to be lost, as an allele frequency that fluctuates up and down randomly will eventually reach 0 or 1. 
The loss is fast in small populations with large allele frequency fluctuations. 
Forth, drift causes populations that are initially identical to become different. 
Fifth, an allele can be fixed in populations by drift, even though the allele provides no benefit of natural selection. 

\section{The genealogy of genes}
The genealogy of genes is the paths of their inheritance across generations. 
Looking backward in time, when the lineages of two gene copies merge, they coalesce. 
If we go back far enough, we are certain to arrive at the most recent common ancestor of all copies of the gene now present in the population. 
This typically happens in the recent past when the population size is small. 
In many cases, each part of the genome has a different genealogy. 

\section{Effective population size: the strength of genetic drift}
Genetic drift is a random process that is always at work. 
It is stronger in small populations, and weaker in large populations. 
Drift is also affected by distribution of age: if most individuals in the population are too young or too old to reproduce, drift is stronger than it would be if all individuals were reporductive. 
Changes in population size and unequal numbers of reproducing males and females also affect drift. 
To account for all these factors, consider an idealized population of constant size in which all individuals have an equal chance of leaving offspring. 
This constant size should make the idealized population the same strength of drift as the actual population. 
This constant size, or effective populatin size ($N_e$), thus measures strength of drift. 
A small value of $N_e$ means that drift is strong, while a large value means that drift is weak. 
If a locus evolves neutrally without any selection in a diploid organism, the average time back to the most recent common ancestor of any pair of gene copies in an extant population is $2N_e$ generations. 

\section{Populations that change in size}
Population bottleneck is the situation in which a population is reduced to a small size in a small number of generations. 
A population bottleneck causes intense genetic drift for a brief time and reduces genetic variation. 
A similar phenomenon is founder event, when a new population is begun from a small number of individuals. 

\section{Drift and genetic variation within species}
Within a species, variation of DNA sequence at a specific locus is polymorphism. 
Polymorphism can be quantified by heterozygosity $\pi$, the chance that two copies of a specific locus within a population are different. 
The expected heterozygosity resulted from neutral mutations evolving by drift in a diploid species 
\begin{equation}
    \pi \approx 4 N_e \mu_n (\pi < 0.1)
\end{equation}
where $\mu_n$ is the neutral mutation rate, the chance per generation that the locus mutates to another allele that does not change the fitness. 

\par

Polymorphism increases with the effective population size and the neutral mutation rate. 
Mutations in introns and regions between genes are largely neutral. 
For these non-coding regions, the nuetral mutation rate approximately equals the mutation rate. 
For coding regions, most mutations are nonsynonymous (protein sequence changed) and harmful, and are excluded from the population via purifying selection. 
Loci that experience purifying selection are said to be under selective constraint.
Therefore, the nuetral mutation rate of coding regions is often much lower than the mutation rate. 
As a result, polymorphism in non-coding regions is typically higher than coding regions. 
Within a coding locus, many mutations to the third codon sites are synonymous and neutral, resulting high polymorphism in these sites, compared with the first/second codon sites. 

\par

Chromosome regions with high recombination rates tend to be polymorphic.
When selection eliminates deleterious mutations in a locus and constraints it, polymorphism of nearby regions is also reduced (background selection). 
It affects large chromosome regions where recombination is rare. 


\section{Estimating population size}
Estimation of effective population size $N_e$ is from heterozygosity $\pi$ and mutation rate $\mu$. 
By sequencing several individuals in a population, heterozygosity $\pi$ of neutral regions can be estimated. 
By sequencing parents and offsprings, mutation rate $\mu$ can be estimated. 
For neutral regions, $\mu \approx \mu_n$. 
Thus, we have $N_e \approx \frac{\pi}{4\mu}$. 
For human, $N_e$ is roughly 10,000. 

\section{Drift and selection}



\end{sloppypar}
\end{document}
