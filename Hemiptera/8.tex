\documentclass[11pt]{article}

% \usepackage[UTF8]{ctex} % for Chinese 

\usepackage{setspace}
\usepackage[colorlinks,linkcolor=blue,anchorcolor=red,citecolor=black]{hyperref}
\usepackage{lineno}
\usepackage{booktabs}
\usepackage{graphicx}
\usepackage{float}
\usepackage{floatrow}
\usepackage{subfigure}
\usepackage{caption}
\usepackage{subcaption}
\usepackage{geometry}
\usepackage{multirow}
\usepackage{longtable}
\usepackage{lscape}
\usepackage{booktabs}
\usepackage{natbib}
\usepackage{natbibspacing}
\usepackage[toc,page]{appendix}
\usepackage{makecell}
\usepackage{amsfonts}
 \usepackage{amsmath}

\title{Insect symbiosis and immunity: the bean bug-Burkholderia interaction as a case study}
\author{}
\date{}

\linespread{1.5}
\geometry{left=2cm,right=2cm,top=2cm,bottom=2cm}

\begin{document}
  \maketitle

  \linenumbers
Primary symbionts refers to maternally transmitted obligated symbionts. 
They associate with host at evolutionary time scale and become vital to host survival. 
The adaptive evolution of primary symbionts is accompanied with dramatic reduction in genome size and loss of genes essential for free-living (Moran, 2003; McCutcheon and Moran, 2012). 
Secondary or facultative symbionts are vertically/horizontally transmitted, and are not essential for host survival. 
Their association with hosts is recent, and still possess free-living ability without collapsed genomes.

\section{Models for insect-bacteria symbiosis}
\subsection{Pea aphid (\textit{Acyrthosiphon pisum})}
Pea aphid (\textit{Acyrthosiphon pisum}) possesses primary symbiont \textit{Buchnera aphidicola}. 
Aphids with an unbalanced diet of plant sap are dependent on \textit{Buchnera} for acquiring essential amino acids (Douglas, 1998). 
\textit{Buchnera} lives within specialized aphid cells (bacteriocytes) that are assembled in bacteriome. 
The start of their symbiosis is estimated to be 150 million years ago. 
The genome of \textit{Buchnera} reduced greatly in size with loss of genes essential for free-living and conservation of amino acid synthesis pathways (Shigenobu and Wilson, 2011; Shigenobu \textit{et al.}, 2000).

\newline

Secondary symbionts of pea aphid include \textit{Hamiltonella defensa}, \textit{Serratia symbiotica} and \textit{Regiella insecticola}. 
They play a role in defence against parasitoid wasps and fungi, in heat tolerance and in adaption to host plants (Koga \textit{et al.}, 2003; Oliver \textit{et al.}, 2010).

\subsection{Tsetse fly (\textit{Glossina spp.})}
\textit{Wigglesworthia} is a primary symbiont of tsetse fly (Aksoy, 1995). 
It provides essential vitamin metabolites to host that is only fed on vertebrate blood. 
\textit{Wigglesworthia} symbionts are localized within bacteriocytes, and their genomes has undergone dramatic size decrease (Akman \textit{et al.}, 2002). 

\newline

\textit{Sodalis glossinidius} is a secondary symbiont of tsetse fly (Wang \textit{et al.}, 2013). 
It can colonize different organs intra- and extracellularly, and is vertically transmitted. 
\textit{Sodalis} retains many functions associated with free-living, but has lost genes involved in energy metabolism (Akman \textit{et al.}, 2001). 
\textit{In vitro} culture of \textit{Sodalis} is available (Sassera \textit{et al.}, 2013). 

\newline

\textit{Wolbachia} is located in germ line of tsetse flies and manipulates host reproduction (Werren \textit{et al.}, 2008). 
There are few offsprings between infected males and uninfected females, and no male offspring from infected females.

\subsection{Fruit fly (\textit{Drosophila})}
\textit{Wolbachia} and \textit{Spiroplasma} are vertically transmitted secondary symbionts of \textit{Drosophila} (Mateos \textit{et al.}, 2006). 
They have effects on reproductive manipulation and defence against enemies (Hamilton and Perlman, 2013).

\section{Bean bug and \textit{Burkholderia}}
Bean bug \textit{Riptortus pedestris} is a member of order Hemiptera feeding on plant sap. 
The midgut of bean bug is divided into morphologically distinct regions called M1, M2, M3, M4B and M4, where M4 is symbiotic. 
Symbiotic organ in M4 has two rows of crypts, whose lumens are densely colonized by betaproteobacterial symbionts of genus \textit{Burkholderia}. 
The symbionts of bean bug are acquired orally from rhizosphere environment during early nymphal stage. 
\textit{Burkholderia} is a soil bacterium. 
They retain free-living ability after association with bean bug and are easily cultured in labs.

\section{Symbiont \textit{Burkholderia} increases host survival under bacterial challenges}
Under bacterial challenges, symbiotic bean bugs exhibits better survival than aposymbiotic (Kim \textit{et al.}, 2015). 
This better survival retains after inhibition of cellular immunity, indicating stronger humoral immunity induced by \textit{Burkholderia}. 
Without bacterial challenges, antimicrobial peptide (AMP) (riptocin, rip-defensin and rip-thanatin) expression in fat body of symbiotic and aposymbiotic bean bugs is similar. 
However, AMP expression significantly increases in symbiotic bean bugs compared with aposymbiotic ones (Kim \textit{et al.}, 2015).

\section{Molecular changes of symbiont \textit{Burkholderia}}
\textit{Burkholderia} is Gram-negative bacterium. 
Its cell envelop consists of inner- and outer membranes. 
Lipopolysaccharide (LPS) is located at the outer part of outer membrane. 
It is composed of lipid A embedded in outer membrane and oligosaccharide connecting with O-antigen. 

\newline

In symbiotic \textit{Burkholderia}, O-antigen is lost when compared with free-living \textit {Burkholderia} (Kim \textit{et al.}, 2005). 
However, lipid A and oligosaccharide are retained. 
Besides, symbiotic \textit{Burkholderia} are more susceptible to detergent than free-living ones, indicating compromised cell membrane integrity.

\section{Symbiotic \textit{Burkholderia} is susceptible to host immunity}
Free-living \textit{Burkholderia} are resistant to antimicrobial activity of bean bug haemolymph (Loutet and Valvano, 2011), while symbiotic ones are highly susceptible (Kim \textit{et al.}, 2015). 
When bean bug AMPs are purified, symbiotic \textit{Burkholderia} are more susceptible to riptoctin and rip-defensin than free-living ones (Kim \textit{et al.}, 2015). 
Ultimately, injected symbiont \textit{Burkholderia} are removed much faster by bean bug immunity than free-living ones (Kim \textit{et al.}, 2015).

\section{Suppression of bean bug immunity in symbiotic organ}
Systemic injection of symbiotic \textit{Burkholderia}, free-living \textit{Burkholderia} and \textit{Escherichia coli} triggers similar level of AMP expression in fat body. 
However, in midgut M4 region, the expression of AMPs is similar in symbiotic and aposymbiotic bean bugs. 
The AMP expression in M4 region is lower than basal expression of AMPs in fat body (Kim \textit{et al.}, 2015). 
These indicate potential immune privilege of symbiotic organ. 

\section{Bean bug regulates symbiont population by immunity}
The susceptibility of \textit{Burkholderia} to bean bug immunity could be an advantage for easy management of symbiotic population. 
During nymphal stage, the size of symbiont population increases. 
However, pattern of transient decrease of symbiotic population is observed prior to moulting period in each instar stage. 
This transient decrease is corresponding to increase of antimicrobial activity of symbiont organ, including up-regulated expression of c-type lysosome and riptocin (Kim \textit{et al.}, 2014). 

\newline

Another mechanism for symbiont population management in bean bugs is related to M4B region of midgut. 
M4B region of symbiotic bean bug exhibits strong antimicrobial activity, while in aposymbiotic bean bug, little antimicrobial activity is exhibited. 
Besides, died bacterial symbionts present in M4B region. 
These indicate that the antimicrobial activity of M4B midgut is induced by \textit{Burkholderia} (Kim \textit{et al.}, 2013). 
One of components responsible for antimicrobial activity of M4B region is cathepsin-L-like protease (Byeon \textit{et al.}, 2015), which is highly and preferentially expressed in M4B region of symbiotic bean bugs (Futahashi \textit{et al.}, 2013). 

\newline

Antimicrobial activity exhibited by M4B and M4 region of bean bugs is only effective for symbiont \textit{Burkholderia}, but free-living \textit{Burkholderia} are resistant (Byeon \textit{et al.}, 2015; Kim \textit{et al.}, 2013). 

\end{document}