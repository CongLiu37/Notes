\documentclass[11pt]{article}

% \usepackage[UTF8]{ctex} % for Chinese 

\usepackage{setspace}
\usepackage[colorlinks,linkcolor=blue,anchorcolor=red,citecolor=black]{hyperref}
\usepackage{lineno}
\usepackage{booktabs}
\usepackage{graphicx}
\usepackage{float}
\usepackage{floatrow}
\usepackage{subfigure}
\usepackage{caption}
\usepackage{subcaption}
\usepackage{geometry}
\usepackage{multirow}
\usepackage{longtable}
\usepackage{lscape}
\usepackage{booktabs}
\usepackage{natbib}
\usepackage{natbibspacing}
\usepackage[toc,page]{appendix}
\usepackage{makecell}
\usepackage{amsfonts}
 \usepackage{amsmath}

\title{Convergent patterns in the evolution of mealybug symbioses involving different intrabacterial symbiosis}
\author{}
\date{}

\linespread{1.5}
\geometry{left=2cm,right=2cm,top=2cm,bottom=2cm}


\begin{document}
\begin{sloppypar}
  \maketitle

  \linenumbers
Manna mealybug \textit{Trabutina mannipara} contains a betaproteobacterial symbiont \textit{Tremblaya}, which contains a gammaproteobacterial symbiont \textit{Trabutinella endobia}. 
Genomic sequences of \textit{Tremblaya} are highly syntenic and harbored nearly identical sets of genes in the manna/citrus mealybug in accordance with a monophyletic origin of the outer symbionts among mealybugs. 
The genome of the intrabacterial symbiont \textit{Trabutinella} shows only minimal synteny with that of \textit{Moranella} from citrus mealybug, which is consistent with the distinct evolutionary origin of these symbionts. 
Although \textit{Trabutinella} genome is much smaller than \textit{Moranella} genome (McCutcheon & von Dohlen, 2011), \textit{Trabutinella} genome is likely still in the process of reduction as indicated by the presence of 27 pseudogenes (Moran and
Bennett, 2014). 

\newline

Genes of bacterial origin are found in manna mealybug genome. 
These genes are involved in synthesis of essential amino-acid, biotin and riboflavin. 
Many laterally acquired genes in genome of manna/citrus mealybug appear as sisters in phylogenetic tree, indicating that they share a common origin and were present in ancestral mealybugs before diversification of manna/citrus mealybug.

\newline

Symbiotic partners of the manna mealybug and citrus mealybug systems partition the synthesis of essential amino acids in a highly similar manner. 
In most of the essential amino acid production pathways, exactly the same steps are carried out by the inner or the outer symbiont in manna mealybug and citrus mealybug, despite the independent origin of the intrabacterial symbionts. 
A similar, yet far less complex situation has been observed among members of Auchenorrhyncha, where \textit{Sulcia} synthesizes eight or seven essential amino acids while the remaining two or three are produced by different co-symbionts in different lineages, for instance. by \textit{Baumannia} in sharpshooters, \textit{Hodgkinia} in cicadas and \textit{Zinderia} in spittlebugs (McCutcheon and Moran, 2010; Bennett and Moran, 2013).

\newline

A conceivable scenario explaining the observed similarities between the manna mealybug and citrus mealybug symbioses would be that before manna and citrus mealybug diverged, the \textit{Tremblaya} ancestor was already infected by an (intra-)bacterial symbiont in ancestral mealybugs and this ancient association would have facilitated reduction of the \textit{Tremblaya} genome and has shaped its gene repertoire. 
The inner symbiont might have been subsequently replaced in the ancestor of citrus and/or manna mealybug, with the new symbiont taking over the functions required by \textit{Tremblaya} and at the same time allowing for loss of further genes from the \textit{Tremblaya} genome, which would account for the observed differences between the two systems.
This scenario is favoured over alternative scenarios such as ancient associations of \textit{Tremblaya} with another bacteriocyte-associated symbiont because of the high level of congruence between the loss of genes in different \textit{Tremblaya} strains at intermediate steps of essential amino-acid synthesis pathways. 

\end{sloppypar}
\end{document}