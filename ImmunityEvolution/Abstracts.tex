\documentclass[11pt]{article}

% \usepackage[UTF8]{ctex} % for Chinese 

\usepackage{setspace}
\usepackage[colorlinks,linkcolor=blue,anchorcolor=red,citecolor=black]{hyperref}
\usepackage{lineno}
\usepackage{booktabs}
\usepackage{graphicx}
\usepackage{float}
\usepackage{floatrow}
\usepackage{subfigure}
\usepackage{caption}
\usepackage{subcaption}
\usepackage{geometry}
\usepackage{multirow}
\usepackage{longtable}
\usepackage{lscape}
\usepackage{booktabs}
\usepackage{natbib}
\usepackage{natbibspacing}
\usepackage[toc,page]{appendix}
\usepackage{makecell}
\usepackage{amsfonts}
 \usepackage{amsmath}

\title{Abstracts}
\author{}
\date{}

\linespread{1.5}
\geometry{left=2cm,right=2cm,top=2cm,bottom=2cm}

\begin{document}
\begin{sloppypar}
  \maketitle

  \linenumbers

\textbf{1. Viljakainen, 2015, Evolutionary genetics of insect innate immunity.} \newline
Toll and IMD pathways are well conserved among insects. \newline
Immune gene families evolution: expansion/shrink. \newline
Selection mode of immune genes. \newline

\par

\textbf{2. Hosokawa et al., 2007, Obigate symbiont involved in pest status of host insect.} \newline
A pest stinkbug species, \textit{Megacopta punctatissima}, performed well on crop legumes, while a closely related non-pest species, \textit{Megacopta cribraria}, suffered low egg hatch rate on the plants. 
When their obligate gut symbiotic bacteria were experimentally exchanged between the species, their performance on the crop legumes was completely reversed: 
the pest species suffered low egg hatch rate, whereas the non-pest species restored normal egg hatch rate and showed good performance. 
The low egg hatch rates were attributed to nymphal mortality before or upon hatching, which were associated with the symbiont from the non-pest stinkbug irrespective of the host insect species.

\par

\textbf{3. Boehm, 2012 Evolution of vertebrate immunity.} \newline
Could it be possible then that an immune system employing structurally diversified antigen receptors facilitated increased species-richness in autochthonous microbial communities, for example, in the intestine? 
The selective advantage of increasing antigen receptor diversity with respect to the species-richness of microbiomes is illustrated by the role of secreted antibodies, such as IgA in mammals, in the maintenance of microbial homeostasis on mucosal surfaces; defective structural diversification of secreted antibodies is associated with dysbiosis, which is characterized by generally lower species diversity and an ‘unhealthy’ composition of the microbiome. 
Autoimmunity can be a price for the evolution of adaptive immunity. 

\textbf{4. McFall-Ngai, 2007, Care for the community.} \newline
A memory-based immune system may have evolved in vertebrates because of the need to recognize and manage complex communities of beneficial microbes. \newline
Invertebrates are no less challenged by the microbial world than vertebrates, nor are they less able to remain healthy by entirely relying on innate immunity. 
Invertebrates often harbor much less diversified symbiont communities compared with vertebrates. 
There are three possible strategies for management of symbionts in invertebrates: 
maintain symbionts intracellularly; 
build physical barrier between host tissue and symbionts; 
express a high number of specific recognition components of immate immunity. 

\textbf{5. Kitao et al., 2006, Robustness trade-offs and host–microbial symbiosis in the immune system.}
Commensal bacterial flora as an integral part of the host defense system

\textbf{6. }
\end{sloppypar}
\end{document}