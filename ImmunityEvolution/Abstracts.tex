\documentclass[11pt]{article}

% \usepackage[UTF8]{ctex} % for Chinese 

\usepackage{setspace}
\usepackage[colorlinks,linkcolor=blue,anchorcolor=red,citecolor=black]{hyperref}
\usepackage{lineno}
\usepackage{booktabs}
\usepackage{graphicx}
\usepackage{float}
\usepackage{floatrow}
\usepackage{subfigure}
\usepackage{caption}
\usepackage{subcaption}
\usepackage{geometry}
\usepackage{multirow}
\usepackage{longtable}
\usepackage{lscape}
\usepackage{booktabs}
\usepackage{natbib}
\usepackage{natbibspacing}
\usepackage[toc,page]{appendix}
\usepackage{makecell}
\usepackage{amsfonts}
 \usepackage{amsmath}

\title{Abstracts}
\author{}
\date{}

\linespread{1.5}
\geometry{left=2cm,right=2cm,top=2cm,bottom=2cm}

\begin{document}
\begin{sloppypar}
  \maketitle

  \linenumbers

\textbf{Viljakainen, 2015, Evolutionary genetics of insect innate immunity.} \newline
Toll and Imd signaling pathways are well conserved across insects. 
Antimicrobial peptides (AMPs) are the most labile component of insect immunity showing rapid gene birth-death dynamics and lineage-specific gene families. 
Immune genes and especially recognition genes are frequently targets of positive selection driven by host-pathogen arms races. 
Homology-based annotation is useful but to some extent restricted approach to find immune-related genes in a newly sequenced genome. 
Novel immune genes have been found in many insects and should be looked for in future research.

\par

\textbf{Boehm, 2012 Evolution of vertebrate immunity.} \newline
Could it be possible then that an immune system employing structurally diversified antigen receptors facilitated increased species-richness in autochthonous microbial communities, for example, in the intestine? 
The selective advantage of increasing antigen receptor diversity with respect to the species-richness of microbiomes is illustrated by the role of secreted antibodies, such as IgA in mammals, in the maintenance of microbial homeostasis on mucosal surfaces; defective structural diversification of secreted antibodies is associated with dysbiosis, which is characterized by generally lower species diversity and an ‘unhealthy’ composition of the microbiome. 
Autoimmunity can be a price for the evolution of adaptive immunity. 

\textbf{McFall-Ngai, 2007, Care for the community.} \newline
A memory-based immune system may have evolved in vertebrates because of the need to recognize and manage complex communities of beneficial microbes. 
Invertebrates are no less challenged by the microbial world than vertebrates, nor are they less able to remain healthy by entirely relying on innate immunity. 
Invertebrates often harbor much less diversified symbiont communities compared with vertebrates. 
There are three possible strategies for management of symbionts in invertebrates: 
maintain symbionts intracellularly; 
build physical barrier between host tissue and symbionts; 
express a high number of specific recognition components of immate immunity. 

\end{sloppypar}
\end{document}