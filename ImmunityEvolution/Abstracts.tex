\documentclass[11pt]{article}

% \usepackage[UTF8]{ctex} % for Chinese 

\usepackage{setspace}
\usepackage[colorlinks,linkcolor=blue,anchorcolor=red,citecolor=black]{hyperref}
\usepackage{lineno}
\usepackage{booktabs}
\usepackage{graphicx}
\usepackage{float}
\usepackage{floatrow}
\usepackage{subfigure}
\usepackage{caption}
\usepackage{subcaption}
\usepackage{geometry}
\usepackage{multirow}
\usepackage{longtable}
\usepackage{lscape}
\usepackage{booktabs}
\usepackage{natbib}
\usepackage{natbibspacing}
\usepackage[toc,page]{appendix}
\usepackage{makecell}
\usepackage{amsfonts}
 \usepackage{amsmath}

\title{Abstracts}
\author{}
\date{}

\linespread{1.5}
\geometry{left=2cm,right=2cm,top=2cm,bottom=2cm}

\begin{document}
\begin{sloppypar}
  \maketitle

  \linenumbers

\textbf{Viljakainen, 2015, Evolutionary genetics of insect innate immunity.} \newline
Toll and Imd signaling pathways are well conserved across insects. 
Antimicrobial peptides (AMPs) are the most labile component of insect immunity showing rapid gene birth-death dynamics and lineage-specific gene families. 
Immune genes and especially recognition genes are frequently targets of positive selection driven by host-pathogen arms races. 
Homology-based annotation is useful but to some extent restricted approach to find immune-related genes in a newly sequenced genome. 
Novel immune genes have been found in many insects and should be looked for in future research.

\par

\textbf{Boehm, 2012 Evolution of vertebrate immunity.} \newline
Could it be possible then that an immune system employing structurally diversified antigen receptors facilitated increased species-richness in autochthonous microbial communities, for example, in the intestine? 
The selective advantage of increasing antigen receptor diversity with respect to the species-richness of microbiomes is illustrated by the role of secreted antibodies, such as IgA in mammals, in the maintenance of microbial homeostasis on mucosal surfaces; defective structural diversification of secreted antibodies is associated with dysbiosis, which is characterized by generally lower species diversity and an ‘unhealthy’ composition of the microbiome. 
Autoimmunity can be a price for the evolution of adaptive immunity. 

\par

\textbf{McFall-Ngai, 2007, Care for the community.} \newline
A memory-based immune system may have evolved in vertebrates because of the need to recognize and manage complex communities of beneficial microbes. 
Invertebrates are no less challenged by the microbial world than vertebrates, nor are they less able to remain healthy by entirely relying on innate immunity. 
Invertebrates often harbor much less diversified symbiont communities compared with vertebrates. 
There are three possible strategies for management of symbionts in invertebrates: 
maintain symbionts intracellularly; 
build physical barrier between host tissue and symbionts; 
express a high number of specific recognition components of immate immunity. 

\par

\textbf{Hoang & King, 2022, Symbiont-mediated immune priming in animals through an evolutionary lens.} \newline
While research on symbiont-mediated immune priming (SMIP) has focused on ecological impacts and agriculturally important organisms, the evolutionary implications of SMIP are less clear. 
Here, we review recent advances made in elucidating the ecological and molecular mechanisms by which SMIP occurs. 
We draw on current works to discuss the potential for this phenomenon to drive host, parasite, and symbiont evolution. 
We also suggest approaches that can be used to address questions regarding the impact of immune priming on host-microbe dynamics and population structures. 
Finally, due to the transient nature of some symbionts involved in SMIP, we discuss what it means to be a protective symbiont from ecological and evolutionary perspectives and how such interactions can affect long-term persistence of the symbiosis. 

\par

\textbf{Sharp & Hoster, 2022, Host control and the evolution of cooperation in host microbiomes.} \newline
It is often suggested that the mutual benefits of host-microbe relationships can alone explain cooperative evolution. 
Here, we evaluate this hypothesis with evolutionary modelling. 
Our model predicts that mutual benefits are insufficient to drive cooperation in systems like the human microbiome, because of competition between symbionts. 
However, cooperation can emerge if hosts can exert control over symbionts, so long as there are constraints that limit symbiont counter evolution. 
We test our model with genomic data of two bacterial traits monitored by animal immune systems. 
In both cases, bacteria have evolved as predicted under host control, tending to lose flagella and maintain butyrate production when host-associated. 
Moreover, an analysis of bacteria that retain flagella supports the evolution of host control, via toll-like receptor 5, which limits symbiont counter evolution. 
Our work puts host control mechanisms, including the immune system, at the centre of microbiome evolution. 

\par

\textbf{Barribeau \textit{et al.}, 2015, A depauperate immune repertoire precedes evolution of sociality in bees.} \newline
We find that the immune systems of these bumblebees, two species of honeybee, and a solitary leafcutting bee, are strikingly similar. 
Transcriptional assays confirm the expression of many of these genes in an immunological context and more strongly in young queens than males, affirming Bateman’s principle of greater investment in female immunity. 
We find evidence of positive selection in genes encoding antiviral responses, components of the Toll and JAK/STAT pathways, and serine protease inhibitors in both social and solitary bees. 
Finally, we detect many genes across pathways that differ in selection between bumblebees and honeybees, or between the social and solitary clades.

\par

\textbf{Martinez \textit{et al.}, 2016, Addicted? Reduced host resistance in populations with defensive symbionts.} \newline
Heritable symbionts that protect their hosts from pathogens have been described in a wide range of insect species. 
By reducing the incidence or severity of infection, these symbionts have the potential to reduce the strength of selection on genes in the insect genome that increase resistance. 
Therefore, the presence of such symbionts may slow down the evolution of resistance. 
Here we investigated this idea by exposing \textit{Drosophila melanogaster} populations to infection with the pathogenic Drosophila C virus (DCV) in the presence or absence of \textit{Wolbachia}, a heritable symbiont of arthropods that confers protection against viruses. 
After nine generations of selection, we found that resistance to DCV had increased in all populations. 
However, in the presence of \textit{Wolbachia} the resistant allele of \textit{pastrel}—a gene that has a major effect on resistance to DCV—was at a lower frequency than in the symbiont-free populations. 
This finding suggests that defensive symbionts have the potential to hamper the evolution of insect resistance genes, potentially leading to a state of evolutionary addiction where the genetically susceptible insect host mostly relies on its symbiont to fight pathogens.

\par



\end{sloppypar}
\end{document}