\documentclass[11pt]{article}

% \usepackage[UTF8]{ctex} % for Chinese 

\usepackage{setspace}
\usepackage[colorlinks,linkcolor=blue,anchorcolor=red,citecolor=black]{hyperref}
\usepackage{lineno}
\usepackage{booktabs}
\usepackage{graphicx}
\usepackage{float}
\usepackage{floatrow}
\usepackage{subfigure}
\usepackage{caption}
\usepackage{subcaption}
\usepackage{geometry}
\usepackage{multirow}
\usepackage{longtable}
\usepackage{lscape}
\usepackage{booktabs}
\usepackage{natbib}
\usepackage{natbibspacing}
\usepackage[toc,page]{appendix}
\usepackage{makecell}
\usepackage{amsfonts}
 \usepackage{amsmath}

\title{Evolution of animal immunity in the light of beneficial symbioses}
\author{}
\date{}

\linespread{1.5}
\geometry{left=2cm,right=2cm,top=2cm,bottom=2cm}

\bibliographystyle{plain}
\bibliographystyle{unsrtnat}
\bibliographystyle{plainnat}
\bibliographystyle{dinat}
\bibliographystyle{abbrvnat}
\bibliographystyle{rusnat}
\bibliographystyle{ksfh_nat}
\setcitestyle{round}

\begin{document}
\begin{sloppypar}
  \maketitle

  \linenumbers
There are three key ways for host-symbiont interactions. 
First, host immunity can play a role in regulation of symbionts. 
Second, symbionts can protect host against pathogens. 
THird, symbionts can influence the maturation of host immune system. 

\section*{Host immunity in regulation of symbionts}
Host can regulate symbiont populations. 
For example, cereal weevil requires endosymbionts for exoskeleton development, after which symbionts are eliminated by apoptosis of bacteriocytes. 
Bean bug \textit{Riptortus pedestris} up-regulates immunity and digests bacteriocytes before moulting, reducing symbiont populations as moulting is energy-costing and leaves bean bug vulnerable to infections and injuries. 

\par

The influence of host immunity on symbiosis likely dependent on symbiont transmission mode. 
In horizontal transmission, host select proper symbionts from pools of microorganisms. 
In vertical transmission, symbionts are transmitted vertically from parents (often mothers) to offsprings. 

\par

For vertebrate hosts, horizontal transmission is the major mode. 
Vertebrates often harbour a large and diversified community of microbes. 
By leveraging innate and adaptive immunity, vertebrates can mount rapid and robust responses to large numbers of microorganisms. 
In mice, immune system discriminates between pathogens and symbionts, and segregates symbionts to proper host tissues. 
In mice gut, epithelium tissue is protected from lumen by a mucus layer. 
Both pathogens and symbionts can enter gut lumen. 
Pathogens are segregated from mucus layer by immune responses. 
Symbionts enter mucus layer and are segregates from epithelium by antimicrobial peptides and immunoglobulins. 

\par

Invertebrates likely to regulate horizontally-transmitted symbiosis by compartmentalized innate immune responses. 
By compartmentalization, hosts can invest the most energy into screening microorganisms and mounting immunity in regions exposed to a wide array of microorganisms, while reduce immune investment elsewhere. 
For example, Hawaiian bobtail squid \textit{Euprymna scolopes} screens large quantities of microbes by defenses including physical barrier, morphological changes and innate immunity. 
Thus, squid limits colonization in light organ to bacteria with specific characteristics, including symbiotic molecular patterns, biofilm formation, bioluminescence and nitric oxide resistance. 
In this way, squid limits colonization of specific strains of \textit{Vibrio fischeri} in light organ. 
Fruit fly \textit{Drosophila melanogaster} uses physical barrier, morphological changes and compartmentalized immune expressions to eliminate pathogens and to limit few symbionts in gut microbiome. 

\par

In vertical transmission, hosts pass few symbionts directly to offsprings in ways including providing symbiont-enclosed capsules, smearing eggs with symbionts, and symbiotic infection of embryo. 
Passaged symbionts undergo population bottlenecks and have little chance for getting virulance factors via horizontal gene transfer with environmental microbes. 
In many cases, vertical transmission is coupled with sequestration of symbionts into specialized cells. 
Sequestration allows host to limit symbiont populations and reduce horizontal gene transfer with fewer investment. 
For example, in cereal weevil, antimicrobial peptides are not expressed in bacteriocytes except ColA, whose knock-out leads to symbiont over-proliferation and escape to other tissues. 

\par

The evolution of immunity in symbiont regulation is likely dependent on transmission mode. 
Vertical transmission has evolved multiple times among invertebrates. 
Compared with horizontal transmission, it may allow reduced investment in symbiont regulation because 
(1) horizontal transmission requires screening for symbionts and discarding pathogens from environmental pools; 
(1) fitness of vertically-transmitted symbionts is dependent on host fitness, and therefore, they are less likely to exploit hosts; 
(2) vertical transmission limits chance for horizontal gene transfer from environments, reducing possibility that symbionts acquire virulance factors; 
(3) vertically-transmitted symbionts are often sequestered into host cells, allowing tightly control via nutrition availability. 
Therefore, it is possible that selection pressure on immunity is weaker in hosts with vertically-transmitted symbionts than hosts with horizontally-transmitted symbionts. 
However, vertical transmission provides less flexibility in the face of changing environment conditions, which can be especially important for long-living hosts. 
Adaptive immunity, in turn, is assumed to have evolved to affording regulation of a diversified symbiont communities, as complex symbiont communities are often found in vertebrates. 
However, adaptive immunity only evolved independently twice in jawed vertebrates and jawless vertebrates, indicating the co-occurrence of complex microbiomes and adaptive immunity is resulted from common ancestors instead of convergent evolution under selection. 


\end{sloppypar}
\end{document}