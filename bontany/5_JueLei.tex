\documentclass[11pt]{article}

\usepackage[UTF8]{ctex} % for Chinese 

\usepackage{setspace}
\usepackage[colorlinks,linkcolor=blue,anchorcolor=red,citecolor=black]{hyperref}
\usepackage{lineno}
\usepackage{booktabs}
\usepackage{graphicx}
\usepackage{float}
\usepackage{floatrow}
\usepackage{subfigure}
\usepackage{caption}
\usepackage{subcaption}
\usepackage{geometry}
\usepackage{multirow}
\usepackage{longtable}
\usepackage{lscape}
\usepackage{booktabs}
\usepackage{natbib}
\usepackage{natbibspacing}
\usepackage[toc,page]{appendix}
\usepackage{makecell}
\usepackage{amsfonts}
 \usepackage{amsmath}

\title{蕨类(Pteridophyta)}
\author{}
\date{}

\linespread{1.5}
\geometry{left=2cm,right=2cm,top=2cm,bottom=2cm}

\begin{document}
\begin{sloppypar}
  \maketitle

  \linenumbers
蕨类植物和种子植物具有维管系统(vascular system),称为维管植物。
维管植物中轴部分称为中柱,可根据初生木质部和初生韧皮部的排列方式分类。
原生中柱(protostele)中央为柱状木质部,无髓部,外侧为圆筒状韧皮,分为单中柱(haplostele),星状中柱(actinostele)和编织中柱(plectostele)。
单中柱中央为圆柱状木质部。
星状中柱中央木质部横切为星状。
编织中柱中央木质部横切为分离片状。
管状中柱(siphonostele)中央为髓部,木质部包围髓部成圆筒状,分为双韧管状中柱(amphiphloic siphonostele)和外韧管状中柱(ectophloic siphonostele)。
双韧管状中柱之韧皮部位于木质部内外两侧。
外韧管状中柱之韧皮部围绕木质部外侧。
网状中柱(dictyostele)源于管状中柱。
由于茎节间短,分支密集,包裹着髓部的维管束互相连接成网状。
真中柱(eustele)是种子植物的初生维管柱,其木质部与韧皮部内外并生成为维管束,在横切面上围绕髓部辐射排列,间隔为薄壁组织。
散生中柱(atactostele)木质部和韧皮部内外并生多束,散生在基本组织内。

\par

蕨类植物为高等孢子植物和低等维管植物,异形世代交替,大多孢子体世代占优。
孢子体多为多年生,体表被附属。
根状茎多在地下横走,或匍匐地面,二叉分枝或单轴分枝,亦有不分枝。
茎上常生不定根。
木质部为厚壁管胞和薄壁组织。
韧皮部为筛胞,筛管和薄壁组织。
叶有小型叶(microphyll)和大型叶(macrophyll)两种。
小型叶系延生起源,无叶隙,无叶柄。
大型叶系顶枝起源,有叶柄,叶片,叶隙,叶脉分枝。
多数蕨类植物的叶兼司营养代谢和产生孢子。
孢子囊源于叶表皮细胞。
原始类群孢子囊源自一群细胞,大,无柄,厚壁,有气孔。
高等类群孢子囊源自单细胞,小,有三列细胞构成的柄,薄壁。

\par

低等蕨类配子体辐射对称,无叶绿体,通过菌根获得营养,精子器和颈卵器埋在内部。
高等类群配子体绿色,腹背分化,腹面有精子器和颈卵器。
精子有鞭毛,借水游动至颈卵器与卵结合,受精卵发育后配子体死亡。

\section{松蕨亚门(Psilophytina)}
孢子体有匍匐根状茎和直立茎,二叉分枝。
根状茎上生假根,司固着,吸收。
原生中柱,横切星状。
外始式木质部,螺纹或梯纹管胞,无髓部。
无叶,茎上有螺旋排列的绿色叶状瓣。
厚壁孢子囊生于枝端或孢子叶近端。
孢子圆形。
仅存松蕨目(Psilotales),含松叶蕨(\textit{Psilotum})和梅溪蕨(\textit{Tmesipteris})。

\section{石松亚门(Lycophytina)}
孢子体有根茎叶的分化。
茎多二叉分枝,原生中柱,外始式木质部,梯纹管胞为主。
小型叶,有中肋。
厚壁孢子囊侧生于叶腋或叶腋上方茎枝。
孢子叶聚于分枝顶端,形成孢子叶球。

\subsection{石松目(Lycopodiales)}
孢子体茎直立,少数悬垂,有分枝根状茎。
叶轮生或排成紧密螺旋,无叶舌。
孢子囊肾形,有短柄,生于叶腋。
含石松科(Lycopodiaceae)和石杉科(Huperziaceae)。
石松科孢子叶和营养叶异形,孢子叶聚为孢子叶球,孢子壁有网状或颗粒状纹。
石杉科孢子叶和营养叶同形,孢子叶散生,孢子壁有蜂窝状纹。
如石松(\textit{Lycopodium})。

\subsection{卷柏目(Selaginellales)}
孢子体仅卷柏(\textit{Selaginella})。
草本,平卧,有背腹之分。
匍匐茎分枝处有向下的细长根托,先端丛生不定根。
茎表皮无气孔,皮层与中柱间有大间隙。
小型叶,鳞片状,腹面基部有叶舌。
叶多排为四纵列。
孢子叶聚集成孢子叶穗。

\section{水韭亚门(Isoephytina)}
仅水韭(\textit{Isoetes})。
近水生。
茎粗短块状,下有二叉分枝的根托,上生须状不定根。
茎顶部螺旋排列叶丛。
最外围的叶不育,向内分化为大孢子叶,小孢子叶,未成熟的孢子叶和幼叶。
叶基宽,向上突然收缩为锥状。
孢子囊生于叶舌下方凹穴。

\section{楔叶蕨亚门(Sphenophytina)}
俗称木贼。
孢子体有根茎叶之分化。
茎二叉或单轴分枝,分节,节间中空,管状中柱转化为具节中柱,有中央空腔和原生木质部空腔。
内始式木质部,有梯纹,孔纹管胞,间有导管。
叶小,轮生鞘状。
茎分枝与叶互生。
有孢子叶聚于枝顶,成球状。
仅木贼科(Equisetaceae),含问荆(\textit{Equisetum})和木贼(\textit{Hippochaete})。

\section{真蕨亚门(Filicophytina)}
孢子体有根茎叶分化。
除树蕨,均有根状茎,二叉分枝至单轴分枝。
中柱、管胞多样,偶有导管。
叶大,顶枝起源,有叶片和叶柄。
有叶轴,叶脉多样。
孢子囊群生于叶缘、叶背或孢子叶。
配子体小,绿色叶状,精子器和颈卵器位于腹面。

\subsection{厚囊蕨纲(Eusporangiopsida)}
孢子囊源于一群细胞。
孢子囊壁为多层细胞,有气孔和短柄,有数层源自孢子囊壁的绒毡层。
孢子囊生于特化孢子叶或普通叶背面。
配子体地下生,有菌根。

\subsubsection{瓶尔小草目(Ophioglossales)}
草本。
菌根无毛。
茎稍肉质,深埋土中。
每年生一营养叶。
叶有鞘状托叶。
孢子囊穗有柄,生于总叶柄顶端或营养叶基部。
如瓶尔小草(\textit{Ophioglossum})。

\subsubsection{观音坐莲目(Angiopteriales)}
茎块状,连同叶基、托叶形成硕大莲座状结构,外被毛或鳞片。
羽状或掌状复叶,叶柄基部一对托叶。
孢子囊聚合成群,有孔缝或裂缝,顶端有环带状增厚细胞。
配子体心形,分背腹,有中脉。
如莲座蕨(\textit{Angiopteris})。

\subsection{原始薄壁蕨纲(Protoleptosporangiopsida)}
孢子囊源自单个细胞,囊壁为单层细胞,绒毡层源自单个囊壁细胞。
孢子囊一侧有数个厚壁细胞。
配子体叶状长心形。
如紫萁(\textit{Osmunda})。

\subsection{薄囊蕨纲(Leptosporangiopsida)}
孢子囊源于一个圆锥形原始细胞。
囊壁为单层细胞,有环带,两层绒毡层源于单个囊壁细胞。
孢子囊群生于叶背、叶缘或成孢子囊果。
配子体地上生,分背腹。

\subsubsection{水龙骨目(Polypodiales)}
多陆生或附生。
孢子囊聚集成群。
原始类群孢子囊同时发育,较演化的类群孢子囊顺序向基部发育,高等类群孢子囊发育无次序。
孢子同型。
如蕨(\textit{Pteridium})。

\subsubsection{苹目(Marsileales)}
小型水生蕨类。
根状茎有双韧管状中柱,根发达,叶两列。
有特化的孢子囊果,内有多个孢子囊群,群内大小孢子囊混生。
孢子囊果壁由羽片变态而成。
仅苹科(Marsileaceae)。
如四叶苹(\textit{Marsilea quadrifolia})。

\subsubsection{槐叶苹目(Salviniales)}
漂浮水生植物。
有孢子囊果,果壁源自囊群盖。
孢子囊果单性。
孢子囊向基顺序发育。
如槐叶苹(\textit{Salvinia})、满江红(\textit{Azolla})。

\end{sloppypar}
\end{document}