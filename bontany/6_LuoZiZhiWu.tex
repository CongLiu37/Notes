\documentclass[11pt]{article}

\usepackage[UTF8]{ctex} % for Chinese 

\usepackage{setspace}
\usepackage[colorlinks,linkcolor=blue,anchorcolor=red,citecolor=black]{hyperref}
\usepackage{lineno}
\usepackage{booktabs}
\usepackage{graphicx}
\usepackage{float}
\usepackage{floatrow}
\usepackage{subfigure}
\usepackage{caption}
\usepackage{subcaption}
\usepackage{geometry}
\usepackage{multirow}
\usepackage{longtable}
\usepackage{lscape}
\usepackage{booktabs}
\usepackage{natbibspacing}
\usepackage[toc,page]{appendix}
\usepackage{makecell}
\usepackage{amsfonts}
 \usepackage{amsmath}
\usepackage[utf8]{inputenc}
\usepackage{amssymb}
\usepackage{amsthm}
\usepackage{enumerate}
\usepackage{comment}

\usepackage[backend=bibtex,style=authoryear,sorting=nyt,maxnames=1]{biblatex}
\bibliography{} % Reference bib

\title{裸子植物(Gymnospermae)}
\author{}
\date{}

\linespread{1.5}
\geometry{left=2cm,right=2cm,top=2cm,bottom=2cm}

\setlength\bibitemsep{0pt}

\begin{document}
\begin{sloppypar}
  \maketitle

  \linenumbers
种子植物门(Spermayophyta)以种子繁殖,包括裸子植物亚门(Gymnospermae)和被子植物亚门。
种子植物孢子体发达。
根系发达。
有真中柱,内外并生型维管束,有形成层,可次生生长。
大型叶,内部结构复杂,发展出表皮气孔和毛被等附属。
输导组织最初只有管胞和筛胞分司运输水和营养,后发展为导管、筛管和伴胞。
支持组织也由兼司输水和支撑的管胞发展为专司支撑的木纤维。

\par

种子植物配子体高度简化。
孢子异形。
大小孢子囊内的大小孢子发育为雌雄配子体,均不离开孢子体独立生活。
小孢子在小孢子囊内发育为原叶细胞、管细胞和生殖细胞。
大孢子母细胞减数分裂为四个链状排列的大孢子,其中位于远珠孔端的大孢子发育为雌配子体。
雄配子经花粉管,在胚珠内完成受精,产生种子。
种子包括种皮、胚和胚乳。
种皮源自珠被,胚是精卵结合的产物,胚乳是营养组织。
裸子植物胚乳是雌配子体的一部分,被子植物胚乳是一个精子和两个极核结合而来的。

\par

胚珠、花粉管和种子是种子植物最本质的结构。
裸子植物的胚珠和种子生于开放的大孢子叶,花粉粒在胚珠中萌发。
被子植物的胚珠和种子为心皮包裹,形成由子房、花柱、柱头构成的雌蕊,花粉粒在柱头萌发,形成果实。

\par

裸子植物为多年生木本植物,多为单轴分枝的高大乔木,主根发达。
分枝有长枝和短枝之分。
长枝细长,无限生长,叶在枝上螺旋排列。
短枝粗短,生长缓慢,叶簇生于顶。
有真中柱和形成层,次生生长。
输导组织多管胞、筛胞。
叶针形、条形或鳞片状。
条形叶面气孔单列成气孔线,叶被气孔多列成浅色气孔带。
小孢子叶(雄蕊)下生贮满小孢子(花粉)的囊,聚为孢子叶球。
大孢子叶(心皮)丛生或聚生。
胚珠由珠被(大孢子囊外侧附属物)、珠心、珠柄组成,顶端有珠孔。
雌配子体由大孢子发育而来,下端发育为胚乳,顶端有颈卵器。
雄配子体在小孢子囊发育,花粉经风传播到珠孔,发育形成花粉管,释放两个精子。
种子包括胚、胚乳和种皮。胚来自受精卵。
种皮来自珠被。
亦有大孢子叶变态而来的假种皮包裹种子。
大孢子叶聚生为球果。

\section{苏铁纲(Cycadopsida)}
茎柱状,鲜有分枝,生长缓慢,皮层与髓部发达,维管束相对皮层和髓部较少,常有黏液沟。
有羽状复叶和鳞片叶,羽状叶脱落后在茎上留下叶基。
大小孢子叶球单性异株。
大孢子叶球疏松,从羽状分裂到盾状,胚珠生于孢子叶两侧。
珠被两层,均有维管束。
小孢子叶球球果状,小孢子叶鳞状。
小孢子囊聚生于小孢子叶背面。
精子多鞭毛。
种子大,种皮厚。
外种皮肉质,中种皮骨质,内种皮膜质。
子叶两枚,胚乳丰富。
仅苏铁科(Cycadaceae),如苏铁(\textit{Cycas})。

\section{银杏纲(Ginkgopsida)}
落叶大乔木,多分枝。
叶扇状,顶端二裂,二叉脉序。
孢子叶球单性异株,精子多鞭毛。
种子核果状。
仅银杏(\textit{Ginkgo biloba})。

\section{松柏纲(Coniferae)}
木本,茎多分枝,有树脂道。
叶针状或鳞片状。
孢子叶排成球果状,单性,多同株。

\subsection{松科(Pinaceae)}
叶互生或簇生,针形或线形。
孢子叶球单性同株。
小孢子叶有两个小孢子囊。
小孢子多有气囊。
大孢子叶球的苞鳞和珠鳞分离,珠鳞发达,近轴面基部有两枚胚珠。
种子常有翅。

\subsubsection{油杉(\textit{Keteleeria})}
叶条形,扁平,中脉在叶面隆起。
球果直立,当年成熟,种鳞不脱落。
种子连翅与种鳞等长。

\subsubsection{冷杉(\textit{Abies})}
叶条形,扁平,中脉在叶面凹下。
枝上有圆形微凹的叶痕。
球果直立,当年成熟,种鳞脱落。

\subsubsection{铁杉(\textit{Tsuga})}
叶条形,扁平,单生,叶被有白色气孔带。
小枝有微隆起的叶枕。
球果下垂,当年成熟,种鳞不脱落。

\subsubsection{银杉(\textit{Cathaya})}
叶条形,扁平,中脉在叶面凹下,单生。
有长短枝。
球果腋生,初直立,后下垂。
苞鳞短,不露出,种鳞宿存。

\subsubsection{云杉(\textit{Picea})}
叶棱状条形,四面或仅叶面有气孔线。
小枝生隆起的叶枕。
球果下垂,苞鳞短于珠鳞,种鳞宿存。

\subsubsection{金钱松(\textit{Pseudolarix})}
落叶乔木。叶条状扁平,簇生。
小孢子叶球簇生。
苞鳞短于珠鳞。
种鳞木质,脱落。

\subsubsection{落叶松(\textit{Larix})}
叶条状扁平,簇生。
落叶。
小孢子叶球单生。
种鳞革质,宿存。

\subsubsection{黄杉(\textit{Pseudotsuga})}
常绿乔木。
叶扁平,叶基扭转成二列。
球果下垂,苞鳞长于珠鳞,先端三裂。

\subsubsection{松(\textit{Pinus})}
分布极广。
木材柔软,富含树脂。

\subsubsection{雪松(\textit{Cedrus})}
叶针状,硬,三棱或四棱,生于嫩枝上的单生或互生,生于老枝或短枝上的丛生。
球果直立,种鳞脱落。

\subsection{杉科(Taxodiaceae)}
叶两型,与小枝一起脱落。
小孢子囊与胚珠多于两个。
苞鳞小,与珠鳞合生。
珠鳞成盾状或覆瓦状排列,腹面生多个胚珠。
种子两侧有窄翅或下部有翅。

\subsubsection{水松(\textit{Glyptostrobus})}
三型叶,条状、针状稍晚或鳞状。
叶互生。
生条形叶的小枝冬季脱落,有鳞形叶的小枝不脱落。
种鳞木质,先端有裂齿。
种子下端有长翅。
仅水松(\textit{Glyptostrobus pensilis})。

\subsubsection{水杉(\textit{Metasequoia})}
叶条形,交互对生成两列,落叶。
种鳞盾形,木质,交互对生。
种子扁平,周围有翅。
如水杉(\textit{Metasequoia glyptostroboides})。

\subsubsection{杉(\textit{Cunninghamia})}
叶互生,条状披针形,有锯齿。
苞鳞大,种鳞小。
种子两侧有翅。

\subsection{柏科(Cupressaceae)}
叶对生或轮生,鳞状或刺形。
苞鳞与珠鳞合生。
种鳞盾形,木质或肉质,交互对生或轮生,鲜有螺旋状着生。
种子两侧有窄翅或无翅,或上部有一长一短的翅。

\subsubsection{侧柏(\textit{Thuja})}
叶鳞形,交互对生,小枝扁平。
孢子叶球单性同株,单生于短枝顶端。
球果木质,当年成熟。
种鳞四对,扁平,背部近顶端有反曲的箭头。
种子无翅,有棱脊。
仅侧柏(\textit{Thuja orientalis})。

\subsubsection{柏木(\textit{Cupressus})}
叶鳞形,交互对生,先端尖,小枝扁平下垂。
孢子叶球单性同株,单生于枝顶。
球果木质。
种鳞四对,盾形。
种子有窄翅。

\subsubsection{圆柏(\textit{Sabina})}
叶鳞形或刺形。
孢子叶球单性异株,单生于枝顶。
球果木质,种鳞愈合,种子无翅。

\subsection{南洋杉科(Araucariaceae)}
常绿乔木,有树脂。
大枝轮生。
叶螺旋状着生或交互对生,革质。
花单性异株。
花粉粒无气囊。
珠鳞舌状,不发达。
苞鳞发达。
种子无翅。
如南洋杉(\textit{Araucaria cunninghamia})。

\section{紫杉纲(Taxopsida)}
木本,多分枝。
叶为条形或条状披针形。
孢子叶球单性异株。
大孢子叶特化为鳞片状的珠托或瓮状套被。
种子有肉质假种皮或外种皮。

\subsection{罗汉松科(Podocarpaceae)}
常绿乔木或灌木。
管胞有单列具缘纹孔,木射线单列,有树脂细胞,无树脂道。
单叶互生,针状、鳞片状或阔长椭圆形。
孢子叶球单性异株。
小孢子叶球单生,或稀聚为柔荑花序状。
小孢子叶螺旋排列,小孢子囊两个。
花粉粒有气囊。
大孢子叶球生于叶腋或托苞片腋,在主轴上排列成各式球序。
大孢子叶变态为囊状套被,包围胚珠,或在胚珠基部缩小为杯状,有时完全与珠被合生。
种子成熟时,珠被分化为薄而石质的外层和厚而肉质的内层,套被变为革职假种皮;或珠被变为石质种皮,套被变为肉质假种皮。
托苞片与大孢子叶球轴愈合的种托。

\subsubsection{罗汉松(\textit{Podocarpus})}
大孢子叶球腋生,套被与珠被合生。
种子核果状,有肉质假种皮。

\subsubsection{陆均松(\textit{Dacrydium})}
叶异型,镰状针状或鳞形钻形。
大孢子叶生于小枝顶端,套被与珠被离生。
种子坚果状,卵圆锥形,横生,仅基部为肉质。

\subsection{三尖杉科(Cephalotaxaceae)}
仅三尖杉(\textit{Cephalotaxus})。
常绿小乔木或灌木,近对生或轮生枝条,鳞芽。
叶条形或披针状条形,交互对生或近对生,在侧枝基部扭转为二列。
管胞有单列纹孔和大型螺纹增厚。
单列射线,髓心有树脂道。
孢子叶球单性异株。
小孢子叶组成球状总序,花粉球形无气囊。
大孢子叶有三到四对交互对生的珠托或套被,生于小枝基部。
大孢子叶变态为囊状肉质套被,包裹种子。
种皮石质,内种皮膜质。

\subsection{红豆杉科(Taxaceae)}
常绿乔木或灌木,有鳞芽。
管胞有大型螺纹增厚。
单列射线。
叶披针形或条形,互生或对生。
叶中脉在叶面凹下,在叶背隆起。
叶背有两条气孔带。
孢子叶球单性异株。
小孢子叶球多单生。
小孢子叶辐射对称,花粉粒球形无气囊。
大孢子叶球多单生,基部有成对苞片,顶端有珠托。
种子核果状或坚果状,假种皮肉质。

\subsubsection{红豆杉(\textit{Taxus})}
叶螺旋排列,无树脂道。
气孔淡黄或淡绿。
孢子叶球单生。
假种皮红色杯状肉质。

\subsubsection{白豆杉(\textit{Pseudotaxus})}
小枝近对生或轮生,叶螺旋排列,无树脂道。
气孔有白粉。
孢子叶球单生。
假种皮白色杯状肉质。
仅白豆杉(\textit{Pseudotaxus chienii})。

\subsubsection{穗花杉(\textit{Amentotaxus})}
叶交互对生,有树脂道。
小孢子叶球聚生穗状。
大孢子叶球单生。
假种皮红色肉质囊状,种子顶端漏出。

\subsubsection{榧树(\textit{Torreys})}
叶交互对生,有树脂道。
小孢子叶球单生,大孢子叶球对生。
胚珠生于漏斗状珠托。
种子完全包裹于肉质假种皮。

\section{买麻藤纲(Gnetopsida)}
次生木质部有导管,无树脂道。
叶对生。
孢子叶球序二叉分枝,有盖被。
珠被延伸为珠孔管。
精子无鞭毛,有假种皮。

\subsection{麻黄科(\textit{Ephedraceae})}
仅麻黄(\textit{Ephedra})。
旱生植物,多分枝灌木,少数为藤本或小乔木。
小枝对生或轮生,绿色,节间多细纵纹。
叶对生或轮生。
止住顶端叶针状,底端叶鳞状。
孢子叶球单性异株,小孢子叶球中时有不孕胚珠。
胚珠外围增厚的囊状盖被。
珠被身长为珠孔管。

\subsection{买麻藤科(\textit{Gnetaceae})}
仅买麻藤(\textit{Gnetum})。
多为藤本。
枝有膨大关节,有多列圆形管胞和导管,有具缘纹孔。
阔叶对生,羽状网脉。
孢子叶球在轮状苞片内腋生。

\subsection{百岁兰科(Welwitschiaceae)}
仅百岁兰(\textit{Welwitschia bainesii})。
旱生,茎块状。
仅一对大型带状叶,平行叶脉,有斜向横脉。

\end{sloppypar}
\end{document}

