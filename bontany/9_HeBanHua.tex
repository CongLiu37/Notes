\documentclass[11pt]{article}

\usepackage[UTF8]{ctex} % for Chinese 

\usepackage{setspace}
\usepackage[colorlinks,linkcolor=blue,anchorcolor=red,citecolor=black]{hyperref}
\usepackage{lineno}
\usepackage{booktabs}
\usepackage{graphicx}
\usepackage{float}
\usepackage{floatrow}
\usepackage{subfigure}
\usepackage{caption}
\usepackage{subcaption}
\usepackage{geometry}
\usepackage{multirow}
\usepackage{longtable}
\usepackage{lscape}
\usepackage{booktabs}
\usepackage{natbibspacing}
\usepackage[toc,page]{appendix}
\usepackage{makecell}
\usepackage{amsfonts}
 \usepackage{amsmath}
\usepackage[utf8]{inputenc}
\usepackage{amssymb}
\usepackage{amsthm}
\usepackage{enumerate}
\usepackage{comment}

\usepackage[backend=bibtex,style=authoryear,sorting=nyt,maxnames=1]{biblatex}
\bibliography{} % Reference bib

\title{合瓣花亚纲(Sympetalae)}
\author{}
\date{}

\linespread{1.5}
\geometry{left=2cm,right=2cm,top=2cm,bottom=2cm}

\setlength\bibitemsep{0pt}

\begin{document}
\begin{sloppypar}
  \maketitle

  \linenumbers
花瓣连合成花冠管,雄蕊生于花冠管上。
珠被常一层。

\section{杜鹃花目(Ericales)}
木本,单叶,无托叶。
两性花,辐射对称或稍左右对称。
雄蕊与花瓣互生或为二倍。
花药有附属物,顶孔开裂。
子房上位或下位,中轴胎座,胚珠多,有胚乳。

\subsection{杜鹃花科(Ericaceae)}
木本。
单叶互生全缘或锯齿,背面有深槽或闭合腔,上表皮厚角质化,有贮水组织,被毛或鳞,无托叶。
花两性。
花萼宿存,四或五裂。
花瓣四或五,合生。
雄蕊为花瓣两倍,分为两轮,外轮与瓣对生;或雄蕊与花瓣同数互生。
花药二室,有附属物,顶孔开裂。
子房上位或下位,多室,中轴胎座,胚珠多,珠被一,花柱单生。
蒴果、浆果或核果,有胚乳。
  
\subsubsection{杜鹃花(\textit{Rhododendron})}
灌木,常绿或落叶。
叶全缘,常聚生枝顶。
顶生、伞形花序式的总状花序。
萼瓣五裂。
雄蕊五或十或多,花药无芒,顶孔开裂,有具齿缺的花盘。
子房五至二十室。
蒴果室间开裂。

\subsubsection{吊钟花(\textit{Enkianthus})}
落叶灌木,叶全缘或有翅。
花药有芒,芒直立或上升。
蒴果三至五角或有翅,室背开裂。

\subsubsection{越橘(\textit{Vaccinium})}
灌木。
雄蕊内藏,不抱花柱。
子房下位,浆果。

\section{报春花目(Primulales)}
木本或草本,叶有腺点。
两性花,辐射对称,合瓣。
雄蕊与花冠裂片对生。
子房上位或半下位,一室,胚珠多,珠被二,特立中央胎座。

\subsection{报春花科(Primulaceae)}
草本,有腺点。
单叶,无托叶。
两性花,辐射对称,有苞片,总状或伞形花序。
花萼五裂,宿存。
花冠合生五裂。
雄蕊与花冠裂片同数对生,生于花冠管。
子房上位,一室,特立中央胎座,胚珠多。
蒴果。
种子平滑或有棱角,胚乳多。

\subsubsection{报春花(\textit{Primuka})}
叶基生。
花冠裂片在花蕾中覆瓦或镊合排列。
花冠管长与花冠裂片。
伞形花序,有苞片。

\subsubsection{排草(\textit{Lysimachia})}
花冠裂片螺旋排列,蒴果瓣裂。

\section{柿树目(Diospyrales)}
木本。
单叶互生,无托叶。
花辐射对称,雄蕊与花瓣同数或二倍。
子房上位或下位,中轴胎座,胚珠一至多,珠被一至二。

\subsection{山榄科(Sapotaceae)}
灌木或乔木,有乳汁。
单叶互生,全缘,常无托叶。
两性花,辐射对称,单生或簇生于叶腋。
萼四至八裂,一或二轮。
花冠管短,花冠裂片与萼片同数或二倍,常有全缘或裂片状附属体。
雄蕊与花冠裂片同数对生。
心皮一轮,与雄蕊同数或二倍。
胚珠一,浆果。

\subsubsection{紫荆木(\textit{Madhuca})}
萼片四,雄蕊十六以上,无退化雄蕊。

\subsubsection{金叶树(\textit{Chrysophyllum})}
萼片五,花冠裂片无附属物。
雄蕊五至十,无退化雄蕊。

\subsubsection{铁榄(\textit{Sinosiideroxylon})}
叶互生,无托叶,萼片五。
花冠裂片无附属物。
有退化雄蕊。
花丝基部无毛。
子房五室。
种子疤痕基生。

\subsection{柿树科(Diospyraceae)}
木本,木材黑褐色,称为乌木。
单叶互生无托叶。
花单性,雌雄异株,单生或伞形花序。
花萼四裂,果实成熟时增大宿存。
花冠钟状或壶状,四或五裂,旋转排列。
雄蕊常十六,分离或合生成束。
子房上位,二至十六室,胚珠一或二。
浆果,胚乳硬质。
如柿(\textit{Diospyros kaki})。

\subsection{野茉莉科(Styracaceae)}
木本,单叶互生,无托叶。
嫩枝和叶被星状毛。
两性花,辐射对称,腋生或顶生总状花序。
萼钟状或管状,四或五裂。
花冠四或五裂,或八裂。
雄蕊为花冠裂片二倍。
花丝基部合生。
子房上位到下位,基部三至五室,上部一室。
胚珠一至多。
浆果或核果,或干燥开裂为三瓣,或有翅。

\subsubsection{鸦头梨(\textit{Melliodendron})}
冬芽有鳞苞,先花后叶,花单生或双生。
花冠五裂,子房半下位。
果硬木质,有肋,无翅。

\subsubsection{木瓜红(\textit{Rehderodendron})}
花多,总状或圆锥花序。

\subsubsection{赤叶杨(\textit{Alniphyllum})}
冬芽无鳞苞,先叶后花。
花丝下半部合生。
果梗弯曲不明显,有关节,蒴果,室背开裂,种子有翅。

\subsubsection{野茉莉(\textit{Styrax})}
子房稍半下。
果与萼筒分离,不规则三瓣开裂。
种子种脐大。

\subsection{山矾科(Symplocaceae)}
仅山矾(\textit{Symplocos})。
木本,单叶互生,无托叶。
两性花,辐射对称。
花萼五裂,裂片镊合或覆瓦排列,常宿存。
花冠裂至基部或中部,裂片五,覆瓦排列。
雄蕊多至四,生于花冠。
子房下位或半下位,一至五室,垂生胚珠二至四,花柱纤细。
浆果或核果,顶部有宿存花萼,种子胚乳多。
如白檀(\textit{Symplocos paniculata})。

\section{木犀目(Oleales)}
仅木犀科(Oleaceae)。
如油橄榄(\textit{Olea europaea}),茉莉花(\textit{Jasminum sambac}),女贞(\textit{Ligustrum lucidum}),连翘(\textit{Forsythia suspensa})。

\subsection{木犀科(Oleaceae)}
直立木本或藤本。
单叶、三出复叶或羽状复叶,对生,无托叶。
两性或单性花,圆锥、聚生或丛生花序。
萼四裂,花冠四至九裂。
雄蕊二,花药二室,纵裂。
子房上位,二心皮,二室,中轴胎座,胚珠一至三,珠被一。
花柱单生,柱头二尖裂。
核果、蒴果、浆果或翅果。
有胚乳。

\subsubsection{梣(\textit{Fraxinus})}
复叶,花序间或有叶状苞片。
翅果,翅在果实顶端伸长。
如白蜡树(\textit{Fraxinus chinensis})。

\subsubsection{丁香(\textit{Syringa})}
花多紫或红。
花冠裂片比花冠筒短。
蒴果,种子有翅。

\subsubsection{木犀(\textit{Osmanthus})}
花芳香,簇生或短圆锥花序。
花冠裂片在芽中覆瓦排列。
核果。
如桂花(\textit{Osmanthus fragrans})。

\section{龙胆目(gentianales)}
木本或草本、藤本,叶对生。
两性花,辐射对称,花萼四或五。
花冠管状,裂片四至五。
雄蕊与花冠裂片同数。
心皮二,子房上位到下位。
中轴胎座。

\subsection{马钱科(Loganiaceae)}
草本、灌木或乔木,或攀援。
叶对生或轮生,单叶,托叶退化。
两性花,有花序。
萼、瓣、雄蕊均四至五。
子房上位,二室,胚珠二,花柱单生二裂。
蒴果、浆果或核果。
如断肠草(\textit{Gelsemium elegans})。

\subsubsection{马钱(\textit{Strychnos})}
乔木、灌木或附生。
枝有时变为钩刺。
叶脉三至五出,或离基三至五出。
种子含马钱碱,剧毒。

\subsection{夹竹桃科(Apocynaceae)}
木本、藤本或草本,有乳汁或水液。
单叶,对生或轮生,全缘,无托叶,叶柄基部有腺体或腺鳞。
两性花,辐射对称,单生或聚伞、圆锥花序。
花萼五裂,合生成钟状或筒状,覆瓦排列,基部有腺体。
花冠裂片五,旋转状覆瓦排列,喉部有毛或鳞。
雄蕊五,花药二室纵裂。
有花盘。
子房上位二室,花柱结合。
胚珠一或二。
浆果、核果、蒴果或蓇葖果,种子有长丝毛或翅。
有胚乳。
如夹竹桃(\textit{Nerium indicum}),萝芙木(\textit{Rauvolfia})。

\subsection{萝摩科(Asclepiadaceae)}
草本、藤本或灌木,有乳汁,常有毒。
单叶对生或轮生,无托叶,叶柄顶端常有腺体或鳞腺。
两性花,辐射对称,聚伞花序。
花萼筒短,裂片五,覆瓦或镊合排列,内面基本常有腺体。
花冠裂片五,旋转状、覆瓦状或镊合状排列。
副花冠由五个分离或基部合生短裂片或鳞片构成,连生于花冠筒上。
雄蕊五。
花药合生成管,包围雌蕊。
花药与柱头黏合成中心柱。
花粉结为花粉块或四合花粉,位于花粉器上。
子房上位,二离生心皮。
花柱二,合生,柱头基本有五棱。
胚珠多。
蓇葖果,有种毛,胚乳多。
如夜来香(\textit{Telosma cordata})。

\subsubsection{白叶藤(\textit{Cryptolepis})}
木质藤本。
花蕾端部长圆形,顶端尾部渐尖。
副花冠与花丝生于花冠筒内面中部以上,与花丝离生。
副花冠裂片卵形,顶端钝,四合花粉。

\subsubsection{杠柳(\textit{Periploca})}
副花冠与花丝着生于花冠基部,花丝筒状,副花冠裂片异形,四合花粉。

\subsubsection{匙羹藤(\textit{Gymnema})}
茎缠绕。
副花冠有五个硬肉质条带或两列纵毛,位于花冠喉部。

\subsection{茜草科(Rubiacece)}
木本、草本或藤本。
单叶,对生或轮生,全缘。
托叶二,分离或合生成鞘状,宿存。
两性花,辐射对称。
花萼筒与子房连生,或有一枚萼齿增大成叶状。
花冠裂片四至六,雄蕊与花冠裂片同数互生。
子房下位,常二室,胚珠多至一。
蒴果、核果或浆果。
种子有胚乳,或有翅。
如栀子(\textit{Gardenia jasminoides}),金鸡纳树(\textit{Cinchona ledgerian})。

\subsubsection{水团花(\textit{Adina})}
乔木或灌木。
球形头状花序。
花有小苞片,萼檐五裂。
蒴果,中轴宿存,顶部有星状萼檐裂片。

\subsubsection{钩藤(\textit{Uncaria})}
藤本。
不育花序梗成钩状,司攀登。
头状花序,无小苞片。

\subsubsection{玉叶金花(\textit{Mussaenda})}
直立或攀援灌木。
萼檐裂片或有一枚扩大成有柄叶状体。
花冠黄色,雄蕊五,浆果。

\subsubsection{山黄皮(\textit{Randia})}
有刺或无刺,灌木或乔木。
花单生或聚伞花序,多腋生。

\subsubsection{巴戟天(\textit{Morinda})}
木本,直立或攀援状。
头状花序,聚花果。

\subsubsection{咖啡(\textit{Coffea})}
灌木或小乔木,花冠高脚碟状,浆果,种皮角质。
种子含咖啡碱。

\section{管花目(Tubiflorae)}
两性花。
萼片五,花瓣五,合生。
雄蕊四或五,心皮二,子房上位。

\subsection{旋花科(Convolvulaceae)}
多藤本,或有乳汁。
单叶互生,无托叶。
两性花,辐射对称,单生或聚伞花序,有苞片。
萼片五,分离,覆瓦排列,宿存。
花冠五浅裂,旋转排列。
雄蕊五,生于花冠基部,与花冠裂片互生。
子房上位,常为环状分裂花盘包围,一至四室。
胚珠一或二,花柱顶生,柱头二。
蒴果或浆果,种子二至四,胚乳少。
如菟丝子(\textit{Cuscuta chinensis})。

\subsubsection{打碗花(\textit{Calystegia})}
萼片近相同。
花萼包藏于两片大苞片内。
柱头二,长圆,扁平。

\subsubsection{鱼黄草(\textit{Merremia})}
花冠黄色,花瓣中带有五条暗色的脉。
花粉粒无刺,蒴果开裂。

\subsubsection{番薯(\textit{Ipomoea})}
花冠白、红或紫。
花瓣中有两条脉,花粉粒有刺,子房二至四室。

\subsubsection{牵牛(\textit{Pharbitis})}
萼片顶端长而渐狭尖,子房三室,胚珠六。

\subsection{紫草科(Boraginaceae)}
木本或草本,常被毛。
单叶互生,无托叶。
两性花,辐射对称,单歧或二歧聚伞花序。
萼片五,离生或基部合生,覆瓦排列。
花冠裂片五,覆瓦排列,喉部有附属体。
雄蕊与花冠裂片对生同数。
子房上位, 二室,胚珠二,柱头二至四。
核果或坚果,无胚乳。
如紫草(\textit{Lithospermum erythrorhizon}),厚壳树(\textit{Ehretia thyrsiflora}),勿忘草(\textit{Myosotis silvatica})。

\subsection{马鞭草科(Verbenaceae)}
木本或草本,叶对生,无托叶。
两性花,两侧对称,穗状或聚伞花序,或再由聚伞花序排成圆锥状、头状或伞房状。
花萼合生,四或五裂,宿存。
花冠四至五裂,覆瓦排列。
雄蕊四,着生于花冠管上。
花药二室,分叉状,纵裂。
子房上位,二心皮,四室,胚珠一或二。
花柱顶生,柱头二裂或不裂。
核果或蒴果,无胚乳。

\subsubsection{马鞭草(\textit{Verbena})}
方茎,叶不规则分裂,穗状花序,花淡蓝紫色,中轴胎座,子房四室。

\subsubsection{大青(\textit{Clerodendrum})}
聚伞或圆锥花序。
结果时花萼增大。
花冠筒直。
雄蕊四。
如鬼灯笼(\textit{Clerodendrum fortunatum}),大青(\textit{Clerodendrum crytophyllum})。

\subsubsection{牡荆(\textit{Vitex})}
乔木或灌木,掌状复叶。
结果时花萼不增大。
花萼绿色。
花冠二唇形,下唇中央裂片大。
如黄荆(\textit{Vitex negundo})。

\subsection{唇形科(Labiatae)}
草本或灌木,常含挥发性芳香油,茎四棱形。
叶对生或轮生,有腺点,有香气。
两性花,两侧对称,轮伞花序。
花萼合生,唇形,五齿裂,宿存。
花冠合瓣有毛环,二唇形,上唇两裂,下唇三裂。
雄蕊四或二,分叉状花药二室,有花盘。
子房上位,四室,二心皮深裂,一胚珠。
花柱生于子房裂隙基部,柱头二尖裂,胚珠倒生。
果由四个小坚果构成。
胚乳少。
如夏枯草(\textit{Pogostemon vulgaris}),薄荷(\textit{Mentha haplocalyx}),荆芥(\textit{Schizonepeta tenuifolia})。

\subsubsection{黄芩(\textit{Scutellaria})}
草本或亚灌木,花对生。
萼钟状,花后封闭。
花冠管长而突出,上唇兜状且背部有一大鳞片。

\subsubsection{藿香(\textit{Agastache})}
叶不分裂。
花萼筒内部无毛环,萼十五脉。
花冠上唇外凸,后一对雄蕊下垂,前一对雄蕊上升。
花冠裂片相等,不明显。
花冠下唇中裂片无爪状狭柄。

\subsubsection{益母草(\textit{Leonurus})}
花萼漏斗状,五脉,前二裂片靠合反折,成尖三角形。
花冠筒内有微柔毛或毛环,其上伸直或囊状膨大。
花冠上唇微外凸而基部狭窄,下唇直伸或平展。

\subsubsection{鼠尾草(\textit{Salvia})}
花萼喉部毛少或无。
雄蕊二,花药条形,药隔线形。
如丹参(\textit{Salvia splendens})。

\subsection{茄科(Solanaceae)}
草本或灌木,直立或攀援,双韧维管束。
单叶互生,或为一大一小双生叶,无托叶。
两性花,辐射对称,单生或聚伞花序。
花萼四至六裂,宿存,果实成熟时增大。
花冠常辐射状五裂。
雄蕊五,与花冠裂片互生,药两室,纵裂或孔裂。
有花盘。
子房上位,二室,胚珠多。
浆果或蒴果,有胚乳。
如茄(\textit{Solanum melongena}),马铃薯(\textit{Solanum tuberosum}),番茄(\textit{Lycopersicum esculentum}),辣椒(\textit{Capsicum frutescens}),烟草(\textit{Nicotiana tabacum}),曼陀罗(\textit{Datura stramonium})。

\subsection{玄参科(Scrophulariaceae)}
草本或木本。
叶多对生,无托叶。
两性花,两侧对称。
萼片四或五,分离或合生,宿存。
花冠合瓣,二唇形,裂片四或五,花蕾时覆瓦排列。
雄蕊四,二强,与花冠裂片互生。
花盘环状或一侧退化。
子房上位,二室,胚珠多,中轴胎座,花柱顶生。
蒴果,有宿存花柱,种子多,有胚乳。
如地黄(\textit{Rehmannia glutinosa})。

\subsubsection{泡桐(\textit{Paulownia})}
落叶乔木,叶对生。
花大,顶生圆锥花序,上唇二裂反卷。
蒴果室背开裂。

\subsubsection{毛麝香(\textit{Adenosma})}
草本,有香气,基部或木质化。
叶背有腺点,小苞片二,花萼裂至基部。

\subsubsection{玄参(\textit{Scrophularia})}
草本,叶对生,有透明腺点。
花冠管球形或卵形,蒴果。

\subsubsection{母草(\textit{Lindernia})}
花萼有五棱,无唇形,不规则分裂。
蒴果隔膜宿存。

\subsection{爵床科(Acanthaceae)}
草本或灌木。
单叶对生,全缘或分裂,无托叶。
两性花,多左右对称,罕单生。
花常有苞片和小苞片。
萼五深裂,花冠二唇形或五裂。
雄蕊四或二,生于花冠管上。
花药二或一室。
子房上位,二室,胚珠一至多。
蒴果,种子有上弯的种钩。

\subsubsection{山牵牛(\textit{Thunbergia})}
藤本,两小苞片似佛焰苞状。
花萼退化,仅存一边环或小齿。
蒴果胎座无种钩。
如大花老鸦嘴(\textit{Thunbergia grandiflora})。

\subsubsection{老鼠簕(\textit{Acanthus})}
直立灌木。
叶柄两侧各有一刺。
叶缘有深波状带刺的齿。
花冠淡唇形,上唇退化。
如老鼠簕(\textit{Acanthus ilicifolius}),为红树植物。

\subsection{苦苣苔科(Gesneriaceae)}
草本或灌木。
单叶基生或对生,或不等大。
两性花,左右对称,单生或各式聚伞花序。
花萼管状,五裂。
花冠合瓣,上部偏斜,多五裂。
雄蕊生于花冠管上,常四枚,二强,或有两枚退化。
子房上位或下位,侧膜胎座,胚珠多。
蒴果,果瓣旋卷。

\subsubsection{芒毛苣苔(\textit{Aeschynanthus})}
木本,多附生。
叶对生或轮生,革质或肉质。
花鲜红,单生或簇生。
花瓣二唇形,雄蕊四,花盘环状。
蒴果线形,种子有长毛。

\subsubsection{马铃苣苔(\textit{Oreocharis})}
草本,近乎无茎。
叶基生,有网脉。
花茎长,花萼、花冠均五裂。
花蓝或紫。
雄蕊四,花药分生,药隔无硬毛。
花盘环形,全缘或浅裂。
蒴果室裂,种子无毛。

\subsubsection{唇柱苣苔(\textit{Chirita})}
花冠两侧对称,能育雄蕊二。
柱头斜,不等二裂。

\section{川续断目(Dipsacales)}
草本或木本。
叶对生或轮生。
两性花,子房下位或半下位,心皮二至三,一至多室,胚珠一至多。

\subsection{忍冬科(Caprifoloaceae)}
木本,稀为草本。
叶对生,单叶,稀为羽状复叶,无托叶。
两性花,聚伞花序或簇生。
花萼筒与子房贴生,四或五裂。
花冠筒裂片四或五,覆瓦排列。
雄蕊四或五,生于花冠管,与花冠裂片互生。
无花盘,子房下位,二至五室,胚珠一至多。
浆果、蒴果或核果。
有胚乳。

\subsubsection{忍冬(\textit{Lonicera})}
藤本或灌木。
花冠二唇形或五等裂,浆果。
如忍冬(金银花)(\textit{Lonicera japonica})。

\subsubsection{荚蒾(\textit{Viburnum})}
灌木。
花冠辐状,核果。

\subsubsection{接骨木(\textit{Sambucus})}
奇数羽状复叶对生,或有托叶与小托叶。

\section{桔梗目(Campanulales)}
两性花,整齐或两侧对称。
聚药雄蕊,子房下位,胚珠多至一。

\subsection{菊科(Compositae)}
草本、灌木或藤本。
叶互生,无托叶。
花两性或单性,头状花序,有总苞片构成的总苞。
花序托凸、扁或圆柱状,平滑或有窝孔,裸露或被托片。
头状花序单生或排成花序。
头状花序内的花可同型(管状花或舌状花),可异型(外围舌状花,中央管状花),可多型。
萼片成冠毛状、刺状或鳞片状。
花冠合瓣成管状、舌状、二唇形、假舌状、漏斗状,四或五裂。
管状花为辐射对称的两性花;
二唇红为两侧对称的两性花,上唇二裂,下唇三裂;
舌状花为两侧对称的两性花,五个裂瓣结成舌状瓣片;
假舌状花为两侧对称的雌花或中性花,三个裂瓣结成舌状片;
漏斗状花无性。
雄蕊四至五,花药从侧面合生成筒状,基部钝或有尾。
花药内向纵裂,花丝分离。
子房下位,一室,一胚珠。
花柱顶端二裂成长段不一的柱头壁,柱头顶端有附器,柱头内表面有乳突,柱头下或有毛环。
下位瘦果,顶端有附属物。
种子无胚乳。

\subsubsection{筒状花亚科(Carduoideae)}
无乳汁。
头状花序中央的盘花非舌状。

\par

斑鸠菊(\textit{Vernonia})头状花序分散,全为两性筒状花。
冠毛多,宿存,外冠管毛或成膜片状。
果五棱或十棱。

\par

胜红蓟(\textit{Ageratum})叶对生,同型筒状花,冠毛鳞片状。

\par

泽兰(\textit{Eupatorium})冠毛毛状,多,分离。

\par

紫菀(\textit{Aster})有舌状雌花,白、紫或蓝色。
花盘中央的盘花两性管状,黄色。
冠毛一或二层,外层短膜片状。

\par

艾纳香(\textit{Blumea})管状花,花药基部有尾,冠毛细毛状。

\par

苍耳(\textit{Xanthium})头状花序单性,同型花。
雌雄同株,雌花无花冠。
花序托在两性花间有毛状托片。
雄花序总状或穗状排列,总苞片一层,分离。
雌花序无柄,内层总苞片结合成蒴果状,有喙和钩刺。

\par

豨莶(\textit{Siegesbeckia})叶对生,头状花序稀疏圆锥状排列,有花序梗。
总苞片二列,外裂的线装匙形扩展,被腺毛;内列卵形或矩圆形,半包果实。
花序托有托片,瘦果无冠毛。

\par

蟛蜞菊(\textit{Wedelia})叶对生。
头状花序有异型花,黄色。

\par

鬼针草(\textit{Bidens})叶对生,分裂或一至二回羽状复叶。
头状花序单生或成束。
总苞片一列,舌状花一列,黄色或白色,无中性花。
花盘中央的盘花黄色,两性。
瘦果顶硬刺二至四条,刺上有倒毛刺。

\par

向日葵(\textit{Helianthus})草本,头状花序单生或伞房状,顶生。
总苞片多轮,外轮叶状。
花盘边缘的缘花假舌状,中性;中央的盘花筒状,两性。
果顶端有两个鳞片状脱落的芒。

\par

菊(\textit{Dendranthema})叶互生,全缘、齿牙状或分裂。
头状花序大,总苞多列,异型花,花序托无托片,舌状花一列。
瘦果有棱或翅,顶部或有鳞片状的杯。

\par

蒿(\textit{Artemisia})草本或亚灌木,有苦味或芳香,多被柔毛或蛛丝状毛。
叶不分裂或有缺刻,或一至三回羽状全裂。
头状花序小,集成总状或圆锥状,全部筒状花。
花盘中央的盘花两性,边缘的花雌性。
总苞片边缘膜质,多列。
花药顶端附片钝或钻状。
果无冠毛,有微棱。

\subsubsection{舌状花亚科}
有乳汁。
头状花序有同型舌状花。
花柱分枝细长条形,无附器。
叶互生。

\par

莴苣(\textit{Lactuca})一或多年生草本,叶全缘或羽裂。
花黄色,总苞片多列。
花序托扁平,秃裸。
果扁平,有喙,有棱。
冠毛多,白色,软,基部有一短毛环。
如莴苣(\textit{Lactuca sativa}),变种有莴笋(\textit{Lactuca sativa} var. \textit{angustata}),卷心莴苣(\textit{Lactuca sativa} var. \textit{capitata}),生菜(\textit{Lactuca sativa} var. \textit{romana}),玻璃生菜(\textit{Lactuca sativa} var. \textit{crispa})。

\par

黄鹌菜(\textit{Crepis})一年生草本。
叶基生,全缘、齿缺或羽状深裂。
头状花序小,花黄。
总苞圆筒状,外侧苞片小,内侧苞片线形。
果小,有线纹,两端渐狭,无喙,冠毛白色柔软。

\par

蒲公英(\textit{taraxacum})多年生草本,叶茎生。
头状花序单生于无叶花茎上,黄或白色。
总苞钟状或矩圆状,苞片草质。
花序托扁平,秃裸。
果柱状,四或五棱,棱上有突点或小刺,喙细长。
冠毛多,白色。

% 蓇葖果
\end{sloppypar}
\end{document}
