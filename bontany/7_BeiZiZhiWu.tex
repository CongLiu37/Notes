\documentclass[11pt]{article}

\usepackage[UTF8]{ctex} % for Chinese 

\usepackage{setspace}
\usepackage[colorlinks,linkcolor=blue,anchorcolor=red,citecolor=black]{hyperref}
\usepackage{lineno}
\usepackage{booktabs}
\usepackage{graphicx}
\usepackage{float}
\usepackage{floatrow}
\usepackage{subfigure}
\usepackage{caption}
\usepackage{subcaption}
\usepackage{geometry}
\usepackage{multirow}
\usepackage{longtable}
\usepackage{lscape}
\usepackage{booktabs}
\usepackage{natbibspacing}
\usepackage[toc,page]{appendix}
\usepackage{makecell}
\usepackage{amsfonts}
 \usepackage{amsmath}
\usepackage[utf8]{inputenc}
\usepackage{amssymb}
\usepackage{amsthm}
\usepackage{enumerate}
\usepackage{comment}

\usepackage[backend=bibtex,style=authoryear,sorting=nyt,maxnames=1]{biblatex}
\bibliography{} % Reference bib

\title{被子植物(Angiospermae)}
\author{}
\date{}

\linespread{1.5}
\geometry{left=2cm,right=2cm,top=2cm,bottom=2cm}

\setlength\bibitemsep{0pt}

\begin{document}
\begin{sloppypar}
  \maketitle

  \linenumbers
被子植物的胚珠有心皮(carpel)包裹,形成子房(ovary),进而发育为果实。

\par

被子植物有真正的花。
花被的出现加强了保护作用,提高了传粉效率。
雄蕊由花丝和花药构成。
雌蕊由子房、花柱、柱头组成。
组成雌蕊的单位为心皮。
多数被子植物心皮完全愈合。

\par

被子植物开花后,经传粉受精,胚珠发育为种子,子房发育为果实。
有时花萼、花托、花序轴也一并发育为果实。
果实在种子成熟前起保护作用,亦有助于种子传播。

\par

被子植物有特殊的双受精作用,产生三倍体胚乳。
胚乳在受精后才能发育,较裸子植物在受精前由大孢子发育为胚乳更经济。

\par

被子植物孢子体高度分化。
木质部由导管、纤维和薄壁组织构成。
导管和纤维均由管胞发育而来,分司输水和机械支撑。
韧皮部有筛管和伴胞。
完善的输导组织提高了运输效率,进而适应更大的叶面积和更强的光合作用,进而产生大量花、果实、种子。

\par

被子植物配子体进一步简化。
雄配子体仅一个粉管细胞、两个精子。
大部分被子植物花粉散布时仅一个粉管细胞和一个精子。
花粉在柱头上萌发为雄配子体,精子分裂一次。
部分被子植物花粉有一个粉管细胞、两个精子。
雌配子体仅一个卵、两个助细胞、两个极核和三个反足细胞。

\par

被子植物适应性强,营养方式多样,适应不同的生态环境,传粉方式多样化。

\section{双子叶植物纲(Dicotyledoneae)}
花常为四或五基数,花粉常三沟孔。
种子一般两枚子叶。
主根发达。
茎内维管束排成圆柱状,有形成层。
叶常有网脉,无叶鞘。

\section{单子叶植物纲(Monocotyledoneae)}
花常为三基数,花粉常单孔或散孔。
种子一般一枚子叶。
多有须根。
茎内维管束散生,无形成层。
叶有平行脉或弧形脉,有叶鞘。

\end{sloppypar}
\end{document}
