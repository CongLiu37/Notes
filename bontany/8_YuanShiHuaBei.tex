\documentclass[11pt]{article}

\usepackage[UTF8]{ctex} % for Chinese 

\usepackage{setspace}
\usepackage[colorlinks,linkcolor=blue,anchorcolor=red,citecolor=black]{hyperref}
\usepackage{lineno}
\usepackage{booktabs}
\usepackage{graphicx}
\usepackage{float}
\usepackage{floatrow}
\usepackage{subfigure}
\usepackage{caption}
\usepackage{subcaption}
\usepackage{geometry}
\usepackage{multirow}
\usepackage{longtable}
\usepackage{lscape}
\usepackage{booktabs}
\usepackage{natbibspacing}
\usepackage[toc,page]{appendix}
\usepackage{makecell}
\usepackage{amsfonts}
 \usepackage{amsmath}
\usepackage[utf8]{inputenc}
\usepackage{amssymb}
\usepackage{amsthm}
\usepackage{enumerate}
\usepackage{comment}

\usepackage[backend=bibtex,style=authoryear,sorting=nyt,maxnames=1]{biblatex}
\bibliography{} % Reference bib

\title{双子叶植物-原始花被亚纲(Archichlamydeae)}
\author{}
\date{}

\linespread{1.5}
\geometry{left=2cm,right=2cm,top=2cm,bottom=2cm}

\setlength\bibitemsep{0pt}

\begin{document}
\begin{sloppypar}
  \maketitle

  \linenumbers
原始花被亚纲(Archichlamydeae)花瓣分离。
雄蕊生在花托上,珠被两层,种子有胚乳。

\section{木麻黄目(Casuarinales)}
乔木或灌木。
小枝轮生或假轮生,有节,纤细。
叶退化为鳞状,环状轮生成鞘。
花单性。
雄花轮生于花序轴,成柔荑花序。
雄花苞片二,杯状合生,花被片一或二,雄蕊一。
雌花序头状,苞片一,无花被,心皮二枚合生,花柱短,柱头二,子房二(后为单室),胚珠二。
扁平小坚果,有翅。
种子单生,种皮膜质。
仅木麻黄科(Casuarinaceae)。

\section{胡桃目(Juglandales)}
芳香乔木,多有树脂。
羽状复叶,互生。
雄花单花被。
雌花三花被,子房下位,一或二室,胚珠一个,无胚乳。

\subsection{胡桃科(Juglandaceae)}
落叶乔木,有树脂。
羽状复叶,互生,无托叶。
花单性,雌雄同株。
雄花排成下垂的柔荑花序,花被与苞片相连,三至六裂,雄蕊一般三个。
雌花为穗状花序,直立无柄,小苞片一或二枚,花被与子房连生,三至五裂,子房下位一室或不完全二至四室,羽状花柱二,基生胚珠一。
核果或具翅坚果。
种子单生,无胚乳。
子叶皱褶含油。

\subsubsection{胡桃(\textit{Juglans})}
核果。
外果皮肉质,干后纤维质,不规则开裂。
内果皮有雕纹。
不完全二至四室。

\subsubsection{山核桃(\textit{Carya})}
核果。
外果皮木质四裂。
内果皮平滑,纵棱,四裂。

\subsubsection{枫杨(\textit{Pterocarya})}
总状果序,坚果有两翅。

\subsection{杨梅科(Myricaceae)}
单叶,有芳香腺体。
花单性,风媒,无花被。
肉质核果。
如杨梅(\textit{Myrica rubra})。

\section{杨柳目(Salicales)}
仅杨柳科(Salicaceae)。
木本,单叶互生,有托叶。
单性花,雌雄异株,柔荑花序,每一苞片内一花,无小苞片,无花被,一杯状花盘或二腺状鳞片。
雄蕊二至多个。
子房一室,花柱一或二至四,二心皮,侧膜胎座,胚珠多,倒生。
蒴果,两瓣开裂。
种子细小,由珠柄生出柔毛。
胚直生,外包一层参与内胚乳。

\subsection{杨(\textit{Populus})}
叶宽阔,苞片条裂,杯状花盘,雄蕊多个,花序下垂。

\subsection{柳(\textit{Salix})}
叶窄,苞片全缘,无花盘,有分泌蜜汁多腺体。
雄蕊二,雄花序直立。

\section{壳斗目(Fagales)}
木本。
单叶互生,有托叶。
单性花,雌雄同株,单被。
雄花组成柔荑花序,雄蕊多数至二。
雌花单生,子房下位,一至六室。
坚果。

\subsection{壳斗科(Fagaceae)}


  
\end{sloppypar}
\end{document}
