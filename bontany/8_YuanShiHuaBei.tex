\documentclass[11pt]{article}

\usepackage[UTF8]{ctex} % for Chinese 

\usepackage{setspace}
\usepackage[colorlinks,linkcolor=blue,anchorcolor=red,citecolor=black]{hyperref}
\usepackage{lineno}
\usepackage{booktabs}
\usepackage{graphicx}
\usepackage{float}
\usepackage{floatrow}
\usepackage{subfigure}
\usepackage{caption}
\usepackage{subcaption}
\usepackage{geometry}
\usepackage{multirow}
\usepackage{longtable}
\usepackage{lscape}
\usepackage{booktabs}
\usepackage{natbibspacing}
\usepackage[toc,page]{appendix}
\usepackage{makecell}
\usepackage{amsfonts}
 \usepackage{amsmath}
\usepackage[utf8]{inputenc}
\usepackage{amssymb}
\usepackage{amsthm}
\usepackage{enumerate}
\usepackage{comment}

\usepackage[backend=bibtex,style=authoryear,sorting=nyt,maxnames=1]{biblatex}
\bibliography{} % Reference bib

\title{双子叶植物-原始花被亚纲(Archichlamydeae)}
\date{}

\linespread{1.5}
\geometry{left=2cm,right=2cm,top=2cm,bottom=2cm}

\setlength\bibitemsep{0pt}

\begin{document}
\begin{sloppypar}
  \maketitle

  \linenumbers
原始花被亚纲(Archichlamydeae)花瓣分离。
雄蕊生在花托上,珠被两层,种子有胚乳。

\section{木麻黄目(Casuarinales)}
乔木或灌木。
小枝轮生或假轮生,有节,纤细。
叶退化为鳞状,环状轮生成鞘。
花单性。
雄花轮生于花序轴,成柔荑花序。
雄花苞片二,杯状合生,花被片一或二,雄蕊一。
雌花序头状,苞片一,无花被,心皮二枚合生,花柱短,柱头二,子房二(后为单室),胚珠二。
扁平小坚果,有翅。
种子单生,种皮膜质。
仅木麻黄科(Casuarinaceae)。
如木麻黄(\textit{Casuarina equisetifolia})。

\section{胡桃目(Juglandales)}
芳香乔木,多有树脂。
羽状复叶,互生。
雄花单花被。
雌花三花被,子房下位,一或二室,胚珠一个,无胚乳。

\subsection{胡桃科(Juglandaceae)}
落叶乔木,有树脂。
羽状复叶,互生,无托叶。
花单性,雌雄同株。
雄花排成下垂的柔荑花序,花被与苞片相连,三至六裂,雄蕊一般三个。
雌花为穗状花序,直立无柄,小苞片一或二枚,花被与子房连生,三至五裂,子房下位一室或不完全二至四室,羽状花柱二,基生胚珠一。
核果或具翅坚果。
种子单生,无胚乳。
子叶皱褶含油。

\subsubsection{胡桃(\textit{Juglans})}
核果。
外果皮肉质,干后纤维质,不规则开裂。
内果皮有雕纹。
不完全二至四室。
如胡桃(\textit{Juglans regia})。

\subsubsection{山核桃(\textit{Carya})}
核果。
外果皮木质四裂。
内果皮平滑,纵棱,四裂。
如山核桃(\textit{Carya cathayensis})。

\subsubsection{枫杨(\textit{Pterocarya})}
总状果序,坚果有两翅。
如枫杨(\textit{Pterocarya stenoptera})。

\subsection{杨梅科(Myricaceae)}
单叶,有芳香腺体。
花单性,风媒,无花被。
肉质核果。
如杨梅(\textit{Myrica rubra})。

\section{杨柳目(Salicales)}
仅杨柳科(Salicaceae)。
木本,单叶互生,有托叶。
单性花,雌雄异株,柔荑花序,每一苞片内一花,无小苞片,无花被,一杯状花盘或二腺状鳞片。
雄蕊二至多个。
子房一室,花柱一或二至四,二心皮,侧膜胎座,胚珠多,倒生。
蒴果,两瓣开裂。
种子细小,由珠柄生出柔毛。
胚直生,外包一层参与内胚乳。

\subsection{杨(\textit{Populus})}
叶宽阔,苞片条裂,杯状花盘,雄蕊多个,花序下垂。
如毛白杨(\textit{Populus tomentosa})。

\subsection{柳(\textit{Salix})}
叶窄,苞片全缘,无花盘,有分泌蜜汁多腺体。
雄蕊二,雄花序直立。
如垂柳(\textit{Salix babylonica})。

\section{壳斗目(Fagales)}
木本。
单叶互生,有托叶。
单性花,雌雄同株,单被。
雄花组成柔荑花序,雄蕊多数至二。
雌花单生,子房下位,一至六室。
坚果。

\subsection{壳斗科(Fagaceae)}
乔木。
单叶互生,革质,全缘或有锯齿,侧脉直出齿尖,有托叶。
单性花,雌雄同株。
雄花排成柔荑花序,花被八至四,合生,裂片覆瓦排列。
雄蕊七至四或更多,花丝细长,花药两室纵裂。
雌花位于雄花序基部,单生或二歧聚伞状,外围总苞,总苞有鳞、刺或毛。
花被和子房合生,四至七裂;子房下位,二至六室;花柱与子房数相等。
胚珠两个,珠被两层。
结实时总苞膨大变硬,成壳斗状。
坚果,无胚乳。

\subsubsection{石栎(\textit{Lithocarpus})}
常绿乔木。
雄花三至七朵簇生,花序直立。
雌花有退化雄蕊,子房三室。
总苞杯状,多不完全封闭,外被鳞片,内有坚果一枚。
如绵柯(\textit{Lithocarpus henryi})。

\subsubsection{栲(\textit{Castanopsis})}
常绿乔木。
雄花花序直立。
雌花单生,稀三朵歧伞排列,子房三室。
总苞封闭,有刺。
如桂林栲(\textit{Castanopsis chinensis})。

\subsubsection{栗(\textit{Castanea})}
落叶乔木。
雄花花序直立。
雌花子房六室。
总苞封闭,外面密生针状长刺,内有一至三枚坚果。
如板栗(\textit{Castanea mollissima})。

\subsubsection{三棱栎(\textit{Trigonobalanus})}
总苞裂瓣,常有三至五枚果实。
果实三棱形。
如三棱栎(\textit{Trigonobalanus doichangensis})。

\subsubsection{栎(\textit{Quercus})}
多为落叶乔木。
雄花花序下垂。
雌花一或二朵簇生,子房三至五室。
总苞鳞片覆瓦状或宽刺状。
如栓皮栎(\textit{Quercus variabilis})。

\subsubsection{水青冈(\textit{Fagus})}
落叶乔木。
雄花为下垂的头状花序。
雌花对生于有柄总苞内。
总苞有刺或瘤突。
坚果三角形。
如水青冈(\textit{Fagus longipetiolata})。

\subsection{桦木科(Betulaceae)}
乔木或灌木。
单叶互生,有托叶,具重锯齿,羽状叶脉直通齿尖。
单性花,雌雄同株,柔荑花序下垂。
花被片零至六,雄蕊一至六。
子房下位,二室,每室两胚珠,仅一胚珠发育。
一层珠被。
瘦果、坚果或具两翅的翅果。
如白桦(\textit{Betula platyphylla}),榛(\textit{Corylus heterophylla})。

\section{荨麻目(Urticales)}
草本或木本。
叶多互生,常有托叶。
花两性或单性,辐射对称,单被或无被。
雄蕊与花被对生。
子房上位,二心皮,一或二室,胚珠一或二个。
风媒或专性虫媒。

\subsection{榆科(Ulmaceae)}
草本或木本。
单叶互生,叶基偏斜,托叶早落。
花小,单生、簇生、短聚伞花序或总状花序。
雌雄同株。
花单被,萼片状,四至八裂。
雄蕊四至八,与花被对生,花丝直立。
花粉粒球形,二至五个萌发孔,外壁有脑纹或颗粒纹饰。
雄花有退化的雌蕊。
子房上位,二心皮,一或二室,每室一胚珠。
双珠被,花柱二,柱头头状。
翅果、坚果或核果,无胚乳。

\subsubsection{榆(\textit{Ulmus})}
乔木。
树皮有黏液,芽鳞片多,叶有重锯齿。羽状叶脉直通叶缘。
花两性。
翅果,果核扁平,子叶扁平。

\subsubsection{朴(\textit{Celtis})}
落叶。
叶基三出脉,侧脉不直达边缘。
核果,近球形。

\subsection{桑科(Moraceae)}
木本或草本,常有乳汁,有钟乳体。
叶多互生,托叶明显,早落。
花单性,单被或无被,稍肉质,宿存。
雄蕊一,或与花被同数对生。
子房上位,一室,二心皮,其中一个常不发育。
花柱二,胚珠一,倒生或弯生。
胎座基生或顶生。
瘦果、坚果或浆果,常成聚花果。
胚弯曲。

\subsubsection{桑(\textit{Morus})}
乔木或灌木。
叶互生,穗状花序,花丝向内弯曲。
肉质花被内包子房。

\subsubsection{榕(\textit{Ficus})}
有乳汁。
托叶大,抱茎,脱落后在茎上留有环痕。
花单性,生于肉质花序托构成的隐头花序内壁。
花序托开口处有多个总苞片。
雄花花被一至六,雄蕊一至二。
雌花花柱长。
另有瘿花,不育,花柱短粗,柱头喇叭口状,子房内有瘿蜂卵。
瘿蜂在产卵过程中传粉。

\subsection{荨麻科(Urticaceae)}
草本或灌木。
表皮细胞有钟乳体,茎皮纤维丰富。
单叶,互生或对生,两侧常不对称,有托叶。
茎叶生刺毛。
花细小,多单性,聚伞花序,密集成头状。
雄花花被四或五,与花被对生,花丝伸直。
雌花花被二至五,子房一室,胚珠一,直立或基生。
瘦果或核果,有胚乳。

\subsubsection{荨麻(\textit{Urtica})}
草本,被刺毛。
刺毛前段中空尖锐,基部有多细胞腺体。
叶对生。
花被四,雄蕊四,柱头毛帚状。

\subsubsection{苎麻(\textit{Boehmeria})}
灌木或小乔木。
叶三出脉,有锯齿。
花排成团伞花序,或再排成穗状或圆锥花序。
雄花被三至五,雄蕊三至五,退化雄蕊球形或梨形。
雌花被管状,二至四齿裂。
子房内藏,一室,花柱柔弱。
瘦果包裹于花萼内。

\subsubsection{杜仲科(Eucommiaceae)}
落叶乔木,有乳汁。
单叶互生,边缘有锯齿,无托叶。
单性花,雌雄异株,先叶开放,无花被。
雄花簇生,雄蕊五至十,花丝短,花药四室。
雌花单生小枝下部,有苞片,二心皮合生。
子房有柄,顶端二裂,柱头位于裂口内侧,胚珠二。
果扁平,不开裂,翅果先端二裂。
胚乳多,子叶肉质,外果皮膜质。

\section{檀香目(Santalales)}
木本或草本,常寄生。
叶互生、对生或退化,无托叶。
花两性或单性,辐射对称。
雄蕊与花被对生。
子房下位,一至五室,中轴胎座。
有胚乳。

\subsection{桑寄生科(Loranthaceae)}
寄生灌木。
叶对生或轮生,或退化为鳞片无托叶。
花萼与子房合生。
雄蕊与花瓣对生。
子房下位,一室,胚珠一。
子房与花托杯联合为浆果或核果。
肉质花托与外果皮间有黏液层。
种子一,胚乳多。

\subsubsection{桑寄生(\textit{Loranthus})}
羽状叶脉,两性花。

\subsubsection{槲寄生(\textit{Viscum})}
叶三脉,或退化。
花单性,单被。

\section{蓼目(Polygonales)}
仅蓼科。

\subsection{蓼科(Polygonaceae)}
草本、灌木或藤本。
叶互生,托叶膜质,托叶鞘包茎。
花两性或单性,辐射对称,单被或异被。
花被三至六,覆瓦排列,结实时增大为膜质。
雄蕊三至六,花药二室,纵裂,花盘环状或鳞片状。
子房上位,一室,花柱二至四,基生胚珠一。
坚果或瘦果,三棱形或凸镜形。
胚乳发达。

\subsubsection{荞麦(\textit{Fagopyrum})}
叶三角形。
两性花,花被五。
雄蕊八,花柱三,柱头头状。
果卵形,三锐棱。

\subsubsection{蓼(\textit{Polygonum})}
草本或藤本,膜质托叶鞘。
花被三至五。
瘦果。
如何首乌(\textit{Polygonum multiflorum}),虎杖(\textit{Polygonum cuspidatum})。

\subsubsection{大黄(\textit{Rheum})}
草本。
根状茎粗,黄色。
叶掌状浅裂。
两性花,花被外轮反卷。
坚果有翅。

\section{中央子目(Centrospermae)}
多草本。
多两性花,辐射对称。
雄蕊一或二轮,与花被对生。
子房上位,胚珠弯曲。
中轴胎座至特立中央胎座。
胚弯曲,有外胚乳。

\subsection{商陆科(Phytolaccaeae)}
草本或木本。
总状花序或聚伞花序。
两性花,辐射对称,花被四至五。
雄蕊三至多。
子房上位,心皮一至多,离生或合生,每室胚珠一个。
浆果或坚果,鲜有蒴果,有外胚乳。

\subsection{石竹科(Caryophyllaceae)}
多草本,茎节膨大,单叶对生,无托叶。
两性花,辐射对称,单生或二歧聚伞花序。
花萼分离或连成管状,四或五裂,膜质边缘,花瓣与萼片同数。
雄蕊五至十,或一至二。
花药二室,纵裂。
子房上位,一室,特立中央胎座,下半部为中轴胎座。
花柱连合或离生,胚珠多至一。
蒴果顶端齿裂或瓣裂。
胚弯曲包围外胚乳。

\subsubsection{石竹(\textit{Dianthus})}
一年或多年草本。
花萼脉清楚,下有苞片。
花瓣五,有爪,先端全缘,有齿或细裂。

\subsubsection{繁缕(\textit{Stellaria})}
花瓣五,二裂,雄蕊十,心皮三。

\subsection{藜科(Chenopodiaceae)}
常草本,泡状毛破裂干萎后成粉状或皮屑状。
单叶互生,肉质,无托叶。
花小,单被。绿色,辐射对称,集于叶腋或成穗状、圆锥花序。
花萼三至五裂,无花瓣。
雄蕊与萼片对生,花丝分离或基部连合。
花药二室纵裂。
子房上位或陷入萼基,一室,心皮二至五,花柱一至三,基生胚珠一。
胞果位于花萼或花苞内,不开裂。
种子扁平,胚弯曲包围外胚乳。
如梭梭(\textit{Haloxylon ammodendron}),猪毛菜(\textit{Salsola collina})。

\subsection{苋科(Amaranthaceae)}
萼片干膜质,有颜色,花丝基部连合。
蒴果,盖裂,胚弯曲包围外胚乳。
如鸡冠花(\textit{Celosia cristata})。

\section{木兰目(Magnoliales)}
木本。
花单生,花托发达,两性。
花各部螺旋排列或轮状排列。
雄蕊多,花粉粒单沟至三沟。
心皮一至多,离生。
胚小,胚乳丰富。

\subsection{木兰科(Magnoliaceae)}
木本,有香气。
单叶互生,全缘,托叶大。
种子大多一枚。
包被幼芽,早落,留下环痕。
两性花,大,单生于枝顶或叶腋,外有非绿色大型苞片。
花柄上或有绿叶状苞片。
花被多轮。
雄蕊多数,分离,螺旋排列于花托下半,花丝短。
花药长,二室,纵裂,药隔突出。
花粉粒舟状,单沟。
心皮多,螺旋排列于花柄延长形成的轴上,离生。
聚合蓇葖果,少数为带翅坚果。
种子悬挂于丝状珠柄维管束。
胚乳多,嚼烂状。

\subsubsection{木莲(\textit{Manglietia})}
多为大乔木。
有托叶痕,花顶生。
胚珠四至十四。

\subsubsection{木兰(\textit{Magnolia})}
花顶生,花被多轮。
心皮分离,胚珠二。
如玉兰(\textit{Magnolia denudata})。

\subsubsection{含笑(\textit{Michelia})}
花腋生。
结实时雌蕊轴伸长。
胚珠二至多。

\subsubsection{观光木(\textit{Tsoongiodendron})}
乔木,花腋生。
结果时心皮连合。
胚珠八至十。
聚合果硬木质。

\subsubsection{鹅掌楸(\textit{Liriodendron})}
叶分裂,先端截形,柄长,托叶二。
单花顶生,杯状,黄绿色。
萼片三,花瓣六。
花药向外开裂。
翅果不开裂。

\subsection{单心木兰科(Degeneriaceae)}
仅单心木兰(\textit{Degeneria vitiensis})。
乔木,有香气。
单叶互生,无托叶。
花腋生,柄长,两性,花被分化,花下有小苞片。
花萼三,花瓣十二至十八,排成三至五轮。
雄蕊多,三至四轮,无花丝。
花药四室,纵裂,嵌入侧脉与中脉之间。
发育雄蕊内侧有退化雄蕊。
心皮一,无花柱,心皮腹线边缘反折成柱头。
胚珠多。
果不开裂,种子有垂丝。

\subsection{蕃荔枝科(Annonaceae)}
乔木、灌木或藤本,常绿或落叶。
单叶互生,全缘,无托叶。
花香,单生或簇生叶腋,辐射对称,两性。
花被略分化,外轮萼状,内二轮瓣状,每轮三枚。
雄蕊多,螺旋排列,药隔突出。
心皮一至多,离生。
蓇葖果或聚合浆果,有长柄。
胚小,有假种皮,富嚼烂状胚乳。

\subsection{五味子科(Schisandraceae)}
木质藤本,含精油细胞。
花单性,雌雄异株。
花托肉质近球形。
心皮连合成肉质球。
浆果。

\subsection{八角茴香科(Illiciaceae)}
木本,有香气。
叶互生,无托叶。
花单生或聚生,两性。
花被七至三十三,多轮,有腺体。
雄蕊七至二十一,螺旋排列。
心皮八至二十,一轮。
胚珠一。
聚合蓇葖果。
胚小,油质胚乳发达。
仅八角(\textit{Illicium})。

\subsection{樟科(Lauraceae)}
木本,仅无根藤(\textit{Cassytha})为无叶寄生藤本。
叶和树皮有油细胞或黏液细胞。
单叶互生,全缘,三出或羽状脉,叶被有灰白色粉,无托叶。
两性花,辐射对称,圆锥、总状或丛生花序,腋生或近顶生。
花各部轮生。
花被留至四,不分化,两轮。
花被、花丝、花托合生为被丝托。
雄蕊九,每轮三枚,第四轮雄蕊退化。
第一、第二轮雄蕊花药向内,第三轮花药向外。
花药基部有腺体。
花药二至四室,瓣裂,花粉无萌发孔。
子房上位,三心皮,两枚前侧方心皮退化。
花柱单生,柱头二或三裂。
胚珠单生,悬垂。
浆果或核果,胚乳退化。

\subsubsection{润楠(\textit{Machilus})}
羽状叶脉。
结果时被丝托不增粗,花被向外反卷。
如泡花楠(\textit{Machilus pauhoi})。

\subsubsection{楠木(\textit{Phoebe})}
结果时花被裂片伸长增厚,成革质或木质,包裹果实基部。

\subsubsection{樟(\textit{Cinnamomum})}
叶互生或对生,三出脉或羽状脉。
无总苞,被丝托短。
结果时花被脱落。

\subsubsection{厚壳桂(\textit{Cryptocarya})}
羽状或三出叶脉。
两性花。
被丝托陀螺形或卵形,结果时增大,包住果实。
发育雄蕊九,花药二室。

\subsubsection{檫木(\textit{Sassafras})}
落叶乔木。
叶互生,聚于枝顶。
叶尖二或三裂。
单性花,雌雄异株,有异性退化痕迹。
总状花序,下有互生总苞片。
被丝托浅杯状,结果时增大。

\subsubsection{油丹(\textit{Alseodaphne})}
两性花。
被丝托肉质膨大。

\subsubsection{木姜子(\textit{Litsea})}
羽状叶脉。
花单性异株。
花序下有对生总苞。
如山苍子(\textit{Litsea cubeba})。

\subsubsection{无根藤(\textit{Cassytha})}
仅无根藤(\textit{Cassytha filiformis})。
寄生草质藤本,以盘状吸根侵入寄主。
叶鳞状或退化。
两性花,穗状花序。
被丝托包裹果实。

\subsection{水青树科(Tetracentraceae)}
仅水青树(\textit{Tetracentron sinensis})。
落叶乔木。
掌状叶脉,有锯齿,无托叶。
两性花小,穗状花序。
花被四。
雄蕊与花被对生。
心皮四,与雄蕊互生。
花柱四,直立于花蕾中央,后向外弯曲,最后位于蓇葖果外侧基部。
胚珠四,倒生。

\subsection{莽草科(Winteraceae)}
芳香,无托叶,次生木质部仅有管胞。
花药二室,外向。
花粉粒单孔。
浆果,胚珠多。
如林仙(\textit{Drimys})。

\subsection{腊梅科(Calycanthaceae)}
灌木,树皮芳香。
叶对生,无托叶。
花单生,花托壶状、坛状。
聚合瘦果。

\subsection{昆栏树科(Trochodendraceae)}
仅昆栏树(\textit{Trochodendron aralioides})。
常绿乔木,顶芽大,单叶互生,无托叶。
木质部仅有管胞。
两性花,顶生穗状或二歧聚伞花序,雌雄蕊轮生。
花丝细长,心皮一轮,胚珠多,蓇葖果。

\subsection{领春木科(Eupteleaceae)}
落叶乔木。
芽侧生,外包近鞘状叶柄基部。
单叶互生。
两性花,单生苞片内。
先叶植物,无花被。
心皮有长柄。
簇生聚合翅果。
仅领春木(\textit{Euptelea pleiosperma})和日本领春木(\textit{Euptelea polyandra})。

\subsection{连香树科(Cercidiphyllaceae)}
落叶乔木,叶对生,托叶早落。
花单性异株。
雄花无梗,花丝细长。
雌花有梗,腋生。
聚合蒴果,种子有翅。
仅连香树(\textit{Cercidiphyllum})。

\section{毛茛目(Ranales)}
多草本或木质藤本,含阿朴啡类生物碱。
雄蕊多数,分离,螺旋排列;或与花瓣对生。
心皮多,离生,螺旋排列或轮生。
胚直,胚乳丰富。

\subsection{毛茛科(Ranunculaceae)}
草本。
叶互生或基生,掌状或羽状分裂;或为一至多回小叶或羽状复叶。
两性花。
萼片五,绿色,花瓣状。
花瓣五。
雄蕊多,花药四至二室。
心皮多,离生。
胚珠多。
瘦果或蓇葖果。
种子有胚乳。

\subsubsection{铁线莲(\textit{Clematis})}
攀缘灌木。
羽状复叶对生,无托叶。
花序腋生或顶生。
萼片四或六至八,一般白色。
花瓣缺,雄蕊多,心皮多,羽毛状花柱。
瘦果。
如威灵仙(\textit{Clematis chinensis})。

\subsubsection{侧金盏花(\textit{Adonis})}
叶细裂。
萼片五至八,花瓣状。
花瓣五至十六,黄或红。
心皮多,胚珠下垂,瘦果。

\subsubsection{毛茛(\textit{Ranunculus})}
直立草本。
花白或黄,萼片五,花瓣五,基部常有蜜腺。
雄蕊和心皮均离生多数。
瘦果密集于花托。

\subsubsection{翠雀(\textit{Delphinium})}
草本,叶掌状分裂。
总状或圆锥花序,白色或紫色,萼片五,后面有两个长距。
花瓣小,二至四,后面有两个长距,套在萼片长距内。
心皮三至七,蓇葖果,多种子。
如大花飞燕草(\textit{Delphinium grandiflorum})。

\subsection{睡莲科(Nymphaeaceae)}
水生草本。
叶盾形或心形,有长柄,浮水。
两性花,离生萼四至六。
花瓣多,下位或周位。
雄蕊多。
心皮多,藏于肥大花托。
胚珠多。
如莲(\textit{Nelumbo nucifera}),莼菜(\textit{Brasenia schreberi}),芡(\textit{Euryale ferox})。

\subsection{金鱼藻科(Ceratophyllaceae)}
仅金鱼藻(\textit{Ceratophyllum})。
沉水草本。
茎纤细分枝。
叶轮生,劈裂为二叉裂片。
单性花小,单被。
雄蕊八至二十,花丝细,花药外向,药隔伸出。
子房一室,胚珠一,单层珠被。
坚果,胚大,无胚乳。

\section{胡椒目(Piperales)}
草本或木本。
草本茎有散生维管束。
叶对生或互生,有托叶。
无花被。
子房上位,心皮合生,侧膜胎座至半中轴胎座。
胚小,有胚乳。

\subsection{胡椒科(Piperaceae)}
木质或草质藤本,或肉质小草本。
藤本种类节膨大,生不定根。
叶互生、对生或轮生,有辛辣气,基部两侧不等,离基三出脉,托叶与叶柄合生。
穗状或肉穗花序。
单性花,小,雌雄异株。
有苞片,无花被。
雄蕊一至十,心皮一至四。
子房上位,一室,直立胚珠一,珠被一或二层。
浆果。
维管束散生,导管细小。
如胡椒(\textit{Piper nigrum})。

\section{马兜铃目(Aristolochiales)}
藤本,或为寄生植物。
单叶互生。
花辐射对称。
花被一轮,花瓣状,肉质。
子房下位,胚珠多数,珠被二层。
有胚乳。
蒴果或浆果。

\subsection{马兜铃科(Aristolochiaceae)}
多年生草本或藤本。
根苦辣,有香气。
叶互生,被灰白粉,基部心形,无托叶。
花单生两性。
花被单,花瓣状,合生管状或坛状,三裂或向一侧延长,紫色,臭气。
雄蕊六至多。
子房下位,四至六室。
蒴果。
如马兜铃(\textit{Aristolochia debilis}),细辛(\textit{Asarum heterotropoides})。

\subsection{大花草科(Rafflesiaceae)}
肉质寄生草本。如大花草(\textit{Rafflesia arnoldii})。

\section{藤黄目(Guttiferales)}
木本。
单叶互生或对生。
两性花,辐射对称,异被,覆瓦或螺旋排列。
雄蕊多。
子房上位,中轴或侧膜胎座。
胚珠多,珠被两层。
有胚乳。

\subsection{芍药科(Paeoniaceae)}
仅芍药(\textit{Paeonia})。
多年生草本或灌木。
叶互生长柄,掌状或羽状复叶,全缘。
花辐射对称,两性,单生或总状花序。
萼片五,花瓣五至十。
雄蕊多,离心发育,花丝细长。
心皮一至五,分离,革质,基部有蜜腺联合成的花盘包裹子房。
柱头大,花柱短,双珠被。
蓇葖果,种子有珠柄发育来的假种皮。
胚小,胚乳多。
如牡丹(\textit{Paeonia suffruticosa}),芍药(\textit{Paeonia lactiflora})。

\subsection{猕猴桃科(Actinidiaceae)}
木质藤本,髓实心或片层状。
单叶互生,常有锯齿,被粗毛或星状毛,羽状脉。
聚伞花序。
萼片五,覆瓦状。
花瓣五,雄蕊多。
花药丁字形着生,纵裂或顶裂。
子房上位,五室以上,花柱五。
浆果,被毛,胚乳多。

\subsubsection{猕猴桃(\textit{actinidia})}
常绿,少落叶。
萼片五至二,花瓣五至十二。

\subsection{龙脑香科(Dipterocarpaceae)}
木本,含树脂。
叶革质全缘,有托叶。
圆锥花序,腋生。
花两性,辐射对称,萼管五裂片,花瓣五,雄蕊多。
心皮三,下部合生。
坚果有翅,无胚乳。
如龙脑香(\textit{Dipterocarpus})。

\subsection{山茶科(Theaceae)}
木本,常绿或落叶。
茎叶中有石质细胞。
单叶互生,有胼胝质状锯齿,无托叶。
花两性,辐射对称,单生,或簇生于叶腋,有苞片。
萼片五至六,花瓣五至六。
雄蕊多,多轮。
心皮三至五,子房上位,中轴胎座。
蒴果、核果或浆果。

\subsubsection{山茶(\textit{Camellia})}
常绿乔木或灌木。
苞片与萼合生。
花红色、白色或黄色。
蒴果木质,背开裂。
胚大,无胚乳。
子叶半球形,富油脂。
如油茶(\textit{Camellia oleifera}),茶(\textit{Camellia sinensis}),普洱茶(\textit{Camellia assamica})。

\subsubsection{石笔木(\textit{Tutcheria})}
常绿乔木。
蒴果,种皮厚,子叶折叠状,无胚乳。

\subsubsection{木荷(\textit{Schima})}
乔木。
苞片二至八,脱落。
萼片五或六,宿存。
花瓣五。
子房五室。
蒴果扁球形,顶端平或微凹,中轴棒槌状。
种子有周翅,有胚乳。

\subsubsection{紫茎(\textit{Stewartia})}
有鳞芽。
叶柄不对折。
种子周翅至无翅。

\subsubsection{柃(\textit{Eurya})}
雌雄异株。
花细小,柄短,簇生叶腋。
雄花中有退化雌蕊。
苞片二,萼片五,花瓣五。
子房三至五室。
浆果,种子细小,胚乳多。
如米碎花(\textit{Eurya chinensis})。

\section{罂粟目(Papaverales)}
草本或灌木。
两性花,辐射对称或两侧对称,异被。
雄蕊分离或连合为两束。
心皮合生,子房一室,侧脉胎座。
胚乳多。

\subsection{罂粟科(Papaveraceae)}
草本或灌木,有乳汁。
叶互生或对生,常分裂,无托叶。
花单生,萼片二或三,苞片状,早落。
花瓣四至六或八至十二,两轮。
雄蕊多,分离。
花药两室,纵裂。
子房上位,心皮合成一室,侧膜胎座。
蒴果,瓣裂或孔裂,胚乳油质。

\subsubsection{罂粟亚科(Papaveroideae)}
罂粟(\textit{Papaver})多草本。
基生叶羽状裂,表面有白粉,两面被刚毛,有柄。
茎生叶无柄,有时抱茎。
花单生,总花梗直立,常被刚毛。
花蕾下垂,卵形或球形。
萼片二,开花前脱落,多被刚毛。
花瓣四,着于短花托,倒卵形,二轮,多红色,常早落。
雄蕊多,花丝丝状,花药球形或长圆形。
子房一室,上位,多卵珠形。
心皮四至八,连合,胚珠多,无花柱。
柱头四至十八,辐射状,连合成盘状于子房上,边缘圆齿状或分裂。
蒴果狭圆柱形、倒卵形或球形,或有肋,于辐射状柱头下孔裂。
种子多,小,肾形,黑色、褐色、深灰色或白色。
胚乳白色,肉质,富油。
如罂粟(\textit{Papaver somniferum}),虞美人(\textit{Papaver rhoeas})。

\par

博落回(\textit{Macleaya})直立多年草本。
茎有红色汁液,剧毒。
叶掌状分裂,基部心形,背面粉白。
圆锥花序顶生。
萼片二,乳白色。
无花瓣,雄蕊多,花柱短,柱头二。
蒴果卵形,有短柄,两瓣裂。

\par

绿绒蒿(\textit{Meconopsis})草本,有黄色汁液。
花大,色艳。
萼片二,早落。
花瓣四,雄蕊多。

\subsubsection{荷包牡丹亚科(Fumarioideae)}
无乳汁,有水液。
两性花双面对称。
萼片小,花瓣四,外一或二枚基部囊状或距状。
雄蕊四枚离生,或六枚连成两束。
子房一室,两个侧膜胎座。
如延胡索(\textit{Corydalis yanhusuo})。

\subsection{白花菜科(Capparidaceae)}
草本、灌木或乔木。
叶互生,单叶或掌状复叶,托叶常变态为刺或腺体。
两性花,辐射对称,单生或总状花序。
萼片四至八,花瓣四至八。
花盘环状或鳞片状。
雄蕊四至多,有柄。
子房上位,有柄,侧膜胎座。
蒴果或浆果。
如鱼目(\textit{Crateva religiosa}),醉蝶花(\textit{Cleome spinosa})。

\subsection{十字花科(Cruciferae)}
草本,或有辛辣汁液。
基生叶旋叠,茎上叶互生,无托叶。
花两性,辐射对称,总状花序。
萼片四,花瓣四,十字排列,基部爪状。
花托上有蜜腺,常与萼片对生。
雄蕊六,内轮四枚长,外轮二枚短。
子房上位,二心皮,一室,两个侧膜胎座。
常有一次生膜质假隔将子房分为两室。
柱头二,胚珠多。
长角或短角果,两瓣开裂。
种子无胚乳,子叶弯曲或折叠。

\subsubsection{芸薹(\textit{Brassica})}
一、二年生或多年生,或有块状根。
基生叶大头羽裂。
萼片基部囊状,花瓣黄色或白色。
蒴果有喙,果瓣有中脉,子叶对摺。
如卷心菜(\textit{Brassica oleracea} var. \textit{capitata}),花菜(\textit{Brassica oleracea} var. \textit{botrytis}),芥蓝(\textit{Brassica alboglabra}),大白菜(\textit{Brassica pekinensis}),小白菜(\textit{Brassica chinensis})。

\subsubsection{萝卜(\textit{Raphanus})}
花白或紫。
长角果内种子间稍缩,成熟时节间短开。
如萝卜(\textit{Raphanus sativus})。

\section{蔷薇目(Rosales)}
叶互生,有托叶。
多两性花,异被。

\subsection{悬铃木科(Platanaceae)}
落叶大乔木。
叶掌状,有锯齿,托叶大而抱茎。
单性花,雌雄同株,头状花序。
雄花序无苞片,雌花序有线形苞片。
雄花几乎无柄,托以微小鳞片,药隔顶端钝状。
雄花有退化的雌蕊。
雌花心皮多,分离,柱头生于一侧。
胚珠一或二,下垂。
坚果,有胚乳。
仅悬铃木(\textit{Platanus})。

\subsection{金缕梅科(Hamamelidaceae)}
乔木或灌木,枝叶常被星状毛。
单叶互生,掌状或羽状脉,多有托叶。
雌雄同株。
萼片四至五,合生成筒状。
花瓣四至五,或缺。
雄蕊四至十三。
花药两室,纵裂或瓣裂。
子房下位,两室。
花柱二。
蒴果,先端二喙状。

\subsubsection{双花木(\textit{Disanthus})}
仅双花木(\textit{Disanthus cercidifolius})。
头状花序,含两朵无柄对生的花。
萼片五,花瓣五,雄蕊五,子房上位,胚珠多,花柱短。

\subsubsection{马蹄荷(\textit{Exbucklandia})}
常绿乔木。
叶厚革质,长柄,掌状脉。
托叶大,包裹鳞芽。
头状花序,萼短,种子有翅。

\subsubsection{红苞木(\textit{Rhodoleia})}
羽状叶脉。
两性花,头状花序。
总苞片覆瓦排列。
花瓣二至五,红色,列于花序外侧。

\subsection{虎耳草科(\textit{Saxifragaceae})}
多草本,叶互生,无托叶。
花辐射对称,萼片五,花瓣缺或与萼片互生。
雄蕊十至十五,生于花瓣上。
子房上位或下位,一至三室。
花柱分离,胚珠多,珠被二。
蒴果或浆果。

\subsection{蔷薇科(Rosaceae)}
木本或草本,常有刺。
叶互生。
两性花,辐射对称,周位或上位。
萼片五,雄蕊多,花丝分离,花药二室,子房上位或下位。
种子胚乳不发达。

\subsubsection{绣线菊亚科(Spiraeoideae)}
子房上位。
心皮五,离生或基部连合。
蓇葖果。
如绣线菊(\textit{Spiraea})。

\subsubsection{蔷薇亚科(Rosoideae)}
子房上位,心皮多至一,离生于突出的花托上或壶形花托内。
胚珠一至二。
聚合瘦果或聚合核果。
如玫瑰(\textit{Rosa rugosa}),月季(\textit{Rosa chinensis}),草莓(\textit{Fragaria ananassa}),蛇莓(\textit{Duchesnea indica})。

\subsubsection{苹果亚科(Maloideae)}
心皮五至二,背面和花托内侧愈合。
子房半下位或下位,胚珠一至二。
梨果。
如白梨(\textit{Pyrus bretachneideri}),枇杷(\textit{Eriobotrya japonica}),苹果(\textit{Malus pumila}),山楂(\textit{Crataegus pinnatifida})。

\subsubsection{梅亚科(Prunoideae)}
子房上位。
心皮一,生于凹陷的花托上,但不与花托愈合。
如梅(\textit{Prunus mume}),杏(\textit{Prunus armeniaca}),李(\textit{Prunus salicina}),桃(\textit{Prunus persica}),日本樱花(\textit{Prunus phaeosticta})。

\subsection{豆科(Leguminosae)}
木本或草本。
单叶或复叶,常有叶枕和托叶。
两性花,萼片五,离生花瓣五。
子房上位,心皮一。
荚果,无胚乳。

\subsubsection{含羞草亚科(Mimosoideae)}
多木本。
一或二回羽状复叶,常有托叶。
花辐射对称,穗状或头状花序。
萼片五,或三至六,常合生。
花瓣镊合状排列,分离或连成短筒。
雄蕊多,花药二室,纵裂,顶端有一脱落腺体。
子房上位,胚珠多,荚果有次生横隔膜。
如合欢(\textit{Albizzia julibrissin}),含羞草(\textit{Mimosa pudica})。

\subsubsection{苏木亚科(Caesalpinioideae)}
木本。
一或二回羽状复叶或单叶,托叶常缺。
花稍两侧对称,总状、穗状或聚伞花序。
花瓣上升覆瓦排列。
雄蕊十或较少。
如决明(\textit{Cassia tora}),腊肠树(\textit{Cassia fistula}),番泻叶(\textit{Cassia anustifolia}),皂荚(\textit{Gleditsia sinensis}),羊蹄甲(\textit{Bauhinia})。

\subsubsection{蝶形花亚科(Papilionoideae)}
木本、草本或藤本。
单叶、三小叶或一至多回羽状复叶,有托叶和小托叶。
花两侧对称,蝶形。
花萼五裂。有萼管。
花瓣下降覆瓦排列,最上一片为旗瓣,侧面两片为翼瓣,最下两片龙骨瓣常连合。
花瓣有爪和胼胝体。
荚果开裂或不开裂,或有节荚。
常共生固氮根瘤菌。

\par

红豆(\textit{Ormosia})乔木,奇数羽状复叶,小叶对生。
花白或紫。
花萼钟状,宿存。
雄蕊十,分离。
花柱拳卷,种皮朱红。
如花榈木(\textit{Ormosia henryi})。

\par

槐(\textit{Sophora})灌木或乔木,奇数羽状复叶,小叶对生。
花白或黄,少数蓝紫。
龙骨瓣直立,比旗瓣长。
雄蕊十,分离,或基部合生成环状。
荚果圆柱形,在种子间紧缩成串珠状,不开裂。

\par

苜蓿(\textit{Medicago})草本,羽状三出复叶。
小叶有齿,叶脉伸入齿端。
花黄色或紫色,旗瓣无柄,龙骨瓣短于翼瓣。
雄蕊十,其中九个合生。
花柱短,扁或锥状。
荚果旋卷成陀螺状,或有刺,

\par

大豆(\textit{Glycine})草本,羽状三出复叶。
花白、紫、蓝。
雄蕊十,全部分离或九个合生。

\par

葛(\textit{Pueraria})缠绕植物,羽状三出复叶,托叶基部着生或盾状着生,有小托叶。
花蓝或紫。
萼钟状,萼裂片不等大。
旗瓣有柄有耳。
翼瓣狭窄,中部与龙骨瓣贴生。
雄蕊十,九个合生。
花柱无毛。

\par

豌豆(\textit{Pisum})一或多年生草本。
偶数羽叶。
叶轴顶端为分枝卷须,托叶叶状。
花萼钟状偏斜。
花白、紫或红。
雄蕊十,九个合生。
花柱内侧有纵裂鬃毛。
豆荚肿胀。

\par

甘草(\textit{Glycyrrhiza})多年生草本或半灌木,有刺毛或鳞片状腺体。
奇数羽叶,小叶全缘或有微齿,无小托叶。
雄蕊十,九个合生。
荚果有刺毛状或小疣状突起。

\par

落花生(\textit{Arachis})匍匐草本。
偶数羽叶。
花红,萼管细长。
雄蕊十,合生。
闭花受精后,花柄伸长把子房推入地下,形成荚果。

\par

黄檀(\textit{Dalbergia})攀缘灌木或乔木。
奇数羽状复叶,小叶互生,无小托叶。
花小,白、黄或紫。
花瓣有柄。
雄蕊十,合生或九个合生。
荚果薄,扁平,不裂,在种子处的果瓣常增厚有网纹。

\subsection{牛拴藤科(Connaraceae)}
心皮一至五,分离或近分离。
蓇葖果。
种子有假种皮。

\section{牻牛儿苗目(Geraniales)}
多草本。
花两性或单性,辐射或两侧对称。
萼片三至五,花瓣五至零。
雄蕊四至五或八至十。
常有花盘。
心皮合生,中轴胎座,胚珠多至一。
无胚乳。

\subsection{牻牛儿苗科(Geraniaceae)}
多草本。
叶互生或对生,掌状或羽状裂。
两性花,辐射对称,聚伞花序。
萼五,瓣五,均覆瓦排列。
雄蕊十至十五,两轮,有五个具有退化雄蕊的蜜腺。
子房上位,三至五室。
花柱与心皮同数。
蒴果中轴延伸成喙,室间开裂。
果瓣由基部向上反卷成螺旋状,附于中轴顶端,种子借果瓣反卷弹射出去。
胚乳不发达,子叶摺叠。

\subsubsection{老鹳草(\textit{Geranium})}
被倒向毛。
聚伞花序,具两花。
五萼,五花瓣,五蜜腺。
蒴果有长喙。
果瓣五,内无毛,反卷附于喙顶。

\subsection{大戟科(Euphorbiaceae)}
木本或草本,常有乳汁。
单叶互生,间有对生,叶基部常有腺体,托叶早落。
单性花,多聚伞花序或杯花。
萼片二至五,无花瓣,有花盘或腺体。
雄蕊多,花药二室,纵裂或孔裂。
子房三室,胚珠一或二。
花柱与子房同数。
多为蒴果,胚乳肉质,胚直生,种阜发达。

\subsubsection{大戟(\textit{Euphorbia})}
无花被。
杯状或聚伞花序,其中雌花仅一朵,雄花围绕雌花组成螺旋单歧聚伞花序。
雄花一苞片,一雄蕊。
子房三室,花柱三。
花柱先端二裂。

\subsubsection{重阳木(\textit{Bischofia})}
乔木,三出复叶,小叶有锯齿。
单性异株。
无花瓣,无花盘。
雄花花萼五,分离雄蕊五,有退化雌蕊。
雌花子房三室,花柱长而肥厚,顶端不裂。
浆果球形。

\subsubsection{算盘子(\textit{Glochidion})}
常灌木,叶排成两列,托叶宿存,无乳汁。
雌雄同株,花小无瓣,聚伞花序。
雄花无花盘,萼片六,雄蕊三至八,花丝花药合生成圆筒状。
雌花萼厚,子房三至十五室,花柱合生。
蒴果有纵沟。

\subsubsection{油桐(\textit{Vernicia})}
落叶乔木。
叶互生,全缘或三至五裂,叶柄顶端和叶裂处有粗大腺体。
单性花大,圆锥花序,萼二至三,瓣五,白色或略红。
雄蕊八至二十,花丝短,合生。
雌花子房三至五,胚珠一,花柱二裂。
核果,厚壳状种皮。

\subsubsection{巴豆(\textit{Croton})}
灌木或乔木,被星状毛或鳞片。
叶基部或叶柄顶端有两腺体。
单性花,雌雄同株,总状花序。
雄花萼五,瓣五,花盘五裂,离生雄蕊五至二十。
雌花萼五,无瓣,花盘环状或分裂,子房三室,胚珠一。
花柱三,离生,顶二至四裂。
蒴果。

\subsubsection{乌桕(\textit{Sapium})}
乔木,叶柄顶端有两突起腺体。
单性花,雌雄同序,雌花在下。
无花瓣,无花盘,花萼二至五,雄蕊二至三,花丝分离,子房二至三室,胚珠一,花柱三,柱头外卷。
蒴果,蜡质假种皮。

\subsubsection{五月茶(\textit{Antidesma})}
乔木或灌木,叶全缘,无乳汁。
单性花,异株。
花小,无瓣,萼三至五裂。
雄花花盘垫状,雄蕊二至五。
雌花花盘杯状或环状,子房一室,花柱二至三,顶端二裂。
核果。

\section{芸香目(Rutales)}
多木本,叶常有腺点,鲜有托叶。
两性花,辐射对称。
萼片覆瓦排列,花瓣螺旋状或镊合状。
子房上位,胚珠一或二。

\subsection{芸香科(Rutaceae)}
常有刺,茎叶树皮有柑橘香气。
复叶,有透明油腺点,无托叶。
两性花,辐射对称。
萼片五至三,花瓣五至三。
雄蕊与花瓣同数或加倍。
花盘位于雄蕊内侧。
子房上位,心皮四至五或三至一,多合生。
胚珠一般两个。
浆果、核果或蒴果。

\subsubsection{花椒(\textit{Zanthoxylum})}
灌木,有刺,奇羽状复叶。
花小,单性,异株。
心皮五至一,柄明显。
胚珠二。
蓇葖果,果皮有瘤状突起腺点。

\subsubsection{柑橘(\textit{Citrus})}
有刺灌木或小乔木。
叶互生,花两性。
花萼杯状,二至五裂,结果时增大。
花瓣四至八,雄蕊为花瓣数的四至六倍,花丝合生。
子房七至十五室,柑果。

\section{无患子目(Sapindales)}
多木本。
花多辐射对称,有花盘。
子房上位,心皮一至五。

\subsection{漆树科(Anacardiaceae)}
木本,有树脂道。
叶互生,无托叶。
花小而多,单性,辐射对称,圆锥花序或总状花序。
萼片五至三,花瓣五至三或缺。
雄蕊十至五,花丝分离。
花盘全裂或分裂。
子房上位,花柱一至五,倒生胚珠一。
核果,胚乳不发达。
如芒果(\textit{Mangifera indica}),盐肤木(\textit{Rhus chinensis}),漆树(\textit{Toxicodendron verniciflua}),腰果(\textit{Anacardium occidentale}),开心果(\textit{Pistacia vera})。

\subsection{槭树科(Aceraceae)}
木本,落叶。
叶对生,无托叶。
花辐射对称,总状或圆锥花序。
萼片四至五。
花瓣四至五或缺。
花盘扁平,环状或分裂,或退化成齿,或缺。
雄蕊四至十。
子房上位,二室,二裂,花柱二,胚珠二。
扁平双翅果。
如三角枫(\textit{Acer buergerianum}),金钱槭(\textit{Dipteronia sinensis})。

\subsection{无患子科(Sapindaceae)}
木本,羽状复叶。
两性花,辐射对称,总状、圆锥或伞状花序。
花萼四至五,花瓣四至五或缺。
花盘发达,位于雄蕊之外。
雄蕊八至十。
子房上位,胚珠一至二。
无胚乳。
如无患子(\textit{Sapindus mukorossi}),龙眼(\textit{Dimocarpus longan}),荔枝(\textit{Litchi chinensis}),红毛丹(\textit{Nephelium lappaceum})。

\section{卫矛目(Celastrales)}
木本或藤本。
单叶互生或对生,托叶小或缺。
花辐射对称,花瓣与萼片同数。
雄蕊一轮,与花瓣对生,有花盘。
子房上位,胚珠一或二,有胚乳。

\subsection{冬青科(Aquifoliaceae)}
乔木或灌木,多常绿。
单叶互生,托叶小或缺。
花小,辐射对称,单性,聚伞花序或簇生于叶腋。
花萼细小,三至六裂,覆瓦排列,宿存。
花瓣四或五,雄蕊与花瓣互生。
花药二室,纵裂,无花盘。
子房上位,三至多室。
胚珠一或二,悬垂。
核果,内有分核。
胚小,直立,胚乳多。
如冬青(\textit{Ilex})。

\subsection{卫矛科(Celastraceae)}
木本或藤本,叶互生或对生。
有花盘。
雄蕊四至五,位于花盘边缘。
子房一至五室。
假种皮色彩鲜艳。
如卫矛(\textit{Euonymus}),南蛇藤(\textit{Celastrus orbiculatus}),雷公藤(\textit{Tripterygium wilfordii})。

\section{鼠李目(Rhamnales)}
木本或藤本,单叶互生或对生。
雄蕊与花瓣或萼片对生。
有胚乳。

\subsection{鼠李科(Rhamnaceae)}
木本或藤本,单叶互生或对生,有托叶。
花小,两性,聚伞花序。
花萼筒状,四至六浅裂。
花瓣四至五,或缺,常短于萼片。
雄蕊五,与花瓣对生,包藏于花瓣内。
花药二室,纵裂,花盘周位。
子房上位,二至四室,或陷入花盘。
花柱浅裂,胚珠一,倒生。
核果、翅果或蒴果,胚乳不发达。
如枣(\textit{Ziziphus jujuba}),拐枣(\textit{Hovenia dulcis})。

\subsection{葡萄科(Vitaceae)}
藤本,有卷须。
叶互生,有托叶。
花小,辐射对称,聚伞或圆锥花序,与叶对生。
萼片细小,四或五。
花瓣四或五,镊合排列,早落。
雄蕊四或五,与花瓣对生。
花盘环状或浅裂,紧贴子房。
子房上位,六至二室,胚珠二或一。
浆果,有胚乳。
如葡萄(\textit{Vitis vinifera}),爬山虎(\textit{Parthenocissus tricuspidata})。

\section{锦葵目(Malvales)}
木本或草本,富纤维。
单叶互生。
雄蕊多,子房上位,心皮常合生,中轴胎座,胚珠多,珠被两层。

\subsection{锦葵科(Malvaceae)}
乔木、灌木或草本,被星状毛。
茎皮多纤维,有黏液。
单叶互生全缘,有锯齿或分裂,掌状脉。
托叶二,早落。
两性花,辐射对称,单生或簇生于叶腋,或总状、圆锥花序。
萼片五至三,常基部合生,镊合排列。
花瓣五,螺旋排列,近基部与雄蕊管连合。
雄蕊多,花丝连合成管。
花药一式,纵裂。
子房上位,中轴胎座,花柱一。
蒴果、分果或浆果。
如棉(\textit{gossypium}),洋麻(\textit{Hibiscus cannabinus}),秋葵(\textit{Abelmoschus esculentus})。

\subsection{杜英科(Elaeocarpaceae)}
木本,单叶互生或对生,有托叶。
常两性花,总状或圆锥花序。
萼五至四,瓣与萼同数或缺。
雄蕊多,生于花盘,花药顶孔开裂。
子房多至二室,胚珠多至二。
核果、浆果或蒴果。
如杜英(\textit{Elaeocarpus}),猴欢喜(\textit{Sloanea sinensis})。

\subsection{椴树科(Tiliaceae)}
木本,稀草本,茎皮富纤维。
单叶,多三出脉,有托叶。
两性花,辐射对称,萼五至三,瓣五或缺,基部有腺体。
雄蕊多,或合生成束。
子房上位,十至二室,胚珠一至多。
蒴果、核果或浆果。
如椴树(\textit{Tilia}),刺蒴麻(\textit{Triumfetta bartramia}),黄麻(\textit{Corchorus capsularis})。

\subsection{梧桐科(Sterculiaceae)}
草本、乔木或灌木。
叶互生,单叶或指状复叶,有托叶。
花辐射对称。
萼片五,花瓣五或缺。
雄蕊多,合生成管。
子房上位,二至五室。
果干燥或肉质。
如苹婆(\textit{Sterculia nobilis}),梧桐(\textit{Firmiana simplex}),可可(\textit{Theobroma cacao})。

\subsection{木棉科(Bombacaceae)}
落叶乔木,树干有皮刺。
两性花,大,花萼杯状。
雄蕊五至多轮。
蒴果,种子有绵毛。
如木棉(\textit{Gossampinus malabarica})。

\section{堇菜目(Violales)}
雄蕊多至少,子房上位,侧膜胎座。

\subsection{堇菜科(Violaceae)}
木本或草本。
单叶互生或基生,有托叶。
两性花,辐射对称或两侧对称,单生或圆锥花序。
萼片五,宿存,覆瓦排列。
花瓣五,覆瓦排列,下面一枚大而基部有囊。
雄蕊五,花药分离或围绕子房排成一圈,内向,纵裂。
药隔延伸至药室顶外面,形成膜质附属体。
子房上位,一室,三个侧膜胎座。
花柱单生。
胚珠多至一,倒生。
蒴果或浆果。
胚乳肉质。
如紫花地丁(\textit{Viola philippica} ssp. \textit{munda}),三色堇(\textit{Viola tricolor} var. \textit{hortensis})。

\section{葫芦目(Cucurbitales)}
仅葫芦科(Cucurbitaceae)。
攀缘或匍匐草本,常有卷须。
单叶互生,掌状分裂。
单性花,雌雄同株或异株,单生或总状、圆锥花序。
雄花花萼管状,五裂,合生花瓣五。
雄蕊五,其中两对合生。
花药曲成S形。
雌花子房下位,萼管与子房相连,花瓣合生,五裂,侧脉胎座三,胚珠多。
花柱一至四。
瓠果,肉质最终干燥变硬,不开裂、瓣裂或周裂。
种子多,扁平,无胚乳。
如南瓜(\textit{Cucurbita moschata}),黄瓜(\textit{Cucumis sativa}),冬瓜(\textit{Benincasa hispida}),丝瓜(\textit{Luffa aegyptica}),苦瓜(\textit{Momordica charantia}),葫芦(\textit{Lagenaria siceraria}),佛手瓜(\textit{Sechium edule}),西瓜(\textit{Citrullus lanatus}),甜瓜(\textit{Cucumis melo}),罗汉果(\textit{Momordica grosvenori}),瓜蒌(\textit{Trichosanthes kirilowii})。

\section{桃金娘目(Myrtales)}
草本或木本,单叶。
两性花,辐射对称,异被。
雄蕊两轮,子房多至一室。
花柱多单生,胚珠多至一,中轴胎座。

\subsection{桃金娘科(Myrtaceae)}
常绿灌木或乔木,茎有双韧维管束。
单叶对生,全缘,羽状脉或基出三至五脉,有边脉,有透明腺点,无托叶。
花两性,辐射对称。
萼筒与子房略合生。
萼片三至多。
花瓣四至五,生于花盘边缘或与萼片连成帽状。
雄蕊多,生于花盘边缘。
花丝分离或连成管状,或成束状与花瓣对生。
药隔顶端有一腺体。
子房下位,多至一室,中轴胎座,胚珠多。
浆果、核果或蒴果。
胚直立,无胚乳。
如桉(\textit{Eucalyptus}),莲雾(\textit{Syzygium samarangense}),白千层(\textit{Melaleuca leucadendra}),桃金娘(\textit{Rhodomyrtus tomentosa}),番石榴(\textit{Psidium guajava})。

\subsection{野牡丹科(Melastomataceae)}
草本或木本。
叶对生,基出脉,无托叶。
两性花,辐射对称,少有单生。
萼管五或四裂,或截平。
花瓣五至四。
花药顶孔开裂,药隔有附属体或下延成距。
子房下位或半下位,中轴胎座或特立中央胎座,胚珠多。
浆果或蒴果。
如野牡丹(\textit{Melastoma candidum}),野海棠(\textit{Bredia fordii})。

\subsection{红树科(Rhizophoraceae)}
常绿灌木或小乔木。
单叶革质对生,托叶早落。
两性花,单生或丛生于叶腋,或聚伞花序。
萼管与子房合生或分离。
萼片三至十六,花瓣与萼片同数。
雄蕊与花瓣同数或二倍,常与花瓣对生。
子房下位或半下位,二至六室,胚珠二。
果革质,不开裂。
生于海岸线的品种有胎萌现象,果实成熟后种子在母体上发芽,而后幼苗坠入海滩淤泥。
如红树(\textit{Rhizophora apiculata}),秋茄(\textit{Kandelia candel}),木榄(\textit{Bruguiera gymnorhiza}),竹节树(\textit{Carallia brachiata})。

\section{伞形目(Umbellales)}
草本或木本。
伞形花序,两性,辐射对称。
子房下位,心皮一至五。
胚珠一,悬垂。珠被一层。
胚小,胚乳丰富。

\subsection{五加科(Araliaceae)}
木本,或攀援藤本,或多年生草本,有刺。
单叶,或掌状、羽状复叶,互生,叶柄基部抱茎,托叶边缘革质舌状。
花小,伞形或头状花序。
花萼小,与子房连生。
花瓣五至十,分离。
雄蕊与花瓣互生,或为花瓣两部。
花盘位于子房顶部。
子房下位,一至十五室,倒悬胚珠一。
核果或浆果,胚乳多。

\subsubsection{鹅掌柴(\textit{Schefflera})}
乔木或灌木,有香气。
掌状复叶。
头状、穗状、总状或伞形花序,再组成复圆锥花序。
如鸭脚木(\textit{Schefflera heptaphylla})。

\subsubsection{五加(\textit{Acanthopanax})}
有刺,有香气。
掌状复叶,三至五小叶。
伞形花序,再组成复圆锥花序,花梗无关节。

\subsubsection{人参(\textit{Panax})}
多年生草本,掌状复叶,三至五小叶,轮生茎顶。
如人参(\textit{Panax ginseng}),西洋参(\textit{Panax quinquefolium}),三七(\textit{Panax notoginseng})。

\subsection{伞形科(Umbelliferae)}
草本,有香气,茎中空。
叶互生,常为复叶,叶柄基部膨大或成鞘状,无托叶。
两性花,辐射对称,伞形花序。
萼片五,不明显。
花瓣五,在花蕾时向内弯。
雄蕊五,与花瓣互生,生于花盘周围。
花盘位于花柱基部。
子房下位,二室,胚珠一,花柱二。
果由两个有棱或有翅的心皮构成,成熟时心皮下部分离,上部挂在心皮轴上,称为双悬果。
每个悬果有五条纵行主棱。
棱间有沟,沟内有油管。
胚细小,有胚乳。
如胡萝卜(\textit{Daucus carota} var. \textit{sativa}),芫荽(\textit{Coriandrum sativum}),茴香(\textit{Foeniculum vulgare}),芹菜(\textit{Apium graveolens} var. \textit{dulce})。

\subsubsection{当归(\textit{Angelica})}
叶三出羽状裂或羽状多裂。
叶柄膨大成管状或囊状鞘。
复伞花序,萼小。
果背腹压扁。
如当归(\textit{Angelica sinensis}),兴安白芷(\textit{Angelica dahurica}),杭白芷(\textit{Angelica formosana}),川白芷(\textit{Angelica anomala})。

\subsubsection{柴胡(\textit{Bupleurum})}
单叶全缘,复伞花序,总苞与小总苞宿存,缺萼齿,瓣片背有突起中脉,花柱短,果侧面压扁。
如北柴胡(\textit{Bupleurum chinensis}),狭叶柴胡(\textit{Bupleurum scorzonerifolium})。

\subsection{八角枫科(Alangiaceae)}
木本,叶互生,花两性,聚伞花序。
萼、瓣均四至十,雄蕊八至二十或更多。
子房下位,一或二室。
核果。
仅八角枫(\textit{Alangium})。

% 蓇葖果



\end{sloppypar}
\end{document}
