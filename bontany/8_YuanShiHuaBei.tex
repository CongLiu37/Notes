\documentclass[11pt]{article}

\usepackage[UTF8]{ctex} % for Chinese 

\usepackage{setspace}
\usepackage[colorlinks,linkcolor=blue,anchorcolor=red,citecolor=black]{hyperref}
\usepackage{lineno}
\usepackage{booktabs}
\usepackage{graphicx}
\usepackage{float}
\usepackage{floatrow}
\usepackage{subfigure}
\usepackage{caption}
\usepackage{subcaption}
\usepackage{geometry}
\usepackage{multirow}
\usepackage{longtable}
\usepackage{lscape}
\usepackage{booktabs}
\usepackage{natbibspacing}
\usepackage[toc,page]{appendix}
\usepackage{makecell}
\usepackage{amsfonts}
 \usepackage{amsmath}
\usepackage[utf8]{inputenc}
\usepackage{amssymb}
\usepackage{amsthm}
\usepackage{enumerate}
\usepackage{comment}

\usepackage[backend=bibtex,style=authoryear,sorting=nyt,maxnames=1]{biblatex}
\bibliography{} % Reference bib

\title{双子叶植物-原始花被亚纲(Archichlamydeae)}
\date{}

\linespread{1.5}
\geometry{left=2cm,right=2cm,top=2cm,bottom=2cm}

\setlength\bibitemsep{0pt}

\begin{document}
\begin{sloppypar}
  \maketitle

  \linenumbers
原始花被亚纲(Archichlamydeae)花瓣分离。
雄蕊生在花托上,珠被两层,种子有胚乳。

\section{木麻黄目(Casuarinales)}
乔木或灌木。
小枝轮生或假轮生,有节,纤细。
叶退化为鳞状,环状轮生成鞘。
花单性。
雄花轮生于花序轴,成柔荑花序。
雄花苞片二,杯状合生,花被片一或二,雄蕊一。
雌花序头状,苞片一,无花被,心皮二枚合生,花柱短,柱头二,子房二(后为单室),胚珠二。
扁平小坚果,有翅。
种子单生,种皮膜质。
仅木麻黄科(Casuarinaceae)。
如木麻黄(\textit{Casuarina equisetifolia})。

\section{胡桃目(Juglandales)}
芳香乔木,多有树脂。
羽状复叶,互生。
雄花单花被。
雌花三花被,子房下位,一或二室,胚珠一个,无胚乳。

\subsection{胡桃科(Juglandaceae)}
落叶乔木,有树脂。
羽状复叶,互生,无托叶。
花单性,雌雄同株。
雄花排成下垂的柔荑花序,花被与苞片相连,三至六裂,雄蕊一般三个。
雌花为穗状花序,直立无柄,小苞片一或二枚,花被与子房连生,三至五裂,子房下位一室或不完全二至四室,羽状花柱二,基生胚珠一。
核果或具翅坚果。
种子单生,无胚乳。
子叶皱褶含油。

\subsubsection{胡桃(\textit{Juglans})}
核果。
外果皮肉质,干后纤维质,不规则开裂。
内果皮有雕纹。
不完全二至四室。
如胡桃(\textit{Juglans regia})。

\subsubsection{山核桃(\textit{Carya})}
核果。
外果皮木质四裂。
内果皮平滑,纵棱,四裂。
如山核桃(\textit{Carya cathayensis})。

\subsubsection{枫杨(\textit{Pterocarya})}
总状果序,坚果有两翅。
如枫杨(\textit{Pterocarya stenoptera})。

\subsection{杨梅科(Myricaceae)}
单叶,有芳香腺体。
花单性,风媒,无花被。
肉质核果。
如杨梅(\textit{Myrica rubra})。

\section{杨柳目(Salicales)}
仅杨柳科(Salicaceae)。
木本,单叶互生,有托叶。
单性花,雌雄异株,柔荑花序,每一苞片内一花,无小苞片,无花被,一杯状花盘或二腺状鳞片。
雄蕊二至多个。
子房一室,花柱一或二至四,二心皮,侧膜胎座,胚珠多,倒生。
蒴果,两瓣开裂。
种子细小,由珠柄生出柔毛。
胚直生,外包一层参与内胚乳。

\subsection{杨(\textit{Populus})}
叶宽阔,苞片条裂,杯状花盘,雄蕊多个,花序下垂。
如毛白杨(\textit{Populus tomentosa})。

\subsection{柳(\textit{Salix})}
叶窄,苞片全缘,无花盘,有分泌蜜汁多腺体。
雄蕊二,雄花序直立。
如垂柳(\textit{Salix babylonica})。

\section{壳斗目(Fagales)}
木本。
单叶互生,有托叶。
单性花,雌雄同株,单被。
雄花组成柔荑花序,雄蕊多数至二。
雌花单生,子房下位,一至六室。
坚果。

\subsection{壳斗科(Fagaceae)}
乔木。
单叶互生,革质,全缘或有锯齿,侧脉直出齿尖,有托叶。
单性花,雌雄同株。
雄花排成柔荑花序,花被八至四,合生,裂片覆瓦排列。
雄蕊七至四或更多,花丝细长,花药两室纵裂。
雌花位于雄花序基部,单生或二歧聚伞状,外围总苞,总苞有鳞、刺或毛。
花被和子房合生,四至七裂;子房下位,二至六室;花柱与子房数相等。
胚珠两个,珠被两层。
结实时总苞膨大变硬,成壳斗状。
坚果,无胚乳。

\subsubsection{石栎(\textit{Lithocarpus})}
常绿乔木。
雄花三至七朵簇生,花序直立。
雌花有退化雄蕊,子房三室。
总苞杯状,多不完全封闭,外被鳞片,内有坚果一枚。
如绵柯(\textit{Lithocarpus henryi})。

\subsubsection{栲(\textit{Castanopsis})}
常绿乔木。
雄花花序直立。
雌花单生,稀三朵歧伞排列,子房三室。
总苞封闭,有刺。
如桂林栲(\textit{Castanopsis chinensis})。

\subsubsection{栗(\textit{Castanea})}
落叶乔木。
雄花花序直立。
雌花子房六室。
总苞封闭,外面密生针状长刺,内有一至三枚坚果。
如板栗(\textit{Castanea mollissima})。

\subsubsection{三棱栎(\textit{Trigonobalanus})}
总苞裂瓣,常有三至五枚果实。
果实三棱形。
如三棱栎(\textit{Trigonobalanus doichangensis})。

\subsubsection{栎(\textit{Quercus})}
多为落叶乔木。
雄花花序下垂。
雌花一或二朵簇生,子房三至五室。
总苞鳞片覆瓦状或宽刺状。
如栓皮栎(\textit{Quercus variabilis})。

\subsubsection{水青冈(\textit{Fagus})}
落叶乔木。
雄花为下垂的头状花序。
雌花对生于有柄总苞内。
总苞有刺或瘤突。
坚果三角形。
如水青冈(\textit{Fagus longipetiolata})。

\subsection{桦木科(Betulaceae)}
乔木或灌木。
单叶互生,有托叶,具重锯齿,羽状叶脉直通齿尖。
单性花,雌雄同株,柔荑花序下垂。
花被片零至六,雄蕊一至六。
子房下位,二室,每室两胚珠,仅一胚珠发育。
一层珠被。
瘦果、坚果或具两翅的翅果。
如白桦(\textit{Betula platyphylla}),榛(\textit{Corylus heterophylla})。

\section{荨麻目(Urticales)}
草本或木本。
叶多互生,常有托叶。
花两性或单性,辐射对称,单被或无被。
雄蕊与花被对生。
子房上位,二心皮,一或二室,胚珠一或二个。
风媒或专性虫媒。

\subsection{榆科(Ulmaceae)}
草本或木本。
单叶互生,叶基偏斜,托叶早落。
花小,单生、簇生、短聚伞花序或总状花序。
雌雄同株。
花单被,萼片状,四至八裂。
雄蕊四至八,与花被对生,花丝直立。
花粉粒球形,二至五个萌发孔,外壁有脑纹或颗粒纹饰。
雄花有退化的雌蕊。
子房上位,二心皮,一或二室,每室一胚珠。
双珠被,花柱二,柱头头状。
翅果、坚果或核果,无胚乳。

\subsubsection{榆(\textit{Ulmus})}
乔木。
树皮有黏液,芽鳞片多,叶有重锯齿。羽状叶脉直通叶缘。
花两性。
翅果,果核扁平,子叶扁平。

\subsubsection{朴(\textit{Celtis})}
落叶。
叶基三出脉,侧脉不直达边缘。
核果,近球形。

\subsection{桑科(Moraceae)}
木本或草本,常有乳汁,有钟乳体。
叶多互生,托叶明显,早落。
花单性,单被或无被,稍肉质,宿存。
雄蕊一,或与花被同数对生。
子房上位,一室,二心皮,其中一个常不发育。
花柱二,胚珠一,倒生或弯生。
胎座基生或顶生。
瘦果、坚果或浆果,常成聚花果。
胚弯曲。

\subsubsection{桑(\textit{Morus})}
乔木或灌木。
叶互生,穗状花序,花丝向内弯曲。
肉质花被内包子房。

\subsubsection{榕(\textit{Ficus})}
有乳汁。
托叶大,抱茎,脱落后在茎上留有环痕。
花单性,生于肉质花序托构成的隐头花序内壁。
花序托开口处有多个总苞片。
雄花花被一至六,雄蕊一至二。
雌花花柱长。
另有瘿花,不育,花柱短粗,柱头喇叭口状,子房内有瘿蜂卵。
瘿蜂在产卵过程中传粉。

\subsection{荨麻科(Urticaceae)}
草本或灌木。
表皮细胞有钟乳体,茎皮纤维丰富。
单叶,互生或对生,两侧常不对称,有托叶。
茎叶生刺毛。
花细小,多单性,聚伞花序,密集成头状。
雄花花被四或五,与花被对生,花丝伸直。
雌花花被二至五,子房一室,胚珠一,直立或基生。
瘦果或核果,有胚乳。

\subsubsection{荨麻(\textit{Urtica})}
草本,被刺毛。
刺毛前段中空尖锐,基部有多细胞腺体。
叶对生。
花被四,雄蕊四,柱头毛帚状。

\subsubsection{苎麻(\textit{Boehmeria})}
灌木或小乔木。
叶三出脉,有锯齿。
花排成团伞花序,或再排成穗状或圆锥花序。
雄花被三至五,雄蕊三至五,退化雄蕊球形或梨形。
雌花被管状,二至四齿裂。
子房内藏,一室,花柱柔弱。
瘦果包裹于花萼内。

\subsubsection{杜仲科(Eucommiaceae)}
落叶乔木,有乳汁。
单叶互生,边缘有锯齿,无托叶。
单性花,雌雄异株,先叶开放,无花被。
雄花簇生,雄蕊五至十,花丝短,花药四室。
雌花单生小枝下部,有苞片,二心皮合生。
子房有柄,顶端二裂,柱头位于裂口内侧,胚珠二。
果扁平,不开裂,翅果先端二裂。
胚乳多,子叶肉质,外果皮膜质。

\section{檀香目(Santalales)}
木本或草本,常寄生。
叶互生、对生或退化,无托叶。
花两性或单性,辐射对称。
雄蕊与花被对生。
子房下位,一至五室,中轴胎座。
有胚乳。

\subsection{桑寄生科(Loranthaceae)}
寄生灌木。
叶对生或轮生,或退化为鳞片无托叶。
花萼与子房合生。
雄蕊与花瓣对生。
子房下位,一室,胚珠一。
子房与花托杯联合为浆果或核果。
肉质花托与外果皮间有黏液层。
种子一,胚乳多。

\subsubsection{桑寄生(\textit{Loranthus})}
羽状叶脉,两性花。

\subsubsection{槲寄生(\textit{Viscum})}
叶三脉,或退化。
花单性,单被。

\section{蓼目(Polygonales)}
仅蓼科。

\subsection{蓼科(Polygonaceae)}
草本、灌木或藤本。
叶互生,托叶膜质,托叶鞘包茎。
花两性或单性,辐射对称,单被或异被。
花被三至六,覆瓦排列,结实时增大为膜质。
雄蕊三至六,花药二室,纵裂,花盘环状或鳞片状。
子房上位,一室,花柱二至四,基生胚珠一。
坚果或瘦果,三棱形或凸镜形。
胚乳发达。

\subsubsection{荞麦(\textit{Fagopyrum})}
叶三角形。
两性花,花被五。
雄蕊八,花柱三,柱头头状。
果卵形,三锐棱。

\subsubsection{蓼(\textit{Polygonum})}
草本或藤本,膜质托叶鞘。
花被三至五。
瘦果。
如何首乌(\textit{Polygonum multiflorum}),虎杖(\textit{Polygonum cuspidatum})。

\subsubsection{大黄(\textit{Rheum})}
草本。
根状茎粗,黄色。
叶掌状浅裂。
两性花,花被外轮反卷。
坚果有翅。

\section{中央子目(Centrospermae)}
多草本。
多两性花,辐射对称。
雄蕊一或二轮,与花被对生。
子房上位,胚珠弯曲。
中轴胎座至特立中央胎座。
胚弯曲,有外胚乳。

\subsection{商陆科(Phytolaccaeae)}
草本或木本。
总状花序或聚伞花序。
两性花,辐射对称,花被四至五。
雄蕊三至多。
子房上位,心皮一至多,离生或合生,每室胚珠一个。
浆果或坚果,鲜有蒴果,有外胚乳。

\subsection{石竹科(Caryophyllaceae)}
多草本,茎节膨大,单叶对生,无托叶。
两性花,辐射对称,单生或二歧聚伞花序。
花萼分离或连成管状,四或五裂,膜质边缘,花瓣与萼片同数。
雄蕊五至十,或一至二。
花药二室,纵裂。
子房上位,一室,特立中央胎座,下半部为中轴胎座。
花柱连合或离生,胚珠多至一。
蒴果顶端齿裂或瓣裂。
胚弯曲包围外胚乳。

\subsubsection{石竹(\textit{Dianthus})}
一年或多年草本。
花萼脉清楚,下有苞片。
花瓣五,有爪,先端全缘,有齿或细裂。

\subsubsection{繁缕(\textit{Stellaria})}
花瓣五,二裂,雄蕊十,心皮三。

\subsection{藜科(Chenopodiaceae)}
常草本,泡状毛破裂干萎后成粉状或皮屑状。
单叶互生,肉质,无托叶。
花小,单被。绿色,辐射对称,集于叶腋或成穗状、圆锥花序。
花萼三至五裂,无花瓣。
雄蕊与萼片对生,花丝分离或基部连合。
花药二室纵裂。
子房上位或陷入萼基,一室,心皮二至五,花柱一至三,基生胚珠一。
胞果位于花萼或花苞内,不开裂。
种子扁平,胚弯曲包围外胚乳。
如梭梭(\textit{Haloxylon ammodendron}),猪毛菜(\textit{Salsola collina})。

\subsection{苋科(Amaranthaceae)}
萼片干膜质,有颜色,花丝基部连合。
蒴果,盖裂,胚弯曲包围外胚乳。
如鸡冠花(\textit{Celosia cristata})。

\section{木兰目(Magnoliales)}
木本。
花单生,花托发达,两性。
花各部螺旋排列或轮状排列。
雄蕊多,花粉粒单沟至三沟。
心皮一至多,离生。
胚小,胚乳丰富。

\subsection{木兰科(Magnoliaceae)}
木本,有香气。
单叶互生,全缘,托叶大。
种子大多一枚。
包被幼芽,早落,留下环痕。
两性花,大,单生于枝顶或叶腋,外有非绿色大型苞片。
花柄上或有绿叶状苞片。
花被多轮。
雄蕊多数,分离,螺旋排列于花托下半,花丝短。
花药长,二室,纵裂,药隔突出。
花粉粒舟状,单沟。
心皮多,螺旋排列于花柄延长形成的轴上,离生。
聚合蓇葖果,少数为带翅坚果。
种子悬挂于丝状珠柄维管束。
胚乳多,嚼烂状。

\subsubsection{木莲(\textit{Manglietia})}
多为大乔木。
有托叶痕,花顶生。
胚珠四至十四。

\subsubsection{木兰(\textit{Magnolia})}
花顶生,花被多轮。
心皮分离,胚珠二。
如玉兰(\textit{Magnolia denudata})。

\subsubsection{含笑(\textit{Michelia})}
花腋生。
结实时雌蕊轴伸长。
胚珠二至多。

\subsubsection{观光木(\textit{Tsoongiodendron})}
乔木,花腋生。
结果时心皮连合。
胚珠八至十。
聚合果硬木质。

\subsubsection{鹅掌楸(\textit{Liriodendron})}
叶分裂,先端截形,柄长,托叶二。
单花顶生,杯状,黄绿色。
萼片三,花瓣六。
花药向外开裂。
翅果不开裂。

\subsection{单心木兰科(Degeneriaceae)}
仅单心木兰(\textit{Degeneria vitiensis})。
乔木,有香气。
单叶互生,无托叶。
花腋生,柄长,两性,花被分化,花下有小苞片。
花萼三,花瓣十二至十八,排成三至五轮。
雄蕊多,三至四轮,无花丝。
花药四室,纵裂,嵌入侧脉与中脉之间。
发育雄蕊内侧有退化雄蕊。
心皮一,无花柱,心皮腹线边缘反折成柱头。
胚珠多。
果不开裂,种子有垂丝。

\subsection{蕃荔枝科(Annonaceae)}
乔木、灌木或藤本,常绿或落叶。
单叶互生,全缘,无托叶。
花香,单生或簇生叶腋,辐射对称,两性。
花被略分化,外轮萼状,内二轮瓣状,每轮三枚。
雄蕊多,螺旋排列,药隔突出。
心皮一至多,离生。
蓇葖果或聚合浆果,有长柄。
胚小,有假种皮,富嚼烂状胚乳。

\subsection{五味子科(Schisandraceae)}
木质藤本,含精油细胞。
花单性,雌雄异株。
花托肉质近球形。
心皮连合成肉质球。
浆果。

\subsection{八角茴香科(Illiciaceae)}
木本,有香气。
叶互生,无托叶。
花单生或聚生,两性。
花被七至三十三,多轮,有腺体。
雄蕊七至二十一,螺旋排列。
心皮八至二十,一轮。
胚珠一。
聚合蓇葖果。
胚小,油质胚乳发达。
仅八角(\textit{Illicium})。

\subsection{樟科(Lauraceae)}
木本,仅无根藤(\textit{Cassytha})为无叶寄生藤本。
叶和树皮有油细胞或黏液细胞。
单叶互生,全缘,三出或羽状脉,叶被有灰白色粉,无托叶。
两性花,辐射对称,圆锥、总状或丛生花序,腋生或近顶生。
花各部轮生。
花被留至四,不分化,两轮。
花被、花丝、花托合生为被丝托。
雄蕊九,每轮三枚,第四轮雄蕊退化。
第一、第二轮雄蕊花药向内,第三轮花药向外。
花药基部有腺体。
花药二至四室,瓣裂,花粉无萌发孔。
子房上位,三心皮,两枚前侧方心皮退化。
花柱单生,柱头二或三裂。
胚珠单生,悬垂。
浆果或核果,胚乳退化。

\subsubsection{润楠(\textit{Machilus})}
羽状叶脉。
结果时被丝托不增粗,花被向外反卷。
如泡花楠(\textit{Machilus pauhoi})。

\subsubsection{楠木(\textit{Phoebe})}
结果时花被裂片伸长增厚,成革质或木质,包裹果实基部。

\subsubsection{樟(\textit{Cinnamomum})}
叶互生或对生,三出脉或羽状脉。
无总苞,被丝托短。
结果时花被脱落。

\subsubsection{厚壳桂(\textit{Cryptocarya})}
羽状或三出叶脉。
两性花。
被丝托陀螺形或卵形,结果时增大,包住果实。
发育雄蕊九,花药二室。

\subsubsection{檫木(\textit{Sassafras})}
落叶乔木。
叶互生,聚于枝顶。
叶尖二或三裂。
单性花,雌雄异株,有异性退化痕迹。
总状花序,下有互生总苞片。
被丝托浅杯状,结果时增大。

\subsubsection{油丹(\textit{Alseodaphne})}
两性花。
被丝托肉质膨大。

\subsubsection{木姜子(\textit{Litsea})}
羽状叶脉。
花单性异株。
花序下有对生总苞。
如山苍子(\textit{Litsea cubeba})。

\subsubsection{无根藤(\textit{Cassytha})}
仅无根藤(\textit{Cassytha filiformis})。
寄生草质藤本,以盘状吸根侵入寄主。
叶鳞状或退化。
两性花,穗状花序。
被丝托包裹果实。

\subsection{水青树科(Tetracentraceae)}
仅水青树(\textit{Tetracentron sinensis})。
落叶乔木。
掌状叶脉,有锯齿,无托叶。
两性花小,穗状花序。
花被四。
雄蕊与花被对生。
心皮四,与雄蕊互生。
花柱四,直立于花蕾中央,后向外弯曲,最后位于蓇葖果外侧基部。
胚珠四,倒生。

\subsection{莽草科(Winteraceae)}
芳香,无托叶,次生木质部仅有管胞。
花药二室,外向。
花粉粒单孔。
浆果,胚珠多。
如林仙(\textit{Drimys})。

\subsection{腊梅科(Calycanthaceae)}
灌木,树皮芳香。
叶对生,无托叶。
花单生,花托壶状、坛状。
聚合瘦果。

\subsection{昆栏树科(Trochodendraceae)}
仅昆栏树(\textit{Trochodendron aralioides})。
常绿乔木,顶芽大,单叶互生,无托叶。
木质部仅有管胞。
两性花,顶生穗状或二歧聚伞花序,雌雄蕊轮生。
花丝细长,心皮一轮,胚珠多,蓇葖果。

\subsection{领春木科(Eupteleaceae)}
落叶乔木。
芽侧生,外包近鞘状叶柄基部。
单叶互生。
两性花,单生苞片内。
先叶植物,无花被。
心皮有长柄。
簇生聚合翅果。
仅领春木(\textit{Euptelea pleiosperma})和日本领春木(\textit{Euptelea polyandra})。

\subsection{连香树科(Cercidiphyllaceae)}
落叶乔木,叶对生,托叶早落。
花单性异株。
雄花无梗,花丝细长。
雌花有梗,腋生。
聚合蒴果,种子有翅。
仅连香树(\textit{Cercidiphyllum})。

\section{毛茛目(Ranales)}
多草本或木质藤本,含阿朴啡类生物碱。
雄蕊多数,分离,螺旋排列;或与花瓣对生。
心皮多,离生,螺旋排列或轮生。
胚直,胚乳丰富。

\subsection{毛茛科(Ranunculaceae)}
草本。
叶互生或基生,掌状或羽状分裂;或为一至多回小叶或羽状复叶。
两性花。
萼片五,绿色,花瓣状。
花瓣五。
雄蕊多,花药四至二室。
心皮多,离生。
胚珠多。
瘦果或蓇葖果。
种子有胚乳。

\subsubsection{铁线莲(\textit{Clematis})}
攀缘灌木。
羽状复叶对生,无托叶。
花序腋生或顶生。
萼片四或六至八,一般白色。
花瓣缺,雄蕊多,心皮多,羽毛状花柱。
瘦果。
如威灵仙(\textit{Clematis chinensis})。

\subsubsection{侧金盏花(\textit{Adonis})}
叶细裂。
萼片五至八,花瓣状。
花瓣五至十六,黄或红。
心皮多,胚珠下垂,瘦果。

\subsubsection{毛茛(\textit{Ranunculus})}
直立草本。
花白或黄,萼片五,花瓣五,基部常有蜜腺。
雄蕊和心皮均离生多数。
瘦果密集于花托。

\subsubsection{翠雀(\textit{Delphinium})}
草本,叶掌状分裂。
总状或圆锥花序,白色或紫色,萼片五,后面有两个长距。
花瓣小,二至四,后面有两个长距,套在萼片长距内。
心皮三至七,蓇葖果,多种子。
如大花飞燕草(\textit{Delphinium grandiflorum})。

\subsection{睡莲科(Nymphaeaceae)}
水生草本。
叶盾形或心形,有长柄,浮水。
两性花,离生萼四至六。
花瓣多,下位或周位。
雄蕊多。
心皮多,藏于肥大花托。
胚珠多。
如莲(\textit{Nelumbo nucifera}),莼菜(\textit{Brasenia schreberi}),芡(\textit{Euryale ferox})。

\subsection{金鱼藻科(Ceratophyllaceae)}
仅金鱼藻(\textit{Ceratophyllum})。
沉水草本。
茎纤细分枝。
叶轮生,劈裂为二叉裂片。
单性花小,单被。
雄蕊八至二十,花丝细,花药外向,药隔伸出。
子房一室,胚珠一,单层珠被。
坚果,胚大,无胚乳。

\section{胡椒目(Piperales)}
草本或木本。
草本茎有散生维管束。
叶对生或互生,有托叶。
无花被。
子房上位,心皮合生,侧膜胎座至半中轴胎座。
胚小,有胚乳。

\subsection{胡椒科(Piperaceae)}
木质或草质藤本,或肉质小草本。
藤本种类节膨大,生不定根。
叶互生、对生或轮生,有辛辣气,基部两侧不等,离基三出脉,托叶与叶柄合生。
穗状或肉穗花序。
单性花,小,雌雄异株。
有苞片,无花被。
雄蕊一至十,心皮一至四。
子房上位,一室,直立胚珠一,珠被一或二层。
浆果。
维管束散生,导管细小。
如胡椒(\textit{Piper nigrum})。

\section{马兜铃目(Aristolochiales)}
藤本,或为寄生植物。
单叶互生。
花辐射对称。
花被一轮,花瓣状,肉质。
子房下位,胚珠多数,珠被二层。
有胚乳。
蒴果或浆果。

\subsection{马兜铃科(Aristolochiaceae)}
多年生草本或藤本。
根苦辣,有香气。
叶互生,被灰白粉,基部心形,无托叶。
花单生两性。
花被单,花瓣状,合生管状或坛状,三裂或向一侧延长,紫色,臭气。
雄蕊六至多。
子房下位,四至六室。
蒴果。
如马兜铃(\textit{Aristolochia debilis}),细辛(\textit{Asarum heterotropoides})。

\subsection{大花草科(Rafflesiaceae)}
肉质寄生草本。如大花草(\textit{Rafflesia arnoldii})。

\section{藤黄目(Guttiferales)}
木本。
单叶互生或对生。
两性花,辐射对称,异被,覆瓦或螺旋排列。
雄蕊多。
子房上位,中轴或侧膜胎座。
胚珠多,珠被两层。
有胚乳。

\subsection{芍药科(Paeoniaceae)}
仅芍药(\textit{Paeonia})。
多年生草本或灌木。
叶互生长柄,掌状或羽状复叶,全缘。
花辐射对称,两性,单生或总状花序。
萼片五,花瓣五至十。
雄蕊多,离心发育,花丝细长。
心皮一至五,分离,革质,基部有蜜腺联合成的花盘包裹子房。
柱头大,花柱短,双珠被。
蓇葖果,种子有珠柄发育来的假种皮。
胚小,胚乳多。
如牡丹(\textit{Paeonia suffruticosa}),芍药(\textit{Paeonia lactiflora})。

\subsection{猕猴桃科(Actinidiaceae)}
木质藤本,髓实心或片层状。
单叶互生,常有锯齿,被粗毛或星状毛,羽状脉。
聚伞花序。
萼片五,覆瓦状。
花瓣五,雄蕊多。
花药丁字形着生,纵裂或顶裂。
子房上位,五室以上,花柱五。
浆果,被毛,胚乳多。

\subsubsection{猕猴桃(\textit{actinidia})}
常绿,少落叶。
萼片五至二,花瓣五至十二。

\subsection{龙脑香科(Dipterocarpaceae)}
木本,含树脂。
叶革质全缘,有托叶。
圆锥花序,腋生。
花两性,辐射对称,萼管五裂片,花瓣五,雄蕊多。
心皮三,下部合生。
坚果有翅,无胚乳。
如龙脑香(\textit{Dipterocarpus})。

\subsection{山茶科(Theaceae)}
木本,常绿或落叶。
茎叶中有石质细胞。
单叶互生,有胼胝质状锯齿,无托叶。
花两性,辐射对称,单生,或簇生于叶腋,有苞片。
萼片五至六,花瓣五至六。
雄蕊多,多轮。
心皮三至五,子房上位,中轴胎座。
蒴果、核果或浆果。

\subsubsection{山茶(\textit{Camellia})}
常绿乔木或灌木。
苞片与萼合生。
花红色、白色或黄色。
蒴果木质,背开裂。
胚大,无胚乳。
子叶半球形,富油脂。
如油茶(\textit{Camellia oleifera}),茶(\textit{Camellia sinensis}),普洱茶(\textit{Camellia assamica})。

\subsubsection{石笔木(\textit{Tutcheria})}
常绿乔木。
蒴果,种皮厚,子叶折叠状,无胚乳。

\subsubsection{木荷(\textit{Schima})}
乔木。
苞片二至八,脱落。
萼片五或六,宿存。
花瓣五。
子房五室。
蒴果扁球形,顶端平或微凹,中轴棒槌状。
种子有周翅,有胚乳。

\subsubsection{紫茎(\textit{Stewartia})}
有鳞芽。
叶柄不对折。
种子周翅至无翅。

\subsubsection{柃(\textit{Eurya})}
雌雄异株。
花细小,柄短,簇生叶腋。
雄花中有退化雌蕊。
苞片二,萼片五,花瓣五。
子房三至五室。
浆果,种子细小,胚乳多。
如米碎花(\textit{Eurya chinensis})。

\section{罂粟目(Papaverales)}


% 蓇葖果



\end{sloppypar}
\end{document}
