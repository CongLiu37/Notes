\documentclass[11pt]{article}

\usepackage[UTF8]{ctex} % for Chinese 

\usepackage{setspace}
\usepackage[colorlinks,linkcolor=blue,anchorcolor=red,citecolor=black]{hyperref}
\usepackage{lineno}
\usepackage{booktabs}
\usepackage{graphicx}
\usepackage{float}
\usepackage{floatrow}
\usepackage{subfigure}
\usepackage{caption}
\usepackage{subcaption}
\usepackage{geometry}
\usepackage{multirow}
\usepackage{longtable}
\usepackage{lscape}
\usepackage{booktabs}
\usepackage{natbibspacing}
\usepackage[toc,page]{appendix}
\usepackage{makecell}
\usepackage{amsfonts}
 \usepackage{amsmath}
\usepackage[utf8]{inputenc}
\usepackage{amssymb}
\usepackage{amsthm}
\usepackage{enumerate}
\usepackage{comment}

\usepackage[backend=bibtex,style=authoryear,sorting=nyt,maxnames=1]{biblatex}
\bibliography{} % Reference bib

\title{单子叶植物(Monocotyledineae)}
\author{}
\date{}

\linespread{1.5}
\geometry{left=2cm,right=2cm,top=2cm,bottom=2cm}

\setlength\bibitemsep{0pt}

\begin{document}
\begin{sloppypar}
  \maketitle

  \linenumbers
草本,鲜有木本。
须根,散生中柱,无永久形成层。
叶脉多平行脉或弧形脉,鲜有网状。
花三基数。
胚有一顶生子叶。

\section{沼生目(Helobieae)}
草本,水生或沼生。
花的各部分轮生或半轮生,心皮离生。
无胚乳,花粉传播时为三细胞。

\subsection{泽泻科(Alismataceae)}
球茎。
叶基生,基部鞘状。
两性或单性花,辐射对称,轮生,总状或圆锥花序。
花被片六,两轮。
外轮花被萼片状,宿存;内轮花瓣状,脱落。
雄蕊六至多,分离,生于扁平花托或螺旋生于球形花托。
花柱宿存,胚珠倒生。
瘦果。

\subsubsection{泽泻(\textit{Alisma})}
叶椭圆或卵圆,圆锥花序,两性花,花托扁平,雄蕊六,心皮一环。

\subsubsection{慈菇(\textit{Sagittaria})}
沉水叶带状,漂浮叶椭圆,气生叶箭形。
总状花序,单性,雄蕊多,心皮多轮生于球形花托。

\subsubsection{毛茛泽泻(\textit{Ranalisma})}
叶有羽状脉,花葶直立,上生花一至三朵。
两性花,花托突出成球形。
雄蕊九,心皮多,聚合瘦果。

\subsection{花蔺科(Butomaceae)}
有乳汁。
叶基生,条形或椭圆。
两性花,辐射对称,伞形花序,有苞片。
外轮花被萼状,三枚,宿存。
内轮花被花瓣状,三枚,早落。
雄蕊九至多,外轮或不育。
花丝长,基部宽。
心皮六至多,离生或基部合生。
子房一室,胚珠多,散生于网状分枝多侧膜胎座上。
蓇葖果,腹缝开裂,花柱宿存。
种子多而细小。

\subsection{水鳖科(Hydrocharitaceae)}
浮水或沉水草本,淡水或咸水生。
叶互生、对生或轮生。
花单生、成对或成花序,常有佛焰状苞片或为两个对生苞片包裹,多单性,辐射对称。
雄花多伞形花序,雌花单生。
花被一或二轮,每轮三片,外轮萼片状,内轮花瓣状。
雄蕊多,向心发育。
花药外向,线形或椭圆形。
花粉球形,无孔沟或单沟。
子房下位,心皮三至六。
花柱与心皮同数,或有二裂。
三至六个侧膜胎座,胚珠多。
果实肉质,开裂。

\section{百合目(Liliiflorae)}
多草本,有地下茎。
心皮合生,子房三室或一室。
蒴果或浆果,种子有内胚乳。
花粉粒有两细胞。

\subsection{百合科(Liliaceae)}
草本,根茎、鳞茎或块茎。
单叶互生,或退化为鳞片状。
总状花序。
两性花,辐射对称,花被六,雄蕊六。
花药二室,基生或丁字着生,直裂或孔裂。
子房上位,三室,中轴胎座。
蒴果或浆果,胚乳肉质。

\subsubsection{天门冬(\textit{Asparagus})}
叶成鳞片状,枝条为小而狭长的绿色叶状枝,花火花序生于叶状枝腋内。

\subsubsection{菝葜(\textit{Smilax})}
攀缘灌木,块状根茎,茎上有刺。
叶互生,有掌状脉和网状小脉。
叶柄两侧有卷须。
单性花,雌雄异株,腋生伞形花序。
浆果。
如菝葜(\textit{Smilax china}),土茯苓(\textit{Smilax glabra})。

\subsubsection{黄精(\textit{Polygonatum})}
根茎圆柱形,有节和疤痕。
叶互生、对生或轮生。
叶顶端有卷须。
花腋生,单生或伞形花序。
花被合生管状,无副花冠。

\subsubsection{土麦冬(\textit{Liriope})}
多年生簇生草本。
根茎和块根短厚。
叶狭。
花白或紫,直立,总状花序。
子房上位。
果肉质,开裂。

\subsubsection{沿阶草(\textit{Ophiopogon})}
花下垂,子房半下位。

\subsubsection{黄花菜(\textit{Hemerocallis})}
根茎短,叶基生狭长。
花大,花被基部合生成漏斗状,黄色或橘黄色。
雄蕊六,花药背部着生,蒴果。

\subsubsection{知母(\textit{Anemarrhena})}
仅知母(\textit{Anemarrhena asphodeloides})。
根茎短,条形叶根生。
花小,间断总状花序。
花被六,宿存。
雄蕊三,背着药,蒴果。

\subsubsection{蜘蛛抱蛋(\textit{Aspidistra})}
草本植物,花单生,花梗从根状茎的鳞片腋内抽出。
花被合生,柱头盾状。

\subsubsection{葱(\textit{Allium})}
多年生草本,有辛辣气味。
鳞茎,叶鞘封闭。
花葶空心,伞形花序,外被总苞。
总苞一侧开裂或裂成数片。
花被分离或仅基部合生,子房上位。
如葱(\textit{Allium fistulosum}),韭菜(\textit{Allium tuberosum}),洋葱(\textit{Allium cepa}),蒜(\textit{Allium sativum})。

\subsubsection{藜芦(\textit{Veratrum})}
鳞茎不膨大,基部残存叶鞘撕裂成纤维状。
圆锥花序,被毛,花被六。
花药肾形,一室,横裂。
花柱三,宿存。
蒴果。

\subsubsection{百合(\textit{Lilium})}
鳞茎无被,鳞瓣肥厚。
叶椭圆至条形。平行叶脉。
花大,单生或总状花序。
花被合生成漏斗状,花药丁字着生。

\subsubsection{贝母(\textit{Fritillaria})}
有鳞茎。
叶互生或轮生。
花有苞片,垂下,单生或伞形、总状花序。
花被基部合生成钟状或漏斗状,裂片不反转,基部有腺穴。
花药基部着生,蒴果。

\subsubsection{郁金香(\textit{Tulipa})}
多年生草本,鳞茎有膜状或纤维状外被。
叶基生,花葶单生直立。
花被六,分离。
雄蕊六,内藏,无蜜腺。
花柱短或缺。
蒴果室背开裂。
种子多,扁平或有狭翅。

\subsection{石蒜科(Amaryllidaceae)}
草本,有鳞茎。
叶基生,细长,全缘。
花鲜艳,两性,单生或顶生伞形花序,佛焰状总苞。
花被瓣状,六枚,基部或合生成筒。
雄蕊六,二轮,花丝基部合生成筒,或花丝间有鳞片。
子房上位或下位,三室。
蒴果,有胚乳。

\subsubsection{石蒜(\textit{Lycoris})}
有被鳞茎,叶带状或条状。
花葶是心,花被基部合生成漏斗状,无副花冠。
花丝分离,花丝间有鳞片。
子房下位。

\subsubsection{水仙(\textit{Narcissus})}
卵圆形鳞茎,叶带状直立。
花葶中空,花被高脚碟状,筒部三棱,副花冠杯状,蒴果。

\subsection{薯蓣科(Dioscoreaceae)}
草质缠绕藤本,块茎或根茎。
叶互生或中部以上对生,叶腋内有珠芽。
单叶或指状、掌状复叶。
基出掌状叶脉,有网脉。
单性花,辐射对称,同株或异株,穗状、总状或圆锥花序。
花被六,离生,二轮。
雄蕊六,雌花有退化雄蕊。
子房下位,三室。
蒴果有翅,三瓣裂,种子有翅。
如山药(\textit{Dioscorea opposita})。

\section{灯芯草目(Juncales)}
草本或灌木。
叶有鞘,或叶片退化。
花小,两性,辐射对称。
花被片六,二轮,革质或干膜质。
雄蕊三至六,子房上位,蒴果。

\subsection{灯芯草科(Juncaceae)}
草本,常有根茎,茎多簇生。
叶扁平、圆柱状、披针形、条形、毛发状或芒刺状。
叶鞘开放或闭锁,常有叶耳。
花顶生或假侧生,小,两性,辐射对称,或有二先出叶。
雄蕊六,内轮三枚或退化。
花药二室,子房上位,一或三室。
蒴果,三瓣裂。

\subsubsection{地杨梅(\textit{Luzula})}
叶鞘闭合,叶边缘有毛,花有先出叶,蒴果内三枚种子。

\subsubsection{灯芯草(\textit{Juncus})}
叶鞘开放,叶边无毛,花或无先出叶,蒴果内种子多。

\section{鸭跖草目(Commelinales)}
多陆生,叶鞘闭合。
两性花,单生或圆锥、聚伞花序。
花被片六,外轮三枚萼状,内轮三枚有爪。
花药纵裂或孔裂。
子房上位,蒴果,种子有胚乳。

\subsection{鸭跖草科(Commelinaceae)}
草本,茎有明显的节和节间。
叶互生,有鞘。
两性花,辐射对称,蝎尾状聚伞花序,或缩短成头状,或伸长集成圆锥状,或单生。
花被片六,二轮,分生,或中部合生而两端分离。
雄蕊六,花丝有念珠状长毛。
子房上位,二至三室,胚珠直生。
蒴果,胚乳多,种脐背面或侧面有盘状胚盖。

\subsubsection{鸭跖草(\textit{Commelina})}
匍匐或直立草本,总苞片佛焰苞状,聚伞花序,花瓣离生,能育雄蕊三。

\subsubsection{水竹叶(\textit{Murdannia})}
匍匐或直立草本,总苞不成佛焰苞状,圆锥或聚伞花序,花瓣离生,三枚能育雄蕊与萼片对生,退化雄蕊不裂或成三小体。

\section{禾本目(Graminales)}
仅禾本科(Gramineae)。

\subsection{禾本科(Gramineae)}
草本或木本。
须根,有根茎。
地上茎特称杆,节明显,实心。
单叶互生,二列,有叶片和叶鞘。
叶片狭长,纵向平行脉,或有横向小脉。
叶鞘包杆、开放或闭合。
叶片与叶鞘连接处常有膜质或纤毛状叶舌,外侧叶颈稍厚,两侧有突起状或纤毛状叶耳。
花小,多两性。
花被二或三,特化为透明肉质鳞片,称为浆片。
雄蕊三,花药丁字着生。
子房上位,一室,倒生胚珠一,花柱二,柱头羽毛状。
小花几乎无柄,生于苞片(外稃)和小苞片(内稃)间,一至多数,两列着生于小穗轴(花轴)上,其基部有两枚不育苞片(颖),构成小穗。
小穗中的花火无雌雄蕊,或雌雄蕊不育形成中性小穗,或颖腋内生芽。
小穗成对生于穗轴各节。
小穗常两侧或背腹压扁,基部有时增厚为基盘。
小穗轴延伸,或颖上方小穗节间或颖下方有关节,使得小穗成熟时脱节于颖上或颖下,颖宿存或脱落。
花序以小穗为基本单位,排成圆锥、总状、穗状或头状花序。
小穗和花序基部有时有总苞。
多为颖果,粉质胚乳发达,胚小。

\subsubsection{竹亚科(Bambusoideae)}
杆木质化,灌木、乔木或藤本。
地下茎细长或粗短。
杆节间中空,圆柱或四方、扁圆。
杆节隆起。
杆生叶特化为杆箨,分为箨鞘和箨叶。
箨鞘抱杆,厚革质,外侧有刺毛,内侧光滑,鞘口有毛,与箨叶连接处有箨舌和箨耳。
箨叶缩小,无明显主脉,直立或反折。
枝生叶有明显的中脉和小横脉,有柄,与叶鞘连接处有关节。

\par

簕竹(\textit{Bambusa})地下茎短粗合轴丛生,杆节间圆筒形,每节分枝多。
箨叶直立,基部与箨鞘顶端等宽,箨耳明显。
如凤尾竹(\textit{Bambusa multiplex} var. \textit{nana})。

\par

慈竹(\textit{Dendrocalamus})地下茎短粗合轴丛生,杆节间圆筒形。
幼时杆顶细长弯垂。
箨鞘顶端截平,一至二倍宽于箨叶基部。
箨叶反转,箨耳不发达。
小穗轴极短,不易折。
小穗古铜或紫棕色。
外稃顶端不成芒状。
如麻竹(\textit{Dendrocalamus latiflorus})。

\par

毛竹(\textit{Phyllostachys})地下茎细长单轴散生。
杆节分枝二枚。
箨鞘顶端渐窄,箨叶狭长皱缩。
小穗丛间夹杂许多顶端有缩小叶片的苞片。

\subsubsection{禾亚科(Agrostidooideae)}
草本。
杆为草质或木质。
杆生叶有明显中脉,无叶柄,不易从叶鞘上脱落。

\par

稻(\textit{Oryza})小穗两性,两侧压扁有脊,含三小花。
下方二小花退化,仅存极小的外稃,位于顶生两性小花下。
颖退化,在小穗柄顶端呈半月状痕迹。
两性小花外稃或有芒,内稃三脉。
如水稻(\textit{Oryza sativa}),有两个亚种:籼稻(\textit{Oryza sativa} subsp. \textit{indica})米粒细长不黏;粳稻(\textit{Oryza sativa} subsp. \textit{japonica})米粒短圆黏糯。

\par

芦苇(\textit{Phragmites})大型禾草,根茎粗壮。
圆锥花序,小穗有四至七小花,脱节于颖上。
外稃无毛,基盘延长,有丝状绒毛。

\par

小麦(\textit{Triticum})一年生或越年生草本。
穗状花序。
小穗两侧压扁,常单生于轴的各节,成熟时不自基部脱落,穗轴不逐节断落。
颖卵形,背部有脊,三至数脉。
小穗含三至九小花,上部小花不结实。

\par

黑麦(\textit{Secale})越年生草本。
穗状花序。
小穗两侧压扁,常单生于穗轴个节,二小花。
颖锥状,仅一脉。

\par

大麦(\textit{Hordeum})多年或越年生草本,穗状花序。
小穗两侧压扁,三枚生于同一节,各含一花。
颖基本宽或针状。

\par

燕麦(\textit{Avena})一年生草本,圆锥花序开展。
小穗两侧压扁,下垂。含二至数朵小花,下部小花两性。
颖等长,七至十一脉,外稃有芒。

\par

䅟(\textit{Eleusine})一年生草本,丛生。
穗状花序,指状排列于枝顶。
小穗无柄,两侧压扁,覆瓦状紧密排列于较宽扁的穗轴一侧,含三至六小花。
囊果,种子有皱纹。

\par

黍(\textit{Panicum})一年或多年生草本。
圆锥花序,开展。
小穗背腹压扁,两性,疏生,脱节于颖下。
内颖等长或稍短于小穗。
外稃无芒,基部无附属物,无凹痕。

\par

狗尾草(\textit{Setaria})多年或一年生草本。
圆锥花序,柱状。
小穗背腹压扁,两性,含两小花,有宿存刚毛状不育小枝。
小穗脱节于颖下。
如小米(\textit{Setaria italica})。

\par

甘蔗(\textit{Saccharum})多年生草本,杆直立粗壮实心。
圆锥花序,银白色。
小穗两性,背腹压扁或略圆筒形,成对生于穗轴各节。
穗状有关节,各节连同着生其上的无柄小穗一起脱落。
小穗基盘、颖河小穗柄上的毛均长于小穗。

\par

高粱(\textit{Sorghum})一年或多年生草本,圆锥花序。
小穗两性,背腹压扁或略圆筒形,成对生于穗轴各节或顶生三枚。
无柄小穗结实,仅生一花。
基盘短而钝圆。
外颖下部革质,平滑有光泽。
有柄小穗不育。

\par

玉蜀黍(\textit{Zea})仅玉米(\textit{Zea mays})。
一年生,杆粗壮实心。
小穗单性,花序单性。
雄花为顶生圆锥花序。
雌花为腋生肉穗状花序,鞘苞多。

\section{棕榈亩(Palmales)}
仅棕榈科(Palmae)。

\subsection{棕榈科(Palmae)}
乔木或灌木,茎常不分枝,单生或丛生,直立或攀援,覆盖老叶柄基或有叶痕。
叶大,全缘或羽状、掌状分裂,芽摺叠聚生于茎顶。
叶柄基部扩大为富纤维的鞘。
花小,有苞片或小苞片,辐射对称,肉穗花序,有大型佛焰状总苞,生于叶丛中或叶鞘下。
花被片六,分离或合生,镊合或覆瓦排列。
雄蕊六,二轮,花药二室纵裂。
子房上位,一至三室。
花柱不发达,柱头三。
浆果、坚果或核果,外果皮纤维质,或盖覆瓦排列的鳞片。
种子与内果皮分离或黏合。
胚乳均匀或嚼烂状。

\subsubsection{蒲葵(\textit{Livistona})}
乔木。
叶掌状分裂,裂不超过二分之一,裂片先端渐尖且再裂为两片。
叶柄下有逆刺两列。
两性花,子房有三个近离生的心皮,三室。
核果橄榄状。

\subsubsection{棕竹(\textit{Rhapis})}

% 蓇葖果
\end{sloppypar}
\end{document}
