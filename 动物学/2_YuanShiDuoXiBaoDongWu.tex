\documentclass[11pt]{article}

\usepackage[UTF8]{ctex} % for Chinese 

\usepackage{setspace}
\usepackage[colorlinks,linkcolor=blue,anchorcolor=red,citecolor=black]{hyperref}
\usepackage{lineno}
\usepackage{booktabs}
\usepackage{graphicx}
\usepackage{float}
\usepackage{floatrow}
\usepackage{subfigure}
\usepackage{caption}
\usepackage{subcaption}
\usepackage{geometry}
\usepackage{multirow}
\usepackage{longtable}
\usepackage{lscape}
\usepackage{booktabs}
\usepackage{natbib}
\usepackage{natbibspacing}
\usepackage[toc,page]{appendix}
\usepackage{makecell}

\title{原始多细胞动物}
\date{}

\linespread{1.5}
\geometry{left=2cm,right=2cm,top=2cm,bottom=2cm}

\begin{document}

  \maketitle

  \linenumbers

\section{中生动物门(Mesozoa)}
中生动物代表单细胞的原生动物和多细胞的后生动物(Metazoa)之间的过渡。
中生动物呈蠕虫状,营寄生生活,虫体细胞数目恒定。
中生动物虫体外层是单层有纤毛的体细胞,这些细胞包围着内层的轴细胞。
体细胞具有营养功能,轴细胞则能形成生殖细胞。

\subsection{菱形虫纲(Rhombozoa)}
寄生于头足类肾脏。
体前细胞排成两圈,以附着宿主。
能行有性生殖和无性生殖。
体细胞大致螺旋排列。

\subsection{直泳虫纲(Orthonecta)}
多寄生于海洋无脊椎动物,大多雌雄异体。
体前端纤毛指向前,体后端纤毛指向后。
体细胞排列成环形。

\section{扁盘动物门(Placozoa)}
该门下仅丝盘虫(\textit{Trichoplar adhaerens})一个物种。
虫体表面的上皮细胞有鞭毛,背面细胞扁平,腹面细胞呈柱状,二者之间为实质组织,内有众多变形细胞。
丝盘虫虫体无对称性,无固定体形,无器官,无体腔,行出芽生殖或有性生殖,以其它原生动物为食。
扁盘动物的分类地位不确定。
  
\section{多孔动物门(Porifera)}
多孔动物又名海绵动物,体型一般不对称,无器官和明确的组织,营固着生活。
多孔动物的顶端为出水孔,体壁由两层上皮细胞和其间的中胶层组成。
体壁外侧由起保护作用的扁平细胞(pinacocyte)和孔细胞(porocyte)组成。
孔细胞为管状,贯穿体壁,水流经此进入体腔。
扁平细胞起保护作用,亦通过收缩和舒张调节孔细胞管道的开合。
中胶层为胶状物质,内含钙质、硅质的骨针(spicule)或(和)类蛋白质的海绵质纤维(spongin fiber)以及散在的变形细胞(amoebocyte)、成骨针细胞(scleroblast)、成海绵质细胞(spongioblast)、原细胞(archeocyte)和具神经传导作用的芒状细胞(collencyte)。
原细胞可负责消化食物或形成配子。
体壁内侧为领细胞(choanocyte)层。
领细胞有一透明领围绕一根鞭毛。鞭毛摆动,使水流通过体壁。
食物颗粒落在领上,而后进入细胞,形成食物泡,由领细胞或变形细胞进行消化。
部分淡水海绵细胞中还有伸缩泡。

\newline

水沟系(canal system)为多孔动物特有结构。
根据体壁中的水沟的分支程度从低到高的顺序,水沟系分为单沟型(ascon type)、双沟型(sycon type)和复沟型(leucon type)。

\newline

多孔动物的无性生殖有出芽和形成芽球两种形式。
出芽即母体体壁向外突出,形成芽体。
芽体可脱离母体,形成新个体。
芽球(gemmule)是中胶层的原细胞聚集成堆,外包几丁质膜和骨针形成的。
成体死亡后,大量芽球可以生存,待到环境适宜时发育为新个体。

\newline

多孔动物亦进行有性生殖,且胚胎发育过程中有独特的逆转(inversion)现象。
领细胞吞噬精子后变成变形虫状,将精子带入位于中胶层的卵子,形成合子。
合子卵裂为囊胚。
囊胚动物极细胞向囊胚腔内生出鞭毛,植物极则形成开口。
而后动物极细胞从植物极开口处翻转出来,鞭毛朝向囊胚表面,形成两囊幼虫(amphiblastula)。
幼虫随出水孔水流流出,具鞭毛的细胞内陷,形成体壁内层,而原植物极细胞形成体壁外层。
而其它多细胞生物的囊胚在发育过程中,多为植物极细胞内陷为内胚层,动物极细胞形成外胚层。
幼虫游动后不久营固着生活。

\newline

领细胞、骨针、水沟系、胚胎发育的逆转现象均为多孔动物所特有,故认为其是多细胞生物演化过程中的侧支。

\subsection{钙质海绵纲(Calcarea)}
钙质骨针,水沟系简单,体型小,多生活于浅海。

\subsubsection{同腔目(Homocoela)}
体壁薄,无褶皱,领细胞连续分布于中央腔,单沟型水沟系。
如白枝海绵(\textit{Leucosolenia})。

\subsubsection{异腔目(Heterocoela)}
体壁厚,有褶皱,领细胞位于鞭毛室内,双沟或复沟型。
如毛壶(\textit{Grantia})。

\subsection{六放海绵纲(Hexactinellida)}
骨针硅质、六放型,复沟型水沟系,体型较大,生活于深海。

\subsubsection{六放星目(Hexasterophora)}
骨针三轴六放型。
如偕老同穴(\textit{Euplectella})。

\subsubsection{双盘海绵目(Amphidiscophora)}
骨针双盘型,两端有钩。
如拂子介(\textit{Hyalonema})

\subsection{寻常海绵纲(Demospongiae)}
硅质骨针或海绵质纤维,复沟型水沟系,部分物种生活于淡水环境。

\subsubsection{胶海绵目(Myxospongide)}
无骨针,无海绵丝。
如糊海绵(\textit{Oscurella})。

\subsubsection{同骨海绵目(Carnosa)}
大小骨针相似。

\subsubsection{异骨海绵目(Choristida)}
有大小不等的骨针。

\subsubsection{韧海绵目(Hadromerina)}
有大骨针和星状小骨针,无海绵丝。
如穿贝海绵(\textit{Cliona})。

\subsubsection{软海绵目(Halichondrina)}
有一至二种大骨针,小骨针杆状,几乎无海绵丝。

\subsubsection{细芽海绵目(Poecilosclerina)}
大小骨针复杂。
如细芽海绵(\textit{Microcoina})。

\subsubsection{筒骨海绵目(Haplosclerina)}
一种大骨针,小骨针可能没有,有海绵丝。
如淡水的针海绵(\textit{Spongilla})
  
\end{document}