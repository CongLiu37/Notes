\documentclass[11pt]{article}

\usepackage[UTF8]{ctex} % for Chinese 

\usepackage{setspace}
\usepackage[colorlinks,linkcolor=blue,anchorcolor=red,citecolor=black]{hyperref}
\usepackage{lineno}
\usepackage{booktabs}
\usepackage{graphicx}
\usepackage{float}
\usepackage{floatrow}
\usepackage{subfigure}
\usepackage{caption}
\usepackage{subcaption}
\usepackage{geometry}
\usepackage{multirow}
\usepackage{longtable}
\usepackage{lscape}
\usepackage{booktabs}
\usepackage{natbib}
\usepackage{natbibspacing}
\usepackage[toc,page]{appendix}
\usepackage{makecell}

\title{线虫动物门(Nematoda)}
\date{}

\linespread{1.5}
\geometry{left=2cm,right=2cm,top=2cm,bottom=2cm}

\begin{document}

  \maketitle

  \linenumbers
线虫虫体呈圆柱形,体表无纤毛,头部不明显,体前端有辐射对称的口。
虫体表面有四条由下皮层向内加厚形成的线,在背面者为背线(dorsal cord),在腹者为腹线(ventral cord),在两侧者为侧线(lateral cord)。
部分物种体表有环纹,出现假分节现象。
线虫体表被角质膜(cuticle),系上皮细胞分泌物,起保护作用。
然而,角质膜限制虫体的生长,故线虫生长过程中需蜕皮(ecdysis)。

\newline

线虫体壁自外向内分别为角质膜、上皮细胞和源自中胚层的肌肉,但线虫只有纵肌,缺乏环肌。
体壁内为源自囊胚腔的假体腔(pseudocol)。
假体腔只有外壁源自中胚层,内壁为源自内胚层的肠上皮。
假体腔内部充满体腔液,司循环和支撑。

\newline

线虫有完整的消化道。
其口源于原肠胚胚孔,即原口。
与胚孔相对的另一侧开口发育为肛门。
线虫消化道从口至肛门依次为前肠、中肠、后肠。
前肠为原口处外胚层内陷形成,内壁有角质层,分化为口、口腔和咽。
中肠源自内胚层,司消化吸收。
后肠由外胚层内陷形成,内壁有角质层。

\newline

线虫的排泄器官起源于外胚层,是一种独特的原肾管结构,分为腺型(glandular type)和管型(tubular type)。
腺型排泄器官仅一到二个原肾细胞,位于咽后端腹面,开口于腹线。
管型排泄器官由一个原肾细胞特化形成,包括侧线下的两条纵排泄管和二管之间的横管,整个细胞呈“H”形,开口于腹线。
此外,代谢废物亦可由体壁和消化管排出。

\newline

线虫咽的周围有围咽神经环(circumenteric ring),向前、向后各分出六条神经。
向前的神经分布到体前的感觉器官。
向后的神经中,一条背神经,一条腹神经,两对侧神经。
侧神经离开围咽神经环后很快合并为一对。
这些神经中,腹神经最为发达。

\newline

线虫的感觉器官主要分布于头尾。
头部有唇、乳突、感觉毛和头感器。
唇和乳突为角质突起。
感觉毛实为特化的纤毛,司触觉。
头感器是体表的内陷物,司化学感受。
水生种类咽两侧有一对眼点,司视觉。
线虫尾部的感受器官为尾感器,开口于尾端两侧。

\newline

线虫动物大多雌雄异体异型,少数种类为雌雄同体或无雄性。
线虫生殖腺为盲管。
雄虫大多有一个精巢,精巢后接输精管,再向后为肌肉发达的射精管(ejaculatory duct),最终与后肠相接于泄殖腔(cloaca)。
射精管周围有前列腺(prostatic gland)。
大多数线虫雄性泄殖腔向外伸出两个囊,内有角质的交合刺(spicule)。
在交配时,交合刺伸出,撑开雌虫阴门。
雌虫大多有两个卵巢,后接输卵管和子宫。
两个子宫后端相接,经肌肉质阴道,开口于虫体中部腹线,形成阴门。
  
\section{无尾感器纲(Aphasmida)}
虫体尾端无尾感器,而有尾腺。
排泄器官腺型。
海产线虫全部属此纲,亦有寄生品种。

\subsection{刺嘴亚纲(Enoplia)}
前端细长,后端膨大,咽腺多。

\subsubsection{刺嘴目(Enoplida)}
头感器排成三圈。
第一圈为六个唇乳突,第二圈为六个感觉毛,第三圈为四个感觉毛。
咽长圆锥形,基部膨大。
五个咽腺,一个在背面,四个在腹面。
排泄器官为单细胞腺体。
三个尾腺。
如刺嘴虫(\textit{Enoplus})。

\subsubsection{单齿目(Monochida)}
头感器排成两圈。
第一圈为六个锥形突起,第二圈为十个锥形突起。
口角质化,有一个块状齿。
咽腺五个,排泄器官退化。
如单齿虫(\textit{Monochus})。

\subsubsection{矛线目(Dorylainida)}
口腔内有一可伸缩长矛刺。
咽前端细长肌肉质,后端膨大。
头部两圈突起,第一圈六个,第二圈十个。
如矛线虫(\textit{Dorylaimus})。

\subsubsection{毛首目(Trichocephalida)}
幼虫口内有伸缩毛刺,成虫消失。
体前有两个大细胞构成的咽。
无唇片,寄生鸟类或哺乳类,或以节肢动物为中间宿主。
如鞭虫(\textit{Trichuris trichiura})、旋毛虫(\textit{Trichinella spiralis})。

\subsubsection{索虫目(Mermithida)}
体细长如索。
成虫无口囊,有十六个头感器。
咽细长,肠特化为两行大营养细胞。
幼虫寄生无脊椎动物,成虫自由生活。
如索虫(\textit{Mermis})。

\subsection{色矛亚纲(Chromadoria)}
咽圆柱形,前后为球形。
咽腺三个,单细胞。

\subsubsection{色矛目(Chromadorida)}
有螺旋形化感器,前端有头感器。
口囊内有齿。
体表角质层有纹。
自由生活。
如色矛虫(\textit{Chromadora})。

\subsubsection{疏毛目(Araeolaimida)}
头感器三圈,第三圈为四个细长头毛。
口前端漏斗形,口囊内无齿。
体表有环纹。
多海产。
如\textit{Plectus}。

\subsubsection{带线虫目(Desmocolecida)}
体粗短,体表有鳞、毛、刺、瘤等。
体前端有色素小点或小眼。
口囊退化,多海产。
如链头线虫(\textit{Desmoscolex})。

\subsubsection{单宫目(Monohysterida)}
头端有分散刚毛,化感器环状。
体表可能有环纹,或四/八纵列刚毛。
单个卵巢。
如咽管线虫(\textit{Siphonolaimus})。

\section{尾感器纲(Phasmida)}
虫体尾端有一对尾感器,排泄器官管型,为陆生或淡水生。

\subsection{小杆亚目(Rhabditida)}
咽分三部分,末端球内常有瓣膜。
雄性交合囊发达。

\subsubsection{小杆目(Rhabditda)}
头感器乳突状,口囊长管状。
土壤生活或寄生脊椎动物。
如小杆线虫(\textit{Rhabditis})。

\subsubsection{圆线虫目(Strongylida)}
口囊柱形。
雄性交合囊三叶,两侧叶各有六条放射肋。
幼虫自由生活,成虫寄生脊椎动物小肠。
如十二指肠钩口线虫(\textit{Ancylostoma duodenale})。

\subsubsection{蛔虫目(Ascaridia)}
唇片三或六个,无口囊,咽柱形,大多无肌肉质咽球。
雄虫有两个等长交合刺。
如蛔虫(\textit{Ascaris lumbricoides})、蛲虫(\textit{Enterobius vermicularia})。

\subsection{旋尾亚纲(Spiruria)}
\subsubsection{旋尾目(Spirurida)}
体细长,尾部盘曲。
头部两个侧唇,口囊角质化。
雄虫交合刺不等长。
成虫寄生脊椎动物消化道、呼吸道,中间宿主为节肢动物。
如马来丝虫(\textit{Brugia malayi})。

\subsection{双胃线虫亚纲(Diplogasteria)}
体表有环纹、刻点。
唇不发达。
咽分四个部分,有瓣膜。

\subsubsection{双胃线虫目(Diplogastterida)}
大多有齿。
寄生昆虫。
如双胃线虫(\textit{Diplogaster})。

\subsubsection{垫刃线虫目(Tylenchida)}
唇发达,唇区光滑,化感器位于唇上,口囊内有长刺。
排泄器官为一纵管,位于体侧。
雄虫一对尾感器。
寄生昆虫、植物。
如垫刃线虫(\textit{Tylenchulus})。

\end{document}