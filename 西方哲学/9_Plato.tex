\documentclass[11pt]{article}

\usepackage{setspace}
\usepackage[colorlinks,linkcolor=blue,anchorcolor=red,citecolor=black]{hyperref}
\usepackage{lineno}
\usepackage{booktabs}
\usepackage{graphicx}
\usepackage{float}
\usepackage{floatrow}
\usepackage{subfigure}
\usepackage{caption}
\usepackage{subcaption}
\usepackage{geometry}
\usepackage{multirow}
\usepackage{longtable}
\usepackage{lscape}
\usepackage{booktabs}
\usepackage{natbib}
\usepackage{natbibspacing}
\usepackage[toc,page]{appendix}
\usepackage{makecell}

\title{Plato}
\date{}

\linespread{1.5}
\geometry{left=2cm,right=2cm,top=2cm,bottom=2cm}

\begin{document}

  \maketitle

  \linenumbers

Socrates does not construct a system of thought, although this is necessary if the work begun by the master was to be completed. 
The problems of human are naturally connected with each other. 
The meaning of human life, of human knowledge, of morality, of human conduct and of human institutions must be seen as a whole so that they could be answered completely. 
Plato and Aristotle, two great philosophers, the peaks of Greek thought, are going to concern all problems that their pioneers have argued and arrange them into comprehensive systems, which still have their influence in modern world. 
Here we shall begin with Plato (427 BC-347 BC).

\section{Dialectic and Theory of Knowledge}
Plato argues that if we rely solely on perceptions and opinions, there can not be genuine knowledge as Sophists have proposed. 
Sense perceptions can only show us appearance instead of the true reality of things. 
As for opinions, some of them may be true, but they are proofed by persuasion or feelings, for which they cannot be true knowledge. 
Genuine knowledge, on the other hand, is based on reason and authenticate itself. 
The majority of people think and act according to senses, feelings and opinions, which are taken for knowledge wrongly. 
They follow impulse, customs and habits instinctively, like ants or bees, without knowing why they think and act as they do.

\newline

Hence, Plato distinguishes knowledge from perceptions and opinions in theory. 
To rise ourselves to true knowledge, a desire for the truth is needed. 
It is the impetus that drives us to gain knowledge instead of being trapped in opinions, and dialectic is the method for this. 
In this method, we start with scattering particulars and summarize them into one idea, then the idea is checked, modified and divided into different parts. 
In this way, we keep particularizing and generalizing from concept to concept, upward and downward, until our concepts, the genuine knowledge, are finally carved out from senses and opinions, like a beautiful sculpture is carved out from a block of stone or wood. 
Dialectic is this art of thinking in concepts and finding true knowledge, and the usage of it in Plato’s works gains marvelous artistic effect.

\newline

Furthermore, Plato emphasizes that genuine knowledge, which equals to concepts, does not originate from experience. 
Knowledge starts with concepts, but nothing in experience can exactly correspond with a concept. 
None of things in the world of sense can be absolutely true, beautiful or good, although we have the concept of the true, the beautiful and the good. 
An analogy with geometry would make it clear. 
One can think about the concept of line, which is infinitely long but has no width, but he can never find a line which is exactly the same with the concept of the line in real life, no matter how long or how narrow it could be. 
The real role played by things of the world of sense in the discovery of knowledge is as merely tools used in explicating concepts that exist obscure and implicitly. 
In this process of discovering concepts, we achieve true knowledge. 
Thus, man is indeed the measure of everything, for the discovery of universal concepts, which are the start point of all principles.

\newline

So, what is the origin of knowledge? 
To Plato, knowledge originates from something real, permanent, and outside human beings. 
Knowledge is the correspondence of thought and reality. 
If a concept have any value as knowledge, something real must correspond to it. 
Since there are concepts, there must be entities correspond to them. 
However, nothing in the world of sense can exactly correspond to a concept. 
Hence, the entities of concepts are not in the world of sense, but upwards. 
Since genuine knowledge is universal and permanent, the entities correspond to concepts must be universal and permanent. 
The entities of concepts are called \textit{Ideas}, which is the central concept of Platonic philosophy and will be studied latter.

\newline

There is another way to gain the same result. 
The world of sense is always in ceaseless change, so it cannot be true being, which is permanent, but appearance or illusion. 
To have genuine knowledge instead of being trapped by illusions, we must rise ourselves from illusions to unchangeable essences of things, the permanent beings, the \textit{Ideas}, and this can only be done by conceptual thinking.

\newline

Plato’s theory of knowledge is summarized in the figure of a divided line presented in \textit{Republic}. 
A vertical line is divided into four parts, each of which represents a level of cognition and has its own object of study and method of inquiry. 
The lowest part is conjecture, a kind of sensuous knowledge conversant with images, shadows, dreams and so on. 
Conjectural knowledge is mere guesswork, but still, it may provide some clues to the characteristics of objects. 
The second part is called belief, sensible knowledge of objects, originated from the induction of experience. 
The basis of sense perceptions makes belief more reliable than conjecture. 
The third segment is discursive knowledge. 
It abandons sensuous particulars, but relies on unproved hypotheses and further rational deduction. 
To Plato, this mainly refers to mathematics. 
Mathematics rests on assumptions instead of self-evident principles, and uses perceptions as symbolized tools to help conduct deductions. 
For example, in geometry, drawn dots, lines and surfaces are used as symbols of ideal dots, lines and surfaces. 
That makes mathematics the third segment of the line. 
The highest part is rational insight, the knowledge of \textit{Ideas} and its method is dialectic. 
It rests on absolute principles instead of hypotheses and completely abandons perceptions. 
To Plato, conjecture and belief are all opinions, while discursive knowledge and rational insight belong to knowledge.

\newline

Hence, cognition is divided into four levels, mainly based on how abstract it is and how much it casts off perceptions, and the two higher levels, or knowledge, are further divided into different levels. 
The hierarchy of knowledge starts with arithmetic and concludes with dialectic. 
The lowest is arithmetic, the abstract science of numbers and numerical relations. 
It liberates intellect from senses by resolving contradictions in perceptions, and promotes abstract thinking. 
The second level is geometry. 
It also draws minds from perceptions to eternal forms and promote abstract thinking. 
The third is astronomy, the science of the motions of heavenly bodies and the principles that governs such motions. 
It requires abstract thinking, directs minds to the law and harmony of celestial motion, and thus paves the way to dialectical study of eternal forms, the \textit{Ideas}. 
The fourth is harmonics, the study of motions that produce harmonious sound. 
It leads minds to the harmony of \textit{Ideas}. 
Finally it is dialectic. 
It is concerned with the eternal forms and serves as a guide in morals and statecraft. 
It is the completion of inquiry.

\newline

Overall, to Plato, genuine knowledge is permanent, abstract, harmonious, based on reason instead of perceptions and gained by dialectic. 
Its contents are conceptions that corresponds to eternal forms, or \textit{Ideas}. 
Thus, only knowledge of \textit{Ideas} is qualified as genuine knowledge.
  

\section{Doctrine of \textit{Ideas}}
Now we shall study \textit{Ideas}, the central conception of Platonic philosophy. 
Plato denies the world of sense, criticizing it as illusion. 
It is defective and fulfilled with errors. 
Human minds must rise from senses and perceptions to conceptions, to learn the perfect and eternal forms, the \textit{Ideas}. 

\newline

\textit{Ideas} are not thoughts, but independent entities that exist prior to things. 
The particular objects we perceive are merely imperfect copies or reflections of \textit{Ideas}. 
Particulars may keep changing, but their forms, or \textit{Ideas} are permanent and uninfluenced by the fluxing of things, for they are apart from things. 
An \textit{Idea} is corresponded with a category of particulars of similar form. 
Since things can be classified into a infinite number of categories, there must be infinite \textit{Ideas}.

\newline

\textit{Ideas} are not disoraganized, but relate with each other and constitute a well-ordered cosmos of reason. 
They are arranged in logical order and forms an organic unity. 
Among all \textit{Ideas}, the \textit{Idea} of Good is supreme and is the source of all the rest. 
It is the realest, for the truly real and truly good are identical. 
It is the purpose and highest principle of the cosmos, the \textit{logos}. 
The \textit{Ideas} cannot be grasped by senses, for they only perceive imperfect and fleeting copies of \textit{Ideas}, but never rise to understand the perfect forms.

\newline

Theory of \textit{Ideas} is the core of Platonic philosophy and is Plato’s most original achievement, although a comprehensive inheritance of earlier philosophers can be seen. 
In \textit{Ideas}, it is easy to find the influence from the distinction of forms and objects of Pythagoras; 
the thought of ceaseless change in accordance with \textit{logos} from Heraclitus; 
the emphasis of permanence from Parmenides; 
the infinite kinds of elements from Anaxagoras; 
and most of all the theory of concepts from Socrates. 
The theory of \textit{Ideas} can be summarized as the following statements without over-simplification: 
(1) Forms or \textit{Ideas} are real entities that exactly correspond to conceptions. 
(2) There are infinite \textit{Ideas}, and every category of things has its \textit{Ideas}. 
There are \textit{Ideas} of objects like house or dog; 
of qualities like whiteness or toughness; 
of relations like equality or resemblance; 
of values like justice or good. 
(3) The \textit{Ideas} constitute an eternal hierarchy of abstract entities logically, separated from concrete particulars in time and space. 
(4) The \textit{Ideas} are superior to particulars in terms of reality and value. 
The \textit{Ideas} are reality while particulars are appearances and imperfect copies of \textit{Ideas}. 
(5) \textit{Ideas} are independent from any mind.
  
\section{Philosophy of Nature}
Hence, Plato develops a dualism between the world of forms and the world of objects. 
The former is real, ordered and good, while the latter is fake, chaotic and bad. 
Our knowledge about the former is rational, conceptional and dialectical; 
while for the latter, it is based on perceptions and opinions and is not qualified as genuine knowledge. 
As for the relations of the two worlds, the world of objects is an imperfect copy or reflection of the world of forms. 
How is this to be interpreted? 
How could perfect and eternal forms be responsible for the incomplete and ever-changing world of sense? 
How does the world of \textit{Ideas} impose its impact on the sensible world while being separated from it?

\newline

Thus, Plato proposes another principle, the matter, which is opposed to \textit{Ideas}. 
It is the basis of the sensible world, for it serves as raw material on which are somehow impressed by forms. 
The sensible world is resulted from interplay of \textit{Ideas} and matter. 
Things owe their being to forms, while matter is responsible for their imperfection. 
The diversity of things of one \textit{Ideas} is also attributed to matter. 

\newline

Then, there are two principles, \textit{Ideas} and matter. 
The former is true reality, the thing of greatest value, that to which everything owes its form and essence, the principle of law and order; 
while the latter is secondary, dull, irrational, an unwilling slave of the former. 
Matter, somehow, is influenced by \textit{Ideas}, though imperfectly. 
\textit{Ideas} provide the forms of the sensible world, while matter provides material for it. 
\textit{Ideas} pose a tendency towards order and perfection on matter, which is characterized by a tendency towards chaos. 
The two opposite tendencies act on matter and result in sensible world of imperfection.

\newline

The conception of matter is an unsuccessful note about the relation of the world of forms and the world of sense, which Plato does not define with enough preciseness. 
Since sensible world is merely illusion, how could it have a material origin? 
In the discussion of the matter, it is indicated that matter is real to some extent, or on which \textit{Ideas} would impress? 
If so, how could the real matter results in illusion of sensible world? 
To deal this contradiction, there are two ways on theory. 
The first one is giving up the conception of matter. 
The sensible world can be explained as shadows of \textit{Ideas} that appear in human senses. 
Thus, one should rises himself to the level of \textit{Ideas} using his reason instead of being trapped by senses, to escape from illusions of sensible world. 
This is similar with Buddhism. 
Another way is admitting that the world of sense is not mere illusion, but reality to some extent. 
This way is adopted by Aristotle.
  
\section{Cosmology} 
Plato tries to explain the origin of universe in \textit{Timaeus} and thus develops his cosmology, intermingling with mythical elements and often contradicting with his other teachings. 
However, it is nothing more than probability for Plato. 

\newline

At the very beginning, there are ideal \textit{Ideas} and material matter. 
Using the material and following the pattern of the ideal, the Demiurge creates the universe, guided by the highest \textit{Idea} of Good. 
The universe he forms is as similar as perfect \textit{Ideas}, but is hampered by the material. 
The universe so generated is composed of four material elements: 
water, fire, earth and air. 
There is also an immaterial animating soul, the world-soul, which both knows the ideal and perceives the material. 
It is an intermediary between the ideal and the material, and has its own original motion. 
The world-soul moves in accordance with fixed laws, and is the cause of all motions and the source of all laws and order, for it is diffused throughout the whole universe.

\newline

Beside the world-soul, the Demiurge also creates gods and the rational part of human souls. 
The irrational part of human souls and animals are left to gods to create. 
Everything has been made for human: 
plants to nourish him; 
animal-bodies to serve as habitations of fallen souls losing rational parts. 
The total world of \textit{Ideas} includes the \textit{Ideas} of the Good, the Demiurge, the world-soul, the planetary souls, and the gods of popular religion.

\newline

The cosmology of Plato is characterized by teleology. 
He attempts to explain the universe as purposeful and well-ordered. 
The cosmos is intelligent, guided by reason and directed towards an ethical goal, the \textit{Idea} of the highest Good. 
Purposes are the real causes of the world, while the material is merely cooperative.

\newline

Another important point in Plato’s myth of the creation of the universe is an attempt to differentiate the causes, or creative factors of actual world. 
Plato’s story of creation is not intended to describe the origin of universe, but analyse the causes of the universe by considering it as if it had been created through a process. 
In \textit{Timaeus}, four causes are listed: 
(1) The Demiurge, the active and dynamic cause. The Demiurge, an impersonal spirit, constructs the world using the ideal and the material. 
It is the source and principle of all motions in nature and in mind. 
(2) The ideal, the pattern of the world. 
It is the form of the everything. 
(3) The material. 
(4) The form of the Good. 
It is the purpose of everything. 
If the universe is likened to a building; 
then the ideal would be the design diagram, which is ordered; 
and the material would be the bricks and concrete, which is disordered; 
and the form of Good would be the usage of the building. 
  
\section{Psychology}
Plato has distinguished genuine knowledge based on reason and opinions based on perceptions. 
This dualism is also seen in his theory of the soul, which is divided into rational part and irrational part, as it is indicated in Platonic cosmology. 
The opinions are attributed to the body and irrational soul, while genuine knowledge is attributed to rational soul which has beheld the world of \textit{Ideas}. 
Thus the body is an impediment to knowledge, from which the rational soul must free itself to see the true \textit{Ideas}. 
The things in sensible world merely activate the rational soul to think, to remember what it has seen, but do not produce knowledge. 
The rational soul must rise to the eternal forms from its contact with experience. 
Thus, rational soul must exist and sees the world of forms before it is added with the mortal and irrational part to fit the existence in the sensible world. 
Gaining genuine knowledge is a process of the purification of the rational soul to get rid of the impediment of body and recall what it has seen. 
As for the irrational part of the soul, it is further divided into the spirited part, by which Plato means nobler emotional impulses like honour and ambition, situated in the heart; 
and desire, by which he means lower passions and appetites for bodily satisfactions situated in the liver.

\newline

Hence, Plato divides soul into three faculties: the rational, the spirited and the appetitive. 
The latter two are irrational. 
This separation is based on the principle of contradiction that one cannot produce opposite effects at one and the same time. 
So the rational and irrational has its own faculty respectively, and they may pull the soul to different directions. 
This is dependent on their interactions. 
For example, when a thirst man gets poisonous drink, the rational will prevent drinking it while the irrational will want to drink it. 
However, the rational and the irrational still can be in conformity, for example, a moderate desire for food and drink.
  
\section{Immortality of Soul}
Plato attributes immortality to souls and provides several arguments in his dialogues. 
The most characteristic one is based on the doctrine of reminiscence. 
In dialectic, conceptions are gradually carved out from crude opinions, indicating that they are from within, but have been forgotten, and dialectic helps souls remember what they have known. 
Thus, the soul must pre-exist and know the world of forms before it is trapped in the body; 
and for similarity between subject and object is required if the subject wants to learn about the object, the soul must share some common characteristics with \textit{Ideas}, or the effect of dialectic cannot be explained. 
Since \textit{Ideas} are immortal, souls are also immortal, and the pre-existence of the soul before the body also supports this point.

\newline

Plato also proposes metaphysical proofs, some of which are especially prominent. 
First, the soul is by nature simple and indivisible. 
So it cannot originate by composition, and cannot be destroyed by degradation. 
Thus, the soul is immortal. 
Second, the soul is the principle of life, and of course it should be immortal, or there would be no life. 
Third, the soul, as the source of motion, is immortal, or there would be no motion.
Finally, there are arguments of value and moral, ascribing immortality to the soul. 
Plato declares that the soul is of superior dignity and value over mortals. 
Thus, the soul is immortal, or the superiority has no way to begin. 
Another argument is related with Platonic cosmology that the universe is purposive and ethical. 
The universe is moral, rational and just. 
One would be rewarded or punished because of his goodness or badness, and this is based on the requirement of justice of the world. 
The body may decay without being rewarded or punished, and if the soul also dies at the same time, how the justice of the world is supposed to be conducted? 
On what the rewards or punishments should be imposed? 
Thus, the soul must be immortal to deal this problem.

\newline

One more thing: 
how does the immortal and rational soul combine with the mortal and irrational body? 
To answer this question, again Plato turns to mythical elements and further intermingles with his psychological doctrines. 
The Demiurge creates a soul and places it on a planet, which can be understood as the ideal world of forms. 
However, the soul has its irrational part and possesses a desire for a world of sense. 
So it enters a material body. 
In the world of sense, the incomplete copies of the forms would activate the memories of the contemplated beauty of the world of \textit{Ideas}, resulting in a yearning of rising back to the world of forms. 
This tendency is against with the irrational part of the soul, which seeks sensible impulses. 
In this conflict, if the rational takes advantage, the soul is purified and will go back to the world of \textit{Ideas} until it has completely abandoned the irrational; 
if the irrational takes advantage, the soul will fall to the bodies of different animals. 
The whole process may be accompanied with several turns of the birth and death of the body. 
Clearly the influence of metempsychoses can be seen.
  
\section{Ethics}
Now we shall turn to more practical problems about human. 
How should a rational being act? 
What is moral and virtue? 
What is the way, or ways, by which the soul purifies itself to get close with the ideal world? 
Socrates has raised these problems and answered them without proposing a complete and systematic philosophy of life. 
Plato, on the other hand, seeks to solve it in the light of his comprehensive philosophical system, involving the meaning of human life and institutions to the larger questions of world and nature by putting ethics and politics on the basis of metaphysics.

\newline

Objects of sense are merely imperfect reflections of the world of \textit{Ideas}, which is perfect, rational, ordered, and eternal. 
Thus, sensible world has no value, for it is imperfect, irrational, disordered and mortal. 
Reason, which alone is conversant with \textit{Ideas}, has true value. 
Hence the rational part of human soul is true part, and one’s ideal must be to cultivate his reason, the immortal side. 
The body as a sensible object, is the prison house of the reason, a fetter from which the reason must fly away from to reach the beautiful world of ideas, the ultimate end of life.

\newline

To rise to forms, the soul must surpass the body with its reason, but the soul itself is not pure reason. 
It is composed of pure reason, the spirited and appetitive. 
How should soul deal with its irrational part? 
To Plato, one is wise when reason rules over other impulses of the soul, knowing what is advantageous. 
The spirited is subject to reason and allies with it. 
The rational should trains, educates, and unite with the spirited. 
After the unison of the rational and the spirited, they exercise control over the appetitive. 
The spirited fights against the appetitive by its bravery, following the orders of the rational. 
Finally, both the spirited and the appetitive are subjective to the rational and follow its instructions. 
Thus, each one of the three parts is doing its proper work and harmony is introduced to the soul. 
The individual would be wise, brave, self-control, temperate and just with the authority of his reason and the harmony of his soul.

\newline

The ethical ideal, therefore, is a well-ordered soul under the authority of reason. 
A life of reason is of virtue and of genuine happiness. 
Pleasure is accompanied with the gratification of desire; 
and the more reasonable the desire is, the more pleasure would be gained with its gratification.
  
\section{Politics}
Plato’s politics is based on his ethics. 
Since individual should purify his soul to rise to the \textit{Ideas}, to the highest Good, and this cannot be done in isolation, the duty of the state is to promote virtue among its people, to enable as many men as possible to be good; 
that is, to secure the welfare of the majority. 
For an individual, social life is a mean to his perfection and purification, not an end in itself. 
So the individual must subordinate his own interests to the public welfare, devote himself to social life, for his true benefit: the purification of soul and ascension to pure reason. 
If all men were rational and good, there would be no need of states and laws: 
everyone was governed by internal reason and followed its instructions, and external laws were useless. 
The organization of the state should follow the pattern of the world of forms, for it is rational and well-ordered. 
In the world of forms, \textit{Ideas} are arranged in a hierarchy logically and the \textit{Idea} of Good is supreme to all the others. 
The human institution, thus, should be in the ascendency.

\newline

Plato also analogizes human society to human soul. 
The soul is composed of parts of different functions. 
Things are the same to human society, and all the parts of society have the harmonious relationships corresponding to those in a healthy soul. 
Those who have philosophical insight represents the rational and ought to be the ruling class. 
The warrior class represents the spirited and its task is defense. 
The agriculturists, artisans and merchants are corresponded to the appetitive, and their function is to produce material goods. 
Thus, each class does its own work and focuses on its own business without meddling the work of other classes; 
the desire of the vulgar of the many is ruled by the rational desire of wisdom of the few; 
every individual has his occupation in the state and his capacity is best adapted; 
there is no argument about who the governors or governed should be. 
So the state would be wise, brave, self-control, temperate and just, as that of a healthy soul ruled by reason.

\newline

The influence of Sparta is obvious in Platonic politics. 
The Peloponnesian War (431 BC-404 BC) perhaps have impressed Plato greatly, in which democratic and cultural Athens is beaten by militarist Sparta. 
More impressions of Sparta would be found in detailed devices of Plato’s ideal state. 
Plato opposes private property and monogamous marriage, for an ideal society itself should be a unity, a large family. 
He recommends, especially for the higher two classes, communism and the common possession of wives and children. 
Other recommendations include supervision of marriages and births, exposure of weak children, compulsory state education, education of women for war and government, censorship of art and literature. 
Plato denies art as imitation of sensible world, which itself is an imperfect imitation of ideal world, although he thinks it can be helpful in promote moral culture.

\newline

Since the aim of the state is leading its people to reason and goodness, education must have its place. 
Thus, Plato proposes a plan of education for the children of higher classes, which shall be the same for citizens of both sexes in the first twenty years of life. 
It includes: 
myths selected by their ethical influence; 
gymnastics, which develops the body and the spirited part of the soul; 
reading and writing; 
poetry and music, which arouse the sense of beauty, harmony and proportion, and encourage philosophical thought; 
mathematics, which draws the mind from the sensible and concrete to the abstract, the universal and the real. 
When these young people turn to twenty, superior individuals would be chosen to learn the interrelations of subjects they learned before and survey them as a whole. 
At the age of thirty, the most outstanding individuals would be chosen to learn dialectic for five years. 
Then they would be put to tests of holding military commands and other political affairs. 
At the age of fifty, selection would be conducted again and the chosen people would devote themselves to the study of philosophy and administer affairs of the state in turn reluctantly, for the sake of the country and public welfare.

\newline

Now we shall end the study of Plato’s ideal state, which he calls utopian and is described in \textit{Republic}. 
It shall be learned that Plato is of fully awareness that his utopian can be never actually realized. 
It is a guidance in the organization and administration of actual societies, and some of its characteristics can be found in the policy of Sparta of that time, showing that an actual state have the ability to develop towards the scheme of ideal utopian. 
Still, it is of impracticality. 
Perhaps based on such considerations, Plato abandons some idealistic traits of his ideal state in his latter work, \textit{Laws}. 
Besides insight and reason, friendship and freedom gain their importance in a good state, and all citizens should be free and have a share in the government. 
They are all landowners, and all trades and commerce are given to serf and foreigners. 
The family is restored to its natural position and the importance of reason and knowledge is weaken, for there are other motives of virtue, like pleasure, friendship, pain, hate and so on. 
However, the aim of the state does not change. 
Its duty still is to promote virtue and lead its people to good and reason.
  
\section{Summary}
Plato constructs a splendid system of philosophy, in which rational thinking, poetic passion and mythical imagination intermingles, bringing it marvelous charm even with the existence of discordance and contradictions. 
It is difficult to find a philosophical question that was not studied philosophically by Plato; 
and seldom scholars have the ability to make their works so rich in passion and have the amazingly artistic effect, while conserve rational arguments that makes sense at the same time. 
Perhaps in modern time, Plato’s enthusiasm for knowledge, the art of thinking and arguing, and all his questions are more important and meaningful compared with his actual thoughts. 
Some scholars have summarized the whole western philosophy as a series notes of Platonic philosophy and it is not based on nothing.
  
\end{document}