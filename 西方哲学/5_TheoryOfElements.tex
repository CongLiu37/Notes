\documentclass[11pt]{article}

\usepackage{setspace}
\usepackage[colorlinks,linkcolor=blue,anchorcolor=red,citecolor=black]{hyperref}
\usepackage{lineno}
\usepackage{booktabs}
\usepackage{graphicx}
\usepackage{float}
\usepackage{floatrow}
\usepackage{subfigure}
\usepackage{caption}
\usepackage{subcaption}
\usepackage{geometry}
\usepackage{multirow}
\usepackage{longtable}
\usepackage{lscape}
\usepackage{booktabs}
\usepackage{natbib}
\usepackage{natbibspacing}
\usepackage[toc,page]{appendix}
\usepackage{makecell}

\title{Theory of Elements}
\date{}

\linespread{1.5}
\geometry{left=2cm,right=2cm,top=2cm,bottom=2cm}

\begin{document}

  \maketitle

  \linenumbers

Heraclitus and Parmenides proposed two opposite views on the problem of change: 
Heraclitus holds the view that everything is always changing while Parmenides believes that change is impossible. 
Still, this riddle is not solved. 
Things seem to persist, like mountains and oceans and all those heavenly bodies. 
However, things also seem to change, like the transformation between water and moisture. 
Younger philosophers try to blend such two opposite ideas in a mechanical way. 
Empedocles, Anaxagoras and the atomists agree with Parmenides that absolute change is impossible, but there is change in relative term. 
Things are composed of permanent particles, or elements. 
Through the combination or separation of these particles, things originate or decay. 
So change and transformation are attributed to motion. 
These philosophers differ in the answers of the following two questions: what is the nature of these particles and what causes them to combine or separate? 

\section{Empedocles}
Empedocles (495 BC-435 BC) is famous for his theory of four elements. 
There are four kinds of elements, or “root of things”, of different qualities: 
water, fire, earth and air. 
These permanent particles combine, separate or rearrange their relationships, resulting in the origin, loss or transformation of objects respectively.

\newline

Then, what drives the motion of these particles? 
Empedocles attributes it to two mythical forces, Love and Hate. 
Love makes elements come together while Hate drives the separation. 

\newline

The effect of the two opposite forces results in the cyclic recurrence of the universe. 
At the very beginning, all particles are intermingled together, forming a sphere and Love reigns supreme. 
The universe is in a state of complete combination. 
Gradually, Hate grows, leading to the separation. 
The universe comes to a state of partially combined and partially separated. 
In this process, air appears first, forming the arch of heavens. 
Then it is fire, forming the sphere of stars beneath. 
Water is pressed from earth by rotation and is evaporated to moisture, resulting seas and lower atmosphere respectively. 
Organisms arise from earth: 
first plants, then the different parts of animals, like arms, legs, heads and eyes. 
All these parts combined with each other randomly and produce animals of weird forms, which would separate again because they are not fit to live. 
Such process continues until animals of normal forms, which are fit to survive, are produced. 
These are perpetuated by generation. 
During this process of separation, Hate is keeping growing. 
Finally, Hate gains the upper hand and leads to the universe to a state of complete separation. 
Elements are completely separated from each other and no objects would exist. 
Then Love grows, and the whole universe moves to the state of complete combination again and things would recur. 
Briefly, the universe is always changing from one extreme state to another, from complete combination to complete separation or the reverse. 
Things only exist when the universe is in the state intermediate between two extremes. 

\newline

Man is composed of the four elements and that is why we can sense all these elements. 
We can feel water, fire, earth, air and objects composed of them by the same elements in us. 
For example, we see earth by earth, and water by water. 
Particles from the things we sense meet the same particles in us, by which we see, hear, smell and taste.
  
\section{Anaxagoras}
Anaxagoras (500 BC-428 BC) disagrees with Empedocles’ four elements. 
He argues that a world so rich of different qualities cannot be explained by a so limited number of kinds of elements. 
Besides, water, fire, earth and air are not elements. 
For example, man’s body is composed of skin, bone, blood, flesh and so on. 
They are all of different qualities. 
Since the body is nourished by food, food must contains all the elements of which the body is composed. 
While food gets its nutrition from earth, air, water and the sun, so the later must contains the elements which compose food and body. 
So they are not elements, but mixtures of multiple kinds of elements. 
Hence, Anaxagoras assumes an infinite number of kinds of elements. 

\newline

What causes these elements to move? 
Perhaps being inspired by the revolution of heavenly bodies, Anaxagoras assumes a rapid and forcible rotation. 
At the very beginning, all elements combined with each other. 
At a certain point in this confusing mass, a rotation started and extended, bringing similar particles together, and will continue until the initial mixture is finally disentangled. 
The rotation caused the separation of the dense and the rare, the warm and the cold, the bright and the dark, the dry and the moist. 
Then the dense, the cold, the dark and the moist were collected at where earth now is, while the rare, the hot, the light and the dry departed toward the further part of the ether. 
Some solid masses on earth were hurled from the earth by the force of rotation and became heavenly bodies. 
As for the formation of things on earth, the heat of sun dried up the moist earth, and from the elements filling the air which fell into the earth-slime by rain, organisms arose. 
The motion of creatures is attributed to souls.

\newline

Another problem: how did the rotation start since it is the cause of all the following motion? 
To answer this question, Anaxagoras finally gives up mechanical principles and attributes it to \textit{nous}, an absolutely simple and homogeneous substance, absolutely separated from any other elements. 
It has power over all being and is the free source of all motion in the world. 
It knows all things past, present and future. 
It arranges all things and is the cause of all tings. 
The \textit{nous} is purposive. 
It forms only one unique and perfect universe. 
So the universe cannot recur as other philosophers like Empedocles proposes. 

\newline

Sometimes, parallels are drawn between \textit{nous} and spirit. 
The word \textit{nous} originates from Greek, meaning mind, intelligence, intuitive apprehension. 
It may sound like something immaterial and spiritual, but in fact, it is difficult to draw a conclusion whether the \textit{nous} of Anaxagoras is material, immaterial or something neither not entirely material nor not entirely immaterial. 
To Anaxagoras, \textit{nous} is something distinct from other elements, which makes it seem to be immaterial. 
However, he also expresses it as the most rarefied of all things, suggesting it is material. 
Perhaps Anaxagoras himself is not really sure about it. 
A reference from Aristotle may help understand Anaxagoras’ attitude on spirit and material: 
  
\textit{Anaxagoras uses mind as a device by which to construct the universe, and when he is at a loss for the cause why anything necessarily is, then he drags it in, but in other cases he assigns any other cause rather than mind for what comes into being.}

Maybe to Anaxagoras, mind is the last sort in his theory. 
He wishes to explain everything using mechanical principles and only uses mind unwillingly when the former fails. 
  
\end{document}