\documentclass[11pt]{article}

% \usepackage[UTF8]{ctex} % for Chinese 

\usepackage{setspace}
\usepackage[colorlinks,linkcolor=blue,anchorcolor=red,citecolor=black]{hyperref}
\usepackage{lineno}
\usepackage{booktabs}
\usepackage{graphicx}
\usepackage{float}
\usepackage{floatrow}
\usepackage{subfigure}
\usepackage{caption}
\usepackage{subcaption}
\usepackage{geometry}
\usepackage{multirow}
\usepackage{longtable}
\usepackage{lscape}
\usepackage{booktabs}
\usepackage{natbib}
\usepackage{natbibspacing}
\usepackage[toc,page]{appendix}
\usepackage{makecell}
\usepackage{amsfonts}
 \usepackage{amsmath}

\title{Abstracts}
\author{}
\date{}

\linespread{1.5}
\geometry{left=2cm,right=2cm,top=2cm,bottom=2cm}

\begin{document}
\begin{sloppypar}
  \maketitle

  \linenumbers

\textbf{1. Viljakainen, 2015, Evolutionary genetics of insect innate immunity.} \newline
Toll and IMD pathways are well conserved among insects. \newline
Immune gene families evolution: expansion/shrink. \newline
Selection mode of immune genes. \newline

\par

\textbf{2. He et al., 2021, Evidence for reduced immune gene diversity and activity during the evolution and activity during the evolution of termites.}
Immune genes of 18 species of termites and cockroach were identified by transcriptomics, to explore relationship between evolution of immunity and transition of solitary, subsocial and eusocial lifestyles. 
Phylogenetic signal analysis shows immune gene diversity loss during termite evolution. 
Gene families of immune receptors and effectors contracted in termites, and some re-expanded in higher termites.

\end{sloppypar}
\end{document}