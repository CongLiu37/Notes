\documentclass[11pt]{article}

% \usepackage[UTF8]{ctex} % for Chinese 

\usepackage{setspace}
\usepackage[colorlinks,linkcolor=blue,anchorcolor=red,citecolor=black]{hyperref}
\usepackage{lineno}
\usepackage{booktabs}
\usepackage{graphicx}
\usepackage{float}
\usepackage{floatrow}
\usepackage{subfigure}
\usepackage{caption}
\usepackage{subcaption}
\usepackage{geometry}
\usepackage{multirow}
\usepackage{longtable}
\usepackage{lscape}
\usepackage{booktabs}
\usepackage{natbib}
\usepackage{natbibspacing}
\usepackage[toc,page]{appendix}
\usepackage{makecell}
\usepackage{amsfonts}
 \usepackage{amsmath}

\title{Comparative genomics reveals the origin and diversity of arthropod immune system}
\author{}
\date{}

\linespread{1.5}
\geometry{left=2cm,right=2cm,top=2cm,bottom=2cm}

\begin{document}
\begin{sloppypar}
  \maketitle

  \linenumbers

Immune gene families are searched in genomes of arthropods, including 
insect \textit{Drosophila melanogaster}, crustacean \textit{Daphnia pulex} (water flea), 
myriapod \textit{Strigamia maritima} (coastal centipede) and 
five chelicerates: \textit{Mesobuthus martensii} (Chinese scorpion), 
                   \textit{Parasteatoda tepidariorum} (house spider),
                   \textit{Ixodes scapularis} (deer tick),
                   \textit{Metaseiulus occidentalis} (western orchard predatory mite),
                   \textit{Tetranychus urticae} (red spider mite).

\par

Arthropod Toll-like receptors (TLRs) are a dynamically evolving gene family that includes relatives of vertebrate TLRs.

\par

The Toll signaling pathway is conserved across arthropods.

\par

The IMD signaling pathway is highly reduced in chelicerates.

\par

The JAK/STAT signaling pathway is highly conserved.

\par

Peptidoglycan-recognition proteins.

\par

$\beta$-1,3 glucan recognition proteins ($\beta$GRPs) have been lost in chelicerates.

\par

Arthropod TEPs include relatives of vertebrate C3 complement factors and proteins lacking the thioester motif.

\par

Gene duplication generates diversity in the immune receptor Down syndrome cell adhesion molecule (Dscam).

\par

Fibrinogen-related proteins (FREPs) and Nimrod-like proteins.

\par

Prophenoloxidase and melanization.

\par

Dual oxidase (DUOX).

\par

AMPs and lysozymes.

\par

Antiviral RNAi.

\end{sloppypar}
\end{document}