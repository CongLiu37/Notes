\documentclass[11pt]{article}

\usepackage[UTF8]{ctex} % for Chinese 

\usepackage{setspace}
\usepackage[colorlinks,linkcolor=blue,anchorcolor=red,citecolor=black]{hyperref}
\usepackage{lineno}
\usepackage{booktabs}
\usepackage{graphicx}
\usepackage{float}
\usepackage{floatrow}
\usepackage{subfigure}
\usepackage{caption}
\usepackage{subcaption}
\usepackage{geometry}
\usepackage{multirow}
\usepackage{longtable}
\usepackage{lscape}
\usepackage{booktabs}
\usepackage{natbibspacing}
\usepackage[toc,page]{appendix}
\usepackage{makecell}
\usepackage{amsfonts}
 \usepackage{amsmath}
\usepackage[utf8]{inputenc}
\usepackage{amssymb}
\usepackage{amsthm}
\usepackage{enumerate}
\usepackage{comment}

\usepackage[backend=bibtex,style=authoryear,sorting=nyt,maxnames=1]{biblatex}
\bibliography{} % Reference bib

\title{双子叶植物-合瓣花亚纲(Sympetalae)}
\date{}

\linespread{1.5}
\geometry{left=2cm,right=2cm,top=2cm,bottom=2cm}

\setlength\bibitemsep{0pt}

\begin{document}
\begin{sloppypar}
  \maketitle

  \linenumbers
花瓣连合成花冠管,雄蕊生于花冠管上,珠被常一层。

\section{杜鹃花目(Ericales)}
木本,单叶,无托叶。
两性花,基本辐射对称。
花五数,雄蕊与花瓣互生。
花药有附属物,顶孔开裂。
子房上位或下位,胚珠多,有胚乳。

\subsection{杜鹃花科(Ericaceae)}
木本。
单叶互生,全缘或有锯齿,背面有深槽或闭合腔,上表皮厚角质化,有贮水组织,被毛或鳞,无托叶。
两性花单生或簇生。
花萼宿存,四或五裂。
花瓣四或五,合生成钟状、漏斗状或壶状。
中轴胎座,花柱单生。
蒴果、浆果或核果。

\subsubsection{杜鹃花(\textit{Rhododendron})}
多灌木,常绿或落叶。
叶全缘,多聚生枝顶。
花排成顶生、伞形花序式的总状花序。
萼瓣五裂,有齿缺花盘。
子房五至二十室。
蒴果。

\subsubsection{吊钟花(\textit{Enkianthus})}
落叶灌木,叶全缘或有齿,花药有芒。
蒴果,三至五角或有翅,室背开裂。

\subsubsection{越橘(\textit{Vaccinium})}
灌木。
花冠坛形或钟状。
雄蕊内藏,不抱花柱。
子房下位,浆果。

\section{报春花目(Primulales)}
木本或草本,叶有腺点。
两性花,辐射对称。
雄蕊与花冠裂片对生。
子房上位或半下位,一室,胚珠多,特立中央胎座。

\subsection{报春花科(Primulaceae)}
草本,有腺点。
单叶,无托叶。
两性花,辐射对称,有苞片,总状或伞形花序。
花萼五裂,与花冠合生。
雄蕊与花冠裂片对生,生于花冠管上。
子房上位,一室,特立中央胎座。
蒴果,胚乳多。

\subsubsection{报春花(\textit{Primula})}
叶基生,花冠裂片在花蕾中覆瓦或镊合排列,花冠管长与花冠裂片。
伞形花序,有苞片。

\subsubsection{排草(\textit{Lysimachia})}
花冠裂片在花蕾中螺旋排列。
蒴果瓣裂。
  
\section{柿树目(Diospyrales)}
木本,单叶互生,无托叶。
花辐射对称。
子房上位或下位,中轴胎座,珠被一或二。
胚乳不发达。

\subsection{山榄科(Sapotaceae)}
乔木或灌木,有乳汁。
单叶互生,全缘,无托叶。
两性花,辐射对称,单生或簇生于叶腋。
萼四至八裂,裂片一或二轮。
花冠管短,裂片有全缘或裂片状附属。
雄蕊与花冠对生,生于花冠管。
心皮一轮,胚珠一,浆果。

\subsubsection{紫荆木(\textit{Madhuca})}
萼片四,能育雄蕊十六以上,无退化雄蕊。

\subsubsection{金叶树(\textit{Chrysophyllum})}
萼片五,花冠裂片无附属物,雄蕊五至十,无退化雄蕊。

\subsubsection{铁榄(\textit{Sinosiideroxylon})}
叶互生,无托叶,萼片五,花冠裂片无附属物,有退化雄蕊,花丝基部光滑。
子房五室,种子疤痕基生。

\subsection{柿树科(Diospyraceae)}
木本,木材多黑褐色。
单叶互生,无托叶。
单性花,雌雄异株,单生或伞形花序。
花萼四裂,果熟时增大宿存。
花冠钟状或壶状,四或五裂,螺旋排列。
雄蕊常十六,或合生成束。
子房上位,二至十六室,胚珠一或二。
浆果,有硬质胚乳。
如柿(\textit{Diospyros kaki})。

\subsection{野茉莉科(Styracaceae)}
木本,单叶互生,无托叶。
嫩枝和叶常被星状毛。
两性花,总状花序,腋生或顶生,花冠四或五裂。
雄蕊为花冠裂片二倍,花丝基部合生。
子房上位到下位,基部三至五室,上部一室,胚珠一至多。
浆果或核果,或干燥开裂为三瓣。

\subsubsection{鸦头梨(\textit{Melliodendron})}
先花后叶,冬芽有鳞苞。
花单生或双生,花冠五裂。
子房半下位,果有肋无翅,硬木。

\subsubsection{木瓜红(\textit{Rehderodendron})}
花多,总状或圆锥花序。

\subsubsection{赤叶杨(\textit{Alniphyllum})}
先叶后花,冬芽无鳞苞,花丝基部合生。
果梗弯曲不明显,有关节。
蒴果,室背开裂,种子有翅。

\subsubsection{野茉莉(\textit{Styrax})}
子房稍半下位。
果与萼筒分离,不规则三瓣开裂。
种脐大。

\subsection{山矾科(Symplocaceae)}
仅山矾(\textit{Symplocos})。
木本,单叶互生,无托叶。
花两性,辐射对称。
花萼五裂,萼片镊合或覆瓦排列,多宿存。
花冠裂至基部或中部。
雄蕊多至四,生于花冠。
子房下位或半下位,一至五室,垂生胚珠二至四,花柱细。
浆果或核果,顶部有宿存花萼,胚乳多。

\section{木犀目(Oleales)}
仅木犀科(Oleaceae)。
如连翘(\textit{Forsythia suspensa}),茉莉(\textit{Jasminum sambac}),油橄榄(\textit{Olea europaea})。

\subsection{木犀科(Oleaceae)}
木本或藤本。
叶对生,单叶或三出复叶或羽状复叶,无托叶。
圆锥、聚伞或丛生花序。
萼常四裂,花冠裂片四至九。
雄蕊二,花药纵裂二室。
子房上位,二室,二心皮,中轴胎座,胚珠一至三,珠被一。
花柱单生,柱头二尖裂。
有胚乳。
核果、浆果、蒴果或翅果。

\subsubsection{梣(\textit{Fraxinus})}
复叶。
花序间或有叶状苞片。
翅果,果实顶端翅伸长。
如白蜡树(\textit{Fraxinus chinensis})。

\subsubsection{丁香(\textit{Syringa})}
花紫或红,鲜有白色。
花冠裂片比花冠管短。
蒴果,种子有翅。

\subsubsection{木犀(\textit{Osmanthus})}
花芳香,簇生或短圆锥花序。
花冠裂片在芽中覆瓦排列,核果。
如木犀(桂花)(\textit{Osmanthus fragrans})。

\section{龙胆目(Gentianales)}
叶常对生。
两性花,辐射对称,花萼四或五。
花冠管状,裂片四或五,雄蕊与花冠裂片同数。
心皮二,子房上位到下位,中轴胎座。


\subsection{马钱科(Loganiaceae)}
草本、灌木或乔木。
叶对生或轮生,单叶,托叶退化。
花常两性,萼、瓣、雄蕊各四或五。
子房上位,二室,胚珠二,花柱单生二裂。
蒴果、浆果或核果。
如断肠草(\textit{Gelsemium elegans})。

\subsubsection{马钱(\textit{Strychnos})}
灌木或乔木,或为缠绕、攀缘、附生植物。
枝有时变为钩刺。
叶脉三至五出。
种子含马钱碱,剧毒。

\subsection{夹竹桃科(Apocynaceae)}
木本、藤本或草本,有乳汁或水液。
单叶,对生或轮生,全缘,无托叶,叶柄基部有腺体或腺鳞。
两性花,辐射对称,单生或聚伞、圆锥花序。
花萼合生成筒状或钟状,五裂,覆瓦排列,基部有腺体。
花冠合瓣,五裂,旋转状覆瓦排列,喉布有毛或鳞片。
雄蕊五,生于花冠管上或喉。
花药二室,纵裂。
花盘环状、杯状或为腺体。
子房上位,二室,花柱合生,胚珠一或二。
种子常有长丝毛或翅,有胚乳。
如黄花夹竹桃(\textit{Thevetia peruviana}),夹竹桃(\textit{Nerium indicum}),蛇根木(\textit{Rauvolfia serpentina})。

\subsection{萝摩科(Asclepiadaceae)}
草本、藤本或灌木,有乳汁,常有毒。
单叶,对生或轮生,无托叶,叶柄顶端有腺体或鳞腺。
两性花,辐射对称,聚伞花序。
花萼筒短,五裂,覆瓦或镊合排列,内面基部常有腺体。
花冠合瓣,五裂,旋转、覆瓦或镊合排列。
有五个裂片或鳞片构成副花冠,连生于花冠筒。
雄蕊五,花药合生成管包围雌蕊。
花药与柱头黏合成中心柱。
花粉结成花粉块或四合花粉,位于匙形花粉器上。
子房上位,二离生心皮。
花柱二,合生,柱头基部五棱,胚珠多。
蓇葖果。
种子有种毛,胚乳薄。
如夜来香(\textit{Telosma cordata})。

\subsubsection{白叶藤(\textit{Cryptolepis})}
木质藤本。
花蕾基本长圆形,顶部尾状渐尖。
副花冠与花丝生于花冠筒内面中部以上,与花丝离生。
副花冠裂片卵形,顶端钝。
花粉器单个匙形,下有粘盘,四合花粉。

\subsubsection{杠柳(\textit{Periploca})}
副花冠与花丝生于花冠基部,花丝筒装,副花冠裂片异形,四合花粉。

\subsubsection{匙羹藤(\textit{Gymnema})}
缠绕茎。
副花冠有五个硬肉质条带或两纵列毛,位于花冠喉部。

\subsection{茜草科(Rubiacece)}
木本、草本或藤本。
单叶,对生或轮生,全缘。
托叶二,分离或合生成鞘,宿存。
两性花,辐射对称。
花萼筒与子房连生。
花冠合瓣,裂片四至六。
雄蕊与花冠裂片互生。
子房下位,常二室,胚珠多至一。
蒴果、核果或浆果。
种子有胚乳,或有翅。
如栀子(\textit{Gardenia jasminoides}),金鸡纳树(\textit{Cinchona ledgerian})。

\subsubsection{水团花(\textit{Adina})}
灌木或乔木。
花多,球状或头状花序。
有小苞片,萼檐五裂。
蒴果,中轴宿存,顶部有星状萼檐裂片。

\subsubsection{钩藤(\textit{Uncaria})}
藤本。
不育花序梗成钩状,司攀登。
头状花序,无小苞片。

\subsubsection{玉叶金花(\textit{Mussaenda})}
直立或攀援灌木。
萼檐裂片中的一枚常扩大成有柄叶状体。
花冠黄色,雄蕊五,浆果。

\subsubsection{山黄皮(\textit{Randia})}
灌木或乔木,花多腋生。

\subsubsection{巴戟天(\textit{Morinda})}
木本,直立或攀援。
头状花序,聚花果。

\subsubsection{咖啡(\textit{Coffea})}
灌木或小乔木。
花冠高脚碟状。
浆果,种皮角质。
种子含咖啡碱。

\section{管花目(Tubiflorae)}
两性花,萼片五,花瓣五,雄蕊四或五,心皮二,子房上位。

\subsubsection{旋花科(Convolvulaceae)}
多藤本,或有乳汁。
单叶互生,无托叶。
两性花,辐射对称,单生或聚伞花序,有苞片。
萼片五,分离,覆瓦排列,宿存。
花冠钟状或漏斗状,五浅裂,开花前旋转排列。
雄蕊五,生于花冠基部,与花冠裂片互生。
子房上位,常为环状花盘包围,一至四室,胚珠一或二。
花柱顶生,柱头二。
蒴果或浆果,胚乳少。

\subsubsection{打碗花(\textit{Calystegia})}
萼片近相等,花萼包于两片大苞片内。
柱头二,长圆或椭圆,扁平。

\subsubsection{鱼黄草(\textit{Merremia})}
花冠黄色,瓣中有五条暗色的脉。
花粉粒无刺。
蒴果四瓣裂或不规则开裂。

\subsubsection{番薯(\textit{Ipomoea})}
花冠白、红或紫,瓣中两条脉。
花粉粒有刺,子房二或四室。
如甘薯(\textit{Ipomoea batatas})。

\subsubsection{牵牛(\textit{Pharbitis})}
萼片顶端长二狭,渐尖。
子房三室,胚珠六。

\subsection{紫草科(Boraginaceae)}

\subsection{马鞭草科(Verbenaceae)}
\subsection{唇形科(Labiatae)}
\subsection{茄科(Solanaceae)}
\subsection{玄参科(Scrophulariaceae)}
\subsection{爵床科(Acanthaceae)}
\subsection{苦苣苔科(Gesneriaceae)}

% 蓇葖果
\end{sloppypar}
\end{document}
