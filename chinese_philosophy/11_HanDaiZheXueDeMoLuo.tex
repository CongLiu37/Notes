\documentclass[11pt]{article}

\usepackage[UTF8]{ctex} % for Chinese 

\usepackage{setspace}
\usepackage[colorlinks,linkcolor=blue,anchorcolor=red,citecolor=black]{hyperref}
\usepackage{lineno}
\usepackage{booktabs}
\usepackage{graphicx}
\usepackage{float}
\usepackage{floatrow}
\usepackage{subfigure}
\usepackage{caption}
\usepackage{subcaption}
\usepackage{geometry}
\usepackage{multirow}
\usepackage{longtable}
\usepackage{lscape}
\usepackage{booktabs}
\usepackage{natbib}
\usepackage{natbibspacing}
\usepackage[toc,page]{appendix}
\usepackage{makecell}

\title{汉代哲学的没落}
\date{}

\linespread{1.5}
\geometry{left=2cm,right=2cm,top=2cm,bottom=2cm}

\begin{document}

  \maketitle

  \linenumbers
  
汉代本是中国历史上最为强盛的王朝之一,此处言其哲学,奈何判之以“没落”?
试请诸君不假思索地说出几位中国古代哲学家。
若非专门学者,所列哲学家之年代,大多在先秦与宋明。
由汉至唐,中国几无大思想家。
为理解这一现象,需先考察秦汉之历史环境。
秦燹之后,子学尽失,造成学术思想一大混乱,具体表现为儒家的堕落和道家的肢解。

\section{秦燹}
前221年,始皇26年,秦扫六合。
后秦相李斯倡议焚书。《史记·秦始皇本纪》载:

\textit{始皇置酒咸阳宫,博士七十人前为寿。仆射周青臣进颂曰:“他时秦地不过千里,赖陛下神灵明圣,平定海内,放逐蛮夷,日月所照,莫不宾服。以诸侯为郡县,人人自安乐,无战争之患,传之万世。自上古不及陛下威德。”始皇悦。博士齐人淳于越进曰:“臣闻殷周之王千馀岁,封子弟功臣,自为枝辅。今陛下有海内,而子弟为匹夫,卒有田常、六卿之臣,无辅拂,何以相救哉?事不师古而能长久者,非所闻也。今青臣又面谀以重陛下之过,非忠臣。”始皇下其议。丞相李斯曰:“五帝不相复,三代不相袭,各以治,非其相反,时变异也。今陛下创大业,建万世之功,固非愚儒所知。且越言乃三代之事,何足法也?异时诸侯并争,厚招游学。今天下已定,法令出一,百姓当家则力农工,士则学习法令辟禁。今诸生不师今而学古,以非当世,惑乱黔首。丞相臣斯昧死言:古者天下散乱,莫之能一,是以诸侯并作,语皆道古以害今,饰虚言以乱实,人善其所私学,以非上之所建立。今皇帝并有天下,别黑白而定一尊。私学而相与非法教,人闻令下,则各以其学议之,入则心非,出则巷议,夸主以为名,异取以为高,率群下以造谤。如此弗禁,则主势降乎上,党与成乎下。禁之便。臣请史官非秦记皆烧之。非博士官所职,天下敢有藏诗、书、百家语者,悉诣守、尉杂烧之。有敢偶语诗书者弃市。以古非今者族。吏见知不举者与同罪。令下三十日不烧,黥为城旦。所不去者,医药卜筮种树之书。若欲有学法令,以吏为师。”制曰:“可。”}

可见,最初的争议在于是否实行周朝之分封制度。
淳于越谓不法古者不能长久,提倡分封子弟功臣为辅。
李斯则以为当法今而非古,以儒生法古之论为“惑乱黔首”。
又谓战国之乱,其根源在于私学兴起,百家各执一词,遂生思想混乱。
臣民各以其学非议君王,损害君王权威;
又彼此结成党羽,难以驾驭。
故李斯提倡加强思想控制,维护君主权威,其主要措施则是将诗书及百家之文收于官府,禁止其在民间流传。
医药卜筮种树之书则不在禁止之列。
李斯所要求者, 实为将一切权力和是非标准收归于人主,以驾驭臣民,其主要立场与韩非同。
及秦亡,楚汉相争,阿房焦土,诸子之学弗传矣。

\section{献书}
汉兴,废秦法。
自孝惠降,政府致力于整理典籍,献书解经成风,经学遂兴。
《汉书·艺文志》载:

\textit{汉兴,改秦之败,大收篇籍,广开献书之路。迄孝武世,书缺简脱,礼坏乐崩,圣上喟然而称曰:“朕甚闵焉!”于是建藏书之策,置写书之官,下及诸子传说,皆充秘府。至成帝时,以书颇散亡,使谒者陈农求遗书于天下。诏光禄大夫刘向校经传诸子诗赋,步兵校尉任宏校兵书,太史令尹咸校数术,侍医李柱国校方技。每一书已,向辄条其篇目,撮其指意,录而奏之。}

汉儒所求之经,有老儒口诵录之者,用汉代通行文字,谓之今文经;
亦有先秦遗留简策,用战国古文字,谓之古文经。
今文经和古文经往往内容迥异,更兼子学失传,故汉代经学之训诂注释问题甚为繁重,学者多皓首于此,其哲学思想则难复精严。

\par

此外,自战国晚期,阴阳家之学渐与占卜合流。
秦之焚书,不禁卜筮。
故先秦诸学,独阴阳家不绝。
由此,阴阳五行理论渐为知识分子之共识,影响汉代哲学,造成汉代儒家之堕落。

\section{儒家的堕落}
先秦儒学,以孔孟为主脉,其价值根源归于人之自觉,强调道德之价值,将外在的道德和社会秩序归因于内在自觉。
汉儒受阴阳家影响,反以宇宙论为中心,兴“天人感应”之说,实以一“天”为超越主宰和价值根源,顺“天”即为有德,系原始信仰之阴霾,可谓之为堕落。

\section{道家的肢解}
先秦道家,以老庄为主脉,视道为万物存在之总原理,谓自觉心应脱离包括形躯的具体事物以明道,对世界持观赏之态度,达逍遥之境界,不受经验世界的影响。
老庄之学在汉代分裂为三个部分,皆误执形躯,非老庄本意。

\par

第一是道教对长生和神通的追求。
老庄论自我境界,皆不以形躯为意。
道教不解观赏世界之逍遥,反误执形躯,求长生之术。
老庄谓明道之自我超越具体事物,不受其影响,反对经验世界有支配力量。
道教则将其曲解为某种神秘力量,即所谓神通,例如呼风唤雨、撒豆成兵,并追求此种支配力量。

\par

第二是以观照之智慧行权术阴谋。
先秦道家谓自觉心观道,遂生支配力量,具体举措为无为守柔。
但老庄之价值根源在于万物之自然流转,以无为守柔之力量干预世界并无意义。
对道家来说,此种力量只是自觉心超越万物的自然结果,自身并无意义。
但后人揪住此种支配力,以此行权术阴谋,大失老庄本旨。

\par

第三是放荡之风的兴起。老庄仅肯定一观赏世界之自我,否定道德及知识。
后人不解其精意,独扣住对道德和知识的否定。
否定道德,故反对一切规范;
否定知识,故不求学问之进步。
到头来耽于纵欲行乐,行为放荡,谓之风流,全然不解老庄观赏世界之逍遥。
这一倾向,于汉末方见端倪,兴于魏晋。

\end{document}