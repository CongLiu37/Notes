\documentclass[11pt]{article}

\usepackage[UTF8]{ctex} % for Chinese 

\usepackage{setspace}
\usepackage[colorlinks,linkcolor=blue,anchorcolor=red,citecolor=black]{hyperref}
\usepackage{lineno}
\usepackage{booktabs}
\usepackage{graphicx}
\usepackage{float}
\usepackage{floatrow}
\usepackage{subfigure}
\usepackage{caption}
\usepackage{subcaption}
\usepackage{geometry}
\usepackage{multirow}
\usepackage{longtable}
\usepackage{lscape}
\usepackage{booktabs}
\usepackage{natbib}
\usepackage{natbibspacing}
\usepackage[toc,page]{appendix}
\usepackage{makecell}

\title{真常之学}
\date{}

\linespread{1.5}
\geometry{left=2cm,right=2cm,top=2cm,bottom=2cm}

\begin{document}

  \maketitle

  \linenumbers

印度大乘佛教除中观、唯识二支外,尚有论及“如来藏”之经文,似承认一真常之主体。
此可视为大乘佛教的第三支,即真常之学,所依经文包括《妙法莲华经》(即《法华经)、《大方广佛华严经》(即《华严经》)和《大般涅槃经》(即《涅槃经》)。
理论内容来看,真常之学正面肯定主体之自由,实能统摄大乘佛教教义,在理论上应有极高地位。
然而,三经在印度均无著名疏论,唯中国僧人依此立论开宗。
故真常之学独盛于中国。
本篇就印度原有文献,撮述真常一系的观点,结束对印度佛教教义的叙述。

\section{一乘}
所谓“一乘”,即一切众生皆可成佛之主张。
传统佛教本有“声闻”“缘觉”“菩萨”三乘,而以“佛”和“菩萨”为同一等级。
真常之学主张,佛之说法立教,只有一目的,即使众生皆能成佛。
《法华经》云:

\textit{舍利弗,如来但以一佛乘故,为众生说法;无有余乘,若二,若三。}

此即明白宣示一乘之义。

\par

但一乘观念显然与传统不合。
对此,《法华经》称,立三乘教义是为引导众生之方便,实际上只有“一佛乘”。
盖众生禀赋能力有高低之分,故立种种法门,欲使其易于受益。
《法华经》云:

\textit{佛平等说,如一味雨,随众生性,所受不同;如彼草木,所禀各异。佛以此喻,方便开示,种种言辞,演说一法。}

故佛以种种言辞演说一法,使一切众生得以各随其自身能力而得开悟,以使众生得同样的无上正果,非谓众生中根性愚劣者仅能得较低等级之果。
《法华经》云:

\textit{今为汝等,说最实事:诸声闻众,皆非灭度;汝等所行,是菩萨道,渐渐修学,悉当成佛。}

此即辨明天赋禀性只影响入门之难易,不影响最终成就。

\par

倘若就理论脉络而言,佛教素来否定一切独立实有,仅肯定最高主体性之自由。
则此一最高主体,断不能受任何外在条件之限制,否则主体之最高不能成立。
如此则众生之天赋禀性皆不能影响主体之显现,故众生皆能成佛。
因此,“一乘”观念实为佛教教义发展之必然归宿。

\section{佛性}
众生皆能成佛,即谓众生皆有成佛之能力,即“佛性”。
印度佛经中,唯《大般涅槃经》对此一观念详加论述。
《涅槃经》云:

\textit{知无有我无有我所,知众生皆有佛性;以佛性故,一阐提等舍离本心,悉当得成阿耨多罗三藐三菩提。}

“一阐提”即梵文Icchantika,意为“不具信”,或称“断善根”。
“阿耨多罗三藐三菩提”即梵文Anuttara Samyak Sambodhi,意为无上正等正觉。
印度传统观念中有“一阐提不能成佛”之说,而《涅槃经》谓众生皆有佛性,皆可得阿耨多罗三藐三菩提。
此说直承《法华经》“一乘”理论。

\par

众生皆有佛性,是谓众生皆有成佛之能力,非谓众生皆必然成佛。《涅槃经》云:

\textit{善男子,如汝所言,以何义故名佛性者;善男子,佛性者,即是一切诸佛阿耨多罗三藐三菩提中道种子。}

引文中“中道”指超越“有”和“无”之真正自我或者说主体性,承中观之学。
谓佛性为“种子”,便是区分“佛性”和“成佛”。
佛性为阿耨多罗三藐三菩提之种子,则此种子之存在是一个问题,而种子之发用是另一问题,二者互不妨碍。
换言之,众生皆有成佛之能力,而此能力是否发用以及发用之效果如何,并不妨碍此能力的存在。
这种“能力”和“能力之发用”的区分,为一切价值理论必须加以澄清之处。
此理本不难明,但如不仔细分辨,易生迷乱。

\par

佛性既存在于主体,则其发用亦完全由主体决定。
换言之,能否成佛,系自觉努力之问题。
然此种自觉努力如何为可能?
依佛教教义,一切法皆非独立实有,皆为虚妄。
此种虚妄,是主体活动的产物。
主体如此活动,其结果是主体陷溺于众法。
然而,主体一旦明众法之虚妄,遂停止此中活动,则主体之最高自由即刻显现,佛性亦得以发用。

\par

佛性之发用,结果为主体性之完满实现,即“法身”。
此境界为主体最高自由之境界,故不能视之为对象,不能加以思考或感知。
此主体为含有一切之至高,即一切法皆为主体所立。
  
\end{document}