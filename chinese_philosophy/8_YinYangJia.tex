\documentclass[11pt]{article}

\usepackage[UTF8]{ctex} % for Chinese 

\usepackage{setspace}
\usepackage[colorlinks,linkcolor=blue,anchorcolor=red,citecolor=black]{hyperref}
\usepackage{lineno}
\usepackage{booktabs}
\usepackage{graphicx}
\usepackage{float}
\usepackage{floatrow}
\usepackage{subfigure}
\usepackage{caption}
\usepackage{subcaption}
\usepackage{geometry}
\usepackage{multirow}
\usepackage{longtable}
\usepackage{lscape}
\usepackage{booktabs}
\usepackage{natbib}
\usepackage{natbibspacing}
\usepackage[toc,page]{appendix}
\usepackage{makecell}

\title{阴阳家}
\date{}

\linespread{1.5}
\geometry{left=2cm,right=2cm,top=2cm,bottom=2cm}

\begin{document}

  \maketitle

  \linenumbers

阴阳家之学起源于方术,以宇宙论为主。
此类学说带有迷信色彩。
盖古代宇宙论想象成分居多,不乏诉诸超自然力量者。
于自然科学发达之今日观古代宇宙论,其幼稚浅薄不言自明。
此为古代哲学常有之事,不足为奇,亦不足为病。
然而,方术之学终究是与事实打交道,以积极态度解释客观世界,致力于诠释其规律,预测其发展,有征服自然之态度。
此类学说代表科学精神,在历史发展中往往成为古代科学之萌芽。

\par

先秦时期,此类学说大致可分为论阴阳者和论五行者。
这两种思想各自独立发展,而后逐渐交汇,统称为阴阳家。
《汉书·艺文志》谓阴阳家之典籍,有二十一家,三百六十九篇,今已失传殆尽。
现论阴阳家,所依资料系《易经》以及包括《易传》《尚书》《礼记》《吕氏春秋》等杂辑之籍中涉及阴阳五行及宇宙论者。
此类资料,时代难定。
《易经》之卦爻体系反映上古原始思想;其余较晚近者,最晚可至汉初,但大部分可推测为战国之著作,至少反映了战国时期的思想。

\par

阴阳家于战国末年盛行,且渐与卜筮合流。
秦始皇焚书,不去卜筮。
故经秦燹,诸子之学皆绝而阴阳家独存。
自汉以降,阴阳家之学混入儒道,反蔽孔老精意。
  
\section{卦爻}
言阴阳者多称《易》。
今传《易》中,《易经》言卦爻,系上古时代宇宙论之资料,其时代不晚于周初;《易传》始论阴阳,系自战国至汉杂辑而成。
此处论《易经》之卦爻体系。

\par

《易经》本占卜之书。
殷人以甲骨受火龟裂之纹路占卜,即为“卜”。
此类纹路千变万化,不易解读。
周以降,“卜”逐渐被淘汰,“筮”逐渐流行。
所谓“筮”,即取一束蓍草,每次从中取出两株。
最后所剩,若为单数,则以“—”记之;若为双数,则以“--”记之。
“—”和“--”即为“爻”。
“卦”则为三“爻”上下排列而成,故共有八“卦”,即乾坎艮震,巽离坤兑。
其形状为“乾三连,坤六断,震仰盂,艮覆碗,离中虚,坎中满,兑上缺,巽下断”。
八“卦”中任意两“卦”上下排列,遂有六十四“重卦”。
重复蓍草占卜六次,便可得一重卦。
对照《易经》中的卦爻辞,便可解其含义。

\par

卦爻之组织本是一种符号游戏,但重卦之命名与排列反映一种宇宙秩序的观念。
六十四重卦各代表宇宙历程之某一阶段,其中各爻又代表此一阶段中的某一部分。
合而观之,便是整个宇宙之运行历程。
其中,“乾”“坤”为重卦之首。
“乾”指“发生”;“坤”本意为“地”,指发生之质料。
“乾”“坤”为重卦之首,即发生之形式动力与所凭之质料为宇宙过程之基本条件。
自此,宇宙经历各个阶段,以“既济”和“未济”结束。
“既济”即已经完成。
然而宇宙系无限之时空,其运行永不停止,故最后以“未济”,即未完成结束。

\par

重卦既用于占卜,则其意义不仅指宇宙秩序。
六十四重卦所代表的阶段,既属宇宙历程,又属人生历程。
由此,宇宙历程和人生历程遂有对应关系。
换言之,六十四重卦谓宇宙历程和人生历程各有六十四给阶段,而每个阶段又有六个小阶段。

\par

再论卦爻辞。
卦爻辞本是解释各卦或各爻之吉凶或意义,体例颇不一致。
其中所透露出的理论,大致可概括为“物极必反”和“居中为上”两点。
“物极必反”表现为:若重卦为吉,其最后一爻反凶;若重卦为凶,其最后一爻反吉。
“居中为上”则表现为重卦之第二、第五爻多吉。
盖每一重卦由两卦组成,则第二、第五爻分别为两卦之正中,多为吉爻。
此可视为基于宇宙论的价值观。
  
\section{阴阳}
阴阳之观念主要见于《易传》。
其大意为,阴和阳两种相反相成的力量相互作用,产生宇宙万物并驱动其运行。
此一观念颇为古老。
《国语·周语上》载周幽王二年(780 BC)地震,史官伯阳父评论曰:

\textit{阳伏而不能出,阴迫而不能烝,于是有地震。}

此即以阴阳运行解释自然现象。

\par

《易传》中,阴阳和卦爻相关联,谓“—”为阳爻,“--”为阴爻。
如此则八卦各分阴阳。
其中乾之三爻皆为阳,故乾为纯阳;坤之三爻皆为阴,故坤为纯阴。
其余各卦,皆为乾和坤相交而成。
故乾和坤被视为父母,其余诸卦皆为其子女。
《易传·说卦传》云:

\textit{乾天也,故称父,坤地也,故称母;震一索而得男,故谓之长男;巽一索而得女,故谓之长女;坎再索而男,故谓之中男;离再索而得女,故谓之中女;艮三索而得男,故谓之少男;兑三索而得女,故谓之少女。}

更进一步,《易传》将八卦与具体事物相对应,其大要为:乾为天,坤为地,震为雷,巽为风,坎为水,离为火,艮为山,兑为泽。
由此,则乾和坤之相交产生其余六卦,即阳和阴相交,产生世间万物。
  
\section{五行}
五行之说,现有最早的可靠资料是《尚书·洪范》。
此篇传说是武王伐纣后,箕子为武王讲述传自大禹的治国之法。
现代学者考《洪范》之年代,以为当在公元前四世纪。

\par

《尚书·洪范》所描述的五行,为水、火、木、金、土。
又谓人类社会和自然世界相关联,国君的恶行会伴随自然界的不正常现象,如地震、洪水、干旱等。
需要指出的是,《洪范》中的五行观念比较原始,仍局限于具体的水、火等事物。
后世言五行,则指五种相生相灭的抽象力量,仍冠以水、火、木、金、土之名。

\par

《礼记·月令》进一步发展《尚书·洪范》中的五行观念,将五行和时间、空间相对应,并强调人类活动和自然现象之关系。
《月令》谓火为南方,为夏季,盖南方和夏天比较热;
水为北方,为冬季,盖北方和冬季比较冷;
金为西方,为秋季,盖秋季草木凋零,盛行西风,且五行中木被金克;
木为东方,为春季,盖春季草木茂盛,盛行东风;
土在时间上为一年之中,即夏末秋初的一段时间,在空间上则为中央。
此外,五行之相互作用亦是四季流转之原因。盖木生火,火生土,土生金,金生水,水生木;一年四季亦由此周而复始。
由此,一个由五行观念所构建的宇宙之时空构架得以完成。

\par

《月令》还将自然现象和人类活动相关联,规定君王每个月应当做什么。
此类规定大可涉及政治活动,如政令之发布或者战争;小可包括日常生活琐事,如君王之衣食住行。
反之,如果君王违背此类规定,则会引起不正常的自然现象。

\par

至此,五行与宇宙之时空框架相关联,而时空又与人类活动相关联。
《吕氏春秋·有始览》更进一步,以五行解释朝代更迭:
  
\textit{凡帝王者之将兴也,天必先见祥乎下民。黄帝之时,天先见大蚓大蝼。黄帝曰:“土气胜。”土气胜,故其色尚黄,其事则土。及禹之时,天先见草木秋冬不杀。禹曰:“木气胜。”木气胜,故其色尚青,其事则木。及汤之时,天先见金刃生於水。汤曰:“金气胜。”金气胜,故其色尚白,其事则金。及文王之时,天先见火赤乌衔丹书集于周社。文王曰:“火气胜。”火气胜,故其色尚赤,其事则火。代火者必将水,天且先见水气胜。水气胜,故其色尚黑,其事则水。水气至而不知数备,将徙于土。}

此将朝代兴衰归因于五行生克,且以此规律对未来作一预测。
至此,宇宙运行和人类活动皆归因于五行生克。
  
\section{数字}
阴阳和五行在理论上相汇于数字。
《易传》将宇宙之运行归于阴阳,而又将阴阳归于数字,谓单数为阳,偶数为阴。
五行亦产生并完成于数字:水生于一,完成于六;火生于二,完成于七;木生于三,完成于八;金生于四,完成于九;土生于五,完成于十。
如此,阴阳和五行借由数字相联系。
  
\end{document}