\documentclass[11pt]{article}

\usepackage[UTF8]{ctex} % for Chinese 

\usepackage{setspace}
\usepackage[colorlinks,linkcolor=blue,anchorcolor=red,citecolor=black]{hyperref}
\usepackage{lineno}
\usepackage{booktabs}
\usepackage{graphicx}
\usepackage{float}
\usepackage{floatrow}
\usepackage{subfigure}
\usepackage{caption}
\usepackage{subcaption}
\usepackage{geometry}
\usepackage{multirow}
\usepackage{longtable}
\usepackage{lscape}
\usepackage{booktabs}
\usepackage{natbib}
\usepackage{natbibspacing}
\usepackage[toc,page]{appendix}
\usepackage{makecell}

\title{周敦颐}
\date{}

\linespread{1.5}
\geometry{left=2cm,right=2cm,top=2cm,bottom=2cm}

\begin{document}

  \maketitle

  \linenumbers

儒学复兴始于唐代末期,经宋至明,为中国哲学发展之主流。
其旨在于回归孔孟之心性论哲学。
韩愈、李翱为此运动之序幕人物。
至北宋初年,学者已有自造系统抗拒佛教之意,却多混杂宇宙论与形而上学,以此为价值理论之基础。
此为儒学复兴运动早期,以周敦颐、张载等为代表。
儒学复兴运动中期以程颐、程颢、朱熹等为代表。
其学说已洗去宇宙论成分,将价值理论建议在纯粹形而上学的基础上。
晚期则始于南宋陆九渊,集大成于王阳明。
最高主体自由得明。

\newline

本篇论周敦颐之学。周敦颐(1017-1073),字茂叔,号濂溪,其学见诸《太极图说》与《通书》。

\section{宇宙论}
周敦颐之宇宙论主要见诸《太极图说》,其文曰:

\textit{无极而太极。太极动而生阳,动极而静,静而生阴,静极复动。一动一静,互为其根。分阴分阳,两仪立焉。阳变阴合,而生水火木金土。五气顺布,四时行焉。五行一阴阳也,阴阳一太极也,太极本无极也。}

\textit{五行之生也,各一其性。无极之真,二五之精,妙合而凝。乾道成男,坤道成女。二气交感,化生万物。万物生生而变化无穷焉。}

\textit{唯人也得其秀而最灵。形既生矣,神发知矣。五性感动而善恶分,万事出矣。圣人定之以中正仁义而主静,立人极焉。}

\textit{故圣人“与天地合其德,日月合其明,四时合其序,鬼神合其吉凶”,君子修之吉,小人悖之凶。故曰:“立天之道,曰阴与阳。立地之道,曰柔与刚。立人之道,曰仁与义。”又曰:“原始反终,故知死生之说。”大哉易也,斯其至矣!}

第一节言宇宙之发生历程。
“无极”产生“太极”;
“太极”之“动”生“阳”,其“静”生阴;
“阳变阴合”而生五行。
第二节言阴阳五行“化生万物”。

\newline

第三、四节涉及价值理论。
需要注意的是,从“无极”到“万物生生”,再到“唯人也得其秀而最灵”,均系对事实的描述,仅涉及“实际如此”,不涉及“应该如此”,即不涉及价值观念。
纵然言人“秀而最灵”,亦只是化生万物的一种状态,仅仅是一事实,不能产生善恶之分。
此为周敦颐学说之困难所在。

\section{“诚”}
周敦颐《通书》论“诚”,曰:

\textit{诚者,圣人之本。}

此“诚”为形而上学概念。
下文就此发挥,曰:

\textit{大哉乾元,万物资始,诚之源也。}

又曰:

\textit{乾道变化,各正性命,诚斯立焉。纯粹至善者也。}

此处引《易》,谓“诚”为“乾元”“天道”之内容。
盖“诚”即本性之充分实现。
就万有之根源而言,根源创生万有,遂有本性实现之意,故曰“诚之源”。
就已出现之万有而言,万有“各正性命”,实现本性,故曰“诚斯立焉”。
“纯粹至善”则言“诚”为价值判断之基础,即以“天道”之方向为最高价值。

\newline

“诚”亦为成圣之工夫。
《通书》云:

\textit{圣,诚而已矣。诚,五常之本,百行之源也。}

此谓“诚”为一切德性之基础,“诚”之实现为最高境界。
如此则必须解释错误罪恶如何成为可能。

《通书》曰:

\textit{五常百行,非诚,非也。邪暗塞也。}

此谓“邪暗”之“塞”,使得“诚”不能达成,遂生邪恶。
《通书》进一步以“几”诠释“邪暗塞”,曰:

\textit{寂然不动者,诚也;感而遂通者,神也;动而未形有无之间者,几也。}

此处周敦颐预设人之自觉心。
“寂然不动者,诚也;感而遂通者,神也”言自觉心遵照“天道”之状态。
“动而未形有无之间者,几也”则言自觉心可依“天道”,亦可背离之。

\newline

如此,成圣工夫全在自觉心之发用,而周敦颐以一“动”字言之。
《通书》曰:

\textit{动而正曰道,用而和曰德。}

既以“动”之“正”释德性,则一切邪恶皆出自“动”之不”正“。
故《通书》曰:

\textit{邪动,辱也;甚焉,害也。故君子甚动。}

\newline

至此,可对周敦颐之价值理论作一总结。
周敦颐先设定一形而上的至高之理,其内容为“诚”。
此理创生万物,亦为万物演进之方向。
又谓依理而动,遂得正和,而为圣人境界;
反之则有辱害,邪恶生焉。
然人在其宇宙论构架中的位置为一大问题。
理既为创生万物之至高,则人不能背理而行。
此问题只能就主体自由一面加以论述,而周敦颐未加注意,仅言人为理所创生之万物中“秀而最灵”者。

\section{其他理论}
《通书》言“性”,全然就“才性”而言,仅将其视为客观事实,并以朦胧语说之。。

\newline

周敦颐论学,或者说成圣工夫,仅言“无欲”二字。
“无欲”成圣,得“明通公溥”。
则”有欲“遂坠凡。
然此”无欲“意义不明。
依周敦颐之宇宙论,一切存有皆为被绝定者,皆受“无极太极阴阳五行万物”这一历程之支配。
就此宇宙历程言“无欲”“有欲”,仅为客观事实,不能支持价值判断。
至于“明通公溥”四字,意义不明。

\newline

周敦颐论天人关系,谓“天”有“仁”“义”。
圣人遵天道,亦有“仁”“义”。
此说与汉儒观念相通。
若不直面主体性,此类问题终不可解。

\end{document}