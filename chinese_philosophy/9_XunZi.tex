\documentclass[11pt]{article}

\usepackage[UTF8]{ctex} % for Chinese 

\usepackage{setspace}
\usepackage[colorlinks,linkcolor=blue,anchorcolor=red,citecolor=black]{hyperref}
\usepackage{lineno}
\usepackage{booktabs}
\usepackage{graphicx}
\usepackage{float}
\usepackage{floatrow}
\usepackage{subfigure}
\usepackage{caption}
\usepackage{subcaption}
\usepackage{geometry}
\usepackage{multirow}
\usepackage{longtable}
\usepackage{lscape}
\usepackage{booktabs}
\usepackage{natbib}
\usepackage{natbibspacing}
\usepackage[toc,page]{appendix}
\usepackage{makecell}

\title{荀子}
\date{}

\linespread{1.5}
\geometry{left=2cm,right=2cm,top=2cm,bottom=2cm}

\begin{document}

  \maketitle

  \linenumbers
  
荀子(289 BC-238 BC),名况,字卿,赵人。
荀子之学,以文化为中心,强调文化之价值,却苦于文化之起源,最终被迫将其归因于圣人,衍生出尊君之说,遂逐坠入权威主义。
此说后为法家所继承,故李斯、韩非皆荀门子弟。
秦遵循法家之学说,巩固自身之统治,建一强力秩序,遂有秦燹之祸。
及汉兴,诸子之精义不传,新学说无法突然降生,中国思想遂陷入混乱。
此一历史演进,其伏笔埋于荀子之学。

\par

然考荀子及法家之学,自有其历史必然性,不可仅就其导致子学之终结而非之。
周人建礼法,其本来目的是构建一强有力的中央政府,加强中央对地方的控制,遂生出“大一统”的理念。
所谓“大一统”,即“天下”应当统一,诸侯混战为一不正常之乱象。
此为诸子之共识。
此处需强调,古籍所言“天下”,实指“世界”,尽管古人所知仅限于中国。
换言之,所谓“大一统”,即世界之统一。
对大一统之渴求在春秋战国时期不断增长,于战国末年达到顶峰。
此种历史条件下,权威主义自然有其市场。
由此观之,荀子之崇君及韩非对建立绝对君权的追求,皆反映了对大一统之渴求,为先秦历史发展之必然。

\section{性恶}
性恶为荀子理论中最为人熟知者,俗论多谓其与孟子性善论针锋相对。
但荀孟所言之“性”,其含义本不相同。
孟子所言之“性”,为人之本质,系人之定义性属性。
有此“性”即为人,无此“性”即不为人。
此“性”为价值根源,是德性之种子。
荀子所言之性,则是人“生而有”者。
就范围论,人生而有者包含人之本质,亦包含动物性本能。
由此则性恶之说无难矣。
《荀子·性恶》载:

\textit{人之性恶,其善者伪也。今人之性,生而有好利焉,顺是,故争夺生而辞让亡焉;生而有疾恶焉,顺是,故残贼生而忠信亡焉;生而有耳目之欲,有好声色焉,顺是,故淫乱生而礼义文理亡焉。然则从人之性,顺人之情,必出于争夺,合于犯分乱理,而归于暴。}

此谓人性为恶,故有导致混乱冲突之倾向。

\section{文化之重要性}
人性本恶。
顺从人性只会导致混乱,社会便无从谈起。
然而,荀子谓人之生活必须依赖于一定形式的社会组织。
《荀子·王制》云:

\textit{力不若牛,走不若马,而牛马为用,何也?曰:人能群,彼不能群也。人何以能群?曰:分。分何以能行?曰:义。故义以分则和,和则一,一则多力,多力则强,强则胜物。}

此谓人本身是一弱小的生物,但因群居而强大,故能使自然为己所用。
此外,人需要在社会组织中互相帮助、支持彼此来提高生活水平。
《荀子·富国》云:

\textit{故百技所成,所以养一人也。而人不能兼技,人不能兼官,离居不相待则穷。}

此谓个人之能力有限,若不互相合作,难免于穷。

\par

总之,为了生存与生活,需要社会组织。
然而人皆有欲,故求其所喜,远其所恶。
考自然界,人之所喜者寡,人之所恶者多。
此种现实条件下,由于人性本恶,共同生活之人必然为满足自身之欲望而无所不用其极,则争斗冲突在所难免,社会组织亦无法构建。
为解决此种状况,文化应运而生。
人受文化熏陶,故能克服其性之恶,不汲汲于欲望之满足,社会组织遂可形成。

\par

至此,文化仅为组织社会之工具。
文化之所以重要,是因为社会组织之重要。
然荀子并不止步于此,而是从宇宙论的角度,再次强调文化的重要性。
荀子谓天、地、人为构成宇宙的三种力量,各有其分工。
天地之作用系自然界之运行;而人的作用便是利用自然创造文化,此为与自然运行同等重要之事。
故天、地、人三者在宇宙中同等重要,无高下之分。
由此,荀子肯定人之地位,强调文化之价值。
此为古代思想所罕有之主张。
盖古代科学技术落后,人面对不可解释之自然现象,往往持畏惧态度,进而将其神秘化,遂生各种迷信;
亦或由此否定人之意识和活动,谓人于自然界不可能有所成就,遂生消极之态度。
荀子从宇宙论的角度言文化之价值,表现出对人的高度自信,其气魄胸襟不可谓不宏大。

\section{文化之内容}
荀子谓文化为人生活所需,且亦是人在宇宙中的地位之凭据。
则荀子所言之“文化”,其内容究竟为何?
答曰:礼。
荀子屡屡强调礼之重要性。
《荀子·劝学》云:

\textit{故书者,政事之纪也;诗者,中声之所止也;礼者,法之大兮,类之纲纪也。故学至乎礼而止矣,夫是之谓道德之极。}

《荀子·礼论》载:

\textit{礼起于何也?曰:人生而有欲,欲而不得,则不能无求。求而无度量分界,则不能不争;争则乱,乱则穷。先王恶其乱也,故制礼义以分之,以养人之欲,给人之求,使欲必不穷于物,物必不屈于欲。两者相持而长,是礼之所起也。}

此处言“礼”,取制度义。
谓人有欲望,遂生争执。
礼法既立,人皆服从于某一制度,乱即平矣。

\par

总之,荀子所谓之人类文化,尤重社会制度。
有此一制度,则人之欲望可依照制度得到限制,争斗冲突可免,而社会组织生焉,人之生存以及生活之改善成为可能,人在宇宙中的尊贵地位亦得确立。

\par

然而,礼亦可指仪式。
荀子所言之礼,亦包括仪式、礼仪。
盖儒者所言之仪式,往往不出周礼之范围。
产生于周初的仪式,不可避免地受原始信仰的影响,带有迷信和神话色彩。
儒家对这种迷信往往持否定态度。
孔子虽重视礼法,却不语怪力乱神,反对鬼神信仰。
又将礼仪从原始信仰中剥离,将其归于仁义。
更兼将祭祀祖先解释为因祖先之功业于今日之生活有所贡献,故需通过此类仪式向祖先酬恩。
此类学说便暗含对迷信的否定。
荀子则进一步明确赋予这些仪式新的含义,将其中的迷信色彩诠释为情意之体现。
以古代颇为普遍的祭祀为例。
从智性上可知,人死不能复生,灵魂永生亦不可知,故祭祀死者本是自我欺骗,并无意义。
但人尚有情感之需求。
人之情感希望死者可以复生,或者以灵魂的形式永生,故祭祀死者是表达对死者的尊敬和思念,是情感之宣泄。
在祭祀仪式中,人是在欺骗自我,且知道这是自我欺骗。
由此,祭祀仪式即由迷信转变为情感之宣泄。
荀子对其它仪式,亦持此种态度,如为求降雨而祈祷,为做重大决定而占卜,皆是表示重视;而不是认为祈祷可导致降雨,或者占卜可预知未来。

\par

如此,荀子将祭祀、求雨、占卜等充满宗教色彩的仪式解释为诗意而否定其中的宗教色彩。
盖宗教与诗意皆为想象,皆为人类情感之宣泄,为生命情意之体现。
僧侣在想象之时,并不知道他正在想象,遂将想象等同于现实,而生种种迷乱与荒谬。
而诗人在想象之时,明确知道他正在想象,故不会混淆想象与现实。
诗人之想象,仅仅作为审美之对象,不能取代人之积极进取。
荀子对周礼的诠释,实际上是以诗意取代宗教,以美学取代宗教。

\section{文化之实践}
文化终究是限制人之欲望,与性之恶相悖。
由此则引出一实践性问题,即人如何能够按照文化之要求限定自身之行为。
换言之,人如何能够循礼?
如何将文化付诸实践?
此一功夫,荀子概括为一“学”字,即人需受文化之熏陶改造,方能限制其性之恶,而成就其善。
故荀子强调教化和改造历程。
《荀子·性恶》谓:

\textit{令人之性恶,必将待师法然后正,得礼义然后治。}

\par

人既待为学而后善,则人自身必有学习之能力。
荀子将此能力称为“心”。
《荀子·解蔽》云:

\textit{心者,形之君也,而神明之主也,出令而无所受令。自禁也,自使也,自夺也,自取也,自行也,自止也。故口可劫而使墨云,形可劫而使诎申,心不可劫而使易意,是之则受,非之则辞。}

此强调“心”之主宰性。
可据此而言,“心”为人之自觉主宰性,为人之本质,与孟子所论之“性”意义相同。

\par

然而,孟子之“性”为善之源头,为万理之源。
荀子之“心”,只能观照于理,而非理之源头。
《荀子·解蔽》载:

\textit{故人心譬如盘水,正错而勿动,则湛浊在下,而清明在上,则足以见鬒眉而察理矣。微风过之,湛浊动乎下,清明乱于上,则不可以得大形之正也。心亦如是矣。故导之以理,养之以清,物莫之倾,则足以定是非决嫌疑矣。}

此处以水喻心。
水可照见万物,而万物皆在水外;一如“心”之观理,而理在“心”外,故可以理导“心”。

\par

荀子进一步强调,人人皆有学习文化之能力。
《荀子·性恶》云:

\textit{凡禹之所以为禹者,以其为仁义法正也。然则仁义法正有可知可能之理。然而涂之人也,皆有可以知仁义法正之质,皆有可以能仁义法正之具,然则其可以为禹明矣。}

至此,荀子强调人需通过学习文化改造其性之恶,而后方可为善;且此学习文化之能力为人之本质。

\par

此外,荀子理论中尚有涉及逻辑学和知识论者。
依荀子,此类理论有助人为学之功效。
此处所谓的“学”,指学习文化,尤其是学习礼,学习社会制度和各种仪式,而非思辨之学。
故在荀子理论中,逻辑学和知识论仅有助学之用,属文化实践之工具,为学说之附属。
尽管如此,荀子论此类问题,颇有独到之处。兹于此论之。

\par

先论荀子之逻辑学,自“名”开始。
《荀子·正名》谓:

\textit{名无固宜,约之以命,约定俗成谓之宜,异于约则谓之不宜。名无固实,约之以命实,约定俗成,谓之实名。}

此谓一切名皆系约定而来。
对一类事物,命之以一名,则此名之意义即定。
这一命名过程本身只是将一符号与一类事物对应,二者之关系完全系人为约定。
名实之对应关系既定,沿用成俗,故此名有固定意义,不可违背。

\par

故“名”本质系人为约定的与事物对应之符号。
荀子更进一步,论名之范围。
《荀子·正名》云:

\textit{故万物虽众,有时而欲无举之,故谓之物;物也者,大共名也。推而共之,共则有共,至于无共然后止。有时而欲偏举之,故谓之鸟兽。}

此处“共”与“别”指普遍和特殊,“大”则就名之范围而言。
所谓“大共名”,即一切事物皆有的“物”之名,为包括一切事物的最普遍之概念,故谓“共”与“大”。
“大别名”则指除“大共名”外范围较广之概念。
例如“鸟兽”之名,非“大共名”,故言其“别”;然其所包括之事物不可谓不多,故言其“大”。

\par

荀子如此论名后,即对各种诡辩进行反驳。
依荀子,诡辩之错误可分为三类。
第一类诡辩是以名乱名。
此类诡辩系混淆各概念之范围,澄清所涉及概念之定义条件即可破之。
例如,“杀盗非杀人”,谓“盗”与“人”不同,故杀盗算不得杀人。
然所谓“盗”,为“有劫夺行为之人”,故“盗”为包含于“人”之概念。
由此,杀盗为杀有劫夺行为之人,亦是杀人。
第二类诡辩是以实乱名。
此类诡辩就事物差异之相对性而生诡异之论,就所涉及的认知条件观察事物即可破之。
盖荀子强调事物差异之客观性和人对客观差异的认知能力。
例如,“山与渊平”,谓山之高与渊之低皆相对某一标准而言;若采用不同标准,则山可低,渊可高,二者相平矣。
然以视力认知山与渊,则山高渊低,明矣。
第三类诡辩是以名乱实。
此类诡辩混淆各概念之关系,对其作一澄清即可破之。
例如,“白马非马”,“白马”和“马”为两个不同概念,故不相等,即为相离。
对此只需澄清两概念除相等或相离外,亦可相交、相包含,则此类诡辩破矣。

\par

再论荀子之知识理论。
荀子之立论尚嫌简陋,但其经验主义立场甚明。
荀子谓人依靠感觉产生对事物之印象,经人之自觉心进行归纳整理,作出解释,遂有知识。

\section{文化之根源}
至此,荀子之理论以文化为核心。
荀子谓人性为恶。
而人之所以能作价值判断,可言事物之正当与否而不是仅仅顺从人性、言其是否有利,系文化之功效。
如此,人之价值观源于文化,而文化必有一起源。
则文化之根源为何?
换言之,荀子哲学之价值根源最终究竟落于何处?
此为荀子的文化哲学得立之根本问题,不可不作回答。
荀子对此重大问题,纠结颇苦,最终亦无一合理解答,可谓荀子哲学之一大败笔。

\par

首先,荀子不以形躯之利为价值根源。
荀子论性恶时,已言逐利为恶。
《荀子·荣辱》又明确贬利崇义:

\textit{先义而后利者荣,先利而后义者辱。}

《荀子·不苟》论小人,又云:

\textit{言无常信,行无常贞,唯利所在,无所不倾,若是则可谓小人矣。}

可见,荀子不以求形躯之利为价值。

\par

其次,荀子不以人之自觉为价值根源。
荀子谓自觉心只有观照于理、学习文化之能力,自身并不包含理,非价值根源。
此点前文已述。

\par

此外,荀子又坚持所有人之性皆相同。
《荀子·性恶》云:

\textit{凡人之性者,尧舜之与桀跖,其性一也。君子之与小人,其性一也。}

因此,不应当有某些能力超群之人取创造文化。
至此,价值根源必不在人自身。

\par

荀子亦不以“天”为价值根源。
《荀子·天论》云:

\textit{大天而思之,孰与物畜而制之?从天而颂之,孰与制天命而用之?}

又谓:

\textit{强本而节用,则天不能贫;养备而动时,则天不能病;修道而不贰,则天不能祸。}

由此观之,荀子不以自然事物之运行为价值根源,遂有“制天”之说;且反对将人事归因于“天”,不立一宗教式的超越性主宰,故有“天”不能“贫”“病”“祸”之说。
总之,荀子不以自然之“天”或人格化之“天”为价值根源。

\par

价值根源不落于形躯之利,此为诸子共识,不足为奇。
倘落于人之自觉心,则为孔孟一脉;
倘落于自然之“天”,则为老庄一脉;
倘落于人格化之“天”,则为原始信仰及墨学一脉。
然而这些主张全部为荀子所否定,则就理论而言,价值根源已无归宿矣。

\par

然而价值根源之归宿问题,为荀子哲学中不可不作答之根本性问题。
故荀子强行将价值根源归于“圣人”,谓圣人可违背人性之恶,作价值判断,遂创立文化。
《荀子·性恶》云:

\textit{凡礼义者,是生于圣人之伪,非故生于人之性也。}

又谓:

\textit{圣人积思虑,习伪故,以生礼义而起法度。然则礼义法度者,是生于圣人之伪,非故生于人之性也。}

总之,圣人创立了文化。
则圣人从何而来?
《荀子·性恶》又云:

\textit{故圣人者,人之所积而致矣。}

荀子言“圣人”,真真糊涂账也。
前述,所有人之性彼此相同,皆为恶,断非价值根源。
如此则自然不能有可作价值判断并创立文化之圣人。
又谓人经努力可成为圣人。
然而人只有学习已有文化成就之能力,无价值判断能力,自然不能成为创立文化之圣人。
此真为荀子哲学之败笔。

\section{走向权威主义}
尽管有着诸多困难,荀子还是将价值根源归于人类中的某些特殊个体,即圣人。
这种带有英雄主义色彩的论断最终将荀子引向权威主义。
盖圣人创造制度,若无权力,则制度只是空中楼阁。
毕竟,制度之运行必须依赖于一权力。
故荀子文中“君”与“师”并举。
《荀子·礼论》云:

\textit{礼有三本:天地者,生之本也;先祖者,类之本也;君师者,治之本也。}

此谓君师为礼之本。
《礼论》又云:

\textit{先王恶其乱也,故制礼义以分之,以养人之欲,给人之求,使欲必不穷于物,物必不屈于欲。两者相持而长,是礼之所起也。}

此处直言先王制礼义。
由此,执掌权力之君王与创制文化之圣人逐渐重合,荀子之价值根源过度至“君”,遂走向权威主义。

\par

君为价值根源。
《荀子·君道》云:

\textit{君者,民之原也。原清则流清,原浊则流浊。}

又谓:

\textit{请问为国?曰:闻修身,未尝闻为国也。君者仪也,民者景也,仪正而景正。君者盘也,民者水也,盘圆而水圆。君者盂也,盂方而水方。君射则臣决。楚庄王好细腰,故朝有饿人。故曰:闻修身,未尝闻为国也。}

此处直言君为最高规范,权威主义色彩甚明。

\par

荀子重君,遂有对建立君权之讨论。
《荀子·君道》谓:

\textit{故天子不视而见,不听而聪,不虑而知,不动而功,块然独坐,而天下从之如一体,如四肢之从心。夫是之谓大形。}

又谓:

\textit{故人主无便嬖左右足信者,谓之暗;无卿相辅佐足任使者,谓之独;所使于四邻诸侯者非其人,谓之孤;孤独而晻,谓之危。国虽若存,古之人曰亡矣。}

此类论述已大类法家之言,故李斯、韩非皆出荀门,不足为奇。

\par

但荀子终究不是法家,仍然坚持传统儒家之德治。
前引《荀子·君道》中君民盘水之喻,尤言君之修身,强调君王德性对国家之化成作用,可知荀子思想终究与法家尚隔一层。

\end{document}