\documentclass[11pt]{article}

\usepackage[UTF8]{ctex} % for Chinese 

\usepackage{setspace}
\usepackage[colorlinks,linkcolor=blue,anchorcolor=red,citecolor=black]{hyperref}
\usepackage{lineno}
\usepackage{booktabs}
\usepackage{graphicx}
\usepackage{float}
\usepackage{floatrow}
\usepackage{subfigure}
\usepackage{caption}
\usepackage{subcaption}
\usepackage{geometry}
\usepackage{multirow}
\usepackage{longtable}
\usepackage{lscape}
\usepackage{booktabs}
\usepackage{natbib}
\usepackage{natbibspacing}
\usepackage[toc,page]{appendix}
\usepackage{makecell}

\title{佛教在中国的传播}
\date{}

\linespread{1.5}
\geometry{left=2cm,right=2cm,top=2cm,bottom=2cm}

\begin{document}

  \maketitle

  \linenumbers
  
佛教传入中国,原在汉代。
学者多以永平十年,公元67年,汉明帝命人由西域迎沙门竺法兰和迦叶摩腾至洛阳,为佛教传入中国之始。
此说颇有争议,然思想之流传本是一渐进过程,非突发之事,故不必勉强确定具体年份。
大致而言,东汉初年,公元1世纪初,佛教已渐入中国。
此时中国学者仅将佛教视为神仙方术之流,谓之浮屠。
所传教义,零星浅薄。
东汉末年,公元2世纪,西域安息国之安世高来华,译述小乘诸经,为翻译佛经之始。
时值龙树立大乘中观之学(约150年),《般若经》诸部亦渐来中国。
三国至西晋,皆有僧众译经,但鲜有涉及佛教理论者。
衣冠南渡后,中国南北分裂。
佛教在北中国和南中国分别流传,印度佛教理论渐为中国学者所知。
隋唐之时,方有中国学者所立之佛教教义,即中国佛教。
现论述自西晋灭亡(316)至隋朝建立(581)时,佛教在北中国和南中国的流传。

\section{北中国的佛教}
\subsection{道安}
佛教典籍之翻译,本多为西域僧众学习中文后所为,其措辞用语常有谬误及欠通之处;
中国学者译经,又多依附道家思想,谓之“格义”,自不能得佛教本旨。
这一状况自道安(312-385)整理典籍后,方渐为改善。
此外,道安亦制定清规戒律,其后僧众多则而从之;
又派遣门下弟子往各地传教。
总之,道安为两晋佛教之中心人物,系佛教传播之先驱。

\par

然而,道安本人的理论浅薄。
道安及其同时代的僧众,多治中观之学,其说驳杂。
虽有“六家七宗”之盛,而皆不得中观之旨。
其主要问题在于受道家影响,以“道”释“空”。
“道”原为形而上的存在,以此解释否定一切存在的“空”,自不能得其主旨。
真正讲明中观之学,需待鸠摩罗什及其门下译传诸经之后。

\subsection{鸠摩罗什}
鸠摩罗什(343-413),龟兹人,原习小乘。
曾游历西域,于沙勒受中观之学。
385年,鸠摩罗什至凉州,并于401年抵达长安。
鸠摩罗什平生译经极多,又以阐明中观教义为核心,然其著作鲜有流传者。
据今存残篇,鸠摩罗什基本承龙树诸论。
其门下的僧肇可以说是中国学者中最能阐明中观之旨者。

\subsection{僧肇}
僧肇(384-414)为鸠摩罗什门下治中观之代表性人物。
兹先观其《不真空论》。

\par

“不真空”者,意指“空”即“不真”。
“不真”者,判一切独立实有也。
此说直承龙树“众因缘生法”之意。
《不真空论》云:

\textit{夫有若真有,有自常有,岂待缘而后有哉?譬彼真无,无自常无,岂待缘而后无也?若有不自有,待缘而后有者,故知有非真有。有非真有,虽有,不可谓之有矣。不无者,夫无则湛然不动,可谓之无。万物若无,则不应起,起则非无,以明缘起,故不无也。}

引文谓“有”“无”皆依“缘”起,故皆非真。
非有非无,即龙树“中道”之意。
故《不真空论》总结云:

\textit{说法不有亦不无,以因缘故诸法生。}

一切法皆决定于“因缘”,即条件,故皆非独立,则不可谓之有,亦不可谓之无。
此为中观之基本旨趣,而僧肇得之。

\par

独立实有既被否定,则不可言其有任何形式的变化,此即僧肇《物不迁论》之旨。
所谓“物不迁”者,即时空变化等观念,皆依认知活动而立,系主体活动之产物,故变化本身并无意义。
《物不迁论》云:

\textit{求向物于向,于向未尝无;责向物于今,于今未尝有。于今未尝有,以明物不来;于向未尝无,故知物不去。复而求今,今亦不往。是谓昔物自在昔,不从今以至昔;今物自在今,不从昔以至今。}

“向物”仅见诸“向”,故曰“不去”;
又不曾至“今”,故曰“不来”。
不去不来,则证其非动非静。
《物不迁论》云:

\textit{夫人之所谓动者,以昔物不至今,故曰动而非静;我之所谓静者,亦以昔物不至今,故曰静而非动。动而非静,以其不来;静而非动,以其不去。}

“昔物不至今”,故判其为“动”,或者“不来”,否则其可至今矣。
然而,“昔物不至今”意味着“昔物仅见诸昔”,故可判其为“静”,或者“不去”。
如此事物既静又动,则动静皆非实有。
如此则变化观念破矣。此说就理论脉络而言,可谓龙树“八不中道”之推论。
盖一切独立实有皆不成立,则一切形容独立实有的谓词亦不成立,故言实有之变化并无意义。

\par

此外,僧肇又有《般若无知论》,论主体境界,即论中所谓之“圣智”。
其大要谓,“圣智”为超越言论思考者,故不可以“虚实”“有无”等词汇描述。
“圣智”代表主体自由,故不受任何事象及规律之限制。
更进一步,此种自由非认知主体所具有。
盖认知活动基于主体和客体的二元对立,而二者互为限制,故认知主体不自由。
僧肇论中特标“无知”一词,否定认知主体之自由也。

\par

总之,僧肇为魏晋南北朝时期最能解中观之学者。
其后,中观之学虽流传甚广,其理论成就未有超过僧肇者。

\subsection{北方四宗}
414年,僧肇卒;
439年,北魏统一北中国,与南中国刘宋对立;
446年,北魏太武帝灭佛。
灭佛运动有其政治、经济等方面的原因,非理论之争。
但北魏以下,北中国僧人治学驳杂,常有论阴阳方术之流,更兼其生活作风每每可议。
总之,僧肇之后,北中国的佛教已日渐没落。

\par

但此时期,北方仍有许多佛教宗派。
北齐僧人慧远(523-592)记载各家学说,其中有立性、破性、破相、显实四宗。
一言以蔽之:
立性宗近唯识教义;
破性宗系小乘佛教;
破相宗承中观教义;
涅槃宗近真常教义。

\section{南中国的佛教}
\subsection{慧远}
南中国的佛教运动,首推慧远(334-416)。
此系道安门下弟子于南中国传教者,非前述北齐僧人慧远。
此人承道安之学,以中观教义为本。
但慧远本人并不固守门派,对佛教各宗之活动均加以支持鼓励,为佛教运动之领袖。
就理论而言,慧远之学杂取中观、小乘教义以及儒道之说,其学驳杂,无甚成就。

\subsection{竺道生}
竺道生(355-434)为南方佛教另一重要人物,初习小乘,后从鸠摩罗什习中观。
南归后,竺道生思想另有发展,结合417年译出的《六卷泥洹》(为《大涅槃经》的一部分),立佛性之说,谓众生皆可成佛,遭僧众抨击,大抵在428或429年。
421年,北凉昙无谶译出全部《大涅槃经》,其译文于430年传入南方,与竺道生相合,其学遂盛。

\par

就理论而言,竺道生为中国最早治真常教义的学者,于主体自由问题有所得悟。
然其著作失传者多。兹就现有资料,论竺道生之学。

\par

先论竺道生之“佛性我”论点。
佛教一切理论最终归于对最高主体的肯定,即对真我之肯定。
然诸宗中,小乘无真我;
中观重在破独立实有,仅预设主体性;
唯识重在解释对象界之虚妄,亦不直面主体。
竺道生则直接肯定真我。
《维摩经注》云:

\textit{无我本无死生中我,非不有佛性我也。}

此以佛性代表主体性,亦即真我,故谓“佛性我”。

\par

以佛性代指真我,乃就“觉”而言,但严格而论,“觉”“迷”皆依真我而立。
换言之,真我之自觉活动,可“觉”可“迷”。
更进一步,此主体超越对象界,遂不受一切外在决定,故“佛性”为“本有”。
《法华经注》云:

\textit{良由众生本有佛知见分,但为垢障不现耳。佛为开除,则得成之。}

\par

“佛”作为最高主体,超越一切对象,故不可以任何言及对象界的词汇加以描述,由此引出“佛无净土”之论点。
《维摩经注》云:

\textit{夫国土者,是众生封疆之域;其中无秽,谓之为净;无秽为无,封疆为有。有生于惑,无生于解。其解若成,其惑方尽。}

盖“净土”之类的词汇系对对象界的描述,断不能用于具有超越性的主体。
更进一步,因果报应亦是经验界的现象,故不能与主体相关。
此外,“佛”作为最高主体,亦不能与释迦牟尼其人等同。
此等论点,作为超越性主体之推论,于理论上全无困难。
然而,佛教本是哲学和宗教的混合。
竺道生之论,实是将佛教中的哲学成分剥离出来,并以此否定佛教中的宗教成分,自不能见容于教徒信众。

\par

最高主体自由既立,则“一阐提皆得成佛”无难矣,否则主体自由安在?
然竺道生之作失传,仅存征引之语。
日本沙门宗和尚《一乘佛性慧日抄》引《名僧传》云:

\textit{阐提是含生之类,何得独无佛性?盖此经度未尽耳。}

此就《六卷泥洹》而言,其言佛性,而无一阐提成佛之论。
竺道生则由此判定经文不全。
据今传传文,竺道生遭僧众抨击,似以此论调为主因。

\par

由最高主体自由亦引申出“顿悟”。
盖“迷”与“悟”皆为主体之状态。
主体既不受外物决定,则处于何种状态完全由主体决定,一切程序或外在条件均不能产生影响。
僧众之修持及其它为得悟所做之努力,皆为未悟之事,系悟之不必要不充分条件。
如此,一切程序均不能对悟产生影响,则迷和悟之间无中间状态。
换言之,一悟便全悟,非先悟一部分,而后再悟另一部分,故名之曰“顿”。

\par

总之,竺道生之学基本为真常一脉,为隋唐时期中国佛教之先声。

\subsection{真谛}
竺道生之后,南中国的佛教以真谛(499-569)为代表。
真谛系印度僧人,于548年,携梵文经卷,由南海抵达建业。
其理论承唯识教义,是南北朝时期,南方佛教最后一阶段之代表。

\par

唯识教义,侧重于解释对象界为何为虚妄。
然对破除虚妄之动力问题,其解释庞杂不明,主要纠结于阿赖耶识之染净。
真谛依《决定藏论》和《摄大乘论》,以阿赖耶识为对象界之根源,需加以破除;
而另立阿摩罗识,以为解脱之动力。
此为真谛之理论立场,然其著作不传,其中细节,殊难考证。

\subsection{《大乘起信论》}
最后,撮《大乘起信论》之要旨,结束本篇。

\par

《大乘起信论》旧传为古印度僧人马鸣所作,其时代约为1至2世纪;
其译本有真谛所作之梁译和实叉难陀(652-710)所作之唐译。
经考,此论非马鸣所作,其译者亦成问题。
且玄奘西行时,印度已不传此论。
故《大乘起信论》是由梵文翻译而来,还是中国学者所作,亦成问题。
无论如何,此论确于南朝末年见诸中国,其理论系真常一脉,与隋唐时中国佛教之三宗相近。

\par

《大乘起信论》先立主体性,云:

\textit{所言法者,谓众生心,是心则摄一切世间法、出世法。}

引文以“心”摄一切法,则此“心”即为主体,能观一切法,亦是一切法之起源。
如此则“心”必不能受外物制约,遂为众生皆有,故云“众生心”。
换言之,每一个体之自觉,皆通最高主体,故每一个体皆能成佛。

\par

“心”既是主体,则一切染净迷觉,皆完全决定于心。
换言之,迷与觉为“心”之活动的两个方面,即“一心二门”。
论云:

\textit{显示正义者,依一心法有二种门。云何为二?一者心真如门,二者心生灭门。是二种门皆各总摄一切法。此义云何?以是二门不相离故。}

“显示正义”,即《大乘起信论》所肯定之理论。
“心”之活动的两种方向,即“二门”,分别为“真如”和“生灭”。
此二门乃相依而立之概念,“真如”即主体超越“生灭”,“生灭”即主体违背“真如”,故云“二门不相离”。
“二门”皆为主体之活动方向,而主体为一切法之本,故谓“是二种门皆各总摄一切法”。
此为佛教之惯用表述,于今观之,实欠严格。

\par

“心真如”,即“真如”之“心”,指超越生灭、得觉与净的最高主体,为一切法之根源。
故云:

\textit{心真如者,即是一法界大总相法门体,所谓心性,不生不灭。}

“心真如”之语实欠通顺。
大抵此论作者不精中文。

\par

“真如”本身亦可标出两种意义。
《大乘起信论》云:

\textit{此真如者,依言说分别,有二种义。云何为二?一者如实空,以能究竟显实故;二者如实不空,以有自体具足无漏性功德故。}

所谓“如实空”,即通过点明对象之虚妄,破除独立实有,以明主体性。
故云“如实空”“能究竟显实”,即通过破除对象性以“显”主体性之“实”。
所谓“如实不空”,即点明主体为一切法之起源。
一切法皆为主体活动之产物,非独立实有,故皆为虚妄。
倘主体停止此种之活动,则一切法皆不存在,主体性得显现矣,遂有“无漏性功德”。
“无漏性”一词强调主体之超越性。
盖主体停止产生虚妄之活动,系超经验意义,非在对象界之事象。
此为唯识一系之用语。

\par

再观“心生灭”。
论云:

\textit{心生灭者,依如来藏故有生灭心,所谓不生不灭与生灭和合,非一非异,名为阿黎耶识。}

“心真如”表最高主体自由,“心生灭”为其反方向。
但“生灭”亦是由主体活动呈现,与外在不相干,故云“依如来藏故”,即依主体而立。
“不生不灭”表主体自由显现,即“心真如”;
“生灭”指主体自由不显现。
二者为同一主体之不同状态,故曰“和合”。
主体处于“生灭”之状态时,生出种种迷障,主体自由不显现,故云“非一”;
但此种迷障乃主体所生,作迷障活动者仍是此主体,故云“非异”。
“生灭”之状态,称为“阿黎耶识”,即阿赖耶识。
此系唯识一系之用语。

\par

阿赖耶识有“觉”和“不觉”两种含义。
论云:

\textit{此识有二种义,能摄一切法,生一切法。云何为二?一者觉义,二者不觉义。}

“心真如”和“心生灭”皆就同一主体而言。
主体处于“生灭”状态,主体自由不显现,为迷障所蔽,故曰“不觉”。
然受蔽之主体仍有破除迷障之能力,主体自由仍在,故曰“觉”。
换言之,主体自由之显现和主体自由之存在互不干扰。
主体无论是处于“真如”还是处于“生灭”,其自由皆存在;
只是于“真如”状态下,主体自由显现,而于“生灭”状态下不显。

\par

主体自由不显现,“心”为迷障所蔽,遂有主客之分裂,而生出一切幻妄对象。
故云:

\textit{是故三界虚伪,唯心所作。离心则无六尘境界。此义云何?以一切法皆从心起妄念能生,一切分别即分别自心。}

此种迷障源自主体活动,其消除亦依赖主体。
论云:

\textit{若能观察,知心无念,即得随顺入真如门故。}

“心”之“无念”,即主体停止产生迷障之活动,则一切法皆不成立,主体性显现,遂“顺入真如门故”。

\par

总之,《大乘起信论》肯定一真常主体,并以此解释对象界。
是论若出印度,则必为晚出之书,在中观唯识兴起之后,不可托于马鸣;
若为中国学者所作,则此论可称中国佛教之先声。

\end{document}