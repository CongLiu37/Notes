\documentclass[11pt]{article}

\usepackage[UTF8]{ctex} % for Chinese 

\usepackage{setspace}
\usepackage[colorlinks,linkcolor=blue,anchorcolor=red,citecolor=black]{hyperref}
\usepackage{lineno}
\usepackage{booktabs}
\usepackage{graphicx}
\usepackage{float}
\usepackage{floatrow}
\usepackage{subfigure}
\usepackage{caption}
\usepackage{subcaption}
\usepackage{geometry}
\usepackage{multirow}
\usepackage{longtable}
\usepackage{lscape}
\usepackage{booktabs}
\usepackage{natbib}
\usepackage{natbibspacing}
\usepackage[toc,page]{appendix}
\usepackage{makecell}

\title{早期儒学复兴运动}
\date{}

\linespread{1.5}
\geometry{left=2cm,right=2cm,top=2cm,bottom=2cm}

\begin{document}

  \maketitle

  \linenumbers

儒学复兴始于唐代末期,经宋至明,为中国哲学发展之主流。
其旨在于回归孔孟之心性论哲学,反对汉儒宇宙论哲学以及佛教解脱之说。
韩愈、李翱为此运动之序幕人物。
至北宋初年,学者已有自造系统抗拒佛教之意,却多混杂宇宙论与形而上学,以此为价值理论之基础。
此为儒学复兴运动早期,以周敦颐、张载等为代表。
同一时期的邵雍则肯定认知主体自由,一反儒释道之基本立场。
儒学复兴运动中期以程颐、程颢、朱熹等为代表。
其学说已洗去宇宙论成分,将价值理论建议在纯粹形而上学的基础上。
晚期则始于南宋陆九渊,集大成于王阳明。
最高主体自由得明。

\section{周敦颐}
周敦颐(1017-1073),字茂叔,号濂溪。
其学基本与汉儒之宇宙论中心哲学相同,立一形而上的至高作为万物主宰和价值根源。

\par

周敦颐之至高主宰曰“诚”。
《通书》曰:

\textit{大哉乾元,万物资始,诚之源也。}

盖“诚”即本性之充分实现。
万有之产生,皆因其本性之实现,故曰“诚之源也”。

《通书》又曰:

\textit{乾道变化,各正性命,诚斯立焉。纯粹至善者也。}

万有在演变过程中“正性命”,充分实现其本性,是为“诚”之显现,故曰“诚斯立焉”。
“纯粹至善”则言“诚”为价值判断之基础。
《通书》对此进一步论述,曰:

\textit{圣,诚而已矣。诚,五常之本,百行之源也。}

此谓“诚”为一切德性之基础,“诚”之实现为最高人生境界。
总之,“诚”为万物之根源,又为万物演变之方向,又为纯粹至善之最高价值。
如此则万物不能违背“诚”,不能不充分实现其本性,故万物皆善,遂不能有罪恶。
此说蹈汉儒覆辙,混淆了“应该如此”和“必然如此”两个不同的领域。

\par

周敦颐之《太极图说》主要涉及宇宙论,其文曰:

\textit{无极而太极。太极动而生阳,动极而静,静而生阴,静极复动。一动一静,互为其根。分阴分阳,两仪立焉。阳变阴合,而生水火木金土。五气顺布,四时行焉。五行一阴阳也,阴阳一太极也,太极本无极也。
\newline
五行之生也,各一其性。无极之真,二五之精,妙合而凝。乾道成男,坤道成女。二气交感,化生万物。万物生生而变化无穷焉。
\newline
唯人也得其秀而最灵。形既生矣,神发知矣。五性感动而善恶分,万事出矣。圣人定之以中正仁义而主静,立人极焉。
\newline
故圣人“与天地合其德,日月合其明,四时合其序,鬼神合其吉凶”,君子修之吉,小人悖之凶。故曰:“立天之道,曰阴与阳。立地之道,曰柔与刚。立人之道,曰仁与义。”又曰:“原始反终,故知死生之说。”大哉易也,斯其至矣!}

第一节言宇宙之发生历程:
“无极”产生“太极”;
“太极”之“动”生“阳”,其“静”生阴;
“阳变阴合”而生五行。
第二节言阴阳五行“化生万物”。
第三、四节涉及价值理论。
需要注意的是,从“无极”到“万物生生”,再到“唯人也得其秀而最灵”,均系对事实的描述,仅涉及“实际如此”,不涉及“应该如此”,即不涉及价值观念。
纵然言人“秀而最灵”,亦只是化生万物的一种状态,仅仅是一事实,不能产生善恶之分。

\section{张载}
张载(1020-1078),字子厚。
其说坚守儒学基本立场,但仍混合宇宙论和形而上学,未能正面肯定主体性。

\par

张载以“气”为万物之根源。
《正蒙·太和篇》曰:

\textit{太虚无形,气之本体。其聚其散,变化之客形尔。至静无感,性之渊源;有识有知,物交之客感尔。}

此处谓“太虚”或“气”为万物之本体。
“气”本无形,由“气”之聚散而变化为有形之万物。
“气”为“性”之根源,本身无感,但在与万物的交互中呈现为“有识有知”。
如此则万物与人皆由“气”所产生,故《西铭》云:

\textit{民吾同胞,物吾与也。}

\par

张载以“神”、“化”联系形而上学和价值理论。
《正蒙·神化篇》曰:

\textit{神,天德;化,天道。德,其体;道,其用。一于气而已。}

以“神”、“化”为“天”之“道”、“德”,则人应当以此为目标,以求明“神”知“化”,合于“天”之“道”、“德”。
故《正蒙·神化篇》又云:

\textit{神化者,天之良能,非人能。故大而位天德,然后能穷神知化。}

\par

张载论“性”,取本质意义。
《正蒙·诚明篇》曰:

\textit{性者,万物之一源,非有我之得私也。}

“性”为万物共有之本质,即为至高价值。
故《正蒙·诚明篇》又云:

\textit{性与天道合一,存乎诚。}

“诚”即实现。
“性”与“天道”为同一,其实现即为“诚”。
“性”之实现则依赖“尽性”、“穷理”之活动。
《正蒙·诚明篇》曰:

\textit{自明诚,由穷理而尽性也;自诚明,由尽性而穷理也。}

此处张载已触及主体问题,盖“尽性”、“穷理”皆主体之自觉活动。
张载对主体性观念探究有限,遂有自我矛盾之处。
如《正蒙·诚明篇》云:

\textit{心能尽性,人能弘道也;性不知检其心,非道弘人也。}

此命题实为孔孟心性论立场。
倘依张载之宇宙论立场,万物之本质(“性”)与天道合一,人仅能被动地依照天道实现其“性”,实为天道宏人。

\par

万物之方向均由天道决定,故均向乎善,其恶则由第二序的“气质”产生。
《正蒙·诚明篇》曰:

\textit{形而后有气质之性。善反之,则天地之性存焉。故气质之性,君子有弗性者焉。}

盖万物由无形之“气”聚为有形之存在,必有其发生之条件,即“气质”。
“气质”另有一不同于天道之本质,即“气质之性”。
如此则人可向乎“气质之性”而走向恶,依可向乎天道之性而走向善。
然天道既然为一切之根源,“气质之性”为何能违背天道?
张载对此无明确论述,反而转论如何克服“气质之性”,即“学”。

\par

张载论“学”,坚持儒者基本立场,以成德为目的,其枢纽落于改造“气质之性”、实现天道之性。
此处张载实已经接近心性论,其说时时透露主体性,显现道德自觉。
如《理窟·义理篇》曰:

\textit{学贵心悟,守旧无功。}

成德之关键在于自觉,故落于一“悟”字。
主体既“悟”,天道之性自然觉醒。
此事与经验知识无观,故《正蒙·大心篇》曰:

\textit{德性所知,不萌于见闻。}

主体之觉醒为一自觉过程,并不源于信仰,故《正蒙·中正篇》曰:

\textit{笃信不好学,不越为善人信士而已。}

“悟”不仅不依赖于经验知识和信仰,亦不必依赖经籍,故《理窟·义理篇》曰:

\textit{凡经义,不过取证明而已;故虽有不识字者,何害为善?}

总之,张载论“气质之性”及“学”,实处心性论立场而不自知,遂与其宇宙论和形而上学理论矛盾。

\section{邵雍}
邵雍(1011-1077),字尧夫,谥康节。
总的来说,邵雍以河图洛书《易传》等汉儒杂辑资料为依据,构建一宇宙论系统,并肯定认知主体自由。
此说与儒学道德化成之基本方向相左。
然朱熹极尊邵氏,谓其学出自孔子,乃误判邵庸所依据的资料为孔子所作。

\par

邵氏之宇宙论以《先天图》为中心。
《先天图》相传出自河图,仅为爻卦符号之排列组合。
“—”为阳爻,“--”为阴爻。
两爻排列组合,遂有四种符号。
三爻排列组合,则有八种,即所谓八卦。
邵雍《观物篇》谓阴爻为“静”,阳爻为“动”。
两爻组合,有“柔”、“刚”、“阴”、“阳”四种。
三爻组合,有“太柔”、“太刚”、“少柔”、“少刚”、“少阴”、“少阳”、“太阴”、“太阳”八种,实为八卦。
八卦中任意两种上下排列,有六十四重卦。
将这些符合排列为同心圆,最外层为六十四重卦,其内依次为八卦、四种两爻组合、阴阳爻,遂得圆图。
邵雍谓此圆图表寒暑循环。
六十四重卦,凡三百八十四爻。
抽出乾、坤、坎、离,余三百六十爻,与一年三百六十日相对。

\par

《先天图》亦说明世界之历程。
邵雍谓三十年为一“世”,十二“世”为一“运”,三十“运”为一“会”,十二“会”为一“元”。
如就年往下推,一年为十二月,一月为三十日,一日为十二时,一时为三十分,一分为十二秒。
此种计算,不过十二与三十交替,所言分秒,非今日之时间单位。
一“元”为世界由始至终所需时间,即十二万九千六百年。一“元”分十二“会”,对应十二辟卦,即复、临、泰、大壮、夬、乾、姤、遁、否、观、剥、坤。
一会开天,二会辟地,三会生人。
六会乾卦,世界最盛。
此会第三十运中的第九世对应唐尧时代。
至十一会,一切衰落。
十二会,天地崩坏。
一元已终,另一元复始。
依次观念,人类历史自唐尧之后,一代不如一代,着实为一极度悲观之命定论历史观。

\par

邵雍又以八卦对应“日、月、星、辰、水、火、土、石”。
前四者属于“天”,后四者属于“地”。
如此,八卦与宇宙之发生相关联。
“天”“地”之“变”产生非生物之宇宙,而后方有生物。
生物之“性情形体”源于“天”,“走飞草木”源于“地”。
“天”“地”分别有一万七千二十四种“变数”,故生物有$17,024^2=289,816,576$种“变数”,谓之“动植通数”。

\par

邵雍之说亦有涉及形而上学之论点。
《观物内篇》曰:

\textit{所以谓之理者,物之理也;所以谓之性者,天之性也;所以谓之命也,处理性者也;所以能处理性者,非道而何?是知道为天地之本,天地为万物之本。以天地观万物,则万物为物;以道观天地,则天地亦为万物。}

“理”指事物之特殊性,“性”则指万物之共性。
“命”则“处理性”,为“理”与“性”之间的关系,所指不甚明晰。
而“命”又因“道”成立。
“道”为“天地之本”,进而为“万物之本”。
此“道”为形而上之实有。

\par

至此,邵庸判世界之演变为一既定历程,则人之价值何在?
此处邵庸一反儒释道之精神,肯定认知主体。
《观物内篇》曰:

\textit{是知人也者,物之至者也;圣也者,人之至者也。人之至者,谓其能以一心观万心,一身观万身,一世观万世。}

人为万物之至,而圣人为人之至。
人之所以有此种地位,是因为人有认知能力。
《观物内篇》又曰:

\textit{道之道尽于天矣,天之道尽于地矣,地之道尽于物矣,天地万物之道尽于人矣。人能知天地万物之道,所以尽于人者,然后能尽民也。}

“尽”即实现。
“道”于“天”中实现,“天”于“地”中实现,“地”于“物”中实现,而“天地万物”又在人中实现。
人能“知”“天地万物之道”,故有此地位。
此处邵雍正面肯定认知主体。
此主体能就个别存在各观其理,不局限于自身之存在,故超越万物且能知万物,并不局限于形躯这一客观存在。
故《观物内篇》曰:

\textit{不我物,则能物物。}

盖认知主体非物,故能统摄一切对象,显示主体之超越性。
另一方面,认知主体亦有不显现之自由,即“以我徇物”。


\end{document}