\documentclass[11pt]{article}

\usepackage[UTF8]{ctex} % for Chinese 

\usepackage{setspace}
\usepackage[colorlinks,linkcolor=blue,anchorcolor=red,citecolor=black]{hyperref}
\usepackage{lineno}
\usepackage{booktabs}
\usepackage{graphicx}
\usepackage{float}
\usepackage{floatrow}
\usepackage{subfigure}
\usepackage{caption}
\usepackage{subcaption}
\usepackage{geometry}
\usepackage{multirow}
\usepackage{longtable}
\usepackage{lscape}
\usepackage{booktabs}
\usepackage{natbib}
\usepackage{natbibspacing}
\usepackage[toc,page]{appendix}
\usepackage{makecell}

\title{《礼记·中庸》}
\author{}
\date{}

\linespread{1.5}
\geometry{left=2cm,right=2cm,top=2cm,bottom=2cm}

\begin{document}

  \maketitle

  \linenumbers
先秦儒学以心性论为核心,所关切者为人之德性,有以道德化成世界之态度,不关注形而上学。
阴阳家之学则以宇宙论为核心,以积极的态度解释自然、认知事物,自身带有向形而上学发展的趋势。
秦燹之后,儒学与阴阳学逐渐融合。
在这一历史进程中,有学者基于形而上学构建价值规范。
今传《中庸》即代表此方向。

\section{性与道}
《中庸》开篇云:

\textit{天命之谓性,率性之谓道,修道之谓教。}

此处之“天命”,可理解为一形而上学或神学概念。
但《中庸》主旨在于价值规范,本无形而上学或神学旨趣。
立一“天命”,仅将其视为外在的价值根源,对“天命”本身并未有更多关注。
此天命赋予万物以“性”。
率性而为即为“道”。
可见,《中庸》持孟子之性善理论立场。
如若不然,则率性焉能之谓道?
“修道之谓教”则进一步强调价值所在。
至此,《中庸》之基本理论框架已明:
有一外在的、至高的天命赋予万物不同的性,顺性即为价值。

\newline

《中庸》又云:

\textit{道也者,不可须臾离也,可离非道也。}

此处谓“不可须臾离也”,盖事物运行,均决定于其固有属性,故万物皆依道而行。
由此,《中庸》认为,万物不仅应该顺性,而且必然顺性。
下文又谓:

\textit{喜怒哀乐之未发,谓之中;发而皆中节,谓之和;中也者,天下之大本也;和也者,天下之达道也。致中和,天地位焉,万物育焉。}

此处论人之情绪。
首先,“喜怒哀乐”可“未发”,则不能无有一超越情绪之自我。
否则,自我始终在“喜怒哀乐”之中,何言“未发”?
其次,“发而皆中节”暗示一价值规范。
换言之,情绪之“发”应该“中节”。
如此可使整个宇宙按照价值规范运行,即“天地位焉,万物育焉”。
由此可见《中庸》之形而上学立场。

\newline

此处还有一理论问题。
道为决定万象之规律,则万物断不可违道背性而行。
而且此规律亦是价值规范,则万象皆为应该。
如此则情绪运行焉能不“中节”?
倘若此道只表价值规范,不决定万象,则何言“不可须臾离也”?
可见,《中庸》混淆了“应该如此”和“必然如此”两个概念。
前者涉及价值规范,而后者涉及客观事物的运行规律,二者本不相干。

\section{尽性}
论《中庸》尽性之说,先训“诚”字。
《中庸》所言之“诚”,可指个人之不欺骗,又指“完全实现”之状态。
前者为日常用语,后者则涉及形而上学,为《中庸》之特殊用法。
但就理论脉络而言,完全实现之诚为不欺之诚的进一步发展。
《中庸》载:

\textit{诚者,天之道也;诚之者,人之道也。诚者,不勉而中,不思而得,从容中道,圣人也。诚之者,择善而固执之者也。}

“诚”和“诚之”分别指有待实现的境界和实现此境界所需的功夫。
谓诚为“天之道”,是将价值规范上归于天,而人之义务在于努力实现诚之境界,故谓“人之道”。

\newline

则为实现诚,需下何种功夫?
答曰:尽性。
《中庸》云:

\textit{唯天下至诚,为能尽其性;能尽其性,则能尽人之性;能尽人之性,则能尽物之性;能尽物之性,则可以赞天地之化育;可以赞天地之化育,则可以与天地参矣。}

诚之旨在于充分发挥事物之性。
总之,《中庸》所立之价值规范系事物本性的充分实现。
盖对某一事物,人常常为其假定某种理想状态。
与理想状态相符即为好,否则即为坏。
而此理想状态即为事物之本性。
就人类而言,其本性是内在的道德自觉。
盖《中庸》预设孟子性善理论,而以形而上学对其加以诠释。

\newline

达到诚之境界,不仅仅是充分实现自身之性,亦需“尽人之性”“尽物之性”,帮助宇宙实现价值,故曰“可以赞天地之化有”。
故《中庸》云:

\textit{诚者非自成己而已也,所以成物也。成己仁也;成物知也。性之德也,合外内之道也,故时措之宜也。}

此处强调以道德化成世界。
此为儒者一贯主张,但其所化成之“世界”仅限于人类。
换言之,其道德化成仅就政治生活而言。
而《中庸》立论,欲化成整个宇宙,其形而上学立场甚明。
此类论述,《中庸》文中时有出现。如:

\textit{仲尼祖述尧舜,宪章文武:上律天时,下袭水土。辟如天地之无不持载,无不覆帱,辟如四时之错行,如日月之代明。万物并育而不相害,道并行而不相悖,小德川流,大德敦化,此天地之所以为大也。}

又如:

\textit{大哉圣人之道!洋洋乎,发育万物,峻极于天。优优大哉!礼仪三百,威仪三千。待其人然后行。故曰:苟不至德,至道不凝焉。}

此处值得注意的是“苟不至德,至道不凝焉”。
如前所述,《中庸》混淆了“应该如此”和“必然如此”两个概念。
如此则价值之实现为一必然,无需人之自觉努力。
而引文又将至德视为实现道的先决条件。
所谓圣人,指能自尽其性、尽人之性、尽物之性的人。
而价值的实现又依赖于圣人,实际上仍依赖于人的自觉努力。
此为《中庸》之立说不严。
  
\end{document}