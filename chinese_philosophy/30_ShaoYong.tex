\documentclass[11pt]{article}

\usepackage[UTF8]{ctex} % for Chinese 

\usepackage{setspace}
\usepackage[colorlinks,linkcolor=blue,anchorcolor=red,citecolor=black]{hyperref}
\usepackage{lineno}
\usepackage{booktabs}
\usepackage{graphicx}
\usepackage{float}
\usepackage{floatrow}
\usepackage{subfigure}
\usepackage{caption}
\usepackage{subcaption}
\usepackage{geometry}
\usepackage{multirow}
\usepackage{longtable}
\usepackage{lscape}
\usepackage{booktabs}
\usepackage{natbib}
\usepackage{natbibspacing}
\usepackage[toc,page]{appendix}
\usepackage{makecell}

\title{邵雍}
\date{}

\linespread{1.5}
\geometry{left=2cm,right=2cm,top=2cm,bottom=2cm}

\begin{document}

  \maketitle

  \linenumbers
邵雍(1011-1077),字尧夫,谥康节。
其说多零散论点,难成体系。
总的来说,邵雍以河图洛书《易传》等汉儒杂辑资料为依据,构建一宇宙论系统,并将认知主体置于其中。
此说与儒学基本方向相左。
然朱熹极尊邵氏,谓其学出自孔子,乃误以为《易传》等为孔子所作。
本篇论邵雍之说。

\section{宇宙论}
邵氏之说以《先天图》为中心。
《先天图》相传出自河图,仅为爻卦符号之排列组合。
“—”为阳爻,“--”为阴爻。
两爻排列组合,遂有四种符号。
三爻排列组合,则有八种,即所谓八卦。

\newline

邵雍《观物篇》谓阴爻为“静”,阳爻为“动”。
两爻组合,有“柔”、“刚”、“阴”、“阳”四种。
三爻组合,有“太柔”、“太刚”、“少柔”、“少刚”、“少阴”、“少阳”、“太阴”、“太阳”八种,实为八卦。
八卦中任意两种上下排列,有六十四重卦。
将这些符合排列为同心圆,最外层为六十四重卦,其内依次为八卦、四种两爻组合、阴阳爻,遂得圆图。

\newline

邵雍谓圆图表寒暑循环。
六十四重卦,凡三百八十四爻。
抽出乾、坤、坎、离,余三百六十爻,与一年三百六十日相对。

\newline

《先天图》亦说明世界之历程。
邵雍谓三十年为一“世”,十二“世”为一“运”,三十“运”为一“会”,十二“会”为一“元”。
如就年往下推,一年为十二月,一月为三十日,一日为十二时,一时为三十分,一分为十二秒。
此种计算,不过十二与三十交替,所言分秒,非今日之时间单位。
一“元”为世界由始至终所需时间,即十二万九千六百年。一“元”分十二“会”,对应十二辟卦,即复、临、泰、大壮、夬、乾、姤、遁、否、观、剥、坤。
一会开天,二会辟地,三会生人。
六会乾卦,世界最盛。
此会第三十运中的第九世对应唐尧时代。
至十一会,一切衰落。
十二会,天地崩坏。
一元已终,另一元复始。
依次观念,人类历史自唐尧之后,一代不如一代,着实为一极度悲观之命定论历史观。

\newline

邵雍又以八卦对应“日、月、星、辰、水、火、土、石”。
前四者属于“天”,后四者属于“地”。
如此,八卦与宇宙之发生相关联。
“天”“地”之“变”产生非生物之宇宙,而后方有生物。
生物之“性情形体”源于“天”,“走飞草木”源于“地”。
“天”“地”分别有一万七千二十四种“变数”,故生物有$17024^2$种“变数”,谓之“动植通数”。

\section{形而上学}
邵雍之说亦有涉及形而上学之论点。
《观物内篇》曰:

\textit{所以谓之理者,物之理也;所以谓之性者,天之性也;所以谓之命也,处理性者也;所以能处理性者,非道而何?是知道为天地之本,天地为万物之本。以天地观万物,则万物为物;以道观天地,则天地亦为万物。}

“理”指事物之特殊性,“性”则指万物之共性。
“命”则“处理性”,为“理”与“性”之间的关系。
而“命”又因“道”成立。
“道”为“天地之本”,进而为“万物之本”。
此“道”为形而上之实有。

\section{人之地位}
邵雍谓人区别于他物。
《观物内篇》曰:

\textit{是知人也者,物之至者也;圣也者,人之至者也。人之至者,谓其能以一心观万心,一身观万身,一世观万世。}

人为万物之至,而圣人为人之至。

\newline

人之所以有此种地位,是因为人有认知能力。
《观物内篇》曰:

\textit{道之道尽于天矣,天之道尽于地矣,地之道尽于物矣,天地万物之道尽于人矣。人能知天地万物之道,所以尽于人者,然后能尽民也。}

“尽”即实现。
“道”于“天”中实现,“天”于“地”中实现,“地”于“物”中实现,而“天地万物”又在人中实现。
人能“知”“天地万物之道”,故有此地位。
此处邵雍正面肯定认知活动。

\newline

人之认知能力表现为“观物”。
人能就个别存在各观其理,而不局限于自身之存在,故能知天地万物。
此中关键载于不以人自身为“物”,故《观物内篇》曰:

\textit{不我物,则能物物。}

换言之,认知主体不自限于“人”这一客观存在,能统摄一切对象,遂显示主体之超越性。
另一方面,认知主体亦有不显现之自由,即“以我徇物”。

\end{document}