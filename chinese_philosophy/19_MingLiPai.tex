\documentclass[11pt]{article}

\usepackage[UTF8]{ctex} % for Chinese 

\usepackage{setspace}
\usepackage[colorlinks,linkcolor=blue,anchorcolor=red,citecolor=black]{hyperref}
\usepackage{lineno}
\usepackage{booktabs}
\usepackage{graphicx}
\usepackage{float}
\usepackage{floatrow}
\usepackage{subfigure}
\usepackage{caption}
\usepackage{subcaption}
\usepackage{geometry}
\usepackage{multirow}
\usepackage{longtable}
\usepackage{lscape}
\usepackage{booktabs}
\usepackage{natbib}
\usepackage{natbibspacing}
\usepackage[toc,page]{appendix}
\usepackage{makecell}

\title{名理派}
\date{}

\linespread{1.5}
\geometry{left=2cm,right=2cm,top=2cm,bottom=2cm}

\begin{document}

  \maketitle

  \linenumbers

名理一派,承老庄之形而上学旨趣。
盖清谈之辈,多作俏皮语、抖机灵者。
其中能把握老庄之部分理论者,已属高士矣。
现就代表人物和著作,论此派大要。

\section{王弼}
史传王弼(226-249)“好论儒道,辞才逸辩”,大抵论述颇多。
今传王弼著作,有《老子》注和《易》注。

\newline

王弼注《老子》,基本能把握其形而上学理论。
其旨谓“道”或“无”为超越经验世界的至高存有,又为经验世界的源头。
万物流变不息,皆受永恒的“道”的支配。
就此而言,王注虽多文人铺排之语,但大旨与老子合。

\newline

然而,老子之学尚有特殊的价值观念,即自我境界问题。
对此,《老子》中所论尚欠周全,而《庄子》内篇所论甚为严整。
就理论而言,道家之形而上学观念是为支撑其价值理论而立,故后者较前者更为重要。
就内容而言,道家之自我系观道之自我,超越经验世界,于经验世界中无所追求,仅观赏其在道的支配下的运行。
换言之,老庄之自我不在经验世界寻找价值。
此超越性自我系老庄之精意,而王弼于此全未涉及。

\newline

王弼注《易》,不过借古人之文阐述自己的思想,非阐述《易》之本意。
此为古代学者常有之弊病,不必在意。
其中值得注意者,不过谓“一”为“多”之起源,并支配“多”。
此实为老庄之形而上学观念,用语不同耳。

\newline

总之,王弼基本能正确把握道家之形而上学观念,却不解道家观赏世界之境界。

\section{《庄子注疏》}
《庄子注疏》为魏晋玄学之重要文献,传为郭象(252-312)所作。
然《晋书·郭象传》谓郭象窃向秀(227-272)所作之《庄子注》为己作。
但《晋书·向秀传》又谓郭象仅“述而广之”,不曾窃取向注。
此中细节,本文不作考证,仅就今传《庄子注疏》观名理一派之理论。

\newline

道家之形而上学,以“道”为经验世界之起源和支配力量,即由“无”生“有”。
《庄子注疏》则否认有生于无的说法,其文谓:

\textit{无既无矣,则不能生有;有之未生,又不能为生,然则生生者谁哉?块然而自生耳。自生耳,非我生也。我既不能生物,物亦不能生我,则我自然矣。自己而然,谓之天然。}

此谓“无”不能生“有”。那么万物为何如此?
答曰:万物自己如此。
既然万象皆“自己而然”,则其存在和运行不依赖于任何外部条件,即“无待”;
经验世界的一切事象完全由自身决定,故曰“独化”。

\newline

至此,万物本身皆自然如此,无需增减,则人之自觉于经验世界不能有所成就。
因此,面对经验世界,人应不加干预,使万物“各任其自为”,即“无为”。
由此,自觉心于经验世界,只能持观赏态度,而不能有任何作为。
此为庄子所论之逍遥境界,而《庄子注疏》称之为“无待”,言自我不执于物也。

\newline

总之,《庄子注疏》能把握道家以超越性自我观赏经验世界的境界,其对道家学说的了解亦仅限于此。  
  
\end{document}