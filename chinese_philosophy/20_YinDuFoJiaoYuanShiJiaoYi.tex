\documentclass[11pt]{article}

\usepackage[UTF8]{ctex} % for Chinese 

\usepackage{setspace}
\usepackage[colorlinks,linkcolor=blue,anchorcolor=red,citecolor=black]{hyperref}
\usepackage{lineno}
\usepackage{booktabs}
\usepackage{graphicx}
\usepackage{float}
\usepackage{floatrow}
\usepackage{subfigure}
\usepackage{caption}
\usepackage{subcaption}
\usepackage{geometry}
\usepackage{multirow}
\usepackage{longtable}
\usepackage{lscape}
\usepackage{booktabs}
\usepackage{natbib}
\usepackage{natbibspacing}
\usepackage[toc,page]{appendix}
\usepackage{makecell}

\title{印度佛教原始教义}
\date{}

\linespread{1.5}
\geometry{left=2cm,right=2cm,top=2cm,bottom=2cm}

\begin{document}

  \maketitle

  \linenumbers

佛教源于印度,其创始人乔达摩·悉达多(Gotama Siddhattha, 565 BC-486 BC),又称释迦牟尼(Sakyamuni)。
东汉时,佛教传入中国,经三国两晋南北朝,于隋唐时为显学。
在这一过程中,佛教逐渐中国化,产生了不同于印度佛教的新理论,即为中国佛学。
欲观佛教在中国的流传和演变,有必要先返溯印度佛教理论。

\newline

佛教自创立以来,演变甚繁。
释迦牟尼之教义,可视为原始教义。
释迦逝世后,教徒分为诸多派别,佛教遂进入“部派时期”。
至约150年,龙树(Nagarjuna)立中观之学,大乘(Mahayana)佛教始现。
三至四世纪间,有弥勒(Maitreya)立新说,并由无著(Asanga)和世亲(Vasubandhu)于四世纪时构建系统学说,即唯识之学,为大乘佛教的第二阶段。
大乘佛教除中观、唯识二支外,尚有真常之学。
此一支对中国佛学的产生颇有影响,但于印度本土则不甚发达。

\newline

本文论原始教义之大要,所涉及材料为阿含(Agama)经集。
相传佛陀逝世后,其弟子摩诃迦叶(Mahakasyapa)组织僧众整理教义。
此时释迦弟子阿难(Ananda)诵先师之训,后又整理为阿含经集。
中译阿含有《长阿含》(Dighanikaya)、《中阿含》(Majjimanikaya)、《杂阿含》(Sanryuttan)、《增一阿含》(Anguttarayn)四种。

\newline

就古印度传统的吠陀(Veda)哲学而言,佛教可谓一革命性理论。
盖吠陀经之理论虽繁杂,但始终假定某种外在的客观存在。
然而佛教否定一切外在存在,仅肯定主体性。
吠陀经学和佛教的对立为古印度学者之共识,但二者所关注的根本问题相同,即如何脱离生命之苦。
由此,吠陀经学和佛教需先承认生命之苦。
现论原始教义,即从此开始。

\newline

需要指出,佛教本为一宗教。然印度传统中,宗教与哲学历来相混。此处论佛教或者佛学,仅就其哲学方面而言,并不涉及宗教方面。
    
\section{三法印}
三法印,即代表佛教的基本立场的三个命题。
佛教诸经中,三法印的表述甚多。
兹取《法句经》(Dhammapada)经文作为标准叙述。
《法句经》列出的三法印为:
(1)诸行无常(Sabbe Sankhara Anicca);
(2)诸行皆苦(Sabbe Sankhara Dukkha);
(3)诸法无我(Sabbe Dhamma Anatta)。

\newline

先论“苦”(Dukkha)。
生命恒有需求,每一需求形成一种满足该需求的压力,即成为生命之苦。
在需求得到满足时,生命之苦得以解除,遂有“乐”。
故“乐”仅为“先在之苦”的暂停,其意义需通过“苦”而界定。
但“苦”依生命本身而立,所以“苦”先在于“乐”。
由此,生命之真相实为诸多痛苦。

\newline

“行”(Sankhara)指意欲活动。
“苦”既为生命之真相,则一切意志活动皆不能离苦,故曰“诸行皆苦”。
生命有诸多需求,而意志活动断不能仅致力于满足某种需求而不论其它。
故生命活动必然忽而求某需求之满足,忽而转向另一需求,此种状态即为“无常”,故曰“诸行无常”。

\newline

所谓“诸法无我”,即一切事象皆无自主性可言,皆由条件或者“因缘”决定,故皆为被决定者。
更进一步,则生命中一切事象亦受条件决定,意志活动无法对其加以干预,故有生命之苦。

\section{四谛}
四谛,即释迦教义的四个基本概念,为佛教理论之总纲。
四谛包括“苦”(Dukkha)、“集”(Samudaya)、“灭”(Nirodha)、“道”(Magga)。

\newline

“苦”即前述由需求造成的生命之苦。
“集”则指决定事象的条件“集合”,其意与“诸法无我”相通。
更进一步,“集”包括两种条件关系。
其一是同时互依,即甲是乙的原因,而乙又同时是甲的原因,二者互为条件,同时显现。
《杂阿含》载佛陀解释同时互依曰:

\textit{如两束芦,互倚不倒。}

其二便是因果关系,即甲是乙的原因,并先于乙显现。

\newline

总而言之,“苦”“集”二谛涉及生命和世界的真相。
“苦”谛论生命之痛苦,“集”谛论经验世界之不自由。
“灭”“道”二谛则涉及佛教所肯定之境界以及达到此种境界的方法。

\newline

所谓“灭”,即“苦”“集”之破灭,指自我摆脱生命之苦和经验世界之束缚而得超生。
换言之,即真正自我超越经验世界,免于痛苦和束缚。
此境界似与道家所论相似但不相同。
佛道二家均谓自我应超越经验世界。
佛教初入中国时,译佛经者多借用老庄之语,甚至出现老子化胡为佛的传说,便是基于此点。
但道家于经验世界,持超越和观赏的态度,此即庄子“胜物”“养生”之意。
而佛教视经验世界为痛苦和束缚之根源,遂对其持超越和舍离的态度,以求自我之“离苦”。
此种差异,甚为微妙,不可不察。

\newline

“灭”谛点明自我应该超越并舍离一切事象,即所谓“解脱”(Mokka)和“涅槃”(Nibbana)。
前者指免于经验事象之束缚,后者指免于生命之苦。
倘若自我未能超越并舍离经验世界,便会在不同的生命历程中流转。
生命结束,则该生命中所显现的自我遂进入另一生命历程。
如此,自我在生命中不断流转,即为“轮回”(Samsara)。
换言之,即灵魂之转世,此为宗教常有之观念。

\newline

灵魂转世受“业”(Kamma)的决定。
在生命中显现的自我或灵魂处于昏迷状态,不表现出自主性。
在昏迷状态下,自我盲目地推动生命活动的进行,由此产生的结果即为“业”。
“业”又决定自我之轮回。
这一历程完全为被决定者,无自主性可言。

\newline

如此,则如何达到“灭”之境界?
“道”谛之主旨即为此一实践问题,而阿含经中所言甚为繁琐。
盖佛教本为一宗教,故言及求“灭”之方法,不可避免地涉及诸多宗教仪式以及各种清规戒律。
本文不涉及宗教,仅论其哲学方面。

\newline

自我之所以为经验世界所囚禁,是因为自我处于昏迷状态。
故免于轮回之根本在于自我之觉醒,释迦谓之“觉”。
盖自我既有超越现实世界之潜力,其自身必有不受经验事象影响之能力。
因此,自我之“觉”与“不觉”,完全由其自身决定。

\newline

自我之“觉”,尚涉及培养功夫。
此类努力,包括“戒”(Sila)、“定”(Samadhi)、“慧”(Panna)三种。
“戒”指约束行为,即佛教种种清规。
“定”指意志之锻炼,即所谓“禅定”。
“慧”指对生命、世界之真相的领悟。
三者相合,乃成“正觉”。

\section{十二因缘}
“十二因缘”涉及灵魂之特殊性的形成以及经验世界的起源,为佛教原始教义的重要组成。
十二因缘包括:无明(Avijja),行(Sankhara),识(Vinnana),名色(Namarupa),六入(Sadayatana),触(Phassa),受(Vedana),爱(Tanha),取(Upadana),有(Bhava),生(Jati),老死(Jaramarana)。

\newline

“无明”即自我处于昏迷状态,不表现出自主性。
在“无名”状态下,自我只能进行盲目的意志活动,即为“行”。
由“行”生“识”,即认知主体的出现。
此主体同时为认知对象,即为“名色”。
在认知主体和认知对象对立的境遇中,认知主体分化为各种感觉能力,即“六入”。

\newline

感觉能力出现后方有在经验世界的种种活动。
“触”及感觉能力与经验世界的种种事象接触,遂产生接触之感受,即为“受”。
“爱”则是对某种感受之不舍。
由“爱”生“取”,即对某种感受之依恋追逐。
在这一过程中,不同的自我或者灵魂在“触”“受”“爱”“取”等不同方面产生差异,即为“有”。
这些彼此不同的灵魂在生命中显现,即为“生”;
生命之消逝即为“老死”。
至此,灵魂遂在经验世界不断“轮回”。

\newline

可见,十二因缘之说乃释迦对世界之真相的解释。
若就自我之真正显现而言,则应先从“生”和“老死”之循环中脱离,随后则是逐个破除从“有”至“无明”的种种因缘。
如此则自我得“正觉”,达到“灭”谛之舍离境界。

\end{document}