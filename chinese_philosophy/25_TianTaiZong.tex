\documentclass[11pt]{article}

\usepackage[UTF8]{ctex} % for Chinese 

\usepackage{setspace}
\usepackage[colorlinks,linkcolor=blue,anchorcolor=red,citecolor=black]{hyperref}
\usepackage{lineno}
\usepackage{booktabs}
\usepackage{graphicx}
\usepackage{float}
\usepackage{floatrow}
\usepackage{subfigure}
\usepackage{caption}
\usepackage{subcaption}
\usepackage{geometry}
\usepackage{multirow}
\usepackage{longtable}
\usepackage{lscape}
\usepackage{booktabs}
\usepackage{natbib}
\usepackage{natbibspacing}
\usepackage[toc,page]{appendix}
\usepackage{makecell}

\title{天台宗}
\date{}

\linespread{1.5}
\geometry{left=2cm,right=2cm,top=2cm,bottom=2cm}

\begin{document}

  \maketitle

  \linenumbers

佛教传入中国,经长期讲论,于隋唐时产生中国佛教徒自立之宗派,包括天台宗、华严宗、禅宗三支。
天台、华严依印度佛经,自造诸论,构建理论;
禅宗则不依经论,亦不重视宗教传统。
三宗教义虽殊,然皆为真常一系。
印度徒有真常诸经,其中所主张之主体自由等观念,未能流行。
故真常之教盛于中国。

\par

三宗皆重主体自由,颇类儒道,然儒释道三家理论立场有重大分歧。
儒家重视道德文化和社会秩序,以道德化成世界,主体自由体现于人类行为之应当和不应当。
道家和佛教之主体皆具有超越性,故主体自由为超越世界之自由。
然道家之主体旨在观赏世界之自然运转,佛教之主体旨在舍离、否定世界。
盖佛教视一切实有为虚妄幻影,故需加以破除。

\par

佛教与中国本有之思想在理论方向上根本不同,然就中国佛教之特有观念而言,仍有受中国本土哲学影响之处。
由主体之最高自由可引申出三个命题:
(1)主体皆有取得成就之能力;
(2)此能力是否发用,即主体是否取得成就,取决于主体本身;
(3)主体不仅有取得成就之自由,亦有堕落之自由。
这三个命题中的任意一个不成立,则主体自由不存。
儒道二家对“成就”一词理解不同,但于这三个命题,皆有涉及。
儒家重视道德成就,而道德实践由人之本性决定,故人人皆能有所成就。
然人亦可受利之影响,行不道德之事,遂堕落矣。
道家所定义的成就,系超越并观赏世界,如此则主体必有实现此成就之能力。
若此能力未发用,则主体陷溺于物,即为堕落。
换言之,儒道皆肯定主体自由,即赋予主体“上升”和“下降”之能力,遂与印度之种姓观念有不可调和之冲突。
而真常一系,论述“佛性”,即主体取得成就之能力,有进一步肯定主体自由之倾向,故不能为印度学者坦然接受,而其在中国的传播、讲论则无此困难。
故真常之学能盛于中国。
更兼秦燹之后,中国本土哲学式微,两汉经学、魏晋玄学之理论鄙陋浅薄,皆不能与佛教理论抗衡。
故迨至隋唐,佛教已占据中国思想之主位,遂有三宗之盛。

\par

此外,唐时又有玄奘(602-664)致力于返归印度佛教,力倡《成唯识论》,立法相宗。
其学全为印度唯识教义,兹不赘述。

\par

中国佛教之三宗,天台宗立教最早,故先论述。
天台宗之理论,始于智者大师(538-597),所依佛经为《妙法莲华经》和《大涅槃经》。

\section{百界千如三世间}
“百界千如三世间”统摄一切法。
兹先观“十如是”和“十界”之说。

\par

“十如是”指如是性(本性)、如是相(表象)、如是体(实体)、如是力(功能)、如是作(活动)、如是因(主要先在条件)、如是缘(辅助性先在条件)、如是果(直接后果)、如是报(间接后果)、如是本末究竟等(以上九者合成之全体过程)。
“十界”是十种自我境界,其中最上四界为佛、菩萨、缘觉、声闻,称为“四圣”;
以下六界为天、人、阿修罗、畜生、地狱、恶鬼。
十界相通,比如有人界,有倾向佛界的人界,有倾向畜生界的人界,如此遂有百界。
每一界皆有十如是,遂成千如。

\par

“三世间”即众生世间、国土世间、五阴世间。
盖一界有三世间,每一世间又有十如是。
由此,百界遂产生三千种组合,每一组合皆是主体可处之境界。
总之,“百界”“千如”“三世间”之说对主体境界进行分级,且各级互通,更进一步,则有“一念三千”之说。

\section{一念三千}
“三千”即前述三千种主体境界,“一念”则表主体自由。
盖自我在任何境界中,均可通往其它境界。
由此,凡可成圣,圣可堕凡,升降进退,悉归自我,主体自由立矣。
故《摩诃止观辅行传弘决》云:

\textit{当知己心,具一切佛法矣。}

另一方面,若将百界、千如、三世间皆视为对象,则一切对象交互相融。
万法之所以能相融相通,是由于万法皆主体所生,皆非独立实有。
此为龙树空义,智者以“三观”“三谛”论之。

\section{三观三谛}
“三观”,即“假”“空”“中”三观。
此说出自龙树《中论》,而智者从主体一面解释。
三观皆出自一心,此心即为主体。
主体取“假观”,则一切皆假;
取“中观”“空观”亦然。
如此,一切皆由主体决定,皆由主体所生,故一心生万法,万法交相融。

\par

“三谛”就客体一面而言,称为“即空”“即假”“即中”,其旨在于破除万法。
诸法呈现是假,诸法无独立性是空,诸法皆为一心所作,是中。
三观三谛,种种言论,一言蔽之,即一切客体皆为主体活动之产物。
故《摩诃止观辅行传弘决》云:

\textit{三界无别法,唯是一心作。}

\section{六即}
“六即”指成佛之努力次序,即主体逐渐显现之过程。
“理即”者,即主体空有上升之能力。
倘仅通过言语文字,了解到主体本有成佛能力,则为“名字即”。
更进一步,在实践中能自觉向成佛努力,则为“观行即”。
自觉努力渐有成效,接近正觉,遂成“相似即”。
再进一步,已然觉悟,初现佛性,为“分真即”。
“分真”者,得部分之真也。
天台宗最高境界称“究竟即”,如此则完全成佛矣。
  
\end{document}