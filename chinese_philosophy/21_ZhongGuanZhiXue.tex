\documentclass[11pt]{article}

\usepackage[UTF8]{ctex} % for Chinese 

\usepackage{setspace}
\usepackage[colorlinks,linkcolor=blue,anchorcolor=red,citecolor=black]{hyperref}
\usepackage{lineno}
\usepackage{booktabs}
\usepackage{graphicx}
\usepackage{float}
\usepackage{floatrow}
\usepackage{subfigure}
\usepackage{caption}
\usepackage{subcaption}
\usepackage{geometry}
\usepackage{multirow}
\usepackage{longtable}
\usepackage{lscape}
\usepackage{booktabs}
\usepackage{natbib}
\usepackage{natbibspacing}
\usepackage[toc,page]{appendix}
\usepackage{makecell}

\title{中观之学}
\date{}

\linespread{1.5}
\geometry{left=2cm,right=2cm,top=2cm,bottom=2cm}

\begin{document}

  \maketitle

  \linenumbers

中观之学,始于2世纪,由龙树(Nagarjuna)所创。
龙树弟子提婆(Deva)继为宣说,遂兴盛一时。
此派学说,以《大般若波罗蜜多经》(\textit{Maha Prajna Paramita})为据,故又称“般若宗”,其主要著作有龙树所著之《大智度论》和《中论》。
中国佛教徒又称中观之学为“空宗”或“三论总”。
现撮其大要如下。

\section{假名}
《大般若波罗蜜多经》言“假名”,意即一切对象皆只在言说过程中获得意义,一切法皆非实有。
不仅经验事物如此,即使觉悟境界和觉悟过程亦非独立于主体而自存之实在。
一切法皆主体所立。
所谓“有”,实为主体“使之有”;
所谓“无”,实为主体“使之无”。
此即为《中论》所论之“空”。
“空”者,一言以蔽之,即一切独立实有本身为不可解,故皆不存在。
如此,则一切法皆归于“空”。
此即《中论》所谓:

\textit{以有空义故,一切法得成;若无空义者,一切则不成。}

\section{因缘}
佛教言“因缘”,其中“因”(Hetu)指有决定力的条件,“缘”(Paccaga)指辅助条件。
故“因缘”遂指一切条件。
龙树以“因缘”释“空”。
《中论》云:

\textit{众因缘生法,我说即是空,亦为是假名,亦是中道义。}

诸法皆因缘生,故皆为被条件决定者,故皆非独立实有。
由此,“一切皆空”与“一切法皆因缘生”、“一切法皆假名”均是否定独立实有的存在。
“中道义”则是针对“有”和“无”。
既无任何独立实有,一切“有”“无”皆是主体“使之有”“使之无”,本质上无所谓“有”“无”,故云“中道”。

\newline

因缘所生之法非独立实有,因缘自身亦非实有,而仅依主体活动显现。
倘若依照《中论》文本之论辩方式,因缘若为独立实有,遂为“不可解”者。
此即龙树“破因缘”之义,即破“独立实有”。

\newline

由此可见中观教义之基本旨趣,即破除对“独立实有”之“执”。
换言之,即揭明所谓的“有”和“无”根本是主体活动的结果,完全受主体决定。
此处所言“独立实有”不仅包括经验世界的种种事象,且包括一切概念、一切意义、一切原则或原理。
一切一切,皆依主体显现,皆非独立实有。

\section{八不中道}
独立实有既不能成立,则用于描述实有之谓语皆不可用。
龙树取四对相反谓语为例,以示此义,即“八不中道”。
《中论》云:

\textit{不生亦不灭,不常亦不断,不一亦不异,不来亦不去。能说此因缘,善灭诸戏论。我稽首礼佛,诸说中第一。}

此偈语表明一切谓词均是用于描述实有者。
实有既不能成立,故谓语亦不成立。

\section{涅槃}
至此,一切实有皆被否定,主体遂达到真正的自由,即“涅槃”。
此时,主体不成为对象,不受一切条件决定,即“不受因缘”。
《中论》云:

\textit{受诸因缘故,轮转生死中。不受诸因缘,是名为涅槃。}

此为佛教所肯定的最高境界。
    
\end{document}