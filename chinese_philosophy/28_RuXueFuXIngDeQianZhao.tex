\documentclass[11pt]{article}

\usepackage[UTF8]{ctex} % for Chinese 

\usepackage{setspace}
\usepackage[colorlinks,linkcolor=blue,anchorcolor=red,citecolor=black]{hyperref}
\usepackage{lineno}
\usepackage{booktabs}
\usepackage{graphicx}
\usepackage{float}
\usepackage{floatrow}
\usepackage{subfigure}
\usepackage{caption}
\usepackage{subcaption}
\usepackage{geometry}
\usepackage{multirow}
\usepackage{longtable}
\usepackage{lscape}
\usepackage{booktabs}
\usepackage{natbib}
\usepackage{natbibspacing}
\usepackage[toc,page]{appendix}
\usepackage{makecell}

\title{儒学复兴的前兆}
\date{}

\linespread{1.5}
\geometry{left=2cm,right=2cm,top=2cm,bottom=2cm}

\begin{document}

  \maketitle

  \linenumbers
印度佛教自汉传入中国,经魏晋南北朝之演变,产生中国佛教之三宗,于隋唐大兴。
佛教哲学之精彩处,在于对主体性的肯定,其实质是一与儒道不同的心性论哲学。
中国佛教之兴盛,一方面是由于秦燹之后,古学失传。
汉儒之宇宙论,魏晋之玄学,皆粗鄙之论,不能与佛教相抗衡。
另一方面,佛教也在不断适应中国。
禅宗判客观世界为中性,于得悟无害亦无益,已是佛教对中国心灵作出的最大程度的适应。
然舍离世界为佛教不可让步之根本原则,终不能与中国精神相契合。
欲与佛教论短长,必须检讨汉儒之宇宙论,回归心性论哲学,直面主体和精神境界之问题。
此思想倾向滥觞于中晚唐。

\section{道教内丹派}
唐代,道教内丹派兴盛,借佛老儒之观念言修炼理论,有探索精神境界问题之倾向。
然道教理论驳杂无章,于思想发展之影响极小。
此处仅撮述道教之发展以及基本教义。

\par

道教源于先秦“有方之士”。
汉代始有宗教组织形式。
南北朝时,刘宋之陆修静(406-477),元魏之寇谦之(365-448),皆道流之领袖。
其教义以神仙方术为主,与先秦老庄之学相左。
更兼收阴阳五行之说,河图洛书之形,佛老周易之文,基于古代粗鄙幼稚之宇宙论,炼丹斋醮,采药烧香,行医问卜,以求长生不老,白日飞升,呼风唤雨,撒豆成兵。
就思想层面而言,道教代表了积极解释、控制、利用自然之态度,其建树多在科学技术方面。

\par

唐代,皇室李姓,遂与道教之“老君”联宗,道教大盛。
隋代苏元朗(?-?)倡“内丹”之说,自炼精气以求长生,轻视金石丹药。
其说盛行于唐,广传于宋。
此说实为杂取儒释道之说,讲一套修炼理论,对思想界影响极小,但代表了探索内部精神境界问题的趋势。

\section{韩愈}
韩愈(768-824),字退之,唐代反佛教倡儒学之先驱。
就文学而言,韩退之提倡古文,反对隋唐绮丽颓废之风,是“起八代之衰”的大文豪。
《谏佛骨表》可谓对佛教宣战之檄文:

\textit{今无故取朽秽之物,亲临观之,巫祝不先,桃茹不用,群臣不言其非,御史不举其失,臣实耻之。乞以此骨付之有司,投诸水火,永绝根本,断天下之疑,绝后代之惑。使天下之人,知大圣人之所作为,出于寻常万万也。岂不盛哉!岂不快哉!佛如有灵,能作祸祟,凡有殃咎,宜加臣身,上天鉴临,臣不怨悔。}

韩愈反佛教立场鲜明。
然退之不长于理论构建,亦不精坟籍考训。

\par

韩愈主张以孟子为儒学之正统。
《原道》云:

\textit{曰:“斯道也,何道也?”曰:“斯吾所谓道也,非向所谓老与佛之道也。尧以是传之舜,舜以是传之禹,禹以是传之汤,汤以是传之文、武、周公,文、武、周公传之孔子,孔子传之孟轲。轲之死,不得其传焉。}

孟子之后,道不传矣。秦汉以降,所传非道,皆为歧途。
《原道》云:

\textit{周道衰,孔子没,火于秦,黄老于汉,佛于晋、魏、梁、隋之间。其言道德仁义者,不入于杨,则归于墨;不入于老,则归于佛。}

此处实为文人铺排之语,仅表明立场。

\par

孟子之后,道既不传,而天下误以佛老之学为道。
《原道》评儒道曰:

\textit{博爱之谓仁,行而宜之之谓义,由是而之焉之谓道,足乎己而无待于外之谓德。仁与义为定名,道与德为虚位。故道有君子小人,而德有凶有吉。老子之小仁义,非毁之也,其见者小也。坐井而观天,曰天小者,非天小也。彼以煦煦为仁,孑孑为义,其小之也则宜。其所谓道,道其所道,非吾所谓道也。其所谓德,德其所德,非吾所谓德也。凡吾所谓道德云者,合仁与义言之也,天下之公言也。老子之所谓道德云者,去仁与义言之也,一人之私言也。}

《谏佛骨表》又评佛教曰:

\textit{夫佛本夷狄之人,与中国言语不通,衣服殊制;口不言先王之法言,身不服先王之法服;不知君臣之义,父子之情。假如其身至今尚在,奉其国命,来朝京师,陛下容而接之,不过宣政一见,礼宾一设,赐衣一袭,卫而出之于境,不令惑众也。况其身死已久,枯朽之骨,凶秽之馀,岂宜令入宫禁?}

韩愈行文气势非凡,然其旨不过言儒学与佛老殊途。

\par

韩愈反佛老立场鲜明,然其理论浅薄。
《原道》驳斥佛老,确立儒学,曰:

\textit{帝之与王,其号虽殊,其所以为圣一也。夏葛而冬裘,渴饮而饥食,其事虽殊,其所以为智一也。今其言曰:“曷不为太古之无事”?”是亦责冬之裘者曰:“曷不为葛之之易也?”责饥之食者曰:“曷不为饮之之易也?”传曰:“古之欲明明德于天下者,先治其国;欲治其国者,先齐其家;欲齐其家者,先修其身;欲修其身者,先正其心;欲正其心者,先诚其意。”然则古之所谓正心而诚意者,将以有为也。今也欲治其心而外天下国家,灭其天常,子焉而不父其父,臣焉而不君其君,民焉而不事其事。孔子之作《春秋》也,诸侯用夷礼则夷之,进于中国则中国之。经曰:“夷狄之有君,不如诸夏之亡。”《诗》曰:“戎狄是膺,荆舒是惩”今也举夷狄之法,而加之先王之教之上,几何其不胥而为夷也?}

行文立场鲜明,情感充沛,但全然不见理论力量。

\par

韩愈另有《原性》之文,亦多朦胧语。
开篇云:

\textit{性也者,与生俱生也;情也者,接于物而生也。}

言“性”为生而有者,可见韩愈不解孟子性善之说。
下文谓性分上中下三品,曰:

\textit{曰性之品有上、中、下三。上焉者,善焉而已矣;中焉者,可导而上下也;下焉者,恶焉而已矣。}

又曰性包含仁义礼智信五种美德,曰:

\textit{其所以为性者五:曰仁、曰礼、曰信、曰义、曰智。}

性既“与生俱生”,分为三品,而又包含诸德。
此理不通。

\par

《原性》下文又谓:

\textit{孟子之言性曰:人之性善;苟子之言性曰:人之性恶;扬子之言性曰:人之性善恶混。夫始善而进恶,与始恶而进善,与始也混而今也善恶,皆举其中而遗其上下者也,得其一而失其二者也。}

可见韩愈不皆孟荀扬之学。

\par

总之,韩愈反佛倡儒立场坚定,为儒学复兴之先驱,但其理论浅薄。

\section{李翱}
李翱(?-?),字习之,大致与韩愈同时。
其说主要见诸《复性书》。

\par

《复性书》开篇先论“性”与“情”,其文曰:

\textit{人之所以为圣人者,性也,人之所以惑其性者,情也。喜怒哀惧爱恶欲,七者皆情之所为也。情既昏,性斯匿矣,非性之过也,七者循环而交来,故性不能充也。}

此处言“性”为成圣之基础。
人实现其“性”,遂为圣人。
然“性”会受“情”之阻碍,其不能实现“非性之过”。

\par

“情”可阻碍“性”之实现,非消灭“情”之意,只是将“情”置于“性”之统帅。
《复性书》上云:

\textit{虽然,无性则情无所生矣。是情由性而生,情不自情,因性而情。性不自性,由情以明。性者天之命也,圣人得之而不惑者也;情者性之动也,百姓溺之而不能知其本者也。}

“情”之存在依附于“性”,故曰“因性而情”。
“情”为“性”之显现,故曰“由情以明”。
凡俗之人陷溺于“情”,“性”不显现。
圣人之“性”充分发用,故曰“得之而不惑”。
然圣人亦非无“情”,凡俗亦非无“性”。
下文又谓:

\textit{圣人者岂其无情耶?圣人者,寂然不动,不往而到,不言而神,不耀而光,制作参乎天地,变化合乎阴阳,虽有情也,未尝有情也。然则百姓者,岂其无性耶?百姓之性与圣人之性弗差也,虽然,情之所昏,交相攻伐,未始有穷,故虽终身而不自睹其性焉。}

总之,人人皆有“性”“情”。
倘“性”占支配地位,其发用使人成圣;
反之,则“情”使“性”昏,遂坠凡俗。
故《复性书》以“尽其性”为价值所在。
《复性书》上云:

\textit{《易》曰:“夫圣人者,与天地合其德,日月合其明,四时合其序,鬼神合其吉凶,先天而天不违,後天而奉天时。天且勿违,而况於人乎?况於鬼神乎?”此非自外得者也,能尽其性而已矣。}

\end{document}