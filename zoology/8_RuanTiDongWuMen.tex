\documentclass[11pt]{article}

\usepackage[UTF8]{ctex} % for Chinese 

\usepackage{setspace}
\usepackage[colorlinks,linkcolor=blue,anchorcolor=red,citecolor=black]{hyperref}
\usepackage{lineno}
\usepackage{booktabs}
\usepackage{graphicx}
\usepackage{float}
\usepackage{floatrow}
\usepackage{subfigure}
\usepackage{caption}
\usepackage{subcaption}
\usepackage{geometry}
\usepackage{multirow}
\usepackage{longtable}
\usepackage{lscape}
\usepackage{booktabs}
\usepackage{natbib}
\usepackage{natbibspacing}
\usepackage[toc,page]{appendix}
\usepackage{makecell}

\title{软体动物门(Mollusca)}
\date{}

\linespread{1.5}
\geometry{left=2cm,right=2cm,top=2cm,bottom=2cm}

\begin{document}

  \maketitle

  \linenumbers
软体动物各类群形态差异较大,但都有头、足和内脏团(visceral mass)。
头位于躯体前端,足位于腹侧,内脏团位于足的背侧。
体背侧皮肤褶形成外套膜(mantle),常包裹内脏团。
外套膜和内脏团之间的空腔称为外套腔(mantle cavity),联通体外,上常有鳃、足、肾孔、生殖孔、肛门等。
外套腔壁处的上皮有纤毛,促进水流在外套腔的循环。
外套膜外层上皮的分泌物,能形成贝壳。
左右两片外套膜后缘处常有一或两处愈合,形成出水孔(exhalant siphon)和入水孔(inhalant siphon)。

\newline

贝壳(shell)是软体动物的重要特征,起保护和维持体型的作用。
其主要成分是碳酸钙和贝壳素(conchiolin)。
贝壳最外一层为角质层(periostracum),薄且透明,有光泽,主要成分为贝壳素,不受酸碱侵蚀。
中间一层为壳层(ostracum),主要成分为碳酸钙。
最内为壳底(hypostracum),即珍珠质层(pearl layer),有光泽。
角质层和壳层的生长受环境影响,并非连续不断的,由此形成贝壳表面的生长线。

\newline

软体动物的消化道完整,消化腺发达。
多数种类口腔底部有颚片(mandible)和齿舌(radula)。
颚片可辅助捕食。
齿舌表面有横列的角质齿,呈锉刀状。
摄食时,齿舌前后伸缩,刮取食物。

\newline

软体动物体腔退化,仅残留围心腔(pericardinal cavity)、生殖腺和排泄器官内腔等。
假体腔则见于各种组织间隙,形成血窦。
心脏位于内脏团背侧围心腔内,由一个能搏动的心室和数对心耳组成。
心室和心耳之间有瓣膜,防止血液逆流。
血管分化为动脉和静脉。
血液经心脏流入动脉,而后进入血窦,再经静脉流回心脏。
故软体动物的循环系统为开管式循环(open circulation),在循环过程中血液进入组织间隙。
开管式循环的效率不如闭管式。
软体动物的开管式循环,与其大部分种类低下的运动能力相适应。

\newline

软体动物中,水生种类以鳃呼吸。
鳃为外套腔内面皮肤伸展形成的,位于外套腔内。
陆生种类无鳃,外套腔内部分区域的微细血管集中分布,形成肺,司气体交换。
其排泄器官为肾管,分为腺质部分和管状部分。
腺质部分富血管,开口于围心腔,肾口部分有纤毛;
管状部分内壁有纤毛,肾孔开口于外套腔。

\newline

软体动物的神经系统变化较大。
原始种类仅有围咽神经环和向体后伸出的一对足神经索(pedal cord)和一对侧神经索(pleural cord)。
较高等的种类主要有四对神经节,彼此以神经相连。
脑神经节(cerebral ganglion)位于食管背侧,向前发出神经;
足神经节(pedal ganglion)位于足的前端,向足部发出神经;
侧神经节(pleural ganglion)向鳃和外套膜发出神经;
脏神经节(visceral ganglion)向内脏发出神经。
软体动物的皮肤、外套膜内层和触角均可司感觉,司感光的眼结构繁简不一;
另有嗅检器(osphradium)和平衡囊等感觉器官。

\newline

软体动物大多雌雄异体,一般为间接发育,有担轮幼虫期。
  
\section{无板纲(Aplacophora)}
呈蠕虫状,体表背具有石灰质细棘的角质外皮,无贝壳。
口位于体前腹侧。
腹侧中央有一腹沟,沟内或有具纤毛的足。
体后有泄殖腔,腔内一般有一对鳃。
无感觉器官,血管系统退化。

\subsection{毛皮贝目(Chaetodermoida)}
体圆筒状,头部通过收缩部与体躯分开。
体被角质带棘歪批,无腹沟。
排泄腔内有发达的羽状鳃。
中肠有盲囊,起肝的作用。
肾亦作为生殖输送管。
雌雄异体,无交接器。
如毛皮贝(\textit{Chaetoderma})。

\subsection{新月贝目(Neomenioida)}
头、排泄腔与体躯界限不明显。
口位于腹面近前,排泄器接近体后端。
有腹沟,沟内有足,有足腺。
鳃位于肛门边缘,褶叠状。
中肠无盲囊。
如龙女簪(\textit{Proneomenia})、新月贝(\textit{Neomenia})。

\section{单板纲(Monoplacophora)}
有一笠形贝壳,壳顶在中央靠前处。
壳表有围绕壳顶的环状生长线。
足发达,五或六对鳃环列于足的周围。
头不明显,齿舌发达。
六对肾,一对开口于体前,其余五对开口于鳃的基部。
雌雄异体,两对生殖腺在围心腔前。
多化石种类。
现代品种如新碟贝(\textit{Neopilina galathea})。

\section{多板纲(Polyplacophora)}
呈椭圆形。
背侧有八块贝壳,呈覆瓦状排列。
体前背侧第一块贝壳呈半月形,称为头板(cephalic plate);
体后最后一块呈元宝状,称为尾板(tail plate);
中间六块为中间板(intermediate plate)。
各板可移动,故多板纲动物可卷曲起来。
贝壳下方有一圈外套膜,上丛生针状或棘状突起。
头不发达,位于体前腹侧,有一向下的短吻,吻中央为口。
足宽大,吸附力腔。
足和外套膜之间有一圈狭窄的外套腔,腔内两侧生鳃。

\newline
  
口腔内有齿舌,前有一对唾液腺。
食管后有一对食道腺,胃周围为肝。
真体腔发达。
排泄器官为一对后肾管,肾口开于围心腔,肾孔位于外套腔。
神经系统包括环食管的神经环和向后伸出的侧神经索、足神经索。
侧神经索伸至外套膜和内脏,足神经索伸至足。
神经索之间有神经相连,呈梯状。
雌雄异体,生殖导管开口于外套腔。
  
\subsection{鳞侧石鳖目(Lepidopleurida)}
体狭小,贝壳无嵌入片。
多深海产。
浅海常见如鳞侧石鳖(\textit{Lepidopleurus})。

\subsection{石鳖目(Chitonida)}
体长椭圆形,壳板翼部发达。
多见于沿海潮间带,如锉石鳖(\textit{Ischnochiton})。

\subsection{毛肤石鳖目(Acanthochitonida)}
头板嵌入片有齿裂,中间板各侧一个齿裂或无。
环带发达,生鳞片针束。
如毛肤石鳖(\textit{Acanthochiton})

\section{腹足纲(Gastropoda)}
头发达,有眼和触角。
足发达,呈叶状,位于腹侧,上有单细胞黏液腺。
体外一般有一个螺旋形贝壳。
壳一般为右旋。
壳分为两部分,在上者为包含内脏器官的螺旋部(spire)和容纳头、足的体螺层(body whorl)。
体螺层开口称壳口(aperture)。
壳口常有由足后端分泌的盖,称为厣。

\newline

口腔有颚片和齿舌,唾液腺分泌无消化功能的黏液。
肝脏发达。水生种类以鳃呼吸,陆生种类以肺呼吸。
肾呈长形,两端开口于围心腔和外套腔。
神经系统包括脑、足、侧、脏四对神经节,感觉器官有眼、触角、嗅检器、味蕾、平衡囊等。

\newline

海产种类多雌雄异体,陆生种类多雌雄同体。
异体受精,有交配行为。
生殖腺位于内脏团背面,生殖管开口于体前右侧,即为生殖孔。
雌雄同体的种类,生殖腺后为两性管和输精卵管。
输精卵管后端分为通向交接器的输精管和通向阴道的输卵管。
交接器和阴道均由生殖孔通向体外。
  
\subsection{前鳃亚纲(Prosobranchia)}
外壳螺旋形,外套腔位于体前,头部有一对触角。
鳃位于心室前方,侧脏神经连索左右交叉成8字形,雌雄异体。

\subsubsection{原始腹足目(Archaeogastropoda)}
鳃楯状,一般两个心耳,神经系统集中不明显。
腹神经节长索状,左右两个脏神经节彼此原理,一个脑下食管神经索。
嗅检器不发达,位于鳃神经上。
平衡器内多耳沙。
肾一对,开口于乳状突起。
生殖腺一般开口于右肾。
吻或水管缺失,齿舌小齿极多。
如鲍(\textit{Haliotis})。

\subsubsection{中腹足目(Mesogastropoda)}
神经系统集中,常缺失唇神经连索。
平衡器一个,内有一枚耳石。
唾液腺位于食管神经节后,无食管附属腺、吻和水管。
右侧排泄、呼吸器官退化。
一心耳,不被直肠穿过。
鳃一枚,栉状,附于外套膜。
肾开口于体表。
有生殖孔和交接器。
如圆田螺(\textit{Cipangopaludina})、沼螺(\textit{Parafossarulus})、福寿螺(\textit{Ampullaria})、钉螺(\textit{Oncomelania})、宝贝(\textit{Cypraea})。

\subsubsection{新腹足目(Neogastropoda)}
神经系统集中,有外壳和水管沟。
食管神经环位于唾液腺后,没有被唾液腺疏松管穿过。
胃肠神经节位于脑神经中枢附近。
口吻发达,食管腺不成对。
外套膜部分包卷为水管。
嗅检器羽毛状。
雌雄异体,雄性有交接器。
海产。
如芋螺(\textit{Conus})。

\subsection{后鳃亚纲(Opisthobranchia)}
贝壳不发达,部分无壳。
鳃位于心室后方,侧脏神经连索左右不交叉。
多雌雄同体,全部海产。

\subsubsection{头楯目(Cephalaspidae)}
贝壳发达,或多或少成螺旋形,大多无厣。
外套腔发达,外套膜后部成大叶状。
头无触角,头背面由挖掘用的楯板。
眼无柄,侧足发达。
胃中有咀嚼板。
侧神经连索长。
如泡螺(\textit{Hydatina})。

\subsubsection{无楯目(Anaspidae)}
俗称海兔。
无头楯,两对触角。
贝壳薄,被外套膜包裹。
足两侧位于贝壳上。
如海兔(\textit{Aplysia})。

\subsubsection{被壳翼足目(Thecosomata)}
有石灰质壳或软骨厚皮。
有厣。
足前侧翼状,用来浮游。
如龟螺(\textit{Cavolinia})。

\subsubsection{裸体翼足目(Gymnosomata)}
成体无外套膜,无贝壳。
足两侧为翼状,用来浮游。
如海若螺(\textit{Clione})。

\subsubsection{囊舌目(Sacoglossa)}
壳、外套膜、本鳃均消失。
触手一对。
齿舌藏于背侧囊内,其上一列小齿。
如长足螺(\textit{Oxynoe})。

\subsubsection{无壳目(Acochlidiacea)}
成体无贝壳,有骨针。
内脏团长,位于体后。
背部无附属物。
如无壳螺(\textit{Acochlidium})。

\subsubsection{背楯目(Notaspidea)}
无侧足,无外套腔,栉鳃发达。
如伞螺(\textit{Umbraculum})。

\subsubsection{裸鳃目(Nudibranchia)}
无壳,无外套膜,无本鳃。
体背有次生鳃。
内脏团平坦,齿舌🍀列小齿。
如海牛(\textit{Doris})。

\subsection{肺螺亚纲(Pulmonata)}
陆生或淡水生。
无鳃,外套膜特化为肺囊,部分水生品种有次生鳃。
水生种类一对触角,陆生品种两对。
无厣,侧脏神经连索左右不交叉,雌雄同体。

\subsubsection{基眼目(Basommatophore)}
有外壳,一对触角,眼位于触角基部。
如椎实螺(\textit{Lymnaea})、扁卷螺(\textit{Hippeutis})。

\subsubsection{柄眼目(Stylommatophore)}
两对触角。
眼位于后触角顶端。
如华蜗牛(\textit{Cathaica})、蛞蝓(\textit{Agriolimax})。

\section{瓣鳃纲(Lamellibranchia)}
身体侧扁,体侧有两片贝壳,两片外套膜分别位于贝壳内面。
头部退化,足呈斧状,鳃呈瓣状。
贝壳背面有突出的壳顶(umbo),壳顶前后一般分别有小月面和楯面。
壳边缘较厚,有互相咬合的齿和齿槽,构成铰合部(hinge)。
铰合部连接两壳的背缘有角质韧带(ligament),可连接两片贝壳。

\newline

外套膜薄而透明,边缘较厚,常有触手。
外套膜上有两个连接左右两侧的闭壳肌分别位于体前和体后。
鳃位于外套腔中,从体前向后延伸至肛门,司呼吸和滤食。
足位于腹面,两侧扁平,前端呈斧状。
口位于体前,具纤毛,两侧各有一对三角形唇瓣,无齿舌和口腔腺。
胃壁厚,位于内脏团。肠细长,直肠穿过围心腔,开口于体后。
肾管一对,开口于围心腔和外套腔。
神经系统有脑、足、脏神经节各一对。
一般雌雄异体,生殖管开口于肾管内或肾孔附近。
体外受精。

\subsection{古多齿亚纲(Palaeotaxodonta)}
两壳等大,可完全闭合。
壳表面有黄绿色壳皮。
铰齿多,沿前后背缘分布。
有内外韧带。
前、后闭壳肌相同。
栉鳃,成体无足丝。
滤食。

\subsubsection{胡桃蛤目(Nuculoida)}
壳小而厚,卵圆。
鳃小,鳃丝横列。
如云母蛤(\textit{Yoldia})。

\subsection{隐齿亚纲(Cryptodonta)}
多等壳,小而薄。
铰齿少,外韧带。
闭壳肌多为等柱。
海产,滤食。

\subsubsection{蛏螂目(Solemyoida)}
如蛏海螂(\textit{Solemya})。

\subsection{翼形亚纲(Pterimorphia)}
外韧带。
外套膜完整。
有丝鳃或瓣鳃。
有足丝。

\subsubsection{魁蛤目(Arcoida)}
多不等壳。
壳表面有放射肋,内侧腹缘有细齿。
铰齿盘笔直,多细齿。
后闭壳肌大于前闭壳肌。
韧带面三角形。
有血红蛋白。
如魁蛤(\textit{Arca})、毛蚶(\textit{Senilia})。

\subsubsection{狐蛤目(Limoida)}
两壳相等,桨状,铰合线沿三角形韧带凹槽的两侧倾斜。
壳顶宽圆,放出密集圆肋,肋上布满凹槽状鳞。
壳内缘宽锯状。
如大黄狐蛤(\textit{Acesta marissinica})。

\subsubsection{贻贝目(Mytiloida)}
壳同形,壳皮发达,铰合齿退化为结节状小齿。
后闭壳肌大,前闭壳肌退化。
心脏仅一大动脉,生殖腺扩大至外套膜中,生殖孔位于肾外孔旁,肛门孔明显。
足小,足丝发达。
如贻贝(\textit{Mytilus})。

\subsubsection{牡蛎目(Ostreoida)}
左壳固定于岩石,大于右壳。
铰合齿退化,前闭壳肌退化。
无足,无足丝。
如牡蛎(\textit{Ostrea})

\subsubsection{莺蛤目(Pterioida)}
铰合齿退化。
鳃丝屈折,以纤毛盘相连接。
前闭壳肌小于后闭壳肌。
足不发达。
如栉孔扇贝(\textit{Chlamys farreri})、珍珠贝(\textit{Pteria})。

\subsection{古异齿亚纲(Palaeoheterodonta)}
壳文石质,内壳层珠母质。
等柱闭壳肌,外韧带,外套线完整。

\subsubsection{三角蛤目(Trigonioida)}
壳三角形,壳面有结节和同心脊。
如三角蛤(\textit{Trigonia})。

\subsubsection{河蚌目(Unionoida)}
铰合齿少,闭壳肌发达等大。
鳃丝、鳃瓣间以血管相连。
出入水孔多形成水管。
如河蚌(\textit{Anodonta})。

\subsection{异齿亚纲(Heterodonta)}
壳文石质,内壳层无珠母质。
等柱闭壳肌,外韧带。

\subsubsection{海螂目(Myoida)}
壳薄,壳顶不凸出。
如象拔蚌(\textit{Panopea abrupta})、船蛆(\textit{Teredo navalis})。

\subsubsection{帘蛤目(Veneroida)}
等壳,铰合部发达。主齿强大。
水管发达。
如砗磲(\textit{Tridacna})、竹蛏(\textit{Solen})。

\subsection{异韧带亚纲(Anomalodesmacea)}
壳不等,内壳层珠母质,铰合齿不发达。
韧带位于壳顶内面槽中,常有石灰质小片。

\subsubsection{笋螂目(Pholadomyoida)}
壳薄,后端圆形。
壳表面粗糙。
铰合线上无齿,但有短小脊骨供韧带附着。
如中国杓蛤(\textit{Cuspidaria chinensis})。

\section{头足纲(Cephalopoda)}
头位于体前,其顶端为口,口周围有口膜。
头两侧有发达的眼,眼后有椭圆形小窝,为嗅觉陷。
足环列于头前口周,形成数十只、十只或八只腕。
亦有一部分足形成位于头和躯干之间的腹面的漏斗,其开口朝向体前。
漏斗腔和外套腔相连。
仅少数种类有外壳,多数种类外壳包埋于外套膜,形成内壳。
部分物种内壳退化。
内壳可支撑身体,有利于保持平衡。
头足类软骨发达,主要包括包围中枢神经系统和平衡囊的头软骨、颈软骨和腕软骨。
大部分种类皮下有扁平状、富有弹性的色素细胞(chromatophore),周围有肌纤维。
肌纤维的收缩控制色素细胞的舒张,改变皮肤颜色。

\newline

头足类口腔内有颚片和齿舌,肝脏发达。
部分种类直肠末端有梨形盲囊,称为墨囊(ink sac)。
囊内腺体分泌墨汁,经位于外套腔的肛门排出。
鳃呈羽状,位于外套腔内。
循环系统近似闭管式。
心脏位于体后腹部中央的围心腔内。
由心脏向前、向后各伸出一大动脉。

\newline

头足类神经系统和感觉器官发达,有眼、嗅觉陷、平衡囊等。
眼的结构复杂。
瞳孔(pupil)周围为虹膜。
瞳孔外侧覆透明的角膜(cornea),瞳孔后为晶状体。
晶状体两侧有与虹膜平行分布的睫状肌。
眼的内侧为视网膜,视网膜内层为感光细胞,外层神经纤维于眼后端汇合为视神经。

\newline

头足类雌雄异体,有求偶和交配行为。

\subsection{四鳃亚纲(Tetrabranchia)}
有外壳。
腕无吸盘。
鳃、心耳、肾各两对。
大多为化石种,如菊石(\textit{Ammononite})、箭石(\textit{Belemnite})。
生活种仅鹦鹉螺属(\textit{Nautilus})四种。

\subsection{二鳃亚纲(Dibranchia)}
无外壳,部分有内壳。
腕生吸盘。
鳃、心耳、肾各一对。

\subsubsection{十腕目(Decapoda)}
有腕十对,右侧第五腕为茎化腕,吸盘有柄,有内壳。
如曼氏无针乌贼(\textit{Sepiella maindroni})、中国枪乌贼(\textit{Loligo chinensis})、柔鱼(\textit{Ommatostrephes})。

\subsubsection{八腕目(Octopoda)}
有腕四对,右侧第三腕为茎化腕,吸盘无柄,无内壳。
躯干短,近乎球形。
如章鱼(\textit{Octopus})。

\section{掘足纲(Scaphopoda)}
全部海产,贝壳呈象牙形,粗的一端为前端,上有较大的头足孔;
细的一端为后端,上有较小的肛门孔。
壳凸出的一面为背侧,凹的一面为腹侧。
外套膜呈管状,前后有开口。
头不明显,前端有不能伸缩的吻,吻基部有可伸缩的头丝(captacula),司触觉和摄食。
吻内为口球,内有颚片和齿舌。
足在吻基部之后,柱状,可伸长以挖掘泥沙。
肛门位于足基部腹侧,开口于外套腔。
以外套膜交换气体。
无血管,仅有血窦。
肾一对,位于胃的侧面。
雌雄异体。

\subsection{角贝目(Dentaliacea)}
贝壳角状。
足前端尖,两个翼状褶。
口周围八个叶状唇瓣。
如大角贝(\textit{Dentalium vernedei})。

\subsection{管角贝目(Siphonodentaliacea)}
贝壳梭形,足末端盘状。口周无唇瓣。
如棱角贝(\textit{Cadulus})。
  
\end{document}