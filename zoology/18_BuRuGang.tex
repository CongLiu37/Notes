\documentclass[11pt]{article}

\usepackage[UTF8]{ctex} % for Chinese 

\usepackage{setspace}
\usepackage[colorlinks,linkcolor=blue,anchorcolor=red,citecolor=black]{hyperref}
\usepackage{lineno}
\usepackage{booktabs}
\usepackage{graphicx}
\usepackage{float}
\usepackage{floatrow}
\usepackage{subfigure}
\usepackage{caption}
\usepackage{subcaption}
\usepackage{geometry}
\usepackage{multirow}
\usepackage{longtable}
\usepackage{lscape}
\usepackage{booktabs}
\usepackage{natbib}
\usepackage{natbibspacing}
\usepackage[toc,page]{appendix}
\usepackage{makecell}

\title{哺乳纲(Mammalia)}
\date{}

\linespread{1.5}
\geometry{left=2cm,right=2cm,top=2cm,bottom=2cm}

\begin{document}

  \maketitle

  \linenumbers
哺乳动物通过胎生(vivipary)和哺乳,有效地提高后代成活率。
哺乳动物胚胎的绒毛膜、尿囊膜和母体子宫壁内膜结合,形成胎盘(placenta),联系母体和胚胎。
绒毛膜上生指状突起,插入子宫内膜,负责母体和胚胎间的物质交换。
胎生为发育中的胚胎提供稳定的营养供给和环境条件,最大程度减少外界环境对胚胎发育成长的不利影响。

\newline

胚胎在母体内完成发育的过程称为妊娠(gestation)。
妊娠结束,产出幼儿,称为分娩。
母体分泌乳汁,哺育幼崽,即为哺乳。
哺乳为后代提供优越的营养条件,更兼哺乳动物有较完善的保护幼崽行为,有效提高幼崽成活率。
与之相应,哺乳动物产崽数较低。

\newline

哺乳动物皮肤结构致密防水,可有效抵抗张力和病原菌入侵。
体表被毛(hair),司触觉和保温。
皮肤腺发达,有皮脂腺(sebaceous gland)、汗腺(sweat gland)、乳腺(mammary land)和臭腺(scent gland)。
哺乳动物通过汗液蒸发调节体温并排出部分代谢废物。
皮肤特化形成爪(claw)和角(horn)。

\newline

哺乳动物骨骼系统发达,骨化完全。
头骨骨块多彼此愈合,以满足对坚固性和轻便性的需求。
有完整的次生腭,分割口腔和鼻腔。
鼻腔扩大,鼻甲骨发达。
脑颅腔扩大。
下颌由单一骨块构成。
头骨两侧有颧弓(zygomatic arch),为咀嚼肌提供支点。
脊柱富有韧性,分为颈椎、胸椎、腰椎、荐椎、尾椎。
颈椎七枚。
胸椎附生肋骨,肋骨下端连胸骨,构成胸廓。
附肢下移至腹面,与地面垂直。
肢骨长而强健,以前后运动为主。

\newline

肌肉系统与爬行动物类似,但皮肤肌发达,咀嚼肌强大。
此外,哺乳动物出现膈肌,位于胸廓后端的肋骨后缘,分割胸腔和腹腔。
膈肌运动,改变胸腔容积,完成呼吸运动。

\newline

哺乳动物消化系统发达。出现肉质的唇(lip),参与摄食,辅助咀嚼。
口缩小,牙齿外侧出现颊(cheek),避免咀嚼时食物掉落。
肌肉质的舌(tongue)发达,表面生味蕾(taste bud)。
齿型出现分化,分为司切割的门齿(incisor)、司撕裂的犬齿(canine)和司切压、研磨的臼齿(molar)。
出现口腔消化。
口腔和鼻腔均开口于咽,咽后通食管和气管。
气管和咽交界处,即喉门处,生会厌软骨(epiglottis)。
吞咽时会厌软骨封闭喉门,避免食物进入气管。
小肠高度分化,出现乳糜管(lacteal),为输送脂肪的淋巴管。
小肠和大肠交界处有盲肠,为通过发酵消化植物的场所。
直肠直通肛门,开口于体外,无泄殖腔。
消化腺发达,包括唾液腺、肝和胰。
唾液入口腔,胆汁和胰液入十二指肠。

\newline

哺乳动物以肺呼吸。
鼻腔膨大,出现伸入头骨的鼻旁窦,以温暖、湿润、过滤空气,兼司发声共鸣。
气管前端膨大为喉,通咽,上生会厌软骨。
气管下端分为支气管,支气管在肺中不断分支,盲端为肺泡。
通过膈肌和胸廓的运动进行呼吸。

\newline

哺乳动物为恒温动物,有完全的双循环系统,心脏两心房两心室。
右心室血液经肺动脉、肺静脉至左心房,构成肺循环。
左心房血液进入左心室,经体动脉、体静脉回到右心房,构成体循环。
右心房血液再进入右心室。
哺乳动物仅有左体动脉弓,向后延申为背大动脉,沿途分支至全身。
前、后大静脉各一条,无肾门静脉和腹静脉。
淋巴系统极为发达。
淋巴管收集组织液,经胸导管(thoracic duct)入前大静脉。
淋巴通路中常有淋巴节,可阻拦异物,保护机体,亦是淋巴细胞发育场所,司免疫。

\newline

哺乳动物主要排泄器官为肾,皮肤亦有排泄功能。
肾小管汇集为集合管(collecting tubule),二者皆有重吸收水分、浓缩尿液的功能。
集合管再汇集为输尿管,入膀胱。膀胱以尿道直接或间接通体外。

\newline

神经系统高度发达。
大脑、小脑体积增大。
大脑皮层加厚,表面有褶皱。
中脑相对萎缩。
脑神经十二对。
延脑后接脊髓。
植物神经系统发达,负责调节内脏器官、腺体、心脏、血管的活动,其中枢位于脑干、胸椎、腰椎、荐椎等特定部位,传出神经在自主神经节内更换神经元后通效应器。

\newline

嗅觉发达,鼻腔扩大,鼻甲骨发达。
听觉敏锐,内耳下端形成发达的耳蜗(cochlea),中耳内有三块相关联的听骨,外耳发达可运动。
大部分哺乳动物色感受能力差。

\newline

雄性有一对睾丸,位于阴囊(scrotum)。
睾丸由精小管(seminiferous tubule)构成。
精小管经输出小管(vas efferens),入附睾(epididymis)。
附睾下端经输精管,通入尿道。
尿道被海绵体包裹,构成阴茎,为交配器官。
雌性有一对卵巢。输卵管上端开口于腹腔,下通子宫。
子宫下通阴道,尿道亦与阴道汇合。
哺乳动物性成熟后,再一年中的某些季节,规律性地进入发情期,称为动情。
雌性卵子于动情期间成熟并排出。

\section{原兽亚纲(Prototheria)}
卵生,雌性有孵卵行为。
乳腺仍为特化的汗腺,无乳头。
肩带结构类似爬行类。
有泄殖腔。
雄性无交配器官。
大脑皮质不发达。
成体无齿。
体温波动较大。
如鸭嘴兽(\textit{Ornithorhynchus anatinus})、针鼹(\textit{Tachyglossus aculeatus})。

\section{后兽亚纲(Metatheria)}
胎生,但无真正的胎盘。
胚胎通过卵黄囊和母体子宫壁接触。
妊娠期短,幼崽发育不良,出生后需在雌性腹部育儿袋中长期发育。
泄殖腔退化,但仍有残留。
有乳头、乳腺和异型齿。
门齿较多。
大脑皮质不发达。
体温波动较小。
如灰袋鼠(\textit{Macropus giganteus})、负鼠(\textit{Didelphis spp.})。

\section{真兽亚纲(Eutheria)}
有真正的胎盘,胚胎通过尿囊与母体子宫壁接触。
胎儿发育完全后再产出。
无泄殖腔。
乳腺发育充分,有乳头。
肩带为单一肩胛骨。

\subsection{食虫目(Insetivora)}
原始类群。
个体较小,吻尖细,四肢短小,有爪。
齿结构原始。
体被绒毛或硬刺。
多食昆虫、蠕虫。
如刺猬(\textit{Erinaceus europaeus})、缺齿鼹(\textit{Mogera robusta})。

\subsection{树鼩目(Sandentia)}
外形略似松鼠。
结构似食虫目,但有灵长目的特征,如嗅叶小、脑颅宽大、完整的骨质眼眶环。
如树鼩(\textit{Tupaia glis})。

\subsection{翼手目(Chiroptera)}
飞行,夜行性。
前肢特化,指骨延长。
由指骨末端至肱骨、体侧、后肢、尾,生薄而韧的翼膜。
前肢第一指有爪,第二指可能有爪。
后肢短,有长钩爪,适合悬挂。
胸骨有突起,锁骨发达。
齿尖。
如东方蝙蝠(\textit{Vespertilio sinensis})。

\subsection{灵长目(Primates)}
树栖。
拇指(趾)可与其他指(趾)相对,适合抓握。
锁骨发达。
手掌及跖裸露,有两行皮垫,利攀援。
多有趾甲。
大脑发达。
眼眶周缘有骨,两眼前视,视觉发达。
嗅觉退化。
雌性有月经。

\subsubsection{懒猴科(Lorisidae)}
头圆,吻短。
眼大而向前,眼间距窄。
耳郭半圆,朝前。
四肢粗短,等长。
大拇指与其他四指相距角度大。
第二趾有爪,其他指趾末端为肉垫和扁指甲。
如蜂猴(\textit{Nycticebus bengalensis})。

\subsubsection{卷尾猴科(Cebidae)}
尾可缠卷。
尾端下部有无毛区,司抓握。
三十六枚牙齿。
无颊囊和臀胼胝。
鼻间隔宽阔,两鼻孔距离远且向两侧开口。
如白喉卷尾猴(\textit{Cebus capachinus})。

\subsubsection{猴科(Cercopithecidae)}
吻突出,颚粗壮。
三十二枚牙齿。
鼻间隔窄,鼻孔向下,尾不可缠绕。
有颊囊和臀胼胝。
脸部有裸区。
后肢一般比前肢长。
如猕猴(\textit{Macaca mulatta})

\subsubsection{长臂猿科(Hylobatidae)}
前肢特长,站立时可及地。
手掌、手指关节长。
肩宽臀窄。
犬齿长。
无尾,无颊囊,臀胼胝小。
喉部有音囊。
如黑冠长臂猿(\textit{Hylobates concolor})。

\subsubsection{猩猩科(Pongidae)}
前肢长,过膝。
无臀胼胝,耳和脸毛少,无尾,无颊囊。
大脑发达。
如黑猩猩(\textit{Pans troglodytes})、猩猩(\textit{Pongo pygmaeus})、大猩猩(\textit{Gorilla gorilla})。

\subsubsection{人科(Hominidae)}
直立步行,前肢不过膝,体毛退化,手足分工。
犬齿和下第一前臼齿退化,牙釉质厚。
大脑发达,有语言。
如人(\textit{Homo sapiens})。

\subsection{贫齿目(Edentata)}
牙齿趋于退化,无门齿、犬齿。
可能有缺釉质的单根臼齿。
食虫,大脑几无沟回。
后足五趾。
前足仅二至三个指较为发达,生有利爪。
如大食蚁兽(\textit{Myrmecophaga tridactyla})、三趾树懒(\textit{Bradypus tridactylus})、懶犰狳(\textit{Tolypeutes matacus})。

\subsection{鳞甲目(Pholidota)}
体覆角质鳞甲,鳞间杂有稀疏硬毛。
无齿,吻尖,舌发达,前爪长。
如穿山甲(\textit{Manis pentadactyla})。

\subsection{兔形目(Lagomorpha)}
草食性。
上颌有两对前后排列的门齿,后一对较小。
门齿前后缘有珐琅质,无犬齿。
门齿与前臼齿间有空隙。
上唇有唇裂。
如草兔(\textit{Lepus capensis})、穴兔(\textit{Oryctolagus cuniculus})。

\subsection{啮齿目(Rodentia)}
上下颌各一对门齿,仅前面有釉质,凿状,终生生长。
无犬齿,门齿和前臼齿间有空隙。
嚼肌发达。

\subsubsection{松鼠科(Sciuridae)}
头骨有眶后突,颧骨发达。
上臼齿五枚,下臼齿四枚。
前后肢间无皮翼。
如松鼠(\textit{Sciurus vulgaris})、花鼠(\textit{Tamias sibiricus})、草原旱獭(\textit{Marmata bobak})。

\subsubsection{河狸科(Castoridae)}
为半水栖大型品种,脂肪层厚,绒毛致密。
四肢宽短,后肢粗壮。
后足有蹼。
尾大,上下扁平,覆角质鳞片。
眼小,耳孔小,外耳可折起。
鼻孔有防水灌入的肌肉结构。
头骨扁平,颧骨发达,骨脊高起。
唇肌肉发达。
牙二十枚。
门齿凿状,异常粗大。
臼齿宽阔。
食树枝树皮。
仅河狸(\textit{Castor fiber})和美洲河狸(\textit{Castor canadensis})两种。

\subsubsection{仓鼠科(Cricetulus)}
无前臼齿,颧骨不发达。
如黑线仓鼠(\textit{Cricetulus barabensis})。

\subsubsection{鼠科(Muridae)}
耳短厚,前翻不到眼。
后足粗大。
多有长而裸露的尾,外被鳞片。
无前臼齿,颧骨不发达。
如小家鼠(\textit{Mus musculus})、褐家鼠(\textit{Rattus morvegicus})。

\subsubsection{跳鼠科(Dipodidae)}
生于荒漠,善跳跃。
眼大耳长,吻部细长,顶间骨宽大。
上门齿平滑无沟。
前臼齿圆柱状。
下门齿齿根极长。
后肢长,第一与第五趾退化。
趾骨、跖骨有愈合的趋势。
尾长而有端部丛毛。
如三趾跳鼠(\textit{Dipus sagitta})。

\subsection{鲸目(Cetacea)}
水栖。
体内有储氧短特殊结构,呼吸频率低。
体毛退化,无皮脂腺,皮下脂肪厚。
前肢鳍状,后肢消失,颈椎愈合,尾叉状。
鼻孔位于头顶,边缘有瓣膜。
肺弹性强。
外耳退化。
胸性睾丸位于腹腔内。
雌性乳房在生殖孔两侧,为皮囊遮蔽。
须鲸无齿,上腭角质板成行下垂,形成鲸须,司滤食。
如蓝鲸(\textit{Balaenoptera musculus})。
齿鲸有圆锥形牙齿,如抹香鲸(\textit{Physeter macrocephalus})。

\subsection{食肉目(Carnivora)}
肉食性。
门齿小,犬齿大。
上颌最后一枚前臼齿和下颌第一枚臼齿的齿尖变形,相合成剪刀状,称为裂齿。
头骨上矢状嵴高起。
有骨质耳泡。
下颌关节突位置低。
有利爪,毛厚密。

\subsubsection{犬科(Canidae)}
颜面长而突出,耳尖而直立。
犬齿、裂齿发达。
后足四趾。
爪钝,不能伸缩。
如豺(\textit{Cuon alpinus})、狼(\textit{Canis lupus})、赤狐(\textit{Vulpes vulpes})。

\subsubsection{熊科(Ursidae)}
体粗壮,头圆,颜面长。
五趾,五指。
爪不能伸缩,尾短,裂齿不发达。
如黑熊(\textit{Selenarctos thibetanus})。

\subsubsection{大熊猫科(Ailuropodidae)}
似熊,吻短,食竹。
前肢有五个生爪的指,另有第六指,以便抓握。
毛色黑白相间。
仅大熊猫(\textit{Ailuropoda melanoleuca})一种。

\subsubsection{鼬科(Mustelidae)}
体细长,腿短。
头狭长,耳圆短,裂齿不发达
五趾,五指。
爪不能伸缩。
肛门附近常有臭腺。
如紫貂(\textit{Martes zibellina})、狗獾(\textit{Meles meles})、水獭(\textit{Lutra lutra})。

\subsubsection{猫科(Felidae)}
躯干匀称,尾发达。
头圆吻短,五指,四趾,爪能收缩。
裂齿、犬齿发达,臼齿退化。
如狮(\textit{Panthera leo})、虎(\textit{Panthera tigris})、豹(\textit{Panthera pardus})、猞猁(\textit{Lynx lynx})。

\subsection{鳍脚目(Pinnipedia)}
海产。
皮下脂肪厚。
体纺锤形,头圆,颈短。
外耳不发达,鼻孔、耳孔有瓣膜。
尾小口大,口周有触毛。
无裂齿。
四肢鳍状。
前肢大而无毛,后肢转向体后。
五趾,五指,蹼肥厚。
如斑海豹(\textit{Phoca largha})。

\subsection{长鼻目(Proboscidea)}
鼻和上唇延长并愈合,形成长鼻。
体毛退化,五趾,五指。
脚底有厚弹性肉垫。
上门齿发达,突出唇外。
臼齿咀嚼面多横棱。
如亚洲象(\textit{Elephas maximus})、非洲象(\textit{Loxodonta africana})。

\subsection{奇蹄目(Perissodactyla)}
除第三指趾,其余各指趾退化。
指趾端有蹄。
门齿发达,犬齿退化,臼齿有棱。
胃结构简单。
盲肠发达。

\subsubsection{马科(Equidae)}
三指,三趾,腿细长。
脊背中线有一列鬃毛。
尾毛长。
门齿凿状。
如野马(\textit{Equus przewalskii})、藏野驴(\textit{Equus kiang})。

\subsubsection{犀牛科(Rhinocerotidae)}
体粗壮,腿短,尾短,三指,三趾。
头顶有一至二个毛特化形成的角。
皮厚而裸露。
如亚洲犀牛(\textit{Rhinoceros unicornis})。

\subsection{偶蹄目(Artiodactyla)}
第三、第四指趾发达,其余指趾退化。
尾短,上门齿常退化,臼齿发达。

\subsubsection{猪科(Suidae)}
吻部延伸,在鼻孔处成盘状,内有软骨垫。、
嗅觉发达。
毛鬃状,尾细,四指,四趾。
有门齿,雄性犬齿突出成獠牙。
胃简单,杂食。
如野猪(\textit{Sus scrofa})。

\subsubsection{河马科(Hippopotamidae)}
体粗圆,吻大而圆。
眼突出于背面,耳小,体毛稀疏,皮肤紫褐色。
腿短,四指,四趾。
门齿、犬齿獠牙状。
如河马(\textit{Hippopotamus amphibius})。

\subsubsection{驼科(Camelidae)}
头小,颈长。
上唇延伸,有唇裂。
二指,二趾,指趾宽大,有厚肉垫。
体毛软而细。
胃分三室。
如双峰驼(\textit{Camelus fenrus})。

\subsubsection{鹿科(Cervidae)}
四指,四趾,中间一对较大。
腿细长。
眼窝凹陷。
常有足腺和眶下腺。
雄性多有分叉的鹿角。
上颌无门齿,臼齿有新月状脊棱。
胃四室,反刍。
无胆囊。
如梅花鹿(\textit{Cervus nippon})、马鹿(\textit{Cervus elaphus})、麋鹿(\textit{Elaphurus davidianus})。

\subsubsection{长颈鹿科(Giraffidae)}
颈长,颈背一行鬃毛。
腿长,前肢略长于后肢。
蹄阔大,尾短小。
额宽吻尖,耳大而竖立。
头顶有一对骨质短角,外包毛皮。
如长颈鹿(\textit{Giraffa camelopardalis})。

\subsubsection{牛科(Bovidae)}
雄性多一对洞角。
反刍。
如水牛(\textit{Bubalus bubalus})、山羊(\textit{Cepra hircus})。

\end{document}