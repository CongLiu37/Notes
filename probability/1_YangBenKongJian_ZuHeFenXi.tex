\documentclass[11pt]{article}

\usepackage[UTF8]{ctex} % for Chinese 

\usepackage{setspace}
\usepackage[colorlinks,linkcolor=blue,anchorcolor=red,citecolor=black]{hyperref}
\usepackage{lineno}
\usepackage{booktabs}
\usepackage{graphicx}
\usepackage{float}
\usepackage{floatrow}
\usepackage{subfigure}
\usepackage{caption}
\usepackage{subcaption}
\usepackage{geometry}
\usepackage{multirow}
\usepackage{longtable}
\usepackage{lscape}
\usepackage{booktabs}
\usepackage{natbibspacing}
\usepackage[toc,page]{appendix}
\usepackage{makecell}
\usepackage{amsfonts}
 \usepackage{amsmath}
\usepackage[utf8]{inputenc}
\usepackage{amssymb}
\usepackage{amsthm}
\usepackage{enumerate}
\usepackage{comment}

\usepackage[backend=bibtex,style=authoryear,sorting=nyt,maxnames=1]{biblatex}
% \bibliography{} % Reference bib

\title{样本空间与组合分析}
\author{}
\date{}

\linespread{1.5}
\geometry{left=2cm,right=2cm,top=2cm,bottom=2cm}

\setlength\bibitemsep{0pt}

\begin{document}
\begin{sloppypar}
  \maketitle

  \linenumbers
\section{样本空间与事件}
样本空间 \newline
不可分解的事件(样本点)之集合 \newline
\newline
事件 \newline
样本空间之子集 \newline
\newline
离散样本空间 \newline
有限或可数的样本空间 \newline

\section{离散样本空间中的概率}
离散样本空间中的概率 \newline
考虑离散样本空间$\mathfrak{E} = \{E_1,E_2,\dots\}$,对每一点$E_i$赋予一数$\Pr\{E_i\}$,称此数为$E_i$之概率并使之满足
\begin{equation}
\begin{align}
    & \forall i, \Pr\{E_i\} \ge 0 \\
    & \sum_i \Pr\{E_i\} = 1 \\
\end{align}
\end{equation}
\newline
事件$A$的概率为其中包含的样本点概率之和 \newline
\newline
事件$A_1,A_2$,有$\Pr\{A_1+A_2\} = \Pr\{A_1\} + \Pr\{A_2\} - \Pr\{A_1A_2\}$ \newline
\newline
Boole不等式 \newline
对事件$A_1,A_2,A_3,\dots$,有$\Pr\{\sum_i A_i\} \le \sum_i \Pr\{A_i\}$ \newline

\section{排列数与组合数}
有序样本 \newline
有限样本空间$\{a_1,a_2,a_3,\dots,a_n\}$中的$r$个元素的有序排列为一大小为$r$的有序样本 \newline
\newline
排列数 \newline
含$n$个样本点的总体中大小为$r$的有序样本数为$A^n_r =n(n-1)\dots(n-r+1) =\frac{n!}{(n-r)!}$ \newline
\newline
无序样本 \newline
有限样本空间$\{a_1,a_2,a_3,\dots,a_n\}$中的$r$个元素为一大小为$r$的无序样本 \newline
\newline
组合数 \newline
含$n$个样本点的总体中大小为$r$的无序样本数为$C^n_r = \frac{n!}{r!(n-r)!}$ \newline

\section{超几何分布}
$n$个元素的总体中,$n_1$个元素有某性质。现任取$r$个元素,其中有某性质的元素个数$k$服从超几何分布,有
\begin{equation}
    \Pr{k} = \frac{C^{n_1}_k C^{n-n_1}_{r-k}}{C^n_r}
\end{equation}

\section{二项式定理}
\begin{equation}
    (a+b)^n = \sum_{r=0}^n C^n_r a^rb^{n-r}
\end{equation}

\section{Stering公式}
\begin{equation}
    n! \sim \sqrt{2\pi} n^{n+\frac{1}{2}} e^{-n}, n \rightarrow \infty
\end{equation}

\end{sloppypar}
\end{document}