\documentclass[11pt]{article}

\usepackage[UTF8]{ctex} % for Chinese 

\usepackage{setspace}
\usepackage[colorlinks,linkcolor=blue,anchorcolor=red,citecolor=black]{hyperref}
\usepackage{lineno}
\usepackage{booktabs}
\usepackage{graphicx}
\usepackage{float}
\usepackage{floatrow}
\usepackage{subfigure}
\usepackage{caption}
\usepackage{subcaption}
\usepackage{geometry}
\usepackage{multirow}
\usepackage{longtable}
\usepackage{lscape}
\usepackage{booktabs}
\usepackage{natbib}
\usepackage{natbibspacing}
\usepackage[toc,page]{appendix}
\usepackage{makecell}

\title{原生动物门(Protozoa)}
\date{}

\linespread{1.5}
\geometry{left=2cm,right=2cm,top=2cm,bottom=2cm}

\begin{document}

  \maketitle

  \linenumbers
原生动物大多为单细胞动物,少数物种会出现由多个个体聚集形成的群体(clony)。
群体中的细胞无形态、功能上的分化,故不能称之为多细胞生物。
原生动物细胞质外侧透明、致密,称为外质(ectoplasm);
内侧流动性大且含颗粒物质,称为内质(endoplasm)。
原生动物一般只有一个细胞核,部分种类有多核。
一些原生动物有多倍体的大核(macronucleus),负责代谢;
和二倍体的小核(micronucleus),负责繁殖。
其运动依靠伪足(pseudopodium)爬行或通过鞭毛(flagellum)、纤毛(cilium)游动。

\newline

原生动物包含生物界全部营养方式,包括利用无机物合成有机物的自养性营养(holophytic nutrition)、通过体表渗透作用摄取环境中的有机物的腐生性营养(saprophytic nutrition)和通过非跨膜方式摄取食物并形成食物泡(food vacuole),而后再进一步消化并以非跨膜方式排出残渣的动物性营养(holozoic nutrition)。

\newline

原生动物的呼吸和部分代谢废物的排泄是通过体表渗透作用进行的。
此外,原生动物亦可通过伸缩泡(contractile vacuole)完成排泄以及胞内水平衡的维持。
伸缩泡为胞内膜状结构,有开口通向胞外。
伸缩泡变大时,胞内水分和代谢废物进入伸缩泡;
伸缩泡收缩时,其中的物质被排到胞外。

\newline

原生动物的生殖有多种形式。
其中无性生殖(asexual reproduction)包括二分裂(binary fission)、出芽(budding)、复分裂(multiple fission)和质裂(plasmotomy)。
二分裂和出芽本质上为有丝分裂(mitosis),但前者形成的子细胞大小相近,而后者形成的子细胞大小差异明显。
复分裂即细胞核先进行分裂,而后细胞质再分裂,形成单核子细胞。
此外,原生动物亦营有性生殖(sexual reproduction),包括配子生殖(gamogenesis)和接合生殖(conjugation)。
配子生殖即亲本减数分裂(maiosis)形成的配子彼此结合,形成子代。
接合生殖则为纤毛虫所特有。
纤毛虫有大核和小核。
亲本细胞相贴时,大核解体,小核进行减数分裂,形成的四个子细胞中三个解体,一个再进行二分裂。
亲本细胞交换小核后分离。
每个细胞中的两个小核融合并二分裂三次,形成八个细胞核。
这八个细胞核中,四个变为大核,三个解体,一个和细胞质一同分裂两次。
如此,每个亲代细胞产生四个子细胞。

\newline

大多数自由生活的原生动物可形成具有保护作用的包囊(cyst),将自身包裹起来。
部分营寄生生活的原生动物,其合子亦会分泌囊壁,形成起保护作用的卵囊(oocyst)。
虫体可在卵囊中分裂繁殖。

\newline

原生动物和其它物种的关系包括共栖(commensalism)、共生(symbiosis)和寄生(parasitism)。
共栖关系中一方受益,一方无益无害;
共生关系中双方受益;
寄生关系中一方受益,一方受害。

\section{鞭毛纲(Mastigophora)}
虫体有鞭毛。
鞭毛有运动、捕食、附着、感觉等功能。
细胞膜表面有纹路,细胞内部有感光的眼点(eye spot)、储蓄泡和伸缩泡。
鞭毛虫的营养方式有自养性、腐生性和动物性。
部分种类营混合性营养,即在有光条件下可进行光合作用,无光时营腐生性营养。
生殖方式则主要为二分裂和配子生殖。

\subsection{植鞭亚纲(Phytomastigina)}
多有色素体,能进行光合作用,水生,为浮游生物的重要组成。

\subsubsection{金滴虫目(Chrysomonadina)}
一至二鞭毛,无胞咽,无淀粉体。
色素体一至二个,黄色或褐色。
胞外有胶质囊。
如钟罩虫(\textit{Dinobryon})。

\subsubsection{隐滴虫目(Cryptomonadina)}
两鞭毛,有胞咽,有储蓄泡,有淀粉体。
两个色素体,黄色或褐色。
亦有无色素体品种。
如唇滴虫(\textit{Chilomonas})。

\subsubsection{植滴虫目(Phytomonadina)}
鞭毛二/四/八根,多小端绿色色素体。
无胞咽,有淀粉体。
群体现象普遍。
如盘藻(\textit{Gonium})、团藻(\textit{Volvox})。

\subsubsection{眼虫目(Euglenoidina)}
体长圆形,一至二鞭毛,有胞咽,有淀粉体。
部分品种无色素体。
常有眼点,表面纹陆明显。
如扁眼虫(\textit{Phacus})。

\subsubsection{腰鞭目(Dinoflagellata)}
两根鞭毛,一根围绕在体中部横沟内,使身体旋转;另一根拖曳于后部纵沟,使身体前进。
色素体黄色或褐色。
体表多有甲板。
如沟腰鞭虫(\textit{Gonyaulax})、裸甲腰鞭虫(\textit{Gymnodinium}),大量繁殖造成赤潮。

\subsection{动鞭亚纲(Zoomastigina)}
无色素体,多寄生种类。

\subsubsection{领鞭毛目(Choanoflagellina)}
一根鞭毛,鞭毛基部有透明原生质领。
如静钟虫(\textit{Codosiga})。

\subsubsection{根鞭目(Rhizomastgina)}
虫体类似变形虫,有鞭毛和伪足。
如变形鞭毛虫(\textit{Mastigamoeba})。

\subsubsection{动体目(Kinetoplastina)}
细胞内有可自我复制的动体,内含DNA,位于延伸的线粒体内。
体一侧有波动膜。
大多寄生。
如造成人犬黑热病的利什曼原虫(\textit{Leishmania})、多生活于脊椎动物血液的锥虫(\textit{Trypanosoma})。

\subsubsection{曲滴虫目(Retoramonadina)}
有一根鞭毛与腹面胞口相连。
寄生于昆虫或脊椎动物肠道。
如唇鞭毛虫(\textit{Chilomastix})。

\subsubsection{双滴虫目(Trichomonadina)}
有一根鞭毛向后并与体表相连,形成波动膜。
多寄生于昆虫或脊椎动物消化道。
如毛滴虫(\textit{Trichomonas})。

\subsubsection{超鞭毛目(Hypermastigina)}
鞭毛极多。
多共生于昆虫肠道。
如披发虫(\textit{Trichonympha})。


\section{肉足纲(Sarcodina)}
肉足虫多为自由生活,细胞内质、外质分别明显,通过伪足运动、摄食,营动物性营养。
淡水物种有伸缩泡而海水物种没有。
肉足虫虫体多裸露,但亦有很多物种有石灰质、几丁质或硅质的外壳。
其繁殖大多为二分裂,但部分物种具有有性生殖且有世代交替现象。

\subsection{根足亚纲(Rhizopoda)}
伪足无轴丝,呈叶状、指状、丝状或根状。

\subsubsection{变形目(Amoebida)}
细胞裸露。
如大变形虫(\textit{Amoeba proteus})、溶组织阿米巴(\textit{Entamoeba histolytica})。

\subsubsection{有壳虫目(Testacea)}
有外壳。
壳上有单一壳孔,供伪足伸出。
如表壳虫(\textit{Arcella})、砂壳虫(\textit{Difflugia})。

\subsubsection{有孔虫目(Foraminiferida)}
原生质分泌由小室组成的外壳。
生活史复杂,有世代交替。
如球房虫(\textit{Globigerina})。
有孔虫自寒武纪至今皆有分布且数量庞大,不同地质年代的地层常有不同的有孔虫化石,故可用于确定地层年代。

\subsection{辐足亚纲(Actinopoda)}
体多呈球形。
伪足针状,有轴丝。
多水生,浮游生活。

\subsubsection{太阳虫目(Heliozoa)}
球形,无外壳。
伪足放射状,轴丝坚硬有毒。
有骨针。
如太阳虫(\textit{Actinophrys})。

\subsubsection{放射虫目(Radiolarida)}
细胞质分内外两层,中间由骨质中央囊隔开。
中央囊上有小孔,囊外有空泡和共生黄藻。
有硅质外壳,壳面有花纹。
如环骨虫(\textit{Lithocircus})。

\section{孢子纲(Sporozoa)}
孢子纲物种全部营寄生生活,无运动器或仅在生活史的特定阶段以鞭毛、伪足运动。
孢子纲物种有顶复合器,一般认为这与侵入寄主细胞有关。

\newline

孢子虫生活史复杂,且普遍存在世代交替。
一般来说,孢子虫生活史包括三个阶段:
(1)裂体生殖(schizgony)期,进行复分裂;
(2)配子生殖(gametogony)期,包括配子的形成和结合为合子的阶段;
(3)孢子生殖(sporogony)期,合子分裂形成子孢子。
子孢子包裹在孢子囊中,孢子囊又包裹在卵囊中。
此一阶段一般为孢子虫更换宿主的时期。

\subsection{晚孢子亚纲(Telesporia)}
顶复合器发达,有类锥体,能有性生殖,有卵囊和孢子,多胞内寄生。

\subsubsection{簇虫目(Gregarinida)}
成熟滋养体大,胞外寄生。
如簇虫(\textit{Gregarina})。

\subsubsection{球虫目(Coccidia)}
成熟滋养体小,胞内寄生。
如疟原虫(\textit{Plasmodium})。

\subsection{焦虫亚纲(Piroplasmia)}
顶复合器不发达,无类锥体,无卵囊,无孢子,胞内寄生,仅无性生殖。

\subsubsection{焦虫目(Piroplasmida)}
寄生马牛羊端红细胞、白细胞、肝细胞。
通过蜱虫传播。
如巴贝斯虫(\textit{Babesia})。

\section{纤毛纲(Ciliata)}
纤毛虫多营自由生活,虫体表面有纤毛,负责运动和摄食。
纤毛可分散分布,或彼此粘合为小膜(membranella),或由单排纤毛粘合为波动膜(undulating membrane),或成簇粘合为棘毛(cirrus)。
部分纤毛虫质膜下有与质膜垂直排列的杆状结构,即刺丝泡(trichocyst)。
其开口位于质膜,遇刺激时射出内容物,起防御作用。
纤毛虫多营动物性营养,食物的吞入和残渣的排除均通过细胞上固定的位置,分布称为胞口和胞肛。
代谢废物的排除则是通过伸缩泡。
纤毛虫的细胞核有大核和小核两种,营二分裂或接合生殖。

\subsection{前口目(Prostomatida)}
纤毛分布均匀无特化,胞口位置靠前。
如前管虫(\textit{Prorodon})。

\subsubsection{侧口目(Pleurostomatida)}
纤毛分布均匀无特化。
胞口裂缝状,位于侧腹面。
如漫游虫(\textit{Litonotus})。

\subsubsection{毛口目(Trichostomatida)}
胞口位置靠前,口区前庭明显。
口区纤毛排列紧密无特化。
寄生,多寄生脊椎动物消化道。
如肠带虫(\textit{Balantidium})。

\subsubsection{肾形目(Colpodida)}
胞口位置靠前,口区前庭明显,口区纤毛特化为膜状结构。
如肾形虫(\textit{Colpoda})。

\subsubsection{篮口目(Nassulida)}
虫体扁平或筒状,胞口腹面。
纤毛退化或局部集中。
有围口系统,胞咽为刺杆。
如篮口虫(\textit{Nassula})。

\subsubsection{管口目(Cyrtophotida)}
扁平,背部微拱无纤毛。
胞口位于腹面前半部,胞咽为刺杆。
多外寄生。
如斜管虫(\textit{Chilodonella})。

\subsubsection{漏斗毛目(Chonotrichida)}
花瓶形,胞口向外延伸成螺旋口围。
体表除口围外,其余部分无纤毛。
固着生活,多海产,出芽生殖。
如旋漏斗虫(\textit{Spirochona})。

\subsubsection{吸管目(Suctorida)}
成体固着无纤毛,以吸管捕食。
幼体有纤毛。
如足吸管虫(\textit{Podophrya})。

\subsubsection{膜口目(Hymenostomatida)}
纤毛密布全身,胞口有纤毛膜,浮游。
如草履虫(\textit{Paramecium caudatum})。

\subsubsection{盾纤目(Scuticociliatida)}
纤毛分布均匀,有一长尾纤毛,三片口膜。
如康纤虫(\textit{Cohnilenbus})。

\subsubsection{无口目(Astomatida)}
纤毛遍布,无胞口,有用于附着的棘刺,内寄生环节动物。
如射眉虫(\textit{Anoplophyra})。

\subsubsection{缘毛目(Peritrichida)}
倒钟形,无体纤毛。
顶部有可伸缩端口围盘和口围。
多固着集群。
如钟虫(\textit{Vorticella})。

\subsubsection{异毛目(Heterotrichida)}
体大可收缩,纤毛无特化。
围口膜发达。
如喇叭虫(\textit{Stentor})。

\subsubsection{齿口目(Odontostomatida)}
左右侧扁,纤毛少,围口膜退化,体表硬化,体后常有棘突。
如朽纤虫(\textit{Saprodinium})。

\subsubsection{寡毛目(Oligotrichida)}
虫体圆形或椭圆,围口膜位于顶端,体纤毛退化或特化。
如弹跳虫(\textit{Halteria})。

\subsubsection{内毛目(Entodiniomorphida)}
围口膜发达,体表硬化生棘。
内共生于草食性哺乳动物消化道。
如内毛虫(\textit{Entodinium})。

\subsubsection{膜毛目(Hypotrichida)}
背腹扁平,背部略隆起。
腹部纤毛特化,司爬行或支持。
背部纤毛特化为触毛。
围口膜发达。
如棘尾虫(\textit{Stylonychia})、游仆虫(\textit{Euplotes})。

\end{document}