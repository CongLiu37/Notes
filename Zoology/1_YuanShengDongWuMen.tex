\documentclass[11pt]{article}

\usepackage[UTF8]{ctex} % for Chinese 

\usepackage{setspace}
\usepackage[colorlinks,linkcolor=blue,anchorcolor=red,citecolor=black]{hyperref}
\usepackage{lineno}
\usepackage{booktabs}
\usepackage{graphicx}
\usepackage{float}
\usepackage{floatrow}
\usepackage{subfigure}
\usepackage{caption}
\usepackage{subcaption}
\usepackage{geometry}
\usepackage{multirow}
\usepackage{longtable}
\usepackage{lscape}
\usepackage{booktabs}
\usepackage{natbib}
\usepackage{natbibspacing}
\usepackage[toc,page]{appendix}
\usepackage{makecell}

\title{原生动物门(Protozoa)}
\date{}

\linespread{1.5}
\geometry{left=2cm,right=2cm,top=2cm,bottom=2cm}

\begin{document}

  \maketitle

  \linenumbers

\section{一般特征}
原生动物大多为单细胞动物,少数物种会出现由多个个体聚集形成的群体(clony)。
群体中的细胞无形态、功能上的分化,故不能称之为多细胞生物。
原生动物细胞质外侧透明、致密,称为外质(ectoplasm);
内侧流动性大且含颗粒物质,称为内质(endoplasm)。
原生动物一般只有一个细胞核,部分种类有多核。
一些原生动物有多倍体的大核(macronucleus),负责代谢;
和二倍体的小核(micronucleus),负责繁殖。
其运动依靠伪足(pseudopodium)爬行或通过鞭毛(flagellum)、纤毛(cilium)游动。

\newline

原生动物包含生物界全部营养方式,包括利用无机物合成有机物的自养性营养(holophytic nutrition)、通过体表渗透作用摄取环境中的有机物的腐生性营养(saprophytic nutrition)和通过非跨膜方式摄取食物并形成食物泡(food vacuole),而后再进一步消化并以非跨膜方式排出残渣的动物性营养(holozoic nutrition)。

\newline

原生动物的呼吸和部分代谢废物的排泄是通过体表渗透作用进行的。
此外,原生动物亦可通过伸缩泡(contractile vacuole)完成排泄以及胞内水平衡的维持。
伸缩泡为胞内膜状结构,有开口通向胞外。
伸缩泡变大时,胞内水分和代谢废物进入伸缩泡;
伸缩泡收缩时,其中的物质被排到胞外。

\newline

原生动物的生殖有多种形式。
其中无性生殖(asexual reproduction)包括二分裂(binary fission)、出芽(budding)、复分裂(multiple fission)和质裂(plasmotomy)。
二分裂和出芽本质上为有丝分裂(mitosis),但前者形成的子细胞大小相近,而后者形成的子细胞大小差异明显。
复分裂即细胞核先进行分裂,而后细胞质再分裂,形成单核子细胞。
此外,原生动物亦营有性生殖(sexual reproduction),包括配子生殖(gamogenesis)和接合生殖(conjugation)。
配子生殖即亲本减数分裂(maiosis)形成的配子彼此结合,形成子代。
接合生殖则为纤毛虫所特有。
纤毛虫有大核和小核。
亲本细胞相贴时,大核解体,小核进行减数分裂,形成的四个子细胞中三个解体,一个再进行二分裂。
亲本细胞交换小核后分离。
每个细胞中的两个小核融合并二分裂三次,形成八个细胞核。
这八个细胞核中,四个变为大核,三个解体,一个和细胞质一同分裂两次。
如此,每个亲代细胞产生四个子细胞。

\newline

大多数自由生活的原生动物可形成具有保护作用的包囊(cyst),将自身包裹起来。
部分营寄生生活的原生动物,其合子亦会分泌囊壁,形成起保护作用的卵囊(oocyst)。
虫体可在卵囊中分裂繁殖。

\newline

原生动物和其它物种的关系包括共栖(commensalism)、共生(symbiosis)和寄生(parasitism)。
共栖关系中一方受益,一方无益无害;
共生关系中双方受益;
寄生关系中一方受益,一方受害。

\section{原生动物的分类}
\subsection{鞭毛纲(Mastigophora)}
虫体有鞭毛。
鞭毛有运动、捕食、附着、感觉等功能。
细胞膜表面有纹路,细胞内部有感光的眼点(eye spot)、储蓄泡和伸缩泡。
鞭毛虫的营养方式有自养性、腐生性和动物性。
部分种类营混合性营养,即在有光条件下可进行光合作用,无光时营腐生性营养。
生殖方式则主要为二分裂和配子生殖。

\section{肉足纲(Sarcodina)}
肉足虫多为自由生活,细胞内质、外质分别明显,通过伪足运动、摄食,营动物性营养。
淡水物种有伸缩泡而海水物种没有。
肉足虫虫体多裸露,但亦有很多物种有石灰质、几丁质或硅质的外壳。
其繁殖大多为二分裂,但部分物种具有有性生殖且有世代交替现象。

\section{孢子纲(Sporozoa)}
孢子纲物种全部营寄生生活,无运动器或仅在生活史的特定阶段以鞭毛、伪足运动。
孢子纲物种有顶复合器,一般认为这与侵入寄主细胞有关。

\newline

孢子虫生活史复杂,且普遍存在世代交替。
一般来说,孢子虫生活史包括三个阶段:
(1)裂体生殖(schizgony)期,进行复分裂;
(2)配子生殖(gametogony)期,包括配子的形成和结合为合子的阶段;
(3)孢子生殖(sporogony)期,合子分裂形成子孢子。
子孢子包裹在孢子囊中,孢子囊又包裹在卵囊中。
此一阶段一般为孢子虫更换宿主的时期。

\section{纤毛纲(Ciliata)}
纤毛虫多营自由生活,虫体表面有纤毛,负责运动和摄食。
纤毛可分散分布,或彼此粘合为小膜(membranella),或由单排纤毛粘合为波动膜(undulating membrane),或成簇粘合为棘毛(cirrus)。
部分纤毛虫质膜下有与质膜垂直排列的杆状结构,即刺丝泡(trichocyst)。
其开口位于质膜,遇刺激时射出内容物,起防御作用。
纤毛虫多营动物性营养,食物的吞入和残渣的排除均通过细胞上固定的位置,分布称为胞口和胞肛。
代谢废物的排除则是通过伸缩泡。
纤毛虫的细胞核有大核和小核两种,营二分裂或接合生殖。

\end{document}