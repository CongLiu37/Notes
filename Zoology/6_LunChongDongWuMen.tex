\documentclass[11pt]{article}

\usepackage[UTF8]{ctex} % for Chinese 

\usepackage{setspace}
\usepackage[colorlinks,linkcolor=blue,anchorcolor=red,citecolor=black]{hyperref}
\usepackage{lineno}
\usepackage{booktabs}
\usepackage{graphicx}
\usepackage{float}
\usepackage{floatrow}
\usepackage{subfigure}
\usepackage{caption}
\usepackage{subcaption}
\usepackage{geometry}
\usepackage{multirow}
\usepackage{longtable}
\usepackage{lscape}
\usepackage{booktabs}
\usepackage{natbib}
\usepackage{natbibspacing}
\usepackage[toc,page]{appendix}
\usepackage{makecell}

\title{轮虫动物门(Rotifera)}
\date{}

\linespread{1.5}
\geometry{left=2cm,right=2cm,top=2cm,bottom=2cm}

\begin{document}

  \maketitle

  \linenumbers
轮虫大小与原生动物类似,虫体纵长,无色透明。
头部有纤毛组成的头冠(corona),司游泳和摄食。
虫体被角质膜,常在躯干增厚,形成兜甲(lorica),其上多生刺或棘。
尾部呈长筒状,内有足腺,可借其分泌物黏附于其它物体。
轮虫体壁与消化道之间为假体腔,其内有游离的变形细胞,司噬菌。
各个器官、组织细胞互相融合,形成合胞体,但各部分的细胞核数目恒定。

\newline

轮虫消化道分口、咽、胃、肠、肛门等部分。
口位于头部腹面。
咽膨大且肌肉发达,内有咀嚼器(mastax)。
咽侧有唾液腺,咽后经食管通入胃。
胃前有胃腺,其开口通入胃,可分泌消化酶。
胃内壁有纤毛。
胃经肠,通于泄殖腔,泄殖腔开口于躯干和尾部交界处,是为泄殖孔。

\newline

排泄器官为位于虫体两侧的原肾管。
原肾管盲端有鞭毛,称为焰球(flame bulb)。
焰球经排泄管,通入膀胱,后与肠汇合,通入泄殖腔。
轮虫无呼吸器官,通过体壁扩散交换气体。

\newline

轮虫咽背侧有脑神经节。
脑神经节向后伸出两条腹神经索。
感觉器官位于头部,包括眼点和触手。
触手呈短棒状,司触觉。

\newline

轮虫雌雄异体。
雄性寿命短,体内仅有精巢、输精管和阴茎,其余器官退化。
部分种类未发现雄性个体。
环境适宜时,轮虫营孤雌生殖。
环境恶化时,轮虫孤雌生殖产生混交雌体(mictic female)。
混交雌体产生单倍体的卵,与雄性交配后产生合子,否则单倍体的卵发育为雄性。
交配时,雄性阴茎刺破雌性体壁,将精子输入假体腔。
合子分泌卵壳,形成休眠卵(resting egg)。
环境条件改善时,休眠卵发育为非混交雌性。

\section{轮虫动物的分类}
\subsection{单巢纲(Monogononta)}
雌虫有单个卵巢。

\subsection{双巢纲(Digononta)}
雌虫有两个卵巢。

\end{document}