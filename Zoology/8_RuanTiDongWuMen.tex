\documentclass[11pt]{article}

\usepackage[UTF8]{ctex} % for Chinese 

\usepackage{setspace}
\usepackage[colorlinks,linkcolor=blue,anchorcolor=red,citecolor=black]{hyperref}
\usepackage{lineno}
\usepackage{booktabs}
\usepackage{graphicx}
\usepackage{float}
\usepackage{floatrow}
\usepackage{subfigure}
\usepackage{caption}
\usepackage{subcaption}
\usepackage{geometry}
\usepackage{multirow}
\usepackage{longtable}
\usepackage{lscape}
\usepackage{booktabs}
\usepackage{natbib}
\usepackage{natbibspacing}
\usepackage[toc,page]{appendix}
\usepackage{makecell}

\title{软体动物门(Mollusca)}
\date{}

\linespread{1.5}
\geometry{left=2cm,right=2cm,top=2cm,bottom=2cm}

\begin{document}

  \maketitle

  \linenumbers
 
\section{一般特征}
软体动物各类群形态差异较大,但都有头、足和内脏团(visceral mass)。
头位于躯体前端,足位于腹侧,内脏团位于足的背侧。
体背侧皮肤褶形成外套膜(mantle),常包裹内脏团。
外套膜和内脏团之间的空腔称为外套腔(mantle cavity),联通体外,上常有鳃、足、肾孔、生殖孔、肛门等。
外套腔壁处的上皮有纤毛,促进水流在外套腔的循环。
外套膜外层上皮的分泌物,能形成贝壳。
左右两片外套膜后缘处常有一或两处愈合,形成出水孔(exhalant siphon)和入水孔(inhalant siphon)。

\newline

贝壳(shell)是软体动物的重要特征,起保护和维持体型的作用。
其主要成分是碳酸钙和贝壳素(conchiolin)。
贝壳最外一层为角质层(periostracum),薄且透明,有光泽,主要成分为贝壳素,不受酸碱侵蚀。
中间一层为壳层(ostracum),主要成分为碳酸钙。
最内为壳底(hypostracum),即珍珠质层(pearl layer),有光泽。
角质层和壳层的生长受环境影响,并非连续不断的,由此形成贝壳表面的生长线。

\newline

软体动物的消化道完整,消化腺发达。
多数种类口腔底部有颚片(mandible)和齿舌(radula)。
颚片可辅助捕食。
齿舌表面有横列的角质齿,呈锉刀状。
摄食时,齿舌前后伸缩,刮取食物。

\newline

软体动物体腔退化,仅残留围心腔(pericardinal cavity)、生殖腺和排泄器官内腔等。
假体腔则见于各种组织间隙,形成血窦。
心脏位于内脏团背侧围心腔内,由一个能搏动的心室和数对心耳组成。
心室和心耳之间有瓣膜,防止血液逆流。
血管分化为动脉和静脉。
血液经心脏流入动脉,而后进入血窦,再经静脉流回心脏。
故软体动物的循环系统为开管式循环(open circulation),在循环过程中血液进入组织间隙。
开管式循环的效率不如闭管式。
软体动物的开管式循环,与其大部分种类低下的运动能力相适应。

\newline

软体动物中,水生种类以鳃呼吸。
鳃为外套腔内面皮肤伸展形成的,位于外套腔内。
陆生种类无鳃,外套腔内部分区域的微细血管集中分布,形成肺,司气体交换。
其排泄器官为肾管,分为腺质部分和管状部分。
腺质部分富血管,开口于围心腔,肾口部分有纤毛;
管状部分内壁有纤毛,肾孔开口于外套腔。

\newline

软体动物的神经系统变化较大。
原始种类仅有围咽神经环和向体后伸出的一对足神经索(pedal cord)和一对侧神经索(pleural cord)。
较高等的种类主要有四对神经节,彼此以神经相连。
脑神经节(cerebral ganglion)位于食管背侧,向前发出神经;
足神经节(pedal ganglion)位于足的前端,向足部发出神经;
侧神经节(pleural ganglion)向鳃和外套膜发出神经;
脏神经节(visceral ganglion)向内脏发出神经。
软体动物的皮肤、外套膜内层和触角均可司感觉,司感光的眼结构繁简不一;
另有嗅检器(osphradium)和平衡囊等感觉器官。

\newline

软体动物大多雌雄异体,一般为间接发育,有担轮幼虫期。
  
\section{软体动物的分类}
\subsection{无板纲(Aplacophora)}
呈蠕虫状,体表背具有石灰质的角质外皮,无贝壳,口位于体前腹侧,腹侧中央有一腹沟,沟内或有具纤毛的足。
体后有泄殖腔,腔内一般有一对鳃。
无感觉器官,血管系统退化。
  
\section{单板纲(Monoplacophora)}
有一笠形贝壳,壳顶在中央靠前处。
壳表有围绕壳顶的环状生长线。
足发达,五或六对鳃环列于足的周围。
头不明显,齿舌发达。
六对肾,一对开口于体前,其余五对开口于鳃的基部。
雌雄异体,两对生殖腺在围心腔前。

\subsection{多板纲(Polyplacophora)}
呈椭圆形。
背侧有八块贝壳,呈覆瓦状排列。
体前背侧第一块贝壳呈半月形,称为头板(cephalic plate);
体后最后一块呈元宝状,称为尾板(tail plate);
中间六块为中间板(intermediate plate)。
各板可移动,故多板纲动物可卷曲起来。
贝壳下方有一圈外套膜,上丛生针状或棘状突起。
头不发达,位于体前腹侧,有一向下的短吻,吻中央为口。
足宽大,吸附力腔。
足和外套膜之间有一圈狭窄的外套腔,腔内两侧生鳃。

\newline
  
口腔内有齿舌,前有一对唾液腺。
食管后有一对食道腺,胃周围为肝。
真体腔发达。
排泄器官为一对后肾管,肾口开于围心腔,肾孔位于外套腔。
神经系统包括环食管的神经环和向后伸出的侧神经索、足神经索。
侧神经索伸至外套膜和内脏,足神经索伸至足。
神经索之间有神经相连,呈梯状。
雌雄异体,生殖导管开口于外套腔。
  
\section{腹足纲(Gastropoda)}
头发达,有眼和触角。
足发达,呈叶状,位于腹侧,上有单细胞黏液腺。
体外一般有一个螺旋形贝壳。
壳一般为右旋。
壳分为两部分,在上者为包含内脏器官的螺旋部(spire)和容纳头、足的体螺层(body whorl)。
体螺层开口称壳口(aperture)。
壳口常有由足后端分泌的盖,称为厣。

\newline

口腔有颚片和齿舌,唾液腺分泌无消化功能的黏液。
肝脏发达。水生种类以鳃呼吸,陆生种类以肺呼吸。
肾呈长形,两端开口于围心腔和外套腔。
神经系统包括脑、足、侧、脏四对神经节,感觉器官有眼、触角、嗅检器、味蕾、平衡囊等。

\newline

海产种类多雌雄异体,陆生种类多雌雄同体。
异体受精,有交配行为。
生殖腺位于内脏团背面,生殖管开口于体前右侧,即为生殖孔。
雌雄同体的种类,生殖腺后为两性管和输精卵管。
输精卵管后端分为通向交接器的输精管和通向阴道的输卵管。
交接器和阴道均由生殖孔通向体外。
  
\section{瓣鳃纲(Lamellibranchia)}
身体侧扁,体侧有两片贝壳,两片外套膜分别位于贝壳内面。
头部退化,足呈斧状,鳃呈瓣状。
贝壳背面有突出的壳顶(umbo),壳顶前后一般分别有小月面和楯面。
壳边缘较厚,有互相咬合的齿和齿槽,构成铰合部(hinge)。
铰合部连接两壳的背缘有角质韧带(ligament),可连接两片贝壳。

\newline

外套膜薄而透明,边缘较厚,常有触手。
外套膜上有两个连接左右两侧的闭壳肌分别位于体前和体后。
鳃位于外套腔中,从体前向后延伸至肛门,司呼吸和滤食。
足位于腹面,两侧扁平,前端呈斧状。
口位于体前,具纤毛,两侧各有一对三角形唇瓣,无齿舌和口腔腺。
胃壁厚,位于内脏团。肠细长,直肠穿过围心腔,开口于体后。
肾管一对,开口于围心腔和外套腔。
神经系统有脑、足、脏神经节各一对。
一般雌雄异体,生殖管开口于肾管内或肾孔附近。
体外受精。
  
\section{头足纲(Cephalopoda)}
头位于体前,其顶端为口,口周围有口膜。
头两侧有发达的眼,眼后有椭圆形小窝,为嗅觉陷。
足环列于头前口周,形成数十只、十只或八只腕。
亦有一部分足形成位于头和躯干之间的腹面的漏斗,其开口朝向体前。
漏斗腔和外套腔相连。
仅少数种类有外壳,多数种类外壳包埋于外套膜,形成内壳。
部分物种内壳退化。
内壳可支撑身体,有利于保持平衡。
头足类软骨发达,主要包括包围中枢神经系统和平衡囊的头软骨、颈软骨和腕软骨。
大部分种类皮下有扁平状、富有弹性的色素细胞(chromatophore),周围有肌纤维。
肌纤维的收缩控制色素细胞的舒张,改变皮肤颜色。

\newline

头足类口腔内有颚片和齿舌,肝脏发达。
部分种类直肠末端有梨形盲囊,称为墨囊(ink sac)。
囊内腺体分泌墨汁,经位于外套腔的肛门排出。
鳃呈羽状,位于外套腔内。
循环系统近似闭管式。
心脏位于体后腹部中央的围心腔内。
由心脏向前、向后各伸出一大动脉。

\newline

头足类神经系统和感觉器官发达,有眼、嗅觉陷、平衡囊等。
眼的结构复杂。
瞳孔(pupil)周围为虹膜。
瞳孔外侧覆透明的角膜(cornea),瞳孔后为晶状体。
晶状体两侧有与虹膜平行分布的睫状肌。
眼的内侧为视网膜,视网膜内层为感光细胞,外层神经纤维于眼后端汇合为视神经。

\newline

头足类雌雄异体,有求偶和交配行为。
  
\section{掘足纲(Scaphopoda)}
全部海产,贝壳呈象牙形,粗的一端为前端,上有较大的头足孔;
细的一端为后端,上有较小的肛门孔。
壳凸出的一面为背侧,凹的一面为腹侧。
外套膜呈管状,前后有开口。
头不明显,前端有不能伸缩的吻,吻基部有可伸缩的头丝(captacula),司触觉和摄食。
吻内为口球,内有颚片和齿舌。
足在吻基部之后,柱状,可伸长以挖掘泥沙。
肛门位于足基部腹侧,开口于外套腔。
以外套膜交换气体。
无血管,仅有血窦。
肾一对,位于胃的侧面。
雌雄异体。
  
\end{document}