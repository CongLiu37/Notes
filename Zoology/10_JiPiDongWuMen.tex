\documentclass[11pt]{article}

\usepackage[UTF8]{ctex} % for Chinese 

\usepackage{setspace}
\usepackage[colorlinks,linkcolor=blue,anchorcolor=red,citecolor=black]{hyperref}
\usepackage{lineno}
\usepackage{booktabs}
\usepackage{graphicx}
\usepackage{float}
\usepackage{floatrow}
\usepackage{subfigure}
\usepackage{caption}
\usepackage{subcaption}
\usepackage{geometry}
\usepackage{multirow}
\usepackage{longtable}
\usepackage{lscape}
\usepackage{booktabs}
\usepackage{natbib}
\usepackage{natbibspacing}
\usepackage[toc,page]{appendix}
\usepackage{makecell}

\title{棘皮动物门(Echinodermata)}
\date{}

\linespread{1.5}
\geometry{left=2cm,right=2cm,top=2cm,bottom=2cm}

\begin{document}

  \maketitle

  \linenumbers
\section{一般特征}
棘皮动物出现后口。
胚胎发育过程中,原肠胚后端的原口封闭,胚胎前端外胚层内陷,与原肠相连,形成成体的口;
原口则形成成体的肛门。
具有后口的地位称为后口动物(deuterostome),与由原口形成成体的口的原口动物(protostomia)相对应。

\newline

棘皮动物胚胎和幼虫体型两侧对称,成体则为次生性辐射对称。
成体的口面为幼虫左侧,反口面为幼虫右侧。
成体辐射对称的体型与固着生活相适应,运动能力有限。

\newline

棘皮动物成体包括中央盘和腕两部分。
中央盘的中间的一面有口,称为口面;
肛门多在于口的反面,称为反口面。
筛板位于肛门附近,上有小孔。
腕围绕中央盘辐射排列,其腹面有纵沟,沟内有二至四行管足,司运动。
体表有薄膜状的颗粒突起,其内腔与体腔相同,称为皮鳃,司呼吸、排泄。

\newline

棘皮动物成体体型差异较大,可分为:
(1)海星型,呈多角星型,扁平,背面稍拱起,体表有颗粒状突起;
(2)海胆型,呈半球形、卵圆形或盘形,体表骨板愈合形成外壳,其上有孔和棘刺;
(3)海参型,呈长圆筒形,体表有突起,体前口周围有触手;
(4)海百合型,呈树枝状,腕呈羽状。

\newline

棘皮动物体表上皮为一层柱状上皮细胞,以及其外的角质膜、其间散布的腺细胞和神经感觉细胞和其基部基膜下的神经层。
上皮下为真皮,包括结缔组织和肌肉层。
结缔组织分泌骨片,组合形成发达的网状内骨骼,常突出体表形成棘。
肌肉层外环内纵。
反口面沿腕背中线辐射伸出发达的肌肉束。
体壁最内为体腔上皮,具纤毛。

\newline

棘皮动物真体腔发达,围绕消化道和生殖腺,延申至腕顶端。
真体腔的一部分形成水管系统(water vascular system)。
水管系统包括环水管(ring canal)、辐水管(radial canal)、支水管(lateral canal)、管足(tube foot)、石管(stone canal)、筛板。
环水管位于中央盘,呈环状,其内层生帖德曼式体(Tiedmann’s body),是一种小型腺体组织。
环水管外层辐射伸出辐水管,辐水管两侧伸出彼此平行的侧水管。
侧水管连通于管足并与之垂直。
管足司运动,其上部为盲囊状的罍,下部呈管状。
罍收缩时,水进入管足下部,使之伸长,反之则收缩。
环水管向反口面伸出石管,其管壁内有石灰质环。
石管末端为筛板,其上有孔,与外界连通。筛板位于肛门附近。

\newline

此外,棘皮动物真体腔还形成围血系统(perihaemal system),与水管系统走向相同。
围血系统包括口面和反口面的环围血窦,连接二者的轴窦以及由二者辐射伸出的辐围血窦。
循环系统被围血系统包裹,包括口面和反口面的环血窦、轴器和辐血窦。
围血系统和循环系统以海参和海胆的发达。

\newline

棘皮动物的消化系统自口面向反口面延申。
口位于中央盘正中,周围有括约肌和辐射肌纤维,经食道进入充满中央盘的胃。
胃分为进口的贲门胃和进肛门的幽门胃,二者之间有缢缩。
贲门胃大而多褶皱,幽门胃小而扁平,并向各腕内伸出一对幽门盲囊(pyloric caecum)。
海星、蛇尾类消化后的残渣仍由口排出;
海参、海胆类口周围有触手或咀嚼器;
海参直肠壁上有突起,称为呼吸树,司呼吸、排泄。

\newline

棘皮动物成体外神经系统(ectoneural system)位于围血系统下方,与之走向相同,由围口神经环和辐神经干及其分支组成。
下神经系统(hyponeural system)位于围血系统管壁,与之走向相同。
内神经系统(entoneural system)位于反口面体壁,由辐神经干及其分支组成。
三个神经系统均与水管系统平行,并与上皮细胞相连。
外神经系统源于外胚层,下、内神经系统源于中胚层。
中胚层形成的神经系统为棘皮动物所特有。
感觉器官不发达。

\newline

棘皮动物多雌雄异体。
生殖腺位于管足沟之间,成熟时充满体腔。
生殖细胞经生殖管,由反口面排出体外,营体外受精。
其呼吸器官为皮鳃和管足。
排泄器官为皮鳃。
  
\section{棘皮动物的分类}
\subsection{海百合亚门(Crinozoa)}
多营固着生活,反口面生柄,司固着。
下分八纲,今仅存海百合纲(Crinoidea)。
  
\subsection{海星亚门(Asterozoa)}
多呈星状,口面向下。
海星纲(Asteroidea)腕或宽大中空,与体腔相连;
蛇尾纲(Ophiuroidea)腕细长灵活。
  
\subsection{海胆亚门(Echinozoa)}
腕不发达。
下分七纲,今存海参纲(Holothuroidea)和海胆纲(Echinoidea)。

\end{document}