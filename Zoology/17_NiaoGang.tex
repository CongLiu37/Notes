\documentclass[11pt]{article}

\usepackage[UTF8]{ctex} % for Chinese 

\usepackage{setspace}
\usepackage[colorlinks,linkcolor=blue,anchorcolor=red,citecolor=black]{hyperref}
\usepackage{lineno}
\usepackage{booktabs}
\usepackage{graphicx}
\usepackage{float}
\usepackage{floatrow}
\usepackage{subfigure}
\usepackage{caption}
\usepackage{subcaption}
\usepackage{geometry}
\usepackage{multirow}
\usepackage{longtable}
\usepackage{lscape}
\usepackage{booktabs}
\usepackage{natbib}
\usepackage{natbibspacing}
\usepackage[toc,page]{appendix}
\usepackage{makecell}

\title{鸟纲(Aves)}
\date{}

\linespread{1.5}
\geometry{left=2cm,right=2cm,top=2cm,bottom=2cm}

\begin{document}

  \maketitle

  \linenumbers

\section{一般特征}
鸟类具有恒定的体温,称为恒温动物。
恒温动物有良好的体温调节机制,使动物体温相对恒定且通常略高于环境温度,促进体内化学反应的进行,提高并稳定新陈代谢的效率,减少对环境的依赖,扩大其生态位。

\newline

鸟类身体纺锤型,体表被羽(feather),头端有角质喙(bill)。
前肢变为翼(wing),不便之处由长而灵活的颈弥补。
躯干坚实,尾部退化,利于飞行的稳定。
眼发达,有瞬膜,避免飞行时遭气流和异物的伤害。
鼓膜位于耳孔底部,耳孔周围生耳羽,利于收集声波。
后肢发达,四趾。

\newline

皮肤薄而松弛,便于肌肉剧烈运动。
皮肤腺仅有尾脂腺(oil gland),分泌油脂,保护羽毛。
水禽尾脂腺尤其发达。
皮肤衍生物包括羽、喙、爪和鳞。
羽有不同的构造和功能。
正羽(countour feather)由羽轴和羽片组成。
羽轴下端插入皮肤深处,上端密生平行排列的羽,构成羽片。
羽枝多次分支枝为羽小枝,羽小枝上有钩状突起,可互相钩接。
正羽较大,一般位于翼和尾,司飞行。
绒羽(plumule)羽轴纤弱,羽小枝上的钩状突起不发达,呈棉花状,司保温。
纤羽(filoplume)杂生于正羽和绒羽之间,形如毛发,司触觉。
鸟类口、眼附近多生具须(bristle),司触觉,为变形的羽毛。

\newline

羽毛着生于鸟类体表的特定区域,称为羽区(pteryla);
不生羽毛处为裸区(apteria)。
这种着生方式利于剧烈的飞行运动和孵卵。
鸟类羽毛定期更换,称为换羽(molt),受甲状腺控制。
鸟类换羽利于迁徙、越冬和繁殖。

\newline

鸟类骨骼轻而坚固,内多腔隙。
头骨薄轻,骨块愈合,颌骨前申,形成鸟喙。
无齿,咀嚼肌萎缩,以减轻体重,便于飞行。
颅腔膨大,头骨顶部呈圆拱形。
眼眶膨大,颅腔后移。
颈椎及头部异常灵活。
胸椎肋骨和胸骨连接,形成坚固的胸廓,使得胸肌得以剧烈运动并完成呼吸。
胸骨中线多有高耸的龙骨突(keel),增大胸肌的固着面,适应飞行。
部分胸椎和腰荐椎、部分尾椎愈合为综荐骨(synsacrum),又与腰带愈合,以在步行时支持体重,又使躯体重心集中于中央,利于飞行时保持平衡。
前肢特化为翼,手部骨骼愈合或消失,使翼的骨骼构成一个整体,扇动方能有力。

\newline

鸟类背部肌肉萎缩,颈部肌肉和胸肌发达。
支配肢体的肌肉集中于躯体中心,通过肌腱控制肢体运动,使躯体重心集中于中央。
后肢有适应于树栖的肌肉,使鸟类栖于树枝时,体重压迫和腿骨关节弯曲使肌肉拉紧,足趾弯曲。
有特殊的鸣管肌肉,可使鸣管改变形状,发出多变的声音。

\newline

鸟类有角质喙,颌骨轻便,牙齿退化,咀嚼肌萎缩。
口腔内有唾液腺,大多无口腔消化。
部分物种食管的一部分特化为嗉囊(crop),可储藏、软化食物。
胃分为腺胃(glandular stomach)和肌胃(muscular stomach)。
腺胃壁富腺体,分泌消化液;
肌胃肌肉层发达,内壁革质,可磨碎食物。
小肠和大肠交界处有一对盲肠,可吸收水分、消化植物纤维。
大肠短,不贮存粪便。
大肠通泄殖腔。
消化腺主要有肝和胰,分泌胆汁和胰液,注入十二指肠。
总的来说,鸟类消化能力强,消化过程迅速,适应于高效的新陈代谢和飞行。

\newline

鸟类呼吸系统特化明显,有发达的气囊(air sac)与气管相连。
肺体积较小,无弹性。
鸟类栖止时通过胸廓运动呼吸;
飞行时胸骨为胸肌支持点,胸廓趋于稳定,通过两翼扇动引起气囊扩张和收缩交换气体。
扬翼时,气囊扩大,吸入空气;
搧翼时,气囊收缩,呼出空气。
吸气时,空气经支气管,一部分直接进入后气囊,一部分经肺入前气囊;
呼气时,前气囊气体经支气管排出,后气囊气体入肺后经支气管排出。
鸟类呼气和吸气时皆能进行气体交换,称为双重呼吸(dual respiration),效率较高。
此外,气囊亦可减少肌肉、内脏间的摩擦并参与散热。

\newline

鸟类有完全的双循环,动静脉血严格分开,心脏两心房两心室,无静脉窦。
来自体静脉的血液经右心房、右心室、肺动脉入肺;
再经肺静脉、左心房、左心室入体动脉。
鸟类心脏重量与体重之比奇高,心脏容量大,心跳快,动脉血压高,血液循环迅速,适应于较高的代谢水平。
左侧体动脉弓消失,肾门静脉退化。
内脏血液经尾肠系膜静脉进入肝门静脉。

\newline

鸟类的排泄器官为后肾,肾单位数目较多。
肾经输尿管开口于泄殖腔。
鸟类的尿主要为难溶于水的尿酸,可减少水分散失。
无膀胱,尿粪直接排出体外,以减轻体重。
海鸟眼眶上部有开口于鼻间隔的盐腺,排出多余盐分。

\newline

鸟类的脑类似爬行类,伸出十二对脑神经,但司运动协调和平衡的小脑发达。
眼尤为发达。
眼球最外的巩膜前端附生骨片,称为巩膜骨(sclerotic ring),起支持作用,避免飞行时气流压力导致眼球变形。
可通过改变晶状体形状和位置、改变角膜形状调节视力。
听觉器官与爬行类类似,嗅觉大多退化。

\newline

鸟类雄性有成对的睾丸和输精管,开口于泄殖腔,一般无交配器官,通过雌雄鸟泄殖腔口接合受精。
雌性一般仅左侧卵巢有功能,输卵管口喇叭状,开于腹腔。
受精发生于输卵管上端。
卵壳石灰质。

\newline

鸟类繁殖有明显的季节性和多种复杂行为。
鸟类再繁殖期常各自占有一定的领域(territory),不允许其它鸟类进入,称为占区。
占区保证在巢附近有充足的食物供应;
调节营巢地鸟类密度和分布,利于充分利用资源并减少传染病;
减少其它鸟类对生殖活动的干扰。
雄性鸟类多有求偶炫耀(countship display),可激发异性的性活动,帮助辨认同种鸟类和性别。
大部分鸟类有筑巢(nest-building)行为,便于孵卵,亦司保护。
卵产于巢内并需孵化(incubation)。
有育雏行为。
此外,部分鸟类在春秋两季,沿固定路线来往于繁殖地和越冬地,称为候鸟(migrant);
其余终年留在繁殖地,称为留鸟(resident)。

\section{鸟类的分类}
\subsection{平胸总目(Ratitae)}
适应于奔走生活,翼退化,胸骨无龙骨突起,无尾综骨和尾脂腺;
无羽区和裸区之分,羽毛分布均匀。
足趾趋于减少,雄性交配器官发达。
如非洲鸵鸟(\textit{Struthio camelus})、美洲鸵鸟(\textit{Rhea americana})、鸸鹋(\textit{Dromaus novachollandeae})、几维鸟(\textit{Apteryx oweni})。

\subsection{企鹅总目(Impennes)}
仅企鹅目(Sphenisciformes)一目,适应于潜水。
前肢鳍状。
羽毛鳞状,均匀分布。
尾短。
腿短,位于躯体后方,趾间有蹼。
陆上行走时躯体直立,龙骨突起发达。
如王企鹅(\textit{Aptenodytes patagonicus})。

\subsection{突胸总目(Carinatae)}
翼发达,善飞,龙骨突起发达,有尾综骨。
骨骼内有空腔,正羽发达,体表区分羽区和裸区。
雄性多无交配器官。
可大致分为六类生态类群:游禽、涉禽、猛禽、攀禽、陆禽、鸣禽。

\subsubsection{䴙䴘目(Podicipediformes)}
游禽,善潜水。
趾有分离的瓣状蹼。
喙短钝。
尾羽几乎全部为绒羽。
如小䴙䴘(\textit{Tachybaptus ruficolli})。

\subsubsection{信天翁目(Procellariiformes)}
大型海洋鸟类。
体粗壮。
喙大而有钩,被角质片。
鼻孔管状,趾间有蹼,翼长而尖。
如短尾信天翁(\textit{Diomedea albatrus})。

\subsubsection{鹈形目(Pelecaniformes)}
大型游禽。
四趾间有蹼膜相连。
喙大,有钩。
喉囊发达。
如斑嘴鹈鹕(\textit{Pelecanus philippensis})、鸬鹚(\textit{Phalacrocorax carbo})。

\subsubsection{鹳形目(Ciconiiformes)}
水栖涉禽。
喙、颈、腿长。
胫裸露。
趾细长,四趾在同一平面。
如黑鹳(\textit{Ciconia nigra})、东方白鹳(\textit{Ciconia boyciana})、苍鹭(\textit{Ardea cinerea})、大白鹭(\textit{Egretta alba})。

\subsubsection{雁形目(Anseriformes)}
游禽。
喙扁,边缘有梳状栉板,司滤食。
喙端加厚。
腿后移,前三趾间有蹼。
气管基部有膨大的骨质囊,司发声共鸣。
雄鸟有交配器官。
尾脂腺发达。
如绿头鸭(\textit{Anas platyrhynchos})、鸿雁(\textit{Anser cygnoides})、天鹅(\textit{Cygnus cygnus})。

\subsubsection{隼形目(Falconiformes)}
肉食猛禽。
喙端有利钩。
钩爪发达。
善飞,视力发达。
如红脚隼(\textit{Falco vespertinus})、鸢(\textit{Milvus migrans})、鹗(\textit{Pandion haliaetus})、秃鹫(\textit{Aegypius monachus})。

\subsubsection{鸡形目(Galliformes)}
陆生,腿脚强健,爪钝。
喙弓形,嗉囊发达,翼短圆。
如鹌鹑(\textit{Coturnix coturnix})、鹧鸪(\textit{Francolinus pintadeanus})、石鸡(\textit{Alectoris chukar})。

\subsubsection{鹤形目(Gruiformes)}
涉禽。
喙、颈、腿长,胫裸露。
四趾不在同一平面。
如丹顶鹤(\textit{Grus japonensis})、大鸨(\textit{Otis tarda})。

\subsubsection{鸻形目(Charadriiformes)}
分为鸻和鸥两类。
鸻体多沙土色或灰色,善奔善飞。
如金眶鸻(\textit{Charadrius dubius})、燕鸻(\textit{Glareola maldivarum})。
鸥常于水边捕食。
前三趾有蹼,善飞。
如红嘴鸥(\textit{Larus ridibundus})、燕鸥(\textit{Sterna hirundo})。

\subsubsection{鸽形目(Columbiformes)}
陆禽。
喙短,鼻孔外有蜡膜(cere)。
腿脚健壮,四趾位于同一平面。
嗉囊发达,在育雏期间能分泌鸽乳喂雏。
如原鸽(\textit{Columba livia})、山斑鸠(\textit{Streptopelia orientalis})。

\subsubsection{鹦形目(Psittaciformes)}
攀禽。
第四趾朝后。
喙坚硬有钩,上喙可上抬。
如绯胸鹦鹉(\textit{Psittacula alexandri})、虎皮鹦鹉(\textit{Melopsittaccs undulatus})。

\subsubsection{鹃形目(Cuculiformes)}
攀禽。
对趾形。
外形似隼,但喙爪无钩。
多寄生性繁殖,将卵产在其他鸟巢中。
如大杜鹃(\textit{Cuculus canorus})。

\subsubsection{鸮形目(Strigiformes)}
俗称猫头鹰。
夜行性猛禽。
趾能后转,形成对趾。
两眼大而向前。
眼周有放射状细羽,构成脸盘。
耳孔大,周围有褶皱或耳羽,听觉发达。
飞行时无声。
如长耳鸮(\textit{Asio otus})。

\subsubsection{夜鹰目(Caprimulgiformes)}
夜行性攀禽。
前趾基部合并。
羽片柔软,飞行时无声。
口宽,边缘有成排硬毛。
体色近枯枝。
如夜鹰(\textit{Caprimulgus indicus})。

\subsubsection{雨燕目(Apodiformes)}
攀禽。
后趾向前。
如雨燕(\textit{Apus apus})、金丝燕(\textit{Collocalia spp})。

\subsubsection{蜂鸟目(Trochiliormes)}
攀禽,体型小。
食花蜜,管状长舌,翅搧动快。
蜜源不足时有短期休眠行为。
如蓝胸蜂鸟(\textit{Polyerata amabilis})。

\subsubsection{佛法僧目(Coraciiformes)}
攀禽,前趾基部合并,营洞巢。
如翠鸟(\textit{Alcedo atthis})

\subsubsection{䴕形目(Piciformes)}
攀禽,对趾型。
喙似凿,食树皮下昆虫。
尾羽羽轴坚硬有弹性。
如斑啄木鸟(\textit{Picoides major})。

\subsubsection{雀形目(Passeriformes)}
鸣禽。
鸣管和鸣管肌复杂。
足趾分离,三前一后,后趾中趾等长。
如百灵(\textit{Melanocorypha mongolica})、家燕(\textit{Hirundo rustica})、喜鹊(\textit{Pica pica})、画眉(\textit{Garrulax conorus})、麻雀(\textit{Passer montanus})。

\end{document}