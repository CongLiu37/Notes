\documentclass[11pt]{article}

\usepackage[UTF8]{ctex} % for Chinese 

\usepackage{setspace}
\usepackage[colorlinks,linkcolor=blue,anchorcolor=red,citecolor=black]{hyperref}
\usepackage{lineno}
\usepackage{booktabs}
\usepackage{graphicx}
\usepackage{float}
\usepackage{floatrow}
\usepackage{subfigure}
\usepackage{caption}
\usepackage{subcaption}
\usepackage{geometry}
\usepackage{multirow}
\usepackage{longtable}
\usepackage{lscape}
\usepackage{booktabs}
\usepackage{natbib}
\usepackage{natbibspacing}
\usepackage[toc,page]{appendix}
\usepackage{makecell}

\title{鱼类}
\date{}

\linespread{1.5}
\geometry{left=2cm,right=2cm,top=2cm,bottom=2cm}

\begin{document}

  \maketitle

  \linenumbers
鱼类生活于水中,体型多呈纺锤形或侧扁形,适应于游泳运动。
部分底栖鱼类呈平扁形,穴居鱼类则可呈鳗形。
鱼体分为头、躯干和尾,体表多有鳞。
头和躯干以鳃盖后缘鳃孔或最后一对鳃裂为界,躯干和尾以肛门为界。
鱼类有位于身体中线的奇鳍和成对出现的偶鳍。
鳍由鳍膜和支持它的鳍条组成。
奇鳍包括背鳍、尾鳍、臀鳍,偶鳍包括胸鳍和腹鳍。

\newline

口位于头部前方,有上下颌。
颌的出现增进摄食能力,有利于捕食;
且通过附生于颌的牙齿撕咬、碾磨,使原本不能直接利用的物质转化为食物,开拓食物源且促进消化吸收。
因此,颌的出现为脊椎动物脊椎动物结构和功能的复杂化奠定基础。
部分鱼类口周围生触须(barbels),上生味蕾。

\newline

眼一对,生于头两侧。
鼻孔一对,位于口背侧。
头后侧有骨质鳃盖(opercular),其后缘生鳃盖膜(gill membrane)。
鳃盖下为鳃腔,内生鳃弓。
鳃弓向后弯曲,鳃位于鳃弓弯曲外侧。
鳃腔前通于咽,后接鳃孔。
软骨鱼类无鳃盖和鳃腔,咽部有鳃孔,经体表鳃裂直通体外。
躯干两侧各有一条横行的侧线,是皮肤内侧线管开口于体表的小孔连接而成,为鱼类特有的感觉器官。

\newline

皮肤由表皮和真皮组成。
表皮在外,富单细胞黏液腺,分泌的黏液可润滑体表、减小摩擦阻力、调节渗透压、澄清水环境。
表皮层细胞亦可集合并陷入真皮,形成毒腺(venomous gland),常位于牙或鳍条基部,司自卫、捕食、攻击。
真皮位于表皮下方,较厚,常有钙质沉积形成鳞片,司保护。
真皮下为皮下层(subcutis),内有色素细胞。

\newline

内骨骼系统发达,分为中轴骨骼(axial skeleton)和附肢骨骼(appendicular skeleton)。
中轴骨骼包括头骨、脊柱、肋骨。
鱼类头骨骨块极多,有完整的脑颅。
脑颅后接脊柱。
脊柱由椎骨逐节排列而成,分为躯椎和尾椎。
躯椎向腹面伸出肋骨(rib)。
附肢骨骼包括鳍骨和悬挂鳍骨的带骨。
悬挂胸鳍的肩带(pectoral girdle)与头骨相连,悬挂腹鳍的腰带(pelvic girdle)结构简单。
肌肉系统包括头部肌、躯干肌和附肢肌。
头部肌包括控制眼球转动的眼肌(extrinsic eyeball muscles)和控制颌和鳃盖的鳃节肌(branchiomeric muscles)。
躯干肌包括位于躯干两侧的大侧肌(lateralis muscle)和控制奇鳍运动的棱肌(carinate muscle)。
附肢肌控制胸鳍、腹鳍的运动。
部分鱼类肌肉特化为发电器官(electric organ)。

\newline

鱼类消化道内壁常由褶皱和瓣膜,可延缓食物移动,增大吸收面积。
口和咽边界不明显,统称口咽腔,内有齿和舌。
口咽腔内壁上皮富单细胞黏液腺,无消化腺。
口咽腔和鳃腔相连,鳃弓内侧生鳃耙(gill raker),系滤食结构,上有味蕾。
食管(esophagus)短,环肌发达,内壁生味蕾。
胃膨大,通过贲门(cardiac portion)和幽门(pyloric portion)与食管、肠相接。
胃内壁有胃腺(gastric gland),分泌盐酸和各种消化酶。
肠分化不明显,肝和胰分别分泌胆汁和胰液,经肝管和胰管通入肠,以消化食物。

\newline

鱼类以生于鳃弓外侧的鳃呼吸。
软骨鱼类通过游动,使水流经口入咽,经咽壁的鳃孔入鳃,最后于体表鳃裂处排出。
硬骨鱼类鳃膜紧闭,口张开,口咽腔扩大,水进入口咽腔。
而后闭口,口咽腔缩小,将水压入鳃腔,经过鳃,最终重开鳃膜,经鳃孔到体外。

\newline

大部分鱼类有鳔(gas bladder),以调节比重。
鳔内壁有分泌气体的气腺(gas gland)。
气体的排出则是通过鳔管或卵圆窗。
鳔管连通食道;
卵圆窗则连通血管网,气体经循环系统排出。

\newline

鱼类的循环系统为单循环,动脉血和静脉血相混。
心脏位于鳃弓后下方围心腔内,有静脉窦和一心房一心室。
心室前通动脉球(bulbus arteriosus),后通心房,再后为静脉窦。
静脉窦和心房、心房和心室、心室和动脉球之间均有瓣膜,防止血液逆流,提高血压。

\newline

心脏收缩,血液经动脉球进入腹大动脉,在咽下分为左右两支动脉弓。
由动脉弓分出入鳃动脉,并于鳃处发出毛细血管网,与出鳃动脉连通。
出鳃动脉汇入背大动脉,背大动脉通往身体各部分,分为毛细血管网,而后汇合为前主静脉和后主静脉。
前、后主静脉汇合为主静脉。
消化管壁的毛细血管网两端分别接背大动脉和肝门静脉。
肝门静脉于肝出分为毛细血管网,而后汇合为肝静脉。
主静脉和肝静脉均与静脉窦连通。
心脏舒张时,静脉窦内血液进入心脏。

\newline

组织间未被静脉毛细血管吸收的组织液可进入淋巴管(lymphatic vessel),成为淋巴液(lymph)。
淋巴管在最后一节尾椎处汇合,形成一对可搏动的淋巴心,而后通入后主静脉。

\newline

鱼类神经系统包括中枢神经系统(central nervous system)、外周神经系统(peripheral nervous system)和植物性神经系统(vegetative nervous system)。
中枢神经系统包括脑和脊髓。
脑最前端为嗅脑(rhinencephalom),其后为大脑(cerebrum)。
大脑后下方接间脑(diencephalon),间脑上接为中脑(mesencephalon)。
中脑后方为小脑(cerebellum),再后为延脑(medulla oblongata)。
延脑后接脊髓。
外周神经系统包括脑发出的十对脑神经和每节脊髓发出的一对脊神经。
植物性神经亦由中枢神经系统发出,专门支配内脏器官和血管的生理活动。
内分泌系统发达。

\newline

鱼类的感觉器官较多。
眼位于头部两侧,缺乏活动性眼睑。
眼自外向内分别为巩膜、脉络膜和视网膜。
巩膜前部形成透明角膜。
脉络膜(choriod)前部形成虹膜(iris),虹膜中央为瞳孔,瞳孔内为晶状体,其下有晶状体缩肌。
晶状体无弹性,紧挨角膜,故鱼类近视,几无视觉调节能力。
视网膜(retina)外层为感光细胞,内层为神经细胞。
神经细胞于眼后部汇合,穿过感光细胞层,连接视神经。
眼内有晶体。

\newline

内耳一对,包于脑颅内听囊,浸于外淋巴液中。
内耳上部为椭圆囊(utriculus)和与之连通的三个半规管,半规管一端膨大为壶腹(ampulla)。
内耳下部为球囊(sacculus),球囊后方为瓶装囊(lagena)。
球囊和瓶装囊内有石灰质耳石(otolith)。
椭圆囊、球囊和瓶装囊内有感觉上皮,连接听神经。
内耳内为膜迷路(membrane labryinth),充满内淋巴液。
鱼体移位或有声波时,耳石移动,内淋巴液压力变化,刺激听神经。

\newline

鼻孔一对,位于口上方,一为出水孔,一为进水孔。
进水孔通鼻腔,再通出水孔。
鼻腔壁内陷,形成嗅囊。
味蕾分布广泛,见诸口腔、舌、鳃弓、鳃耙、体表、触须、鳍等。
此外,鱼类特有侧线系统(lateral line system)。
侧线管埋于头骨内和体侧皮肤下,开口于体表,形成侧线孔。
侧线管内充满黏液,管壁生感觉细胞。
侧线能感受振动,司定向。
软骨鱼吻部有罗伦氏壶腹(ampulla of Lorenzini),其基部为膨大囊状结构,壁生感觉细胞,称为罗伦瓮。
罗伦瓮经罗伦管连通体外。
罗伦氏壶腹感受振动、温度和电流。

\newline

鱼类有一对肾,司排泄和调节渗透压,位于腹腔背壁。
肾由肾小体(renal corpuscle)组成,肾小体包括肾小球(glomerulus)和肾小管(rental tuble)。
肾小球为毛细血管团。
肾小管为盲管,一侧呈杯状,包裹肾小球。
另一侧盘曲汇集为输尿管。
输尿管通膀胱,膀胱经泄殖腔或泄殖窦,与体表泄殖孔连接。
淡水鱼类体液相对于环境是高渗溶液,环境水分渗入体内,故淡水鱼需排出大量几近于水的尿液。
海水鱼类体液相对环境为低渗溶液,体内水分渗出,故海水鱼需吞饮海水,排出盐分。

\newline

鱼类大多雌雄异体,但部分种类存在雌雄同体、自体受精或性逆转现象。
精巢位于腹腔左右两侧。
硬骨鱼类营体外受精,精巢外膜后延形成输精管,左右输精管后端汇合,与尿道汇合形成泄殖窦,再开口于泄殖孔。
软骨鱼类输尿管兼有输精管的功能,其连通尿殖窦(urogenital sinus),再通于泄殖腔,开口于泄殖孔。
精子由泄殖腔、泄殖孔和鳍上的沟,进入雌性泄殖孔,营体内受精。

\newline

雌性卵巢分为游离型和封闭型。
游离型卵巢无卵囊膜,卵巢内有滤泡,每个滤泡内有一卵子。
滤泡破裂,释放卵子入腹腔,经输卵管腹腔口进入输卵管。
封闭型卵巢外包卵囊膜,卵囊膜延申形成输卵管。
卵子释放后不进入腹腔,直接进入输卵管。
软骨鱼类输卵管通泄殖腔,硬骨鱼类输卵管通泄殖窦。
鱼类大多为卵生,少数物种为卵胎生或胎生。

\newline

部分鱼类生活史中,有规律地在一定时期集群,沿固定路线迁移,以转换生活环境,满足生殖、索饵、越冬等需求,并在一段时间后重新返回原地。
这种行为称为洄游(migration)。
  
\section{软骨鱼纲(Chondrichthyes)}
海产,内骨骼全部为软骨。
体被盾鳞,鼻孔腹位。
体表有多对鳃孔,鳍条皮质,无鳔。
肠内有螺旋瓣,生殖腺和生殖导管不直接相连。
体内受精,有泄殖腔。
尾鳍上半部大而下半部小,成歪尾型。

\subsection{全头亚纲(Holocephali)}
仅银蛟目(Chimaeriformes),如黑线银蛟(\textit{Chimaera phantasma})。
头大而侧扁,尾细,体表光滑无鳞。
上颌与脑颅愈合。
四对鳃裂,鳃腔外有膜质鳃盖,其后有总鳃孔通体外。
背鳍两个,第一背鳍前有一发达硬棘,能竖立。
无泄殖腔,有泄殖孔和肛门。

\subsection{板鳃亚纲(Elasmobranchii)}
体纺锤形或扁平形。
口大,横裂于头部腹面。
鳃裂直接开口于体外。
上颌不与脑颅愈合,有泄殖腔。

\subsubsection{六鳃鲨目(Hexanchiformes)}
鳃孔六至七对。
背鳍一个,无硬棘。
有臀鳍。
底栖。
如灰六鳃鲨(\textit{Hexanchus griseus})。

\subsubsection{锯鲨目(Pristiophoriformes)}
头扁平。
吻长而突出,似剑。
鼻孔前有一对皮须,有瞬膜,无臀鳍。
如日本锯鲨(\textit{Pristiophorus japonicus})。

\subsubsection{扁鲨目(Squatiniformes)}
体平扁,胸腹鳍大,彼此接近。
背鳍一对,较小,位于尾部上方。
如日本扁鲨(\textit{Squatina japonica})。

\subsubsection{角鲨目(Squaliformes)}
背鳍两个,有硬棘。
无瞬膜,无臀鳍。
鳃孔位于胸鳍基部前方。
如长吻角鲨(\textit{Squalus mitsukurii})。

\subsubsection{虎鲨目(Heterodontiformes)}
头大吻钝,眼上有棱。
前牙尖细,后牙平扁。
背鳍两个,有硬棘。
有臀鳍。
如宽纹虎鲨(\textit{Heterodontus japonicus})。

\subsubsection{真鲨目(Carcharhiniformes)}
背鳍无硬棘,有瞬膜,有臀鳍。
如星鲨(\textit{Mustelus spp.})。

\subsubsection{须鲨目(Orectolobiformes)}
有鼻口沟。
前鼻瓣有一对鼻须。
喉部可能有一对皮须。
最后二至四对鳃孔位于胸鳍基部上方。
如鲸鲨(\textit{Rhicodon typus})。

\subsubsection{鼠鲨目(Lamniformes)}
背鳍两个,无硬棘,有臀鳍。
如姥鲨(\textit{Cetorhinus maximus})。

\subsubsection{电鳐目(Torpediniformes)}
体平扁,椭圆形,皮肤光滑。
头侧与胸鳍之间端皮下有发电器官。
如黑斑双鳍电鳐(\textit{Narcine maculata})。

\subsubsection{锯鳐目(Pristiformes)}
吻剑状,狭长平扁,边缘有吻齿。
如锯鳐(\textit{Pristis})。

\subsubsection{鳐形目(Rajiformes)}
吻圆钝,边缘无吻齿。
尾粗大,背鳍两个,无尾刺。
如团扇鳐(\textit{Platyrhina sinensis})。

\subsubsection{鲼形目(Myliobatiformes)}
胸鳍向前延伸至吻部,或向前分化为吻鳍和头鳍。
尾小,背鳍一个或无,常有尾刺。
如蝠鲼(\textit{Mobula japonica})。

\section{硬骨鱼纲(Osteichthyes)}
骨骼大多为硬骨。
部分物种鳞片次生退化。
鼻孔位于吻背面。
鳃间隔不发达。
头两侧有骨质鳃盖,鳃盖后缘有一鳃孔。
鳍条骨质。
大多有鳔,肠内无螺旋瓣。
生殖腺外膜延申为生殖导管,体外受精,无泄殖腔。

\subsection{内鼻孔亚纲(Choanichthyes)}
口腔内有内鼻孔。
偶鳍有发达的肉质基部,其内有基鳍骨,外被鳞片。
肠内有螺旋瓣。
脊索终生保留。

\subsubsection{总鳍总目(Crossopterygiomorpha)}
脊索发达,无椎体。
头下有一喉板(gular plate)。
仅腔棘鱼目(Coelacanthiformes)。

\newline

矛尾鱼(\textit{Latimeria spp.})尾鳍矛状。
尾鳍上下各有一矛状副叶,基部有中轴骨和肌肉。
无鳃盖,肠内有螺旋瓣,动脉圆锥发达,无泄殖腔,卵胎生。

\newline

马兰鱼(\textit{Malania spp.})背鳍两个,后背鳍大。
尾鳍无副叶。
体表被齿鳞。

\subsubsection{肺鱼总目(Dipneustomorpha)}
大部分骨骼为软骨,无次生颌。
脊索保留,有骨片连接脊索。
心脏前有动脉圆锥。
鳔通过鳔管通食管。
鳔管血管丰富,司呼吸。
有特化的齿板,以压碎无脊椎动物甲壳。
肠内有螺旋瓣。
尾鳍矛状,为原尾形。

\newline

单鳔肺鱼目(Ceratodontiformes)仅澳洲肺鱼(\textit{Neoceratodus forsteri})一种。
体侧扁,胸腹鳍粗壮,鳞大,鳔不成对。
幼鱼有无鳃。
成鱼不休眠。

\newline

双鳔肺鱼目(Lepiosireniformes)体呈鳗形,胸鳍狭短,鳞小且埋于皮下,鳔成对。
幼鱼有外鳃。
成鱼枯水季休眠,皮肤分泌黏液形成鱼茧,仅以鳔呼吸。
如美洲肺鱼(\textit{Lepidosiren paradoxa})、非洲肺鱼(\textit{Protopterus annectens})。

\subsection{辐鳍亚纲(Actinopterygii)}
鳍条辐射状,真皮性。
无内鼻孔,无泄殖腔,有泄殖孔和肛门。
生殖管由生殖腺壁延伸而成。

\subsubsection{硬鳞总目(Ganoidomorpha)}
体表有菱形硬鳞。
有动脉圆锥。
肠内有螺旋瓣,尾鳍矛形或歪形。
头下部常有喉板。

\newline

多鳍鱼目(Polypteriformes)背鳍多,有鳍棘。
偶鳍基部肉质,正尾形。
鳔两叶,内多分隔,开口于食管腹面,可司呼吸。
头骨和偶鳍骨有软骨,脊椎完全骨化。
幼鱼有外鳃。
全部分布于非洲淡水河流。
如多鳍鱼(\textit{Polypterus bichir})。

\newline

鲟形目(Acipenseriformes)吻长,口位于头部腹面。
部分品种体被五行纵向骨板。
亦有品种体表裸露,仅尾上叶有硬鳞。
尾鳍歪形。
多软骨,仅头部有膜质硬骨。
脊索发达,无椎体。
如中华鲟(\textit{Acipenser sinensis})、白鲟(\textit{Psephurus gladius})。

\newline

弓鳍鱼目(Amiiformes)体被圆形硬鳞,鳔辅助呼吸。
内骨骼多硬骨。
喉部有大型喉板。
仅弓鳍鱼(\textit{Amia calva})一种。

\newline

雀鳝目(Lepidosteiformes)有长吻,鼻孔位于吻端。
菱形硬鳞厚。
背鳍靠近尾鳍,与臀鳍相对。
鳔辅助呼吸。
如雀鳝(\textit{Lepidosteus oculatus})。

\subsubsection{鲱形总目(Clupeomorpha)}
腹鳍腹位,胸鳍基部位置接近腹缘。
鳍无棘,圆鳞。

\newline

海鲢目(Elopiformes)为低等类群,体被圆鳞,腹鳍腹位,喉板发达,有动脉圆锥。
如海鲢(\textit{Elops saurus})。

\newline

鼠鱚目(Gonorhynchiformes)口小,无颌齿。
部分品种无鳔。
如鼠鱚(\textit{Gonorynchus abbreviatus})。

\newline

鲱形目(Clupeiformes)体有圆鳞,无侧线。
背鳍一个,腹鳍腹位,各鳍无棘。
有鳔管,鳃耙发达。
如凤尾鱼(\textit{Coilia mystus})、鲱鱼(\textit{Clupea pallasi})。

\newline

鲑形目(Salmoniformes)背鳍后有一脂鳍(adipos fin)。
有颌齿,幽门盲囊发达。
如大麻哈鱼(\textit{Oncorhynchus keta})、哲罗鱼(\textit{Hucho taimen})。

\newline

灯笼鱼目(Myctophiformes)口大,颌、颚、舌均有能倒伏的尖齿。
有脂鳍,鳍无棘。
如龙头鱼(\textit{Harpodon nehereus})。

\newline

鲸口鱼目(Cetomimiformes)体软,有发光组织。
眼不发达。
口大,体裸露。
背鳍多与臀鳍相对。
深海产。
如紫辫鱼(\textit{Ateleopus purpureus})。

\subsubsection{骨舌总目(Osteoglosso)}
仅骨舌鱼目(Osteoglossiformes)。
口上位或端位。
副蝶骨、颌、舌均有发达的齿。
胸鳍位低,背鳍、臀鳍靠后。
体被圆鳞,有侧线。
如驼背鱼(\textit{Notopterus notopterus})。

\subsubsection{鳗鲡总目(Anguillomorpha)}
体细常,腹鳍或缺。
背鳍、臀鳍长,与尾鳍相连。

\newline

鳗鲡目(Anguilliformes)体长圆筒形,无腹鳍。
背鳍、臀鳍、尾鳍相连。
各鳍无棘,体有圆鳞或无鳞。
如日本鳗鲡(\textit{Anguilla japonica})。

\subsubsection{鲤形总目(Cyprinomorpha)}
较低等。
腹鳍腹位。
鳔通过管与食道相连。

\newline

鲤形目(Cypriniformes)体有圆鳞或无鳞。
口内无齿,下咽骨有发达的咽齿。
如青鱼(\textit{Mylopharyngodon piceus})、草鱼(\textit{Ctenopharyngodon idella})、鲢鱼(\textit{Hypophthalmichthys molitrix})、鳙鱼(\textit{Aristichthys nobilis})、鲫鱼(\textit{Carassius auratus})、泥鳅(\textit{Misgurnus anguillicaudatus})。

\newline

鲇形目(Siluriformes)体表无鳞,或被骨板。
口须一至四对,有颌齿,咽骨有细齿。
口大,胸鳍位置低。
胸背鳍有骨质鳍棘。
常有脂鳍。
如鲇鱼(\textit{Silurus asotus})。

\subsubsection{银汉鱼总目(Atherinomorpha)}
体被圆鳞。
腹鳍腹位,鳍条五至九枚。
背鳍、臀鳍对生。

\newline

鱂鱼目(Cyprinodontiformes)鳍无棘。
背鳍一个,与臀鳍相对。
无侧线。
如食蚊鱼(\textit{Gambusia affinis})。

\newline

银汉鱼目(Atheriniformes)体圆筒或侧扁,侧线不发达。
两背鳍,第一背鳍棘柔韧。
腹鳍小,胸位或腹位。
被圆鳞或栉鳞。
如麦银汉鱼(\textit{Atherion elymus})。

\newline

颌针鱼目(Beloniformes)鳍无棘。
背鳍一个。
侧线低位。
如斑鳍飞鱼(\textit{Cypselurus poecilopterus})。

\subsubsection{鲑鲈总目(Parapercomorpha)}
体被圆鳞或裸露。
颌部常有小须。

\newline

鳕形目(Gadiformes)背鳍臀鳍长,腹鳍喉位或颏位,无鳍棘。
多被圆鳞,闭鳔。
如江鳕(\textit{Lota lota})、大头鳕(\textit{Gadus macrocephalus})。

\newline

鲑鲈目(Percopsiformes)有脂鳍。
腹鳍腹位,背鳍、臀鳍有一至二个鳍棘。
胸鳍低位,侧线完整。

\subsubsection{鲈形总目(Percomorpha)}
胸鳍胸位或喉位。
鳍多有棘。
常被栉鳞。

\newline

金眼鲷目(Beryciformes)体长椭圆或卵圆形,侧扁。
背鳍臀鳍有棘,被栉鳞、圆鳞或无鳞。
两颌牙群绒状。
海产。
如红锯鳞鱼(\textit{Myripristis pralinia})。

\newline

海鲂目(Zeiformes)体极侧扁,鳞细小。
背鳍、臀鳍基部和胸腹部有棘状骨板。
背鳍棘条发达,腹鳍胸位。
如海鲂(\textit{Zeus faber})。

\newline

月鱼目(Lampriformes)体长带状,侧扁。
头部无棘,口小,颌可伸缩。
齿细小或无,有假鳃,被圆鳞或无鳞,无鳍棘。
如儒氏皇带鱼(\textit{Regalecus russellii})。

\newline

刺鱼目(Gasterosteiformes)体表多有骨板,吻管状,背鳍一至二个。
第一背鳍可能为游离的棘。
常被骨板。
如中华多刺鱼(\textit{Pungitius sinensis})、日本海马(\textit{Hippocampus japonicus})。

\newline

鲻形目(Mugiliformes)背鳍两个,第一背鳍仅有鳍棘。
腹鳍腹位或胸位。
如鲻鱼(\textit{Mugil cephalus})。

\newline

合鳃目(Synbranchiformes)体鳗形,无胸鳍腹鳍。
奇鳍彼此相连,无鳍棘。
左右鳃孔合并于头部腹面。
鳃不发达,咽和肠辅助呼吸。
无鳔无鳞。
如黄鳝(\textit{Monopterus albus})。

\newline

鲈形目(Perciformes)为鱼类第一大目。
腹鳍胸位或喉位。
前背鳍有棘,后背鳍无棘,与臀鳍相对。
多被栉鳞,鳔无鳔管。
如鲈鱼(\textit{Lateolabrax japonicus})、鳜鱼(\textit{Siniperca chuatsi})、河鲈(\textit{Perca fluviatilis})、大黄鱼(\textit{Larimichthys crocea})、罗非鱼(\textit{Oreochromis niloticus})、大弹涂鱼(\textit{Boleophthalmus peotinirostris})、带鱼(\textit{Trichiurus haumela})、金枪鱼(\textit{Thunnus tonggol})、黑鱼(\textit{Channa argus})。

\newline

鲉形目(Scorpaeniformes)第一眶下骨延后形成骨突,连接前鳃盖骨。
头粗壮,常有棘棱或骨板。
胸鳍基部宽大。
如松江鲈鱼(\textit{Trichidermus fasciatas})。

\newline

鲽形目(Pleuronectiformes)俗称比目鱼。
底栖,体侧扁。
成鱼左右不对称,两眼位于身体同侧。
无眼的一侧色浅,并以此侧平卧海底。
被圆鳞或栉鳞。
背鳍、臀鳍基底长,腹鳍胸位或喉位。
肛门位于胸鳍后下方,多不在腹面中线。
无鳔。
如褐牙鲆(\textit{Paralichthys alivaceus})、半滑舌鳎(\textit{Cynoglossus semilaevis})。

\newline

鲀形目(Tetrodontiformes)体短粗。
皮肤裸露,或生小刺、骨板、粒鳞。
上颌骨和前颌骨愈合。
牙齿锥形或门齿状,或愈合为喙状。
鳃孔小,部分种类体内有气囊,可使胸腹膨胀。
腹鳍胸位,或与腰带骨一同消失。
如绿鳍马面鲀(\textit{Thanbaconus septentrionalis})、虫纹东方鲀(\textit{Takifugu vermicularis})、翻车鱼(\textit{Mola mola})。

\subsubsection{蟾鱼总目(Batrachoidomorpha)}
底栖。
体短粗,平扁或侧扁。
皮肤裸露,有小刺或小骨板。
鳃孔小,位于胸鳍外侧腹面。
腹鳍胸位或喉位。

\newline

海蛾鱼目(Pegasiformes)体宽短,平扁。
躯干圆盘状,尾细,头短,吻突出。
眼大,下侧位。
口小,下位,无齿。
无鳞,被骨板。
背鳍一个,与小臀鳍相对,胸鳍宽大。
如飞海蛾鱼(\textit{Pegasus volitans})。

\newline

鮟鱇目(Lophiiformes)体粗短,背腹扁平,底栖,无鳞。
头大,眼位于头背面或侧面。
胸鳍呈足状,司爬行。
腹鳍喉位。
背鳍棘移至头额部,末端肉质,可作为诱饵吸引猎物。
如黄鮟鱇(\textit{Lophius litulon})。

\newline

喉盘鱼目(Gobiesociformes)喉部有吸盘。
体前端平扁,后端侧扁,无鳞。
头大而平扁,口延伸至眼中部下方,唇厚牙小。
头部黏液腺发达,侧线孔不明显。
背鳍臀鳍无棘。
如黄喉盘鱼(\textit{Lepadichthys frenatus})。

\end{document}