\documentclass[11pt]{article}

\usepackage[UTF8]{ctex} % for Chinese 

\usepackage{setspace}
\usepackage[colorlinks,linkcolor=blue,anchorcolor=red,citecolor=black]{hyperref}
\usepackage{lineno}
\usepackage{booktabs}
\usepackage{graphicx}
\usepackage{float}
\usepackage{floatrow}
\usepackage{subfigure}
\usepackage{caption}
\usepackage{subcaption}
\usepackage{geometry}
\usepackage{multirow}
\usepackage{longtable}
\usepackage{lscape}
\usepackage{booktabs}
\usepackage{natbib}
\usepackage{natbibspacing}
\usepackage[toc,page]{appendix}
\usepackage{makecell}

\title{爬行纲(Reptile)}
\date{}

\linespread{1.5}
\geometry{left=2cm,right=2cm,top=2cm,bottom=2cm}

\begin{document}

  \maketitle

  \linenumbers
\section{一般特征}
爬行动物胚胎具有羊膜(amnion),遂能彻底摆脱在个体发育初期对水环境的依赖。
爬行动物胚胎和卵黄相连,从内到外依次包有羊膜、尿囊膜(allantois)、绒毛膜(chorion)、壳膜(shell membrane)、卵壳(shell)。
胚胎位于羊膜腔内,腔内充满羊水。
尿囊膜、绒毛膜、壳膜(shell membrane)、卵壳(shell)在一区域紧贴。
尿囊膜和绒毛膜内壁富血管,可通过多孔的壳膜和卵壳进行气体交换。
尿囊膜和羊膜之间的空腔储存代谢废物,壳膜和绒毛膜之间的空腔内充满蛋白。
卵壳为石灰质或纤维质,司保护。

\newline

爬行动物体表被鳞片,表皮高度角质化,有效防止水分蒸发。
皮肤干燥,皮肤腺不发达,有蜕皮现象。
大部分物种有活动性眼睑。鼓膜内陷,形成外耳道。
一般四肢发达,五趾五指,有爪。

\newline

爬行动物骨骼骨化程度高,鲜有软骨。
头骨高而隆起,出现次生腭(secondary palate),内鼻孔后移。
颅骨两侧眼眶后方一般有一到二个颞孔(temporal fossa)。
咬肌收缩时,肌肉膨大,凸入颞孔。
脊柱进一步分化为颈椎、胸椎、腰椎、荐椎、尾椎。
头部灵活,可进行上下运动和转动。
颈椎、胸椎、腰椎两侧附生肋骨。
部分物种胸椎肋骨和腹部中线的胸骨相接,形成胸廓(throax),以保护内脏。
肋间肌控制胸廓的扩展和收缩,加强呼吸机能。
肩带不与脊柱直接相连,前肢更为灵活。
四肢和躯干位于同一平面,彼此垂直,故只能腹部紧贴地面爬行。

\newline

爬行动物出现皮肤肌(skin muscle)和肋间肌(intercostal muscle)。
皮肤肌调节体表鳞片的活动。
肋间肌位于肋骨之间,调节肋骨升降,引起腹胸腔体积变化。
四肢肌肉发达,躯干肌相对萎缩,尤其是背部肌肉。

\newline

爬行动物口腔和咽腔分界明显。
口腔内出现相对完整的次生腭,内鼻孔后移,出现鼻腔,避免摄食和呼吸相互干扰。
口腔腺体发达,有肌肉质的舌,可司吞咽、感觉、捕食。
牙齿为同型齿,只能咬食,不能咀嚼。

\newline

肺功能完善,无鳃呼吸和皮肤呼吸。
肺在胸腹腔两侧,呈囊状,内部间隔复杂。
部分物种肺后部内壁平滑,形成气囊,不司气体交换。
爬行动物出现支气管。
器官前端膨大为喉头(larynx),后端分支形成支气管,通入左右肺。
爬行动物可通过口底运动进行口咽式呼吸,或通过胸廓活动进行胸腹式呼吸。

\newline

循环系统为不完善的双循环。
心脏为两心房一心室,静脉窦部分并入右心房,无动脉圆锥。
心室内有不完全的室间隔,区分多氧血和少氧血。
肺动脉从心室出发,入肺分支,汇合为肺静脉,经右心房进入心室,构成肺循环。
心室右侧多氧血进入动脉系统,经静脉系统回到左心房,再入心室,构成体循环。
爬行动物肾静脉退化,从后肢进入心脏的静脉,一部分入肾,分散为毛细血管后形成肾门静脉;
另一部分直接汇入后大静脉。

\newline

爬行动物开始出现后肾,紧贴身体后半部背壁,肾单位多。
输尿管不与生殖导管汇合,而是直接通入泄殖腔。
爬行动物所排尿液,尿酸含量高。
尿酸难溶于水,常形成沉淀,随粪便排出。
在此过程中,水分被重吸收,以适应干旱环境。

\newline

爬行动物脑的各部分部在同一平面,大脑增大,其神经活动渐有向大脑集中的趋势,开始具有十二对脑神经。
脊髓长,有明显的胸膨大和腰荐膨大,控制附肢。
大部分物种有活动性眼睑和瞬膜。
通过改变晶状体的位置和形状调节视力。
耳与两栖动物类似,但鼓膜下陷,外耳渐现。
内耳下端瓶装囊扩大、延长,逐渐形成卷曲的耳蜗(cochlea)。
鼻腔内出现鼻甲骨(conchae),上覆嗅上皮。
鼻腔前部有开口于口腔的盲囊,即犁鼻器,司嗅觉。
部分物种有红外线感受器,位于眼鼻之间或唇部。

\newline

营体内受精。
雄性精巢一对,输精管通泄殖腔。
泄殖腔内有交配器,可充血膨大,伸出体外,将精液注入雌性体内。
雌性卵巢一对,输卵管上端为喇叭口,开于体腔。
输卵管中段分泌蛋白,下段分泌卵壳,末端通泄殖腔。

\section{爬行动物的分类}
\subsection{龟鳖目(Chelonia)}
身体宽短,有骨质硬壳,分别称为腹甲和背甲。
头、颈部、四肢、尾外露。
胸腰椎、此处的肋骨和背甲愈合,肩带位于肋骨腹面。
无胸骨,无颞孔,无齿。

\subsubsection{龟科(Testudinidae)}
陆栖,四肢粗壮,爪钝且强。
龟壳完整,壳外覆角质鳞板。
颈部可呈S形缩入壳内。
如乌龟(\textit{Chinemys reevesii})、四爪陆龟(\textit{Testudo horsfieldi})。

\subsubsection{棱皮龟科(Dermochelyidae)}
仅棱皮龟(\textit{Dermochelys coriacea})一种。
大型海产品种,四肢桨状。
甲板外无角质鳞板,覆革皮。
背面七条纵棱,在背甲后方汇合。

\subsubsection{海龟科(Chelonidae)}
四肢桨状,甲板外有角质鳞板,海产。
背腹甲之间以韧带相连。
头颈四肢不能缩入壳内。
如海龟(\textit{Chelonia mydas})、玳瑁(\textit{Ertmochelys imbricata})。

\subsubsection{鳖科(Trionychidae)}
甲板外覆革皮,指趾间生蹼,吻伸长成管状,淡水生。
如甲鱼(\textit{Pelodiscus sinensis})。

\subsection{喙头目(Rhynchocephalia)}
现存仅喙头蜥(\textit{Sphenodon punctatum})一个种。
形似蜥蜴,被细颗粒状鳞片。
头有两个颞孔,嘴长似鸟喙。
顶眼发达,泄殖腔孔横裂。

\subsection{有鳞目(Squamata)}
现存大部分爬行动物皆属此目。
有两个颞孔,体表被鳞。

\subsubsection{蜥蜴亚目(Lacertilia)}
多有附肢、肩带和胸骨。
左右下颌骨在前端合并,愈合处有骨缝。
眼睑可动,多有鼓膜、鼓室、咽鼓管。

\newline
壁虎科(Gekkonidae)眼大,瞳孔垂直,有眼睑。
体被颗粒状鳞。
指趾末端有膨大的吸盘状肉垫。
尾有自残及再生功能。
如无蹼壁虎(\textit{Gekko swinhonis})一种。

\newline

避役科(Chamaeleonidae)树栖,眼大而突出,眼睑厚。
每个眼可独立活动,独立调距。
舌发达。
指趾合并为内外二组,适合握枝。
尾适合缠绕。
皮肤可变色。
如避役(\textit{Chamaeleon vulgaris})。

\newline

石龙子科(Scincidae)体粗壮,四肢短小或消失。
多被圆形光滑鳞盘,覆瓦状排列。
角质鳞下有源自真皮的骨鳞。
眼睑透明。
如蓝尾石龙子(\textit{Eumeces elegans})。

\newline

蜥蜴科(Lacertidae)体鳞具棱嵴。
有大型对称鳞板紧贴头骨。四肢发达有爪。
如丽斑麻蜥(\textit{Eremias argus})。

\newline

蛇蜥科(Anguidae)体蛇形,四肢退化,后肢骨有残痕。
体被圆鳞,鳞下有骨板。
眼小,有活动眼睑/
断尾可再生。
如脆蛇蜥(\textit{Ophisaurus harti})。

\newline

鳄蜥科(Shinisuridae)仅鳄蜥(\textit{Shinisaurus crocodilurus})一种。
尾似鳄鱼,躯体圆柱形,四肢粗壮有爪。
背部有数行纵棱。
舌短,前端分叉。
卵胎生。

\newline

巨蜥科(Varanidae)骨鳞退化,背鳞颗粒状。
头颈长,四肢发达。
如圆鼻巨蜥(\textit{Varanus salvator})。

\subsubsection{蛇亚目(Serpentes)}
附肢退化,无肩带和胸骨。
左右下颌前端以韧带相连。
眼睑不可动。
无外耳。
舌伸缩性强,末端分叉。

\newline

盲蛇科(TYphlopidae)形似蚯蚓,被圆鳞,尾短,眼退化,口小,下颌无齿,腰带退化,后肢有残痕。
如钩盲蛇(\textit{Ramphotyphlops braminus})。

\newline

蟒科(Boidae)鳞片小,腰带退化,有股骨痕迹。
泄殖腔两侧有角质爪,为后肢残留。
肺成对。
部分物种有孵卵形为,以肌肉节律性收缩升高体温。
无毒牙,缠绕绞杀猎物。
多有唇窝(labral pit)作为热感受器。
如蟒蛇(\textit{Python molurus})。

\newline

蝰科(Viperinae)颌骨短而灵活,有管状毒牙。
体粗壮,尾短。
如草原蝰(\textit{Vipera ursini})、蝮蛇(\textit{Agkisirodon halys})、五步蛇(\textit{Agkisirodon acutus})、竹叶青(\textit{Trimeresurus stejnegeri})、烙铁头(\textit{Trimeresurus mucrosquamatus})。

\newline

游蛇科(Colubridae)颌骨水平着生,无沟牙(aglyphous)或后沟牙(opisthoglyphous)。
蛇类近百分之九十为此科。
如赤练蛇(\textit{Dinodon rufozonatum})、黑眉锦蛇(\textit{Elaphe taeniurus})、中国水蛇(\textit{Enhydris chinesis})。

\newline

眼镜蛇科(Elapidae)颌骨短,一对长形前沟牙(proteroglyphous),尾不侧扁。
如眼镜蛇(\textit{Naja naja})、金环蛇(\textit{Bungarus fasciatus})、银环蛇(\textit{Bungarus multicinctus})。

\subsection{鳄目(Crocodiliformes)}
体长大被甲,尾粗壮侧扁,头扁平,吻长。
指趾间有蹼。齿锥形,舌不能外申。
外鼻孔、外耳孔有活动瓣膜。
如扬子鳄(\textit{Alligator sinensis})。

\end{document}