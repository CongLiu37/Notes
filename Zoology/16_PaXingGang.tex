\documentclass[11pt]{article}

\usepackage[UTF8]{ctex} % for Chinese 

\usepackage{setspace}
\usepackage[colorlinks,linkcolor=blue,anchorcolor=red,citecolor=black]{hyperref}
\usepackage{lineno}
\usepackage{booktabs}
\usepackage{graphicx}
\usepackage{float}
\usepackage{floatrow}
\usepackage{subfigure}
\usepackage{caption}
\usepackage{subcaption}
\usepackage{geometry}
\usepackage{multirow}
\usepackage{longtable}
\usepackage{lscape}
\usepackage{booktabs}
\usepackage{natbib}
\usepackage{natbibspacing}
\usepackage[toc,page]{appendix}
\usepackage{makecell}

\title{爬行纲(Reptile)}
\date{}

\linespread{1.5}
\geometry{left=2cm,right=2cm,top=2cm,bottom=2cm}

\begin{document}

  \maketitle

  \linenumbers
\section{一般特征}
爬行动物胚胎具有羊膜(amnion),遂能彻底摆脱在个体发育初期对水环境的依赖。
爬行动物胚胎和卵黄相连,从内到外依次包有羊膜、尿囊膜(allantois)、绒毛膜(chorion)、壳膜(shell membrane)、卵壳(shell)。
胚胎位于羊膜腔内,腔内充满羊水。
尿囊膜、绒毛膜、壳膜(shell membrane)、卵壳(shell)在一区域紧贴。
尿囊膜和绒毛膜内壁富血管,可通过多孔的壳膜和卵壳进行气体交换。
尿囊膜和羊膜之间的空腔储存代谢废物,壳膜和绒毛膜之间的空腔内充满蛋白。
卵壳为石灰质或纤维质,司保护。

\newline

爬行动物体表被鳞片,表皮高度角质化,有效防止水分蒸发。
皮肤干燥,皮肤腺不发达,有蜕皮现象。
大部分物种有活动性眼睑。鼓膜内陷,形成外耳道。
一般四肢发达,五趾五指,有爪。

\newline

爬行动物骨骼骨化程度高,鲜有软骨。
头骨高而隆起,出现次生腭(secondary palate),内鼻孔后移。
颅骨两侧眼眶后方一般有一到二个颞孔(temporal fossa)。
咬肌收缩时,肌肉膨大,凸入颞孔。
脊柱进一步分化为颈椎、胸椎、腰椎、荐椎、尾椎。
头部灵活,可进行上下运动和转动。
颈椎、胸椎、腰椎两侧附生肋骨。
部分物种胸椎肋骨和腹部中线的胸骨相接,形成胸廓(throax),以保护内脏。
肋间肌控制胸廓的扩展和收缩,加强呼吸机能。
肩带不与脊柱直接相连,前肢更为灵活。
四肢和躯干位于同一平面,彼此垂直,故只能腹部紧贴地面爬行。

\newline

爬行动物出现皮肤肌(skin muscle)和肋间肌(intercostal muscle)。
皮肤肌调节体表鳞片的活动。
肋间肌位于肋骨之间,调节肋骨升降,引起腹胸腔体积变化。
四肢肌肉发达,躯干肌相对萎缩,尤其是背部肌肉。

\newline

爬行动物口腔和咽腔分界明显。
口腔内出现相对完整的次生腭,内鼻孔后移,出现鼻腔,避免摄食和呼吸相互干扰。
口腔腺体发达,有肌肉质的舌,可司吞咽、感觉、捕食。
牙齿为同型齿,只能咬食,不能咀嚼。

\newline

肺功能完善,无鳃呼吸和皮肤呼吸。
肺在胸腹腔两侧,呈囊状,内部间隔复杂。
部分物种肺后部内壁平滑,形成气囊,不司气体交换。
爬行动物出现支气管。
器官前端膨大为喉头(larynx),后端分支形成支气管,通入左右肺。
爬行动物可通过口底运动进行口咽式呼吸,或通过胸廓活动进行胸腹式呼吸。

\newline

循环系统为不完善的双循环。
心脏为两心房一心室,静脉窦部分并入右心房,无动脉圆锥。
心室内有不完全的室间隔,区分多氧血和少氧血。
肺动脉从心室出发,入肺分支,汇合为肺静脉,经右心房进入心室,构成肺循环。
心室右侧多氧血进入动脉系统,经静脉系统回到左心房,再入心室,构成体循环。
爬行动物肾静脉退化,从后肢进入心脏的静脉,一部分入肾,分散为毛细血管后形成肾门静脉;
另一部分直接汇入后大静脉。

\newline

爬行动物开始出现后肾,紧贴身体后半部背壁,肾单位多。
输尿管不与生殖导管汇合,而是直接通入泄殖腔。
爬行动物所排尿液,尿酸含量高。
尿酸难溶于水,常形成沉淀,随粪便排出。
在此过程中,水分被重吸收,以适应干旱环境。

\newline

爬行动物脑的各部分部在同一平面,大脑增大,其神经活动渐有向大脑集中的趋势,开始具有十二对脑神经。
脊髓长,有明显的胸膨大和腰荐膨大,控制附肢。
大部分物种有活动性眼睑和瞬膜。
通过改变晶状体的位置和形状调节视力。
耳与两栖动物类似,但鼓膜下陷,外耳渐现。
内耳下端瓶装囊扩大、延长,逐渐形成卷曲的耳蜗(cochlea)。
鼻腔内出现鼻甲骨(conchae),上覆嗅上皮。
鼻腔前部有开口于口腔的盲囊,即犁鼻器,司嗅觉。
部分物种有红外线感受器,位于眼鼻之间或唇部。

\newline

营体内受精。
雄性精巢一对,输精管通泄殖腔。
泄殖腔内有交配器,可充血膨大,伸出体外,将精液注入雌性体内。
雌性卵巢一对,输卵管上端为喇叭口,开于体腔。
输卵管中段分泌蛋白,下段分泌卵壳,末端通泄殖腔。

\section{爬行动物的分类}
\subsction{龟鳖目(Chelonia)}
身体宽短,有骨质硬壳,分别称为腹甲和背甲。
头、颈部、四肢、尾外露。
胸腰椎、此处的肋骨和背甲愈合,肩带位于肋骨腹面。
无胸骨,无颞孔,无齿。

\subsection{缘头目(Rhynchocephalia)}
现存仅\textit{Sphenodon punctatum}一个种,有两个颞孔。

\subsection{有鳞目(Squamata)}
现存大部分爬行动物皆属此目,下分蜥蜴亚目(Lacertilia)、蛇亚目(Serpentes)、蚓蜥亚目(Amphisbaenia)。
有两个颞孔,体表被鳞。

\subsection{鳄目(Crocodiliformes)}
体长大,尾粗壮侧扁,头扁平,吻长。
指趾间有蹼。齿锥形,舌不能外申。
外鼻孔、外耳孔有活动瓣膜。


\end{document}