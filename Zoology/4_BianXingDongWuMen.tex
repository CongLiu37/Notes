\documentclass[11pt]{article}

\usepackage[UTF8]{ctex} % for Chinese 

\usepackage{setspace}
\usepackage[colorlinks,linkcolor=blue,anchorcolor=red,citecolor=black]{hyperref}
\usepackage{lineno}
\usepackage{booktabs}
\usepackage{graphicx}
\usepackage{float}
\usepackage{floatrow}
\usepackage{subfigure}
\usepackage{caption}
\usepackage{subcaption}
\usepackage{geometry}
\usepackage{multirow}
\usepackage{longtable}
\usepackage{lscape}
\usepackage{booktabs}
\usepackage{natbib}
\usepackage{natbibspacing}
\usepackage[toc,page]{appendix}
\usepackage{makecell}

\title{扁形动物门(Platyhelminthes)}
\date{}

\linespread{1.5}
\geometry{left=2cm,right=2cm,top=2cm,bottom=2cm}

\begin{document}

  \maketitle

  \linenumbers
\section{一般特征}
从扁形动物出现两侧对称(bilateral symmetry)的体型,虫体分出前后、左右、腹背。
体背面司保护,腹面司运动。
向前的一端总是先接触新环境,与之相对应的是神经系统和感觉器官向体前集中,逐渐出现头部,动物的运动由不定项变为定向,对环境的感应也更为准确、迅速。

\newline

三胚层结构亦出现于扁形动物,即在内外胚层中间出现中胚层。
中胚层为组织、器官、系统的进一步分化和复杂化奠定物质基础。
此外,中胚层的出现促进了新陈代谢的加强,遂需加强运动能力以摄取更多食物,产生更多代谢废物,故需更复杂的排泄系统。
对运动能力的需求存进肌肉系统的复杂化,进而使得动物更多接触变化多端的环境,促进神经系统和感觉器官的进一步发展。
总之,中胚层的出现促进了动物结构的复杂化和功能的完备化。

\newline

扁形动物表皮起源于外胚层,腹面表皮有纤毛。
营自由生活的物种表皮中有杆状体,遇刺激时杆状体排出,弥散有毒粘液,供捕食和防御。
表皮以下是源自中胚层的三层肌肉,从外到内分别为环肌、斜肌和纵肌。

\newline

扁形动物内胚层形成盲管,即肠。
肠的开口兼司口和肛门的作用。
扁形动物无呼吸、循环器官,依靠体表扩散作用交换气体,但有原肾管(protonephridium)的排泄系统。
原肾管是外胚层内陷形成的有分支的盲管,位于虫体两侧。
盲管末端为焰细胞(flame cell)。焰细胞的鞭毛伸入原肾管,鞭毛打动,推动虫体内液体经焰细胞的过滤,进而排除代谢废物。

\newline

由于两侧对称的体型,扁形动物神经细胞向体前集中,形成原始的脑,并由此向体后端分出若干纵神经索(longitudinal nerve cord)。
纵神经索之间又有横神经(transverse commisure),整个神经系统形状如同梯子。

\newline

扁形动物大多雌雄同体,部分种类雌雄异体或雌雄异形,有固定的生殖腺和特定的生殖导管,以及一系列附属腺。
不同类群的生殖系统结构差异颇大,但均有交配行为和体内受精。

\section{扁形动物的分类}
\subsection{涡虫纲(Turbellaria)}
主要营自由生活,体表被纤毛,肌肉系统、神经系统和感觉器官发达。

\subsection{吸虫纲(Trematoda)}
全部营寄生生活,肌肉系统、神经系统和感觉器官不发达,消化系统简单,生殖系统复杂,体表有发达的吸附器以附着寄主。

\subsection{绦虫纲(Cestoida)}
全部营寄生生活。
虫体分节,由头节(scolex)、幼节(neck)、成节(mature proglottid)和孕节(gravid proglottid)组成带状链体。
其中头节为吸附器官,幼节负责产生新的节片,孕节内含虫卵。
每个节片均有发达的生殖系统。


\end{document}