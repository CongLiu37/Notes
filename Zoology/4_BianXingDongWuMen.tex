\documentclass[11pt]{article}

\usepackage[UTF8]{ctex} % for Chinese 

\usepackage{setspace}
\usepackage[colorlinks,linkcolor=blue,anchorcolor=red,citecolor=black]{hyperref}
\usepackage{lineno}
\usepackage{booktabs}
\usepackage{graphicx}
\usepackage{float}
\usepackage{floatrow}
\usepackage{subfigure}
\usepackage{caption}
\usepackage{subcaption}
\usepackage{geometry}
\usepackage{multirow}
\usepackage{longtable}
\usepackage{lscape}
\usepackage{booktabs}
\usepackage{natbib}
\usepackage{natbibspacing}
\usepackage[toc,page]{appendix}
\usepackage{makecell}

\title{扁形动物门(Platyhelminthes)}
\date{}

\linespread{1.5}
\geometry{left=2cm,right=2cm,top=2cm,bottom=2cm}

\begin{document}

  \maketitle

  \linenumbers
\section{一般特征}
从扁形动物出现两侧对称(bilateral symmetry)的体型,虫体分出前后、左右、腹背。
体背面司保护,腹面司运动。
向前的一端总是先接触新环境,与之相对应的是神经系统和感觉器官向体前集中,逐渐出现头部,动物的运动由不定项变为定向,对环境的感应也更为准确、迅速。

\newline

三胚层结构亦出现于扁形动物,即在内外胚层中间出现中胚层。
中胚层为组织、器官、系统的进一步分化和复杂化奠定物质基础。
此外,中胚层的出现促进了新陈代谢的加强,遂需加强运动能力以摄取更多食物,产生更多代谢废物,故需更复杂的排泄系统。
对运动能力的需求存进肌肉系统的复杂化,进而使得动物更多接触变化多端的环境,促进神经系统和感觉器官的进一步发展。
总之,中胚层的出现促进了动物结构的复杂化和功能的完备化。

\newline

扁形动物表皮起源于外胚层,腹面表皮有纤毛。
营自由生活的物种表皮中有杆状体,遇刺激时杆状体排出,弥散有毒粘液,供捕食和防御。
表皮以下是源自中胚层的三层肌肉,从外到内分别为环肌、斜肌和纵肌。

\newline

扁形动物内胚层形成盲管,即肠。
肠的开口兼司口和肛门的作用。
扁形动物无呼吸、循环器官,依靠体表扩散作用交换气体,但有原肾管(protonephridium)的排泄系统。
原肾管是外胚层内陷形成的有分支的盲管,位于虫体两侧。
盲管末端为焰细胞(flame cell)。焰细胞的鞭毛伸入原肾管,鞭毛打动,推动虫体内液体经焰细胞的过滤,进而排除代谢废物。

\newline

由于两侧对称的体型,扁形动物神经细胞向体前集中,形成原始的脑,并由此向体后端分出若干纵神经索(longitudinal nerve cord)。
纵神经索之间又有横神经(transverse commisure),整个神经系统形状如同梯子。

\newline

扁形动物大多雌雄同体,部分种类雌雄异体或雌雄异形,有固定的生殖腺和特定的生殖导管,以及一系列附属腺。
不同类群的生殖系统结构差异颇大,但均有交配行为和体内受精。

\section{扁形动物的分类}
\subsection{涡虫纲(Turbellaria)}
主要营自由生活,体表被纤毛,肌肉系统、神经系统和感觉器官发达。

\subsubsection{无肠目(Acoela)}
海生,无消化道,以源自内胚层的细胞团进行吞噬、消化。
无原肾管,神经系统不发达,无输卵管。
如旋涡虫(\textit{Convoluta spp.})。

\subsubsection{大口虫目(Macrostomida)}
水生,有简单的咽和囊状的肠,一对腹索神经。
生殖系统完整,但常行无性生殖。
虫体横分裂侯不分开,形成虫链。
如大口虫(\textit{Macrostomum spp.})、微口涡虫(\textit{Microstomum spp.})。

\subsubsection{多肠目(Polycladida)}
海产,虫体前缘或背部有一对触手(tentacle)。
多眼,肠自体中央向四周分出众多盲管。
有折叠咽(plicate pharynx),神经系统和生殖系统完整。
如平角涡虫(\textit{Planocera spp.})。

\subsubsection{三肠目(Tricladida)}
有折叠咽。
肠分三支,一支向前,两支向后,每支上各有分支。
原肾管一对,卵巢一对。
如日本三角涡虫(\textit{Dugesia japonica})、蛭态涡虫(\textit{Bdelloura spp.})、陆生的笄蛭涡虫(\textit{Bipalium spp.})。

\subsection{吸虫纲(Trematoda)}
全部营寄生生活,肌肉系统、神经系统和感觉器官不发达,消化系统简单,生殖系统复杂,体表有发达的吸附器以附着寄主。

\subsubsection{单殖亚纲(Monogenea)}
体外寄生,直接发育。
常缺少口吸盘,体后有附着器官。
排泄空开口于体前端。
如三代虫(\textit{Gryodactylus spp.})、指环虫(\textit{Dactylogyrus spp.})皆寄生鱼类。

\subsubsection{盾腹亚纲(Aspidogastrea)}
腹部有覆盖整个腹面的大吸盘,或一纵列吸盘。
多内寄生于鱼类、爬行类的消化道,或软体动物的围心腔、肾腔。
多无寄主专一性。
如盾腹虫(\textit{Aspidogaster spp.})。

\subsubsection{复殖亚纲(Digenea)}
体内寄生。
一般幼虫期寄生软体动物,成虫期寄生脊椎动物。
或寄生于肝、胆管,称为肝吸虫,如肝片吸虫(\textit{Fasciola hepatica});
或寄生于肠,称为肠吸虫,如布氏姜片虫(\textit{Fasciolopsis buski});
或寄生于血液,称为血吸虫,如日本血吸虫(\textit{Schistosoma japonicum})。

\subsection{绦虫纲(Cestoida)}
全部营寄生生活。
虫体分节,由头节(scolex)、幼节(neck)、成节(mature proglottid)和孕节(gravid proglottid)组成带状链体。
其中头节为吸附器官,幼节负责产生新的节片,孕节内含虫卵。
每个节片均有发达的生殖系统。

\subsubsection{单节亚纲(Cestodaria)}
缺乏头节、成节、孕节。
虫体仅有雌雄同体的生殖系统。
主要寄生鲨鱼、鳐、原始硬骨鱼的消化道或体腔,中间寄主为水生无脊椎动物。
如旋缘绦虫(\textit{Gyrocotyle spp.})。

\subsubsection{多节亚纲(Cestoda)}
又称真绦虫。
成虫寄生脊椎动物消化道。
如猪带绦虫(\textit{Taenia solium})、牛带绦虫(\textit{Taenia saginatus})、细粒棘球绦虫(\textit{Echinococcus granulosus})。

\end{document}