\documentclass[11pt]{article}

\usepackage[UTF8]{ctex} % for Chinese 

\usepackage{setspace}
\usepackage[colorlinks,linkcolor=blue,anchorcolor=red,citecolor=black]{hyperref}
\usepackage{lineno}
\usepackage{booktabs}
\usepackage{graphicx}
\usepackage{float}
\usepackage{floatrow}
\usepackage{subfigure}
\usepackage{caption}
\usepackage{subcaption}
\usepackage{geometry}
\usepackage{multirow}
\usepackage{longtable}
\usepackage{lscape}
\usepackage{booktabs}
\usepackage{natbib}
\usepackage{natbibspacing}
\usepackage[toc,page]{appendix}
\usepackage{makecell}

\title{线虫动物门(Nematoda)}
\date{}

\linespread{1.5}
\geometry{left=2cm,right=2cm,top=2cm,bottom=2cm}

\begin{document}

  \maketitle

  \linenumbers
  
\section{一般特征}
线虫虫体呈圆柱形,体表无纤毛,头部不明显,体前端有辐射对称的口。
虫体表面有四条由下皮层向内加厚形成的线,在背面者为背线(dorsal cord),在腹者为腹线(ventral cord),在两侧者为侧线(lateral cord)。
部分物种体表有环纹,出现假分节现象。
线虫体表被角质膜(cuticle),系上皮细胞分泌物,起保护作用。
然而,角质膜限制虫体的生长,故线虫生长过程中需蜕皮(ecdysis)。

\newline

线虫体壁自外向内分别为角质膜、上皮细胞和源自中胚层的肌肉,但线虫只有纵肌,缺乏环肌。
体壁内为源自囊胚腔的假体腔(pseudocol)。
假体腔只有外壁源自中胚层,内壁为源自内胚层的肠上皮。
假体腔内部充满体腔液,司循环和支撑。

\newline

线虫有完整的消化道。
其口源于原肠胚胚孔,即原口。
与胚孔相对的另一侧开口发育为肛门。
线虫消化道从口至肛门依次为前肠、中肠、后肠。
前肠为原口处外胚层内陷形成,内壁有角质层,分化为口、口腔和咽。
中肠源自内胚层,司消化吸收。
后肠由外胚层内陷形成,内壁有角质层。

\newline

线虫的排泄器官起源于外胚层,是一种独特的原肾管结构,分为腺型(glandular type)和管型(tubular type)。
腺型排泄器官仅一到二个原肾细胞,位于咽后端腹面,开口于腹线。
管型排泄器官由一个原肾细胞特化形成,包括侧线下的两条纵排泄管和二管之间的横管,整个细胞呈“H”形,开口于腹线。
此外,代谢废物亦可由体壁和消化管排出。

\newline

线虫咽的周围有围咽神经环(circumenteric ring),向前、向后各分出六条神经。
向前的神经分布到体前的感觉器官。
向后的神经中,一条背神经,一条腹神经,两对侧神经。
侧神经离开围咽神经环后很快合并为一对。
这些神经中,腹神经最为发达。

\newline

线虫的感觉器官主要分布于头尾。
头部有唇、乳突、感觉毛和头感器。
唇和乳突为角质突起。
感觉毛实为特化的纤毛,司触觉。
头感器是体表的内陷物,司化学感受。
水生种类咽两侧有一对眼点,司视觉。
线虫尾部的感受器官为尾感器,开口于尾端两侧。

\newline

线虫动物大多雌雄异体异型,少数种类为雌雄同体或无雄性。
线虫生殖腺为盲管。
雄虫大多有一个精巢,精巢后接输精管,再向后为肌肉发达的射精管(ejaculatory duct),最终与后肠相接于泄殖腔(cloaca)。
射精管周围有前列腺(prostatic gland)。
大多数线虫雄性泄殖腔向外伸出两个囊,内有角质的交合刺(spicule)。
在交配时,交合刺伸出,撑开雌虫阴门。
雌虫大多有两个卵巢,后接输卵管和子宫。
两个子宫后端相接,经肌肉质阴道,开口于虫体中部腹线,形成阴门。
  
\section{线虫动物的分类}
\subsection{无尾感器纲(Aphasmida)}
虫体尾端无尾感器,排泄器官腺型。
海产线虫全部属此纲。
  
\section{尾感器纲(Phasmida)}
虫体尾端有一对尾感器,排泄器官管型,为陆生或淡水生。
  
\end{document}