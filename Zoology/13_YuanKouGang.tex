\documentclass[11pt]{article}

\usepackage[UTF8]{ctex} % for Chinese 

\usepackage{setspace}
\usepackage[colorlinks,linkcolor=blue,anchorcolor=red,citecolor=black]{hyperref}
\usepackage{lineno}
\usepackage{booktabs}
\usepackage{graphicx}
\usepackage{float}
\usepackage{floatrow}
\usepackage{subfigure}
\usepackage{caption}
\usepackage{subcaption}
\usepackage{geometry}
\usepackage{multirow}
\usepackage{longtable}
\usepackage{lscape}
\usepackage{booktabs}
\usepackage{natbib}
\usepackage{natbibspacing}
\usepackage[toc,page]{appendix}
\usepackage{makecell}

\title{圆口纲(Cyclostomata)}
\date{}

\linespread{1.5}
\geometry{left=2cm,right=2cm,top=2cm,bottom=2cm}

\begin{document}

  \maketitle

  \linenumbers
  
\section{一般特征}
体呈鳗鱼形。
头部背侧中央有一短管状单孔鼻(nostril),其后方皮下有松果眼(pineal eye),松果眼下有顶体(parietal body)。
松果眼和顶体系退化的感光器官。
头两侧各有一眼,无眼睑(eye lid)。
眼后有鳃裂。
体表光滑无鳞,单细胞腺发达。
口呈吸盘状,内侧生角质齿,无颌,营寄生或半寄生生活。
无成对附肢,有背鳍两个、尾鳍一个。
肛门位于尾的基部,其后为泄殖孔。

\newline

终生具有脊索。
咽鳃裂经鳃囊、鳃裂连通体外,鳃囊司呼吸。
有软骨组成的原始头骨,背神经管从前到后依次分化为大脑、间脑、中脑、小脑、延脑,依次排列于同一平面。
听觉、视觉和嗅觉器官集中于头部。
血液循环为单循环,动脉血和静脉血未分开。
心脏包括静脉窦和一心房一心室。

\newline

消化道中,胃不明显,肠内有黏膜褶和螺旋瓣膜,以增加表面积。
口腔后有一对唾液腺,以细管通舌下。
有集中的肾,经输尿管通入泄殖窦(urogenital sinus),再经泄殖孔到体外。
生殖腺单个,无生殖导管。
生殖腺表面破裂,释放生殖细胞于泄殖窦,再经泄殖孔到体外,营体外受精。
  
\section{圆口纲的分类}
\subsection{七鳃鳗目(Petromyzoniformes)}
多营半寄生生活,以口吸附于宿主体表,以角质齿挫破表皮,食其血肉。
雌雄异体,变态发育。
如海七鳃鳗(\textit{Petromyrzon marinus})、东北七鳃鳗(\textit{Lampetra morii})。

\subsection{盲鳗目(Myxiniformes)}
营寄生生活,钻入宿主体内,食其血肉。
无背鳍,口位于身体最前方,无口漏斗,有四对口缘触须。
眼退化,雌雄同体,无变态发育。
如盲鳗(\textit{Myxine glutinosa})、杨式黏盲鳗(\textit{Paramyxine yangi})。
\end{document}