\documentclass[11pt]{article}

\usepackage[UTF8]{ctex} % for Chinese 

\usepackage{setspace}
\usepackage[colorlinks,linkcolor=blue,anchorcolor=red,citecolor=black]{hyperref}
\usepackage{lineno}
\usepackage{booktabs}
\usepackage{graphicx}
\usepackage{float}
\usepackage{floatrow}
\usepackage{subfigure}
\usepackage{caption}
\usepackage{subcaption}
\usepackage{geometry}
\usepackage{multirow}
\usepackage{longtable}
\usepackage{lscape}
\usepackage{booktabs}
\usepackage{natbib}
\usepackage{natbibspacing}
\usepackage[toc,page]{appendix}
\usepackage{makecell}

\title{节肢动物门(Arthropoda)}
\date{}

\linespread{1.5}
\geometry{left=2cm,right=2cm,top=2cm,bottom=2cm}

\begin{document}

  \maketitle

  \linenumbers
节肢动物为动物界第一大类群,亦是原口动物中最为进化的类群。
节肢动物虫体异律分节(heteronomous segmentation)。
各体节结构和功能发生分化,节数因种类而不同。
相似的体节又多组合在一起,形成虫体不同部分。
总之,体节的分化和组合,提高了节肢动物对不同生境的适应能力。
节肢动物躯体第一节称为顶节(acron),最后一节称为尾节(telson),其腹部或末端有肛门。
顶节和尾节非胚胎分节形成的,故并不是真正的体节。
二者之间才是真正意义上的体节。

\newline

节肢动物体被坚硬厚实的外骨骼,其主要成分为几丁质(chitin)和蛋白质。
组成外骨骼的物质是其内层的一层上皮细胞分泌的。
外骨骼一旦形成,遂不能生长。
因此,节肢动物有蜕皮(moulting)现象。
蜕皮时,上皮细胞脱离旧外骨骼,产生新外骨骼,并分泌蜕皮液(moulting fluid)于二者之间。
蜕皮液中有几丁质酶和蛋白酶,可分解旧外骨骼。
旧外骨骼逐渐变薄并破裂。虫体由裂缝处钻出并膨胀。
柔软的新外骨骼随之扩张,完成虫体的生长。
过一段时间后,新外骨骼变硬,生长停止。
外骨骼主要司支持和保护,亦是肌肉的附着点。
此外,外骨骼有效地防止水分的丢失,其上的各种突起可司感觉。

\newline

节肢动物有发达的附肢。
附肢通过关节于本体相连,其内有发达的肌肉。
附肢本身亦分节,故称节肢(arthropodium)。
节肢往往具有不同的结构,以执行不同的功能,如形成口器、触角、外生殖器、适于不同运动方式的足、呼吸器官等。

\newline

节肢动物肌肉系统发达。
肌纤维集合为肌肉束,其两端附着于外骨骼,以调整和放大肌肉运动,增强效能。
节肢动物的肌肉束多成对排列,相互拮抗,使躯干或附肢得以向不同方位运动。
每个体节的躯干肌包括一对背纵肌和一对腹纵肌,前者收缩使躯体向上弯曲,后者收缩使躯体向下弯曲。
每只附肢一般有三对附肢肌,使之得以前后、上下、内外运动。

\newline

节肢动物的真体腔退化,形成生殖管腔、排泄管和围心腔。
在胚胎发育过程中,围心腔壁消失,围心腔和假体腔混合,形成混合体腔(mixocoel)。
混合体腔内充满血液,故称血腔(haemocoel)。
节肢动物的心脏呈管状,向前发出一条短动脉。
短动脉伸入头部,末端开口。
血液经心脏和短动脉,进入血腔。
身体各部分的组织直接浸沐于血液中。
而后血液经心脏上的心孔回到心脏。
此外,肠道直接浸润在血液中,养分可经过肠壁直接进入血液。
总之,节肢动物的循环系统为开管式。

\newline

节肢动物呼吸器官多样。
水生种类有体壁外突形成的鳃,一些小型水生种类通过体表进行气体交换。
陆生种类则以气管(trachea)呼吸。
气管是体壁内陷形成的,其外端通过气门与外界相通,其内端不断分支,伸入组织间,直接和细胞接触。
由此,陆生节肢动物组织可直接与外界进行气体交换。
气管是动物界最为高效的呼吸器官。
除鳃和气管外,节肢动物的呼吸器官还包括书肺、书鳃、足鳃、气管鳃等。

\newline

节肢动物运动能力强,代谢兴旺,排泄器官发达。
节肢动物的排泄器官主要有两种类型。
其一是与后肾管同源的绿腺(green gland),包括腺基部和膀胱。
含氮废物通过渗透进入腺基部,而后进入膀胱部,经排泄孔排出。
另一种为马氏管(Malpighian tube),系从中肠和后肠之间发出的多条细管。
马氏管浸沐于血腔,可收集血液中的代谢废物,使之通过后肠,随粪便一同排出。

\newline

节肢动物有发达的捕食、摄食、碎食结构。
部分种类有发达的中肠突出物,司储存养料。
还有部分种类肠道周围有大量脂肪细胞。
大部分节肢动物有六个直肠垫(rectal papillae),可从食物残渣中回收水分。
这些结构于节肢动物强大的运动能力和较高的代谢效率相适应。

\newline

节肢动物运动能力的增强伴随着神经系统和感觉器官的复杂化。
节肢动物的神经系统呈链状结构,但部分前后相邻的神经节常愈合为大的神经团。
其感觉器官相当复杂,司平衡、触觉、视觉、味觉、嗅觉、听觉的器官俱全。
眼有单眼和复眼之分。
雌雄异体异形,常为体内受精,部分种类可营孤雌生殖。

\section{螯肢亚门(Chelicerata)}
躯体分为头胸部和腹部,无触角。
头胸部每个体节各有一对足,另有一对螯肢和一对脚须,分别位于口器前后。
螯肢一般分二到三节,末节有爪。
大部分种类的螯肢有毒腺,司捕食和防御。
脚须可司感觉,亦可演化为交配器,或演化为钳。
腹部附肢或退化,或演化为吐丝板(蜘蛛)或鳃(鲎)。

\newline

螯肢动物大多以动植物汁液为食,鲜有摄取固体食物者。
其消化道较窄。
排泄器官为绿腺和马氏管。
循环系统为开管式,部分小型种类无循环系统。
神经系统呈链状,体表有司感觉的刚毛和其它化学感受器。
水生种类有复眼和单眼,陆生种类一般仅有单眼。
雌雄异体,生殖腺位于腹部消化管腹侧,其左右各伸出一生殖管,汇合后经生殖孔开口于腹部腹侧。
有交配行为,体内或体外受精。
水生种类间接发育,陆生种类直接发育。

\newline

螯肢动物呼吸器官多样。
水生种类腹部附肢演化为鳃。
陆生种类和呼吸器官包括书肺和气管,二者可并存。
书肺是虫体腹部腹侧体表内陷形成的,呈书页状。
页间空腔汇合后经书肺孔与外界连通,页内与血腔相连。
螯肢动物的气管起源于书肺,非体壁内陷形成。
一些小型种类由体表呼吸。

\subsection{肢口纲(Merostomata)}
大多为化石种。
今存四种:东方鲎(\textit{Tachypleus tridentatus})、南方鲎(\textit{Tachypleus gigas})、圆尾鲎(\textit{Carcinoscorpius rotundicauda})、美洲鲎(\textit{Limulus polyphemus})。
浅海生。

\newline

头胸部背甲宽阔,呈马蹄形,上有单眼、复眼各一对。
六对附肢。
第一对为螯肢,三节,短小。
第二至第五对为步足,六节,末端钳状。
第六对分七节,末端特化,适挖掘、步行。
雄性第二对附肢末端钩状。
第二至第六对附肢基部列于口两侧。
第五对步足之后有一唇状瓣(chilarium)。

\newline

腹部背面坚硬,六角形,两侧有缺刻和短刺。尾部有长大尾剑(telson)。
腹部不分节,第一对附肢愈合形成生殖厣(genital operculum)遮盖生殖孔。
其后五对附肢内缘愈合,彼此重叠似书叶,是为书鳃(book gill)。

\subsection{蛛形纲(Arachnida)}
头胸部除螯肢和脚须,另有四对步足。
腹部无附肢,或形成书肺。

\subsubsection{蜱螨目(Acarina)}
包括螨(mites)、蜱(ticks)两类。
头胸部和腹部合并。
以气管或体壁呼吸。
螯肢和脚须组成颚体(gnathosoma),其周围有围头沟。
无脑,无眼。
如人疥螨(\textit{Sarcoptes scabie})、柑桔叶螨(\textit{Panonychus citri})、全沟蜱(\textit{Ixodes persulcatus})。

\subsubsection{鞭蛛目(Amblypygi)}
体平扁,头胸部宽短,有背甲。
螯肢两节,内无毒腺。
中眼两个,两组侧眼各三个。
腹部十二节。第一节成柄。

\subsubsection{蜘蛛目(Aranceae)}
头胸部和腹部之间有细柄。
头胸部有隆起背甲,前方有单眼。
螯肢两节,基部短粗,末端有爪,爪尖有毒腺开口。
脚须六节,细长,基部位于口两侧,其内缘有刚毛、细齿以撕碎猎物。
雄性脚须末端膨大为交配器。
腹部膨大,多不分节。
腹面前部正中有附肢形成的生殖板,盖住下方生殖孔。
生至板两侧为裂缝状书肺孔。
腹面后部正中有气门,气门后有二至三对纺织突,源自腹部附肢。
腹腔后部有丝腺,开口于纺织突,以吐丝结网。
常见品种如大腹园蛛(\textit{Araneus ventricosus})、拉土蛛(\textit{Lactouchia})、水蛛(\textit{Argyronecta aquatica})、蝇虎(\textit{Plexippus})、黑寡妇(\textit{Latrodectus mactaus})、穴居狼蛛(\textit{Lycosa singoriensis})。

\subsubsection{盲蛛目(Opiliones)}
头胸部与腹部连接处宽阔,腹部分节不明显。
步足细长。
背甲中部隆起,其两侧各有一眼,其前缘有一对臭腺体。
如长踦盲蛛(\textit{Phalangium})。

\subsubsection{脚须目(Palpigradi)}
头胸部不分节,腹部最末三节变细,后有细长分节鞭尾。
螯肢短,脚须发达。
第一步足细长,向前伸出,司感觉。
如鞭蝎(\textit{Mastigoprocutus})。

\subsubsection{伪蝎目(Pseudoscorpionids)}
似蝎,脚须发达,无尾刺,无前腹后腹之分。
如螯蝎(\textit{Chelifer})。

\subsubsection{节腹目(Ricinulei)}
体粗短,表膜厚而有纹。
背甲方形,前缘为可动短头盖。
螯肢成钳。
脚须短小,成小钳。
腹部前方成柄,后端突起。

\subsubsection{裂盾目(Schizomida)}
体柔软。
腹部第一节成柄。
背板、腹板相连。
腹部末端有短鞭。
无眼。
螯肢两节,钳状。
脚须细长,六节,司感觉。
如裂盾蝎(\textit{Schizomus})。

\subsubsection{蝎目(Scorpionida)}
头胸部短小,背面中央有一对大单眼,两侧有小单眼。
螯肢小而向前。
脚须发达,末端钳状。
腹部分前腹和后腹。
前腹宽短,七节,第一节腹面有生殖孔,第二节腹面有司感觉的栉状器(pectines),第三至六节各有一对缝状呼吸孔。
后腹狭长,五节,可向背面弯曲。末节为毒刺。
以书肺呼吸。
常见品种如东亚钳蝎(\textit{Buthus martensi})。

\subsubsection{避日目(Solifugae)}
头胸部六节,前三节成头,后三节独立。
腹部膨大。
螯肢大,钳状。
脚须成步足状。
气管呼吸。
如避日蛛(\textit{Galeodes})。

\subsubsection{有鞭目(Uropygi)}
眼分三组排列,螯肢两节。
有腹柄,腹板,背板。
体后有三十至四十节短尾鞭。

\subsection{海蜘蛛纲(Pycnogonida)}
海产,形似蜘蛛。
头前有较大的吻,其顶端有三角形的口。
吻基部至第一对步足基部为头,系三个体节愈合而成。
头背面有突起,其上有四个单眼。
螯肢位于吻基部上方。
触肢位于螯肢后外侧,多节,顶端有感觉毛。
抱卵肢在触肢后下,顶端有爪,末四节卷曲,内缘生锯齿。
头后方体节生细长步足,八节,顶端有爪。
腹部退化,不分节,无附肢,末端为肛门。

\newline

口周围有带刺的几丁质颚。
食管中有筛器,滤除大颗粒。
中肠向步足发出盲管。
开管循环,血腔分为背腹两部分。
通过体表或肠壁排泄、交换气体。
雌雄异体,生殖导管进入步足,开口于步足基部腹面。
体外受精。

\section{甲壳亚门(Crustacea)}
甲壳动物体节较多,其中头节和胸节愈合为头胸部。
原第六体节背面上皮层褶皱向后延申,包裹头胸部的背侧,其外层上皮细胞分泌的外骨骼形成头胸甲(carapace)。
腹部有六个体节,腹部末端另有尾节。
虾类腹部发达,蟹类腹部退化并向前曲折,俗称蟹脐。
甲壳动物头部第一节有一对复眼。
除头部第一节和腹部最后一节外,其余各体节一般均有一对附肢。
甲壳动物的附肢结构、功能多样,可司感觉、摄食、咀嚼、运动、生殖、呼吸等。

\newline

小型甲壳动物的消化系统比较简单,而大型物种有复杂的消化系统。
一般来说,甲壳动物的消化道分为前肠、中肠、后肠。
前肠包括短小的食道和膨大的胃。
部分物质胃内面有角质膜,可研磨食物。
中肠长短不一,有管道与发达的消化腺相接。
后肠直通肛门。

\newline

大型甲壳动物以位于头胸甲下两侧的鳃呼吸,虾类附肢基部突起形成的足鳃(podobranchia)亦司呼吸。
亦有发达的肌肉系统和完整的开管式循环系统。
小型种类多无呼吸器官和循环系统,或者循环系统不完整,只有心脏,无血管。
甲壳动物的排泄器官为绿腺和鳃。

\newline

低等甲壳动物的神经系统呈梯形。
高等种类神经节愈合明显,有脑和食管下神经节。
感觉器官发达,有眼、嗅毛、触毛、平衡囊等。
其眼自外到内分别为角膜、晶状体、视小网膜、色素细胞和视神经。
平衡囊为高等甲壳动物特有,虾类尤其发达,一般为第一对触角基部凹陷形成的囊,囊低有触毛。
触毛基部与双极感觉细胞的一极相连。
这些感觉细胞另一极束集入脑。
触毛顶部黏着大量细小的平衡石(statolith)。
平衡石可为自身分泌物,亦可为外界泥沙。
动物运动时,触毛承受来自平衡石的压力改变,这种刺激经双极感觉细胞入脑,使动物得以感知自身体位。

\newline

甲壳动物具有内分泌系统(endocrine system),以调节生殖、发育、蜕皮等生理活动。
一般雌雄异体,生殖腺经生殖管道,开口于生殖孔。
甲壳动物一般营两性生殖。
交配时,雄性排出由输精管上皮细胞分泌的物质包裹精子形成的精荚(spermatophore)。
精荚经生殖孔进入雌性体内。
卵子成熟后,精荚破裂,释放精子,完成体内受精。
甲壳动物多有抱卵行为,即产下的受精卵黏附于雌性附肢,直至孵化。
甲壳动物为间接发育,幼体形态多样。

\subsection{头虾纲(Cephalocarida)}
小型海生品种。
体长,略呈桶状。
头六节,胸九节,腹十一节。
头与第一胸节愈合,头胸背甲盾形扁平。
其余胸节宽短,侧甲明显。
各个胸节有附肢一对。
仅短足目(Brachypoda)一目,如哈琴头虾(\textit{Hutchinsoniella})。

\subsection{鳃足纲(Branchiopoda)}
多淡水生小型品种。
头胸背整片背甲,或两片介壳,或无甲。
复眼相近,甚至部分愈合。
两复眼之间常有一单眼。
胸部附肢扁平,可调节渗透压。
腹部多无附肢,体末节多有尾叉。

\subsubsection{背甲目(Notostraca)}
头胸背整片盾形背甲。
躯干后部细长,圆筒形,末节有一对细长柱状尾叉。
背甲后缘中部内凹,前端背面一对无柄复眼,一单眼。
第一触角细小,第二触角消失。
躯干前十一节各生一对附肢,以后各节多对附肢。
附肢司运动、呼吸。
如蟹形鲎虫(\textit{Triops cancriformis})。

\subsubsection{介甲目(Conchosraca)}
躯干包裹于两片介壳内,壳外有同心环纹。
壳顶内面有韧带连接躯干。
体前上方有肌肉控制介壳开闭。
头顶部一对复眼,无柄。
第一触角细小不分节。
第二触角发达,司运动。
如扁豆渔乡蚌虫(\textit{Limnadia lenticularis})。

\subsubsection{枝角目(Cladocera)}
体短,侧扁,分节不明显。
躯干包裹于两片介壳内。
尾缘背侧有针状壳刺。
第一触角细小。
第二触角发达,司运动。
腹部短小,尾节爪状。
胸肢司滤食、呼吸。
体背面有囊状心脏,有心孔。
生殖腺位于肠两侧。
常孤雌生殖。
环境不利时营有性生殖。
如水蚤(\textit{Daphnia})。

\subsubsection{无甲目(Anostraca)}
体长圆筒形。
头部短小,前端中央一单眼,两侧各生有柄复眼。
额下一片上唇,向腹面延伸,覆盖口。
雄性第二触角发达。
生殖孔后的腹节无跗肢。
胸足叶状,尾叉不分节。
多淡水生,如南京丰年虫(\textit{Chirocephalus nankinensis})。

\subsection{桨足纲(Remipedia)}
躯干细长,三十二体节,每节生桨状附肢,无眼。
生高盐地下水体或深海,仅泳足目(Nectiopoda)。

\subsection{须虾纲(Mystacocarida)}
体小细长,后部稍宽。
额向前突出。
头背面前侧有横沟。
尾节以前各节背面两侧各有一斜沟。
第一胸节有颚足。
腹部各节无附肢。
第一触角自额下方伸出。
第二触角自额前侧伸出。
仅须虾目(Mystacocaridida)。

\subsection{蔓足纲(Cirripedia)}
头胸甲完全包裹躯干和附肢。
头胸部发达,腹部退化,常有尾叉。
第一触角在幼虫期司固着,成体时退化。
第二触角常消失。
胸肢六对。
成体仅有单眼。
常雌雄同体。

\subsubsection{囊胸目(Ascothoracica)}
小触角和腹部保留。
胸肢蔓状,寄生珊瑚或棘皮动物。
如合囊虱(\textit{Synagoga})。

\subsubsection{围胸目(Thoracica)}
体节消失。
外背石灰质板,板内有皮肤形成短外套。
有柄类虫体分为头部和柄部。
颚发达,胸肢六对,固着生活。
如藤壶(\textit{Balanus})。

\subsubsection{尖胸目(Acrothoracica)}
触角和腹部退化。
六对蔓状胸肢,体表裸露,有几丁质附着盘。
在钙质物上钻穴生活。
如\textit{Trypetesa}。

\subsubsection{根头目(Rhizocephala)}
躯干退化,外表柔软无壳。
无附肢,无体节。
体内器官仅生殖腺和退化的神经器官。
以根状结构寄生宿主。
如蟹奴(\textit{Sacculina})。

\subsection{桡足纲(Copepoda)}
六个头节和第一胸节愈合为头胸部,背面一个单眼,腹面六对附肢。
其余胸节各生一对胸足。
腹部短小。
胸性第一腹节为生殖节。
有一对尾叉。
雌雄异体异形。

\subsubsection{哲水蚤目(Calanoida)}
圆筒形,头胸部宽大,腹部狭小。
雌性腹部四节,第一、第二节可能愈合。
雄性腹部五节。
腹部第一节为生殖节,雌性腹面有生殖突起和一对生殖孔,雄性仅左侧一个生殖孔。
如近镖水蚤(\textit{Tropodiaptomus})。

\subsubsection{猛水蚤目(Harpacticoida)}
第六胸节末为活动关节。
雄性第一触角四至六节膨大,可司执握。
雌性腹部四至五节,雄性五节。
有尾叉,胸足同形。
多海生底栖。
如猛水蚤(\textit{Harpacticus})。

\subsubsection{剑水蚤目(Cyclopoida)}
头胸部粗大卵圆形,腹部细小。
第一胸节多与头部愈合,最末胸节常与第一腹节愈合。
第一触角特化为执握肢。
第二触角幼虫期发达,成体后退化。
第四、五胸节之间为活动关节。
雌性腹部四节,雄性五节。
生殖节一对附肢,一对生殖孔。
如拟剑水蚤(\textit{Paracyclops})。

\subsubsection{鱼虱目(Caligoida)}
头胸部宽扁,卵圆形,前端有额板,板上常有一对吸附器官。
前三胸节与头部愈合,第三、第四胸节间为活动关节。
腹部窄小,尾叉形状多样。
寄生。
如鱼虱(\textit{Caligus euthynus})。

\subsubsection{颚虱目(Lernaeopodoida)}
头胸部长圆柱形,头端略粗。
躯干扁平,后端略宽。
第一触角小,第二触角双肢型,粗壮,位于口管两侧。
口管圆锥形,开口处有一圈刚毛。
大颚在口管内,小颚在口管基部腹侧。
第一颚足司吸附。
雄性小,附于雌性。
寄生鱼类。
如柱颚虱(\textit{Clavella adunca})。

\subsubsection{怪水蚤目(Monstrilloida)}
圆筒形,前体宽。
幼体寄生,成体浮游生活。
如赫耳兰怪水蚤(\textit{Monstrilla heloglandica})。

\subsection{鳃尾纲(Branchiura)}
平扁,略呈圆形。
头与第一胸节愈合。
背侧头胸甲向侧后扩展为侧叶。
第五、六胸节与腹部愈合。
腹部小,后端中央凹入。
尾叉小。
第一触角基部粗壮,司执握。
大颚和第一小颚包于上下唇合成短口管内,用于刺吸。
第二小颚成圆形吸盘。
复眼一对,大而能动。
多寄生鱼类。
仅鱼虱目(Argulidea)。

\subsection{介形纲(Ostracoda)}
头胸甲由两瓣钙质介壳构成,包裹虫体。
介壳表面有突起或花纹,背面相互铰合,有闭壳肌。
体末向腹弯曲,有尾叉。
第一触角单肢,第二触角双肢。

\subsubsection{壮肢目(Myodocopida)}
触角发达,有复眼。
胸部两对单肢型附肢。
介壳前端凹陷,触角自此伸出。
如海萤(\textit{Cypridina})。

\subsubsection{海介虫目(Halocyprida)}
介壳前端无凹陷。
仅头部五对附肢。
第二触角内外肢均发达。
无复眼,尾叉片状,边缘有爪刺。
海产,底栖。

\subsubsection{简肢目(Platycopida)}
介壳前端无凹陷。
头部五对附肢,一对躯干肢。
第二触角内外肢均发达。
无复眼,尾叉叶状,边缘有刚毛。
海产。

\subsubsection{尾肢目(Podocopida)}
大触角单肢型,两对胸足。

\subsection{软甲纲(Malacostraca)}
体节数保守:头六节,胸八节,腹七或八节,尾一节。
胸腹各节均有附肢一对。
背甲覆盖胸节不等。
雌雄异体。
雄性生殖孔在第六胸节,雌性生殖孔在第八胸节。
大多水生。

\subsubsection{叶虾目(Nebeliacea)}
胸部及部分腹部包有两瓣介形甲,之间有壁壳肌。
胸部附肢叶状,抱卵。
腹部前四对附肢为游泳足,第五、六附肢小,第七附肢消失。
多生浅海。

\subsubsection{口足目(Stomatopoda)}
体扁平,背甲小。
第一二胸节愈合,三四胸节退化。
腹部及尾节发达,背部生棘。
第一触角三叉,第二触角鳞片宽大。
前五对胸足生螯,第二胸足发达。
后三对胸足细长无螯。
腹部发达有鳃。
多穴居。
如虾蛄(\textit{Squilla})。

\subsubsection{地虾目(Bathynellacea)}
八个胸节全部游离,最末腹节与尾节愈合,有尾叉。
生高盐地下水体或深海,仅泳足目(Nectiopoda)。
如拟地虾(\textit{Parabathynella})。

\subsubsection{山虾目(Anaspidacea)}
第一胸节与头部愈合。
六腹节游离,尾节游离,无尾叉。
淡水生。
如塔斯马尼亚山虾(\textit{Anaspides tasmaniae})。

\subsubsection{温泉虾目(Thermosbaenacea)}
体圆筒形,略平扁。
头与第一至第三胸节愈合。
腹末节与尾节愈合,无眼。
第一触角粗大,第二触角细小。
胸足双肢,内肢细小,五节,末端爪状。
外肢略呈鞭状,可游泳,自第一对相后节数逐渐减少。
腹肢仅第一、二对,细小单肢,不分节。
尾肢双肢,内肢细小不分节,外肢粗大片状两节。
如奇异温泉虾(\textit{Tethysbaena texana})。

\subsubsection{洞虾目(Spelaeogriphacea)}
仅鳞眼洞虾(\textit{Spelaeogriphus lepidops})。
体筒状,稍平扁。
头与第一胸节愈合。
头胸甲小,两侧形成鳃室,后端覆盖第二胸节。
七个自由胸节,腹部长。
头前端有眼叶。
第一触角三节柄,双鞭。
第二触角外肢小鳞片状,内肢鞭状。
第一胸足颚状,司呼吸。
其余七对胸足六节,步足。
腹肢前四对双肢多节,游泳足。
第五腹肢小棒状。
第六腹肢外肢两节,内肢一节,与尾节构成尾扇。
雌性胸部五对复卵板,构成卵囊,卵在其内发育。

\subsubsection{糠虾目(Mysidacea)}
头胸甲发达,向前突出为额刺,背面有沟,后端背面前凸。
末四胸节游离。
腹七节,无侧甲。
复眼有柄。
第一触角柄三节,双鞭。
第二触角生鳞。
胸肢外肢司游泳,第一或前二对内肢成为颚足。
如何氏糠虾(\textit{Holmesiella})。

\subsubsection{涟虫目(Cumacea)}
头胸粗大,腹部纤细。
胸前三或四节与头愈合。
腹六或七节。
头胸甲前缘、侧缘与头部腹面前缘,胸部腹甲侧缘以及附肢基部完全愈合,形成封闭鳃室。
头胸甲向前突出。
复眼无柄,自有胸节宽短,腹节能向腹面、背面弯曲。
第一触角短小,第二触角发达。
腹肢小,双肢。
尾肢窄长。
海产,底栖。
如丽涟虫(\textit{Lampropus})。

\subsubsection{端足目(Amphiopoda)}
侧扁。
头与第一或第一、二胸节愈合,无头胸甲。
腹部末端二或三节愈合,尾节明显。
复眼无柄。
第一触角外鞭长,第二触角单鞭。
胸肢八对,单肢。
第一胸肢为颚足,底节愈合。
多海生。
如长尾钩虾(\textit{Melita})。

\subsubsection{等足目(Isopoda)}
平扁,头短小盾状,与第一或第一、二胸节愈合。
无头胸甲,腹部短,最末腹节与尾节愈合。
胸肢第一对为颚足,其余七对司游泳、呼吸。
无眼柄。
第一触角小,单肢。
如鼠妇(\textit{Armadillidium})、海蟑螂(\textit{Ligia})。

\subsubsection{原足目(Tanaidacea)}
长筒状,平扁。
头与第一或第一、二胸节愈合,头胸甲小。
胸节大。
腹部短小,六节,末节与尾节愈合。
复眼有柄,不可转动。
第一胸肢为颚足,基部有片状上肢伸入鳃室,司呼吸。
第二胸肢为螯,粗大。

\subsubsection{磷虾目(Euphausiacea)}
小虾状,鳃裸露。
八对胸肢形态相同,双肢,无颚足。
发蓝绿色闪光。
如华丽磷虾(\textit{Euphausia superba})。

\subsubsection{异虾目(Amphionidacea)}
仅\textit{Amphionides reynaudii}。
头胸甲膜质,大而薄,包裹胸肢。
胸肢外肢短小,第一胸肢为颚足。
雄性胸肢七对,雌性六对。
腹肢五对。

\subsubsection{十足目(Decapoda)}
腹部延长成虾形,或缩短扁圆成蟹形。
头与八个胸节全部愈合。
头胸甲发达,包裹鳃。
虾类头胸甲前伸为额角,蟹类头胸甲侧缘折向腹面形成鳃室。
前三对胸肢为颚足,后五对为步足。
复眼有柄。
第一触角柄三节,基节基部背面有平衡囊。
如对虾(\textit{Penaeus orientalis})、螯虾(\textit{Cambarus})、河虾(\textit{Macrobrachium nipponensis})、三疣梭子蟹(\textit{Portunus trituberculatus})、寄居蟹(\textit{Diogenes})、中华绒螯蟹(\textit{Eriocheir sinensis})。

\section{六足亚门(Hexapoda)}
六足动物有二十个体节,分为头、胸、腹三部分。
头部为顶节和前六个体节愈合形成,外骨骼愈合为头壳(head capsul)。
第一、第三体节附肢退化,第二体节的附肢形成形态功能各异的触角。
头部第四、五、六体节的附肢参与形成口器,包括一对大颚、一对小颚和一片下唇。
此外,口器还包括上唇和舌。
随着食性不同,六足动物口器形态多样。

\newline

胸部有三个体节,称为前、中、后胸节。
各胸节相互愈合,不能自由活动,但彼此界限可辨。
其上的三对附肢演变为三对足。
足的形态、功能多样。此外,中、后胸节分别有一对前翅和一对后翅。
翅源于胸节左右侧面的扁平褶突,称为翅芽(wing pad)。
在发育过程中,翅芽上下两侧的上皮逐渐愈合、退化,形成仅有两层角质膜的翅。
翅内有高度角质化的翅脉(vein),其内中空,起支持作用,亦是气管、血管、神经出入翅的通路。
不同物种的翅形态功能不一,可司飞行、保护、平衡等。

\newline

腹部包括十一个体节和尾节,但成虫尾节多退化。
腹部体节中,仅末三节常彼此愈合,余者彼此游离。
第十一节有附肢发育形成的尾须,第八、九节常有附肢形成外生殖器(genitalia)。
第一至第八腹节两侧各有一个气门,内通气管。

\newline

六足动物躯体最外层为体壁(integument),仅一层皮细胞,源于外胚层。
皮细胞内层为血细胞分泌的底膜,外层为皮细胞分泌的表皮层,包括外骨骼和外骨骼表明的蜡层。
蜡层系皮细胞在蜕皮前分泌,而后扩散到虫体表面的,可防止水分流失。

\newline

六足动物的循环系统为开管式,呼吸器官为气管,消化道分为前肠、中肠、后肠。
中肠源于内胚层,肠壁的肌肉内环外纵。
中肠壁细胞向肠腔内分泌几丁质等物质,形成肠壁内层的围食膜,司保护,营养物质可渗透通过围食膜。
前肠和后肠为外胚层内陷形成,内衬角质层,在蜕皮时脱落。
其肌肉层内纵外环。

\newline

六足动物的排泄器官为马氏管,系消化道伸出的盲管,仅一层细胞和外层的底膜,由外胚层发育而来。
马氏管底膜外有肌肉组织,管腔内侧细胞生微绒毛。
呼吸器官为外胚层内陷形成的气管。

\newline

六足动物的中枢神经系统包括脑、食管下神经节和腹神经链。
头部前三个体节的神经节愈合形成脑,位于食管上方;
后三个体节的神经节愈合为食管下神经节;
二者通过一对围食管神经相连。
腹神经链包括左右两个愈合的神经干(nerve trunk),其上的神经节发出神经。
六足动物的感觉器官发达,触角司感觉、嗅觉;
口器上有味觉感受器;
触角基部、前足或第一腹节有听觉感受器;
虫体有司触觉的触毛;
头部有单眼和复眼。

\newline

六足动物雌雄异体,其生殖系统包括腹部末端几个体节的附肢形成的外生殖器和中胚层形成的内生殖器官。
内生殖器官包括生殖腺和生殖管道。
六足动物可营两性生殖或孤雌生殖。
部分物种存在多胚生殖,即一个受精卵形成多个胚胎。
其发育过程一般存在变态(metamorphosis),幼虫和成虫形态差异显著。
完全变态(complete metamorphosis)的物种,其生活史包括卵(egg)、幼虫(larva)、蛹(pupa)、成虫(adult)四个阶段,幼虫和成虫差异极大。
不完全变态(incomplete metamorphosis)的物种生活史仅卵、幼虫、成虫三个阶段。
幼虫多无翅,体型较小,其余特征与成虫相同。

\newline

六足动物有休眠(dormancy)和滞育(diapause)现象,以应对不良环境。
此时虫体停止摄食、运动、生长、繁殖等活动,新陈代谢下降到最低水平。
休眠是直接由环境引起的,环境条件恢复正常,休眠遂终止。
滞育是在环境条件恶化之前,由某些信号引起的。
其终止需要一定的物理或化学刺激。
此外,部分物种有多态现象(polymorphism),即同一物种同一性别的不同个体的形态结构存在明显差异,在社会性昆虫(social insect)中表现最为明显。

\subsection{原尾纲(Protura)}
体细小,头卵圆,内颚式口器,无触角,假眼一对,下颚腺一对。
前胸足长大,生感觉刚毛。
腹部十二节,近腹部后端有外生殖器。
生于土壤,食真菌。

\subsection{弹尾纲(Collembola)}
仅弹尾目(Collembola)。
俗称跳虫。
体细小,腹部有弹器,无翅。

\subsection{双尾纲(Diplura)}
仅双尾目(Diplura)。
体细长扁平,尾部一对发达尾须,线状或钳状。
无翅,无眼,长丝状触角。
咀嚼式口器,藏于头部腹面腔内。
胸足发达。
腹十一节,第一至七节腹面有一对针突。
惧光。

\subsection{昆虫纲(Insecta)}
体分头胸腹三部分。
头部一对触角。
胸三节,各一对足。
中后胸节各有翅一对。
腹部附肢多退化,生殖孔后位。

\subsubsection{石蛃目(Archaeognatha)}
小型,体纺锤形,胸部粗并向背面拱起。
体表密布鳞片,有金属光泽,多棕褐色。
有的背部有黑白花斑。
有单眼,复眼在体中线相接。
中足、后足基节有外叶。
腹部十一节,三根尾须。
如石蛃(\textit{Machilis})。

\subsubsection{衣鱼目(Zygentoma)}
无翅,背鳞,扁平。
触角长丝状,复眼分离。
腹部七至九节有成对刺突,有尾须和中尾丝。
如西洋衣鱼(\textit{Lepisma saccharina})。

\subsubsection{蜉蝣目(Ephemeroptera)}
虫体细长柔弱。
成虫口器退化,复眼大,触角细小。
翅三角形,透明膜质,后翅退化,静息时竖立于背面。
腹末一对长尾须,部分品种有中尾丝。
幼虫淡水生,咀嚼式口器,腹部两侧有气管鳃,腹末一对长尾须,一中尾丝。
不完全变态。
如蜉蝣(\textit{Ephemera})。

\subsubsection{蜻蜓目(Odonata)}
头灵活,复眼大,触角细小,咀嚼式口器,齿发达。
翅两对,透明膜质,不可折叠,翅脉网状,有翅痣(pterostigma)。
尾须小。
若虫水生,下唇特化为捕食器官,不完全变态。

\subsubsection{等翅目(Isoptera)}
俗称白蚁。
体软色浅。
触角短小,佛珠状。
咀嚼式口器。
尾须短,不完全变态,社会性昆虫。
在一定季节产生有翅繁殖蚁。
前后翅相似,狭长,透明膜质,静息时平放于背面。
如大白蚁(\textit{Macroterms})。

\subsubsection{蜚蠊目(Blattaria)}
俗称蟑螂。
体扁平。
前胸背板盾形,盖住头部。
足相似,基节宽大。
若虫发育中翅芽不反转。
雌性产卵管短小,藏于第七腹节。
雄性外生殖器多不对称,为第九腹节短腹刺掩盖。
如美洲大蠊(\textit{Periplaneta americana})、中华地鳖(\textit{Eupolyphaga sinensis})。

\subsubsection{螳螂目(Mantodea)}
头灵活,三角形。
前足腿节、胫节有刺。
胫节镰刀状,向腿节折叠,形成捕捉足。
前翅皮质,后翅扇状膜质,静息时叠于背面。
腹部肥大。
如刀螂(\textit{Tenodera})。

\subsubsection{蛩蠊目(Grylloblattodea)}
体扁长,暗灰色,无翅,口在头前端,咀嚼式口器。
触角丝状,多刚毛。
复眼小,无单眼。
胸部发达。
有长尾须。
产于寒冷区域。
如中华蛩蠊(\textit{Galloisiana sinensis})。

\subsubsection{螳螂竹节虫目(Mantophasmatodea)}
有颚,有两个小牙,咀嚼式口器。
长丝状触角。

\subsubsection{竹节虫目(Phasmatodea)}
体棒状,少数品种叶状,行动笨拙,长于拟态。
触角细长多节,复眼小,咀嚼式口器,翅退化,尾须短。
雄性少,常孤雌生殖,产卵器不明显。
卵单产,散落地面。
不完全变态。

\subsubsection{纺足目(Embioptera)}
体长扁柔软,雌雄异型。
头大,触角念珠状,无单眼。
雌虫复眼小,雄虫复眼发达。
咀嚼式口器,前胸比头窄。
雌虫无翅。
雄虫翅膜质狭长多毛。
前足第一跗节膨大,内有纺丝腺,可吐丝。
雄虫第十腹节背板中部裂开。
尾须两节,雄虫左尾须基节膨大,内侧生小瓣。
如足丝蚁(\textit{Oligotoma})。

\subsubsection{直翅目(Orthoptera)}
为中大型昆虫。
丝状触角,咀嚼式口器,前胸背板大。
前翅为革质覆翅,狭而厚,静息时覆盖后翅。
后翅膜质透明扇状。
后足为跳跃足。
尾须短而不分节。
不完全变态。
雌虫产卵器发达。
分为两类。
一类触角长过体半,听器位于前足胫节,雄虫前翅摩擦发声,如蟋蟀(\textit{Gryllulus})、北京油葫芦(\textit{Teleogryllus emma})。
另一类触角短于体半,听器位于第一腹节两侧,雄虫前翅后足摩擦发声,如东亚飞蝗(\textit{Locusta migratoria})、中华稻蝗(\textit{Oxya chinensis})。

\subsubsection{革翅目(Dermaptera)}
体长扁,头宽,圆形复眼,线形触角。
上颚发达,前生小齿。
前胸节近方形,后胸节有背板。
腹板宽。
不完全变态。
如蠼螋(\textit{Labidura})。

\subsubsection{缺翅目(Zoraptera)}
三角形头,念珠状触角,咀嚼式口器。
翅退化或膜质,尾须短小不分节。
如中华缺翅虫(\textit{Zorotypus sinensis})。

\subsubsection{啮虫目(Psocoptera)}
无翅品种体色淡,无单眼,前胸大。
有翅品种体色浓,三个单眼,前胸小。
头大,上方有Y形蜕裂缝。
触角丝状或鬃状。
咀嚼式口器,上颚发达。
如窃虫(\textit{Atropos pulsatorium})。

\subsubsection{虱目(Phthiraptera)}
体扁平,有复眼,无单眼。
刺吸式口器。
胸节完全愈合,无翅,无尾须,不完全变态。
寄生,食血。
如人虱(\textit{Pediculus humanus})。

\subsubsection{缨翅目(Thysanoptera)}
锉吸式口器,左右不对称。
翅狭长,翅缘扁长有毛。
无尾须,不完全变态。
如蓟马(\textit{Haplothrips})。

\subsubsection{半翅目(Hemiptera)}
口器刺吸式,无尾须,渐变态。
同翅亚目(Homoptera)触角刚毛状,前后翅均为膜质,静息时成屋脊状放置背面,口器自前足基部之间伸出。
部分品种雄性腹部第一节腹面有发声器。
如黑蚱(\textit{Cryptotympana atrata})、棉蚜(\textit{Aphis gossipii})、灰飞虱(\textit{Laodeopax striatella})、介壳虫。
异翅亚目(Heteroptera)俗称蝽,前翅为半鞘翅,基部革质,末端膜质,静息时覆盖膜质后翅。
口器自头前端伸出。
后足基部有臭腺。
如荔枝蝽(\textit{Tessaratoma papillosa})、温带臭虫(\textit{Cimex lectularius})、红带猎蝽(\textit{Triatoma rubrofasciata})、蝎蝽(\textit{Nepa chinesis})。

\subsubsection{脉翅目(Neuroptera)}
体柔软。
触角丝状多节。
复眼大,有三单眼。
咀嚼式口器。
前后翅等大,膜质,翅脉网状,静息时成屋脊状放置背面。
外缘翅脉分叉。
完全变态。
如草蛉(\textit{Chrysopa perla})。

\subsubsection{广翅目(Megaloptera)}
口在头前端,咀嚼式口器。
上颚发达,下颚须五节,下唇须三节。
复眼凸出。
前胸节大,方形。
中后胸节有背片和气门。
翅膜质,网状翅脉,后翅可折叠。
外缘翅脉不分叉。
无尾须。
如鱼蛉(\textit{Corydalus})。

\subsubsection{蛇蛉目(Raphidioptera)}
体细长,多褐色或黑色。
头长,后端变细,成三角形。
复眼发达,丝状触角,咀嚼式口器。
前胸节延长成颈状,中后胸节宽短。
翅膜质,狭长,网状翅脉。
无尾须,完全变态。
雄虫尾部有肛上板和抱握器,雌虫有长针状产卵器。
如蛇蛉(\textit{Agulla})。

\subsubsection{鞘翅目(Coleoptera)}
俗称甲虫。
体坚硬,有光泽,常无单眼,咀嚼式口器。
前胸节发达,背板宽大。
中胸节小,背板三角形。
前翅为鞘翅,角质,坚厚,无翅脉,左右翅相遇于中线。
后翅膜质,静息时折叠于鞘翅下。
部分缺后翅品种,鞘翅在中线处愈合。
无尾须,全变态。
如金星步甲(\textit{Calosoma maderae})、龙虱(\textit{Cybister})、谷斑皮蠹(\textit{Trogoderma granarium})、黄粉虫(\textit{Tenebrio molitor})。

\subsubsection{捻翅目(Strepsiptera)}
雌虫蛆形,无翅无足,内寄生。
雄虫头宽,复眼大而凸出,无单眼,咀嚼式口器不发达。
触角扇状。
胸部长,后胸节大,后背片发达。
前翅退化为棒状,后翅宽大扇状,无尾须。
如土蝽(\textit{Triozocera macroscyti})。


\subsubsection{双翅目(Diptera)}
俗称蚊、蝇。
触角丝状,刺吸式或舐吸式口器。
复眼发达。
蚊无单眼,蝇有三个单眼。
前翅膜质,后翅退化为平衡棒,隐于前翅基部瓣下。
全变态,幼虫无足。
蚊成虫身细足长,触角长于头胸总长。
幼虫为孑孓,有足,头明显。
如库蚊(\textit{Culex})、伊蚊(\textit{Iedes})、按蚊(\textit{Anopheles})。
蝇成虫身粗足短,触角短于头胸总长。
幼虫为蛆,无足,头不明显。
如家蝇(\textit{Musca domestica})、斑黑带食蚜蝇(\textit{Episyrphus balteatus})。

\subsubsection{长翅目(Mecoptera)}
体细长,略侧扁。
口在头下,向下延长成喙状,咀嚼式。
复眼发达,触角丝状,前胸短小,足细长。
翅膜质,狭长,前后翅相似。
尾须短小不分节。
雄性外生殖器膨大成球形,上举似蝎尾。
如蝎蛉(\textit{Boreus})。

\subsubsection{蚤目(Siphonaptera)}
无翅,体坚硬侧扁,外寄生哺乳类、鸟类,刺吸式口器,食血。
完全变态。
幼虫自由生活,无足,体圆柱形,咀嚼式口器。
如人蚤(\textit{Pulex irritans})。

\subsubsection{毛翅目(Trichoptera)}
俗称石蛾。
复眼发达,触角丝状多节,咀嚼式口器不发达。
前胸短,足细长,翅膜质生毛。
全变态。

\subsubsection{鳞翅目(Lepidoptera)}
俗称蝶、蛾。
全身被鳞,翅尤甚。
复眼大。
成虫口器虹吸式,特化为卷曲长喙。
部分蛾类成虫口器退化。
翅扁平,前翅较大,少翅脉。
完全变态。
幼虫头发达,口器咀嚼式,三胸节,十腹节,三对胸足。
另有五对腹足,位于第三至第六腹节和第十腹节。
体内有丝腺,吐丝作茧。
蝶触角末端膨大成鼓槌状,静息时翅竖立,腹部细,白天活动,无茧。
如柑桔凤蝶(\textit{Papilio xuthus})、菜粉蝶(\textit{Pieri rapae})。
蛾触角不成鼓槌状,静息时翅平放于背面或呈屋脊状,腹部粗大,夜行,有茧。
如棉铃虫(\textit{Heliothis armigera})、松毛虫(\textit{Dendrolimus})、家蚕(\textit{Bombyx mori})、柞蚕(\textit{Antheraea pernyi})。

\subsubsection{膜翅目(Hymenoptera)}
俗称蜂、蚁。
触角丝状、膝状或锤状,口器咀嚼式或嚼吸式。
翅膜质,前翅大,翅脉少。
后翅前缘有一列小钩连接前翅后缘。
静息时翅平展于腹部背面。
腹部第一节并入胸部。
无尾须。
完全变态。
如蜜蜂(\textit{Apis})、熊蜂(\textit{Bombus})、胡蜂(\textit{Vespa})、红火蚁(\textit{Solenopsis invicta})。

\section{多足亚门(Myriapoda)}
体节分化程度不高。
前六个体节愈合形成头部,保留三至四对附肢。
第一体节的附肢演化为触角。
躯干分节,每一节上生一至二对步足。
呼吸器官为气管,心脏细长,贯穿虫体。
链状神经系统,异体受精,有求偶行为。
体表无蜡层,故其生境一般较为湿润。

\subsection{唇足纲(Chilopoda)}
俗称蜈蚣。
体角质化程度底,体节各一对足。
第一对足为颚足,粗大,末端为爪,有毒腺开口。
生殖孔位于倒数第二节腹面中央。
如少棘蜈蚣(\textit{Scolopendra mutilans})、多棘蜈蚣(\textit{Scolopendra multidans})、蚰蜒(\textit{Thereuopoda})。

\subsection{倍足纲(Diplopoda)}
俗称马路。
体多圆筒形,触角短,外骨骼发达。
前五体节愈合为头。
第一小颚愈合为颚片,第二小颚消失。
躯干前四节为胸部,各一对足。
其后的体节是由胚胎期的两个体节合并而成的,每节两对足,两对气门,两对心孔,两对腹神经节,称为倍节(diplosomite)。
生殖孔为于躯干第三节腹面。
如雅丽酸带马路(\textit{Oxidus gracilis})。

\subsection{少足纲(Pauropoda)}
体小,十一体节,一尾节。
六背板,步足九对。
如\textit{Pauropus huxleyi}。

\subsection{综合纲}
体小,似蜈蚣。
虫体乳白色,十二对步足。
如\textit{Scutigerella}。

\end{document}