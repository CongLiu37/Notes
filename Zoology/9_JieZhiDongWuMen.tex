\documentclass[11pt]{article}

\usepackage[UTF8]{ctex} % for Chinese 

\usepackage{setspace}
\usepackage[colorlinks,linkcolor=blue,anchorcolor=red,citecolor=black]{hyperref}
\usepackage{lineno}
\usepackage{booktabs}
\usepackage{graphicx}
\usepackage{float}
\usepackage{floatrow}
\usepackage{subfigure}
\usepackage{caption}
\usepackage{subcaption}
\usepackage{geometry}
\usepackage{multirow}
\usepackage{longtable}
\usepackage{lscape}
\usepackage{booktabs}
\usepackage{natbib}
\usepackage{natbibspacing}
\usepackage[toc,page]{appendix}
\usepackage{makecell}

\title{节肢动物门(Arthropoda)}
\date{}

\linespread{1.5}
\geometry{left=2cm,right=2cm,top=2cm,bottom=2cm}

\begin{document}

  \maketitle

  \linenumbers
  
\section{一般特征}
节肢动物为动物界第一大类群,亦是原口动物中最为进化的类群。
节肢动物虫体异律分节(heteronomous segmentation)。
各体节结构和功能发生分化,节数因种类而不同。
相似的体节又多组合在一起,形成虫体不同部分。
总之,体节的分化和组合,提高了节肢动物对不同生境的适应能力。
节肢动物躯体第一节称为顶节(acron),最后一节称为尾节(telson),其腹部或末端有肛门。
顶节和尾节非胚胎分节形成的,故并不是真正的体节。
二者之间才是真正意义上的体节。

\newline

节肢动物体被坚硬厚实的外骨骼,其主要成分为几丁质(chitin)和蛋白质。
组成外骨骼的物质是其内层的一层上皮细胞分泌的。
外骨骼一旦形成,遂不能生长。
因此,节肢动物有蜕皮(moulting)现象。
蜕皮时,上皮细胞脱离旧外骨骼,产生新外骨骼,并分泌蜕皮液(moulting fluid)于二者之间。
蜕皮液中有几丁质酶和蛋白酶,可分解旧外骨骼。
旧外骨骼逐渐变薄并破裂。虫体由裂缝处钻出并膨胀。
柔软的新外骨骼随之扩张,完成虫体的生长。
过一段时间后,新外骨骼变硬,生长停止。
外骨骼主要司支持和保护,亦是肌肉的附着点。
此外,外骨骼有效地防止水分的丢失,其上的各种突起可司感觉。

\newline

节肢动物有发达的附肢。
附肢通过关节于本体相连,其内有发达的肌肉。
附肢本身亦分节,故称节肢(arthropodium)。
节肢往往具有不同的结构,以执行不同的功能,如形成口器、触角、外生殖器、适于不同运动方式的足、呼吸器官等。

\newline

节肢动物肌肉系统发达。
肌纤维集合为肌肉束,其两端附着于外骨骼,以调整和放大肌肉运动,增强效能。
节肢动物的肌肉束多成对排列,相互拮抗,使躯干或附肢得以向不同方位运动。
每个体节的躯干肌包括一对背纵肌和一对腹纵肌,前者收缩使躯体向上弯曲,后者收缩使躯体向下弯曲。
每只附肢一般有三对附肢肌,使之得以前后、上下、内外运动。

\newline

节肢动物的真体腔退化,形成生殖管腔、排泄管和围心腔。
在胚胎发育过程中,围心腔壁消失,围心腔和假体腔混合,形成混合体腔(mixocoel)。
混合体腔内充满血液,故称血腔(haemocoel)。
节肢动物的心脏呈管状,向前发出一条短动脉。
短动脉伸入头部,末端开口。
血液经心脏和短动脉,进入血腔。
身体各部分的组织直接浸沐于血液中。
而后血液经心脏上的心孔回到心脏。
此外,肠道直接浸润在血液中,养分可经过肠壁直接进入血液。
总之,节肢动物的循环系统为开管式。

\newline

节肢动物呼吸器官多样。
水生种类有体壁外突形成的鳃,一些小型水生种类通过体表进行气体交换。
陆生种类则以气管(trachea)呼吸。
气管是体壁内陷形成的,其外端通过气门与外界相通,其内端不断分支,伸入组织间,直接和细胞接触。
由此,陆生节肢动物组织可直接与外界进行气体交换。
气管是动物界最为高效的呼吸器官。
除鳃和气管外,节肢动物的呼吸器官还包括书肺、书鳃、足鳃、气管鳃等。

\newline

节肢动物运动能力强,代谢兴旺,排泄器官发达。
节肢动物的排泄器官主要有两种类型。
其一是与后肾管同源的绿腺(green gland),包括腺基部和膀胱。
含氮废物通过渗透进入腺基部,而后进入膀胱部,经排泄孔排出。
另一种为马氏管(Malpighian tube),系从中肠和后肠之间发出的多条细管。
马氏管浸沐于血腔,可收集血液中的代谢废物,使之通过后肠,随粪便一同排出。

\newline

节肢动物有发达的捕食、摄食、碎食结构。
部分种类有发达的中肠突出物,司储存养料。
还有部分种类肠道周围有大量脂肪细胞。
大部分节肢动物有六个直肠垫(rectal papillae),可从食物残渣中回收水分。
这些结构于节肢动物强大的运动能力和较高的代谢效率相适应。

\newline

节肢动物运动能力的增强伴随着神经系统和感觉器官的复杂化。
节肢动物的神经系统呈链状结构,但部分前后相邻的神经节常愈合为大的神经团。
其感觉器官相当复杂,司平衡、触觉、视觉、味觉、嗅觉、听觉的器官俱全。
眼有单眼和复眼之分。
雌雄异体异形,常为体内受精,部分种类可营孤雌生殖。

\section{节肢动物的分类}
\subsection{螯肢亚门(Chelicerata)}
躯体分为头胸部和腹部。
头胸部每个体节各有一对足,另有一对螯肢和一对脚须,分别位于口器前后。
螯肢一般分二到三节,末节有爪。
大部分种类的螯肢有毒腺,司捕食和防御。
脚须可司感觉。
但雄性蜘蛛脚须演化为交配器,蝎类脚须演化为大钳,以捕捉猎物。
腹部附肢或退化,或演化为吐丝板(蜘蛛)或鳃(鲎)。

\newline

螯肢动物大多以动植物汁液为食,鲜有摄取固体食物者。
其消化道较窄。
排泄器官为绿腺和马氏管。
循环系统为开管式,部分小型种类无循环系统。
神经系统呈链状,体表有司感觉的刚毛和其它化学感受器。
水生种类有复眼和单眼,陆生种类一般仅有单眼。
雌雄异体,生殖腺位于腹部消化管腹侧,其左右各伸出一生殖管,汇合后经生殖孔开口于腹部腹侧。
有交配行为,体内或体外受精。
水生种类间接发育,陆生种类直接发育。

\newline

螯肢动物呼吸器官多样。
水生种类腹部附肢演化为鳃。
陆生种类和呼吸器官包括书肺和气管,二者可并存。
书肺是虫体腹部腹侧体表内陷形成的,呈书页状。
页间空腔汇合后经书肺孔与外界连通,页内与血腔相连。
螯肢动物的气管起源于书肺,非体壁内陷形成。
一些小型种类由体表呼吸。

\subsection{甲壳亚门(Crustacea)}
甲壳动物体节较多,其中头节和胸节愈合为头胸部。
原第六体节背面上皮层褶皱向后延申,包裹头胸部的背侧,其外层上皮细胞分泌的外骨骼形成头胸甲(carapace)。
腹部有六个体节,腹部末端另有尾节。
虾类腹部发达,蟹类腹部退化并向前曲折,俗称蟹脐。
甲壳动物头部第一节有一对复眼。
除头部第一节和腹部最后一节外,其余各体节一般均有一对附肢。
甲壳动物的附肢结构、功能多样,可司感觉、摄食、咀嚼、运动、生殖、呼吸等。

\newline

小型甲壳动物的消化系统比较简单,而大型物种有复杂的消化系统。
一般来说,甲壳动物的消化道分为前肠、中肠、后肠。
前肠包括短小的食道和膨大的胃。
部分物质胃内面有角质膜,可研磨食物。
中肠长短不一,有管道与发达的消化腺相接。
后肠直通肛门。

\newline

大型甲壳动物以位于头胸甲下两侧的鳃呼吸,虾类附肢基部突起形成的足鳃(podobranchia)亦司呼吸。
亦有发达的肌肉系统和完整的开管式循环系统。
小型种类多无呼吸器官和循环系统,或者循环系统不完整,只有心脏,无血管。
甲壳动物的排泄器官为绿腺和鳃。

\newline

低等甲壳动物的神经系统呈梯形。
高等种类神经节愈合明显,有脑和食管下神经节。
感觉器官发达,有眼、嗅毛、触毛、平衡囊等。
其眼自外到内分别为角膜、晶状体、视小网膜、色素细胞和视神经。
平衡囊为高等甲壳动物特有,虾类尤其发达,一般为第一对触角基部凹陷形成的囊,囊低有触毛。
触毛基部与双极感觉细胞的一极相连。
这些感觉细胞另一极束集入脑。
触毛顶部黏着大量细小的平衡石(statolith)。
平衡石可为自身分泌物,亦可为外界泥沙。
动物运动时,触毛承受来自平衡石的压力改变,这种刺激经双极感觉细胞入脑,使动物得以感知自身体位。

\newline

甲壳动物具有内分泌系统(endocrine system),以调节生殖、发育、蜕皮等生理活动。
一般雌雄异体,生殖腺经生殖管道,开口于生殖孔。
甲壳动物一般营两性生殖。
交配时,雄性排出由输精管上皮细胞分泌的物质包裹精子形成的精荚(spermatophore)。
精荚经生殖孔进入雌性体内。
卵子成熟后,精荚破裂,释放精子,完成体内受精。
甲壳动物多有抱卵行为,即产下的受精卵黏附于雌性附肢,直至孵化。
甲壳动物为间接发育,幼体形态多样。

\section{六足亚门(Hexapoda)}
六足动物有二十个体节,分为头、胸、腹三部分。
头部为顶节和前六个体节愈合形成,各节界限消失,外骨骼愈合为头壳(head capsul)。
第一、第三体节附肢退化,第二体节的附肢形成形态功能各异的触角。
头部第四、五、六体节的附肢参与形成口器,包括一对大颚、一对小颚和一片下唇。
此外,口器还包括上唇和舌。
随着食性不同,六足动物口器形态多样。

\newline

胸部有三个体节,称为前、中、后胸节。
各胸节相互愈合,不能自由活动,但彼此界限可辨。
其上的三对附肢演变为三对足。
足的形态、功能多样。此外,中、后胸节分别有一对前翅和一对后翅。
翅源于胸节左右侧面的扁平褶突,称为翅芽(wing pad)。
在发育过程中,翅芽上下两侧的上皮逐渐愈合、退化,形成仅有两层角质膜的翅。
翅内有高度角质化的翅脉(vein),其内中空,起支持作用,亦是气管、血管、神经出入翅的通路。
不同物种的翅形态功能不一,可司飞行、保护、平衡等。

\newline

腹部包括十一个体节和尾节,但成虫尾节多退化。
腹部体节中,仅末三节常彼此愈合,余者彼此游离。
第十一节有附肢发育形成的尾须,第八、九节常有附肢形成外生殖器(genitalia)。
第一至第八腹节两侧各有一个气门,内通气管。

\newline

六足动物躯体最外层为体壁(integument),仅一层皮细胞,源于外胚层。
皮细胞内层为血细胞分泌的底膜,外层为皮细胞分泌的表皮层,包括外骨骼和外骨骼表明的蜡层。
蜡层系皮细胞在蜕皮前分泌,而后扩散到虫体表面的,可防止水分流失。

\newline

六足动物的循环系统为开管式,呼吸器官为气管,消化道分为前肠、中肠、后肠。
中肠源于内胚层,肠壁的肌肉内环外纵。
中肠壁细胞向肠腔内分泌几丁质等物质,形成肠壁内层的围食膜,司保护,营养物质可渗透通过围食膜。
前肠和后肠为外胚层内陷形成,内衬角质层,在蜕皮时脱落。
其肌肉层内纵外环。

\newline

六足动物的排泄器官为马氏管,系消化道伸出的盲管,仅一层细胞和外层的底膜,由外胚层发育而来。
马氏管底膜外有肌肉组织,管腔内侧细胞生微绒毛。
呼吸器官为外胚层内陷形成的气管。

\newline

六足动物的中枢神经系统包括脑、食管下神经节和腹神经链。
头部前三个体节的神经节愈合形成脑,位于食管上方;
后三个体节的神经节愈合为食管下神经节;
二者通过一对围食管神经相连。
腹神经链包括左右两个愈合的神经干(nerve trunk),其上的神经节发出神经。
六足动物的感觉器官发达,触角司感觉、嗅觉;
口器上有味觉感受器;
触角基部、前足或第一腹节有听觉感受器;
虫体有司触觉的触毛;
头部有单眼和复眼。

\newline

六足动物雌雄异体,其生殖系统包括腹部末端几个体节的附肢形成的外生殖器和中胚层形成的内生殖器官。
内生殖器官包括生殖腺和生殖管道。
六足动物可营两性生殖或孤雌生殖。
部分物种存在多胚生殖,即一个受精卵形成多个胚胎。
其发育过程一般存在变态(metamorphosis),幼虫和成虫形态差异显著。
完全变态(complete metamorphosis)的物种,其生活史包括卵(egg)、幼虫(larva)、蛹(pupa)、成虫(adult)四个阶段,幼虫和成虫差异极大。
不完全变态(incomplete metamorphosis)的物种生活史仅卵、幼虫、成虫三个阶段。
幼虫多无翅,体型较小,其余特征与成虫相同。

\newline

六足动物有休眠(dormancy)和滞育(diapause)现象,以应对不良环境。
此时虫体停止摄食、运动、生长、繁殖等活动,新陈代谢下降到最低水平。
休眠是直接由环境引起的,环境条件恢复正常,休眠遂终止。
滞育是在环境条件恶化之前,由某些信号引起的。
其终止需要一定的物理或化学刺激。
此外,部分物种有多态现象(polymorphism),即同一物种同一性别的不同个体的形态结构存在明显差异,在社会性昆虫(social insect)中表现最为明显。

\section{多足亚门(Myriapoda)}
体节分化程度不高。
前六个体节愈合形成头部,保留三至四对附肢。
第一体节的附肢演化为触角。
躯干分节,每一节上生一至二对步足。
呼吸器官为气管,心脏细长,贯穿虫体。
链状神经系统,异体受精,有求偶行为。
体表无蜡层,故其生境一般较为湿润。


\end{document}