\documentclass[11pt]{article}

\usepackage[UTF8]{ctex} % for Chinese 

\usepackage{setspace}
\usepackage[colorlinks,linkcolor=blue,anchorcolor=red,citecolor=black]{hyperref}
\usepackage{lineno}
\usepackage{booktabs}
\usepackage{graphicx}
\usepackage{float}
\usepackage{floatrow}
\usepackage{subfigure}
\usepackage{caption}
\usepackage{subcaption}
\usepackage{geometry}
\usepackage{multirow}
\usepackage{longtable}
\usepackage{lscape}
\usepackage{booktabs}
\usepackage{natbib}
\usepackage{natbibspacing}
\usepackage[toc,page]{appendix}
\usepackage{makecell}

\title{半索动物门(Hemichordata)}
\date{}

\linespread{1.5}
\geometry{left=2cm,right=2cm,top=2cm,bottom=2cm}

\begin{document}

  \maketitle

  \linenumbers
  
\section{一般特征}
半索动物呈蠕虫状,分为吻(proboscis)、领(collar)和躯干(trunk)。
吻为圆锥形,位于体前,内有吻体腔(proboscis coelom)。
通过吻体腔内液体压力的变化,吻可伸缩。
领呈环状,内有一对领体腔(collar coelom),亦可伸缩。
口位于吻领交界处的腹面。
躯干长,前端两侧有鳃孔和生殖嵴(genital ridge)。
鳃孔后的的躯干分别为肝区和肠区,末端为肛门。

\newline

半索动物体表有纤毛,表皮外层为柱状细胞,其下依次为神经层和基膜。
基膜下依次为肌肉层和体腔膜。
除肝区外的表皮内有多种腺细胞,可分泌黏液。
吻腔通过位于后背部的吻孔与外界连接。
领和躯干内有领腔和躯干腔。
吻腔、领腔和躯干腔均由真体腔发育而来。

\newline

消化道壁肌肉较少。
口位于吻和领交界处腹面,口腔背壁向前伸出一盲管至吻基部,是为口索(stomochord)。
胃不明显,消化道靠后段的背侧有若干对肝盲囊(hepatic caecum),故此段躯干称为肝区。
而后消化道直达虫体末端,开口于肛门。

\newline

呼吸器官为鳃囊。
口后咽部有大量开口,即鳃裂。
鳃裂经鳃囊,开口于体表鳃孔。
鳃囊间有大量微血管。
水和泥沙经口入咽,而后水经鳃裂、鳃囊,由鳃孔排出,完成气体交换;
泥沙则进入消化道,其中的营养物质被消化吸收。

\newline

循环系统为开管式,包括背血管、腹血管和血窦。
血液在背血管中向前流动,在腹血管中向后流动。
背血管在吻腔基部膨大形成静脉窦,再向前则为中央窦。
中央窦附近有心囊。
心囊搏动,使血液进入其前放的脉球。
脉球司排泄,将血液中的代谢废物滤至吻腔,再经吻孔排出。
自脉球伸出四条血管,两条前行至吻,另两条后行并于领腹面汇合,连接腹血管。

\newline

神经系统发达。
体壁表皮有大量神经感觉细胞,背中线和腹中线各有一条神经索,称为背神经索和腹神经索。
二者于领处相连成环。
雌雄异体,生殖腺位于躯干生殖嵴内。
生殖腺开口于鳃孔处,营体外受精,间接发育。

\section{半索动物的分类}
\subsection{肠鳃纲(Enteropneusta)}
个体生活,多穴栖于潮间带或潮下带,食藻类、原生动物。
如柱头虫(\textit{Balanoglossus spp.})。

\subsection{羽鳃纲(Pterobranchia)}
固着于深海海底,聚生。
躯干呈囊状,消化道呈U形,有触手腕。
可营无性生殖。
如头盘虫(\textit{Cephalodiscus dodecalophus})。

\subsection{浮球纲(Planctosphaeroidea)}
仅\textit{Planctosphaera pelagica}一种。
其幼虫为透明球状,生活于深海。
未发现成虫。

\end{document}