\documentclass[11pt]{article}

\usepackage[UTF8]{ctex} % for Chinese 

\usepackage{setspace}
\usepackage[colorlinks,linkcolor=blue,anchorcolor=red,citecolor=black]{hyperref}
\usepackage{lineno}
\usepackage{booktabs}
\usepackage{graphicx}
\usepackage{float}
\usepackage{floatrow}
\usepackage{subfigure}
\usepackage{caption}
\usepackage{subcaption}
\usepackage{geometry}
\usepackage{multirow}
\usepackage{longtable}
\usepackage{lscape}
\usepackage{booktabs}
\usepackage{natbib}
\usepackage{natbibspacing}
\usepackage[toc,page]{appendix}
\usepackage{makecell}

\title{两栖纲(Amphibia)}
\date{}

\linespread{1.5}
\geometry{left=2cm,right=2cm,top=2cm,bottom=2cm}

\begin{document}

  \maketitle

  \linenumbers
\section{一般特征}
现存两栖动物体型分为蚓螈型、鲵螈型和蛙蟾型。
蚓螈型物种眼和四肢退化,尾短,形似蚯蚓,营穴居。
鲵螈型四肢短小,尾部发达侧扁,多营水栖。
蛙蟾型躯体短宽,四肢发达,无尾,适应于陆栖。

\newline

两栖动物头部扁平略尖,口宽。
吻两侧各有一具鼻瓣的外鼻孔。
陆栖种类眼大而突出,具活动性眼睑和瞬膜。
蛙蟾类眼后有鼓膜(tympanic membrane),为耳的一部分。
雄性可具有声囊(vocal sac)。
咽壁向体侧扩展形成的皮肤囊称为外声囊,位于口角两侧;
咽或下颌腹面肌肉褶皱向外突出形成的声囊称为内声囊。
声囊起共鸣作业,可扩大鸣叫声。
颅骨后缘至泄殖孔为躯干部,背面或有褶皱。
附肢两对。

\newline

古两栖动物体表有鳞,现存物种鳞退化,皮肤裸露,富于腺体。
皮肤经皮下结缔组织,疏松地于肌肉层向连。
表皮黏液腺发达,黏液保持体表湿润,减少水分散失,兼司体温调节和维持皮肤气体交换。

\newline

两栖动物头骨宽扁,脑腔狭小,骨块数目少,骨化程度不高。
次生颌为膜性硬骨。
脊柱进一步分化,包括颈椎(cervical vertebra)、躯干椎、荐椎(sacral vertebra)和尾椎。
颈椎使头部可上下运动。
腰带经荐椎,与脊柱连接。
肩带与头骨分离,附着于躯干椎,使头部和前肢活动范围扩大。
胸部正中有胸骨,不于与躯干椎相连。
附肢五趾。

\newline

除幼体和蚓螈外,肌肉不再分节排列,而是彼此愈合或移位,分化为形状各异的肌肉。
躯干背部肌肉退化,附肢肌强大而复杂。
另有肌肉控制控制咽喉和舌的活动。

\newline

消化道分为口、口咽腔、食管、胃、小肠、大肠、泄殖腔。
口咽腔内有牙、舌和内鼻孔、耳咽管孔、喉门、食管等开口。
两栖动物无咀嚼活动,牙起捕食和防止食物滑出的作用。
舌位于口咽腔底部。
蛙蟾类舌根附与下颌前端,舌尖朝向体后。
其舌可翻出,司捕食。
食道短,经贲门连通胃,胃司研磨和消化食物。
胃经幽门同小肠。
小肠前段为十二指肠(duodenum),后段为回肠(ileum)。
回肠通大肠。大肠宽阔,可吸收水分,后通泄殖腔的腹面,经泄殖孔至体外。
肝位于体腔前半部,分叶。
胆囊位于肝叶之间。
肝脏分泌胆汁,经肝管、胆囊管,储存于胆囊;
后经胆囊管、输胆管入十二指肠。
胰位于胃和十二指肠之间,分泌胰液,经胰管、输胆管,进入十二指肠。

\newline

两栖动物幼体水生,以鳃呼吸。
变态登陆后,口咽腔腹侧分化为呼吸道。
外鼻孔经鼻腔、内鼻孔,通口咽腔。
口咽腔经喉头,通入位于其腹侧的气管。
气管分为两支,分别通入两个肺(lung)。
肺位于心、肝背侧,为中空薄壁囊状,内分诸多小室,称为肺泡(alveolus)。
肺泡壁富微血管,司气体交换。
两栖动物皮肤薄湿,皮下血管亦司气体交换。
口腔粘膜亦参与呼吸。

\newline

两栖动物营口咽式呼吸。
外鼻孔瓣膜张开,喉门关闭,口底下降,空气进入口咽腔。
而后或由口腔粘膜进行气体交换,口底抬升,空气由鼻排出;
或外鼻孔瓣膜关闭,喉门打开,口底抬升,空气入肺,进行气体交换,再通过肺本身的弹性、口底下降和腹肌收缩,使空气回到口咽腔,关闭喉门,打开外鼻孔,口底抬升,排出空气。
蛙蟾类喉门内侧多附生一对弹性纤维带,即声带(vocal cord)。
空气出肺,声带振动,发出声音。
雄性或有声囊,通过共鸣使声音更加洪亮。

\newline

两栖动物的血液循环过程中,动脉血和静脉血初步分离,但未完全分开,故为不完全的双循环。
幼体心脏一房一室,紧挨头部,单循环系统。
成体心脏位于胸腔,两房一室。
心脏收缩始于静脉窦,窦内少氧血进入右心房。
而后心房收缩,左、右心房的血液分别进入心室左、右两侧。
心室左侧为少氧血,右侧为多氧血,中间为混合血。
心室右侧通动脉圆锥,发出三对动脉弓,分别为颈动脉弓(carotid arch)、体动脉弓(systemic arch)和肺皮动脉弓(pulmocutaneous arch)。
心室自右向左收缩,其右侧血液率先进入最近的肺皮动脉弓,其中部血液随后入体动脉弓,左侧多氧血再后进入颈动脉弓。
来自两个心房的血液在心室中并未被严格区分。

\newline

一对颈动脉弓通往头部。
一对体动脉弓在分出锁骨下动脉(subclavian artery)至前肢和食管后,汇合为背大动脉,向后延申并分支到内脏和后肢。
一对肺皮动脉弓分别通往肺泡壁和皮下,均形成毛细血管网,司气体交换;
再汇合为肺静脉(pulmonary vein),通左心房。
来自头部和躯干前部的静脉汇为前大静脉,通静脉窦。
来自躯干后半部和尾的静脉汇合后分为两对,一对沿肾外缘,形成肾门静脉(renal portal vein),入肾,分为肾小球,再汇合为肾静脉(renal vein),与生殖腺静脉(genital vein)汇合,通后大静脉(postcava);
另一对为盆骨静脉(pelvic vein),于腹壁汇合为腹静脉(abdominal vein),向前与来自消化系统的静脉汇合为肝门静脉入肝,再由一对肝静脉出肝,通入后大静脉。
后大静脉通静脉窦。
两栖动物淋巴系统发达,以收集从血管和组织细胞内渗出的淋巴液。
淋巴心发达,以使淋巴液回流至心脏。

\newline

两栖动物双循环系统不完全,动脉血含氧较低,新陈代谢教慢,又无良好的保温条件和完善的体温调节机制,故不能维持体温恒定。
这种体温随环节温度变化的动物,称为变温动物(poikilothermal)。

\newline

两栖动物的排泄器官包括皮肤、肺和肾,以肾为主。
两肾外缘连接输尿管,分别通泄殖腔背面。
雄性肾前部无泌尿功能,肾小管与精巢伸出的精细管相通,精子经输尿管进入泄殖腔。
雌性的泌尿系统和生殖系统不相通。
蛙蟾类泄殖腔腹面突出,形成膀胱,储存尿液,回收水分。
但两栖动物保水机制仍不完善,故不能长时间远离水源。

\newline

两栖动物脑分嗅脑、大脑、间脑、中脑、小脑、延脑。
各部分分化程度不高,基本排列于同一平面。
延脑后接脊髓,脊髓有两个膨大部分,即颈膨大和腰膨大,分别控制前后肢。
外周神经系统有脑神经十对,脊神经数目不一。
部分脊神经汇合为臂神经丛和腰荐神经丛,分别进入前后肢。
植物性神经系统较鱼类发达。

\newline

幼体有侧线,变态后基本消失,仅水栖鲵螈头部、躯干尚有所保留。
具有能活动的眼睑、瞬膜以及泪腺(lachrymal gland)、哈式腺(Harderian gland),以保持眼球湿润,免遭伤害。
晶状体弹性差,通过晶状体牵引肌改变晶状体位置,调节视力能力较差。
角膜突出,远离晶状体,适于远视。
鼻腔以内外鼻孔通外界和口咽腔,其内有嗅粘膜。
从此,鼻腔兼司呼吸和嗅觉。
两栖动物的耳开始兼司听觉和平衡。
内耳下端球状囊分化出听壶(lagena),可感受声音。中耳(middle ear)内有耳柱骨(columella),其两端紧贴鼓膜内壁和内耳外壁。
一对耳咽管(eustachin tube)连通口咽腔和中耳。
鼓膜位于体表。

\newline

雄性的一对精巢伸出输精管,连通肾前端的肾小管,经输尿管,通泄殖腔。
雌性有一对囊状卵巢。卵子成熟后进入腹腔,再入输卵管前端漏斗。
输卵管开口于泄殖腔背侧。
两栖动物多营体外受精,有求偶行为,蛙蟾类交配时抱对(amplexus)。
卵外包角质。
两栖动物多变态发育。

\section{两栖纲的分类}
\subsection{蚓螈目(Caeciliformes)}
体细长,似蚯蚓,体表有环状皮肤褶皱,褶皱内或生骨质鳞片。
四肢及带骨退化,尾不发达。
眼耳退化,鼻眼兼近颌部有可伸缩的突触。
多营穴居。

\subsection{有尾目(Caudata)}
体呈圆筒形,四肢短,尾长而侧扁,形似蜥蜴。
再生能力强,多水栖。

\subsection{无尾目(Anura)}
体宽短,四肢发达,无尾,善跳跃,有活动眼睑和瞬膜。

\end{document}