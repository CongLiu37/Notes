\documentclass[11pt]{article}

\usepackage[UTF8]{ctex} % for Chinese 

\usepackage{setspace}
\usepackage[colorlinks,linkcolor=blue,anchorcolor=red,citecolor=black]{hyperref}
\usepackage{lineno}
\usepackage{booktabs}
\usepackage{graphicx}
\usepackage{float}
\usepackage{floatrow}
\usepackage{subfigure}
\usepackage{caption}
\usepackage{subcaption}
\usepackage{geometry}
\usepackage{multirow}
\usepackage{longtable}
\usepackage{lscape}
\usepackage{booktabs}
\usepackage{natbib}
\usepackage{natbibspacing}
\usepackage[toc,page]{appendix}
\usepackage{makecell}

\title{环节动物门(Annelida)}
\date{}

\linespread{1.5}
\geometry{left=2cm,right=2cm,top=2cm,bottom=2cm}

\begin{document}

  \maketitle

  \linenumbers
  
\section{一般特征}
环节动物躯体出现分节,除体前端两节和体末端一节外,其余各节形态和内部结构基本相同,故称同律分节(homonomous metamerism)。
同律分节起源于中胚层,各种器官亦按体节重复。
分节增强运动能力,亦是生理分工的开始。

\newline

环节动物出现了真体腔(true coelom),或称体腔(coelom)。
真体腔源于中胚层内部的空腔,这些空腔不断扩大,取代了假体腔,亦使得中胚层组织附着于内胚层外层,发育为脏体腔膜(visceral peritoneum)和肌肉。
另有部分中胚层附着于外胚层内面,发育为壁体腔膜(parietal peritoneum)和肌肉。
由此,消化道外壁附着肌肉,增强消化道的蠕动,提高消化能力,促进新陈代谢,进而促进了各个系统的进一步完善。
环节动物体腔上皮形成双层的隔膜,将体腔依照体节分为小室,各室有孔相连。

\newline

环节动物出现了真正意义上的循环系统。
随着真体腔的发展,假体腔被压缩,最终成为血管腔和心脏内的空腔,尽管环节动物还没有出现真正意义上的心脏。
环节动物的循环系统主要有背血管、腹血管和连接二者的微血管网。
血液始终在血管中流动,不进入组织间隙,故称闭管式循环系统(closed vascular system)。
但部分种类成体的真体腔被组织填充,残留的真体腔形成血窦,无血管。
此外,环节动物的血红蛋白一般存在于血浆,其血细胞无色。

\newline

环节动物的排泄器官为肾管(nephridium)。
肾管一端开口于体外,称肾孔(nephridiopore);
另一端为漏斗状开口,称肾口(nephrostome),开口于体腔。
肾管司排泄和(或)生殖。
根据起源的不同,肾管分为三种。
体腔管(coelomoduct)起源于中胚层体腔上皮;
后肾管(metanephridium)是胚胎发育过程中原肾管向体腔延伸,与体腔上皮形成的肾口相接形成的;
混合肾(nephromixium)是原肾管和体腔管嫁接而成的。

\newline

环节动物体前咽被侧有一对彼此愈合的咽上神经节(suprapharygeal ganglion),形成类似脑的结构。
由此分别向左、向右伸出围咽神经(circumpharygeal connective)。
围咽神经于咽下相连并向体后延伸,形成腹神经索(ventral nerve cord),贯穿虫体。
腹神经索在每个体节都有膨大的神经节,故呈链状。
环节动物有多种感觉器官,类群间差异明显。

\newline

环节动物的运动器官有疣足(parapodium)和刚毛(seta)。
疣足是体壁凸出的扁平片状结构,体腔也伸入其中,一般每体节一对。
疣足上有可伸缩的刚毛。

\newline

环节动物的生殖细胞来自中胚层,不同物种生殖系统差异较大。
部分种类有固定的生殖腺;
另一些仅在生殖季节由体腔上皮产生生殖细胞,无生殖腺。
成熟的生殖细胞或突破体壁进入环境,或通过体腔膜外延形成的生殖管道离体。

\newline

陆生和淡水生的环节动物直接发育,无幼虫期。
海产品种的原肠胚先发育为担轮幼虫(trochophore larva)。
担轮幼虫呈陀螺形,体中部有两圈纤毛环。
口位于纤毛环附近,后接胃和肠,最终通向虫体下端的肛门。
担轮幼虫上端有司感觉的纤毛束,其基部位神经细胞组成的感觉板和眼点。
担轮幼虫的排泄器官为原肾管,假体腔发达。
担轮幼虫在海水中营浮游生活,后沉入水底,下端伸长,发育为成虫。
  
\section{环节动物的分类}
\subsection{原环虫纲(Archiannelida)}
全部海产,虫体细长,无疣足和刚毛,体表无明显环节,被有纤毛。
口前有眼和触手,雌雄异体,生殖腺回旋于各体节。
发育过程中有显著的担轮幼虫期。
  
\subsection{吸口虫纲(Myzostomida)}
虫体扁平,腹部有刚毛。
体表无明显环节,但神经系统分节。
多寄生于棘皮动物。

\subsection{多毛纲(Polychaeta)}
虫体呈圆柱形,分节明显。
体前端有发达的口前叶,上有多种感觉器官,如眼、触手、腹侧的触须和纤毛。
口前叶后为围口节(peristomium),上有司感觉的围口触须。
口位于口前叶和围口节之间的腹面。
许多物种的咽可翻出。
咽上有一对颚和细齿,司捕食。
躯干部体节类似,每节有一对疣足。
虫体末端为肛节(pygidium),上有肛门。

\newline

多毛虫体被角质层,内有柱状表皮细胞。
部分种类的表皮细胞可分泌发光物质。
表皮内依次为肌层和壁体腔膜。
消化道贯穿虫体,包括口、咽、食管、胃、肠、直肠和肛门。
消化道肌肉层明显,可蠕动。
部分物种有发达的食管盲囊,司消化。

\newline

循环系统包括背血管、腹血管和连接二者的环血管。
腹血管在每个体节发出一对分支到疣足,一对分支到体壁,一个分支到后肾管,一个分支到肠。
多毛虫的呼吸器官一般为体壁突起形成的鳃,其中有血管丛。
许多种类的鳃系疣足上半部分变形而成。
一些小型多毛虫无呼吸器官,通过体表扩散交换气体。

\newline

雌雄异体,生殖腺仅在生殖季节出现,无生殖导管。
卵子破体壁而出,精子则由后肾管排出。
营体外受精。

\subsubsection{蠕形亚纲(Scolecida)}
口前叶无触须或其他附属物,尾节有触须。
有能突出的球形吻。
如海蚯蚓(\textit{Arenicola spp.})、臭海蛹(\textit{Travisia spp.})、阿曼吉虫(\textit{Armandia intermedia})。

\subsubsection{触须亚纲(Palpata)}
口前叶生一对触须,司感觉。

\newline

足刺超目(Aciculata)有发达的疣足,背叶或腹叶上至少有一足刺(aciculum),围口节上生触手。
如沙蚕(\textit{Nereis spp.})、鳞沙蚕(\textit{Aphrodita spp.})、吻沙蚕(\textit{Glycera spp.})。

\newline

管栖触须超目(Canalipalpata)多穴居。
口前叶或围口节有触手。
如磷沙蚕(\textit{Chaetopterus spp.})、龙介虫(\textit{Serpula spp.})、蛰龙介(\textit{Terebella spp.})。

\subsection{寡毛纲(Oligochaeta)}
大部分物种俗称蚯蚓,生活于土壤环境。
虫体体表由黏液腺,亦通过背孔分泌体腔液,以湿润皮肤,便于在土壤中钻动。
蚯蚓头部、疣足、眼点退化,司运动的刚毛生于体壁,口前叶可以伸缩,某些体节形成生殖带(clitellum)。

\newline

寡毛虫外被角质层,向内分别为上皮和肌层。
环肌层狭窄,纵肌层发达。
静止时,体节纵肌层收缩,环肌层舒张,体节变短粗。
同时由于体腔内充满体腔液,体节变硬。
此时体壁上斜向后伸的刚毛伸出,插入土壤。在运动时,某一体节纵肌层舒张,环肌层收缩,体节变细长,刚毛缩回,虫体前移。

\newline

消化道纵行于体腔中央,肌层发达。口位于体前,口腔可从口翻出。
口后的咽肌肉发达,有单细胞咽腺,可分泌黏液和消化酶。
食管短细,有食管腺,可分泌钙质,中和酸性食物。
咽后为砂囊(gizzard),其肌肉发达,内衬厚角质膜,可磨碎食物。
从口至砂囊的消化道称为前肠,源于外胚层。
砂囊后为胃,胃前有一圈分泌消化酶和黏液的胃腺。
胃通于肠,肠背侧中央有盲管,可增大消化吸收面积。
肠的后段两侧向前伸出一对锥状盲肠(caeca),系重要消化腺。
胃和肠统称中肠,源于内胚层。
肠后至肛门胃后肠,后肠无盲道,无消化功能,以肛门开口于体后端。

\newline

背血管(dorsal vessel)较粗,可搏动,血液自后向前流动;
腹血管(ventral vessel)较细,血液自前向后流动。
腹血管下为腹神经索,其下有更细的神经下血管(subneural vessel)。
食管两侧各一条较细的食管侧血管(lateral oesophageal vessel)。
四至五对环血管围绕消化道,其位置因种类不同而异。
环血管内有瓣膜,可搏动。无动静脉之分,其血浆中含血红蛋白,亦无呼吸器官,仅通过体壁交换气体。
其排泄器官为后肾管。

\newline

第三体节背侧有一对咽上神经节,第三和第四体节之间腹侧有一对咽下神经节,二者以围咽神经相连。
咽下神经节向后伸出腹神经索,腹神经索在每个体节均有一神经节,其上发出三对神经,分布于各个器官。
咽上和咽下神经节均有向体前伸出的神经。
感觉器官不发达,体壁上有司触觉的乳突,口腔内有司味觉和嗅觉的感受器,体前几节的背面有辨别光强弱的感受器。

\newline

寡毛虫一般雌雄同体,生殖器官仅限于体前的部分体节,有交配行为,体内受精。

\subsubsection{水蚓目(Limniolae)}
水栖。
体型小,体壁多透明。
生殖带多为单层细胞。
如颤蚓(\textit{Tubifex spp.})、尾鳃蚓(\textit{Branchiura spp.})、头鳃蚓(\textit{Branchiodrilus spp.})。

\subsubsection{陆蚓目(Terricolae)}
陆生。
生殖带多为多层细胞。
如环毛蚓(\textit{Pheretima spp.})、杜拉蚓(\textit{Drawida spp.})、正蚓(\textit{Lumbricus})。

\subsection{蛭纲(Hirudinea)}
俗称蚂蝗。
虫体背腹扁平,头部不明显,无疣足和刚毛。
体前和体后各有一个吸盘,称为前吸盘和后吸盘,可辅助运动。

\newline

蛭类体腔退化,无血管系统,而代之以血窦。
大部分蛭类吸食宿主体液或血液,口腔内有三片颚,颚上生密齿。
咽内有单细胞腺体,分泌具有抗凝血功能的蛭素(hirudin)。
食管短,嗉囊发达,嗉囊两侧有数对盲囊,可储存食物。
一般通过体表进行气体交换,少数种类有鳃。
雌雄同体,有交配行为,体内受精。

\subsubsection{棘蛭目(Acanthobdellida)}
仅\textit{Acanthobdella peledina}一种,生于芬兰和西伯利亚的淡水湖泊。
有30体节,只有后吸盘,前端体节生刚毛。
体腔明显,有隔膜。

\subsubsection{鳃蛭目(Branchiobdellida)}
15体节,有前、后吸盘,无刚毛,体腔明显。
仅鳃蛭科一科(Branchiobdellidae),如鳃蛭(\textit{Ozobranchus spp.})

\subsubsection{水蛭目(Hirudinida)}
33体节,有前、后吸盘,无刚毛。
体腔退化为诸多腔隙管道。

\newline

吻蛭亚目(Rhynchobdellae)头端有能伸出的吻,如扁蛭(\textit{Glossiphonia spp.})、盾蛭(\textit{Placobdella spp.})、鱼蛭(\textit{Piscicola spp.})。

\newline

无吻蛭亚目(Arhynchobdellae)头端无吻,如山蛭(\textit{Haemadipsa spp.})、医蛭(\textit{Hirudo spp.})、蚂蟥(\textit{Whitmania spp.})。

\end{document}