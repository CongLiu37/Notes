\documentclass[11pt]{article}

\usepackage[UTF8]{ctex} % for Chinese 

\usepackage{setspace}
\usepackage[colorlinks,linkcolor=blue,anchorcolor=red,citecolor=black]{hyperref}
\usepackage{lineno}
\usepackage{booktabs}
\usepackage{graphicx}
\usepackage{float}
\usepackage{floatrow}
\usepackage{subfigure}
\usepackage{caption}
\usepackage{subcaption}
\usepackage{geometry}
\usepackage{multirow}
\usepackage{longtable}
\usepackage{lscape}
\usepackage{booktabs}
\usepackage{natbib}
\usepackage{natbibspacing}
\usepackage[toc,page]{appendix}
\usepackage{makecell}

\title{脊索动物门(Chordata)}
\date{}

\linespread{1.5}
\geometry{left=2cm,right=2cm,top=2cm,bottom=2cm}

\begin{document}

  \maketitle

  \linenumbers
  
\section{一般特征}
脊索动物的主要特征为脊索、背神经管和咽鳃裂。

\newline

脊索(notochord)位于背部,介于背神经管和消化道之间,源于中胚层,起支持体轴的作用。
脊索细胞富含液泡,液体压力使得脊索兼有弹性和硬度。
脊索细胞分泌形成脊索鞘(notochord sheath),包裹脊索。
低等脊索动物终生有脊索;
高等脊索动物仅胚胎期有脊索,成体中脊索被分节的骨质脊柱(vertebral column)取代。

\newline

背神经管(dorsal tubular nerve cord)是脊索动物的中枢神经系统,位于脊索背侧,是外胚层内陷形成的。
高等脊索动物中,背神经管前端分化为脑,后端分化为脊髓。

\newline

脊索动物咽部两侧有多个裂孔,直接开口于体表或以一共同开口与外界连通,是为咽鳃裂(pharyngeal gill slits)。
低等水生脊索动物咽鳃裂终生存在,并附生富血管的鳃,司呼吸。
高等陆生脊索动物咽鳃裂仅出现于胚胎期或幼体期,随着个体发育而消失。

\newline

此外,脊索动物的循环系统为闭管式,心脏位于消化管腹面,肛门后方多有肛后尾。

\newline

脊索的出现是动物演化历程中的重大事件。
脊索是支持躯体的主梁,为内脏器官提供保护,为肌肉提供支点,避免因肌肉收缩导致躯体变形,使得动物有向大型化发展的可能。
此外,脊索亦使定向运动更为有效,提高动物的运动能力。
总之,脊索的出现促进动物结构和功能的复杂化和多样化。
  
\section{脊索动物的分类}
\subsection{尾索动物亚门(Urochordata)}
成体大多营固着生活,体表有表皮分泌的被囊。
成体体表有相邻的两个孔,分别为入水孔和出水孔。
入水孔下为口,再下为宽大的咽。
咽和体壁之间的空腔称围鳃腔,系体腔的一部分。
咽上有诸多鳃裂,通围鳃腔。
鳃裂周围密生纤毛,纤毛的摆动使水经入水孔、口、咽和鳃裂进入围鳃腔,完成气体交换。
水中的营养物质则经咽进入胃、肠,肛门和生殖腺开于围鳃腔。
围鳃腔经出水孔,连通外界。

\newline

心脏位于胃附近的肌肉囊内,无收缩机能。
围心腔壁收缩,使血液循环。
血管内血流方向不定。
神经系统退化,仅在咽背侧有一神经节,向身体各个部分发出神经。
无成形的感觉器官和排泄器官。

\newline

大多雌雄同体,体外受精。
生殖腺位于胃附近,开口于围鳃腔。
亦可营出芽生殖。
幼体形似蝌蚪,尾内脊索发达,脊索背侧有神经管,其前端膨大为脑泡(cerebral vesicle),内含眼点,心脏位于消化道腹侧。

\newline

幼体体前有附着突起(adhesive papillae),以黏附于其它物体上,从而开始变态。
变态过程中,尾和脊索消失,神经管退化为一个神经节,咽部扩大,口移至背部,体壁分泌被囊。
尾索动物变态发育过程中,部分重要构造消失,结构变得简单,称为逆行变态(retrogressive metamorphosis)。
  
\subsection{头索动物亚门(Cephalochordata)}
仅一纲一目,即文昌鱼目(Amphioxiformes)。
文昌鱼为半透明小鱼状生物,终生有脊索、背神经管和咽鳃裂。
脊索延申至背神经管前方,无真正的头和脑。

\newline

文昌鱼体前腹面为漏斗状口笠(oral hood),内为前庭(vestibule),前庭通向口,口周围有环形缘膜(velum)。
口笠周围有朝向体外的触须(cirri),缘膜周围有朝向体内的缘膜触手(velar tentacle),皆司保护和过滤。
背侧中线处有低矮的背鳍(dorsal fin),向后与尾部腹面两侧有皮肤下垂形成的腹褶(metapleura fold)。
腹褶和肛前鳍交界处有一腹孔(atripore),系咽鳃裂排水出口。

\newline

腹面有横肌,控制排水;
口部缘膜有括约肌,控制口的大小。
咽宽大,内壁有纤毛。
纤毛和缘膜触手摆动,水流入咽,其中的食物颗粒被咽壁细胞分泌物粘结成团,进入肠;
水则经咽鳃裂,进入围咽腔,经腹孔排出。
咽鳃裂司呼吸,其内壁有纤毛上皮。

\newline

食物团入肠,分解为小颗粒,进入肝盲囊(hepatic diverticulum)。
肝盲囊为肠向体前伸出的一个盲囊,突入咽的右侧,可分泌消化液。
食物小颗粒进入肝盲囊,为肝盲囊细胞吞噬,进行细胞内消化,残渣回到肠,进入后肠,并在此进行进一步的消化吸收。
最后的残渣由肛门排出。

\newline

循环系统为闭管式。
无心脏,腹大动脉(ventral aorta)搏动,推动血液循环。
咽下的腹大动脉两侧伸出成对的鳃动脉,经咽鳃裂进入左右两根背大动脉根(branchial arteries)。
背大动脉根向前,血液经组织间隙进入体壁静脉(parietal vein),再进入一对向体后延申的前主静脉(anterior cardinal vein)。
背大动脉根向后,汇合为一根背大动脉(dorsal aorta),其中的血液经组织间隙,进入一对后主静脉(posterior cardinal vein)和一根尾静脉(caudal vein)。
一对后主静脉向前,与向后的前主静脉汇合,形成一对总主静脉,而后再次汇合为静脉窦(sinus venosus),通入腹大动脉。尾静脉向前,和来自肠壁的毛细血管网汇合,形成肠下静脉(subintestinal vein)。
肠下静脉于肝盲囊处形成毛细血管网,而后再次汇合为肝静脉(hepatic vein),通入静脉窦。

\newline

排泄器官为咽壁背方两侧的肾管(nephridium)。
肾管短而弯曲,弯曲的腹侧开口于围鳃腔,背侧有与肾管相通的管细胞(nephrostome)。
管细胞远肾管端膨大,紧贴体腔,内有鞭毛。
代谢废物进入体腔液,渗透入管细胞,经鞭毛摆动入肾管,再经围鳃腔排出。

\newline

背神经管前端膨大,形成脑泡。
神经管背面未完全愈合,留有背裂(dorsal fissure)。
脑泡发出两对神经;
背神经管两侧发出成对的脊神经,分别向背侧和腹侧延申。
感觉器官不发达。
背神经管两侧有光感受器,称为脑眼(ocelli);
皮肤散在分布感觉细胞。

\newline

雌性异体,生殖腺位于围鳃腔两侧内壁上,无生殖导管。
生殖腺壁裂开,放出生殖细胞,经围鳃腔和腹孔排出,营体外受精。
  
\subsection{脊椎动物亚门(Vertebrata)}
为动物中演化地位最高的类群,包括圆口类、鱼类、两栖类、爬行类、鸟类和哺乳类。
在胚胎期出现脊索、背神经管和咽鳃裂。
背神经管前端分化为脑和眼、耳、鼻等感觉器官,后端分化为脊髓。
骨质脊柱(vertebral column)代替脊索,起支撑作用并保护脊髓。
脊柱前端分化为头骨,保护脑部。
除圆口类外,其余物种头部具有上下颌,支持口部,增强摄食和消化能力。
下颌上举使口闭合为脊椎动物所特有。

\newline

原生水生种类终生有咽鳃裂,以鳃呼吸;
余者仅在胚胎期有咽鳃裂,成体以肺呼吸。
循环系统完善,具有能收缩的心脏。
肾结构复杂。
除圆口类外,均以附肢运动。

\end{document}