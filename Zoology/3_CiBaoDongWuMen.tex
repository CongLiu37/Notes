\documentclass[11pt]{article}

\usepackage[UTF8]{ctex} % for Chinese 

\usepackage{setspace}
\usepackage[colorlinks,linkcolor=blue,anchorcolor=red,citecolor=black]{hyperref}
\usepackage{lineno}
\usepackage{booktabs}
\usepackage{graphicx}
\usepackage{float}
\usepackage{floatrow}
\usepackage{subfigure}
\usepackage{caption}
\usepackage{subcaption}
\usepackage{geometry}
\usepackage{multirow}
\usepackage{longtable}
\usepackage{lscape}
\usepackage{booktabs}
\usepackage{natbib}
\usepackage{natbibspacing}
\usepackage[toc,page]{appendix}
\usepackage{makecell}

\title{刺胞动物门(Cnidaria)}
\date{}

\linespread{1.5}
\geometry{left=2cm,right=2cm,top=2cm,bottom=2cm}

\begin{document}

  \maketitle

  \linenumbers
\section{一般特征}
刺胞动物体型为辐射对称(radial symmetry),有两胚层。
其基本结构为两层的囊。
外层是外胚层形成的皮层(epidermis),内层为内胚层形成的胃层(gastrodermis),两层中间为中胶层(mesoglea)。
胃层内为消化循环腔或腔肠(gastrovascular cavity)。
腔肠只有一个开口,兼作口和肛门。

\newline

刺胞动物有简单的组织分化。
皮层细胞多为立方形,而胃层细胞多为长方形。
组成皮层和胃层的主要细胞为皮肌细胞(epithelio-muscular cell),兼司上皮组织和肌肉组织的功能。
胃层的皮肌细胞还可伸出伪足摄取食物,进行细胞内消化。
腺细胞(gladullar cell)多分布在胃层,可分泌消化酶至腔肠,进行细胞外消化。
口旁的腺细胞分泌粘液,起润滑作用。
间细胞(interstilitial cell)为尚未分化的细胞,多见于皮层。
刺细胞(cnidoblast)为刺胞动物特有,大多分布于皮层。
刺细胞内有刺丝囊(nematocyst),在遇到刺激时,刺丝囊外翻,射出内容物。
部分刺胞动物的刺细胞可射出毒液。

\newline

刺胞动物有网状神经系统(nerve net)。
神经细胞散布于中胶层靠近皮层一侧,与感觉细胞和皮肌细胞相连。
但刺胞动物无神经中枢,神经细胞的信息传递无方向性,信息传导速度较慢。

\newline

刺胞动物兼营细胞外消化和细胞内消化,残渣经口排出。
刺胞动物无呼吸和排泄器官,依靠体表扩散交换气体、排泄废物。

\newline

刺胞动物有水螅型(polyp)和水母型(medusa)两种形态。
水螅型呈圆筒状,适应于固着生活;
水母型呈伞状,适应于漂浮生活。
刺胞动物的无性生殖以出芽为主,亦营有性生殖。
部分种类生活史有世代交替现象,即水螅型个体通过无性生殖产生水母型个体,水母型个体通过有性生殖产生水螅型个体。
  
\section{刺胞动物的分类}
\subsection{水螅纲(Hydrozoa)}
大多生活于海水环境,生活史多存在水螅型和水母型,有世代交替。
胃层无刺细胞,生殖细胞来自皮层。
如于浅海营固着生活的薮枝虫(\textit{Obelia spp.})、淡水生活的水螅(\textit{Hydra spp.})。
  
\subsection{钵水母纲(Scyphozoa)}
全部海产,多为大型水母,水母型发达而水螅型退化。
如海月水母(\textit{Aurelia aurita})、海蛰(\textit{Rhopilema esculentum})。

\subsection{立方水母纲(Cubozoa)}
全部海产,水螅体小而水母体大。
亦有将其作为钵水母纲下一目者。
  
\subsection{珊瑚纲(Anthozoa)}
生活史只有水螅型而没有水母型,且水螅体结构复杂,多为珊瑚礁的造礁生物。  
如海葵、珊瑚虫。

\end{document}