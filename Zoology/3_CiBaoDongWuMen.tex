\documentclass[11pt]{article}

\usepackage[UTF8]{ctex} % for Chinese 

\usepackage{setspace}
\usepackage[colorlinks,linkcolor=blue,anchorcolor=red,citecolor=black]{hyperref}
\usepackage{lineno}
\usepackage{booktabs}
\usepackage{graphicx}
\usepackage{float}
\usepackage{floatrow}
\usepackage{subfigure}
\usepackage{caption}
\usepackage{subcaption}
\usepackage{geometry}
\usepackage{multirow}
\usepackage{longtable}
\usepackage{lscape}
\usepackage{booktabs}
\usepackage{natbib}
\usepackage{natbibspacing}
\usepackage[toc,page]{appendix}
\usepackage{makecell}

\title{刺胞动物门(Cnidaria)}
\date{}

\linespread{1.5}
\geometry{left=2cm,right=2cm,top=2cm,bottom=2cm}

\begin{document}

  \maketitle

  \linenumbers
刺胞动物体型为辐射对称(radial symmetry),有两胚层。
其基本结构为两层的囊。
外层是外胚层形成的皮层(epidermis),内层为内胚层形成的胃层(gastrodermis),两层中间为中胶层(mesoglea)。
胃层内为消化循环腔或腔肠(gastrovascular cavity)。
腔肠只有一个开口,兼作口和肛门。

\newline

刺胞动物有简单的组织分化。
皮层细胞多为立方形,而胃层细胞多为长方形。
组成皮层和胃层的主要细胞为皮肌细胞(epithelio-muscular cell),兼司上皮组织和肌肉组织的功能。
胃层的皮肌细胞还可伸出伪足摄取食物,进行细胞内消化。
腺细胞(gladullar cell)多分布在胃层,可分泌消化酶至腔肠,进行细胞外消化。
口旁的腺细胞分泌粘液,起润滑作用。
间细胞(interstilitial cell)为尚未分化的细胞,多见于皮层。
刺细胞(cnidoblast)为刺胞动物特有,大多分布于皮层。
刺细胞内有刺丝囊(nematocyst),在遇到刺激时,刺丝囊外翻,射出内容物。
部分刺胞动物的刺细胞可射出毒液。

\newline

刺胞动物有网状神经系统(nerve net)。
神经细胞散布于中胶层靠近皮层一侧,与感觉细胞和皮肌细胞相连。
但刺胞动物无神经中枢,神经细胞的信息传递无方向性,信息传导速度较慢。

\newline

刺胞动物兼营细胞外消化和细胞内消化,残渣经口排出。
刺胞动物无呼吸和排泄器官,依靠体表扩散交换气体、排泄废物。

\newline

刺胞动物有水螅型(polyp)和水母型(medusa)两种形态。
水螅型呈圆筒状,适应于固着生活;
水母型呈伞状,适应于漂浮生活。
刺胞动物的无性生殖以出芽为主,亦营有性生殖。
部分种类生活史有世代交替现象,即水螅型个体通过无性生殖产生水母型个体,水母型个体通过有性生殖产生水螅型个体。
  
\section{水螅纲(Hydrozoa)}
大多生活于海水环境,生活史多存在水螅型和水母型,有世代交替。
胃层无刺细胞,生殖细胞来自皮层。

\subsection{水螅目(Hydroida)}
无水母型,无围鞘,个体能运动。
如水螅(\textit{Hydra})。

\subsection{被芽目(Calyptobalstea)}
有角质围鞘。
水母型扁,生殖腺在下伞中辐管下方。
如薮枝虫(\textit{Obelia})。

\subsection{裸芽目(Stylasterina)}
有围鞘,但仅包裹水螅体基部。
水母型钟形,高大于宽。
生殖腺在垂唇上。
如筒螅(\textit{Tubularia})。

\subsection{硬水母目(Trachylina)}
水母型发达,水螅型退化。
平衡囊发生于触手基部内胚层。
如桃花水母(\textit{Craspedacuste})。

\subsection{管水母目(Siphonophora)}
海产,浮游,群体,无围鞘。
如僧帽水母(\textit{Physalia})。

\subsection{水螅珊瑚纲(Hydrocorallina)}
固着群体,基部相连。
皮层分泌石灰质外骨骼。
如多孔螅(\textit{Millepora})。
  
\section{钵水母纲(Scyphozoa)}
全部海产,多为大型水母,水母型发达而水螅型退化。

\subsection{十字水母目(Stauromedusae)}
外伞柄状,无触手囊,无时代交替,固着生活。
如喇叭水母(\textit{Haliclystus})。

\subsection{旗口水母目(Semaeostomae)}
伞部扁平,边缘有触手。
正、间辐管处有触手囊。
有世代交替。
如海月水母(\textit{Aurelia})。

\subsection{根口水母目(Rhizostomae)}
伞半球状,边缘无触手,口腕愈合。
腕口有分枝细管,管外端有吸口。
如海蛰(\textit{Rhopilema})。

\section{立方水母纲(Cubozoa)}
全部海产,水螅体小而水母体大。
独居,有毒。

\subsection{立方水母目(Cubomedusae)}
伞为立方形。
八个触手囊,位于四个正辐管和四个副辐管。
无世代交替。
如灯水母(\textit{Charybadea})。
  
\section{珊瑚纲(Anthozoa)}
生活史只有水螅型而没有水母型,且水螅体结构复杂,多为珊瑚礁的造礁生物。

\subsection{八放珊瑚亚纲(Anthozoa)}
触手、隔膜各八个。
触手羽状,腹面有一条口道沟。

\subsubsection{海鸡冠目(Alcyonacea)}
固着群体,体软,无中轴。
骨骼为散在骨片或骨管。
如海鸡冠(\textit{Alcyonium})。

\subsubsection{海鳃目(Pennatulacea)}
群体,羽状或棒状,柄埋于泥沙。
中轴石灰质或角质。
如海鳃(\textit{Pennatula})。

\subsubsection{柳珊瑚目(Gorgonacea)}
群体,树枝状,中轴石灰质或角质,有散在骨片。
如红珊瑚(\textit{Corallium})。

\subsection{六放珊瑚亚纲(Hexacoralla)}
触手中空,不分枝,有两个口道沟。

\subsubsection{海葵目(Actiniaria)}
体软,触手多,无骨骼。
如细指海葵(\textit{Metridium})。

\subsubsection{角海葵目(Ceriantharia)}
似海葵,体细长,触手两圈。
如角海葵(\textit{Cerianthus})。

\subsubsection{石珊瑚目(Madreporaria)}
群体,外骨骼致密,各体长在外骨骼杯状凹陷。
如脑珊瑚(\textit{Meandrina})。

\subsubsection{角珊瑚目(Antipatharia)}
羽状或树状群体,有黑色角质管轴。
如角珊瑚(\textit{Antipathes})。

\end{document}