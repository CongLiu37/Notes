\documentclass[11pt]{article}

\usepackage[UTF8]{ctex} % for Chinese 

\usepackage{setspace}
\usepackage[colorlinks,linkcolor=blue,anchorcolor=red,citecolor=black]{hyperref}
\usepackage{lineno}
\usepackage{booktabs}
\usepackage{graphicx}
\usepackage{float}
\usepackage{floatrow}
\usepackage{subfigure}
\usepackage{caption}
\usepackage{subcaption}
\usepackage{geometry}
\usepackage{multirow}
\usepackage{longtable}
\usepackage{lscape}
\usepackage{booktabs}
\usepackage{natbib}
\usepackage{natbibspacing}
\usepackage[toc,page]{appendix}
\usepackage{makecell}

\title{墨辩}
\date{}

\linespread{1.5}
\geometry{left=2cm,right=2cm,top=2cm,bottom=2cm}

\begin{document}

  \maketitle
  
  \linenumbers

墨辩,指今传《墨子》中《经上》《经下》《经说上》《经说下》《大取》《小取》六篇,其文风古奥难解,且多文字错讹。
其内容繁杂,有对墨子理论之补充,有涉及自然科学技术者,有涉及逻辑和知识理论以及其它哲学问题者。
其中逻辑和知识论系中国古代哲学于此类问题之主要成就。
\newline
墨辩多驳斥名家理论,亦有涉及庄子理论者,故其时代当在名家诸子之后,为墨家后学。

\section{同异问题}
墨辩谓“同”有四种。
《墨子·经上》云:

\textit{同:重,体,合,类。}

《墨子·经说上》解释云:

\textit{同:二名一实,重同也;不外于兼,体同也;俱处于室,合同也;有以同,类同也。}

此谓“同”有“重”“体”“合”“类”四种意义。
两概念虽各有一名,但所含元素完全相同,即为“重”。
一物之不同部分,虽不相同,但皆属于同一整体,此为“体”。
例如一个人之手足固不相同,但皆为此人之一部分,故为“体”。
两物若处于同一时空范围,则为“合”。
例如两人共处一室。
两物若有某一相同属性,则为“类”。
例如一白马和一白石,二者皆有“白”之属性,属于“白”之概念,即为“类”意义上的相同。
总之,这四种“同”,唯“重”就概念层面而言,其余皆针对具体事物。

\newline

“同”之四种意义既明,则“异”即为“同”之条件之缺乏。
故“异”亦有四种意义,即“不重”、“不体”、“不合”、“不类”。
如此,“同”与“异”之确切意义既明,则惠施“万物毕同毕异”之诡辩可破矣。
万物皆有相同属性,系“类”之同;有不同属性则系“不类”之异。
万物有相同之处与有不同之处本不互相矛盾。
执其中一点而言万物毕同或毕异皆无意义。

\newline

更进一步,墨辩论概念之间的关系。
《墨子·经说下》谓:

\textit{牛与马惟异,以牛有齿,马有尾,说牛之非马也,不可。是俱有,不偏有,偏无有。曰牛之与马不类,用牛有角、马无角,是类不同也。若举牛有角、马无角,以是为类之,不同也,是狂举也。犹牛有齿、马有尾,或不非牛而非牛也,则或非牛或牛而牛也可。
故曰:牛马非牛也未可,牛马牛也未可。则或可或不可,而曰“牛马牛也未可”亦不可。且牛不二,马不二,而牛马二。则牛不非牛,马不非马,而牛马非牛非马,无难。}

盖两概念之同异,需考察其定义性条件,否则即生错误。
兹以“牛”和“马”两概念为例。
若因“牛有齿,马有尾”而判“牛”和“马”为不同概念,即生谬误。
这是因为牛和马皆有齿尾,而非牛有齿马无齿或者马有尾牛无尾。
且有齿者未必为牛,有尾者未必为马。
故“牛有齿”非“牛”之定义性条件,“马有尾”亦非“马”之定义性条件。
依今日之用语,“有齿”非“牛”之充分必要条件,“有尾”亦非“马”之充分必要条件。
“牛有角,马无角”与此同理。
据此类论据言牛马之异者,为“狂举”也。

\newline

更进一步,引文论“牛马”、“牛”、“马”三概念之关系。
“牛马”为“牛”和“马”二概念之相并,自然不与“牛”或“马”相等。
但“牛马”中部分元素与“牛”中元素相同,其余元素与“马”中元素相同,不可简单概括为“牛马”是“牛”或“牛马”不是“牛”。
此处墨家后学已注意到概念之间虽不相等,但可相交或互相包含,较公孙龙为一大进步矣。

\section{坚白问题}
墨辩驳斥公孙龙坚白相离之理论,持唯物主义立场。
盖公孙龙欲论属性概念之独立存在,借知觉历程证之。
墨辩作者谓知觉历程不能证立属性之独立,公孙龙之论证无效。
盖坚与白为石之固有属性。
视之得其白而不得其坚,却无碍于石之坚;拊之得其坚而不得其白,却无碍于石之白。
坚与白皆在于石,并未相离,故知觉历程不能证坚白之相独立。

\section{“名”与“谓”}
《墨子·经上》论“名”与“谓”,颇为精严:

\textit{名,达、私、类。谓,移、举、加。}

\newline

《墨子·经说上》解“名”曰:

\textit{名:物,达也,有实必待之名也。命之马,类也,若实也者,必以是名也。命之臧,私也,是名也,止于是实也。}

“达”即万物。
盖万物皆为物,皆属于“物”之概念。
此概念称为“达”。
“类”指“达”中包含的各种概念,譬如“牛”“马”“坚”“白”等。
“私”则为指代个体事物之名,其效力范围仅为某一物。

\newline

《墨子·经说上》解“谓”曰:

\textit{谓:命狗犬,移也。狗犬,举也。叱狗,加也。}

古语“狗”指“未成豪之犬”。
三“谓”论“是”的三种含义。
“移”指包含关系,如“狗”之概念包含于“犬”这一范围更大的概念,即狗“是”犬。
“举”指定义关系,如将“狗”定义为符合“未成豪”这一条件的“犬”,即狗“是”未成豪的犬。
“加”指将个体归于某一概念,如言这“是”一条狗。

\section{论“辩”}
墨辩反对庄子“泯是非”之立场,谓辩论有胜负,且此胜负有价值。
《墨子·经下》云:

\textit{谓辩无胜,必不当,说在辩。}

《墨子·经说下》解之曰:

\textit{辩也者,或谓之是,或谓之非,当者胜也。}

此谓“辩”乃对某一命题之肯定与否定之争,必有胜者。

\newline
《墨子·小取》又载:

\textit{夫辩者,将以明是非之分,审治乱之纪,明同异之处,察名实之理,处利害,决嫌疑。}

此谓辩论可分清是非、区别治乱、分辨事物之同异、考察事物之道理、分析利害、解决疑虑。
此系借肯定辩之效果来肯定辩之价值,上承墨子之功利主义立场。

\newline

《墨子·经上》解释论证中的条件,云:

\textit{故,所得而后成也。}

《经说上》解之:

\textit{故:小故,有之不必然,无之必不然。体也,若有端。大故,有之必然,无之必不然。若见之成见也。}

“小故”即必要条件,“大故”即充分必要条件。

\newline

《墨子·小取》论辩论之方法:

\textit{辟也者,举也物而以明之也。侔也者,比辞而俱行也。援也者,曰:“子然,我奚独不可以然也?”推也者,以其所不取之同于其所取者,予之也。}

“辟”即举例说明,“侔”即类比,“援”即以对方所肯定者推演出与对方所否定者,“推”即指出对方所肯定者与所否定者之相同处。

\newline

最后,《墨子·小取》谓可用于描述一范围较小概念的词语,可能不能用于描述其所属的范围较大的概念。
《小取》云:

\textit{白马,马也;乘白马,乘马也。骊马,马也;乘骊马,乘马也。获,人也;爱获,爱人也。臧,人也;爱臧,爱人也。此乃是而然者。获之亲,人也;获事其亲,非事人也。其弟,美人也;爱弟,非爱美人也。车,木也;乘车,非乘木也。船,木也;人船,非人木也。盗人人也;多盗,非多人也;无盗,非无人也。奚以明之?恶多盗,非恶多人也;欲无盗,非欲无人也。}

此于近代已为常识,而古代鲜有论述者。

\section{知识问题}
墨辩中涉及知识论者,兹分述之。

\newline

《墨子·经上》谓:
\textit{知,材也。}

《经说上》解之曰:

\textit{知材,知也者,所以知也,而不知,若明。}

此论认知能力和知识。
知识之获得依赖于认知能力,但有认知能力未必有知识,尚有能力运行之问题。

\newline

《经上》分论感觉能力和理解能力:

\textit{知,接也;恕,明也。}

《经说上》解之曰:

\textit{知:知也者,以其知过物而能貌之,若见。恕:恕也者,以其知论物而其知之也著,若明。}

此处“知”指感觉能力,其运行产生感性印象,故曰“能貌之”。
“恕”指理解能力,负责整理感性资料,所产生的知识较为明晰确定,故曰“其知之也著”。

\newline

此外,墨辩论获得知识之方法。
《墨子·经上》曰:

\textit{知:闻、说、亲。名实合为。}

《经说上》解之曰:

\textit{知:传受之,闻也;方不障,说也;身观焉,亲也。所以谓,名也;所谓,实也;名实耦,合也;志行,为也。}

“闻”即由他人传授所得知识;“说”即由推理论证所得知识;“亲”即由实践所得知识。
此为知识之来源。
“名”为用于陈述者;“实”为被陈述者;“名”与“实”相符,即为“合”。
利用知识进行实践,谓之“为”。
此系知识之表述和实践。
  
\end{document}