\documentclass[11pt]{article}

\usepackage[UTF8]{ctex} % for Chinese 

\usepackage{setspace}
\usepackage[colorlinks,linkcolor=blue,anchorcolor=red,citecolor=black]{hyperref}
\usepackage{lineno}
\usepackage{booktabs}
\usepackage{graphicx}
\usepackage{float}
\usepackage{floatrow}
\usepackage{subfigure}
\usepackage{caption}
\usepackage{subcaption}
\usepackage{geometry}
\usepackage{multirow}
\usepackage{longtable}
\usepackage{lscape}
\usepackage{booktabs}
\usepackage{natbib}
\usepackage{natbibspacing}
\usepackage[toc,page]{appendix}
\usepackage{makecell}

\title{唯识之学}
\date{}

\linespread{1.5}
\geometry{left=2cm,right=2cm,top=2cm,bottom=2cm}

\begin{document}

  \maketitle

  \linenumbers

大乘佛教中唯识一支兴起于3-4世纪,由弥勒(Maitreya)立说,经无著(Asanga)及世亲(Vasubandhu)进一步发展,渐有系统学说。
此派学说论著甚繁,此处仅撮其大要。

\section{妙有}
唯识言“有”,非对象之“有”,故称“妙有”。
为解“妙有”之含义,需先回顾表明佛教基本理念的三个命题,即
(1)一切对象,或者独立实有皆是虚妄;
(2)主体需破除一切对象;
(3)破除对象的主体不受一切条件决定,达到充分自由,即中观所言之“空”。
如此,则至少有三种道理可讲。
首先,对象之所以为虚妄,必然有一定的道理。
其次,主体破除虚妄,其中必然有一定的道理可讲。
最后,“空”之境界本身的证立,亦需有一定的道理。
这三种道理必须为“有”,但又不能成为对象或者独立实有,故称“妙有”。
唯识一派,遂根据“妙有”建立“三自性”。

\section{三自性}
三自性,见诸《解深密经》,即“遍计所执性”、“依他起性”、“圆成实性”。

\newline

“遍计所执性”指对象之所以为虚妄的道理。
“遍计”即意识活动。
人之所以会认为对象是实有,是因为人的意识活动赋予对象实有性,即为“所执”。
一切对象依主体而立,即为意识之产物,故皆为虚妄。

\newline

“依他起性”指破除虚妄之道理。
“依他起”谓一切法皆因缘生,皆非独立实有。
人未觉时,对一切法皆执以为实有,即“遍计所执”。
一旦能观因缘,知“依他起性”,遂知一切皆依因缘起,一切皆非实有。

\newline

既知“依他起”,“遍计所执”得破,遂达“圆成实性”,即破除虚妄之后所显现之真。

\newline

三自性作为三种道理,并非具有独立实有性的对象,而是依附主体而立,仅表明主体的三种活动。
故此三理虽可说,但非对象意义之“有”。
于是《解深密经》又立“三无性”,与三自性相对应,以点明三自性依附于主体,而无独立性或脱离主体之存在性。

\section{百法}
三自性既立,则整个对象界遂可得一解释,即揭示一切对象是如何由主体之活动所产生的。
此学说之大纲,见诸世亲所著《百法明门论》。

\newline

《百法明门论》首先将“一切法”划分为“心法”、“心有所法”、“色法”、“心不相应行法”、“无为法”五类。
其中“心法”为“一切最胜”,即一切余法皆由心法所生。
“心法”又分为八,即为“八识”。
“心有所法”则分为五十一,包括种种意识活动和心理活动。
“色法”分为十一,包括经验活动和经验活动之对象。
“心不相应法”分为二十四,皆属形式概念一类。
“无为法”则分为六,包括各种超越对象界之境界。

\newline

此种对对象界的诠释,其中包含种种繁琐枝节。
一言以蔽之,即一切法皆源于“心法”。
故其重点在于“八识”。

\section{八识}
“八识”,即眼、耳、鼻、舌、身、意、末那(Mana)、阿赖耶(Alaya)。
前五者为感官能力,意指心理活动,“末那”即意念。
具有理论意义的是第八识阿赖耶。

\newline

阿赖耶指“个体自我”,即众生皆各有一阿赖耶。
此说类似灵魂,但佛教始终否定实有,故阿赖耶仅取主体意义,具有除存有性外灵魂的一切属性。
更进一步,阿赖耶既为个别自我,则众生之一切特性,皆蕴含于其中。
此即阿赖耶持藏种子之说。

\newline

阿赖耶既为个别自我,其余诸识皆依阿赖耶而运行;
而一切对象皆起源于“识”;
于是阿赖耶遂为一切对象之根源。
换言之,一切法皆依阿赖耶而立。

\section{阿赖耶之染净问题}
至此,阿赖耶既为个体自我,又是对象界之根源。
则此一识与佛教一向提倡的舍离世界,或者说解脱,究竟有何关系?
换言之,阿赖耶是否有价值?
在唯识学派中,此为一有争议的问题。
根据其回答,唯识之学可分为三支。

\newline

第一支以《摄大乘论》《决定藏论》为据,谓阿赖耶为染。
盖阿赖耶既为对象界之根源,则欲脱离一切对象,显现主体之自由,须从根本下手,破除阿赖耶识。
此一支另立“阿摩罗”(Amala)为解脱之根本,即肯定阿赖耶以外的主体性。

\newline

第二支以《十地经论》为据,谓阿赖耶为净,即为真识。
换言之,阿赖耶本身即为解脱之动力,表最终的主体性。

\newline

第三支以《成唯识论》为据,强调阿赖耶中有某种“种子”,为解脱之动力。
阿赖耶本身为中立,即非染非净,却可接受种子的影响而成为种种状态。
换言之,阿赖耶之所以表个别自我,是因为现行之活动和种子的相互影响,即种子(自我之状态)影响自我之活动,自我之活动又反过来影响自我之状态。
而解脱则需要某种特定的种子,即所谓的“无漏种子”。  
  
\end{document}