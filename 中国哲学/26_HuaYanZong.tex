\documentclass[11pt]{article}

\usepackage[UTF8]{ctex} % for Chinese 

\usepackage{setspace}
\usepackage[colorlinks,linkcolor=blue,anchorcolor=red,citecolor=black]{hyperref}
\usepackage{lineno}
\usepackage{booktabs}
\usepackage{graphicx}
\usepackage{float}
\usepackage{floatrow}
\usepackage{subfigure}
\usepackage{caption}
\usepackage{subcaption}
\usepackage{geometry}
\usepackage{multirow}
\usepackage{longtable}
\usepackage{lscape}
\usepackage{booktabs}
\usepackage{natbib}
\usepackage{natbibspacing}
\usepackage[toc,page]{appendix}
\usepackage{makecell}

\title{华严宗}
\date{}

\linespread{1.5}
\geometry{left=2cm,right=2cm,top=2cm,bottom=2cm}

\begin{document}

  \maketitle

  \linenumbers

华严宗依《大方广佛华严经》立教,开山者为贤首法师(643-712)。

\section{法界}
“界”即领域。
万法皆在同一领域之中,遂成“法界”。
法界系一切法之总体,其性质与每一法之性质不在同一理论层级。
华严宗观法界,着眼于万法合为一领域时所具有的性质,而非每一法所具有之性质。
法界可分四层解说,即“四法界”。

\newline

第一为事法界,即专就现象本身而言,以差别为特色。
第二为理法界,即专就现象之理,即其本质或真相而言,以无二无差别为特色。
第三为理事无碍法界,即现象与真相不可分离。
盖现象非真相,却由真相所生;
真相非现象,却在现象中体现。
此无二真相生万法,即为主体。
第四为事事无碍法界。
盖一切现象皆由同一主体所生,故有共同之处,遂交融相通,即所谓“事事无碍”。
故由任一法可见主体,亦可见其余一切法,即“一摄一切,一切摄一”之义。
就主体一面而言,主体之不同境界亦彼此相通无碍,即主体总可由此境界到达彼境界,遂有上升和堕落之能力,主体自由得立。

\newline

至此,华严宗直面主体性之理论特色显现矣。
更进一步,有“十玄门”之说,即以十个论点说明法界,强调一切法彼此交融相通。
其数有十,乃依《华严经》而立,非理论要求。
其中值得注意的唯心回转善成门,点明最高主体。
《华严一乘教分齐章》云:

\textit{九者,唯心回转善成门。以上诸义,唯一是如来藏为自性清净心转也。}

“以上诸义”指前文所列举之十义,指代一切现象。
一切法皆随“自性清净心转”,即一切法皆受自性主体决定。
至此,华严宗立最高主体及主体自由,与天台教义大旨相同。

\section{六相圆融}
“六相”,即“总”“别”“同”“异”“成”“坏”六个概念,分为三对。
贤首论六相,旨在说明一切概念皆互相依存,有此方能有彼,有彼方能有此。
换言之,一概念必须依赖和其它概念的关系,方能有意义。
比如有“总”方能有“别”,有“别”方能有“总”;
倘若无“总”之概念,则“别”即不存在,反之亦然。
于“同”“异”、“成”“坏”,亦是如此。

\newline

一切概念皆互相依赖,佛法亦是如此。
一切佛法,系以种种言论演说同一道理,无二、无三。
此为真常“一乘”之义,《华严一乘教义分齐章》云:

\textit{唯智境界非事识,以此方便会一乘。}

“非事识”即非经验认知。
盖一切现象皆源自主体,皆非真相,故断不能在现象界寻求“唯智境界”。  
  
\end{document}