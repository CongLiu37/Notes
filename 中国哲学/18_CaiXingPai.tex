\documentclass[11pt]{article}

\usepackage[UTF8]{ctex} % for Chinese 

\usepackage{setspace}
\usepackage[colorlinks,linkcolor=blue,anchorcolor=red,citecolor=black]{hyperref}
\usepackage{lineno}
\usepackage{booktabs}
\usepackage{graphicx}
\usepackage{float}
\usepackage{floatrow}
\usepackage{subfigure}
\usepackage{caption}
\usepackage{subcaption}
\usepackage{geometry}
\usepackage{multirow}
\usepackage{longtable}
\usepackage{lscape}
\usepackage{booktabs}
\usepackage{natbib}
\usepackage{natbibspacing}
\usepackage[toc,page]{appendix}
\usepackage{makecell}

\title{才性派}
\date{}

\linespread{1.5}
\geometry{left=2cm,right=2cm,top=2cm,bottom=2cm}

\begin{document}

  \maketitle

  \linenumbers

自东汉末年至魏晋,中国知识分子视品评人物为一大事。
此种风气之形成,固与当时的官吏选拔制度有关,但于品评人物中显现智慧,亦是当时的思想倾向。
此即魏晋玄学中才性一派的根源。

\newline

此辈鉴赏人物,品评其才性,以此判断其德性操守乃至于政治成败。
言者全评其主观感受评价;
闻者或同意,或不同意,鲜有与之辩诘者。
此种观赏之态度实为老庄“观赏世界”境界的衍生,却误执形躯,不解老庄真意。

\newline

才性派的代表性资料,当属刘劭(?-?)所著《人物志》,其时代大致为魏明帝时(227-239)。
其大要为一切人事均由不可改变的才性决定,否定人之自觉努力,实为决定论立场。

\section{情性}
《人物志》开篇论“情性”。
《九征》载:

\section{盖人物之本,出乎情性。情性之理,甚微而玄;非圣人之察,其孰能究之哉?凡有血气者,莫不含元一以为质,禀阴阳以立性,体五行而着形。苟有形质,犹可即而求之。}


引文言情性,仅泛泛而论。
然既诉诸“元一”“阴阳”“五行”,可见情性为被决定的材质,属于已决定者,非人之自觉努力所能改变者。
而后又谓可根据“形质”求之,足见刘劭所论情性,或者说才性,仅为一客观事实。

\newline

才性既为客观事实,故无所谓改造或培养,只有如何了解某人之才性的问题。
故刘劭有“九征”之说,即根据九种表现,判断某人之才性有何特征。
《九征》云:

\section{平陂之质在于神,明暗之实在于精,勇怯之势在于筋,彊弱之植在于骨,躁静之决在于气,惨怿之情在于色,衰正之形在于仪,态度之动在于容,缓急之状在于言。其为人也:质素平澹,中叡外朗,筋劲植固,声清色怿,仪正容直,则九征皆至,则纯粹之德也。九征有违,则偏杂之材也。}

对此中九项,刘劭皆泛泛而论,文人铺排成分居多。
值得注意的是刘劭据此评判人物,谓“九征皆至”为“纯粹之德”,“九征有违”为“偏杂之材”。
此处有两个重要问题。
首先,言某“征”之“至”与“违”,是依何标准?
换言之,如何为“至”?
如何为“违”?
其次,此标准如何涉及价值?
才性既然仅为客观事实,则根据某种标准对其进行划分亦只是一事实,不能据此言各类才性之好坏而将其判为“纯粹之德”或“偏杂之材”。
对此等大问题,刘劭全然不顾,反而直接据此言人物等级,可见其疏漏。

\section{人物等级}
《九征》划分人物等级,曰:

\section{兼德而至,谓之中庸;中庸也者,圣人之目也。具体而微,谓之德行;德行也者,大雅之称也。一至,谓之偏材;偏材,小雅之质也。一征,谓之依似;依似,乱德之类也。一至一违,谓之间杂;间杂,无恒之人也。无恒、依似,皆风人末流;末流之质,不可胜论,是以略而不概也。}

盖最高等级的情性为“中庸”,于各方面皆得圆满,故谓“兼”“至”。
有中庸之情性者为圣人。
“德行”于各方面皆得圆满,但程度不及“中庸”。
此类人为“大雅”。
“一至”仅在某方面得圆满。
此类人为“小雅”。
“一征”亦是在某方面得圆满,但圆满程度不及“一至”。
此类人为“乱德”。
“间杂”则于各个方面偶尔圆满,偶尔不圆满,故此类人为“无恒”。
此外还有更低的等级,兹不赘述。

\newline

等级的划分由才性决定,遂为客观事实,非人之自觉努力所能改变。
但在各个等级内,人仍需做一定的努力以实现其才性。
《体别》载:

\textit{夫学所以成材也,疏所以推情也;偏材之性,不可移转矣。虽教之以学,材成而随之以失;虽训之以恕,推情各从其心。信者逆信,诈者逆诈;故学不道,恕不周物;此偏材之益失也。}

此节直言才性之不可移转,决定论立场甚明。
其所论之“学”,仅在于固有才性之发挥。
故学成之时,才性之优缺点一起显现矣。

\section{才性之作用}
才性对人事起决定性作用。
《人物志》根据不同才性的特点,判断人物之道德。
《九征》载:

\textit{若量其材质,稽诸五物;五物之征,亦各着于厥体矣。其在体也:木骨、金筋、火气、土肌、水血,五物之象也。五物之实,各有所济。是故:骨植而柔者,谓之弘毅;弘毅也者,仁之质也。气清而朗者,谓之文理;文理也者,礼之本也。体端而实者,谓之贞固;贞固也者,信之基也。筋劲而精者,谓之勇敢;勇敢也者,义之决也。色平而畅者,谓之通微;通微也者,智之原也。}

此处所谓“材质”即为才性,并将其归于“五物”,可见才性全为客观事实。
而下文根据才性的不同论仁义礼智信诸德,不过谓道德由才性决定。
此类论述,屡见于《人物志》,用语甚繁,兹不赘引。

\newline

政治成败亦由才性决定。
《流业》列举十二种“人流之业”,并分别论述其政治得失,乃至于可任何种官职。
其文散漫而繁琐,兹不赘述。

\newline

认知活动由才性决定。
《材理》谓:

\textit{夫理多品则难通,人材异则情诡;情诡难通,则理失而事违也。}

此谓“理”有多种,人之才性亦有多种,人能通晓何种理需由其才性决定。
后文列举四种理以及能知理的四种才性,并分别论述其优缺点,其旨不过谓才性决定认知活动。

\newline

总而言之,魏晋玄学中才性一派持决定论立场,以为一切人事之成败均由才性决定,否定人之自觉努力。

\end{document}