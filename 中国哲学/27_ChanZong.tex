\documentclass[11pt]{article}

\usepackage[UTF8]{ctex} % for Chinese 

\usepackage{setspace}
\usepackage[colorlinks,linkcolor=blue,anchorcolor=red,citecolor=black]{hyperref}
\usepackage{lineno}
\usepackage{booktabs}
\usepackage{graphicx}
\usepackage{float}
\usepackage{floatrow}
\usepackage{subfigure}
\usepackage{caption}
\usepackage{subcaption}
\usepackage{geometry}
\usepackage{multirow}
\usepackage{longtable}
\usepackage{lscape}
\usepackage{booktabs}
\usepackage{natbib}
\usepackage{natbibspacing}
\usepackage[toc,page]{appendix}
\usepackage{makecell}

\title{禅宗}
\date{}

\linespread{1.5}
\geometry{left=2cm,right=2cm,top=2cm,bottom=2cm}

\begin{document}

  \maketitle

  \linenumbers

禅宗开山祖师为慧能(638-713)。
其立教直揭主体自由之义,不立文字,不依经论,有否定宗教传统之倾向。

\section{见性成佛}
《六祖坛经》云:

\textit{般若之智亦无大小,为一切众生自心迷悟不同;迷心见外,修行觅佛,未悟自性,即是小根。若开悟顿教,不执外修,但于自心常起正见,烦恼尘劳,常不能染,即是见性。}

引文“自性”一词与佛教习语不同。
佛教言自性,常指独立实有,故云万法皆无自性,即是般若空义。
禅宗言自性,指“自己之性”,即主体性。
盖般若智慧为主体本有,其发用亦决定于主体,不受一切外在之影响。
一旦开悟自性,主体显现,即是成佛;
否则即是众生。
或迷或悟,全然取决于自身,故《六祖坛经》曰:

\textit{自性迷,即是众生;自性觉,即是佛。}

因此,主体之升降,关键为悟。
自己悟,立时成佛;
自己迷,即刻堕落。
而成佛所悟者,仍是主体自身,故云“见性成佛”。

\section{定慧不二}
慧能立教,以主体之悟为核心,遂否定修持禅定之功夫,即有“定慧不二”之说。
盖“定”为主体之境界,“慧”为主体之功能。
主体在此境界,有此功能;
反之亦然。
故“定”与“慧”不可分离。
俗众打坐诵经,烧香拜佛,斋戒施舍,以为禅定,欲求功德。
然种种修持,皆形躯之事,自不能有功于主体之悟。
《六祖坛经》所载,常有否定宗教戒律之辞。
兹引数节如下:

\textit{慧能后至曹溪,又被恶人寻逐。乃于四会,避难猎人队中,凡经一十五载,时与猎人随宜说法。猎人常令守网。每见生命,尽放之。每至饭时,以菜寄煮肉锅。或问,则对曰:“但吃肉边菜”。}

谨以此节,奉送一切素食主义及动物保护主义者。

\textit{公曰:“弟子闻达摩初化梁武帝,帝问云:‘朕一生造寺度僧,布施设斋,有何功德?’达摩言:‘实无功德’。弟子未达此理,愿和尚为说”。师曰:“实无功德。勿疑先圣之言。武帝心邪,不知正法,造寺度僧,布施设斋,名为求福,不可将福便为功德。功德在法身中,不在修福”。}

“公”指韦璩,《六祖坛经》中称为韦刺史。
“功德在法身中”乃就主体之悟而言。
主体不悟,种种修持,皆无功德。

\textit{师曰:汝师若为示众?对曰:常指诲大众,住心观净,长坐不卧。师曰:住心观净,是病非禅。长坐拘身,于理何益?听吾偈曰:生来坐不卧,死去卧不坐。一具臭骨头,何为立功课。}

此为慧能与神秀(606-706)弟子志诚之问答。
慧能借机讥讽禅定修持。

\newline

禅宗既强调主体之悟,则一切外在皆为余事。
不惟禅定无用,一切经论,亦不过为演说方便,非能助力于悟也。
故禅宗立教,不依经论,自称教外别传,实为佛教诸宗中理论地位最高者。
兹引《六祖坛经》数节如下:

\textit{三世诸佛,十二部经,在人性中,本自具有。不能自悟,须求善知识,指示方见。若自悟者,不假外求。若一向执谓须他善知识,望得解脱者,无有是处。何以故?自心内有知识自悟。若起邪迷,妄念颠倒,外善知识虽有教授,救不可得。若起真正般若观照,一刹那间,妄念俱灭。若识自性,一悟即至佛地。}

经论的作用在于告诉人们寻求主体性之悟,而不是经论本身使人解脱。

\textit{师自黄梅得法,回至韶州曹侯村,人无知者。时有儒士刘志略,礼遇甚厚。志略有姑为尼,名无尽藏,常诵《大涅槃经》。师暂听,即知妙义,遂为解说。尼乃执卷问字。师曰:“字即不识,义即请问”。尼曰:“字尚不识,焉能会义”?师曰:“诸佛妙理,非关文字”。}

据传,慧能不识字,是个文盲。然悟见自性,即能成佛,非干文字。

\textit{经有何过,岂障汝念?只为迷悟在人,损益由己。口诵心行,即是转经;口诵心不行,即是被经转。听吾偈曰:心迷法华转,心悟转法华。诵经久不明,与义作仇家。无念念即正,有念念成邪。有无俱不计,长御白牛车。}

有僧法达,常诵《法华经》。引文为慧明训教法达之言。盖主体之迷,唯有主体能解,诵经不能有所加也。

\section{修习之法}
禅宗之旨在于“见性成佛”。
成佛只在主体一念之间,一切外在功夫皆不能于此有所影响。
如此,则实不能有任何修习之法,且主体在对象界的活动皆为中性,无功亦无过。
《六祖坛经》云:

\textit{善知识,我此法门,从上以来,先立无念为宗,无相为体,无住为本。无相者,于相而离相。无念者,于念而无念。无住者,人之本性。}

“无相”谓主体处于客体世界之中而不受客体限制,故可“于相而离相”;
“无念”谓主体在对象界的种种活动无碍于主体超越世界及显现主体自由,故可“于念而无念”;
“无住”谓主体不陷溺于任何对象,此为主体本有能力,故判为“人之本性”。

\newline

总之,主体之悟和主体的其它活动并不相冲突。
因此,成佛功夫,只在念念不失主体上,而外在活动皆为中性。
主体之悟不需要任何特殊努力,禅师们也只是过着寻常生活,做着寻常事情,此即“不修之修”。

\newline

禅宗理论发展至此,主体在对象界显现,在客观世界中的奔忙劳碌不能影响成佛。
然视世界为虚妄为佛教不可让步之根本原则。
一切理论,若违背否定世界之精神,便不可称为佛教。
禅宗判世界为中性,于成佛无益无害,已是佛教对中国心灵所能作出的最大程度之妥协,代表中国佛教哲学之顶峰。
  
\end{document}