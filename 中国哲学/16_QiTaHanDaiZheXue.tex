\documentclass[11pt]{article}

\usepackage[UTF8]{ctex} % for Chinese 

\usepackage{setspace}
\usepackage[colorlinks,linkcolor=blue,anchorcolor=red,citecolor=black]{hyperref}
\usepackage{lineno}
\usepackage{booktabs}
\usepackage{graphicx}
\usepackage{float}
\usepackage{floatrow}
\usepackage{subfigure}
\usepackage{caption}
\usepackage{subcaption}
\usepackage{geometry}
\usepackage{multirow}
\usepackage{longtable}
\usepackage{lscape}
\usepackage{booktabs}
\usepackage{natbib}
\usepackage{natbibspacing}
\usepackage[toc,page]{appendix}
\usepackage{makecell}

\title{其它汉代哲学}
\date{}

\linespread{1.5}
\geometry{left=2cm,right=2cm,top=2cm,bottom=2cm}

\begin{document}

  \maketitle

  \linenumbers
  
汉代哲学中值得注意者,尚有《淮南子》以及扬雄、王充之说。

\section{《淮南子》}
《淮南子》为淮南王刘安(179 BC-122 BC)及其门客编著,共二十一篇,其中《要略》为全书总结。
总的来说,《淮南子》内容杂乱,虽处处以道家自居,却不承老庄之学,为汉代“道家”的典型代表。
现就《要略》篇,论《淮南子》主要思想。

\newline

《要略》释《原道》云:

\textit{《原道》者,卢牟六合,混沌万物,象太一之容,测窈冥之深,以翔虚无之轸,托小以苞大,守约以治广,使人知先后之祸富,动静之利害。诚通其志,浩然可以大观矣。欲一言而寤,则尊天而保真;欲再言而通,则贱物而贵身;欲参言而究,则外物而反情。执其大指,以内治五藏,瀸濇肌肤,被服法则,而与之终身,所以应待万方,鉴耦百变也。若转丸掌中,足以自乐也。}

引文所言之“道”,大抵为一形而上学概念。
“卢牟六合”诸语看似神秘浩大,实不知所云。
“先后之祸福”“动静之利害”,涉及权术阴谋,俨然韩非子语。
此种法家论调于《主术》《兵略》诸篇尤多,兹不赘述。
后文所谓“保真”“贵身”“反情”诸语,皆就形躯而言。
可见,《淮南子》作者根本不理解老庄“观赏世界”之学说。

\newline

此外,《淮南子》多阴阳家语及言天人感应者。
如《要略》释《天文》曰:

\textit{天文者,所以和阴阳之气,理日月之光,节开塞之时,列星辰之行,知逆顺之变,避忌讳之殃,顺时运之应,法五神之常,使人有以仰天承顺,而不乱其常者也。}

此俨然阴阳术士口吻。
而言仰天承顺,实将天视为价值根源。
《精神》更是大谈天人相应。
《要略》释《精神》云:

\textit{精神者,所以原本人之所由生,而晓寤其形骸九窍,取象与天,合同其血气,与雷霆风雨比类其喜怒,与昼宵寒暑并明,审死生之分,别同异之迹,节动静之机,以反其性命之宗,所以使人爱养其精神,抚静其魂魄,不以物易己,紧守虚无之宅者也。}

总之,《淮南子》代表汉代学者心目中的道家,其用语浮夸,内容杂乱,几无可取者。
此系反映汉代老庄之学变质堕落的代表资料,故于此加以论述。

\section{扬雄}
汉代时,儒道两家有融合之趋势。
盖汉儒以宇宙论和形而上学构建价值理论,而先秦道家本有形而上学旨趣,二者相通。
然而,汉代学者不解孔孟心性理论,亦不知老庄观赏世界之境界,故其融合儒道的理论杂乱空虚。
这一趋势日后呈现于为魏晋清谈之流,实为中国哲学之大衰退。
扬雄(53 BC-18 BC)便是这一进程的早期代表。
其主要著作有《法言》《太玄》。
其中《法言》代表儒家立场,而《太玄》接近道家。
现分别加以论述。

\newline

《法言》中,扬雄以孔孟之继承者自居,有“治己以仲尼”“窃自比于孟子”之语;
于老庄则褒贬各半,谓其学虽有可取之处,但存在“捶提仁义,绝灭礼学”的严重错误,即否定文化之价值;
于包括荀子在内的其余诸子则皆持否定态度。
然而,扬雄不解孔孟心性论,对成德治学等问题的论调反而接近荀子立场。
《法言·学行卷第一》载:

\textit{或曰:“学无益也,如质何?”曰:“未之思矣。夫有刀者礲诸,有玉者错诸,不礲不错,焉攸用?礲而错诸,质在其中矣。否则辍。”}

刀玉之喻,强调外在改造的重要性,分明荀子立场。

\newline

扬雄不解心性论,遂有“善恶混”之说。
《法言·修身卷第三》载:

\textit{人之性也善恶混。修其善则为善人,修其恶则为恶人。气也者,所以适善恶之马也与?}

此谓人性中既有善的部分,又有恶的部分。
扩充前者即为善人,扩充后者即为恶人。
“善恶之马”一语,将人之善恶归因于天赋气禀。
此全为常识口吻,可见扬雄于心性论无甚真知。

\newline

此外,就政治而言,扬雄持儒者一贯立场,将政治生活视为个人德性之延申,以修身为行政之本,强调德性教化,并极力宣扬孟子之“仁政”。
由此,一切政治制度、法律条文皆不值得重视。

\newline

《法言》主要代表扬雄身为儒者的一面,而《太玄》则更多的表现出道家的影响。
扬雄论“玄”,谓其为一形而上的存在,自身非经验对象,但又是一切经验对象的根源。
《太玄》中能与老庄扯上关系的,也就“玄”这一形而上学观念了。
总的来说,《太玄》杂取道家之形而上观念和《易传》之神秘主义倾向,却又不持纯宇宙论立场,用语浮夸,实为无聊之作。

\newline

总的来说,扬雄并非一合格的思想家。
因其代表汉代儒道混合之趋势,故就哲学史研究而言,不太好将其忽略。

\section{王充}
汉代知识分子多受阴阳五行学说的影响,大谈天人感应。
王充(27-97)则一反两汉传统。
观其《论衡》之书,反阴阳术数态度鲜明,但相关理论简陋,于先秦诸子不解其深切处,又不能自成体系。
现论王充思想大要。

\newline

王充著书,其主旨在于批判和怀疑,即所谓“疾虚妄”。
如此则必须有判断虚妄的方法。
王充的方法是诉于实证。
《论衡·薄葬篇》曰:

\textit{事莫明於有效,论莫定於有证。空言虚语,虽得道心,人犹不信。}

诉诸实证的方法乍看之下颇有说服力,然而需要注意的是,实证方法仅适用于事实,其作用范围仅限于“实际如此”。
对于涉及规律的“必然如此”和涉及价值的“应该如此”,此方法全无效力。
然而王充对这一问题全未论及。

\newline

王充明确反对天人感应之说。
盖汉儒言天人感应,有分别基于目的论和机械论的两种解释。
前者谓天主动考察人事,并根据其好坏降下祥瑞或灾殃;
后者谓天受人事成败的影响,遂生祥瑞或灾殃。
王充谓天自身无意志,不能主动施加影响于人事,故反对目的论式的天人感应;
又谓人的力量微小,不能影响天之运行,故反对机械论式的天人感应。
然而,王充所言之“天”意义不明,忽而指头上青天,忽而指整个自然界,忽而又赋予天“恬淡无欲”的人格,实在糊涂。

\newline

王充本不是一合格的思想家,其理论既欠严格,又无系统,其论“命”“性”尤其混乱。
王充谓一切生物皆有命,持命定论立场。
然而王充又谓人类生活不完全由命决定。
《论衡·命禄篇》载:

\textit{故夫临事知愚,操行清浊,性与才也;仕宦贵贱,治产贫富,命与时也。}

智慧品行属于才性之领域,而富贵贫贱属于命之领域。
此说乍看无甚问题,但王充所论之才性,实指人之天赋禀性,故仍在命内。
由此,王充所论“命”“性”相混,遂生迷乱。
此外,王充又将价值归因于才性,但其立论混乱,兹不赘述。

\newline

总之,王充有着鲜明的怀疑主义立场,不从流俗。
这种质疑精神固然可贵,但其思想浅陋杂乱,终非合格的思想家。

\end{document}