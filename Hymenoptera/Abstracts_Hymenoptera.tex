\documentclass[11pt]{article}

% \usepackage[UTF8]{ctex} % for Chinese 

\usepackage{setspace}
\usepackage[colorlinks,linkcolor=blue,anchorcolor=red,citecolor=black]{hyperref}
\usepackage{lineno}
\usepackage{booktabs}
\usepackage{graphicx}
\usepackage{float}
\usepackage{floatrow}
\usepackage{subfigure}
\usepackage{caption}
\usepackage{subcaption}
\usepackage{geometry}
\usepackage{multirow}
\usepackage{longtable}
\usepackage{lscape}
\usepackage{booktabs}
\usepackage{natbibspacing}
\usepackage[toc,page]{appendix}
\usepackage{makecell}
\usepackage{amsfonts}
 \usepackage{amsmath}

\usepackage[backend=bibtex,style=authoryear,sorting=nyt,maxnames=1]{biblatex}
\bibliography{} # Reference bib

\title{}
\author{}
\date{}

\linespread{1.5}
\geometry{left=2cm,right=2cm,top=2cm,bottom=2cm}

\setlength\bibitemsep{0pt}
\setcitestyle{braket}

\begin{document}
\begin{sloppypar}
  \maketitle

  \linenumbers

  
\par 

\textbf{Barribeau \textit{et al.}, 2015, A depauperate immune repertoire precedes evolution of sociality in bees.} \newline
We find that the immune systems of these bumblebees, two species of honeybee, and a solitary leafcutting bee, are strikingly similar. 
Transcriptional assays confirm the expression of many of these genes in an immunological context and more strongly in young queens than males, affirming Bateman’s principle of greater investment in female immunity. 
We find evidence of positive selection in genes encoding antiviral responses, components of the Toll and JAK/STAT pathways, and serine protease inhibitors in both social and solitary bees. 
Finally, we detect many genes across pathways that differ in selection between bumblebees and honeybees, or between the social and solitary clades.

\end{sloppypar}
\end{document}