\documentclass[11pt]{article}

\usepackage[UTF8]{ctex} % for Chinese 

\usepackage{setspace}
\usepackage[colorlinks,linkcolor=blue,anchorcolor=red,citecolor=black]{hyperref}
\usepackage{lineno}
\usepackage{booktabs}
\usepackage{graphicx}
\usepackage{float}
\usepackage{floatrow}
\usepackage{subfigure}
\usepackage{caption}
\usepackage{subcaption}
\usepackage{geometry}
\usepackage{multirow}
\usepackage{longtable}
\usepackage{lscape}
\usepackage{booktabs}
\usepackage{natbib}
\usepackage{natbibspacing}
\usepackage[toc,page]{appendix}
\usepackage{makecell}
\usepackage{amsfonts}
 \usepackage{amsmath}

\title{菌类}
\date{}

\linespread{1.5}
\geometry{left=2cm,right=2cm,top=2cm,bottom=2cm}

\begin{document}

  \maketitle

  \linenumbers
菌类(fungi)异养,无光合色素。

\section{黏菌门(Myxomycota)}
在生长期为无细胞壁的原生质团,多核,类似变形虫,称为变形体(plasmodium)。
繁殖时产生有纤维素细胞壁的孢子。
多腐生,少数寄生植物。
如发网菌(\textit{Stemonitis})。

\section{真菌门(Eumycota)}
除少数单细胞品种外,多为分枝丝状体,即菌丝(hyphae)。
菌丝集合为菌丝体(mycedium)。
菌丝有隔或无隔。
隔上有小孔。

\newline

生活史的某个阶段,菌丝体交织形成不同结构的营养或繁殖结构,包括根状菌索(rhizomorph)、子座(stroma)、菌核(sclerotium),由外层拟薄壁组织(pseudoparenchyma)和内层疏丝组织(prosenchyma)构成。
根状菌索外层颜色深,顶端有一生长点。
子座为容纳子实体的基座。
菌核质地坚硬,颜色深。
部分品种菌核无组织分化

\newline

真菌可营腐生或寄生,细胞有几丁质(chitin)构成的细胞壁。
可通过细胞分裂或菌丝断裂营营养生殖;
或产生无性孢子。
亦营有性生殖。

\subsection{鞭毛菌亚门(Mastigomycotina)}
多为分枝丝状体,无隔,多核。
营养繁殖时菌丝生隔,断裂产生子代。
无性孢子单鞭毛或双鞭毛。
有性孢子为卵孢子或休眠孢子。

\subsubsection{水霉(\textit{Saprolegnia})}
常寄生于淡水鱼或淡水动物尸体。
菌丝体白色,绒毛状,细长分枝,以根状菌丝插入寄主组织。

\newline

无性生殖时,菌丝顶部膨大,细胞核进入顶端,生隔形成游动孢子囊。
孢子囊顶端开口,释放游动孢子。
旧孢子囊基部生第二个孢子囊。
游动孢子顶生两条鞭毛,而厚鞭毛消失,形成球形静孢子。
静孢子再生出一条鞭毛,形成次生孢子。

\newline

有性生殖时,菌丝顶端形成精囊和卵囊。
精囊紧靠卵囊,生出丝状输精管。
精核经输精管于卵核结合,形成卵孢子。
卵孢子休眠后萌发,先减数分裂,后形成菌丝体。

\subsection{接合菌亚门(Zygomycotina)}
菌丝无隔多核。
无性生殖产生不动的孢囊孢子。
有性生殖产生二倍体接合孢子。

\subsubsection{根霉(\textit{Rhizopus})}
腐生,常生于富淀粉基质。
菌丝体棉絮状,在基质表面铺开大量匍匐枝,有假根伸入基质。

\newline

无性生殖时,基部向上生出直立孢囊梗(sporangiophore),其顶端膨大为孢子囊。
孢子囊中央有半圆形囊轴(columella),基部有囊托。
成熟后孢子囊破裂,释放孢囊孢子。

\newline

有性生殖时,两个不同宗的菌丝上生出配子囊。
配子囊顶端接触,囊壁融解,细胞核融合为二倍体接合孢子。
接合孢子减数分裂后产生单倍体孢子并释放。

\subsubsection{毛霉(\textit{Mucor})}
类似根霉,但无匍匐枝,孢囊梗从菌丝发出。

\subsection{子囊菌亚门(Ascomycotina)}
大多为多细胞。
菌丝有隔。
有性生殖形成子囊,包于子实体内。
子实体分为三种类型。
子囊盘(apothecium)盘状、杯状或碗状,一侧开口。
闭囊壳(cleistothecium)球形,完全闭合。
子囊壳(perithecium)成瓶状,顶端开口。

\subsubsection{酵母(\textit{Saccharomyces})}
单细胞,圆形或椭圆形。
出芽生殖。
有性生殖时形成子囊孢子。

\subsubsection{赤霉菌(\textit{Gibberella})}
子囊壳蓝色或紫色。
孢子梭形。
有两种无性分生孢子。
一种较大,新月形,有隔,无色;
另一种较小,卵形,粉红色。

\subsubsection{麦角菌(\textit{Claviceps})}
寄生于禾本科植物子房。
菌核萌发出子座。
子座有一长柄,头部膨大球形,其内生子囊壳。
子囊壳椭圆形,孔口突出于子座表面。
子囊壳内有长圆柱形子囊,子囊内有线状子囊孢子。

\subsubsection{青霉(\textit{Penicillium})}
无性分生孢子梗顶端数次分枝,成扫帚状。
最末小枝为小梗,上生一串绿色分生孢子。
鲜有有性生殖。

\subsubsection{白粉菌(\textit{Erysiphe})}
寄生于植物。
闭囊壳,内有子囊。
子囊内有孢子。

\subsubsection{虫草(\textit{Cordyceps})}
子座从昆虫宿主虫体发出,肉质,多为棒状,直立。

\subsubsection{羊肚菌(\textit{Morchella})}
腐生。
子实体有菌盖和菌柄。
菌盖近球形或圆锥形,边缘全部和菌柄相连,表面有网状棱纹。
菌柄平整或有凹槽。

\subsubsection{盘菌(\textit{Peziza})}
子囊盘成盘状,菌病不发达。
子囊圆柱状。
子囊孢子椭圆形,无色,在子囊内排列成一行。

\subsection{担子菌亚门(Basidiomycotina)}
陆生高等真菌,多细胞,菌丝有隔。
初生菌丝体(primary mycelium)细胞单核,在生活史中时间短。
次生菌丝体(secondary mycelium)是单核菌丝或性孢子质配后形成的,细胞双核。
三生菌丝体(tertiary mycelium)是次生菌丝体特化形成的,细胞双核,产生担子和担孢子。

\newline

次生菌丝体和三生菌丝体细胞分裂时产生锁状联合(clamp connection)。
细胞中央侧生出喙状突起,向下弯曲。
一细胞核移入突起基部,另一核在其附近。
两核分裂为四个子核,其中两个在细胞上部,一个在基部,一个在突起中。
而后生隔,母细胞分为三个子细胞。
上方子细胞双核,基部和喙状突起形成的子细胞单核。
喙突向下弯曲的部位连通基部子细胞,形成双核细胞。
两个双核子细胞之间残留喙突。

\newline

通过芽殖或产生节孢子、分生孢子进行无性生殖。
有性生殖时,双核菌丝顶端膨大为担子(basidium),其内的两个单倍体核融合后减数分裂,产生四个核。
担子顶端突出为四个小梗,每个小梗末端形成一个单核担孢子(basidiospore)。

\newline

子实体又称担子果,为产生担孢子的高度组织化结构,一般源于双核菌丝体。

\subsubsection{冬孢菌纲(Teliomycetes)}
不形成子实体。
如玉米黑粉菌(\textit{Ustilago maydis})、禾柄锈菌(\textit{Puccin graminis})。

\subsubsection{层菌纲(Hymenomycetes)}
子实体发达。
担子整齐排列为子实层,分布在菌髓两侧。

\newline

银耳目(Tremellales)子实体多胶质,子实层生于一侧。
担子纵隔,横切面上呈田字分布。

\newline

木耳目(Auriculariales)子实体胶质。
子实层分布于表面,或大部分包埋于胶质内。
担子横隔。
如木耳(\textit{Auricularia auricula})。

\newline

多孔菌目(Polyporales)子实体或为多年生,木质或肉质。
担子单细胞,棒状。
如灵芝(\textit{Ganoderma lucidum})、猴头(\textit{Hericium erinaceus})。

\newline

伞菌目(Agaricales)子实体多肉质,有伞状的菌盖(pileus),下方为菌柄(stipe)。
菌柄多中生,菌盖腹面有放射排列的菌褶(gills)。
子实层位于菌褶两面。
子实体幼时常有内菌幕(partial veil)遮盖菌褶。
菌盖展开,内菌幕破裂,在菌柄残留环状菌环(annulus)。
部分品种子实体幼时有外菌幕(universal veil),包围子实体。
菌柄伸长,外菌幕破裂,在菌柄基部残留菌托(volva),在菌盖顶部可能残留鳞片(scale)。
担子无隔,棒状。
多腐生。
如蘑菇(\textit{Agaricus spp.})、香菇(\textit{Lentinus spp.})。

\subsubsection{腹菌纲(Gasteromycetes)}
子实体发达,外有多层包被(peridium),内为产孢体(gleba)。
产孢体多腔,担子沿腔边缘生出。

\newline

鬼笔目(Phallales)子实体卵形、圆形或梨形。
成熟时包被裂开,包托伸长,包被残留形成菌柄。
产孢组织成熟时有黏性,恶臭。
如白鬼笔(\textit{Phallus impudicus})、长裙竹荪(\textit{Dictyophora indusiata})。

\newline

马勃目(Lycoperdales)子实体梨形,基部有白色根状菌索。
包被多层,可能不裂开。
成熟孢子为粉末状。
如梨形马勃(\textit{Lycoperdon pyriforme})、头状秃马勃(\textit{Calvatia craniiformes})。

\subsection{半知菌亚门(Deuteromycotina)}
单倍体,菌丝有隔。
菌丝体发达。
以分生孢子营无性生殖,无有性生殖。
如稻瘟病菌(\textit{Piriculaxia oryzae})。

\end{document}