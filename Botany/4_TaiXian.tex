\documentclass[11pt]{article}

\usepackage[UTF8]{ctex} % for Chinese 

\usepackage{setspace}
\usepackage[colorlinks,linkcolor=blue,anchorcolor=red,citecolor=black]{hyperref}
\usepackage{lineno}
\usepackage{booktabs}
\usepackage{graphicx}
\usepackage{float}
\usepackage{floatrow}
\usepackage{subfigure}
\usepackage{caption}
\usepackage{subcaption}
\usepackage{geometry}
\usepackage{multirow}
\usepackage{longtable}
\usepackage{lscape}
\usepackage{booktabs}
\usepackage{natbib}
\usepackage{natbibspacing}
\usepackage[toc,page]{appendix}
\usepackage{makecell}
\usepackage{amsfonts}
 \usepackage{amsmath}

\title{苔藓(Bryophyta)}
\date{}

\linespread{1.5}
\geometry{left=2cm,right=2cm,top=2cm,bottom=2cm}

\begin{document}

  \maketitle

  \linenumbers
苔藓为小型有胚植物,无维管组织和根茎叶分化。
生殖器官为多细胞结构。
有世代交替,配子体占优。

\par

配子体为扁平叶状体(thallus)或拟茎叶体。
有单细胞或单列细胞构成的假根,司固着。
部分品种有司运输的长细胞。
拟叶为单层细胞,仅中肋处有多层细胞。
拟叶表面角质层不发达,可直接吸收水分、养料。

\par

营养繁殖时,产生胞芽(gamma)或叶状体断裂。
无性生殖产生孢子,萌发为原丝体(protonema),再发育为配子体。
有性生殖时,配子体形成颈卵器(archegonium)和精子器(antheridium)。
颈卵器形似长颈烧瓶。
其颈部(neck)外壁为单层细胞,中央为一列颈沟细胞(neck canal cell)。
其腹部(venter)外壁为二至三层细胞,中央为一大型卵细胞。
卵细胞和颈沟细胞间为一腹沟细胞(venter canal cell)。
精子器棒状或球状,外壁为单层细胞,内为双鞭毛精子。
卵细胞成熟时,颈沟细胞和腹沟细胞解体,精子借水游入颈卵器。
合子分裂形成胚。
胚在颈卵器内发育为孢子体。
孢子体上端为孢子囊(sporangium),成熟时称孢蒴(capsule)。
孢蒴下为蒴柄(seta)。
再向下为基足(foot),伸入配子体,吸收养料。
无性生殖时,胞蒴内形成孢子母细胞,减数分裂为四分孢子。

\section{苔纲(Hepaticae)}
配子体多两侧对称,有背腹之分。
单细胞假根,常含油体。
叶状体品种为单层或多层细胞,腹面可能有鳞片。
拟茎叶体品种茎细长,全为薄壁组织;
叶单层细胞,无中肋。
孢蒴内有弹丝(elater)。
原丝体不发达,仅发育为一个配子体。
如地钱(\textit{Marchantia})、光萼苔(\textit{Porella})。

\section{角苔纲(Anthocerotae)}
生于潮湿土壤,不耐旱。
配子体叶状、带状或心形,腹面无鳞片。
叶状体内部无组织分化。
细胞薄壁,有一大叶绿体,无油体。
叶绿体内有一淀粉核。
孢子体从配子体上长出,圆柱形,绿色,有基足和蒴轴,无蒴柄。
基足上有分生组织。
新细胞被推向顶端时,分化为蒴壁、蒴轴、造孢组织。
蒴壁细胞含叶绿体。
造孢组织覆盖蒴轴,产生孢子、弹丝。
成熟时孢子囊先端两线状孔裂开。
如角苔(\textit{Anthoceros})。

\section{藓纲(Musci)}
多年生,耐寒。
多为拟茎叶体。
叶有多层细部的中肋,其余叶片为单细胞。
上表面有角质层,下表面无。
茎短小,大多无组织分化。
茎基部有单列细胞假根。
孢子萌发为原丝体,生芽形成配子体。
孢子囊基部有气孔。
蒴柄内有厚壁长细胞,常为导水胞和拟筛管。
孢子囊顶部有帽状蒴盖(operculum)。
蒴盖脱落,露出蒴齿。
蒴齿湿润时向下弯曲,覆盖孢腔。
干燥时蒴齿向外弯曲,放出孢子。
如泥炭藓(\textit{Sphagnum})、黑藓(\textit{Andreaea})、葫芦藓(\textit{Funaria})。

\end{document}