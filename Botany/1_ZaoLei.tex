\documentclass[11pt]{article}

\usepackage[UTF8]{ctex} % for Chinese 

\usepackage{setspace}
\usepackage[colorlinks,linkcolor=blue,anchorcolor=red,citecolor=black]{hyperref}
\usepackage{lineno}
\usepackage{booktabs}
\usepackage{graphicx}
\usepackage{float}
\usepackage{floatrow}
\usepackage{subfigure}
\usepackage{caption}
\usepackage{subcaption}
\usepackage{geometry}
\usepackage{multirow}
\usepackage{longtable}
\usepackage{lscape}
\usepackage{booktabs}
\usepackage{natbib}
\usepackage{natbibspacing}
\usepackage[toc,page]{appendix}
\usepackage{makecell}
\usepackage{amsfonts}
 \usepackage{amsmath}

\title{藻类(algae)}
\date{}

\linespread{1.5}
\geometry{left=2cm,right=2cm,top=2cm,bottom=2cm}

\begin{document}

  \maketitle

  \linenumbers
藻类有光合色素,结构简单,无根、茎、叶之分化,生殖器官单细胞,合子不发育成胚。

\newline

藻类营营养繁殖、无性繁殖和有性繁殖。
营养繁殖时,营养体的一部分脱离母体后发育为新个体。
无性繁殖时,植物体形成孢子,孢子直接发育为新个体。
有性繁殖时,植物体形成配子,配子两两结合为合子,合子发育为新个体,或合子先形成孢子,孢子发育为新个体。
若相结合的两配子形状、结构、大小、运动能力相同,称为同配生殖(isogamy)。
若两配子形状结构相同,大小和运动能力不同,则称异配生殖(anisogamy)。
若两配子形状大小结构均不同,则为卵式生殖(oogamy)。

\newline

有性生殖过程中,减数分裂发生的时间和植物体倍性的不同,形成不同的生活史类型。
合子减数分裂型中,合子萌发时进行减数分裂,发育为单倍体植物体。
配子减数分裂型中,植物体为二倍体,在形成配子时进行减数分裂。
孢子减数分裂型中,合子发育为二倍体植物体,二倍体减数分裂形成孢子,孢子发育为单倍体植物体;
单倍体植物营有性生殖,形成合子。
二倍体(孢子体)和单倍体(配子体)在生活史中交替出现,即世代交替。
若配子体和孢子体形态相似,则为同型世代交替,反之则称异型世代交替。

\section{蓝藻门(Cyanophyta)}
原核生物,无细胞器分化。
胞内有扁平封闭的小囊,为光合片层(photosynthetic lamellae),上有光合色素。
光合产物储藏于胞质。
细胞内有气泡和藻胆体。
藻胆体是藻胆素(phycobilins)和组蛋白结合形成的颗粒体。
藻胆素多呈蓝色。
胞壁主要为黏多糖(glycosaminopeptide)或肽聚糖(peptidoglycan)。
壁外有胶质鞘(gelatinous sheath),成分为果胶酸(pectic acid)和黏多糖。

\newline

以营养繁殖为主。
单细胞可直接分裂。
群体类型则由细胞反复分裂,形成大群体后破裂为多个小群体。
亦有丝状体(filament)类型,断裂为片段,形成子代。
亦可营无性生殖,形成孢子。
营养细胞体积增大,细胞壁加厚,形成厚壁孢子(akinete),可长期休眠。
母细胞多次分裂后细胞壁破裂释放的孢子为内生孢子(endospore)。
母细胞横分裂,较大的子细胞保持分裂能力,较小的子细胞形成外生孢子(exospore)。

\subsection{色球藻目(Chroococcales)}
单细胞或群体,主要通过直接分裂繁殖。

\subsubsection{色球藻(\textit{Chroococcus})}
单细胞成球形,胶质鞘固态透明。
群体细胞半球形或四分体形,每个细胞都有胶质鞘,群体外被胶质鞘。
浮游生活。

\subsubsection{微囊藻(\textit{Microcystis})}
浮游不定型群体,上有穿孔。
细胞球形,埋于胶质。
可形成水华。

\subsection{颤藻目(Osillatoriales)}
单列细胞构成丝状体,常通过断裂形成藻殖段营无性生殖。

\subsubsection{颤藻(\textit{Oscillatoria})}
单列细胞构成不分枝丝状体,丛生成团块状。
细胞短,圆柱状,胶质鞘不发达。
常在双凹形的死细胞或膨大胶化的隔离盘处断裂形成藻殖段。

\subsection{念珠藻目(Nostocales)}
不分枝丝状体,有大而透明的异形胞。
有厚壁孢子,亦通过丝状体断裂生殖。

\subsubsection{念珠藻(\textit{Nostoc})}
丝状体无规则地集合于胶质鞘,形成团块。
单列细胞构成不分枝的丝状体。
细胞球形,念珠状。
淡水生。
如发菜(\textit{Nostoc flagelliforme})。

\subsubsection{鱼腥藻(\textit{Anabena})}
细胞圆形,连接成丝状体,浮于水中,无公共胶质鞘。

\section{红藻门(Rhodophyta)}
常多细胞。
细胞壁内层纤维素,外层果胶,亦有琼胶等红藻特有化合物。
原生质极黏。
有大液泡。
多藻红素,故藻体多呈红色。

\newline

有营养繁殖。
无性生殖产生的孢子无鞭毛。
多雌雄异株。
精子无鞭毛。
雌性生殖结构形似烧瓶,称为果胞(carpogonium),内含一卵。
多有世代交替。
大多海生,固着生活。

\subsection{红毛菜纲(Bangiophyceae)}
\subsubsection{紫球藻(\textit{Porphyridium})}
单细胞,卵圆或椭圆。
外有胶质薄膜。
载色体中轴位,有蛋白核。
常生活于潮湿地面。

\subsubsection{紫菜(\textit{Porphyra})}
藻体叶状,薄,紫红色,边缘有褶皱,外有胶质层。
细胞单核,一载色体。
附着于海滩岩石。
如甘紫菜(\textit{Porphyra tonera})。

\subsection{红藻纲(Florideophyceae)}
分枝丝状体。
载色体多个,周生,盘状。

\subsubsection{多管藻(\textit{Polysiphonia})}
植物体为多列细胞构成的分枝丝状体。
丝状体中央为一列较粗的细胞构成的中轴管(central siphon),外围由较细的细胞构成围轴管(peripheral siphon)。
精子囊葡萄状。

\section{金藻门(Chrysophyta)}
单细胞、群体或分枝丝状体。
单细胞品种和部分群体品种的营养细胞有鞭毛,能运动。
能运动的细胞常无细胞壁,可能外被纤维素或果胶构成的囊壳,囊壳表面镶硅质鳞片。
细胞壁由纤维素和果胶构成,有伸缩泡。
原生质体透明,玻璃状,两侧有大型片状载色体,可能有蛋白核。

\newline

通过细胞纵裂或群体断裂营营养生殖。
有囊壳的细胞原生质体纵裂后,一个子细胞留在原囊壳内,另一个游出后分泌新囊壳。
无性生殖时,细胞停止运动,变圆,分泌纤维素膜。
而后二氧化硅堆积成壳,顶端开孔,形成内生孢子。
不能运动的品种形成有鞭毛的游动孢子(zoospore)。
鲜有有性生殖。

\subsection{金藻纲(Chrysophyceae)}
\subsubsection{色金藻目(Chromulinales)}
单细胞或群体,无细胞壁,有囊壳或硅质鳞片,有鞭毛,载色体周生。

\newline

锥囊藻(\textit{Dinobryon})为单细胞或树状群体,胞外有纤维素质的钟形囊壳,顶端两条不等长鞭毛。

\subsubsection{褐枝藻目(Phaeothamniales)}
丝状体或假薄壁组织体,附着生活,有细胞壁。
载色体一个,片状,周生。

\newline

褐枝藻(\textit{Phaeothamnion})为分枝丝状体,基部有半球形固着器,产游动孢子。
淡水生。

\subsection{黄群藻纲(Synurophyceae)}
\subsubsection{黄群藻目(Synurales)}
自由运动的单细胞或群体,有硅质鳞片和鞭毛。
载色体一或两个,周生。

\newline

黄群藻(\textit{Synura})为球形或椭圆形群体。
细胞前端两条等长鞭毛,细胞外被果胶膜,上有硅质鳞片,鳞片表面有纹或硬刺。
群体以胶质互相黏附,群体外无胶质膜。

\section{黄藻门(Xanthophyta)}
大多有果胶质细胞壁。
单细胞和群体类型细胞壁由两个U形半片套合成工字形。
丝状体类型细胞壁由两个H形半片套合而成。
原生质透明,大多单核。
载色体边位。
两条不等长的近顶生鞭毛,长者向前,短者弯曲向后。

\newline

无性生殖为主,产游动孢子、似亲孢子或不动孢子。

\subsection{黄藻纲(Xanthophyceae)}
\subsubsection{黄丝藻(\textit{Tribonema})}
不分枝丝状体,细胞壁为H形半片。
单核,载色体周生,无蛋白核。
营养繁殖时丝状体断裂或细胞壁半片脱开。
无性生殖产不动孢子、游动孢子或胞囊。
有性生殖同配,但较少。
淡水生。

\subsubsection{气球藻(\textit{Botrydium})}
多核单细胞,下部为分枝假根。
载色体盘状,有中央大液泡。
生于潮湿土壤。

\subsubsection{无隔藻(\textit{Vaucheria})}
细胞分枝多核。基部有假根。
有中央大液泡,胞内储存油,无淀粉。
无性生殖时,陆生种类产静孢子。
水生种类分枝顶端膨大,膨大基部生隔,形成多核游动孢子囊。
其细胞核均匀分布于四周,对应每个细胞核的地方生两条鞭毛。
有性生殖为卵式,同宗或异宗配合。
卵囊和精子囊生于侧短枝,基部有横隔。
多淡水生,少数生于潮湿泥土。

\section{硅藻门(Bacillariophyta)}
单细胞或群体。
细胞壁由两个套合的半片构成,称为上壳(epitheca)和下壳(hypotheca)。
其成分为果胶和硅质,无纤维素。
半片正面为壳面(valve),侧面为环面(girdle)。
上下壳套合重叠的部分为环带(girdle band)。
壳面上有花纹,部分品种有壳缝(raphe)。

\newline

细胞单核,球形或卵形,中央有液泡,载色体盘状或片状。
细胞分裂时,原生质体膨大,撑开上下壳。
母细胞的上壳和下壳形成子细胞的上壳。
如此一个子细胞与母细胞等大,另一个则较小。
细胞随分裂缩小到一定程度时,行有性生殖,产生复大孢子(auxospore)。

\subsection{中心纲(Centricae)}
\subsubsection{小环藻(\textit{Cyllotella})}
单细胞,部分品种壳面连接形成带状群体。
细胞圆盘形或鼓形。
壳面圆形或椭圆形,有辐射对称排列的线纹和孔纹,中央平滑或有颗粒。

\subsection{羽纹纲(Pinnatae)}
\subsubsection{羽纹硅藻(\textit{Pinnularia})}
单细胞或丝状群体。
壳面线状、椭圆形或披针形,两侧平行,有两侧对称横向平行的纹。

\section{褐藻门(Phaeophyta)}
多细胞。
植物体为无分化的分枝丝状体;
或分化为匍匐枝和直立枝的异体丝状体;
或由分枝丝状体紧密结合形成的拟薄壁组织;
或有表皮层、皮层、髓之分化的植物体。
表皮层细胞小,有载色体;
皮层细胞较大,仅外围接近表皮层的细胞有载色体;
髓为无色长细胞,司输导、储存。

\newline

细胞壁内层纤维素,外层藻胶。
色素体粒状或小盘状,呈褐色。
常有大型蛋白核突出于载色体一侧,外包淀粉鞘。

\newline

通过植物体断裂进行营养生殖。
无性生殖产生游动孢子和静孢子。
单室孢子囊(unilocular sporangium)是孢子体的一个细胞体积增大,细胞核经减数分裂和有丝分裂形成的多核细胞;
多室孢子囊(plurilocular sporangium)是孢子体的一个细胞经多次有丝分裂形成的细长多细胞组织。
有性生殖时,配子体形成多室配子囊,结构和起源类似多室孢子囊。
配子结合方式有同配、异配、卵式。
多有世代交替。
多海生。

\subsection{等世代纲(Isogeneratae)}
同型世代交替。

\subsubsection{水云(\textit{Ectocarpus})}
藻体为单列细胞构成的分枝丝状体,分化为匍匐枝和直立枝。
细胞单核。
载色体盘状或带状,有蛋白核。

\newline

无性生殖时,孢子囊发生于侧生小枝顶端,单室或多室。
有性生殖时,配子囊位于侧生小枝顶端,异宗配合。

\subsection{不等世代纲(Heterogeneratae)}
异型世代交替。
孢子体发达,有拟薄壁组织。
配子体为分枝丝状体。

\subsubsection{海带(\textit{Laminaria})}
孢子体分为固着器、柄和带叶。
固着器分枝根状。
柄圆柱形或略侧扁,不分枝,分为表皮、皮层和髓。
带片位于柄顶端,不分裂,无中脉,边缘波浪状,内部构造类似柄。

\newline

孢子体带片两面有深褐色丛生的单室孢子囊,产生梨形游动孢子。
游动孢子发育为雌雄配子体,为几十个细胞构成的分枝丝状体。
精子侧生双鞭毛。
卵子位于配子体枝端卵囊。
合子不离开雌配子体,直接萌发。

\newline

如海带(\textit{Laminaria japonica})。

\subsection{圆孢子纲(Cyclosporae)}
无世代交替。
藻体为二倍体孢子体。

\subsubsection{鹿角菜(\textit{Pelvetia})}
多固着于海浪冲击的岩石上,褐色,软骨质,二叉分枝,有表皮、皮层和髓。
基部有圆盘状的固着器。
有性生殖。
生殖时分枝顶端膨大为生殖托(receptacle),成有柄长角果状,表面有突起,其内为卵囊和精囊。

\section{甲藻门(Pyrrophyta)}
多单细胞。
细胞球形、椭圆形或三角形,略扁,多有两条不等长鞭毛。
大多有含纤维素的细胞壁。
纵列甲藻细胞壁有左右对称的两个半片,无沟。
横裂甲藻细胞壁由多个板片嵌合而成,分为上壳(epitheca)和下壳(hypotheca)。
上下壳中间有一横沟(girdle)。
下壳腹面正中有一纵沟,与横沟垂直。
鞭毛两条,顶生或侧生。
顶生一条伸直向前,另一条弯曲向后。
侧生鞭毛从横沟纵沟交叉处伸出,一条在横沟中,另一条沿纵沟向后。
载色体多周生。
有刺丝胞(trichocyst),受刺激时喷出内容物。
营养繁殖为主。
多海生。

\subsection{甲藻纲(Dinophyceae)}

\subsubsection{多甲藻(\textit{Peridinium})}
细胞背腹扁,背面凸,腹面平或凹。
纵沟横沟明显。
细胞壁有多块板片,板片表面凸起少。
载色体多,粒状,周生。

\subsubsection{角甲藻(\textit{Ceratium})}
细胞不对称。
顶端有板片突出形成的长角,底部有二至三个短角。

\section{裸藻门(Euglenophyta)}
多单细胞,多无细胞壁,多有鞭毛,多无色素体。
细胞前端有胞口(cytostome)和狭长的胞咽(cytopharynx),胞咽下有储蓄泡(reservoir),储蓄泡周围有伸缩泡(contractile vacuole)。
有一条游动鞭毛伸出;
另一根鞭毛退化,其残存与游动鞭毛基部相连。
纵分裂生殖。
仅裸藻目(Euglenales)一目。

\subsection{裸藻(\textit{Euglena})}
单细胞,长纺锤形或圆柱形,前端宽钝,后端尖锐,表面有螺旋形纹。
一根鞭毛自储蓄泡底部经胞咽、胞口伸出;
另一根鞭毛退化,位于储蓄泡内。
核大,圆形。

\subsection{柄裸藻(\textit{Colacium})}
有细胞壁,无鞭毛。
细胞前端有胶质柄,附着浮游动物。
眼点和储蓄泡位于后端。
细胞分裂时子细胞不脱离目体,形成群体。

\subsection{囊裸藻(\textit{Trachelomonas})}
藻体有褐黄色含铁囊壳,表面有颗粒突起,鞭毛自孔口伸出。

\subsection{扁裸藻(\textit{Phacus})}
细胞侧扁,无弹性。
有环状大型裸藻淀粉。

\section{绿藻门(Chlorophyta)}
多不能运动,仅繁殖时形成的孢子、配子有鞭毛。
细胞壁成分为纤维素和果胶。
载色体内有蛋白核。
细胞内有淀粉核。

\newline

营养繁殖时,通过细胞分裂、营养体断裂、产生胶群体等方式形成新个体。
无性生殖时,原生质体收缩为无壁游动孢子;
或形成静孢子。
亦营有性生殖。
若两配子无鞭毛,能变形,则称接合生殖(conjugation)。

\subsection{葱绿藻纲(Prasonophyceae)}
单细胞,有鞭毛。
细胞表面有多糖鳞片。

\subsubsection{四片藻(\textit{Tetraselmis})}
细胞纵扁。
前端突出,中央有四根等长鞭毛。

\subsection{绿藻纲(Chlorophyceae)}
\subsubsection{团藻目(Volvocales)}
浮游,多淡水生。

\newline

衣藻(\textit{Chlamydomonas})单细胞,卵圆、椭圆或圆形,单核。
前端两条鞭毛,下有两个伸缩泡。
眼点橘红色,位于前端。
细胞壁外层果胶,内层纤维素。
色素体形如厚底杯,基部有蛋白核。
营无性生殖或同配有性生殖。

\newline

团藻(\textit{Volvox})细胞卵圆、椭圆或圆形,外包胶质,排列成单层球形。
球内为胶质。
细胞间有原生质丝。
后端部分细胞无鞭毛且大,为生殖细胞(gonidium)。
无性生殖时,生殖细胞分裂、分化,出芽形成单层细胞球。
而后细胞球翻转,原朝向球内部的一面翻转到球的表面,并脱离母体。
亦营卵式生殖。
精子板(sperm packet)的形成类似无性生殖。
卵子由生殖细胞变大而成。
精子板游动至卵子附近后散开,受精形成厚壁合子,脱离母体。
合子萌发后经减数分裂和有丝分裂,发育为细胞群体,后形成藻体。
发育过程类似无性生殖。

\newline

盘藻(\textit{Gonium})为定型群体。
细胞位于同一平面,胞间有间隙。
无性生殖时,群体内所有细胞同时产生游动孢子。
有性生殖为异配。

\newline

实球藻(\textit{Pandorina})为定型群体。
细胞紧密排列,构成实心球。
无性生殖时,群体内所有细胞同时产生游动孢子。
有性生殖为异配。

\newline

空球藻(\textit{Eudorina})群体为空心球。
有性生殖为异配。
部分品种群体内某些营养细胞不产生配子和孢子。

\subsubsection{绿球藻目(Chlorococcales)}
单细胞或群体,营养细胞无鞭毛,产生似亲孢子。
淡水产,浮游。

\newline

小球藻(\textit{Chlorella})多为单细胞浮游生物,圆形或椭圆形,壁薄。
色素体杯形或曲带形,无蛋白核。
无性生殖。
母细胞壁破裂时释放孢子。

\newline

栅藻(\textit{Scenedesmus})为定型群体。
细胞椭圆或纺锤形,单核。
群体中细胞以长轴平行排列成一行,或互相交错排列成两行。
无性生殖。
母细胞壁纵裂放出孢子。

\newline

盘星藻(\textit{Pediastrum})群体细胞辐射状排列于同一平面,可能有穿孔。
外圈细胞有向外的突起。

\newline

水网藻(\textit{Hydrodictyon})细胞长圆柱形,连接成网状。
每五到六个细胞构成网眼。

\newline

空星藻(\textit{Coelastrum})群体为球形或多角形,空心。
细胞以壁上突起互相连接。

\subsubsection{丝藻目(Ulotrichales)}
单列细胞构成不分枝丝状体。

\newline

丝藻(\textit{Ulothrix})基部细胞司固着,色素体颜色浅。
向上为一列短筒形营养细胞,中央有单核。
其色素体大型环带状,多蛋白核。
无性生殖时,非固着器细胞均产生游动孢子,其顶端有四条鞭毛。
有性生殖时,除固着器外的其他细胞产生配子,配子有两条鞭毛,同配生殖。
合子减数分裂后发育为植物体。

\subsubsection{石莼目(Ulvales)}
一或二层细胞构成片状体,有世代交替。

\newline

石莼(\textit{Ulva})
植物体大型多细胞片状体,由两层细胞组成。
基部为无色假根丝,形成固着器。
固着器多年生,每年春天生出新的植物体。
植物体细胞间隙多胶质。
细胞单核,核位于内侧。
色素体片状,有单一蛋白核,位于外侧。
同型世代交替。
孢子体除基部外,其于细胞均可形成孢子囊,经减数分裂形成单倍体游动孢子,发育为配子体。
配子体异宗同配。

\newline

浒苔(\textit{Enteromorpha})为一层细胞的管状体,单核,单载色体。
生活史类似石莼。

\newline

礁膜(\textit{Monostroma})为一层细胞的膜状体,细胞间有厚胶质。
异型世代交替。

\subsubsection{鞘藻目(Oedogoniales)}
丝状体不分枝或少分枝。
细胞分裂时母细胞顶端侧壁有环状裂缝,称为冠环。
游动孢子有一轮鞭毛。
有性生殖卵式。

\newline

鞘藻(\textit{Oedogonium})为单列细胞构成不分枝丝状体。
细胞筒形,核大而明显。
载色体网状,多蛋白核。
基部为盘状固着器。
每个细胞都有分裂能力。
无性生殖产生深绿色游动孢子,顶端有一圈鞭毛。
淡水生。

\subsection{双星藻纲(Zygnematophyceae)}
营养细胞和生殖细胞均无鞭毛。
有性生殖为接合生殖。

\subsubsection{双星藻目(Zygnematales)}
单细胞或不分枝丝状体。

\newline

水绵(\textit{Spirogyra})为一列圆柱形细胞构成不分枝丝状体。
细胞壁内层纤维素,外层果胶,藻体滑腻。
色素体带状,螺旋环绕于胞质中,蛋白核多。
胞内有大液泡。
没有无性生殖。
营接合生殖,同宗或异宗。

\newline

双星藻(\textit{Zygnema})为不分枝丝状体。
载色体两个,星状,轴生,有蛋白核。

\newline

转板藻(\textit{Mougeotia})为不分枝丝状体。
载色体一个,片状,轴生,有多个蛋白核。

\subsubsection{鼓藻目(Desmidiales)}
多单细胞。
细胞中部常缢入,形成上下两个半细胞。

\newline

新月藻(\textit{Closterium})细胞新月形。
载色体两个,位于细胞核两边,有一列蛋白核。
细胞两端各一个液泡。

\newline

鼓藻(\textit{Cosmarium})细胞哑铃状,细胞壁可能有乳突。

\subsection{轮藻纲(Charophyceae)}
植物体初步有根茎叶之分,直立,有轮生的枝,有节和节间的分化。
仅轮藻目(Charales)一目。

\subsubsection{轮藻(\textit{Chara})}
植物体分地上和地下部分。
地下部分为单列细胞构成有分枝的假根。
地上部分有主枝、侧枝、小枝之分,其上分节(node)和节间(internode)。
侧枝互生于主枝。
小枝轮生于主枝、侧枝的节上。

\newline

营卵式生殖。
生殖器位于小枝节上。
精囊球圆形,位于刺状细胞下方,成熟时呈橘红色。
外壁有八个三角形盾细胞(shield cell),其中有橘红色载色体。
盾细胞内侧中央连接一个圆柱形盾柄细胞(manubrium)。
盾柄细胞末端有圆形的头细胞(head cell)。
头细胞上生多个小圆形的次头细胞,其上生多条单列细胞构成的精囊丝。
精囊丝每个细胞产生一个细长精子。
精子顶生两条鞭毛。

\newline

卵囊球位于刺状细胞下方,内含一个卵细胞。
卵细胞外围有五个螺旋状管细胞(tube cell)。
管细胞顶端为冠细胞。
五个冠细胞在卵囊球顶端组成冠(corona)。
卵囊成熟时冠裂开,精子进入,形成合子。
合子休眠后萌发,经减数分裂,发育为子代。

\newline

亦通过藻体断裂进行营养生殖。

\end{document}