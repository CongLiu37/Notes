\documentclass[11pt]{article}

\usepackage[UTF8]{ctex} % for Chinese 

\usepackage{setspace}
\usepackage[colorlinks,linkcolor=blue,anchorcolor=red,citecolor=black]{hyperref}
\usepackage{lineno}
\usepackage{booktabs}
\usepackage{graphicx}
\usepackage{float}
\usepackage{floatrow}
\usepackage{subfigure}
\usepackage{caption}
\usepackage{subcaption}
\usepackage{geometry}
\usepackage{multirow}
\usepackage{longtable}
\usepackage{lscape}
\usepackage{booktabs}
\usepackage{natbib}
\usepackage{natbibspacing}
\usepackage[toc,page]{appendix}
\usepackage{makecell}
\usepackage{amsfonts}
 \usepackage{amsmath}

\title{藻类}
\date{}

\linespread{1.5}
\geometry{left=2cm,right=2cm,top=2cm,bottom=2cm}

\begin{document}

  \maketitle

  \linenumbers
藻类(Algae)有光合色素,结构简单,无根、茎、叶之分化,多为水生。

\section{蓝藻门(Cyanophyta)}
原核生物,无细胞器分化。
胞内有扁平膜状的光合片层(photosynthetic lamellae),上有光合色素。
光合产物储藏于胞质。
细胞内有气泡。
胞壁主要为黏肽(peptidoglycan)。
壁外有胶质鞘(gelatinous sheath),成分为果胶酸(pectic acid)和黏多糖(mucopolysaccharide)。

\newline

营无性生殖。
单细胞可直接分裂。
群体类型则由细胞反复分裂,形成大群体后破裂为多个小群体。
亦有丝状体(filament)类型,断裂为片段,形成子代。

\newline

亦可形成孢子。
母细胞多次分裂后细胞壁破裂释放的孢子为内生孢子(endospore)。
母细胞横分裂,较大的子细胞保持分裂能力,较小的子细胞形成外生孢子(exospore)。

\subsection{色球藻(\textit{Chroococcus})}
单细胞成球形,胶质鞘固态透明。
群体细胞半球形或四分体形,每个细胞都有胶质鞘,群体外被胶质鞘。
浮游生活。

\subsection{微囊藻(\textit{Microcystis})}
浮游不定型群体,上有穿孔。
细胞球形,埋于胶质。
可形成水华。

\subsection{皮果藻(\textit{Dermocarpa})}
单细胞,倒卵形,形成内生孢子。

\subsection{管胞藻(\textit{Chamaesiphon})}
长杆形单细胞,有极性。
基部附着水生植物。
形成外生孢子。

\subsection{颤藻(\textit{Oscillatoria})}
单列细胞构成不分枝丝状体,丛生成团块状。
细胞短,圆柱状,胶质鞘不发达。

\subsection{念珠藻(\textit{Nostoc})}
丝状体无规则地集合于胶质鞘,形成团块。
单列细胞构成不分枝的丝状体。
细胞球形,念珠状。
淡水生。
如发菜(\textit{Nostoc flagelliforme})。

\subsection{鱼腥藻(\textit{Anabena})}
细胞圆形,连接成丝状体,浮于水中,无公共胶质鞘。

\subsection{真枝藻(\textit{Stigonema})}
单或多列细胞构成不规则分枝丝状体。
生于潮湿岩石上。

\section{裸藻门(Euglenophyta)}
多无细胞壁,多有鞭毛,多无色素体。
细胞前端有胞口(cytostome)和狭长的胞咽(cytopharynx),胞咽下有储蓄泡(reservoir),储蓄泡周围有伸缩泡(contractile vacuole)。
纵分裂生殖。

\subsection{裸藻(\textit{Euglena})}
单细胞,长纺锤形或圆柱形,前端宽钝,后短尖锐。
一根鞭毛自储蓄泡底部经胞咽、胞口伸出;
另一根鞭毛退化,位于储蓄泡内。
核大,圆形。
色素体分布于原生质体表面。

\subsubsection{柄裸藻(\textit{Colacium})}
有细胞壁,无鞭毛。
细胞前端有胶质柄,附着浮游动物。
眼点和储蓄泡位于后短。
细胞分裂时子细胞不脱离目体,形成群体。

\section{硅藻门(Bacillariophyta)}
单细胞可直接分裂。
细胞壁由两个套合的半片构成,称为上壳(epitheca)和下壳(hypotheca)。
其成分为果胶和硅质,无纤维素。
半片正面为壳面(valve),侧面为环面(girdle)。
上下壳套合重叠的部分为环带(girdle band)。
壳面上有花纹,部分品种有壳缝(raphe)。

\newline

细胞分裂时,原生质体膨大,撑开上下壳。
母细胞的上壳和下壳形成子细胞的上壳。
如此一个子细胞与母细胞等大,另一个则较小。
细胞随分裂缩小到一定程度时,行有性生殖,产生复大孢子(auxospore)。

\subsection{小环藻(\textit{Cyllotella})}
单细胞,部分品种壳面连接形成带状群体。
细胞圆盘形或鼓形。
壳面圆形或椭圆形,有辐射对称排列的线纹和孔纹,中央平滑或有颗粒。

\subsection{羽纹硅藻(\textit{Pinnularia})}
单细胞或丝状群体。
壳面线状、椭圆形或披针形,两侧平行,有两侧对称横向平行的纹。

\section{绿藻门(Chlorophyta)}
多不能运动,仅繁殖时形成的孢子、配子有鞭毛。
细胞壁成分为纤维素和果胶。
色素体内有蛋白核。
细胞内有淀粉核。

\newline

可营营养繁殖,通过细胞分裂、营养体断裂、产生胶群体等方式形成新个体。
亦营无性生殖,形成孢子(spore)。
亦可有性生殖,产生配子(gamete)。
若相结合的两配子形状、结构、大小、运动能力相同,称为同配生殖(isogamy)。
若两配子形状结构相同,大小和运动能力不同,则称异配生殖(anisogamy)。
若两配子形状大小结构均不同,则为卵式生殖(oogamy)。
若两配子无鞭毛,能变形,则称接合生殖(conjugation)。

\subsection{衣藻(\textit{Chlamydomonas})}
单细胞,卵圆、椭圆或圆形,单核。
前端两条鞭毛,下有两个伸缩泡。
眼点橘红色,位于前端。
细胞壁外层果胶,内层纤维素。
色素体形如厚底杯,基部有蛋白核。
营无性生殖或同配生殖。

\subsection{团藻(\textit{Volvox})}
多细胞。
单个细胞卵圆、椭圆或圆形,外包胶质,排列成单层球形。
球内为胶质。
细胞间有原生质丝。
后端部分细胞无鞭毛且大,为生殖细胞(gonidium)。
无性生殖时,生殖细胞分裂、分化,出芽形成单层细胞球。
而后细胞球翻转,原朝向球内部的一面翻转到球的表面,并脱离母体。
亦营卵式生殖,形成厚壁合子,脱离母体。

\subsection{小球藻(\textit{Chlorella})}
多为单细胞浮游生物,圆形或椭圆形,壁薄。
色素体杯形或曲带形,无蛋白核。
无性生殖。

\subsection{栅藻(\textit{Scenedesmus})}
定型群体。
细胞椭圆或纺锤形,单核。
群体中细胞以长轴平行排列成一行,或互相交错排列成两行。
无性生殖。

\subsection{丝藻(\textit{Ulothrix})}
不分枝丝状体。
基部细胞司固着,色素体颜色浅。
向上为一列短筒形营养细胞,中央有单核。
其色素体大型环带状,多蛋白核。

\subsection{石莼(\textit{Ulva})}
植物体大型多细胞片状体,由两层细胞组成。
基部为无色假根丝,形成固着器。
固着器多年生,每年春天生出新的植物体。
植物体细胞间隙多胶质。
细胞单核,核位于内侧。
色素体片状,有单一蛋白核,位于外侧。

\subsection{刚毛藻(\textit{Cladophora})}
分枝丝状体,固着于基质。
细胞长圆柱形,壁厚,有中央大液泡,多核。
色素体网状,壁生,多蛋白核。
细胞分裂时,侧壁中部生环,向细胞中央生长,分为两个子细胞。
分枝发生于细胞顶端侧面,成二叉状。

\subsection{松藻(\textit{Codium})}
海生,固着于海边岩石。
植物体管状分枝,核多而小。
分枝相互交织,形成大型藻体,似鹿角。
基部为垫状固着器。
分枝中央部分细胞较细,称为髓部。
髓部向四周发出膨大的棒状短枝,称为胞囊(utricle)。
胞囊紧密排列成外围的皮部。
色素体小盘状,位于胞囊远轴端,无蛋白核。

\subsection{水绵(\textit{Spirogyra})}
一列圆柱形细胞构成不分枝丝状体。
细胞壁内层纤维素,外层果胶,藻体滑腻。
色素体带状,螺旋环绕于胞质中,蛋白核多。
胞内有大液泡。
营接合生殖。

\subsection{轮藻(\textit{Chara})}
植物体分地上和地下部分。
地下部分为单列细胞构成有分枝的假根。
地上部分有主枝、侧枝、小枝之分,其上分节(node)和节间(internode)。
侧枝互生于主枝。
小枝轮生于主枝、侧枝的节上。
营卵式生殖。
生殖器位于小枝节上。
精囊球位于刺状细胞下方,成熟时呈橘红色。
卵囊球位于刺状细胞下方。
亦营营养生殖。

\section{红藻门(Rhodophyta)}
常多细胞。
细胞壁内层纤维素,外层果胶,亦有琼胶等红藻特有化合物。
色素体星芒状,多藻红素。
故藻体多呈红色。
无性生殖产生的孢子无鞭毛。
多雌雄异株。
大多海生,固着生活。

\subsection{紫球藻(\textit{Porphyridium})}
单细胞,卵圆或椭圆。
色素体中轴位,有蛋白核。
常生活于潮湿地面。

\subsection{紫菜(\textit{Porphyra})}
藻体叶状,薄,紫红色,固着于海岸岩石。
外有胶质层。
细胞单核,一色素体。
如甘紫菜(\textit{Porphyra tonera})。

\subsection{多管藻(\textit{Polysiphonia})}
植物体为多列细胞构成的分枝丝状体。
丝状体中央为一列较粗的细胞构成的中轴管(central siphon),外围由较细的细胞构成围轴管(peripheral siphon)。
精子囊葡萄状。

\section{褐藻门(Phaeophyta)}
多细胞。
部分品种已初步具有根、茎、叶的分化,亦有表皮、皮层和髓的分化。
细胞壁内层纤维素,外层藻胶。
色素体粒状或小盘状,呈褐色。
精子和游动孢子常有两条不等长的鞭毛。
营有性生殖、无性生殖和营养生殖。
多海生。

\subsection{水云(\textit{Ectocarpus})}
藻体为单列细胞构成的分枝丝状体,匍匐或直立。
孢子体可营无性生殖,亦可营有性生殖产生配子。
配子发育为配子体,产生配子,合子发育为孢子体。
配子体、孢子体同型。

\subsection{海带(\textit{Laminaria})}
孢子体分为固着器、柄和带叶。
固着器分枝根状,柄圆柱形或略侧扁。
带片不分裂,无中脉,边缘波浪状。
带片和柄分为表皮、皮层和髓三层。
孢子体带片两面有深褐色从生的孢子囊。
游动孢子发育为雌雄配子体,为几十个细胞构成的分枝丝状体。
配子体产生配子,合子发育为孢子体。
如海带(\textit{Laminaria japonica})。

\subsection{鹿角菜(\textit{Pelvetia})}
多固着于海浪冲击的岩石上,褐色,软骨质,二叉分枝,有表皮、皮层和中央髓。
基部有圆盘状的固着器。
有性生殖。
生殖时分枝顶端膨大为生殖托(receptacle),成有柄长角果状,表面有突起,其内为卵囊和精囊。

\end{document}