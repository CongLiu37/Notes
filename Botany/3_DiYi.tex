\documentclass[11pt]{article}

\usepackage[UTF8]{ctex} % for Chinese 

\usepackage{setspace}
\usepackage[colorlinks,linkcolor=blue,anchorcolor=red,citecolor=black]{hyperref}
\usepackage{lineno}
\usepackage{booktabs}
\usepackage{graphicx}
\usepackage{float}
\usepackage{floatrow}
\usepackage{subfigure}
\usepackage{caption}
\usepackage{subcaption}
\usepackage{geometry}
\usepackage{multirow}
\usepackage{longtable}
\usepackage{lscape}
\usepackage{booktabs}
\usepackage{natbib}
\usepackage{natbibspacing}
\usepackage[toc,page]{appendix}
\usepackage{makecell}
\usepackage{amsfonts}
 \usepackage{amsmath}

\title{地衣(Lichens)}
\date{}

\linespread{1.5}
\geometry{left=2cm,right=2cm,top=2cm,bottom=2cm}

\begin{document}

  \maketitle

  \linenumbers
地衣系特定真菌与特定藻类形成的共生体。
形成地衣的真菌多为子囊菌,少数为担子菌。
藻类主要为绿藻中的共球藻(\textit{Trebouxia spp.})和橘色藻(\textit{Trentepohlia spp.}),少数为蓝藻。

\newline

地衣有固定的形态。
壳状地衣(crustose lichen)菌丝紧贴基质,或以假根伸入基质,难以剥离。
叶状地衣(foliose lichen)扁平,分背腹面,以假根或脐附着基质,易剥离。
枝状地衣(fruticose lichen)地衣体分枝。

\newline

地衣结构分异层和同层。
异层地衣(heteromerous lichen)自上而下分为上皮层(upper cortex)、藻胞层(algae layer)、髓层(medulla)和下皮层(lower cortex)。
上下皮层由菌丝交织而成,下皮层较薄。
藻胞层有藻细胞分布于菌丝之间。
髓层菌丝疏松。
同层地衣(homolomerous lichen)藻细胞在髓层均匀分布。

\newline

地衣常通过断裂或形成特殊结构进行营养繁殖。
粉芽(soredium)是散布于地衣表面的粉状结构,由菌丝缠绕藻细胞形成。
珊瑚芽(isidium)是地衣上突起的瘤状结构,有皮层,内包藻细胞群。
小裂片(lobule)是叶状地衣边缘形成的扁平突出。

\newline

地衣中的藻类和真菌亦营有性生殖。

\newline

地衣生长缓慢,能适应极端生境。
但地衣容易吸收环境污染,导致死亡。

\section{松萝(\textit{Usnea})}
丛生枝状。
分枝圆柱或棱柱形,常有软骨质中轴。

\section{梅衣(\textit{Parmelia})}
薄叶状,上下皮层间无空腔。
背面灰色、褐色或黄绿,腹面淡色、褐色或黑色。
有假根。

\section{文字衣(\textit{Graphis})}
壳状,生于树皮。
皮层不发达。

\section{地卷衣(\textit{Peltigera})}
叶状或鳞片状。

\section{石蕊(\textit{Cladonia})}
柱状或树枝状,末端可能扩大。

\section{扇衣(\textit{Cora})}
生于树上,有侧生假根。
同层地衣。
地衣体下表面有同心环状排列的突起,实为子实层。

\end{document}