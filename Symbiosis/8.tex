\documentclass[11pt]{article}

% \usepackage[UTF8]{ctex} % for Chinese 

\usepackage{setspace}
\usepackage[colorlinks,linkcolor=blue,anchorcolor=red,citecolor=black]{hyperref}
\usepackage{lineno}
\usepackage{booktabs}
\usepackage{graphicx}
\usepackage{float}
\usepackage{floatrow}
\usepackage{subfigure}
\usepackage{caption}
\usepackage{subcaption}
\usepackage{geometry}
\usepackage{multirow}
\usepackage{longtable}
\usepackage{lscape}
\usepackage{booktabs}
\usepackage{natbib}
\usepackage{natbibspacing}
\usepackage[toc,page]{appendix}
\usepackage{makecell}
\usepackage{amsfonts}
 \usepackage{amsmath}

\title{Repeated replacement of an intrabacterial symbiont in the tripartite nested mealybug symbiosis}
\author{}
\date{}

\linespread{1.5}
\geometry{left=2cm,right=2cm,top=2cm,bottom=2cm}

\begin{document}
\begin{sloppypar}
  \maketitle

  \linenumbers

Citrus mealybug \textit{Planococcus citri} has two bacterial endosymbionts with an unusual nested arrangement: the $\gamma$-proteobacterium \textit{Moranella endobia} lives in the cytoplasm of the $\beta$-proteobacterium \textit{Tremblaya princeps}. 
To test the stability of this three-way symbiosis, host and symbiont genomes for five diverse mealybug species were sequenced. 
$\beta$-proteobacterial genomes from diverse mealybug species are \textit{Tremblaya} with similar genome sizes, while $\gamma$-proteobacteria are from different clades with different genome sizes. 
Therefore, it is inferred that \textit{Tremblaya} is the result of a single infection in the ancestor of mealybugs, while the $\gamma$-proteobacterial symbionts result from multiple replacements of inferred different ages from related but distinct bacterial lineages. 

\par

Three scenario of the order and timing of the $\gamma$-proteobacterial infections are proposed. 
In idosyncratic scenario, there was a single $\gamma$-proteobacterial acquisition in the ancestor of the Pseudococcinae that has evolved idiosyncratically as mealybugs diversified over time, leading to seemingly unrelated genome structures and coding capacities. 
In independent scenario, the $\gamma$-proteobacterial infections occurred independently, each establishing symbioses inside \textit{Tremblaya} in completely unrelated and separate events. 
In replacement scenario, there was a single $\gamma$-proteobacterial acquisition in the Pseudococcinae ancestor that has been replaced in some mealybug lineages over time. 

\par

The idosyncratic scenario can be discarded as phylogenetics of $\gamma$-proteobacterial symbionts reveals that they have originated from clearly distinct and well-supported bacterial lineages. 
The independent and replacement scenarios are more difficult to tell apart. 
Under the independent scenario, \textit{Tremblaya} may experience two rounds of genome corruption: one after association with mealybug ancestor, and one after infection of $\gamma$-proteobacteria. 
Therefore, one should expect diverse genome sizes in $\beta$- and $\gamma$-proteobacteria. 
Conserved genome sizes of $\beta$-proteobacteria and diverse sizes of $\gamma$-proteobacterial genomes favor the replacement scenario. 

\par

Two reasons why $\gamma$-proteobacteria end with living inside $\beta$-proteobacteria are proposed. 
The first is that it was easier to use the established transport system between the insect cell and \textit{Tremblaya} than to evolve a new one. 
The second is that the insect immune system likely does not target \textit{Tremblaya} cells, and so the \textit{Tremblaya} cytoplasm is an ideal hiding place for a newly arrived symbiont.
  
\end{sloppypar}
\end{document}