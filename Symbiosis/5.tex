\documentclass[11pt]{article}

% \usepackage[UTF8]{ctex} % for Chinese 

\usepackage{setspace}
\usepackage[colorlinks,linkcolor=blue,anchorcolor=red,citecolor=black]{hyperref}
\usepackage{lineno}
\usepackage{booktabs}
\usepackage{graphicx}
\usepackage{float}
\usepackage{floatrow}
\usepackage{subfigure}
\usepackage{caption}
\usepackage{subcaption}
\usepackage{geometry}
\usepackage{multirow}
\usepackage{longtable}
\usepackage{lscape}
\usepackage{booktabs}
\usepackage{natbib}
\usepackage{natbibspacing}
\usepackage[toc,page]{appendix}
\usepackage{makecell}
\usepackage{amsfonts}
 \usepackage{amsmath}

\title{Peptidoglycan production by an insect-bacterial mosaic}
\author{}
\date{}

\linespread{1.5}
\geometry{left=2cm,right=2cm,top=2cm,bottom=2cm}

\begin{document}
\begin{sloppypar}
  \maketitle

  \linenumbers
Citrus mealybug \textit{Planococcus citri} has two bacterial symbionts: \textit{Tremblaya princeps} that lives in bacteriocytes of mealybug, and \textit{Moranella endobia} that lives in \textit{Tremblaya}. 

\newline

Genomics predicts a complete pathway for synthesis of peptidoglycan, composed of \textit{Moranella} genes and several genes that were horizontally transferred from bacteria to the nuclear genome of mealybugs. 
No related genes are found in \textit{Tremblaya}. 
Peptidoglycan constitutes are detected in whole insect preparation, and the peptidoglycan-specific molecule D-Ala ss specifically localized at the \textit{Moranella} periphery. 
A peptidoglycan-targeting antibiotic specifically affects the \textit{Moranella} cell envelope. 
A peptidoglycan-related horizontal gene transfer of Alphaproteobacterial origin is localized to the \textit{Moranella} cytoplasm.

\newline

Peptidoglycan-based cell wall is an ancient and defining feature of bacteria, and peptidoglycan biosynthesis pathway is often highly conserved in bacterial genomes. 
There are two exceptions of this pattern. 
The first example of peptidoglycan-related horizontal gene transfers comes from the chromatophore of the rhizarian protist \textit{Paulinella chromatophora}. 
\textit{Paulinella} chromatophore has a peptidoglycan layer (Kies, 1974), which is encoded primarily on the chromatophore genome with the exception of one bacterial horizontal gene transfer to the host protist genome (Nowack \textit{et al.}, 2016). 
The second example comes from the group of photosynthetic eukaryotes whose ancestor formed the original endosymbiosis with the cyanobacterium that became the chloroplast. 
This group, called the Archaeplastida, includes land plants, red algae, green algae, and glaucophyte algae (Lane and Archibald, 2008; McFadden, 2001). 
Many archaeplastidal nuclear genomes encode some peptidoglycan-related endosymbiosis gene transfers and horizontal gene transfers (van Baren \textit{et al.}, 2016; Sato and Takano, 2017), but these genes do not always seem to work together to form a functional peptidoglycan layer at the chloroplast periphery. 
A chloroplast-localized peptidoglycan layer has been verified using fluorescently labeled D-Ala in the moss \textit{Physcomitrella patens} (Hirano \textit{et al.}, 2016), and possible chloroplast peptidoglycan layers have been observed by EM in glaucophytes (Schenk, 1970). 
But in the land plant \textit{Arabidopsis thaliana}, which retains some peptidoglycan-related genes on its nuclear genome, no peptidoglycan layer exists at the chloroplast periphery and at least one peptidoglycan-related enzyme has been coopted for a different function (Garcia \textit{et al.}, 2008). 
These results serve as a cautionary note about inferring function from genomics alone: gene presence is not a reliable predictor of biological function (Doolittle, 2013).

\newline

One important remaining question is the source of D-Ala and D-Glu in \textit{Moranella}’s peptidoglycan, as homologs of Alr (alanine racemase) and MurI (glutamate racemase) do not present as horizontal gene transfers on mealybug genome or present in \textit{Moranella} genome. 
These activities might be moonlighted by other genes. 
For example, GlyA and MetC have been shown to moonlight as alanine racemases in \textit{Chlamydia trachomatis} and \textit{Escherichia coli}, respectively (De Benedetti \textit{et al.}, 2014; Kang \textit{et al.}, 2011; Otten \textit{et al.}, 2018), and eukaryotic homologs for these genes present on the mealybug genome. 
Similarly, DapF has been shown to moonlight as a glutamate racemase in \textit{Chlamydia trachomatis} (Liechti \textit{et al.}, 2018), and this gene exists as an horizontal gene transfer of alphaproteobacterial origin on citrus mealybug genome (Husnik \textit{et al.}, 2013). 
It is also possible that the source of D-Ala and D-Glu is not from these putatively moonlighting enzymes at all, but rather from either the plant sap diet of the insect or from D-amino acids in citrus mealybug produced from normal insect
biochemistry. 
D-amino acids have been found in both plants (Robinson, 1976) and insects (Auclair and Patton, 1950; Corrigan and Srinivasan, 1966; Corrigan,1969), although the levels of these compounds have not been measured in citrus mealybug. 

\newline

MurF encoded by horizontal transferred genes in mealybug genome is localized specifically in \textit{Moranella} cytoplasm. 
Importing enzyme/mRNA into \textit{Moranella} cytoplasm for peptidoglycan synthesis may help avoid immune responses. 
Other insects with long-term endosymbionts devote resources to scavenging peptidoglycan fragments in order to prevent continuous immune activation (Maire \textit{et al.}, 2019). 
By sequestering peptidoglycan production to inside of \textit{Moranella}, citrus mealybug may avoid the need for such contingency pathways, at least until \textit{Moranella} cells are recycled near the end of the mealybug’s life (Kono \textit{et al.}, 2008).

\newline

Most peptidoglycan-related genes on mealybug genome (horizontal transferred) function in the cytoplasmic part of peptidoglycan synthesis, whereas the peptidoglycan-related genes retained by \textit{Moranella} all code for inner membrane- or periplasm-associated proteins. 
It indicates that genes functioning in \textit{Moranella} cytoplasm are more likely to be transferred to insect genome, and this may reflect the mechanisms through which proteins/RNA encoded by insect are transported into symbionts. 

\newline

Host takeover of endosymbiont peptidoglycan production can be an important step in the regulation of endosymbiont cell division and potentially further integration with the host organism (de Vries and Gould, 2018). 
In moss, knocking out a peptidoglycan-related horizontal gene transfer on the nuclear genome results in enlarged chloroplasts
(Machida \textit{et al.}, 2006), and treatment with various peptidoglycan-targeting antibiotics results in fewer and larger chloroplasts per host cell (Katayama \textit{et al.}, 2003). 
Together these data suggest that the movement of peptidoglycan-related genes from organelle genome to the host is a way for hosts to regulate organelle division (de Vries and Gould, 2018; Katayama \textit{et al.}, 2003; Machida \textit{et al.}, 2006). 
In citrus mealybugs, \textit{Tremblaya} was acquired before \textit{Moranella} (Hardy \textit{et al.}, 2008; Thao \textit{et al.}, 2002), and so the host insect must have found a way of controlling \textit{Tremblaya} as the sole endosymbiont prior to the acquisition of \textit{Moranella}. 
It is tempting to speculate that peptidoglycan-related horizontal gene transfers have been retained on the insect genome as a way of controlling the cell division of a bacterium that lives inside of another bacterium inside of insect cells.

\end{sloppypar}
\end{document}