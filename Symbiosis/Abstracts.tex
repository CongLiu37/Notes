\documentclass[11pt]{article}

% \usepackage[UTF8]{ctex} % for Chinese 

\usepackage{setspace}
\usepackage[colorlinks,linkcolor=blue,anchorcolor=red,citecolor=black]{hyperref}
\usepackage{lineno}
\usepackage{booktabs}
\usepackage{graphicx}
\usepackage{float}
\usepackage{floatrow}
\usepackage{subfigure}
\usepackage{caption}
\usepackage{subcaption}
\usepackage{geometry}
\usepackage{multirow}
\usepackage{longtable}
\usepackage{lscape}
\usepackage{booktabs}
\usepackage{natbib}
\usepackage{natbibspacing}
\usepackage[toc,page]{appendix}
\usepackage{makecell}
\usepackage{amsfonts}
 \usepackage{amsmath}

\title{Abstracts}
\author{}
\date{}

\linespread{1.5}
\geometry{left=2cm,right=2cm,top=2cm,bottom=2cm}

\begin{document}
\begin{sloppypar}
  \maketitle

  \linenumbers

\textbf{Gomez-Polo \textit{et al.}, 2017, An exceptional family: Ophiocordyceps-allied fungus dominates the microbiome of soft scale insects (Hemiptera Sternorrhyncha: Coccidae).} \newline
Ribosomal genes from seven soft scale (Coccidae) species showed high prevalence of an \textit{Ophiocordyceps}-allied fungal symbiont, which is from an lineage widely known as entomopathogenic. 
The \textit{Ophiocordyceps}-allied fungus from soft scales is closely related to fungi described from other hemipterans, and they appear to be monophyletic, although the phylogenies of the \textit{Ophiocordyceps}-allied fungi and their hosts do not appear to be congruent. 
Microscopic observations show that the fungal cells are lemon-shaped, are distributed throughout the host’s body and are
present in the eggs, suggesting vertical transmission.

\par

\textbf{Deng \textit{et al.}, 2021, The ubiquity and development-related abundance dynamics of Ophiocordyceps fungi in soft scale insects.} \newline
Nuclear ribosomal internal transcribed spacer (ITS) gene fragment was used to analyze the diversity of fungal communities in 28 soft scale (Coccidae) species. 
Coccidae-associated \textit{Ophiocordyceps} fungi (COF) were prevalent in all 28 tested species with high relative abundance. 
Meanwhile, the first and second instars of \textit{C. japonicus} had high relative abundance of COF, while the relative
abundances in other stages were low, ranging from 0.68\% to 2.07\%. 
The result of fluorescent in situ hybridization showed that the COF were widely present in \textbf{hemolymph} and vertically transmitted from mother to offspring. 

\par

\textbf{Garcia \textit{et al.}, 2014, The symbiont side of symbiosis: do microbes really benefit?} \newline
It has been presumed that microbial symbionts receive host-derived nutrients or a competition-free environment with reduced predation, but there have been few empirical tests, or even critical assessments, of these assumptions. 
Evaluation of these hypotheses based on available evidence indicates reduced competition and predation are not universal benefits for symbionts. 
Some symbionts do receive nutrients from their host, but this has not always been linked to a corresponding increase in symbiont fitness.

\par

\textbf{Costello \textit{et al.}, 2012, The application of ecological theory toward an understanding of the human microbiome.} \newline
Review of three core scenarios of human microbiome assembly: 
development in infants, representing assembly in previously unoccupied habitats; 
recovery from antibiotics, representing assembly after disturbance; 
and invasion by pathogens, representing assembly in the context of invasive species. 

\par

\textbf{Chong & Moran, 2016, Intraspecific genetic variation in hosts affects regulation of obligate heritable symbionts.} \newline
The extent of intraspecific variation in the regulation of a mutually obligate symbiosis, between the pea aphid \textit{Acyrthosiphon pisum} and its maternally transmitted symbiont \textit{Buchnera aphidicola}, using experimental crosses to identify effects of host genotypes. 
Symbiont titer, as the ratio of genomic copy numbers of symbiont and host, as well as developmental time and fecundity of hosts, were measured. 
There was a large (>10-fold) range in symbiont titer among genetically distinct aphid lines harboring the same \textit{Buchnera} haplotype. 
Aphid clones also vary in fitness, measured as developmental time and fecundity, and genetically based variation in titer is correlated with host fitness, with higher titers corresponding to lower reproductive rates of hosts. 
The results show that obligate symbiosis is not static but instead is subject to short-term evolutionary dynamics, potentially reflecting coevolutionary interactions between host and symbiont.

\textbf{Hosokawa \textit{et al.}, 2007, Obigate symbiont involved in pest status of host insect.} \newline
A pest stinkbug species, \textit{Megacopta punctatissima}, performed well on crop legumes, while a closely related non-pest species, \textit{Megacopta cribraria}, suffered low egg hatch rate on the plants. 
When their obligate gut symbiotic bacteria were experimentally exchanged between the species, their performance on the crop legumes was completely reversed: 
the pest species suffered low egg hatch rate, whereas the non-pest species restored normal egg hatch rate and showed good performance. 
The low egg hatch rates were attributed to nymphal mortality before or upon hatching, which were associated with the symbiont from the non-pest stinkbug irrespective of the host insect species.

\par

\textbf{Henry \textit{et al.}, 2013, Horizontally Transmitted Symbionts and Host Colonization of Ecological Niches.} \newline
Facultative or “secondary” symbionts are common in eukaryotes, particularly insects. 
While not essential for host survival, they often provide significant fitness benefits. 
It has been hypothesized that secondary symbionts form a “horizontal gene pool” shuttling adaptive genes among host lineages in an analogous manner to plasmids and other mobile genetic elements in bacteria. 
However, we do not know whether the distributions of symbionts across host populations reflect 
random acquisitions followed by vertical inheritance or whether 
the associations have occurred repeatedly in a manner consistent with a dynamic horizontal gene pool. 
Here we explore these questions using the phylogenetic and ecological distributions of secondary symbionts carried by 1,104 pea aphids, \textit{Acyrthosiphon pisum}. 
We find that not only is horizontal transfer common, but it is also associated with aphid lineages colonizing new ecological niches, including novel plant species and climatic regions. 
Moreover, aphids that share the same ecologies worldwide have independently acquired related symbiont genotypes, suggesting significant involvement of symbionts in their host’s adaptation to different niches. 
We conclude that the secondary symbiont community forms a horizontal gene pool that influences the adaptation and distribution of their insect hosts. 
These findings highlight the importance of symbiotic microorganisms in the radiation of eukaryotes.

\par

\textbf{Hansen & Moran, 2013, The impact of microbial symbionts on host plant utilization by herbivorous insects.} \newline
Herbivory, defined as feeding on live plant tissues, is characteristic of highly successful and diverse groups of insects and represents an evolutionarily derived mode of feeding. Plants present various nutritional and defensive barriers against herbivory; nevertheless, insects have evolved a diverse array of mechanisms that enable them to feed and develop on live plant tissues. For decades, it has been suggested that insect-associated microbes may facilitate host plant use, and new molecular methodologies offer the possibility to elucidate such roles. Based on genomic data, specialized feeding on phloem and xylem sap is highly dependent on nutrient provisioning by intracellular symbionts, as exemplified by Buchnera in aphids, although it is unclear whether such symbionts play a substantive role in host plant specificity of their hosts. Microorganisms present in the gut or outside the insect body could provide more functions including digestion of plant polymers and detoxification of plant-produced toxins. However, the extent of such contributions to insect herbivory remains unclear. We propose that the potential functions of microbial symbionts in facilitating or restricting the use of host plants are constrained by their location (intracellular, gut or environmental), and by the fidelity of their associations with insect host lineages. Studies in the next decade, using molecular methods from environmental microbiology and genomics, will provide a more comprehensive picture of the role of microbial symbionts in insect herbivory.

\par

\textbf{Couret \textit{et al.}, 2019, Even obligate symbioses show signs of ecological contingency: Impacts of symbiosis for an invasive stinkbug are mediated by host plant context.} \newline
Many species interactions are dependent on environmental context, yet the benefits of obligate, mutualistic microbial symbioses to their hosts are typically assumed to be universal across environments. 
We directly tested this assumption, focusing on the symbiosis between the sap-feeding insect \textit{Megacopta cribraria} and its primary bacterial symbiont \textit{Candidatus} Ishikawaella capsulata. 
We assessed host development time, survival, and body size in the presence and absence of the symbiont on two alternative host plants and in the insects' new invasive range. 
We found that association with the symbiont was critical for host survival to adulthood when reared on either host plant, with few individuals surviving in the absence of symbiosis. 
Developmental differences between hosts with and without microbial symbionts, however, were mediated by the host plants on which the insects were reared. 
Our results support the hypothesis that benefits associated with this host–microbe interaction are environmentally contingent, though given that few individuals survive to adulthood without their symbionts, this may have minimal impact on ecological dynamics and current evolutionary trajectories of these partners.

\end{sloppypar}
\end{document}