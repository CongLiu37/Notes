\documentclass[11pt]{article}

% \usepackage[UTF8]{ctex} % for Chinese 

\usepackage{setspace}
\usepackage[colorlinks,linkcolor=blue,anchorcolor=red,citecolor=black]{hyperref}
\usepackage{lineno}
\usepackage{booktabs}
\usepackage{graphicx}
\usepackage{float}
\usepackage{floatrow}
\usepackage{subfigure}
\usepackage{caption}
\usepackage{subcaption}
\usepackage{geometry}
\usepackage{multirow}
\usepackage{longtable}
\usepackage{lscape}
\usepackage{booktabs}
\usepackage{natbib}
\usepackage{natbibspacing}
\usepackage[toc,page]{appendix}
\usepackage{makecell}
\usepackage{amsfonts}
 \usepackage{amsmath}

\title{Abstracts}
\author{}
\date{}

\linespread{1.5}
\geometry{left=2cm,right=2cm,top=2cm,bottom=2cm}

\begin{document}
\begin{sloppypar}
  \maketitle

  \linenumbers

\textbf{1. Gomez-Polo et al., 2017, An exceptional family: Ophiocordyceps-allied fungus dominates the microbiome of soft scale insects (Hemiptera Sternorrhyncha: Coccidae).} \newline
Ribosomal genes from seven soft scale (Coccidae) species showed high prevalence of an \textit{Ophiocordyceps}-allied fungal symbiont, which is from an lineage widely known as entomopathogenic. 
The \textit{Ophiocordyceps}-allied fungus from soft scales is closely related to fungi described from other hemipterans, and they appear to be monophyletic, although the phylogenies of the \textit{Ophiocordyceps}-allied fungi and their hosts do not appear to be congruent. 
Microscopic observations show that the fungal cells are lemon-shaped, are distributed throughout the host’s body and are
present in the eggs, suggesting vertical transmission.

\par

\textbf{2. Deng et al., 2021, The ubiquity and development-related abundance dynamics of Ophiocordyceps fungi in soft scale insects.} \newline
Nuclear ribosomal internal transcribed spacer (ITS) gene fragment was used to analyze the diversity of fungal communities in 28 soft scale (Coccidae) species. 
Coccidae-associated \textit{Ophiocordyceps} fungi (COF) were prevalent in all 28 tested species with high relative abundance. 
Meanwhile, the first and second instars of \textit{C. japonicus} had high relative abundance of COF, while the relative
abundances in other stages were low, ranging from 0.68\% to 2.07\%. 
The result of fluorescent in situ hybridization showed that the COF were widely present in \textbf{hemolymph} and vertically transmitted from mother to offspring. 

\par

\textbf{3. Garcia et al., 2014, The symbiont side of symbiosis: do microbes really benefit?}
It has been presumed that microbial symbionts receive host-derived nutrients or a competition-free environment with reduced predation, but there have been few empirical tests, or even critical assessments, of these assumptions. 
Evaluation of these hypotheses based on available evidence indicates reduced competition and predation are not universal benefits for symbionts. 
Some symbionts do receive nutrients from their host, but this has not always been linked to a corresponding increase in symbiont fitness.

\par

\textbf{Costello et al., 2012, The application of ecological theory toward an understanding of the human microbiome.}
Review of three core scenarios of human microbiome assembly: 
development in infants, representing assembly in previously unoccupied habitats; 
recovery from antibiotics, representing assembly after disturbance; 
and invasion by pathogens, representing assembly in the context of invasive species. 

\end{sloppypar}
\end{document}