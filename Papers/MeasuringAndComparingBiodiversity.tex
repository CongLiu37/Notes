\documentclass[11pt]{article}

\usepackage{setspace}
\usepackage[colorlinks,linkcolor=blue,anchorcolor=red,citecolor=black]{hyperref}
\usepackage{lineno}
\usepackage{booktabs}
\usepackage{graphicx}
\usepackage{float}
\usepackage{floatrow}
\usepackage{subfigure}
\usepackage{caption}
\usepackage{subcaption}
\usepackage{geometry}
\usepackage{multirow}
\usepackage{longtable}
\usepackage{lscape}
\usepackage{booktabs}
\usepackage{natbib}
\usepackage{natbibspacing}
\usepackage[toc,page]{appendix}
\usepackage{makecell}
\usepackage{amsfonts}

\title{Measuring and Comparing Biodiversity}
\date{}

\linespread{1.5}
\geometry{left=2cm,right=2cm,top=2cm,bottom=2cm}

\begin{document}

\bibliographystyle{unsrt}
% \bibliographystyle{plainnat}
% \bibliographystyle{plainnat}

\setcitestyle{round}

  \maketitle

  \linenumbers
\section{Diversity of species, phylogeny and function}
For an assemblage of individuals, its diversity can be measured in ways different in incorporation of species difference \citep{chao2014unifying}. 
In species diversity, all species are assumed to be equally distinct. 
Differences between species
Measures of species diversity can be generalized to incorporate phylogenetic differences. 
All else being equal, an assemblage of closely related species is less diverse in terms of phylogeny than an assemblage composed of highly divergent species. 
When species are described by a set of traits



\section{Classic measures of diversity}

\newpage
  
\bibliography{./MeasuringAndComparingBiodiversity.bib}
  
\end{document}