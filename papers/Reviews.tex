\documentclass[11pt]{article}

% \usepackage[UTF8]{ctex} % for Chinese 

\usepackage{setspace}
\usepackage[colorlinks,linkcolor=blue,anchorcolor=red,citecolor=black]{hyperref}
\usepackage{lineno}
\usepackage{booktabs}
\usepackage{graphicx}
\usepackage{float}
\usepackage{floatrow}
\usepackage{subfigure}
\usepackage{caption}
\usepackage{subcaption}
\usepackage{geometry}
\usepackage{multirow}
\usepackage{longtable}
\usepackage{lscape}
\usepackage{booktabs}
\usepackage{natbibspacing}
\usepackage[toc,page]{appendix}
\usepackage{makecell}
\usepackage{amsfonts}
 \usepackage{amsmath}

\usepackage[backend=bibtex,style=authoryear,sorting=nyt,maxnames=1]{biblatex}
\bibliography{}  %Reference bib

\title{Abstracts: Reviews}
\author{}
\date{}

\linespread{1.5}
\geometry{left=2cm,right=2cm,top=2cm,bottom=2cm}

\setlength\bibitemsep{0pt}

\begin{document}
\begin{sloppypar}
  \maketitle

  \linenumbers
\textbf{Schmidt \textit{et al.}, 2008, Insect and veterbrate immunity: key similarities versus differences.} \newline
Abbreviations: \newline
PAMP: pathogen-associated molecular pattern.\newline
PRR: pattern-recognition receptors.\newline
LPS: lipopolysaccharide.\newline
PGN: peptidoglycan.\newline
LTA: lipoteichoic acid.
\par
Distinction between self and non-self relies on pattern-recognition receptors (PRRs) that bind to diagnostic sites for potential pathogens, or pathogen-associated molecular patterns (PAMPs). 
One precondition for sensing non-self by PAMP recognition is that these molecular patterns are conserved enough to allow the host to evolve binding proteins before the pathogen is able to eliminate or modify the target site. 
Common PAMPs include bacterial lipopolysaccharide (LPS), peptidoglycan (PGN), lipoteichoic acid (LTA) and fungal beta-1, 3-glucans.
\par
Extracellular lipid particles are involved in systemic immune response to pathogens. 
Apolipoprotein III is sensitive to both particle lipid composition and immune elicitors. 
Moreover, lipid particles are associated with typical immune proteins including prophenoloxidase and its upstream proteins, such as LPS- and PGN-binding proteins. 
\par
Recognition of self (histocompatibility and self-incompatibility) and altered-self (apoptotic and tumor cells)
\par
Antimicrobial peptides are defense molecules against microbes by permeation and disruption of target membranes. 
They often kill microorganism via non-receptor-mediated mechanisms, although some bind to bacterial cell wall components (\textit{e.g.} nisin Z). 
\par
Phagocytosis is the cellular uptake of particular substrate. 
It is a fundamental cellular process in eukaryotes and essential for the clearance of damaging objects in multicellular organisms. 
In many animals, specialized cells engage in phagocytosis, such as phagocytes in vertebrates and macrophage-like hemocytes in insects.
\par
Endocytosis
\par
Adaptive immunity of vertebrates is fundamentally different from innate immunity. 
In adaptive immunity, the anticipatory nature of antibody repertoires is capable of binding epitopes never encountered by the organism or its predecessors using direct antibody-epitope specific binding. 
Self-recognizing antibody-producing cells are removed by clonal selection during ontogeny. 
The specific propagation of antibody-producing immune cells provides the basis for an immunological memory. 
Instead, in innate immunity, PRRs are acquired through evolutionary processes resulting from exposure to pathogens over generations. 
Retaining pathogen-binding proteins and removing self-recognizing proteins are facilitated at population level. 
\par
Although lack of adaptive immunity, insects are able to induce immune activity after sub-lethal encounters with pathogens. 
Exposure to sub-lethal concentration of damaging objects enables latter survival under lethal level. 
This immune induction and protection comes with fitness cost, which is often expressed as a delay in development. 
Moreover, the induction of immune defense can be maternally transmitted to subsequent generations, occurring by potential epigenetic mechanisms or the incorporation of female-derived immune-inducible material into oocytes.
\par \newline \newline
\textbf{Strand, 2008, Insect hemocytes and their role in immunity.}
\section{Abbreviations}
AMP: antimicrobial peptide.\newline
PO: phenoloxidase.\newline
PPO1: proPO 1.\newline
JNK: Jun kinase.\newline
PSC: posterior signaling center.\newline
Srp: Serpent.\newline
JAK: Janus kinase.\newline
STAT: signal transducers and activators of transcription.\newline
gcm: glial cell missing.\newline
PRR: pattern recognition receptor.\newline
LPS: lipopolysaccharide.\newline
PGN: peptidoglycan.\newline
LPSBP: LPS-binding protein.\newline
GNBP: Gram-negative binding protein.\newline
PGRP: PGN recognition protein.\newline
GRP: glucan recognition protein.\newline
Dscam: Down's syndrome cell adhesion molecule.\newline
SR: scavenger receptor.\newline
SPZ: Spaetzle.\newline
NF-$\kappa$B: nuclear factor $\kappa$B.
\par
The innate immune system of insects consists of humoral and cellular defense response. 
Humoral defenses refer to soluble molecules including antimicrobial peptides (AMPs), complement-like proteins and products from protealytic cascades such as phenoloxidase (PO) pathway. 
Cellular defenses refer to responses like phagocytosis, encapsulation and clotting that are directly mediated by hemocytes.
\par
Hemocytes have similar function in immunity across insects, but naming of hemocyte types varies among taxa. 
\textit{Drosophila} larvae contain three terminally differentiated hemocyte types: plasmatocytes, crystal cells and lamellocytes. 
Plasmatocytes represent 90-95\% of mature hemocytes, are strongly adhesive \textit{in vitro}. and function as professional phagocytes that engulf pathogens and dead cells. 
Molecular markers for plasmatocytes include extracellular matrix protein peroxidasin and a surface factor P1 antigen. 
Crystal cells represent about 5\% of mature hemocytes. 
They are non-adhesive rounded cells that express PO cascade components such as proPO 1 (PPO1). 
Lamellocytes are absent in healthy \textit{Drosophila} larvae, but rapidly differentiated from prohemocytes after being attacked by parasitoid wasps and during metamorphosis. 
They are large, flat, adhesive cells that express reporters related to Jun kinase (JNK) signaling and L1 antigen. 
The main function of lamellocytes is encapsulation of parasitoids and other large foreign targets. 
Each of these hemocyte types differentiate from precursor prohemocytes that originate from pre-prohemocytes, which mainly reside in hematopoietic organs, and a small number in circulation. 
\par
In Lepidoptera, main differentiated hemocytes in circulation are granulocytes, plasmatocyte, spherule cells and oenocytoids. 
Granulocytes are the most abundant and characterized by the granules in their cytoplasm, the ability to adhere and spread on foreign surface in primary culture, and the tendency to spread systemically. 
They function as professional phagocytes. 
Plasmatocytes are usually larger than granulocytes, spread asymmetrically on foreign surfaces, and are the main capsule-forming hemocytes. 
Non-adhesive hemocytes in larval stage Lepidoptera include oenocytoids that contain PO cascade components, and spherule cells that are potential sources of cuticular components. 
\par
In mosquitoes, hemocyte types include granulocytes, oenocytoids and prohemocytes. 
Granulocytes are strongly adhesive, phagocytic, and the most abundant cell types. 
They express PO activity induced by immune challenge. 
Oenocytoids are non-adhesive and constitutively express PO activity. 
Prohemocytes are characterized by uniform size, rounded morphology and large nuclear. 
It is unknown whether they differentiate into granulocytes/oenocytoids.
\par
Hemocytes arise during two stages of development. 
The first population of hemocytes arises during embryogenesis from head or dorsal mesoderm, and the second is produced during the larval or nymphal stages in mesodermally derived hematopoietic organs. 
The hematopoietic organs of \textit{Drosophila} are lymph glands that form bilaterally along the anterior part of the dorsal vessel during embryogenesis. 
By the third instar, each lymph gland consists of an anterior primary lobe and several posterior secondary lobes separated by pericardial cells. 
The primary lobe has three zones: 
(1) a posterior signaling center (PSC) that contains cells marked by the expression of transcription factor Collier and Notch ligand Serrate; 
(2) a medullary zone that contains quiescent prohemocytes; 
(3) a cortical zone that contains plasmatocytes, crystal cells and following parasitoid attack, lamellocytes. 
Secondary lobes contain pre-prohemocytes, prohemocytes and some plasmatocytes. 
\par
Earliest lymph gland cells, hemocyte precursor cells, are identified by expression of GATA transcription factor homolog Serpent (Srp). 
As transition to pre-prohemocytes, they initiate expression of receptor tyrosine kinase Pvr followed by expressing JAK/STAT (Janus kinase/signal transducers and activators of transcription) signaling pathway receptor Dome, which characterizes maturation of prohemocytes. 
In differentiation of prohemocytes into hemocyte types, Dome is down-regulated. 
Specification of plasmatocytes requires expression of transcription factor glial cell missing (gcm) and gcm2, while crystal cell specification requires Runt-domain protein Lozenge (Lz) and Serrate signaling through Notch. 
The PSC along with JAK/STAT and JNK signaling have been implicated in differentiation of lamellocytes.
\par
The maintenance of hemocytes in circulation involves two aspects: production and release of cells from lymph glands, and proliferation of hemocytes already in circulation. 
Furthermore, the number of circulating hemocytes increases rapidly in response to stress, wounding or infection. 
\par
Immune responses mediated by hemocytes are phagocytosis, encapsulation and clotting. 
Phagocytosis is a conserved defense response in which individual cells internalize and destroy targets. 
It depends on receptor-mediated recognition and binding of the target to a hemocyte followed by formation of a phagosome and engulfment of the target via actin polymerization-dependent mechanisms. 
The phagosome then matures to a phagolysosome by a series of fissin and fusion events with endosomes and lysosomes. 
Insect hemocytes phagocytize bacteria, yeast, fungi, protozoans, apoptotic bodies and inanimate materials like synthetic beads and ink particles. 
\par
Encapsulation refers to the envelopment of large targets by multiple hemocytes. 
In \textit{Drosophila}, the capsules formed around invaders are mainly comprised of lamellocytes. 
In Lepidoptera, formation of capsules is mainly conducted by plasmatocytes, while cooperation of granulocytes are sometimes required for recognition and encapsulation of targets. 
Besides, melanin is often deposited within and around the capsules.
\par
Coagulation of insect hemocytes occurs at sites of external wounding. 
Soft clots initially consist of fibrous matrix embedded with hemocytes, mainly granulocytes (Lepidoptera) or plasmatocytes (\textit{Drosophila}). 
This is followed by clot hardening due to cross-linking of proteins and melanization.
\par
Defense responses including phagocytosis and encapsulation are dependent on recognition of targets as foreign, followed by activation of downstream signaling and effector responses. 
Some foreign invaders are recognized by humoral pattern recognition receptors (PRRs), which bind to targets to enhance recognition by other receptors on hemocyte surface. 
This process is opsonization. 
Other targets are recognized directly by hemocyte surface receptors.
\par
Humoral PRRs can opsonize microorganisms by binding to lipopolysaccharides (LPSs), peptidoglycans (PGNs) and glucans. 
These include 
hemolin, 
LPS-binding proteins (LPSBPs), 
Gram-negative binding protein (GNBPs), 
soluble PGN recognition proteins (PGRP-SA and PGRP-SD), 
glucan recognition proteins (GRPs), 
soluble Down's syndrome cell adhesion molecule (Dscam) 
and complement-like TEP proteins. 
Another group of PGRPs (PGRP-SB1, -SC1a, -SC1b, -SC2) enzymatically degrade PGN. 
This activity kill some bacteria and releases PGN fragments triggering hemocyte effector responses. 
Other humoral molecules implicated in pathogen recognition and opsonization include 
leucine-rich repeat proteins, 
glutamine-rich protein 
and immunolectins. 
The sources of humoral PRRs include hemocytes and other immune tissues, \textit{e.g.} the fat body. 
\par
Cell surface receptors involved in opsonin-independent immunity include 
Peste, a class B scavenger receptor (SR) (or CD36 family member); 
dSR-CI, a class C SR; 
transmembrane protein Eater; 
membrane bound PGRPs (PGRP-LC and its co-receptor PGRP-LE); 
transmembrane form of Dscam; 
class B SR Croquemort; 
low-density lipoprotein (LDL) receptor-related protein LRP1. 
A long version of PGRP-LE can act as intracellular receptor recognizing bacteria. 
Other proteins implicated to be cellular receptors include 
integrins, 
tetraspanin proteins, 
neuroglian (an immunoglobulin superfamily member).
\par
Cytokines are extracellular molecules that regulate hemocyte function. 
These include cysteine-knot-like growth factor Spaetzle (SPZ) that is activated by a protealytic cascade and interacts with Toll receptors located on cell membrane. 
This leads to activation of nuclear factor $\kappa$B (NF-$\kappa$B) transcription factors, which initiate a number of immune genes including several AMPs. 
Upstream PRRs involved in initiating protealytic cascade that lead to SPZ activation include PGRP-SA and soluble GNRPs. 
Cytokine PSP is also processed from a precursor protein by a protealytic cascade. 
After binding to its membrane receptor, PSP simulates plasmatocytes to adhere and spread on foreign surfaces.
\par
In addition to Toll signaling, other pathways also are also activated in hemocytes by cytokine and/or binding of foreign to surface receptors. 
These include Imd pathway activated by PGRP-LC binding with Gram-negative bacteria. 
Imd signaling induces expression of immune effector genes. 
TEP proteins are involved in activation of JAK/STAT signaling, while JNK signaling is associated with phagocytosis and adhesion.  
\par \newline \newline
\textbf{Waterhouse \textit{et al.}, 2020, Characterization of insect immune systems from genomic data.}
Identification of genes involved in physiological processes can be conducted by homology search or transcriptomic analysis. 
Homology search works well for evolutionarily conserved and well-studied canonical gene repertoires, but lose evolutionarily novel or not well-studied genes, which can be complemented by transcriptomic analysis. 
\par
The first step of characterizing canonical gene repertoires in a newly sequenced genome is to compile reference sequences, \textit{i.e.} protein sequences of gene repertoires from reference species that have been characterized. 
It requires define a scope of gene families to be included and select appropriate species from which reference sequences are drawn. 
For immune gene identification, principle components of immune responses should be included, \textit{i.e.} recognition of antigene, signaling transduction and effectors (\ref{Supplement}). 
As for reference species, characterized species of the same order are the most useful, as the lower sequence divergence between more closely related species improves the success of sequence homology searches. 
Besides, closely related species share similar gene family components with less gene gain/loss events. 
\par
\textbf{Gram-negative binding proteins}: Gram-negative binding proteins (GNBPs) or beta-1,3-glucan-­binding proteins (BGBPs) are a family of carbohydrate-binding pattern recognition receptors. \newline
\textbf{Peptidoglycan binding proteins}: PGRPs are pattern recognition receptors capable of recognizing
the peptidoglycan from bacterial cell walls. \newline
\textbf{Fibrinogen-related proteins}: FREPs (also known as FBNs) are a family of pattern recognition
receptors with homology to the C terminus of the fibrinogen beta- and gamma-chains. \newline
\textbf{Galectins}: GALEs bind specifically to beta-galactoside sugars and can function as pattern recognition receptors in innate immunity. \newline
\textbf{MD-2-like proteins}: MLs, also known as Niemann-pick type C-2 proteins, possess myeloid-differentiation-­2-related lipid-­recognition domains involved in recognizing lipopolysaccharide. \newline
\textbf{Nimrods}: NIMs have been shown to bind bacteria leading to their phagocytosis by hemocytes. \newline
\textbf{Scavenger receptors}: SCRs are made up of different classes that function as pattern recognition receptors for a broad range of ligands including from pathogens. \newline
\textbf{Spaetzle-like proteins}: The cleavage of Spaetzle results in binding of the product to the
toll receptor and subsequent activation of the toll pathway; SPZs contain a cystine knot domain. \newline
\par
\textbf{IMD pathway}: Immune deficiency pathway is characterized by peptidoglycan recognition protein receptors, intracellular signal transducers and modulators, and the NF-κB transcription factor relish. \newline
\textbf{Toll pathway}: The intracellular components of Toll pathway signaling are homologous to the Toll-like receptor innate immune pathway in mammals, culminating in activation of the NF-κB transcription factors dorsal and DIF in Drosophila. \newline
\textbf{JAK/STAT pathway}: The Janus kinase protein (JAK) and the signal transducer and activator of transcription (STAT) are two core components of the JAK/STAT pathway, which is involved in cellular responses to stress or injury. \newline
\textbf{RNAi pathway}: RNA interference protects against viral infections employing dicer and Argonaute proteins as well as helicases to identify and destroy exogenous double-­stranded RNAs. \newline
\textbf{Caspase}: Cysteine-aspartic proteases are involved in immune signaling cascades and apoptosis. \newline
\textbf{CLIP-domain serine protease}: Several CLIP proteases have roles as activators or modulators of
immune signaling cascades. \newline
\textbf{Inhibitor of apoptosis}: IAPs are important in antiviral responses and are involved in regulating immune signaling and suppressing apoptotic cell death. \newline
\textbf{Serine protease inhibitors}: Protease inhibition by serpins, or SRPNs, modulates many
signaling cascades; they act as suicide substrates to inhibit their target proteases. \newline
\textbf{Thioester-containing proteins}: TEPs are related to vertebrate complement factors and
alpha2-macroglobulin protease inhibitors; their activation through proteolytic cleavage leads to phagocytosis or killing of pathogens. \newline
\par
\textbf{Antimicrobial peptide}: Antimicrobial peptides (AMPs) are the classical effector molecules of innate immunity; they include defensins, cecropins, and attacins that are involved in bacterial killing by disrupting their membranes. \newline
\textbf{Lysozymes}: LYSs are key effector enzymes that hydrolyze peptidoglycans present in the cell walls of many bacteria, causing cell lysis. \newline
\textbf{C-type lectins}: C-type lections (CTL) are carbohydrate-­binding proteins with roles in pathogen opsonization, encapsulation, and melanization, as well as immune signaling cascades. \newline
\textbf{Prophenoloxidases}: PPOs are key enzymes in the melanization cascade that helps to
kill invading pathogens and is important for wound healing. \newline
\textbf{Peroxidases}: PRDXs are enzymes involved in the metabolism of reactive oxygen species (ROS) that are toxic to pathogens. \newline
\textbf{Superoxide dismutases}: SODs are antioxidant enzymes involved in the metabolism of toxic superoxide into oxygen or hydrogen peroxide.
\par \newline \newline
\textbf{Nazario-Toole \textit{et al.}, 2017, Phagocytosis in insect immunity.}
Abbreviation: \newline
SC: scavenger receptor.\newline
EGF: epidermal growth factor.\newline
PGRP: peptidoglycan recognition receptor.\newline
PGN: peptidoglycan.\newline
AMP: antimicrobial peptide.\newline
Dscam: Down syndrome adhesion molecular.\newline
IgSF: immunoglobulin superfamily.\newline
TEP: thioester-containing protein.\newline
Mcr: macroglobulin complement related.\newline
GEF: guanine nucleotide exchange factor.\newline
ESCRT: endosomal sorting complex required for transport.\newline 
VPS-C: vacuolar protein sorting-C.
\par
Phagocytosis is initiated when phagocytic cell surface receptors recognize their ligands and trigger the engulfment of targets into phagosome. 
Phagocytic receptors can recognize targets directly, or recognize opsonins coating targets. 
Additionally, due to the diversity of targets for phagocytosis, there is overlap and redundancy in receptor-ligand specificities to facilitate recognition. 
It also provides evolutionary advantage as it allows recognition of pathogens that have developed mechanisms to evade detection by a particular receptor. 
\par
Scavenger receptors (SRs) are a family of structurally diversified transmembrane proteins, subdivided into nine classes (Class A-I). 
They exhibit broad ligand specificity, including both altered self and molecular patterns from invaders. 
Croquemort in \textit{Drosophila} is homolog of mammalian CD36 (France \textit{et al.}, 1996). 
It mediates phagocytosis of apoptotic cells and participates immunity against bacteria (France \textit{et al.}, 1999; Stuart \textit{et al.}, 2005). 
Class C SRs are unique to insects, with four members in \textit{Drosophila}: SR-CI, -CII, -CIII and -CIV. 
SR-CI recognizes bacteria, mediating their phagocytosis (Ramet \textit{et al.}, 2001; Ulvila \textit{et al.}, 2006). 
\textit{Drosophila} Peste is a class B SRs and homolog of mammalian CD36. 
It is involved in phagocytosis of bacteria (Philips \textit{et al.}, 2005; Agaisse \textit{et al.}, 2005).
\par
Nimrod family is characterized by epidermal growth factor (EGF)-like repeats called NIM repeats (Kurucz \textit{et al.}, 2007). 
It is divided into three groups: 
(1) draper-type, including \textit{Drosophila} nimrod A and draper; 
(2) nimrod-B type, including \textit{Drosophila} nimrod B 1-5; 
(3) nimrod-C type, including \textit{Drosophila} nimrod C 1-4 and eater. 
\textit{Drosophila} eater protein mediates phagocytosis of bacteria as a pattern recognition receptor (Ramet \textit{et al.}, 2002; Kocks \textit{et al.}, 2005; Chung and Kocks, 2011). 
Nimrod C1 (NimC1) is located on hemocyte plasm membrane and bound to bacteria for phagocytosis (Kurucz \textit{et al.}, 2007). 
\textit{Drosophila} draper is identified as phagocytosis receptor (Freeman \textit{et al.}, 2003).
It is involved in phagocytosis of apoptotic cells (Manaka \textit{et al.}, 2004; Kuraishi \textit{et al.}, 2009; Tung \textit{et al.}, 2013) and bacteria (Cuttell \textit{et al.}, 2008; Hashimoto \textit{et al.}, 2009).
\par
Peptidoglycan-recognition receptors (PGRPs) bind to peptidoglycan (PGN), a polymer restricted to bacterial cell wall. 
There are 13 PGRP genes in \textit{Drosophila}. 
They are upstream of Toll and IMD signaling pathways that regulate the expression of antimicrobial peptides (AMPs) and other effectors. 
In \textit{Drosophila} PGRPs, there are six long (L) form proteins, four of which are located at plasm membrane (Werner \textit{et al.}, 2000). 
The remaining seven short (S) from proteins predicted to be secreted (Werner \textit{et al.}, 2000). 
Members of non-catalytic group (PGRP-SA, -SD, -LA, -LC, -LD, -LE, -LF) serve as pattern recognition receptors. 
They lack the critical cysteine residue in the enzymatic pocket of PGRP domian and are unable to degrade PGN (Mellroth \textit{et al.}, 2003). 
Catalytic PGRPs (PGRP-SC1, -SC2, -LB, -SB1, -SB2) posses amidase activity and degrade PGN (Zaidman-Remy \textit{et al.}, 2011). 
PGRP-SC1a is a receptor for bacteria (Garver \textit{et al.}, 2006). 
Its catalytic activity is required for mediating phagocytosis (Koundakjian \textit{et al.}, 2004). 
PGRP-SA is a pattern recognition receptor with dual roles in \textit{Drosophila} humoral and cellular immunity. 
It activates Toll pathway and thus up-regulates drosomycin, an AMP (Michel \textit{et al.}, 2001). 
PGRP-SA is also important for phagocytosis of Gram-negative bacteria (Garver \textit{et al.}, 2006). 
PGRP-LC is membrane-bound and mediates phagocytosis of Gram-negative but not Gram-positive bacteria (Ramet \textit{et al.}, 2002). 
It is also the major upstream receptor of IMD pathway (Ramet \textit{et al.}, 2002; Choe \textit{et al.}, 2002; Gottar \textit{et al.}, 2002).
\par
Integrin functions as a heterodimer of two transmembrane subunits, $\alpha$ and $\beta$ integrin. 
In \textit{Drosophila}, there are 5 genes coding $\alpha$ integrin, and 2 coding $\beta$ integrin (Brown \textit{et al.}, 2000). 
Integrin heterodimer $\alpha$PS3 and $\beta\nu$ is a receptor for bacteria and apoptotic cells (Nagaosa \textit{et al.}, 2011; Nonaka \textit{et al.}, 2013; Shiratsuchi \textit{et al.}, 2012).
\par
Down syndrome adhesion molecular (Dscam) is a immunoglobulin superfamily (IgSF) in \textit{Drosophila}. 
There are four Dscam-like genes and \textit{Dscam1} is the most extensively characterized (Armitage \textit{et al.}, 2012). 
\textit{Dscam1} is arranged into clusters of variable exons (exon 4, 6, 9, 17) that are flanked by constant exons. 
Via alternative splicing, large isoform repertoires are generated for recognition of diverse ligands (Schmucker \textit{et al.}, 2000). 
\textit{Dscam1} expresses in immune competent tissues of \textit{Drosophila} and acts as phagocytosis receptor (Watson \textit{et al.}, 2005). 
\par
Opsonization is the process by which humoral molecules bind to pathogens and promotes phagocytosis. 
In mammals, antibodies and complement factors act as opsonins. 
Activated complement factors form covalent binding pathogens or altered self, and mark them for phagocytosis. 
\par
Insect thioester-containing proteins (TEPs) share sequence similarity with vertebrate complement factor. 
In \textit{Drosophila}, there are six TEPs (TEPI-VI). 
The present of signal peptide indicates they are secreted proteins. 
TEPV does not seem to be expressed (Lagueux \textit{et al.}, 2000). 
TEPI-IV are closely related to mammalian complement factors as they share a CGEQ motif critical for the formation of thioester bonds with targets. 
\textit{TEPVI}, also called \textit{macroglobulin complement related} (\textit{Mcr}), lacks the critical cysteine residue in the thioester-binding site (Stroschein-Stevenson \textit{et al.}, 2006). 
\par
Signaling from bound phagocytic receptors triggers coordinated rearrangement of the actin cytoskeleton. 
GTPase of Ras superfamily, including Rho-GTPase Cdc42, Rac1 and Rac2, are recruited to the plasma membrane. 
They are activated by binding with GTP, which is facilitated by guanine nucleotide exchange factors (GEFs); 
and inhibited by hydrolysis of GTP by guanine nucleotide disassociation inhibitors. 
\par
\textit{Drosophila} \textit{Zir} is a Rho-GEF that interacts with Cdc42 and Rac2 to mediate larval phagocytosis (Sampson \textit{et al.}, 2012). 
Rac2 activates WAVE. 
WAVE then activates Arp 2/3 complex, which stimulates actin nucleation, the initial step for the formation of new filament structure. 
Cdc42 activates WAS(p), which activates Arp 2/3 complex. 
Cdc 42, Rac1, Rac2 and Arp 2/3 complex are all involved in phagocytosis (Agaisse \textit{et al.}, 2005; Philips \textit{et al.}, 2005; Stroschein-Stevenson \textit{et al.}, 2006; Stuart \textit{et al.}, 2005).
\par
The process of internalization of targets forms a membrane-bound vesicle, the phagosome, which contains targets for degradation. 
Phagosome formation is followed by a series of ordered fission/fusion events with components of endosomal pathway. 
This process, termed as phagocytosome maturation, produces a highly acidic and hydrolytic phagolysosome designed to destroy the targets. 
Phagocytosome maturation involves interactions with early endosomes, recycling endosomes, late endosomes and lysosomes. 
Involved proteins include Rab GTPase, phosphatidylinositol 3-kinase, vacuolar hydrion-ATPase, endosomal sorting complex required for transport (ESCRT) and vacuolar protein sorting-C (VPS-C) complex. 
\par
Phagosome fuse with early endosome quickly (Mayorga \textit{et al.}, 1991). 
GTP ase Dynamin recruits Rab5 to newly formed phagosome (Bucci \textit{et al.}, 1992; Kinchen \textit{et al.}, 2008). 
Rab5 recruits effectors to early endosomal/phagosomal membrane, including early endosome antigen 1 (EEA1), SNARE proteins required for membrane fusion, Vps34 and Vps15 (also called p150, regulatory subunit of Vps34). 
\par
Vps15 is a serine-threonine kinase recruiting Vps34 to early phagosome. 
Vps34 is a class III phosphatidylinositol 3-kinase (PI3-kinase) generating phosphatidylinositol-3-phosphate (PI3P) on early phagosome membrane (Vieira \textit{et al.}, 2001). 
PI3P interacts with Fab1, YOTB, Vac1 and EEA1 via their conserved FYVE domain. 
In \textit{Drosophila}, PI3-kinase 59F (Pi3K59F) is homolog of mammalian Vps34 and functions in cellular immune responses (Qin \textit{et al.}, 2008; Qin \textit{et al.}, 2011). 
Rebenosyn-5, \textit{Drosophila} homolg of EEA1, contains a FYVE domain that binds to PI3P and Rab5 on the phagosome surface, and is required for fusion of early endosomes and phagosomes (Morrison \textit{et al.}, 2008; Simonsen \textit{et al.}, 1998). 
\par
Vaciolar hydrion-ATPase (V-ATPase) comples presents on phagosome membrane and is required for acidification of phagosomal lumen (Beyenbach and Wieczorek, 2006). 
In \textit{Drosophila}, 8 subunits of V-ATPase are important for phagocytosis (Cheng \textit{et al.}, 2005).
\par
During phagosome maturation, multivesicular bodies (MVBs) appear within the phagosome by inward budding and scission of phagosome membrane. 
Transmembrane proteins that are destined for degradation are ubquitinated and sorted into MVBs (Lee \textit{et al.}, 2000). 
\par
After MVB formation, phagosome transitions to late stage, characterized by acidic lumen and several molecules including lysosomal-associated membrane proteins (LAMPs) and hydrolase. 
LAMPs, \textit{e.g.} \textit{Drosophila} Lamp1 (also called CG3305), are required for the last step of phagosome maturation, the fussion of phagosome with lysosome (Huynth \textit{et al.}, 2007; Peltan \textit{et al.}, 2012). 
\par
Additional V-ATPase are acquired by late phagosomes, and the vesicles also acquire Rab GTPase Rab7, a marker of late phagosome (Desjardins \textit{et al.}, 1994). 
Rab7 recruits effectors such as Rab-interacting lysosomal protein, faciliating the movement of phagosome (Harrison \textit{et al.}, 2003; Jordens \textit{et al.}, 2001). 
\par
VPS-C complexes interact with SNAREs and Rabs during phagosome maturation. 
There are two VPS-C complexes: CORVET and HOPS. 
CORVET interacts with Rab5-GTP and promotes early endosome/phagosome fussion. 
HOPS interacts with Rab7-GTP on late endosomes/MVBs and promotes fussion with lysosomes. 
CORVET and HOPS are composed of four shared class C subunits (Vps11, Vps16, Vps18 and Vps33) and two Rab-specific subunits. 
In \textit{Drosophila}, Vps33 and Vps16 have two homologs: car and Vps33B, Vps16A and Vps16B (Li and Blissard \textit{et al.}, 2015; Pulipparacharuvil \textit{et al.}, 2005). 
Vps16A and Vps16B are predicted to associate with HOPS compleses (Pulipparacharuvil \textit{et al.}, 2005). 
Vps16A is required for fussion of autophagosomes with lysosomes (Takats \textit{et al.}, 2015). 
Vps16B mediates phagosome to lysosome fussion (Akbar \textit{et al.}, 2011).
\par
The final step of phagosome maturation is the formation of phagolysosome (pH about 4.5). 
Phagolysosomes are equiped with host factors that impede microbial growth while attacking and degrading pathogens. 
Cofactors of bacterial housekeeping enzymes, such as Fe$^{2+}$, Zn$^{2+}$ and Mn$^{2+}$, are removed from phagolysosome lumen by sequesteration by lactoferrin and removing by membrane-bound protein NRAMP. 
Reactive oxygen (ROS) and nitrogen (RNS) attack bactera. 
ROS is generated by membrane-bound NOX2 NADPH oxidase, which transfers electrons from cytosolic NADPH to molecular oxygen, and releases O$_2^{-}$ to phagolysosome lumen. 
Superoxide dismutase converts O$_2^{-}$ into H$_2$O$_2$, which can be converted into ROS like hypochlorous acid and chloramines. 
RNS is generated by iNOS, the enzyme catalyses the formation of nitric oxide on cytoplasmic side of phagolysosome. 
Nitric oxide dissfuses into phagolysosome lumen, where it encounters ROS and is converted into various RNS that are highly toxic to bacteria. 
Phagosomes are also equiped with onther bactericidal elements: AMPs, peptidase, lipase and hydrolyase.
\par \newline \newline
\textbf{Nakhleh \textit{et al.}, 2017, The melanization response in insect immunity.}
Melanization is an immune response triggered locally in response to cuticle injury or systemically following microbial invasion. 
It is characterized by synthesis of melanin and cross-linking with molecules on microbial surfaces, resulting in killing of invaders. 
Melanization is also linked with coagulation system: coagulation initiates clotting process and melanization contributes to hardening clots (Eleftherianos and Revenis, 2011). 
Besides, it is essential for cuticle sclerotization or tanning that leads to hardening of exoskeleton by cross-linking cuticular proteins by quinones (Andersen, 2010). 
\par
Phenoloxidase (PO) is a key enzyme in melanin synthesis. 
It mediates the oxidation of tyrosine to dihydroxyphenylalanine, and the oxidation of dihydroxyphenylalanine and dopamine to respective quinones, precursors of melanin (Vavricka \textit{et al.}, 2020). 
PO is produced as prophenoloxidase (PPO), which is converted to active PO by a clip domain serine proteinase (CLIP). 
CLIPs are specific to invertebrates and act in cascades to modulate coagulation, melanization and activation of Toll pathway that activates antimicrobial peptides (AMPs) synthesis. 
CLIPs lack one or more of the three residues (His, Asp, Ser) that form catalytic triad are non-catalytic, or called clip-domain containing serine proteinase homologs (cSPHs). 
The rest catalytic CLIPs are known as clip-domain containing serine proteinase (cSP). 
\par
The most upstream proteinase that has been characterized in PPO activation cascades is a modular serine proteinase (ModSp) that lacks clip domain but contains other domain for interactions (Buchon \textit{et al.}, 2009; Ji \textit{et al.}, 2004; Roh \textit{et al.}, 2009; Takahashi \textit{et al.}, 2015). 
ModSps are often autoactivaed and lead to proteolytic cleavage and activation of a CLIPC, which activates a CLIPB that functions as PPO-activating proteinase (Kanost and Jiang, 2015). 
CLIP cascades controling PPO activation is regulated by serpins, a family of serine proteinase inhibitors. 
\par
Insect melanogenesis is initiated by hydroxylation of phenylalanine by phenylalanine 4-monooxygenase (PAH), which forms rate-limiting substrate tyrosine (Futahashi and Fujiwara, 2005; Gorman \textit{et al.}, 2007). 
The tyrosinase-like POs catalyses oxidation of tyrosine into dihydroxyphenylalanine (Dopa), and oxidation of Dopa into dopaquinone. 
With thiol compounds, dopaquinone is converted to cysteinyl and glutathionyl conjugates that mediate synthesis of cutaneous redish pigment phoemelanin. 
Without thiol compounds, dopaquinone undergoes spontaneous cyclization into dopachrome, which in turn is decarboxylated by dopachrome conversion enzyme to generate 5,6-dihyroxyindole (DHI). 
Following PO-meidated DHI oxidation, indole quinones polymerize and give rise to heteropolymer eumelanin. 
DHI-eumelanin can also be derived from dopamine produced early on decarboxylation of dopa by dopa decarboxylase (DDC). 
\par
The infection-induced melanization in \textit{Drosophila melanogaster} requires two CLIPs: MP1 and MP2. 
The proteinase cascade for PPO activation includes MP1 and MP2, while its upstream pattern recognition receptors (PRRs) remain unclear (Tang \textit{et al.}, 2008; An \textit{et al.}, 2013). 
However, PRRs including PGRP-LE (Takehana \textit{et al.}, 2002) and GNBP3 (Matskevich \textit{et al.}, 2010) are involved in melanization without linking to MP1-MP2 module. 
Additionally, another CLIP called Hayan is a key activator of PPO in systemic wound responses (Nam \textit{et al.}, 2012). 
\par
In \textit{Manduca sexta}, beta-glucan recognition proteins betaGRP1 and betaGRP2 trigger PPO activation (Jiang \textit{et al.}, 2004; Ma and Kanost, 2000). 
Binding of betaGRP2 recruits ModSp HP14, which is autoactivated (Wang and Jiang, 2006) and cleaves cSP proHP21 into active HP21. 
HP21 cleaves PPO-activating proteinase-2 zymogen (PAP-2) into active PAP-2, the terminal cSP in the cascade that processes PPO into PO (Wang and Jiang, 2007). 
Additionally, HP21 also cleaves PAP-3 (Gorman \textit{et al.}, 2007), which activates PPO directly (Jiang \textit{et al.}, 1998; Jiang \textit{et al.}, 2003; Jiang \textit{et al.}, 2003). 
PAP-1 is also a direct activator of PPO, but is regulated by a pathway different from HP14-HP21, but requires HP6 (Ann \textit{et al.}, 2009). 
Two cSPHs, SPH1 and SPH2, seem to be required as cofactors for PPO cleavage (Gupta \textit{et al.}, 2005; Yu \textit{et al.}, 2003). 
HP6 also controls Toll pathway by cleaving HP8 (An \textit{et al.}, 2009). 
PPO cascade is subject to a positive feedback. 
PAP-1 activates HP6, hence increases PAP-1 activation (Wang and Jiang, 2008). 
PAP-3 cleaves PPO as well as SPH1, SPH2, PAP-3, and thus leading to a positive feedback loop (Wang \textit{et al.}, 2014). 
Besides, PAP-3 is targeted by several serpins including serpin 1J (Jiang \textit{et al.}, 2003), serpin-3 (Christen \textit{et al.}, 2012), serpin-6 (Wang and Jiang, 2004) and serpin-7 (Suwanchaichinda \textit{et al.}, 2013). 
Serpin-4 and -5 are also involved in regulation of PPO cascade upstream of PAPs (Tong and Kanost, 2005). 
Serpin-4 inhibits HP21, HP6 and HP1 (Tong \textit{et al.}, 2005). 
Serpin-5 inhibits HP6 and HP1 (An and Kanost, 2010). 
\par
In \textit{Tenebrio molitor}, PGRP-SA and GNBP1 act as upstream PRRs of PPO cascade (Park \textit{et al.}, 2006). 
They recruit an autoactivated ModSp, which cleaves downstream cSP called SAE (Kim \textit{et al.}, 2008). 
SPE activates Toll pathway and PPO, and process a precursor of cSPH1 (Kan \textit{et al.}, 2008). 
cSPH1 ligand PO to microbial surface (Zhang \textit{et al.}, 2003). 
PPO cascade is inhibited by serpin 40, serpin 55, serpin 48 (Jiang \textit{et al.}, 2009), and a melanization-inhibiting protein (MIP) inhibits melanization (Zhao \textit{et al.}, 2005).
\par
In \textit{Anopheles gambiae}, complement-like thioester-containing protein 1 (TEP1) promotes melanization (Povelones \textit{et al.}, 2013) and its downstream includes CLIPA8, a cSPH cleaved during melanization response (Volz \textit{et al.}, 2006; Schnitger \textit{et al.}, 2007). 
CLIPA2 is another cSPH that inhibits melanization by controling TEP1 (Volz \textit{et al.}, 2006; Kamareddine \textit{et al.}, 2016; Yassine \textit{et al.}, 2014). 
SPCLIP1 activates TEP1 as cSPH (Povelones \textit{et al.}, 2013). 
Other cSPHs required for melanization include CLIPB17, CLIPB8, CLIPB3 and CLIPB4 (Volz \textit{et al.}, 2006). 
Serpin 2 inhibits PPO cascade by targeting several cSPs (Michel \textit{et al.}, 2005). 
One of its targets is CLIPB9, which is predicted as a PAP (An \textit{et al.}, 2011).
\par 
In \textit{Aedes aegypti}, tissue melanization requires two cSPs, IMP1 and CLIPB8, and is inhibited by serpin-2 (Zou \textit{et al.}, 2010). 
Hemolymph melanization requires two cSPs, IMD1 and IMD2, and is inhibited by serpin-1 (Zou \textit{et al.}, 2010). 
Additionally, modular serine protease CLSP2 also inhibits hemolymph PPO (Wang \textit{et al.}, 2015).
\par
There is extensive crosstalk between PO cascade and other humoral immune pathways, especially the Toll pathway. 
In \textit{Drosophila melanogaster}, this link is through Spn27A (serpin 27A). 
Toll activation requires depletion of Spn27A from Hemolymph, which activates PO cascade (De Gregorio \textit{et al.}, 2002; Ligoxygakis \textit{et al.}, 2002). 
Spn27A inhibits PO cascade by binding to MP2 (An \textit{et al.}, 2013). 
Additionally, PO cascade can be triggered by fungal receptor GNBP3 in a Toll-independent manner (Matskevich \textit{et al.}, 2010). 
In \textit{Anopheles gambiae}, upregulation of Toll leads to increased melanization, which is partially due to increased expression of TEP1 (Frolet \textit{et al.}, 2006). 
Besides, the Imd/Rel2 pathway triggered by PGRP-LC inhibits melanization, which is partially due to activation of CLIPA2 that inhibits TEP1 (Frolet \textit{et al.}, 2006; Meister \textit{et al.}, 2005). 
In \textit{Aedes aegypti}, Toll pathway activates melanization by controling expression of two cSPs (IMP1 and IMP2) and several PPO genes (Zou \textit{et al.},2010).
\par
PO cascade and Toll can be controlled by common upstream signals. 
In \textit{Tenebrio molitor}, SPE cleaves PPO and cSPH1 (Kan \textit{et al.}, 2008), as well as Spz that activates Toll (Kim \textit{et al.}, 2008). 
In \textit{Manduca sexta}, HP6 activates cleaves Spz, resulting in Toll activation; and activates PPO by cleaving proPAP1 (Ann \textit{et al.}, 2009). 
In \textit{Bombyx mori}, serpin-5 regulates both Toll and PPO (Li \textit{et al.}, 2016). 
In \textit{Drosophila}, PGRP-LE activates Imd pathway and PPO cascade (Takehana \textit{et al.}, 2002; Takehana \textit{et al.}, 2004).
\par \newline \newline
\textbf{Hillyer, 2016, Insect immunology and hematopoiesis.}
The most encompassing physical barrier of insects is the cuticle. 
This chitinous, hydrophobic material forms the exoskeleton, and also lines foregut, hindgut and tracheal system. 
Pathogens enter body through cuticle via wound or enzymatic digestion. 
Ingestion is another routine for pathogen entrance. 
\par
Multiple insect cells and tissues are involved in immunity. 
Hemocytes are the primary immune cells. 
They circulate with hemolymph (circulating hemocytes) or attach to tissues (sessile hemocytes). 
These cells drive cellular and humoral immunity. 
Fat body is composed of loosely associated cells that are rich in lipids and glycogen, lines the integument of hemocoel. 
It functions in energy storage and synthesis of vitellogenin precursors that are required for egg production. 
Fat body also produces antimicrobial peptide. 
Midgut mainly functions in digestion and nutrition absorption. 
It produces nitric oxide synthesis and other lytic effectors killing pathogens. 
Salivary glands are primarily involved in feeding and usually located in the anterior of thorax. 
It is involved in immunity.
\par
Immune responses are initiated by recognition of pathogen-associated molecular patterns (PAMPs) by pattern recognition receptors (PRRs). 
Among PRR families are 
\newline
(1) PGRP: peptidoglycan recognition protein, characterized by peptidoglycan-binding domain; \newline
(2) Ig: immunoglobulin domain proteins; \newline
(3) FREP: fibrinogen-related protein, or fibrinogen domain immunolectin (FBN), contain fibrinogen-like domain; \newline
(4) TEP: thioester-containing protein; \newline
(5) betaGRP: beta-1,3-glucan recognition proteins, or Gram-negative bacterial-binding protein (GNBP); \newline
(6) Galectin: bind specifically to beta-galactoside sugars; \newline
(7) CTL: C-type lectin; \newline
(8) LRR: leucin-rich repeat containing protein; \newline
(9) DSCAM: down syndrome cell adhesion molecule; \newline
(10) Nimrod: include eater and draper in \textit{Drosophila melanogaster}; \newline
(11) ML: MD-2-like protein, or Niemann-Pick type C-2 protein, involved in recognizing lipopolysaccharide; \newline
(12) SR: scavenger receptor, include croquemort and peste in \textit{Drosophila melanogaster}; \newline
(13) Integrin.
\par
Toll pathway functions in both development and immunity. 
In immunity, Toll signaling is effective in combating Gram-positive bacteria, fungi and viruses.
Toll pathway includes
\newline
(1) SPZ: Spatzle/spaetzle, extracellular cytokine; \newline
(2) Toll: or toll-like receptor (TLR). \newline 
(3) MyD88: myeloid differentiation primary response 88; \newline
(4) Tube: or interleukin-1 receptor-associated kinase 4 (IRAK4); \newline
(5) Pelle: orthologous to human interleukin 1 receptor associated kinase 1 (IRAK1); \newline
(6) Dorsal; \newline 
(7) Dif: Dorsal-related immune factor; \newline
(8) Cactus: orthologous to human NF-kappaB inhibitor alpha (NFKBIA).
\par
In Toll signaling, SPZ is activated by cleavage. 
SPZ binds to cellular receptor Toll. 
Toll recruits MyD88, Tube and Pelle. 
Pelle acts as serine/threonine-protein kinase, phosphorylating Cactus. 
Thus, NF-kappaB transcription factor Dorsal and Dif are released from Cactus and relocated to nuclear.
\par
Imd signaling is effective in combating Gram-negative bacteria and viruses. 
Players in Imd signaling include: 
\newline
(1) Imd: immune deficiency, or AGAP004959-PA/-PB in \textit{Anopheles gambiae} str. PEST, or receptor-interacting serine/threonine-protein kinase 1-like, or LOC5572865 in \textit{Aedes aegypti}; \newline
(2) TAK1: transforming growth factor (TGF)-beta activated kinase 1, orthologous to human mitogen-activated protein kinase kinase kinase 7 (MAP3K7); \newline
(3) Tab: TAK1-associated binding protein, or MAP3K7 binding protein; \newline
(4) IKKgamma: inhibitor of NF-kappaB (IkappaB) kinase subunit gamma, or Kenny in \textit{Drosophila melanogaster}, or NF-kappaB essential modulator (NEMO), or optineurin; \newline
(5) IKKbeta: IkappaB kinase subunit beta; \newline
(6) Fadd: fas-associated death domain, or Fas associated via death domain; \newline
(7) Dredd: death-related ced3/Nedd2-like caspase, or caspase 8, orthologous to human caspase 10; 
\newline
(8) Relish: or nuclear factor NF-kappa-B p110 subunit; \newline
(9) Diap: death-associated inhibitor of apoptosis, or inhibitor of apoptosis (Iap); \newline
(10) Effete: or ubiquitin-conjugating enzyme E2-17 kDa; \newline
(11) UEV1A: ubiquitin-conjugating enzyme variant 1A, or ubiquitin conjugating enzyme E2 variant 2; \newline
(12) Bendless: or ubiquitin-conjugating enzyme E2 N, or ubiquitin-conjugating enzyme 13 (UBC13); \newline
(13) Caspar: or fas-associated factor 1 (FAF1); \newline
(14) Hemipterous: or dual specific mitogen activated protein kinase kinase 7 (MAP2K7), or MAP2K7; \newline
(15) Basket: or stress-activated protein kinase JNK; \newline
(16) Jra: Jun-related antigen, or transcription factor AP-1, or transcription factor jun-D-like; \newline
(17) Kayak. 
\par
In Imd signaling, PRRs activate cascade composed of Imd, Fadd and Dredd. 
Dredd activates transcription factor Relish. 
Another line for Relish activation includes Imd, Tab, TAK1, IKKgamma and IKKbeta. 
Both signals require ubiquitination mediated by a protein complex including Diap, Effete, UEV1A, Bendless. 
Caspar inhibits Dredd-mediated activation of relish. 
TAK1 also activates JNK (Jun amino-terminal kinase) signaling, including Hemipterous, Basket, Jra and transcription factor Kayak. 
\par
JAK/STAT signaling functions in development and immunity. 
In immunity, it activates antimicrobial genes like nitric oxide synthase and functions in antibacterial and antiviral responses. 
JAK/STAT signaling includes 
\newline
(1) Unpaired \newline
(2) Domeless. \newline
(3) Hopscotch: orthologous to several human genes including JAK1 (Janus kinase 1) and JAK3 (Janus kinase 3). \newline
(4) Stat: signal-transducer and activator of transcription protein. \newline
(5) Socs: suppressor of cytokine signaling; \newline
(6) Pias: protein inhibitor of activated Stat as E3 SUMO-protein ligase, or suppressor of variegation 2-10 (Su(var)2-10) in \textit{Drosophila melanogaster}. 
\par
In JAK/STAT signaling, extracellular protein Unpaired activates membrane protein Domeless. 
Domeless activates Hospotch and Stat. 
Stat relocates to nuclear, acting as transcription factor. 
Socs and Pias inhibit JAK/STAT signaling. 
\par
Phagocytosis is a rapid progress conducted by hemocytes. 
PRRs that have been shown to be involved in phagocytosis include TEPs, Nimrods, DSCAMs, beta-integrins and PGRPs. 
The intracelular signaling in phagocytosis remains poorly understood. 
In mosquitoes, 
\newline
(1) CED2: cell death abnormality 2; \newline
(2) CED5; \newline
(3) CED6 \newline 
are involved in signaling regulate internalization of bacteria (Moita \textit{et al.}, 2005). 
\par
Melanization is an enzymatic process involved in cuticle hardening, egg chorion tanning, wound healing and immunity and is mainly conducted by hemocytes. 
In immunity, melanization functions in killing bacteria, fungi, protozoa parasites, nematode worms and parasitoid wasps. 
It is manifested as a darkened proteinaceous capsule that surrounds pathogens, and kills pathogens via oxidative damage or starvation. 
Players in melanization include:
\newline
(1) PAH: phenylalanine hydroxylase, or phenylalanine 4-monooxygenase, or Henna in \textit{Drosophila melanogaster}; \newline
(2) PO: phenoloxidase, or phenol oxidase, formed via cleavage of prophenoloxidase (PPO); \newline
(3) DCE: dopachrome conversion enzyme or dopachrome decarboxylase/tautomerase, known as yellow in \textit{Drosophila melanogaster}; \newline
(4) DDC: dopa decarboxylase, aromatic-L-amino-acid decarboxylase (AADC or AAAD), tryptophan decarboxylase or 5-hydroxytryptophan decarboxylase, decarboxylates dopa into dopamine, which is oixidized into dopaminequinone by PO, and further converts into dopaminechrome non-enzymatically, and further into DHI non-enzymatically. \newline 
(5) ModSp: modular serine protease that lacks clip domain but contains other domain for interactions; \newline
(6) cSP: clip domain-containing serine protease, includes \textit{Drosophila melanogaster} 
  snake, easter, serine protease 7 (SP7), serine protease immune response integrator (spirit), persephone, spatzle-processing enzyme (SPE), Gram-positive specific serine protease (grass), melanization protease 1 (MP1), hayan, Ser7, lethal (2) k05911, 
activated by ModSp cleavage and activates PO by cleavage. \newline
(7) serpin: serine protease inhibitors. 
\par
In synthesis of melanine, PAH hydroxylates phenylalanine to tyrosine. 
PO oxidizes tyrosine into dihydroxyphenylalanine (Dopa), and further into dopaquinone. 
Dopaquinone is oxidazed into dopachrome non-enzymatically. 
DCE decarboxylates dopachrome into 5,6-dihyroxyindole (DHI). 
Another way from Dopa to DHI is: 
DDC decarboxylates dopa into dopamine, which is oixidized into dopaminequinone by PO. 
Dopaminequinone is further converted into dopaminechrome non-enzymatically, and further into DHI non-enzymatically. 
Finally, following PO-meidated DHI oxidation, indole-5,6-quinones polymerize and give rise to heteropolymer eumelanin. 
PO activity is controlled by ModSP, cSP and Serpin.
\par
Encapsulation is a cellular immune response against pathogens that are too large to be phagocytosed. 
In encapsulation, hemocytes attach to form a capsule surrounding pathogens. 
The capsule may be melanized. 
In Lepidoptera, hemocyte adhesion is dependent on binding of integrin to specific sites defined by Arg-Gly-Asp (RGD) sequence. 
\par
Nodulation is an immune response in which hemocyte adhere to large aggregrates of bacteria and form layers, usually followed by melanization. 
Underlying molecular mechanism of nodulation remains poorly understood, but it relies on eicosanoid-based signaling and extracellular matrix-like protein Noduler. 
\par
Lysis of pathogens is resulted from disruption of cellular membrane by immune effectors including 
\newline
(1) AMP: antimicrobial peptide, small secreted peptide including apisimin, attacin, cecropin, defensin, diptericin, drosocin, drosomycin, gambicin, gloverin, holitricin, jelleine, lebocin, melittin, metchnikowin, moricin, persulcatusin, ponericin, pyrrhocoricin, sapecin; \newline
(2) Lysozymes: or muramidase, or N-acetylmuramide glycanhydrolase, hydrolyze beta-1,4-glycosidic linkage between N-acetylumuramic and N-acetylglucosamine of peptidoglycan; \newline
(3) Transferrin: binds to Fe; \newline
(4) Chitinase: degrades chitin and is involved in antifungal responses.
\par
Reactive species are effect in lysis. 
Synthesis of reactive species include 
\newline
(1) DUOX: dual oxidase, generates hydrogen peroxide; \newline
(2) NOX: NADPH oxidase, generates hydrogen peroxide; \newline
(2) NOS: nitric oxide synthase, generates nitric oxide; \newline
(3) SOD: superoxide dismutase, catalyzes the dismutation (or partitioning) of the superoxide radical into ordinary molecular oxygen and hydrogen peroxide; \newline
(4) peroxidase: also peroxide reductase, peroxiredoxin, break up peroxides.  
\par
In RNA interference (RNAi) pathways, small RNA (sRNA) associates with Argonaute protein, forming RNA induced silencing complex (RISC). 
RISC recognizes targets by complementary bases, and silences targets in an Argonaute-mediated manner. 
RNAi functions in antiviral responses, gene expression regulation and anti-transponson responses. 
In insects, there are three RNAi pathways: micro-RNA (miRNA), small-interfering-RNA (siRNA) and piwi-interacting-RNA (piRNA). 
\par
miRNA pathway is mainly involved in gene expression regulation. 
Players in miRNA pathway include: 
\newline
(1) Drosha; \newline
(2) Pasha: partner of Dosha, or microprocessor complex subunit DGCR8. \newline
(3) Dicer 1: endoribonuclease; \newline
(4) Loquacious: or interferon-inducible doube-stranded RNA-dependent protein kinase activator A homolog, or TARBP2. \newline
(5) Argonaute 1. \newline
miRNA originates from nuclear genome, and is processed by nuclear protein Dorsha and Pasha. 
Matured miRNA relocates to cytoplasm, and is further processed by Dicer 1 and Loquacious. 
Then fully-matured miRNA is loaded to Argonaute 1.
\par 
siRNA pathway is involved in defenses against viral dsRNA and transposonal elements. 
\newline
(6) Dicer 2; \newline
(7) R2D2: or double-stranded RNA-binding protein Staufen homolog; \newline
(8) Argonaute 2. \newline
Viral dsRNA is processed by Dicer 2 and R2D2, forming siRNA. 
siRNA is loaded into Argonaute 2. 
In anti-transposonal elements, dsRNA is processed by Dicer 2 and Loquacious. 
\par
piRNA pathway is involved in defenses against transposonal element in germline. 
\newline
(9) Zucchini: or mitochondrial cardiolipin hydrolase; \newline
(10) Piwi: P-element induced wimpy testis, or Argonaute 3, or Aubergine, or Piwi-like protein Siwi; \newline
Transposon transcripts is processed by Zucchini, forming piRNA. 
piRNA is loaded into Piwi. 
\par
Autophagy is a process of degradation of intracellular materials, and is involved in elimation of intracellular bacteria and viruses. 
In \textit{Drosophila}, autophagy defenses against vesicular stomatitis virus and Rift Valley fever virus, but enhances infection of Sindbis virus. 
Major players in autophagy include: 
\newline
(1) PI3K: phosphatidylinositol 3-kinase, or phosphoinositide 3-kinase; \newline
(2) AKT: or RAC serine/threonine-protein kinase; \newline
(3) TOR: target of rapamycin, protein kinase. \newline
(4) Atg1: autophagy-related (Atg) 1, or unc-51 like autophagy activating kinase (ULK), or unc-51, a serine/threonine protein kinase; \newline
(5) Atg13: serine/threonine protein kinase regulatory subunit; \newline
(6) Atg14: or Beclin 1-associated autophagy-related key regulator; \newline
(7) Vps15: vacuolar protein sorting (Vps) 15, or phosphoinositide 3-kinase regulatory subunit 4; \newline
(8) Vps34: phosphatidylinositol 3-kinase 59F, or phosphatidylinositol 3-kinase catalytic subunit type 3; \newline
(9) Atg5; \newline
(10) Atg12; \newline
(11) Atg8: or gamma-aminobutyric acid receptor-associated protein (GABARAP). 
\par
In immunity, autophagy initiates with PI3K-AKT signaling, inactivating TOR. 
TOR inactivation activates protein complex containing Atg1 and Atg13, which leads to nucleation of autophagosomal membrane via a complex containing Atg14, Vps15 and Vps34. 
Then autophagosome is elongated, dependent on Atg5, Atg12 and Atg8.
\par
Apoptosis is a form of programmed cell death that often functions in antiviral responses. 
Key players include: 
\newline
(1) Dronc: death regulator Nedd2-like caspase, or Nedd2-like caspase (Nc); \newline
(2) Dark: death-associated APAF1-related killer, or apoptotic protease-activating factor 1 (APAF1); \newline
(3) Drice: death related ICE-like caspase; \newline
(4) DCP1: death caspase-1. 
\par
Dronc and Dark form a protein comples, and Dronc activates downstream caspase including Drice and DCP1.
\par \newline \newline
\textbf{Gerardo\textit{et al.}, 2020, Evolution of animal immunity in the light of beneficial symbioses.}
There are three key ways for host-symbiont interactions. 
First, host immunity can play a role in regulation of symbionts. 
Second, symbionts can protect host against pathogens. 
Third, symbionts can influence the maturation of host immune system. 
\par
Host can regulate symbiont populations. 
For example, cereal weevil requires endosymbionts for exoskeleton development, after which symbionts are eliminated by apoptosis of bacteriocytes. 
Bean bug \textit{Riptortus pedestris} up-regulates immunity and digests bacteriocytes before moulting, reducing symbiont populations as moulting is energy-costing and leaves bean bug vulnerable to infections and injuries. 
\par
The influence of host immunity on symbiosis likely dependent on symbiont transmission mode. 
In horizontal transmission, host select proper symbionts from pools of microorganisms. 
In vertical transmission, symbionts are transmitted vertically from parents (often mothers) to offsprings. 
\par
For vertebrate hosts, horizontal transmission is the major mode. 
Vertebrates often harbour a large and diversified community of microbes. 
By leveraging innate and adaptive immunity, vertebrates can mount rapid and robust responses to large numbers of microorganisms. 
In mice, immune system discriminates between pathogens and symbionts, and segregates symbionts to proper host tissues. 
In mice gut, epithelium tissue is protected from lumen by a mucus layer. 
Both pathogens and symbionts can enter gut lumen. 
Pathogens are segregated from mucus layer by immune responses. 
Symbionts enter mucus layer and are segregates from epithelium by antimicrobial peptides and immunoglobulins. 
\par
Invertebrates likely to regulate horizontally-transmitted symbiosis by compartmentalized innate immune responses. 
By compartmentalization, hosts can invest the most energy into screening microorganisms and mounting immunity in regions exposed to a wide array of microorganisms, while reduce immune investment elsewhere. 
For example, Hawaiian bobtail squid \textit{Euprymna scolopes} screens large quantities of microbes by defenses including physical barrier, morphological changes and innate immunity. 
Thus, squid limits colonization in light organ to bacteria with specific characteristics, including symbiotic molecular patterns, biofilm formation, bioluminescence and nitric oxide resistance. 
In this way, squid limits colonization of specific strains of \textit{Vibrio fischeri} in light organ. 
Fruit fly \textit{Drosophila melanogaster} uses physical barrier, morphological changes and compartmentalized immune expressions to eliminate pathogens and to limit few symbionts in gut microbiome. 
\par
In vertical transmission, hosts pass few symbionts directly to offsprings in ways including providing symbiont-enclosed capsules, smearing eggs with symbionts, and symbiotic infection of embryo. 
Passaged symbionts undergo population bottlenecks and have little chance for getting virulance factors via horizontal gene transfer with environmental microbes. 
In many cases, vertical transmission is coupled with sequestration of symbionts into specialized cells. 
Sequestration allows host to limit symbiont populations and reduce horizontal gene transfer with fewer investment. 
For example, in cereal weevil, antimicrobial peptides are not expressed in bacteriocytes except ColA, whose knock-out leads to symbiont over-proliferation and escape to other tissues. 
\par
The evolution of immunity in symbiont regulation is likely dependent on transmission mode. 
Vertical transmission has evolved multiple times among invertebrates. 
Compared with horizontal transmission, it may allow reduced investment in symbiont regulation because 
(1) horizontal transmission requires screening for symbionts and discarding pathogens from environmental pools; 
(1) fitness of vertically-transmitted symbionts is dependent on host fitness, and therefore, they are less likely to exploit hosts; 
(2) vertical transmission limits chance for horizontal gene transfer from environments, reducing possibility that symbionts acquire virulance factors; 
(3) vertically-transmitted symbionts are often sequestered into host cells, allowing tightly control via nutrition availability. 
Therefore, it is possible that selection pressure on immunity is weaker in hosts with vertically-transmitted symbionts than hosts with horizontally-transmitted symbionts. 
However, vertical transmission provides less flexibility in the face of changing environment conditions, which can be especially important for long-living hosts. 
Adaptive immunity, in turn, is assumed to have evolved to affording regulation of a diversified symbiont communities, as complex symbiont communities are often found in vertebrates. 
However, adaptive immunity only evolved independently twice in jawed vertebrates and jawless vertebrates, indicating the co-occurrence of complex microbiomes and adaptive immunity is resulted from common ancestors instead of convergent evolution under selection. 
\par \newline \newline
\textbf{Garcia \textit{et al.}, 2014, The symbiont side of symbiosis: do microbes really benefit?} \newline
It has been presumed that microbial symbionts receive host-derived nutrients or a competition-free environment with reduced predation, but there have been few empirical tests, or even critical assessments, of these assumptions. 
Evaluation of these hypotheses based on available evidence indicates reduced competition and predation are not universal benefits for symbionts. 
Some symbionts do receive nutrients from their host, but this has not always been linked to a corresponding increase in symbiont fitness.
\par
\textbf{Viljakainen, 2015, Evolutionary genetics of insect innate immunity.} \newline
Toll and Imd signaling pathways are well conserved across insects. 
Antimicrobial peptides (AMPs) are the most labile component of insect immunity showing rapid gene birth-death dynamics and lineage-specific gene families. 
Immune genes and especially recognition genes are frequently targets of positive selection driven by host-pathogen arms races. 
Homology-based annotation is useful but to some extent restricted approach to find immune-related genes in a newly sequenced genome. 
Novel immune genes have been found in many insects and should be looked for in future research.
\par
\textbf{Boehm, 2012 Evolution of vertebrate immunity.} \newline
Could it be possible then that an immune system employing structurally diversified antigen receptors facilitated increased species-richness in autochthonous microbial communities, for example, in the intestine? 
The selective advantage of increasing antigen receptor diversity with respect to the species-richness of microbiomes is illustrated by the role of secreted antibodies, such as IgA in mammals, in the maintenance of microbial homeostasis on mucosal surfaces; defective structural diversification of secreted antibodies is associated with dysbiosis, which is characterized by generally lower species diversity and an ‘unhealthy’ composition of the microbiome. 
Autoimmunity can be a price for the evolution of adaptive immunity. 
\par
\textbf{McFall-Ngai, 2007, Care for the community.} \newline
A memory-based immune system may have evolved in vertebrates because of the need to recognize and manage complex communities of beneficial microbes. 
Invertebrates are no less challenged by the microbial world than vertebrates, nor are they less able to remain healthy by entirely relying on innate immunity. 
Invertebrates often harbor much less diversified symbiont communities compared with vertebrates. 
There are three possible strategies for management of symbionts in invertebrates: 
maintain symbionts intracellularly; 
build physical barrier between host tissue and symbionts; 
express a high number of specific recognition components of immate immunity. 
\par
\textbf{Hoang & King, 2022, Symbiont-mediated immune priming in animals through an evolutionary lens.} \newline
While research on symbiont-mediated immune priming (SMIP) has focused on ecological impacts and agriculturally important organisms, the evolutionary implications of SMIP are less clear. 
Here, we review recent advances made in elucidating the ecological and molecular mechanisms by which SMIP occurs. 
We draw on current works to discuss the potential for this phenomenon to drive host, parasite, and symbiont evolution. 
We also suggest approaches that can be used to address questions regarding the impact of immune priming on host-microbe dynamics and population structures. 
Finally, due to the transient nature of some symbionts involved in SMIP, we discuss what it means to be a protective symbiont from ecological and evolutionary perspectives and how such interactions can affect long-term persistence of the symbiosis. 
\par
\textbf{Sharp & Hoster, 2022, Host control and the evolution of cooperation in host microbiomes.} \newline
It is often suggested that the mutual benefits of host-microbe relationships can alone explain cooperative evolution. 
Here, we evaluate this hypothesis with evolutionary modelling. 
Our model predicts that mutual benefits are insufficient to drive cooperation in systems like the human microbiome, because of competition between symbionts. 
However, cooperation can emerge if hosts can exert control over symbionts, so long as there are constraints that limit symbiont counter evolution. 
We test our model with genomic data of two bacterial traits monitored by animal immune systems. 
In both cases, bacteria have evolved as predicted under host control, tending to lose flagella and maintain butyrate production when host-associated. 
Moreover, an analysis of bacteria that retain flagella supports the evolution of host control, via toll-like receptor 5, which limits symbiont counter evolution. 
Our work puts host control mechanisms, including the immune system, at the centre of microbiome evolution. 
\par
\textbf{Costello \textit{et al.}, 2012, The application of ecological theory toward an understanding of the human microbiome.} \newline
Review of three core scenarios of human microbiome assembly: 
development in infants, representing assembly in previously unoccupied habitats; 
recovery from antibiotics, representing assembly after disturbance; 
and invasion by pathogens, representing assembly in the context of invasive species. 
\par
\textbf{Hansen & Moran, 2013, The impact of microbial symbionts on host plant utilization by herbivorous insects.} \newline
Herbivory, defined as feeding on live plant tissues, is characteristic of highly successful and diverse groups of insects and represents an evolutionarily derived mode of feeding. Plants present various nutritional and defensive barriers against herbivory; nevertheless, insects have evolved a diverse array of mechanisms that enable them to feed and develop on live plant tissues. For decades, it has been suggested that insect-associated microbes may facilitate host plant use, and new molecular methodologies offer the possibility to elucidate such roles. Based on genomic data, specialized feeding on phloem and xylem sap is highly dependent on nutrient provisioning by intracellular symbionts, as exemplified by Buchnera in aphids, although it is unclear whether such symbionts play a substantive role in host plant specificity of their hosts. Microorganisms present in the gut or outside the insect body could provide more functions including digestion of plant polymers and detoxification of plant-produced toxins. However, the extent of such contributions to insect herbivory remains unclear. We propose that the potential functions of microbial symbionts in facilitating or restricting the use of host plants are constrained by their location (intracellular, gut or environmental), and by the fidelity of their associations with insect host lineages. Studies in the next decade, using molecular methods from environmental microbiology and genomics, will provide a more comprehensive picture of the role of microbial symbionts in insect herbivory.
\par
\textbf{Webster, 2014, Cooperation, communication, and co-evolution: grand challenges in microbial symbiosis research.} \newline
Horizontal transmission of symbionts often leads to selection on symbiont functions instead of taxa.
\par
\textbf{Shapira, 2016, Gut Microbiotas and Host Evolution: Scaling Up Symbiosis.} \newline
Gut microbiota impacts on animal evolution on different aspects. 
First, gut microbiota could boosts utilization of plant materials. 
For example, primary \textit{Buchnera} endosymbionts of pea aphid enables hosts feeding on plant-sap by providing essential amino acids and vitamins lack from host diet. 
Gut microbiota of mammalian herbivores and some birds contribute to degradation of plants, and some of these microorganisms show evidence for host adaption. 
Second, gut microbiota could boost adaption to new niches by protecting host against pathogens/toxins. 
Third, gut microbiota could impact on host growth and development. 
When gut microbiota disturbed by antibiotics, impaired growth has been observed in termites, \textit{Daphnia}, cotton strainers and red firebugs. 
\textit{Acetobacter pomorum}, a primary gut microorganism of \textit{Drosophila}, impacts host development by modulating insulin signaling. 
Germ-free mice demonstrate diverse impairments in immune maturation, including smaller mucosal Payer patches, fewer lymphocytes, and decreased levels of antimicrobial peptides and IgAs, which cannot be reversed by tranplantation of human gut microbiota (similar in phylum level and gene content) or rat gut microbiota. 
Forth, gut microbiota could impact mate choice. 
Within one generation, \textit{Drosophila} show matting preference towards flies fed on same food, while antibiotics break this preference. 
Fifth, gut microbiota contributes to hybrid incompatibility. 
F2 hybrid males, generated by crossing two closely related \textit{Nasonia} species, were mostly nonviable, but survived when reared under germ-free conditions. 
\par
\textbf{Brucker \textit{et al}., Speciation by symbiosis.} \newline
There are three general observations linking symbiosis with speciation: 
First, microbial symbionts are universal in eukaryotes. 
Second, hosts typically exhibit strong specificity for microbial symbionts and their functions, often as a result of diet, geography and/or phylogenetic history. 
Third, host immune genes are rapidly evolving in response to microbial symbionts and represent a gene family frequently involved in hybrid incompatibilities. 
\par
Symbiosis could mediate pre-mating isolation by mediating parthenogenesis, impacting on mating preference or enabling adaption to new niches. 
Post-mating isolation can be resulted from incompatibility between host genes and/or symbionts. 
\textit{Wolbachia} endosymbionts can mediate cytoplasmic incompatibility. 
\par
Hybridization can lead to breakdown of co-adapted genes, resulting in reduced fitness. 
As the immune system is subject to frequent bouts of positive selection and rapid evolution to combat a pathogenic microbiota and maintain a beneficial one, hybridization could spur negative epistasis between immunity genes from different species and cause an inferior level of resistance than either parental species. This phenomenon is generally referred to as the hybrid susceptibility hypothesis, in which hybrids are more susceptible to infection by pathogens than are non-hybrids. 
For example, \textit{Wolbachia} endosymbionts of testes fly become severe pathogen in hybrid F1 males and causing sterility. 
\par
Epistasis of immune genes in hybrid can cause autoimmunity even in absence of pathogens. 
\par
\textbf{Nyholm & Graf, 2012, Knowing your friends: invertebrate innate immunity fosters beneficial bacterial symbioses.} \newline
MAMP: microbial-associated molecular pattern \newline
DAMP: damage-associated molecular pattern \newline
We propose three possible mechanisms by which hosts differentiate between symbionts and pathogens, and these mechanisms would not be mutually exclusive. 
First, reciprocal signalling between the partners (mediated by MAMP-PRR interactions) could induce an altered immune response that would lead to host tolerance of the symbiont. 
Second, symbionts might be sequestered in specialized tissues, organs or cells, where the molecular dialogue leading to stasis could be regulated locally. 
Last, if the symbiont or other microorganism (such as an invasive pathogen) infiltrates other areas of the host and/or causes tissue damage, signalling events mediated through DAMPS and MAMPs could lead to heightened immune responses and clearance by the host. A lack of such damage signals could also promote maintenance of the symbiont.
\textbf{Zaidman-Rémy \textit{et al.}, 2018, What can a weevil teach a fly, and reciprocally? Interaction of host immune systems with endosymbionts in \textit{Glossina} and \textit{Sitophilus}} \newline
\par
\textbf{Wu \textit{et al.}, 2018, Insect antimicrobial peptides, a mini review.} \newline
\par
\textbf{Skidmore & Hansen, 2017, The evolutionary development of plant-feeding insects and their nutritional endosymbionts.} \newline
\par
\textbf{Vavre & Kremer, 2014, Microbial impacts on insect evolutionary diversification: from patterns to mechanisms.} \newline
\par
\textbf{Eleftherianos \textit{et al.}, 2013, Endosymbiotic bacteria in insects: guardians of the immune system?}
\par
\textbf{Soucy \textit{et al.}, 2015, Horizontal gene transfer: building the web of life.}
\par
\textbf{Thomas \textit{et al.} 2020 Gene content evolution in the arthropods.} \newline
\par
\textbf{Bonning & Saleh \textit{et al.}, 2021, The Interplay Between Viruses and RNAi Pathways in Insects.} \newline
\par
\textbf{Behura & Severson, 2013, Codon usage bias: causative factors,quantification methods and genome-widepatterns: with emphasis on insect genomes.} \newline
\par
\textbf{Booker \textit{et al.}, 2017, Detecting positive selection in the genome.} \newline
\par
\textbf{Govorushko, 2018, Economic and ecological importance of termites: A global review.} \newline
\par
\textbf{Elser \textit{et al.}, 2011, Stoichiogenomics: the evolutionary ecology of macromolecular elemental composition.} \newline
\par
\textbf{Winterbach \textit{et al.}, 2013, Topology of molecular interaction networks.} \newline
\par
\textbf{Cheetham \textit{et al.}, 2019, Overcoming challenges and dogmas to understand the functions of pseudogenes.} \newline
\par
\textbf{Clement \textit{et al.}, 2023, Sidestepping Darwin: horizontal gene transfer from plants to insects.} \newline
\par
\textbf{Wicker \textit{et al.}, 2007, A unified classification system for eukaryotic transposable elements.} \newline
In practice, TE families are usually defined using the 80–80–80 rule, which specifies that insertions are members of the same family if they are longer than 80 bp and share at least 80\% sequence identity over 80\% of their length
\par
\textbf{Wells \textit{et al.}, 2020, A Field Guide to Eukaryotic Transposable Elements.} \newline
\par
\textbf{Gupta \textit{et al.}, 2019, Genetic Basis of Adaptation and Maladaptation via Balancing Selection.} \newline
\par
\textbf{Orr, 2005, The genetic theory of adaptation: a brief history.} \newline
\par


\end{sloppypar}
\end{document}