\documentclass[11pt]{article}

% \usepackage[UTF8]{ctex} % for Chinese 

\usepackage{setspace}
\usepackage[colorlinks,linkcolor=blue,anchorcolor=red,citecolor=black]{hyperref}
\usepackage{lineno}
\usepackage{booktabs}
\usepackage{graphicx}
\usepackage{float}
\usepackage{floatrow}
\usepackage{subfigure}
\usepackage{caption}
\usepackage{subcaption}
\usepackage{geometry}
\usepackage{multirow}
\usepackage{longtable}
\usepackage{lscape}
\usepackage{booktabs}
\usepackage{natbibspacing}
\usepackage[toc,page]{appendix}
\usepackage{makecell}
\usepackage{amsfonts}
 \usepackage{amsmath}

\usepackage[backend=bibtex,style=authoryear,sorting=nyt,maxnames=1]{biblatex}
\bibliography{} % Reference bib

\title{Abstracts: Diptera}
\author{}
\date{}

\linespread{1.5}
\geometry{left=2cm,right=2cm,top=2cm,bottom=2cm}

\setlength\bibitemsep{0pt}

\begin{document}
\begin{sloppypar}
  \maketitle

  \linenumbers
\textbf{Lee \textit{et al.}, 2017, Microbiota, gut physiology, and insect immunity.}
Adult \textit{Drosophila} contains three distinct domains: foregut, midgut and hindgut. 
Foregut is at the anteriormost region and is originated from ectoderm. 
It includes pharynx and oesophagus for passage of ingested food and crop for food storage. 
Midgut is the region from cardia to midgut-hindgut junction where Malpighian tubules are attached. 
It is originated from endoderm and functions for food digestion and nutrient absorption. 
Cardia serves as a valve for food passage regulation. 
Hindgut is ectoderm-derived, extending to the rectum, and is responsible for absorption of water and ions.
\par
Midgut is a single layer of epithelium and a visceral of muscle layer. 
The midgut epithelium contains four cell types: enterocytes (ECs), enteroendocrine cells (EECs), ISCs and enteroblasts (EBs). 
ECs are large polypoid cells secreting digestive enzymes and absorbing nutrients. 
They are the most abundant in midgut epithelium. 
EECs secrets hormones. 
ISCs are dividing progenitor cells. 
EBs are restricted progenitor cells produced by ISCs differentiation, and further differentiates into ECs or EECs. 
The lumenal side of midgut is covered by peritrophic matrix, a chitin polymer layer. 
A mucus layer fills between the epithelium and peritrophic matrix. 
The peritrophic matrix, mucus layer and epithelium act as physical barrier for immunity. 
\par
The midgut is further regionalized into anterior region, copper cell region (CCR) and posterior region. 
Anterior region functions for food breakdown by secreting enzymes. 
CCR is for further digestion with its low pH. 
Posterior region is for absorption of nutrients. 
When radius of gut is measured, midgut can be divided into six regions (R0-R5). 
R0 is cardia, R1-R2 is anterior region, R3 is CCR, and R4-R5 is posterior region. 
\par
The primary immune systems in midgut of \textit{Drosophila} are DUOX pathway (Ha \textit{et al.}, 2009) and IMD pathway (Tzou \textit{et al.}, 2000). 
Toll pathway is dispensable in gut epithelium.
\par
IMD pathway includes: 
(1) recognition of bacterial peptidoglycans; 
(2) intracellular cascade activating Relish, a member of NF-$\kappa$B transcription factor family; 
(3) expression of antimicrobial peptides; 
(4) negative regulation of IMD pathway. 
\par
IMD pathway begins with PGRPs that recognize peptidoglycans. 
PGRP-LC is a transmembrane receptor recognizes DAP-type peptidoglycan characterized by meso-diaminopimelic acid in peptide chain (Choe \textit{et al.}, 2002; Gottar \textit{et al.}, 2002; Ramet \textit{et al.}, 2002). 
PGRP-LE resides in cytoplasm and recognizes DAP-type peptidoglycan, and thus activates IMD pathway (Bosco-Drayon \textit{et al.}, 2012). 
\par
After binding to peptidoglycan, PGRP-LC recruits IMD, Dredd and FADD to form a signaling complex (Georgel \textit{et al.}, 2001; Naitza \textit{et al.}, 2002). 
Dreed cleaves IMD and Relish for their activation. 
Ultimately, N-terminal cleaved Relish translocates into nucleus for target gene expression (Khush \textit{et al.}, 2001). 
\par
Ubquitination and phosphorylation is required for IMD activation. 
Dredd activation requires K63-ubiquitination by IAP2, a E3-ligase (Meinander \textit{et al.}, 2012). 
Dredd cleaves IMD, enabling its binding with IAP2. 
IAP2 generates K63-polyubiquitination, which is required for recruitment of TAK1/TAB2 complex (Vidal \textit{et al.}, 2001). 
TAB2 binds to K63-polyubiquitination of IMD, and TAK1 is a MAPKKK kinase for activation of IKK complex. 
IKK complex is composed of IRD5 (catalytic activity) and Kenny (regulatory subunit). 
Activated IKK complex phosphorylates Relish on multiple sites, which activates its transcription factor activity (Erturk-Hasdemir \textit{et al.}, 2009; Silverman \textit{et al.}, 2000). 
Relish induces expression of genes involved in non-self recognition, signaling pathways, proteolysis and antimicrobial peptides. 
\par
IMD pathway is inhibited by several mechanisms. 
PGRP amidase (PGRP-LB, -SC1a, -SC1b, -SC2) degrades peptidoglycan and thus inhibits IMD pathway (Bischoff \textit{et al.}, 2006; Guo \textit{et al.}, 2014; Paredes \textit{et al.}, 2011). 
PIRK is a transcriptional target of IMD pathway and inhibits IMD pathway (Aggarwal \textit{et al.}, 2008; Kleino \textit{et al.}, 2008; Lhocine \textit{et al.}, 2008). 
It may disrupt IMD signaling as it interacts with PGRP-LC, PGRP-LE and IMD (Aggarwal \textit{et al.}, 2008). 
Other inhibitors of IMD pathway including 
Dnr1 for Dredd inhibition (Foley and O'Farrell, 2004; Guntermann \textit{et al.}, 2009); 
Caspar for Dredd-dependent Relish cleavage inhibition (Kim \textit{et al.}, 2006); 
Trabid targeting TAK1 (Fernando \textit{et al.}, 2014); 
CYLD, a deubiquitinating enzyme (Tsichritzis \textit{et al.}, 2007); 
SkpA, a subunit of SCF-E3 ubquitin ligase targeting Relish (Khush \textit{et al.}, 2002); 
and transcription inhibitors such as caudal (Ryu \textit{et al.}, 2008) and Nubbin (Dantoft \textit{et al.}, 2013).
\par
IMD pathway is inhibited in gut, and its activation leads to pathologic symptoms including mocrobiota dysbiosis and dysplasia (Bosco-Drayon \textit{et al.}, 2012; Guo \textit{et al.}, 2014; Lhocine \textit{et al.}, 2008; Ryu \textit{et al.}, 2008). 
For instance, caudal is gut-specific inhibitor of IMD pathway and its knockdown causes gut cell apoptosis, decreased survival rate and change of microbiome (Ryu \textit{et al.}, 2008). 
Knockdown of PGRP-SC2, an inhibitor of IMD pathway, also leads to mocrobiota dysbiosis and dysplasia (Guo \textit{et al.}, 2014). 
\par
DUOX is a member of nicotinamide adenine dinucleotide phosphate oxidase (NOX) family and is responsibe for bacterial-induced reactive oxygen species (ROS) generation. 
ROS plays an important role in gut immunity, and is degraded by secretory immune-related catalase (IRC) (Ha \textit{et al.}, 2005).
\par
NOX/DUOX family proteins share a catalytic gp91phox domain, and DUOX contains an additional peroxidase homology domain (PHD). 
Generally, NOX generates superoxide anion in extracellular space by electron transfer from NADPH in cytoplasm to oxygen across the membrane. 
Superoxide anion is subsequently converted to H$_2$O$_2$, which can be further converted to HClO by myeloperoxidase. 
\par
In \textit{Drosophila}, there are one NOX and one DUOX (Ha \textit{et al.}, 2005). 
The produce of HClO is DUOX-dependent (Ha \textit{et al.}, 2005). 
DUOX is essential for gut immunity (Ha \textit{et al}, 2005; Ha \textit{et al}, 2009). 
It is activated only with transient microorganisms, but not with commensals. 
Uracil acts as a ligand for DUOX activation (Lee \textit{et al.}, 2013). 
It is secreted by several pathogens, but not by commensals.
\par
DUOX enzymatic activity requires calcium released from ER, and thus is regulated by PLC$\beta$ and G$\alpha$q (Ha \textit{et al.}, 2005; Ha \textit{et al.}, 2009). 
At downstream of G$\alpha$q, PLC$\beta$ is required for generation of 1,4,5-triphosphate, which is recognized by corresponding receptor and enables release of calcium from ER. 
Transcription of DUOX is up-regulated by (Ha \textit{et al.}, 2009) 
(1) peptidoglycan-dependent cascade composed of PGRP-LC, IMD, MEKK1, MKK3, p38 and ATF2; 
(2) uracil-dependent cascade including PLC$\beta$, MEKK1, MKK3, p38 and ATF2. 
ATF2 is a transcription factor. 
\par
Negative regulation of DUOX transcription is mediated by inhibition of peptidoglycan-dependent p38 activation, which requires PLC$\beta$, calcineurin B and MAP kinase phosphatase 3 (MKP3) (Ha \textit{et al.}, 2009). 
It indicates that activation of DUOX requires certain amount of peptidoglycan, and therefore, DUOX remains inactivated under commensals.
\par
Mid gut is dynamic. 
Adult \textit{Drosophila} intestinal epithelium is renewed every 1 week (Micchelli and Perrimon, 2006). 
This gut renewal is dependent on asymmetric division of ISC. 
The two daughter cells of ISC division, one becomes self-renewed ISC and another one differentiates into EB, which further differentiates into EC or EEC. 
The fate decision of ISCs after division is dependent on the antagonism of Delta-Notch signaling and BMP signaling (Tian and Jiang \textit{et al.}, 2014). 
Delta-Notch signaling also plays an important role in differentiation into EC/EEC (Ohlstein and Spradling, 2007; Perdigoto \textit{et al.}, 2011). 
\par
Proliferation of ISC is under tightly control to maintain gut homeostasis. 
Low rate of ISC proliferation leading to reduced replacement of damaged cells, destroying gut integrity and leads to originasm death. 
High rate of ISC proliferation leads to accumulation of unwanted cells, causing pathology (Biteau \textit{et al.}, 2008). 
Several signaling pathways are involved in ISC proliferation activation, including JAK/STAT, EGFR, Hippo, JNK and Wingless (Biteau \textit{et al.}, 2008; Cordero \textit{et al.}, 2012; Jiang \textit{et al.}, 2011; Karpowicz \textit{et al.}, 2010; Lee \textit{et al.}, 2009). 
Myc may be a common downstream of JAK/STAT, EGFR, Hippo and Wingless (Ren \textit{et al.}, 2013). 
Insulin receptor signaling in ISC is required for ISC proliferation (Amcheslavsky \textit{et al.}, 2009). 
\par
Ligands from nearby injured cells are responsible for activation of ISC proliferation. 
In stressed ECs, JAK/STAT ligand Upd3 and EGFR ligand Keren are expressed under control of JNK and Hippo signaling (Jiang \textit{et al.}, 2009; Ren \textit{et al.}, 2010; Jiang \textit{et al.}, 2011). 
EGFR ligands vein and spitz are produced by visceral muscles and progenitors respectively (Jiang \textit{et al.}, 2011). 
Stressed EBs produce Upd2 through activation of Hedgehog pathway (Tian \textit{et al.}, 2015), and Wingless ligand under control of JNK signaling (Cordero \textit{et al.}, 2012). 
Hippo signaling in ISC is controled by intracellular interactions of two cadherins, Fat in ISC and Dachsous (DS) in EC (Karpowicz \textit{et al.}, 2010). 
\par
IMD pathway may regulate ISC proliferation via controling number of gut bacteria (Buchon \textit{et al.}, 2009). 
ROS induced by DUOX signaling also accerlates ISC proliferation. 
ROS may induce ISC proliferation by tissue damaging (Buchon \textit{et al.}, 2009; Karpowicz \textit{et al.}, 2010; Ren \textit{et al.}, 2010; Ren \textit{et al.}, 2013; Shaw \textit{et al.}, 2010; Staley and Irvine, 2010). 
ROS may also activates ISC proliferation directly by targeting redox-sensitive components of signalings. 
ROS activates JAK/STAT by redox-sensitive tyrosine phosphatase (Liu \textit{et al.}, 2004), JNK by thioredoxin (Junn \textit{et al.}, 2000) and Wnt by nucleoredoxin (Funato \textit{et al.}, 2006).
\par
\textbf{2. Husink \textit{et al.}, 2020, Insect-symbiont gene expression in the midgut bacteriocytes of a blood-sucking parasite.}
Sheep ked \textit{Melophagus ovinus} is a species of wingless, blood-sucking insect that is permanently associated with vertebrate host and transmitted via host interactions. 
Its primary bacterial symbiont \textit{Arsenophonus melophagi} lives intracellularly in bacteriocytes that assemble into special structure (bacteriome) in midgut. 
\par
Sheep ked interacts with its symbiont nutritionally. 
Transcription analysis shows symbiont high expression of a pathway that converts proline to L-glutamate through the PutA enzyme (EC 1.5.5.2/1.2.1.88) and its subsequent conversion to D -glutamate by the MurI enzyme (EC 5.1.1.3). 
Proline is almost always the most common amino acid in insect hemolymph (Arrese and Soulages 2010). 
In insects, proline is generally reserved for energy-demanding activity such as flight. 
In wingless sheep ked, proline storage might be used for bacterial symbiont for energy metabolism and peptidoglycan synthesis. 
However, the symbiont expression of B-vitamin synthesis is low, except that of lipoic acid. 
It is possible that B vitamins are not essential for sheep ked, or are only needed during particular life stage. 
It is also possible that sheep ked does not rely on symbiont for B vitamins.
\par
Symbiont \textit{Arsenophonus} might be dependent on host for metal ions. 
High expression of zinc transporters in bacteriocytes and symbionts, together with low expression of zinc protease in bacteriocytes, indicate that symbiont \textit{Arsenophonus} is demanding on zinc. 
Symbiont zinc-dependent proteins might be protease, or a putative metalo-beta-lactamase. 
Beta-lactamases are enzymes that provide bacteria with resistance to beta-lactam antibiotics such as penicillin, ampicillin. 
Metalo-beta-lactamases in particular are well-known for their resistance to a broad spectrum of beta-lactam antibiotics and beta-lactamase inhibitors (Bradford 2001; Drawz and Bonomo 2010). 
The sheep from which sheep keds are collected are often treated with beta-lactam antibiotics. 
Besides, ferritin is highly expressed among whole gut, and transferrin is down-regulated in bacteriocytes. 
Ferritin sequesters iron from a blood meal and blocks iron ions intracellularly, while transferrin mediates transport of iron through blood plasma. 
Transferrin can also act as an antimicrobial protein sequestering iron from pathogens (Yoshiga et al.
2001). 
Down-regulation of transferrin in bacteriome might be a sort of immune privilege.
\par
There is potential immune compromise in sheep kep bacteriome. 
Immune response genes such as attacin (antimicrobial peptide) and two lysozymes, are down-regulated in bacteriocytes. 
PGRP-LB, an amidase that degrades peptidoglycan to inhibit immune responses, is highly expressed along whole gut, while GNBP, a pattern recognition receptor, is lowly expressed along gut. 
Furthermore, sheep ked lacks peritrophic matrix, a physical barrier to protect against pathogens. 
It is consist with the lack of chitin synthesis in sheep ked gut.
\par
\textbf{Waterhouse \textit{et al.}, 2007, Immune-related genes and pathways in disease-vector mosquitoes.} \newline
Immunity-related genes:\newline 
285 \textit{Drosophila melanogaster} (Dm), 338 \textit{Anopheles gambiae} (Ag), and 353 \textit{Aedes aegypti} (Aa) genes from 31 gene families and functional groups implicated in classical innate immunity or defense functions such as apoptosis and response to oxidative stress. \newline
Orthology groups (whole genome): \newline
4951 orthologous trios (1:1:1 orthologs in the three species) and 886 mosquito-specific orthologous pairs (absent from Dm). \newline
Orthology groups (immune-related):\newline
91 trios and 57 pairs, plus a combined total of 589 paralogous genes in the three species. \newline
Immune-related orthology trios are more divergent than that of whole genome: \newline
Phylogenetic distances of genes in each trio is measured by amino acid substitutions. 
With Dm as reference, immune-related trios of Ag and Am are more divergent (on average) compared with trios of whole genome, and several Ag immunity genes are considerably more divergent than their Aa orthologs.\newline
Large variation exists in different immune families in their proportions of orthologous trios,
mosquito-specific pairs and species-specific genes: \newline
(1) Predominantly trio orthologs: 
apoptosis inhibitors (IAPs), 
oxidative defense enzymes [
    superoxide dismutases (SODs), 
    glutathione peroxidases (GPXs), 
    thioredoxin peroxidases (TPXs), 
    heme-containing peroxidases (HPXs)],
class A and B scavenger receptors (SCRs). \newline
(2) Rarely trio orthologs: 
immune effectors, including three antimicrobial peptides.\newline
(3) Intermediately: 
C-type lectins.\newline
Strong divergent evolution of immune recognition genes: \newline
Fruit fly and mosquito recognition proteins mostly form distinct clades within each gene family.
\par
\textbf{Bosco-Drayon \textit{et al.}, 2012, Peptidoglycan sensing by the receptor PGRP-LE in the \textit{Drosophila} gut induces immune responses to infectious bacteria and tolerance to microbiota.} \newline
In \textit{Drosophila}, peptidoglycan recognition protein (PGRP)-LE senses peptidoglycan, and induces NF-kappaB dependent responses to infectious bacteria, but also tolerance to symbionts via up-regulation of pirk and PGRP-LB, which inhibits IMD signaling. 
Loss of PGRP-LE-mediated detection of bacteria in the gut results in systemic immune activation, which can be rescued by overexpressing PGRP-LB in the gut.
\par
\textbf{Wang \textit{et al.}, 2009, Interactions between mutualist \textit{Wigglesworthia} and tsetse peptidoglycan recognition protein (PGRP-LB) influence trypanosome transmission.} \newline
Tsetse flies have coevolved with mutualistic endosymbiont \textit{Wigglesworthia glossinidiae}. 
A tsetse peptidoglycan recognition protein (PGRP-LB) is crucial for symbiotic tolerance and trypanosome infection processes. 
Tsetse \textit{pgrp-lb} is expressed in the \textit{Wigglesworthia}-harboring organ (bacteriome) in the midgut, and its level of expression correlates with symbiont numbers. 
Adult tsetse cured of \textit{Wigglesworthia} infections have significantly lower \textit{pgrp-lb} levels than corresponding normal adults. 
RNA interference (RNAi)-mediated depletion of \textit{pgrp-lb} results in the activation of the immune deficiency (IMD) signaling pathway and leads to the synthesis of antimicrobial peptides (AMPs), which decrease \textit{Wigglesworthia} density. 
Depletion of \textit{pgrp-lb} also increases the host's susceptibility to trypanosome infections. 
Finally, parasitized adults have significantly lower \textitP{pgrp-lb} levels than flies, which have successfully eliminated trypanosome infections. 
When both PGRP-LB and IMD immunity pathway functions are blocked, flies become unusually susceptible to parasitism. 
Based on the presence of conserved amidase domains, tsetse PGRP-LB may scavenge the peptidoglycan (PGN) released by \textit{Wigglesworthia} and prevent the activation of symbiont-damaging host immune responses. 
In addition, tsetse PGRP-LB may have an anti-protozoal activity that confers parasite resistance. 
\par
\textbf{Martinez \textit{et al.}, 2016, Addicted? Reduced host resistance in populations with defensive symbionts.} \newline
Heritable symbionts that protect their hosts from pathogens have been described in a wide range of insect species. 
By reducing the incidence or severity of infection, these symbionts have the potential to reduce the strength of selection on genes in the insect genome that increase resistance. 
Therefore, the presence of such symbionts may slow down the evolution of resistance. 
Here we investigated this idea by exposing \textit{Drosophila melanogaster} populations to infection with the pathogenic Drosophila C virus (DCV) in the presence or absence of \textit{Wolbachia}, a heritable symbiont of arthropods that confers protection against viruses. 
After nine generations of selection, we found that resistance to DCV had increased in all populations. 
However, in the presence of \textit{Wolbachia} the resistant allele of \textit{pastrel}—a gene that has a major effect on resistance to DCV—was at a lower frequency than in the symbiont-free populations. 
This finding suggests that defensive symbionts have the potential to hamper the evolution of insect resistance genes, potentially leading to a state of evolutionary addiction where the genetically susceptible insect host mostly relies on its symbiont to fight pathogens.  
\par
\textbf{You \textit{et al.}, 2014, Homeostasis between gut-associated microorganisms and the immune system in Drosophila.} \newline
Analysis of Drosophila gut immunity revealed that bacterial-derived uracil and uracil-modulated intestinal reactive oxygen species (ROS) generation play a pivotal role in diverse aspects of host–microbe interactions, such as pathogen clearance, commensal protection, intestinal cell regeneration, colitogenesis, and possibly also interorgan immunological communication. 
\par
\textbf{Lima \textit{et al.}, 2021, Evolution of Toll, Spatzle and MyD88 in insects: the problem of the Diptera bias.} \newline
We evaluated the diversity of Toll pathway gene families in 39 Arthropod genomes, encompassing 13 different Insect Orders. 
Through computational methods, we shed some light into the evolution and functional annotation of protein families involved in the Toll pathway innate immune response. 
Our data indicates that: 
(1) intracellular proteins of the Toll pathway show mostly species-specific expansions; 
(2) the different Toll subfamilies seem to have distinct evolutionary backgrounds; 
(3) patterns of gene expansion observed in the Toll phylogenetic tree indicate that homology based methods of functional inference might not be accurate for some subfamilies; 
(4) Spatzle subfamilies are highly divergent and also pose a problem for homology based inference; (5) Spatzle subfamilies should not be analyzed together in the same phylogenetic framework; 
(6) network analyses seem to be a good first step in inferring functional groups in these cases. 
We specifically show that understanding \textit{Drosophila}’s Toll functions might not indicate the same function in other species. 
Our results show the importance of using species representing the different orders to better understand insect gene content, origin and evolution. 
More specifically, in intracellular Toll pathway gene families the presence of orthologues has important implications for homology based functional inference. 
Also, the different evolutionary backgrounds of Toll gene subfamilies should be taken into consideration when functional studies are performed, especially for TOLL9, TOLL, TOLL2_7, and the new TOLL10 clade. 
The presence of Diptera specific clades or the ones lacking Diptera species show the importance of overcoming the Diptera bias when performing functional characterization of Toll pathways.
\par
\textbf{Sackton \textit{et al.}, 2017, Rapid expansion of immune-related gene families in the house fly, \textit{Musca domestica}.} \newline
We apply whole-transcriptome sequencing to identify genes whose expression is regulated in adult flies upon bacterial infection. 
We then combine the transcriptomic data with analysis of rates of gene duplication and loss to provide insight into the evolutionary dynamics of immune-related genes. 
Genes up-regulated after bacterial infection are biased toward being evolutionarily recent innovations, suggesting the recruitment of novel immune components in the \textit{M. domestica} or ancestral Dipteran lineages. 
In addition, using new models of gene family evolution, we show that several different classes of immune-related genes, particularly those involved in either pathogen recognition or pathogen killing, are duplicating at a significantly accelerated rate on the \textit{M. domestica} lineage relative to other Dipterans. 
Taken together, these results suggest that the \textit{M. domestica} immune response includes an elevated diversity of genes, perhaps as a consequence of its lifestyle in septic environments.
\par
\textbf{Christophides \textit{et al.}, 2002, Immunity-related genes and gene families in \textit{Anopheles gambiae}.} \newline
We have identified 242 Anopheles gambiae genes from 18 gene families implicated in innate immunity and have detected marked diversification relative to Drosophila melanogaster. Immune-related gene families involved in recognition, signal modulation, and effector systems show a marked deficit of orthologs and excessive gene expansions, possibly reflecting selection pressures from different pathogens encountered in these insects' very different life-styles. In contrast, the multifunctional Toll signal transduction pathway is substantially conserved, presumably because of counterselection for developmental stability. Representative expression profiles confirm that sequence diversification is accompanied by specific responses to different immune challenges. Alternative RNA splicing may also contribute to expansion of the immune repertoire.
\par
\textbf{Zhang & Zhu, 2009, Drosomycin, an essential component of antifungal defence in \textit{Drosophila}.} \newline
Drosomycin is an inducible antifungal peptide of 44 residues initially isolated from bacteria-challenged \textit{Drosophila melanogaster}. 
The systemic expression of drosomycin is regulated by the Toll pathway present in fat body, whereas inducible local expression in the respiratory tract is controlled by the Immune Deficiency (IMD) pathway. 
Drosomycin belongs to the cysteine-stabilized alpha-helical and beta-sheet (CS$\alpha\beta$β) superfamily and is composed of an alpha-helix and a three-stranded beta-sheet stabilized by four disulphide bridges. 
Drosomycin exhibits a narrow antimicrobial spectrum and is only active against some filamentous fungi. 
However, recent work using recombinant drosomycin expressed in \textit{Escherichia coli} revealed its antiparasitic and anti-yeast activities. 
Two evolutionary epitopes (alpha- and gamma-patch) and the m-loop have been proposed as putative functional regions of drosomycin for interaction with fungi and parasites, respectively. 
Similarity in sequence, structure and biological activity suggests that drosomycin and some defensin molecules from plants and fungi could originate from a common ancestor.
\par
\textbf{The tsetse fly displays an attenuated immune response to its secondary symbiont, \textit{Sodalis glossinidius}.}
\textit{Sodalis glossinidius}, a vertically transmitted facultative symbiont of the tsetse fly, is a bacterium in the early/intermediate state of its transition toward symbiosis, representing an important model for investigating how the insect host immune defense response is regulated to allow endosymbionts to establish a chronic infection within their hosts without being eliminated. 
In this study, we report on the establishment of a tsetse fly line devoid of \textit{S. glossinidius} only, allowing us to experimentally investigate (i) the complex immunological interactions between a single bacterial species and its host, (ii) how the symbiont population is kept under control, and (iii) the impact of the symbiont on the vector competence of the tsetse fly to transmit the sleeping sickness parasite. 
Comparative transcriptome analysis showed no difference in the expression of genes involved in innate immune processes between symbiont-harboring ($Gmm^{Sod+}$) and \textit{S. glossinidius}-free ($Gmm^{Sod–}$) flies. 
Re-exposure of ($Gmm^{Sod–}$) flies to the endosymbiotic bacterium resulted in a moderate immune response, whereas exposure to pathogenic \textit{E. coli} or to a close non-insect associated relative of \textit{S. glossinidius}, i.e., \textit{S. praecaptivus}, resulted in full immune activation. 
We also showed that \textit{S. glossinidius} densities are not affected by experimental activation or suppression of the host immune system, indicating that \textit{S. glossinidius} is resistant to mounted immune attacks and that the host immune system does not play a major role in controlling \textit{S. glossinidius} proliferation. 
Finally, we demonstrate that the absence or presence of \textit{S. glossinidius} in the tsetse fly does not alter its capacity to mount an immune response to pathogens nor does it affect the fly’s susceptibility toward trypanosome infection.
\par
\textbf{Myllymaki \textit{et al.}, 2014, The \textit{Drosophila} Imd signaling pathway.} \newline
\par
\textbf{Kleino & Silverman, 2014, The \textit{Drosophila} IMD pathway in the activation of the humoral immune response.} \newline
\par
\textbf{Ganesan \textit{et al.}, 2010, NF-$\kappa$B/Rel proteins and the humoral immune responses of \textit{Drosophila melanogaster}.} \newline
\par
\textbf{Hill \textit{et al.}, The Genome of \textit{Drosophila innubila} Reveals Lineage-Specific Patterns of Selection in Immune Genes.} \newline
\par
\textbf{Jaenike \textit{et al.}, 2010, Adaptation via Symbiosis: Recent Spread of a Drosophila Defensive Symbiont.} \newline
\textit{Spiroplasma} is a laterally-transmitted secondary bacterial symbiont of \textit{Drosophila neotestacea} and it protects the host against the sterilizing effects of a parasitic nematode \textit{Howardula aoronymphium}, both in the laboratory and the field. 
\par
\textit{Spiroplasma} infection is dynamic and spreading within natural populations of \textit{Drosophila neotestacea}. 
First, PCR screening of museum specimens showed 0-0.14 prevalence of \textit{Spiroplasma} in early 1980s, while prevalence for 2010 is 0.5-0.8. 
Second, almost all nematode-infected \textit{Drosophila neotestacea} recorded in 1980s were sterile; while the fertility distribution of nematode-parasitized flies collected in 1989 was similar to that of nematode-parasitized flies that were uninfected with \textit{Spiroplasma} in 2008. 
It indicates that hosts have not evolved to improved protection against nematode parasits. 
Third, the prevalence of \textit{Spiroplasma} in flies showed geographical variation: prevalence decreasing from east to west. 
Finally, at equilibrium between natural selection favoring \textit{Spiroplasma} infection and imperfect maternal transmission resulting in loss of \textit{Spiroplasma}, the prevalence of \textit{Spiroplasma} infection should be similar among flies with different mtDNA haplotypes; flies carrying all major mtDNA haplotypes should be infected, and all individuals, whether infected or not, should be descended from infected females. 
This is not the case for \textit{D. neotestacea}, as \textit{Spiroplasma} is common in flies carrying certain mtDNA haplotypes but absent from all flies carrying “western” haplotypes. 
Consistent with this, mean within-population mitochondrial diversity is greater in populations where \textit{Spiroplasma} is absent than where it is present. 
Taken together, these four patterns suggest that \textit{Spiroplasma} has recently increased in frequency in the eastern populations of \textit{D. neotestacea} and may now be spreading from east to west across North America. 
\par
Does the apparent recent increase of \textit{Spiroplasma} result from 
(1) recent colonization of \textit{D. neotestacea} by \textit{Spiroplasma}, 
(2) a recent favorable \textit{Spiroplasma} mutation conferring tolerance to an existing parasite challenge, or 
(3) the imposition of a new selective pressure? 
The occurrence of \textit{Spiroplasma} in flies carrying three different mtDNA haplotypes suggests that the colonization of \textit{D. neotestacea} by \textit{Spiroplasma} was not a recent event. 
We previously found a perfect match between two slightly different \textit{Spiroplasma} variants and two closely related mtDNA haplotypes, indicating that sufficient time has elapsed since the original infection for mutations in both \textit{Spiroplasma} and mtDNA to have accumulated in the infected cytoplasmic lineages. 
Thus, \textit{Spiroplasma} was probably present within \textit{D. neotestacea} long before its recent increase. 
We can also rule out a recent favorable mutation, as both of these \textit{Spiroplasma} variants were associated with tolerance to nematode parasitism. 
Finally, we previously hypothesized that \textit{H. aoronymphium} had recently colonized North America, based on our finding of no DNA sequence variation (mtDNA COI) among North American samples of \textit{H. aoronymphium}, as well as sequence identity between North American and European samples of this species. 
Thus, the apparently rapid spread of \textit{Spiroplasma} is most likely due to recently imposed selection on \textit{D. neotestacea} to evolve tolerance of these sterilizing parasites. 
The presumed beneficial function of \textit{Spiroplasma} in \textit{D. neotestacea} before the arrival of \textit{H. aoronymphium} is unknown.

\end{sloppypar}
\end{document}