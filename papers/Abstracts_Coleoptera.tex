\documentclass[11pt]{article}

% \usepackage[UTF8]{ctex} % for Chinese 

\usepackage{setspace}
\usepackage[colorlinks,linkcolor=blue,anchorcolor=red,citecolor=black]{hyperref}
\usepackage{lineno}
\usepackage{booktabs}
\usepackage{graphicx}
\usepackage{float}
\usepackage{floatrow}
\usepackage{subfigure}
\usepackage{caption}
\usepackage{subcaption}
\usepackage{geometry}
\usepackage{multirow}
\usepackage{longtable}
\usepackage{lscape}
\usepackage{booktabs}
\usepackage{natbibspacing}
\usepackage[toc,page]{appendix}
\usepackage{makecell}
\usepackage{amsfonts}
\usepackage{amsmath}

\usepackage[backend=bibtex,style=authoryear,sorting=nyt,maxnames=1]{biblatex}
\bibliography{} % Reference bib

\title{Abstracts: Coleoptera}
\author{}
\date{}

\linespread{1.5}
\geometry{left=2cm,right=2cm,top=2cm,bottom=2cm}

\setlength\bibitemsep{0pt}

\begin{document}
\begin{sloppypar}
\maketitle

\linenumbers

\textbf{Maire \textit{et al.}, 2018, An IMD-like pathway mediates both endosymbiont control and host immunity in the cereal weevil \textit{Sitophilus spp.}.} \newline
In the cereal weevil \textit{Sitophilus spp.}, which houses \textit{Sodalis pierantonius}, endosymbionts are secluded in specialized bacteriocytes that group together as bacteriome. 
At standard conditions, the bacteriome highly expresses the coleoptericin A (colA) antimicrobial peptide (AMP), which was shown to prevent endosymbiont escape from the bacteriocytes. 
However, following the insect systemic infection by pathogens, the bacteriome upregulates a cocktail of AMP encoding genes, including colA. 
The regulations that allow these contrasted immune responses remain unknown. 
Here, evidence shows that an IMD-like pathway is conserved in two sibling species of cereal weevils, \textit{Sitophilus oryzae} and \textit{Sitophilus zeamais}. 
RNA interference (RNAi) experiments showed that \textit{imd} and \textit{relish} genes are essential for 
(i) colA expression in the bacteriome under standard conditions, 
(ii) AMP up-regulation in the bacteriome following a systemic immune challenge, 
(iii) AMP systemic induction following an immune challenge. 
Histological analyses also showed that \textit{relish} inhibition by RNAi resulted in endosymbiont escape from the bacteriome, strengthening the involvement of an IMD-like pathway in endosymbiont control. 
It is concluded that \textit{Sitophilus}’ IMD-like pathway mediates both the bacteriome immune program involved in endosymbiont seclusion within the bacteriocytes and the systemic and local immune responses to exogenous challenges. 
This work provides an example of how a conserved immune pathway, initially described as essential in pathogen clearance, also functions in the control of mutualistic associations.
\par
\textbf{Maire \textit{et al.}, 2019, Weevil pgrp-lb prevents endosymbiont TCT dissemination and chronic host systemic immune activation.}
The maintenance of immune homeostasis in organisms chronically infected with mutualistic bacteria is a challenging task, and little is known about the molecular processes that limit endosymbiont immunogenicity and host inflammation. 
Here, we investigated peptidoglycan recognition protein (PGRP)-encoding genes in the cereal weevil \textit{Sitophilus zeamais}’s association with \textit{Sodalis pierantonius} endosymbiont. 
We discovered that weevil \textit{pgrp-lb} generates three transcripts via alternative splicing and differential regulation. 
A secreted isoform is expressed in insect tissues under pathogenic conditions through activation of the PGRP-LC receptor of the immune deficiency pathway. 
In addition, cytosolic and transmembrane isoforms are permanently produced within endosymbiont-bearing organ, the bacteriome, in a PGRP-LC–independent manner. 
Bacteriome isoforms specifically cleave the tracheal cytotoxin (TCT), a peptidoglycan monomer released by endosymbionts. 
\textit{pgrp-lb} silencing by RNAi results in TCT escape from the bacteriome to other insect tissues, where it chronically activates the host systemic immunity through PGRP-LC. 
While such immune deregulations did not impact endosymbiont load, they did negatively affect host physiology, as attested by a diminished sexual maturation of adult weevils. 
Whereas pgrp-lb was first described in pathogenic interactions, this work shows that, in an endosymbiosis context, specific bacteriome isoforms have evolved, allowing endosymbiont TCT scavenging and preventing chronic endosymbiont-induced immune responses, thus promoting host homeostasis.
\par
\textbf{Hirota \textit{et al.}, 2020, Bacteriome-Associated Endosymbiotic Bacteria of \textit{Nosodendron} Tree Sap Beetles (Coleoptera: Nosodendridae)}
Here we investigated the bacteriomes and the endosymbiotic bacteria of tree sap beetle \textit{Nosodendron coenosum} and \textit{Nosodendron asiaticum} using molecular phylogenetic and histological approaches. 
In adults and larvae, a pair of slender bacteriomes were found along both sides of the midgut. 
The bacteriomes consisted of large bacteriocytes at the center and flat sheath cells on the surface. 
Fluorescence in situ hybridization detected preferential localization of the endosymbiotic bacteria in the cytoplasm of the bacteriocytes. 
In reproductive adult females, the endosymbiotic bacteria were also detected at the infection zone in the ovarioles and on the surface of growing oocytes, indicating vertical symbiont transmission via ovarial passage. 
Transmission electron microscopy unveiled bizarre structural features of the bacteriocytes, whose cytoplasm exhibited degenerate cytology with deformed endosymbiont cells. 
Molecular phylogenetic analysis revealed that the nosodendrid endosymbionts formed a distinct clade in the Bacteroidetes. 
The nosodendrid endosymbionts were the most closely related to the bacteriome endosymbionts of bostrichid powderpost beetles and also allied to the bacteriome endosymbionts of silvanid grain beetles, uncovering an unexpected endosymbiont relationship across the unrelated beetle families Nosodendridae, Bostrichidae and Silvanidae.

\end{sloppypar}
\end{document}