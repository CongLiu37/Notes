\documentclass[11pt]{article}

% \usepackage[UTF8]{ctex} % for Chinese 

\usepackage{setspace}
\usepackage[colorlinks,linkcolor=blue,anchorcolor=red,citecolor=black]{hyperref}
\usepackage{lineno}
\usepackage{booktabs}
\usepackage{graphicx}
\usepackage{float}
\usepackage{floatrow}
\usepackage{subfigure}
\usepackage{caption}
\usepackage{subcaption}
\usepackage{geometry}
\usepackage{multirow}
\usepackage{longtable}
\usepackage{lscape}
\usepackage{booktabs}
\usepackage{natbibspacing}
\usepackage[toc,page]{appendix}
\usepackage{makecell}
\usepackage{amsfonts}
 \usepackage{amsmath}
\usepackage[utf8]{inputenc}
\usepackage{amssymb}
\usepackage{amsthm}
\usepackage{enumerate}
\usepackage{comment}

\usepackage[backend=bibtex,style=authoryear,sorting=nyt,maxnames=1]{biblatex}
\bibliography{} % Reference bib

\title{Other_lineages}
\author{}
\date{}

\linespread{1.5}
\geometry{left=2cm,right=2cm,top=2cm,bottom=2cm}

\setlength\bibitemsep{0pt}

\begin{document}
\begin{sloppypar}
  \maketitle

  \linenumbers
\textbf{Palmer \textit{et al.}, 2015, Comparative genomics reveals the origin and diversity of Arthropod immune system.} \newline
Immune gene families are searched in predicted peptide sets of arthropod species, including 
insect \textit{Drosophila melanogaster}, 
crustacean \textit{Daphnia pulex} (water flea), 
myriapod \textit{Strigamia maritima} (coastal centipede) and 
five chelicerates: \textit{Mesobuthus martensii} (Chinese scorpion), 
                    \textit{Parasteatoda tepidariorum} (house spider),
                    \textit{Ixodes scapularis} (deer tick),
                    \textit{Metaseiulus occidentalis} (western orchard predatory mite),
                    \textit{Tetranychus urticae} (red spider mite). 
Arthropod Toll-like receptors (TLRs) are a dynamically evolving gene family that includes relatives of vertebrate TLRs. 
The Toll signaling pathway is conserved across arthropods. 
The IMD signaling pathway is highly reduced in chelicerates. 
\textit{IMD} is frequently absent. 
\textit{Relish} is often present. 
ANK domain of Relish is occassionally absent, indicating Dredd-independent activation. 
The JAK/STAT signaling pathway is highly conserved. 
$\beta$-1,3 glucan recognition proteins ($\beta$GRPs) have been lost in chelicerates. 
Arthropod TEPs include relatives of vertebrate C3 complement factors and proteins lacking the thioester motif. 
Gene duplication generates diversity in the immune receptor Down syndrome cell adhesion molecule (Dscam). 
Argonaute 2 evolves rapidly and is often duplicated with variable copy numbers.
\par
\textbf{Lan \textit{et al.}, 2022, Endosymbiont population genomics sheds light on transmission mode, partner specificity, and stability of the scaly-foot snail holobiont.} \newline
The scaly-foot snail (\textit{Chrysomallon squamiferum}) inhabiting deep-sea hydrothermal vents in the Indian Ocean relies on its sulphur-oxidising gammaproteobacterial endosymbionts for nutrition and energy. 
In this study, we investigate the specificity, transmission mode, and stability of multiple scaly-foot snail populations dwelling in five vent fields with considerably disparate geological, physical and chemical environmental conditions. 
Results of population genomics analyses reveal an incongruent phylogeny between the endosymbiont and mitochondrial genomes of the scaly-foot snails in the five vent fields sampled, indicating that the hosts obtain endosymbionts via horizontal transmission in each generation. 
However, the genetic homogeneity of many symbiont populations implies that vertical transmission cannot be ruled out either. 
Fluorescence in situ hybridisation of ovarian tissue yields symbiont signals around the oocytes, suggesting that vertical transmission co-occurs with horizontal transmission. 
Results of in situ environmental measurements and gene expression analyses from in situ fixed samples show that the snail host buffers the differences in environmental conditions to provide the endosymbionts with a stable intracellular micro-environment, where the symbionts serve key metabolic functions and benefit from the host's cushion. 
The mixed transmission mode, symbiont specificity at the species level, and stable intracellular environment provided by the host support the evolutionary, ecological, and physiological success of scaly-foot snail holobionts in different vents with unique environmental parameters.
\par
\textbf{Weber \textit{et al.}, 2022, Evolutionary gain and loss of a pathological immune response to parasitism.}


  
\end{sloppypar}
\end{document}