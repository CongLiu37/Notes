\documentclass[11pt]{article}

% \usepackage[UTF8]{ctex} % for Chinese 

\usepackage{setspace}
\usepackage[colorlinks,linkcolor=blue,anchorcolor=red,citecolor=black]{hyperref}
\usepackage{lineno}
\usepackage{booktabs}
\usepackage{graphicx}
\usepackage{float}
\usepackage{floatrow}
\usepackage{subfigure}
\usepackage{caption}
\usepackage{subcaption}
\usepackage{geometry}
\usepackage{multirow}
\usepackage{longtable}
\usepackage{lscape}
\usepackage{booktabs}
\usepackage{natbibspacing}
\usepackage[toc,page]{appendix}
\usepackage{makecell}
\usepackage{amsfonts}
 \usepackage{amsmath}
\usepackage[utf8]{inputenc}
\usepackage{amssymb}
\usepackage{amsthm}
\usepackage{enumerate}
\usepackage{comment}

\usepackage[backend=bibtex,style=authoryear,sorting=nyt,maxnames=1]{biblatex}
\bibliography{} % Reference bib

\title{Other_lineages}
\author{}
\date{}

\linespread{1.5}
\geometry{left=2cm,right=2cm,top=2cm,bottom=2cm}

\setlength\bibitemsep{0pt}

\begin{document}
\begin{sloppypar}
  \maketitle

  \linenumbers
\textbf{Palmer \textit{et al.}, 2015, Comparative genomics reveals the origin and diversity of Arthropod immune system.} \newline
Immune gene families are searched in predicted peptide sets of arthropod species, including 
insect \textit{Drosophila melanogaster}, 
crustacean \textit{Daphnia pulex} (water flea), 
myriapod \textit{Strigamia maritima} (coastal centipede) and 
five chelicerates: \textit{Mesobuthus martensii} (Chinese scorpion), 
                    \textit{Parasteatoda tepidariorum} (house spider),
                    \textit{Ixodes scapularis} (deer tick),
                    \textit{Metaseiulus occidentalis} (western orchard predatory mite),
                    \textit{Tetranychus urticae} (red spider mite). 
Arthropod Toll-like receptors (TLRs) are a dynamically evolving gene family that includes relatives of vertebrate TLRs. 
The Toll signaling pathway is conserved across arthropods. 
The IMD signaling pathway is highly reduced in chelicerates. 
\textit{IMD} is frequently absent. 
\textit{Relish} is often present. 
ANK domain of Relish is occassionally absent, indicating Dredd-independent activation. 
The JAK/STAT signaling pathway is highly conserved. 
$\beta$-1,3 glucan recognition proteins ($\beta$GRPs) have been lost in chelicerates. 
Arthropod TEPs include relatives of vertebrate C3 complement factors and proteins lacking the thioester motif. 
Gene duplication generates diversity in the immune receptor Down syndrome cell adhesion molecule (Dscam). 
Argonaute 2 evolves rapidly and is often duplicated with variable copy numbers.
\par
\textbf{Lan \textit{et al.}, 2022, Endosymbiont population genomics sheds light on transmission mode, partner specificity, and stability of the scaly-foot snail holobiont.} \newline
The scaly-foot snail (\textit{Chrysomallon squamiferum}) inhabiting deep-sea hydrothermal vents in the Indian Ocean relies on its sulphur-oxidising gammaproteobacterial endosymbionts for nutrition and energy. 
In this study, we investigate the specificity, transmission mode, and stability of multiple scaly-foot snail populations dwelling in five vent fields with considerably disparate geological, physical and chemical environmental conditions. 
Results of population genomics analyses reveal an incongruent phylogeny between the endosymbiont and mitochondrial genomes of the scaly-foot snails in the five vent fields sampled, indicating that the hosts obtain endosymbionts via horizontal transmission in each generation. 
However, the genetic homogeneity of many symbiont populations implies that vertical transmission cannot be ruled out either. 
Fluorescence in situ hybridisation of ovarian tissue yields symbiont signals around the oocytes, suggesting that vertical transmission co-occurs with horizontal transmission. 
Results of in situ environmental measurements and gene expression analyses from in situ fixed samples show that the snail host buffers the differences in environmental conditions to provide the endosymbionts with a stable intracellular micro-environment, where the symbionts serve key metabolic functions and benefit from the host's cushion. 
The mixed transmission mode, symbiont specificity at the species level, and stable intracellular environment provided by the host support the evolutionary, ecological, and physiological success of scaly-foot snail holobionts in different vents with unique environmental parameters.
\par
\textbf{Delaux \textit{et al.}, 2014, Comparative Phylogenomics Uncovers the Impact of Symbiotic Associations on Host Genome Evolution.} \newline
Arbuscular mycorrhization (AM) is a widespread land plant symbiont, but it has been independently lost in some lineages. 
Establishment of effective symbiosis requires a set of conserved genes identified in legumes, the so called symbiotic toolkit. 
Targeted phylogenetic analyses in Arabidopsis (order Brassicales) led to the broad classification of the “symbiotic toolkit” genes into two subsets: 
1. a subset called conserved genes that is conserved in \textit{Arabidopsis thaliana} despite the loss of AM symbiosis and play non-symbiotic roles; 
2. a subset of symbiosis-specific genes that are absent in this non-host species. 
Non-host Brassicales have lost many symbiosis-specific genes of the symbiotic toolkit. 
Similar patterns are observed in several other lineages. 
Our results support the reverse hypothesis: the loss of gene(s) from the symbiotic toolkit was the primary cause for the loss of AM symbiosis, and was followed by the emergence of alternative nutrient uptake strategies. 
\par
\textbf{Weber \textit{et al.}, 2022, Evolutionary gain and loss of a pathological immune response to parasitism.}
\par
\textbf{Avina-Padilla |textit{et al.}, 2021, Evolutionary Perspective and Expression Analysis of Intronless Genes Highlight the Conservation of Their Regulatory Role.} \newline
This work aimed to infer the functional role and evolutionary history of IGs centered on the mouse genome. IGs consist of a subgroup of genes with one exon including coding genes, non-coding genes, and pseudogenes, which conform approximately 6\% of a total of 21,527 genes. To understand their prevalence, biological relevance, and evolution, we identified and studied 1,116 IG functional proteins validating their differential expression in transcriptomic data of embryonic mouse telencephalon. Our results showed that overall expression levels of IGs are lower than those of MEGs. However, strongly up-regulated IGs include transcription factors (TFs) such as the class 3 of POU (HMG Box), Neurog1, Olig1, and BHLHe22, BHLHe23, among other essential genes including the β-cluster of protocadherins. Most striking was the finding that IG-encoded BHLH TFs fit the criteria to be classified as microproteins. Finally, predicted protein orthologs in other six genomes confirmed high conservation of IGs associated with regulating neural processes and with chromatin organization and epigenetic regulation in Vertebrata. Moreover, this study highlights that IGs are essential modulators of regulatory processes, such as the Wnt signaling pathway and biological processes as pivotal as sensory organ developing at a transcriptional and post-translational level. Overall, our results suggest that IG proteins have specialized, prevalent, and unique biological roles and that functional divergence between IGs and MEGs is likely to be the result of specific evolutionary constraints.
\par
\textbf{Shultz & Sackton, 2019, Immune genes are hotspots of shared positive selection across birds and mammals.} \newline
\par
\textbf{Acquisti \textit{et al.}, 2009, Signatures of nitrogen limitation in the elemental composition of the proteins involved in the metabolic apparatus.} \newline
\par
\textbf{Seward & Kelly, 2016, Dietary nitrogen alters codon bias and genome composition in parasitic microorganisms.} \newline
\par
\textbf{Seligmann, 2002, Cost-minimization of amino acid usage.} \newline
\par
\textbf{Maeda & Fernie, 2021, Evolutionary history of plant metabolisn.} \newline
\par
\textbf{Hsu \textit{et al.}, 2023,Arginine limitation drives a directed codon-dependent DNA sequence evolution response in colorectal cancer cells.} \newline
\par
\textbf{Christmas \textit{et al.}, 2023, Evolutionary constraint and innovation across hundreds of placental mammals.} \newline
\par
\textbf{Undheim \textit{et al.}, 2021, Phylogenetic analyses suggest centipede venom arsenals were repeatedly stocked by horizontal gene transfer.} \newline
\par


\end{sloppypar}
\end{document}