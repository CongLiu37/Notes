\documentclass[11pt]{article}

% \usepackage[UTF8]{ctex} % for Chinese 

\usepackage{setspace}
\usepackage[colorlinks,linkcolor=blue,anchorcolor=red,citecolor=black]{hyperref}
\usepackage{lineno}
\usepackage{booktabs}
\usepackage{graphicx}
\usepackage{float}
\usepackage{floatrow}
\usepackage{subfigure}
\usepackage{caption}
\usepackage{subcaption}
\usepackage{geometry}
\usepackage{multirow}
\usepackage{longtable}
\usepackage{lscape}
\usepackage{booktabs}
\usepackage{natbibspacing}
\usepackage[toc,page]{appendix}
\usepackage{makecell}
\usepackage{amsfonts}
 \usepackage{amsmath}

\usepackage[backend=bibtex,style=authoryear,sorting=nyt,maxnames=1]{biblatex}
\bibliography{} %Reference bib

\title{Abstracts: Blattodea}
\author{}
\date{}

\linespread{1.5}
\geometry{left=2cm,right=2cm,top=2cm,bottom=2cm}

\setlength\bibitemsep{0pt}

\begin{document}
\begin{sloppypar}
  \maketitle

  \linenumbers
\textbf{Bucek \textit{et al.}, 2019, Evolution of Termite Symbiosis Informed by Transcriptome-Based Phylogenies.} \newline
We sequenced genomes and transcriptomes of 55 termite species and reconstructed phylogenetic trees from up to 4,065 orthologous genes of 68 species. 
We found strong support for a novel sister-group relationship between the bacterial comb-building Sphaerotermitinae and fungus comb-building Macrotermitinae. 
This key finding indicates that comb building is a derived trait within Termitidae and that the creation of a comb-like “external rumen” involving bacteria or fungi may not have driven the loss of protozoa from ancestral termitids, as previously hypothesized. 
Instead, associations with gut prokaryotic symbionts, combined with dietary shifts from wood to other plant-based substrates, may have played a more important role in this symbiotic transition. 
\par
\textbf{Bourguignon \textit{et al.}, 2017, Mitochondrial Phylogenomics Resolves the Global Spread of Higher Termites, Ecosystem Engineers of the Tropics.} \newline

\par

\textbf{He \textit{et al.}, 2021, Evidence for reduced immune gene diversity and activity during the evolution of termites.} \newline
18 cockroach and termite species, including 9 lower termites, 6 higher termites, 2 subsocial wood roaches and 2 solitary roaches were studied via \textit{de novo} assembled transcriptomes. 
Each immune gene family presented in all species, except antifungal dorsomycin, which was lost in termites and wood roaches. 
Phylogenetic signal analysis revealed a loss of total immune gene diversity during termite evolution. 
\par
Antifungal peptide drosomycin was lost in ancestor of wood roach \textit{Cryptocercus} and termites. 
C-type lectin (CTL) underwent two contractions in most recent common ancestor (MRCA) of (1) wood roach \textit{Cryptocercus} and termites; and (2) Rhinotermitidae and Termitidae, but together with lysozymes, re-expanded in late branch of higher termites, \textit{i.e.} MRCA of \textit{Promirotermes} and \textit{Dicuspiditermes}. 
Serine protease CLIP contracted in MRCA of Rhinotermitidae and Termitidae. 
Thiroredoxin peroxidase (TPX) and autophagy-related gene (ATG) contracted in Termitidae, while defensin expanded. 
\par
In bees, immune gene depletion seems to have preceded evolution of eusociality (Barribeau \textit{et al.}, 2015), indicating immune gene depletion is unrelated with transition to sociality. 
Although there was contraction of immune genes in termite evolution, it can be interpreted as an expansion of immune genes in solitary cockroaches (Harrison \textit{et al.}, 2018) followed by returning to a representative level.
\par
\textbf{Meusemann \textit{et al.}, 2020, No Evidence for Single-Copy Immune-Gene Specific Signals of Selection in Termites.} \newline
There are generally two life type in termites: wood-dwelling termites that nest in a single piece of wood that serves as food and shelter, and foraging termites that leave nest and forage outside. 
Wood-dwelling termites can be further divided into two types: living in drywood or decaying dampwood. 
\par
We hypothesized that selective pressure on single-copy immune defense genes (IGs) differs across termites depending on their life type. 
We tested the hypothesis that wood-dwelling termites show relaxed selection on IGs and fewer signs of positive selection compared to soil-foraging species. 
Additionally, we tested whether within the wood-dwelling life type, the dampwood termite \textit{Zootermopsis nevadensis} had stronger signs of selection than other wood-dwellers that nest in sound wood.
\par
In order to test for differences in selective forces acting on IGs between wood-dwelling and foraging species, we analyzed a set of 81 previously identified single-copy IGs in eight termite species. 
Four of the species are wood-dwelling species: \textit{Cryptotermes secundus}, \textit{Incisitermes marginipennis}, \textit{Prorhinotermes simplex}, and \textit{Zootermopsis nevadensis}. 
The remaining four species are foraging species: \textit{Mastotermes darwiniensis}, \textit{Reticulitermes flavipes}, \textit{Coptotermes sp.} and \textit{Macrotermes natalensis}. 
\par
Contrary to our expectation, we did not detect an effect of life type on signs of selection for immune genes. 
Neither did wood-dwelling species differ from the soil foraging species, nor did the dampwood termite \textit{Zootermopsis nevadensis} differ from the other wood-dwelling termites. 
There are several alternative explaniations: 
(1) there are life-type-related selection patterns in multi-copy IGs; 
(2) there are recent selective sweeps in single-copy IGs; 
(3) evolution of social immunity buffers selection pressure on IGs; 
(4) evolution of termite IGs is mainly driven by pressure from modulating gut microbiome, which performs, in principle, similar functions including lignocellulose digestion and nitrogen acquisition. 
\par
\textbf{Bulmer \textit{et al.}, 2021, Termite eusociality and contrasting selective pressure on social and innate immunity.} \newline
Selective pressure on immunity is tested. 
Transition to foraging coincides with relaxed selection on key components of Toll signaling and phenoloxidase cascade, while selection on Gram-negative binding proteins (\textit{GNBP}s) secreted for external antifungal defense (social immunity) is intensified. 
\par
\textbf{Bulmer & Crozier, 2006, Variation in positive selection in termite GNBPs and Relish.} \newline
Two Gram-negative binding genes (\textit{GNBP}s) and \textit{Relish} of \textit{Nasutitermes} are studies. 
Occurance of positive selection ($dN/dS>1$) differs among lineages and genes. 
Positive selection has driven evolution of the three genes in a lineage of three termite species feeding on damp wood. 
\textit{Relish} have experienced highest level of positive selection and selected sites are in regions responsible for activation.
\par
\textbf{Bulmer \textit{et al.}, 2010, Adaptive evolution in subterranean termite antifungal peptides.} \newline
We identified and analysed mRNA sequences of two immune proteins from the subterranean termites \textit{Reticulitermes flavipes} and \textit{Reticulitermes virginicus}. 
These proteins correspond to two immune proteins described in the distantly related termite genus \textit{Nasutitermes}; termicin, which is a small antifungal peptide, and GNBP2, which functions both as a broad pattern recognition receptor and a direct antifungal effector. 
A population genetic analysis of nucleotide intraspecific polymorphism and interspecific divergence indicates that a selective sweep has reduced polymorphism in the termicins. 
Moreover, this selective sweep appears to have been driven by the positive selection of beneficial amino acid changes in the antifungal peptide. 
In contrast, the pattern of polymorphism and divergence in GNBP2 corresponds to the standard neutral model of evolution.
\par
\textbf{Liu \textit{et al.}, 2022, microRNAs shape social immunity: a potential target for biological control of the termite \textit{Reticulitermes chinensis}.} \newline

\par
\textbf{Sabree \textit{et al.}, 2009, Nitrogen recycling and nutritional provisioning by \textit{Blattabacterium}, the cockroach endosymbiont.} \newline
Cockroaches, unlike most terrestrial insects, excrete waste nitrogen within their fat bodies as uric acids, postulated to be a supplement when dietary nitrogen is limited. 
The fat bodies of most cockroaches are inhabited by \textit{Blattabacterium}, which are vertically transmitted, Gram-negative bacteria that have been hypothesized to participate in uric acid degradation, nitrogen assimilation, and nutrient provisioning. 
We have sequenced completely the \textit{Blattabacterium} genome from \textit{Periplaneta americana}. 
Genomic analysis confirms that \textit{Blattabacterium} is a member of the Flavobacteriales (Bacteroidetes), with its closest known relative being the endosymbiont \textit{Sulcia muelleri}, which is found in many sap-feeding insects. 
Metabolic reconstruction indicates that it lacks recognizable uricolytic enzymes, but it can recycle nitrogen from urea and ammonia (uric acid degradation products) into glutamate, using urease and glutamate dehydrogenase. 
Subsequently, \textit{Blattabacterium} can produce all of the essential amino acids, various vitamins, and other required compounds from a limited palette of metabolic substrates. 
The ancient association with \textit{Blattabacterium} has allowed cockroaches to subsist successfully on nitrogen-poor diets and to exploit nitrogenous wastes, capabilities that are critical to the ecological range and global distribution of cockroach species.
\par
\textbf{Neef \textit{et al.}, 2011, Genome Economization in the Endosymbiont of the Wood Roach \textit{Cryptocercus punctulatus} Due to Drastic Loss of Amino Acid Synthesis Capabilities.} \newline
Cockroaches (Blattaria: Dictyoptera) harbor the endosymbiont \textit{Blattabacterium} in their abdominal fat body. 
The genome of \textit{Blattabacterium sp.} of \textit{Cryptocercus punctulatus} (BCpu) was sequenced and compared with those of the symbionts of \textit{Blattella germanica} and \textit{Periplaneta americana}, BBge and BPam, respectively. 
The BCpu genome consists of a chromosome of 605.7 kb and a plasmid of 3.8 kb and is therefore approximately 31 kb smaller than the other two aforementioned genomes. 
The size reduction is due to the loss of 55 genes, 23 of which belong to biosynthetic pathways for amino acids. 
The pathways for the production of tryptophan, leucine, isoleucine/threonine/valine, methionine, and cysteine have been completely lost. 
Additionally, the genes for the enzymes catalyzing the last steps of arginine and lysine biosynthesis, argH and lysA, were found to be missing and pseudogenized, respectively. 
These gene losses render BCpu auxotrophic for nine amino acids more than those corresponding to BBge and BPam. 
BCpu has also lost capacities for sulfate reduction, production of heme groups, as well as genes for several other unlinked metabolic processes, and genes present in BBge and BPam in duplicates. 
Amino acids and cofactors that are not synthesized by BCpu are either produced in abundance by hindgut microbiota or are provisioned via a copious diet of dampwood colonized by putrefying microbiota, supplying host and \textit{Blattabacterium} symbiont with the necessary nutrients and thus permitting genome economization of BCpu.
\par
\textbf{Sabree \textit{et al.}, 2012, Genome Shrinkage and Loss of Nutrient-Providing Potential in the Obligate Symbiont of the Primitive Termite Mastotermes darwiniensis.} \newline
Termite gut microbes collaborate to fix nitrogen, degrade lignocellulose, and produce nutrients, and the absence of \textit{Blattabacterium} in nearly all termites suggests that its nutrient-provisioning role has been replaced by gut microbes. 
\textit{Mastotermes darwiniensis} is a basal, extant termite that solely retains \textit{Blattabacterium}, which would show evidence of relaxed selection if it is being supplanted by the gut microbiome. 
This termite-associated \textit{Blattabacterium} genome is ∼8\% smaller than cockroach-associated \textit{Blattabacterium} genomes and lacks genes underlying vitamin and essential amino acid biosynthesis. 
Furthermore, the \textit{Mastotermes darwiniensis} gut microbiome membership is more consistent between individuals and includes specialized termite gut-associated bacteria, unlike the more variable membership of cockroach gut microbiomes. 
The \textit{Mastotermes darwiniensis} \textit{Blattabacterium} genome may reflect relaxed selection for some of its encoded functions, and the loss of this endosymbiont in all remaining termite genera may result from its replacement by a functionally complementary gut microbiota.
\par
\textbf{Kinjo \textit{et al.}, 2018, Parallel and Gradual Genome Erosion in the \textit{Blattabacterium} Endosymbionts of \textit{Mastotermes darwiniensis} and \textit{Cryptocercus} Wood Roaches.} \newline
Almost all examined cockroaches harbor an obligate intracellular endosymbiont, \textit{Blattabacterium cuenoti}. 
On the basis of genome content, \textit{Blattabacterium} has been inferred to recycle nitrogen wastes and provide amino acids and cofactors for its hosts. 
Most \textit{Blattabacterium} strains sequenced to date harbor a genome of ∼630 kbp, with the exception of the termite \textit{Mastotermes darwiniensis} (∼590 kbp) and \textit{Cryptocercus punctulatus} (∼614 kbp), a representative of the sister group of termites. 
Such genome reduction may have led to the ultimate loss of \textit{Blattabacterium} in all termites other than Mastotermes. 
In this study, we sequenced 11 new \textit{Blattabacterium} genomes from three species of \textit{Cryptocercus} in order to shed light on the genomic evolution of \textit{Blattabacterium} in termites and \textit{Cryptocercus}. 
All genomes of \textit{Cryptocercus}-derived \textit{Blattabacterium} genomes were reduced (∼614 kbp), except for that associated with \textit{Cryptocercus kyebangensis}, which comprised 637 kbp. 
Phylogenetic analysis of these genomes and their content indicates that \textit{Blattabacterium} experienced parallel genome reduction in Mastotermes and \textit{Cryptocercus}, possibly due to similar selective forces. 
We found evidence of ongoing genome reduction in \textit{Blattabacterium} from three lineages of the \textit{C. punctulatus} species complex, which independently lost one cysteine biosynthetic gene. 
We also sequenced the genome of the \textit{Blattabacterium} associated with \textit{Salganea taiwanensis}, a subsocial xylophagous cockroach that does not vertically transmit gut symbionts via proctodeal trophallaxis. 
This genome was 632 kbp, typical of that of nonsubsocial cockroaches. 
Overall, our results show that genome reduction occurred on multiple occasions in \textit{Blattabacterium}, and is still ongoing, possibly because of new associations with gut symbionts in some lineages.

\par

\textbf{Dietrich \textit{et al.}, 2014, The Cockroach Origin of the Termite Gut Microbiota: Patterns in Bacterial Community Structure Reflect Major Evolutionary Events.} \newline
16S profiled gut bacterial communities of 34 termite and cockroach species. 
Generally, gut bacterial communities of cockroaches, lower termites and higher termites were clearly separated. 
Subfamilies of higher termites formed distinct clusters. 
Fungus-cultivating Macrotermitinae (higher termites) showed strong affinity for the cockroaches.
Wood-feeding cockroach \textit{Cryptocercus punctulatus}, the closest relative of termites, was close to lower termites, with which it shares the presence of cellulolytic flagellates. 
Other wood-feeding cockroaches \textit{Panesthia angustipennis} and \textit{Salganea esakii} (family Blaberidae), whose gut microbiota lacks such flagellates, clustered with the omnivorous cockroaches. 
When mapped onto the host tree, the changes in community structure coincided with major events in termite evolution, such as acquisition and loss of cellulolytic protists and the ensuing dietary diversification. 
UniFrac analysis of the core microbiota of termites and cockroaches and construction of phylogenetic tree of individual genus level lineages revealed a general host signal, whereas the branching order often did not match the detailed phylogeny of the host. 
It remains unclear whether the lineages in question have been associated with the ancestral cockroach since the early Cretaceous (cospeciation) or are diet-specific lineages that were independently acquired from the environment (host selection).
\par
\textbf{Warnecke \textit{et al.}, 2007, Metagenomic and functional analysis of hindgut microbiota of a wood-feeding higher termite.} \newline
Only recently have data supported any direct role for the symbiotic bacteria in the gut of the termite in cellulose and xylan hydrolysis. 
Here we use a metagenomic analysis of the bacterial community resident in the hindgut paunch of a wood-feeding ‘higher’ Nasutitermes species (which do not contain cellulose-fermenting protozoa) to show the presence of a large, diverse set of bacterial genes for cellulose and xylan hydrolysis. 
Many of these genes were expressed in vivo or had cellulase activity in vitro, and further analyses implicate spirochete and fibrobacter species in gut lignocellulose degradation. 
New insights into other important symbiotic functions including H2 metabolism, CO2-reductive acetogenesis and N2 fixation are also provided by this first system-wide gene analysis of a microbial community specialized towards plant lignocellulose degradation.
\par
\textbf{Bourguignon \textit{et al.}, 2014, Rampant Host Switching Shaped the Termite Gut Microbiome.} \newline
16S metabarcoding profiled gut bacterial community of 94 termite species (77 higher termites). 
211 lineages containing > 10 OTUs were processed by phylogenetic analysis. 
Phylogenetic trees were classified into 3 categories. 
Category 1 represents trees in which ≥30\% of termite-derived sequences formed a monophyletic group. 
In category 1, termite-specific lineages are often composed of bacteria associated with distinct termite lineages, and are sister groups of bacteria living in invertebrate/vertebrate gut. 
It is hypothesize that the last common ancestors of each termite-specific clade became specialized for termite gut environments and eventually became widespread across a large number of termites through both parent-to-offspring vertical transmission and horizontal colony-to-colony transfer between termites. 
Category 2 represents trees containing monophyletic clades composed of >30\% termite-associated bacteria and > 10\% non-termite-associated bacteria which are often living in guts of other animals. 
It is hypothesized that these bacterial lineages have been specialized for termite gut environments and became widespread through vertical transmission and horizontal transmission between termites and other animals. 
Category 3 represents rest trees in which termite-associated bacteria intersperse with bacteria from other environments. 
Congruence of bacterial and termite phylogeny did not find evidence for strict vertical transmission. 
The results indicate that “mixed-mode” transmission, which combines colony-to-offspring vertical transmission with horizontal colony-to-colony transfer, has been the primary driving force shaping the gut bacterial community of termites.
\textbf{Lo & Evans, 2006, Phylogenetic diversity of the intracellular symbiont \textit{Wolbachia} in termites} \newline
44  termite populations, representing of a total of 30 species, 14 genera, and 6 families were screened for the presence of \textit{Wolbachia}, using PCR assays for the genes 16S rRNA and ftsZ. 
12 out of 44 populations were found to be infected. 
\textbf{Sabree \textit{et al.}, 2012, Genome Shrinkage and Loss of Nutrient-Providing Potential in the Obligate Symbiont of the Primitive Termite \textit{Mastotermes darwiniensis}} \newline
\textit{Mastotermes darwiniensis} is the basal termite harboring \textit{Blattabacterium} endosymbionts, which is present in most cockroaches but absent in other termites. 
\textit{Blattabacterium} endosymbionts of \textit{Mastotermes darwiniensis} has shrinked genome, lossing pathways for synthesis of vitamins and essential amino acids. 
\textit{Mastotermes} gut microbiota contains characteristic termite-associated bacteria.
It is proposed that \textit{Mastotermes} gut microbiome replaces functions of \textit{Blattabacterium}, resulting in relaxed selection.
\par
\textbf{Peterson & Scharf, 2016, Lower Termite Associations with Microbes: Synergy, Protection, and Interplay.} \newline
Grooming and hygienic behavior play an important role in termite immunity. 
Termites with perturbed gut microbiota, by oxygenation or chemical means, display a marked increase in susceptibility to fungal pathogens. 
One biochemical mechanism has been linked to this anti-fungal gut phenomenon in the form of symbiont-derived $\beta$-1, 3-glucanase activity (most likely protist) that is able to act on fungi and prevent their germination. 
Similarly, the inhibition of this antifungal enzyme activity, $\beta$-1, 3-glucanase, results in a marked increase in termite susceptibility to a variety of pathogens and is conserved evolutionarily from woodroaches to termites. 
Termites build nests with fecal material. 
The nests of one species of subterranean termite, \textit{Coptotermes formosanus}, are commonly laden with symbiotic Actinobacteria demonstrated to have antifungal activity ex vivo in nest walls.
\par
\textbf{Pasari \textit{et al.}, 2019, Genome analysis of \textit{Paenibacillus polymyxa} A18 gives insights into the features associated with its adaptation to the termite gut environment.} \newline
\textit{Paenibacillus polymyxa} A18 was isolated from termite \textit{Odontotermes hainanensis} gut and was identified as a potential cellulase and hemicellulase producer in our previous study. 
Considering that members belonging to genus \textit{Paenibacillus} are mostly free-living in soil, we investigated here the essential genetic features that helped \textit{P. polymyxa} A18 to survive in gut environment. 
Genome sequencing and analysis identified 4,608 coding sequences along with several elements of horizontal gene transfer, insertion sequences, transposases and integrated phages, which add to its genetic diversity. 
Many genes coding for carbohydrate-active enzymes, including the enzymes responsible for woody biomass hydrolysis in termite gut, were identified in \textit{P. polymyxa} A18 genome. 
Further, a series of proteins conferring resistance to 11 antibiotics and responsible for production of 4 antibiotics were also found to be encoded, indicating selective advantage for growth and colonization in the gut environment. 
To further identify genomic regions unique to this strain, a BLAST-based comparative analysis with the sequenced genomes of 47 members belonging to genus \textit{Paenibacillus} was carried out. 
Unique regions coding for nucleic acid modifying enzymes like CRISPR/Cas and Type I Restriction-Modification enzymes were identified in \textit{P. polymyxa} A18 genome suggesting the presence of defense mechanism to combat viral infections in the gut. 
In addition, genes responsible for the formation of biofilms, such as Type IV pili and adhesins, which might be assisting \textit{P. polymyxa} A18 in colonizing the gut were also identified in its genome. 
In situ colonization experiment further confirmed the ability of \textit{P. polymyxa} A18 to colonize the gut of termite. 
\par
\textbf{Korb \textit{et al.}, 2015, A genomic comparison of two termites with different social complexity.} \newline
The two termite genomes represent two fundamental termite life types: the wood-dwelling one-piece nesters and the central place foraging lineages that generally differ in social complexity, feeding ecology, gut symbionts, and developmental plasticity. 
\textit{Zootermopsis nevadensis} belongs to the former type and \textit{Macrotermes natalensis} to the latter. 
\par
Wood-dwelling species nest within a single piece of dead wood that serves both as food and nesting habitat so the termites never leave their nest to forage. 
This social syndrome is widely considered to be ancestral and associated with high degrees of developmental plasticity for the individual termites. 
Workers remain totipotent immatures throughout several instars that commonly develop further into sterile soldiers, winged sexuals (alates) that found new nests as primary reproductives, or neotenic reproductives that reproduce within the natal nest. 
\textit{Zootermopsis nevadensis} dwells in decaying dampwood. 
\par
The foraging termite species (also called “multiple piece nesters”) forage for food outside the nest at some point after colony foundation and bring it back to the colony to feed nestmates. 
They represent more than 85\% of the extant termite species. 
They have true workers and an early separation into distinct developmental pathways. 
In the apterous line, individuals are unable to develop wings and can thus never disperse as reproductives. They become workers and soldiers, but can in some species also advance to become neotenic reproductives in their own nest. 
In the nymphal line, however, individuals develop wings and dispersing phenotypes that found new colonies elsewhere. 
The Macrotermitinae to which \textit{Macrotermes natalensis} belongs are special examples of foraging termites because their colonies are dependent on nutrition provided by a \textit{Termitomyces} symbiont (Basidiomycota: Agaricales). 
This fungal symbiosis is evolutionarily derived and comes in addition to more fundamental protist (lower termites) and bacterial gut symbionts (all termites), which have played major roles throughout termite evolution. 
\textit{Macrotermes} species have two (major/minor) worker castes and two (major/minor) soldier castes that may be determined as early as the egg stage (suggested for \textit{Macrotermes michaelseni}). 
Macrotermes colonies often build conspicuous mounds that may harbor several millions of individuals.
\par
All immune signaling pathways were found. 
First, \textit{Zootermopsis nevadensis} has 6 gram-negative binding proteins (GNBPs), whereas only four of these were recovered in \textit{Macrotermes natalensis}. 
These four GNBPs are all termite-specific and some of them were previously shown to be under positive selection in several \textit{Nasutitermes} species, especially in species with arboreal nests. 
The \textit{Macrotermes} genome seems to lack the insect-typical GNBP duplicate and one GNBP gene that has so far only been found in \textit{Zootermopsis nevadensis}. 
Second, while AMPs were not enriched in either termite genome, their identities were completely different with \textit{Z. nevadensis} having 2 AMPs and \textit{M. natalensis} having 3 other AMPs. 
\textit{M. natalensis} has a termite-specific defensin-like gene termicin, a category of genes that seem to have duplicated repeatedly during the radiation of \textit{Nasutitermes} termites. 
After duplication, one copy seems to often be under strong selection, while the other evolves toward neutrality. 
Also in the soil-foraging \textit{Reticulitermes} species these genes seem to be under positive selection.
\par
In contrast to other insects where AMP production is normally induced, these genes seem to be constitutively expressed in fungus-growing termites, as has been shown for \textit{Pseudacanthotermes spiniger}, which might be an adaptation to protect the symbiont against competing fungi. 
Termicin and other defensins were absent in \textit{Z. nevadensis} but this species has GNBPs that are differentially expressed between castes (Terrapon et al., 2014) and may thus serve a similar function in protecting the nest from fungal infections. 
For the arboreal nesting termite \textit{Nasutitermes corniger} it has been shown that GNBP2 has (1,3)-glucanase effector activity and functions as an antifungal agent (Bulmer et al., 2009). 
It is incorporated in the nest building material, where it cleaves and releases pathogenic components while priming termites for improved antimicrobial defense (Bulmer et al., 2009). 
Such a defensive strategy is likely to be most effective for termites with closed nests, consistent with positive selection on GNBP being most pronounced in Nasutitermes that live in arboreal nests (Bulmer and Crozier, 2006). 
Hence, antifungal stategies might differ in termites with different habitats; with GNBPs and termicin possibly playing complementary roles. 
This is supported by the fact that GNBPs in subterranean, foraging Reticulitermes species evolve neutrally while termicin was shown to have been under strong positive selection in these species.
\par
We can reject the possible alternative hypothesis that different defense strategies are linked to the gut symbionts that need different defense strategies to protect the symbiotic partner. 
As lower termites harbor protists as well as bacteria, while higher termites only have bacteria, we would then have expected higher termites having more AMPs and lower termites more GNBPs, but this is not the case because lower Reticulitermes termites have positively selected termicins. 
If there is an association between nesting habit and defense strategy, we expect that GNBPs are under positive selection in other wood-dwelling termites, and termicins are selected in soil-foraging termites. 
Additional genomic data, particularly for wood-dwelling termites, would be needed to validate this hypothesis.
\par
Reduced numbers of immune defense genes were found in ants and the honeybee (Evans et al., 2006; Gadau et al., 2012) but also here there seems to be selection on some of the AMP genes. Similar to termicin, positive selection was detected on defensin in ants (Viljakainen and Pamilo, 2008), but this gene was not overexpressed after experimental fungal infections of leaf-cutting ant colonies, whereas two other AMPs were (Yek et al., 2013). This contrasts with dipterans (Drosophila and Anopheles) for which no evidence was found for positive selection on any AMPs (Sackton et al., 2007; Simard et al., 2007), but instead for immune recognition and signaling proteins (Schlenke and Begun, 2003; Jiggins and Kim, 2005; Sackton et al., 2007). This provides further support for the hypothesis that social insects have responded differently to selection pressure caused by microbial pathogens than solitary insects (Viljakainen and Pamilo, 2008).
\par
The ancestral termite gut microbiota was derived from a cockroach ancestor, but major subsequent changes occurred, most notably when the higher termites evolved (Dietrich et al., 2014). The guts of the wood-dwelling termites are dominated by protists that appear to be primarily adapted to break down wood (Cleveland, 1923; Brugerolle and Radek, 2006), with complementary roles of bacteria that are often symbiotic with the gut-flagellates (Dietrich et al., 2014). The common ancestor of the evolutionarily derived Termitidae lost these flagellate symbionts so their gut microbiotas became dominated by bacteria, which may have facilitated their dietary diversification (Brune and Ohkuma, 2011; Dietrich et al., 2014). The single origin of fungiculture by the Macrotermitinae led to Termitomyces taking over primary plant decomposition and the gut microbiota shifting phylogenetically and functionally to perform complementary roles (Liu et al., 2013; Dietrich et al., 2014; Otani et al., 2014; Poulsen et al., 2014).
\par
Changes in symbiont associations are tightly associated with termite life styles (for a recent review on termite gut symbionts, see Brune, 2014), but this may hardly induce structural genomic changes in the termite hosts, consistent with the similar gene repertoires for plant biomass decomposition found in the two termite genomes (Poulsen et al., 2014). A comparison of carbohydrate-active enzyme (CAZy) profiles of the two termite species showed a reduction in the absolute number of glycoside hydrolase enzymes (85) in M. natalensis compared to Z. nevadensis (97) (Table 5), but very similar relative abundances of specific enzyme families (Poulsen et al., 2014). Profile similarities suggest that plant-biomass decomposition genes may be ancestrally conserved across the termites, but additional termite genomes are needed to shed light on this. Such additional genomic work will need to be accompanied by enzyme function validations to test whether differences in absolute numbers reflect changes in the relative importance of termite-derived enzymes.
\par
\textbf{Poulsen \textit{et al.}, 2013, Complementary symbiont contributions to plant decomposition in a fungus-farming termite.} \newline
We obtained high-quality annotated draft genomes of the termite \textit{Macrotermes natalensis}, its \textit{Termitomyces} symbiont, and gut metagenomes from workers, soldiers, and a queen. 
We show that members from 111 of the 128 known glycoside hydrolase families are represented in the symbiosis, that \textit{Termitomyces} has the genomic capacity to handle complex carbohydrates, and that worker gut microbes primarily contribute enzymes for final digestion of oligosaccharides. 
This apparent division of labor is consistent with the \textit{Macrotermes} gut microbes being most important during the second passage of comb material through the termite gut, after a first gut passage where the crude plant substrate is inoculated with Termitomyces asexual spores so that initial fungal growth and polysaccharide decomposition can proceed with high efficiency. 
Complex conversion of biomass in termite mounds thus appears to be mainly accomplished by complementary cooperation between a domesticated fungal monoculture and a specialized bacterial community. 
In sharp contrast, the gut microbiota of the queen had highly reduced plant decomposition potential, suggesting that mature reproductives digest fungal material provided by workers rather than plant substrate.
\par
\textbf{Ali \textit{et al.}, 2019, Symbiotic cellulolytic bacteria from the gut of the subterranean termite \textit{Psammotermes hypostoma} Desneux and their role in cellulose digestion.} \newline
The subterranean termite \textit{Psammotermes hypostoma} Desneux is considered as an important pest that could cause severe damage to buildings, furniture, silos of grain and crops or any material containing cellulose. 
This species of termites is widespread in Egypt and Africa. 
The lower termite’s ability to digest cellulose depends on the association of symbiotic organisms gut that digest cellulose (flagellates and bacteria). 
In this study, 33 different bacterial isolates were obtained from the gut of the termite \textit{P. hypostoma} which were collected using cellulose traps. 
Strains were grown on carboxymethylcellulose (CMC) as a sole source of carbon. 
Cellulolytic strains were isolated in two different cellulose medium (mineral salt medium containing carboxymethylcellulose as the sole carbon source and agar cellulose medium). 
Five isolates showed significant cellulolytic activity identified by a Congo red assay which gives clear zone. 
Based on biochemical tests and sequencing of 16s rRNA genes these isolates were identified as Paenibacillus lactis, Lysinibacillus macrolides, Stenotrophomonas maltophilia, Lysinibacillus fusiformis and Bacillus cereus.
\par
\textbf{Peppel \textit{et al.}, 2021, Ancestral predisposition toward a domesticated lifestyle in the termite-cultivated fungus \textit{Termitomyces}} \newline
The events leading toward domestication of \textit{Termitomyces} remain unclear. 
To address this, we reconstructed the lifestyle of the common ancestor of \textit{Termitomyces} using a combination of ecological data with a phylogenomic analysis of 21 related non-domesticated species and 25 species of \textit{Termitomyces}. 
We show that the closely related genera \textit{Blastosporella} and \textit{Arthromyces} also contain insect-associated species. 
Furthermore, the genus \textit{Arthromyces} produces asexual spores on the mycelium, which may facilitate insect dispersal when growing on aggregated subterranean fecal pellets of a plant-feeding insect. 
The sister-group relationship between \textit{Arthromyces} and \textit{Termitomyces} implies that insect association and asexual sporulation, present in both genera, preceded the domestication of \textit{Termitomyces} and did not follow domestication as has been proposed previously. 
Specialization of the common ancestor of these two genera on an insect-fecal substrate is further supported by similar carbohydrate-degrading profiles between Arthromyces and Termitomyces. 
We describe a set of traits that may have predisposed the ancestor of Termitomyces toward domestication, with each trait found scattered in related taxa outside of the termite-domesticated clade. 
This pattern indicates that the origin of the termite-fungus symbiosis may not have required large-scale changes of the fungal partner.
\par
\textbf{Bulmer \textit{et al.}, 2006, Variation in Positive Selection in Termite GNBPs and Relish.} \newline
We used different models of nucleotide substitution at nonsynonymous (amino acid altering) and synonymous (silent) sites to compare the different levels and type of selection among three immunity genes in 13 Australian termite species (\textit{Nasutitermes}). 
The immunity genes include two encoding pathogen recognition proteins (gram-negative bacterial-binding proteins) that duplicated and diverged before or soon after the evolution of the termites and a transcription factor (Relish), which induces the production of antimicrobial peptides. 
A comparison of evolutionary models that assign four unrestricted classes of dN/dS (the ratio of the nonsynonymous to synonymous substitution rate) to different \textit{Nasutitermes} lineages revealed that the occurrence of positive selection (dN/dS > 1) varies among lineages and the three genes. 
Positive selection appears to have driven the evolution of all three genes in an ancestral lineage of three subterranean termites. 
It had previously been suggested that there was a transition along this ancestral lineage to termite morphology and ecology associated with a diet of decayed wood, a diet that may expose termites to elevated levels of fungal and bacterial pathogens. 
Relish appears to have experienced the highest levels of selective pressure for change among all three genes. 
Positively selected sites in the molecule are located in regions that are important for its activation, which suggests that amino acid substitutions at these sites are a counter response to pathogen mechanisms that disrupt the activation of Relish.
\par
\textbf{Shijenobu \textit{et al.}, 2022, Genomic and transcriptomic analyses of the subterranean termite Reticulitermes speratus: Gene duplication facilitates social evolution.} \newline
We report the genome, transcriptome, and methylome of the Japanese subterranean termite \textit{Reticulitermes speratus}. 
Our analyses highlight the significance of gene duplication in social evolution in this termite. 
Gene duplication associated with caste-biased gene expression was prevalent in the \textit{R. speratus} genome. 
The duplicated genes comprised diverse categories related to social functions, including lipocalins (chemical communication), cellulases (wood digestion and social interaction), lysozymes (social immunity), geranylgeranyl diphosphate synthase (social defense), and a novel class of termite lineage-specific genes with unknown functions. 
Paralogous genes were often observed in tandem in the genome, but their expression patterns were highly variable, exhibiting caste biases. 
Some of the assayed duplicated genes were expressed in caste-specific organs, such as the accessory glands of the queen ovary and the frontal glands of soldier heads. 
We propose that gene duplication facilitates social evolution through regulatory diversification, leading to caste-biased expression and subfunctionalization and/or neofunctionalization conferring caste-specialized functions.
\par
\textbf{Brune & Dietrich, 2015, The Gut Microbiota of Termites: Digesting the Diversity in the Light of Ecology and Evolution.} \newline
\par
\textbf{Mullins \textit{et al.}, 2021, Soil organic matter is essential for colony growth in subterranean termites.} \newline
\textit{Coptotermes} colony development is correlated with diet nitrogen availability. 
Colonies with organic-rich soil diet develop faster than those lacking organic matters. 
Removing diet nitrogen arrests colony development. 
Change in diet nitrogen does not impacting nitrogenase expression of termite gut microbiome. 
\par
\textbf{Spears & Ueckert, 1976, Survival and Food Consumption by the Desert Termite \textit{Gnathamitermes tubiformans} in Relation to Dietary Nitrogen Source and Levels} \newline
Survival and consumption of artificial diets decreases with increase of dietary non-protein nitrigen. 
As dietary amino acid increases, survival decreases slightly, but comsumption increases. 
Survival and comsumption is lowest on pure cellolose diet. 
\par
\textbf{Machida \textit{et al.}, 2001, Nitrogen recycling through proctodeal trophallaxis in the Japanese damp-wood termite \textit{Hodotermopsis japonica} (Isoptera, Termopsidae)}
Young instars and soilders tend to get proctodeal fluid from old instars. 
Nitrogen-deficient individuals have more proctodeal trophallaxis than controls. 
The nitrogen content in the proctodeal fluid of nitrogen-deficient individuals increased after trophallactic interactions with well-nourished individuals.
AProctodeal fluid of low protein content was frequently exchanged among individuals in nitrogen-poor conditions, while high-protein proctodeal fluid was transferred less frequently under nitrogen-rich conditions. 
\par
\textbf{Shellman-Reeve, 1990, Dynamics of biparental care in the dampwood termite, \textit{Zootermopsis nevadensis} (Hagen): response to nitrogen availability} \newline
Dietary nitrogen availability impacts reproduction and labor division of nest-founding pairs. 
\par
\textbf{Brent & Traniello, 2002, Effect of Enhanced Dietary Nitrogen on Reproductive Maturation of the Termite Zootermopsis angusticollis (Isoptera: Termopsidae).} \newline
Supplemented dietary nitrogen results primaries and neitenics gaining less body mass. 
It also causes females excrete uric acid instead of storing it. 
Although enhancing dietary nitrogen may release newly molted neotenics from nutritional limitations on their fecundity, dietary enhancement with 0.05\% uric acid does not significantly effect the reproductive development of recently dealated primaries. 
\par
\textbf{Nishimura \textit{et al.}, 2020, Division of functional role for termite gut protists revealed by single-cell transcriptomes} \newline
Single cell transcriptomes of four gut protist species of \textit{Coptotermes formosanus} were sequenced. 
Expression pattern of wood-digestion genes differs across protist species. 
One protist, \textit{Cononympha leidyi} degrades chitin and assimilates it into amino acids. 
Two genes for chitin degradation originated from horizontal gene transfer from fungi and Firmicute bacteria.
\par
\textbf{Koshikawa \textit{et al.}, 2008, Genome size of termites (Insecta, Dictyoptera, Isoptera) and wood roaches (Insecta, Dictyoptera, Cryptocercidae).} \newline
C value of \textit{Hodotermopsis sjostedti} and \textit{Zootermopsis nevadensis} differed by a factor of two, indicating polypoidy.
\par

\end{sloppypar}
\end{document}