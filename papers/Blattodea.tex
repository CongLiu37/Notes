\documentclass[11pt]{article}

% \usepackage[UTF8]{ctex} % for Chinese 

\usepackage{setspace}
\usepackage[colorlinks,linkcolor=blue,anchorcolor=red,citecolor=black]{hyperref}
\usepackage{lineno}
\usepackage{booktabs}
\usepackage{graphicx}
\usepackage{float}
\usepackage{floatrow}
\usepackage{subfigure}
\usepackage{caption}
\usepackage{subcaption}
\usepackage{geometry}
\usepackage{multirow}
\usepackage{longtable}
\usepackage{lscape}
\usepackage{booktabs}
\usepackage{natbibspacing}
\usepackage[toc,page]{appendix}
\usepackage{makecell}
\usepackage{amsfonts}
 \usepackage{amsmath}

\usepackage[backend=bibtex,style=authoryear,sorting=nyt,maxnames=1]{biblatex}
\bibliography{} %Reference bib

\title{Abstracts: Blattodea}
\author{}
\date{}

\linespread{1.5}
\geometry{left=2cm,right=2cm,top=2cm,bottom=2cm}

\setlength\bibitemsep{0pt}

\begin{document}
\begin{sloppypar}
  \maketitle

  \linenumbers
\textbf{Bucek \textit{et al.}, 2019, Evolution of Termite Symbiosis Informed by Transcriptome-Based Phylogenies.} \newline
We sequenced genomes and transcriptomes of 55 termite species and reconstructed phylogenetic trees from up to 4,065 orthologous genes of 68 species. 
We found strong support for a novel sister-group relationship between the bacterial comb-building Sphaerotermitinae and fungus comb-building Macrotermitinae. 
This key finding indicates that comb building is a derived trait within Termitidae and that the creation of a comb-like “external rumen” involving bacteria or fungi may not have driven the loss of protozoa from ancestral termitids, as previously hypothesized. 
Instead, associations with gut prokaryotic symbionts, combined with dietary shifts from wood to other plant-based substrates, may have played a more important role in this symbiotic transition. 
\par
\textbf{Bourguignon \textit{et al.}, 2017, Mitochondrial Phylogenomics Resolves the Global Spread of Higher Termites, Ecosystem Engineers of the Tropics.} \newline

\par

\textbf{He \textit{et al.}, 2021, Evidence for reduced immune gene diversity and activity during the evolution and activity during the evolution of termites.} \newline
18 cockroach and termite species, including 9 lower termites, 6 higher termites, 2 subsocial wood roaches and 2 solitary roaches were studied via \textit{de novo} assembled transcriptomes. 
Each immune gene family presented in all species, except antifungal dorsomycin, which was lost in termites and wood roaches. 
Phylogenetic signal analysis revealed a loss of total immune gene diversity during termite evolution. 
\par
Antifungal peptide drosomycin was lost in ancestor of wood roach \textit{Cryptocercus} and termites. 
C-type lectin (CTL) underwent two contractions in most recent common ancestor (MRCA) of (1) wood roach \textit{Cryptocercus} and termites; and (2) Rhinotermitidae and Termitidae, but together with lysozymes, re-expanded in late branch of higher termites, \textit{i.e.} MRCA of \textit{Promirotermes} and \textit{Dicuspiditermes}. 
Serine protease CLIP contracted in MRCA of Rhinotermitidae and Termitidae. 
Thiroredoxin peroxidase (TPX) and autophagy-related gene (ATG) contracted in Termitidae, while defensin expanded. 
\par
In bees, immune gene depletion seems to have preceded evolution of eusociality (Barribeau \textit{et al.}, 2015), indicating immune gene depletion is unrelated with transition to sociality. 
Although there was contraction of immune genes in termite evolution, it can be interpreted as an expansion of immune genes in solitary cockroaches (Harrison \textit{et al.}, 2018) followed by returning to a representative level.
\par
\textbf{Sabree \textit{et al.}, 2009, Nitrogen recycling and nutritional provisioning by \textit{Blattabacterium}, the cockroach endosymbiont.} \newline
Cockroaches, unlike most terrestrial insects, excrete waste nitrogen within their fat bodies as uric acids, postulated to be a supplement when dietary nitrogen is limited. 
The fat bodies of most cockroaches are inhabited by \textit{Blattabacterium}, which are vertically transmitted, Gram-negative bacteria that have been hypothesized to participate in uric acid degradation, nitrogen assimilation, and nutrient provisioning. 
We have sequenced completely the \textit{Blattabacterium} genome from \textit{Periplaneta americana}. 
Genomic analysis confirms that \textit{Blattabacterium} is a member of the Flavobacteriales (Bacteroidetes), with its closest known relative being the endosymbiont \textit{Sulcia muelleri}, which is found in many sap-feeding insects. 
Metabolic reconstruction indicates that it lacks recognizable uricolytic enzymes, but it can recycle nitrogen from urea and ammonia (uric acid degradation products) into glutamate, using urease and glutamate dehydrogenase. 
Subsequently, \textit{Blattabacterium} can produce all of the essential amino acids, various vitamins, and other required compounds from a limited palette of metabolic substrates. 
The ancient association with \textit{Blattabacterium} has allowed cockroaches to subsist successfully on nitrogen-poor diets and to exploit nitrogenous wastes, capabilities that are critical to the ecological range and global distribution of cockroach species.
\par
\textbf{Neef \textit{et al.}, 2011, Genome Economization in the Endosymbiont of the Wood Roach \textit{Cryptocercus punctulatus} Due to Drastic Loss of Amino Acid Synthesis Capabilities.} \newline
Cockroaches (Blattaria: Dictyoptera) harbor the endosymbiont \textit{Blattabacterium} in their abdominal fat body. 
The genome of \textit{Blattabacterium sp.} of \textit{Cryptocercus punctulatus} (BCpu) was sequenced and compared with those of the symbionts of \textit{Blattella germanica} and \textit{Periplaneta americana}, BBge and BPam, respectively. 
The BCpu genome consists of a chromosome of 605.7 kb and a plasmid of 3.8 kb and is therefore approximately 31 kb smaller than the other two aforementioned genomes. 
The size reduction is due to the loss of 55 genes, 23 of which belong to biosynthetic pathways for amino acids. 
The pathways for the production of tryptophan, leucine, isoleucine/threonine/valine, methionine, and cysteine have been completely lost. 
Additionally, the genes for the enzymes catalyzing the last steps of arginine and lysine biosynthesis, argH and lysA, were found to be missing and pseudogenized, respectively. 
These gene losses render BCpu auxotrophic for nine amino acids more than those corresponding to BBge and BPam. 
BCpu has also lost capacities for sulfate reduction, production of heme groups, as well as genes for several other unlinked metabolic processes, and genes present in BBge and BPam in duplicates. 
Amino acids and cofactors that are not synthesized by BCpu are either produced in abundance by hindgut microbiota or are provisioned via a copious diet of dampwood colonized by putrefying microbiota, supplying host and \textit{Blattabacterium} symbiont with the necessary nutrients and thus permitting genome economization of BCpu.
\par
\textbf{Kinjo \textit{et al.}, 2018, Parallel and Gradual Genome Erosion in the \textit{Blattabacterium} Endosymbionts of \textit{Mastotermes darwiniensis} and \textit{Cryptocercus} Wood Roaches.} \newline
Almost all examined cockroaches harbor an obligate intracellular endosymbiont, \textit{Blattabacterium cuenoti}. 
On the basis of genome content, \textit{Blattabacterium} has been inferred to recycle nitrogen wastes and provide amino acids and cofactors for its hosts. 
Most \textit{Blattabacterium} strains sequenced to date harbor a genome of ∼630 kbp, with the exception of the termite \textit{Mastotermes darwiniensis} (∼590 kbp) and \textit{Cryptocercus punctulatus} (∼614 kbp), a representative of the sister group of termites. 
Such genome reduction may have led to the ultimate loss of \textit{Blattabacterium} in all termites other than Mastotermes. 
In this study, we sequenced 11 new \textit{Blattabacterium} genomes from three species of \textit{Cryptocercus} in order to shed light on the genomic evolution of \textit{Blattabacterium} in termites and \textit{Cryptocercus}. 
All genomes of \textit{Cryptocercus}-derived \textit{Blattabacterium} genomes were reduced (∼614 kbp), except for that associated with \textit{Cryptocercus kyebangensis}, which comprised 637 kbp. 
Phylogenetic analysis of these genomes and their content indicates that \textit{Blattabacterium} experienced parallel genome reduction in Mastotermes and \textit{Cryptocercus}, possibly due to similar selective forces. 
We found evidence of ongoing genome reduction in \textit{Blattabacterium} from three lineages of the \textit{C. punctulatus} species complex, which independently lost one cysteine biosynthetic gene. 
We also sequenced the genome of the \textit{Blattabacterium} associated with \textit{Salganea taiwanensis}, a subsocial xylophagous cockroach that does not vertically transmit gut symbionts via proctodeal trophallaxis. 
This genome was 632 kbp, typical of that of nonsubsocial cockroaches. 
Overall, our results show that genome reduction occurred on multiple occasions in \textit{Blattabacterium}, and is still ongoing, possibly because of new associations with gut symbionts in some lineages.
\par
\textbf{Dietrich \textit{et al.}, 2014, The Cockroach Origin of the Termite Gut Microbiota: Patterns in Bacterial Community Structure Reflect Major Evolutionary Events.} \newline
16S profiled gut bacterial communities of 34 termite and cockroach species. 
Generally, gut bacterial communities of cockroaches, lower termites and higher termites were clearly separated. 
Subfamilies of higher termites formed distinct clusters. 
Fungus-cultivating Macrotermitinae (higher termites) showed strong affinity for the cockroaches.
Wood-feeding cockroach \textit{Cryptocercus punctulatus}, the closest relative of termites, was close to lower termites, with which it shares the presence of cellulolytic flagellates. 
Other wood-feeding cockroaches \textit{Panesthia angustipennis} and \textit{Salganea esakii} (family Blaberidae), whose gut microbiota lacks such flagellates, clustered with the omnivorous cockroaches. 
When mapped onto the host tree, the changes in community structure coincided with major events in termite evolution, such as acquisition and loss of cellulolytic protists and the ensuing dietary diversification. 
UniFrac analysis of the core microbiota of termites and cockroaches and construction of phylogenetic tree of individual genus level lineages revealed a general host signal, whereas the branching order often did not match the detailed phylogeny of the host. 
It remains unclear whether the lineages in question have been associated with the ancestral cockroach since the early Cretaceous (cospeciation) or are diet-specific lineages that were independently acquired from the environment (host selection).
\par
\textbf{Warnecke \textit{et al.}, 2007, Metagenomic and functional analysis of hindgut microbiota of a wood-feeding higher termite.} \newline
Only recently have data supported any direct role for the symbiotic bacteria in the gut of the termite in cellulose and xylan hydrolysis. 
Here we use a metagenomic analysis of the bacterial community resident in the hindgut paunch of a wood-feeding ‘higher’ Nasutitermes species (which do not contain cellulose-fermenting protozoa) to show the presence of a large, diverse set of bacterial genes for cellulose and xylan hydrolysis. 
Many of these genes were expressed in vivo or had cellulase activity in vitro, and further analyses implicate spirochete and fibrobacter species in gut lignocellulose degradation. 
New insights into other important symbiotic functions including H2 metabolism, CO2-reductive acetogenesis and N2 fixation are also provided by this first system-wide gene analysis of a microbial community specialized towards plant lignocellulose degradation.
\par
\textbf{Bourguignon \textit{et al.}, 2014, Rampant Host Switching Shaped the Termite Gut Microbiome.} \newline
16S metabarcoding profiled gut bacterial community of 94 termite species (77 higher termites). 
211 lineages containing > 10 OTUs were processed by phylogenetic analysis. 
Phylogenetic trees were classified into 3 categories. 
Category 1 represents trees in which ≥30\% of termite-derived sequences formed a monophyletic group. 
In category 1, termite-specific lineages are often composed of bacteria associated with distinct termite lineages, and are sister groups of bacteria living in invertebrate/vertebrate gut. 
It is hypothesize that the last common ancestors of each termite-specific clade became specialized for termite gut environments and eventually became widespread across a large number of termites through both parent-to-offspring vertical transmission and horizontal colony-to-colony transfer between termites. 
Category 2 represents trees containing monophyletic clades composed of >30\% termite-associated bacteria and > 10\% non-termite-associated bacteria which are often living in guts of other animals. 
It is hypothesized that these bacterial lineages have been specialized for termite gut environments and became widespread through vertical transmission and horizontal transmission between termites and other animals. 
Category 3 represents rest trees in which termite-associated bacteria intersperse with bacteria from other environments. 
Congruence of bacterial and termite phylogeny did not find evidence for strict vertical transmission. 
The results indicate that “mixed-mode” transmission, which combines colony-to-offspring vertical transmission with horizontal colony-to-colony transfer, has been the primary driving force shaping the gut bacterial community of termites.
\textbf{Lo & Evans, 2006, Phylogenetic diversity of the intracellular symbiont \textit{Wolbachia} in termites}
44  termite populations, representing of a total of 30 species, 14 genera, and 6 families were screened for the presence of \textit{Wolbachia}, using PCR assays for the genes 16S rRNA and ftsZ. 
12 out of 44 populations were found to be infected. 
\textbf{Sabree \textit{et al.}, 2012, Genome Shrinkage and Loss of Nutrient-Providing Potential in the Obligate Symbiont of the Primitive Termite \textit{Mastotermes darwiniensis}}

\end{sloppypar}
\end{document}