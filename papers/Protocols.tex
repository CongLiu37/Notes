\documentclass[11pt]{article}

% \usepackage[UTF8]{ctex} % for Chinese 

\usepackage{setspace}
\usepackage[colorlinks,linkcolor=blue,anchorcolor=red,citecolor=black]{hyperref}
\usepackage{lineno}
\usepackage{booktabs}
\usepackage{graphicx}
\usepackage{float}
\usepackage{floatrow}
\usepackage{subfigure}
\usepackage{caption}
\usepackage{subcaption}
\usepackage{geometry}
\usepackage{multirow}
\usepackage{longtable}
\usepackage{lscape}
\usepackage{booktabs}
\usepackage{natbibspacing}
\usepackage[toc,page]{appendix}
\usepackage{makecell}
\usepackage{amsfonts}
 \usepackage{amsmath}

\usepackage[backend=bibtex,style=authoryear,sorting=nyt,maxnames=1]{biblatex}
\bibliography{} %Reference bib

\title{Abstracts: Blattodea}
\author{}
\date{}

\linespread{1.5}
\geometry{left=2cm,right=2cm,top=2cm,bottom=2cm}

\setlength\bibitemsep{0pt}

\begin{document}
\begin{sloppypar}
  \maketitle

  \linenumbers
\textbf{Yang \textit{et al.}, 2019, NCBI’s Conserved Domain Database andTools for Protein Domain Analysis.} \newline
Use Standalone RPS-BLAST and rpsbproc to compute and retrieve domain annotation programmatically.
\textbf{Qu \textit{et al.}, 2020, Effect of sequence depth and length in long-read assembly of the maize inbred NC358.}
We have documented how both the completeness and contiguity of assemblies improve with increasing depth and read length. 
With long-read sequencing (PacBio), the low-copy gene space (including tandem gene arrays) can be well assembled with as low as 30x genomic coverage across a range of read lengths. 
Complete characterization of transposon elements in complex genomes such as maize will require a greater depth of sequence (~40×) and should employ library preparation protocols that maximize read-length N50. 
Finally, complete assembly of highly repetitive genomic features such as heterochromatic knobs, telomeres, and centromeres will require substantially more data. 
In fact, complete assembly of these latter highly repetitive sequences will likely require innovations beyond current sequencing technology.

\end{sloppypar}
\end{document}