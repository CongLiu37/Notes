\documentclass[11pt]{article}

% \usepackage[UTF8]{ctex} % for Chinese 

\usepackage{setspace}
\usepackage[colorlinks,linkcolor=blue,anchorcolor=red,citecolor=black]{hyperref}
\usepackage{lineno}
\usepackage{booktabs}
\usepackage{graphicx}
\usepackage{float}
\usepackage{floatrow}
\usepackage{subfigure}
\usepackage{caption}
\usepackage{subcaption}
\usepackage{geometry}
\usepackage{multirow}
\usepackage{longtable}
\usepackage{lscape}
\usepackage{booktabs}
\usepackage{natbib}
\usepackage{natbibspacing}
\usepackage[toc,page]{appendix}
\usepackage{makecell}
\usepackage{amsfonts}
 \usepackage{amsmath}

\title{Failures of host defense mechanisms}
\author{}
\date{}

\linespread{1.5}
\geometry{left=2cm,right=2cm,top=2cm,bottom=2cm}

\begin{document}
\begin{sloppypar}
  \maketitle

  \linenumbers
Protective immune responses do not always success in defense against infection agents, either due to immune defects in an abnormal individual, or to the evasion or subversion of immune defense in normal hosts. 

\section{Immunodeficiency diseases}
Immunodeficiencies occur when components of immune system are defective and are suggested by a history of repeated infections. 
Primary immunodeficiencies are caused by inherited mutations of immune genes. 
Secondary immunodeficiencies are acquired as a consequence of other diseases or environmental factors

\subsection{Immunodeficiency due to disturbance of T and B cell development}
Defects in T-cell development can cause severe combined immunodeficiencies (SCID), leading to susceptibility to virtually all antigens. 
Such patients exhibit neither T-cell-dependent antibody responses nor cell-mediated immune responses, and thus cannot develop immunological memory. 
X-linked SCID (XSCID) is the most frequent form of SCID and is caused by mutations in \textit{IL2RG} on human X chromosome. 
\textit{IL2RG} encodes interleukin-2 receptor (IL-2R) common gamma chain ($\gamma_c$), which is required in all IL-2 cytokine family (IL-2, -4, -7, -9, -15, -21). 
Therefore, \textit{IL2GR} inactivation causes defects in signaling of IL-2 family cytokines. 
Owing to lack of IL-7 and IL-15, T cells and NK cells fail to develop normally, and due to absence of T-cell help, B cells are not functioning although B-cell development is not impacted. 
XSCID is clinically and immunologically indistinguishable with SCID caused by inactivating mutation in kinase Jak3, a downstream member of $\gamma_c$ signaling. 
Mice with inactivated $\beta_c$ gene \textit{IL2RB} defined a key role for IL-15 as a growth factor for development of T cells and NK cells. 
Mice with inactivated IL-15 or the $\alpha$ chain of its receptor have no NK cells and limited maintenance of memory CD8 T cells. 
Humans with deficiency of IL-7 receptor $\alpha$ chain have no T cells, but normal in NK cell and B cell level. 
However, in mice, IL-7 receptor deficiency causes lack of both T cells and B cells. 
In humans and mice with defective IL-2 production, most T-cell development is normal, although there is impaired development of FoxP3$^+$ T$_{reg}$ cells that predisposed to immune regulatory abnormality and autoimmunity. 

\par

Autosomal recessive SCID can be due to defects in enzymes of the salvage pathway of purine synthesis including adenosine deaminase (ADA) deficiency and purine nucleotide phosphorylase (PNP) deficiency. 
ADA converts adenosine to inosine, and deoxyadenosine to deoxysdenosine. 
ADA deficiency causes accumulation of deoxyadenosine and its precursor S-adenosylhomocysteine, which are toxic to developing T and B cells. 
PNP converts inosine to hypoxanthine, and guanosine to guanine. 
PNP deficiency, as a rare form of SCID, causes accumulation of toxic precursors but affects T cells more severely than B cells.

\par

SCID can be due to defects in antigen receptor gene rearrangement. 
Defects of \textit{RAG1} or \textit{RAG2} causes failure of V(D)J recombination and thus, arrests lymphocyte development at pro- to pre-T cell and B cell transition. 
As a result, T and B cells are completely absent, but NK cell development is normal. 
Hypomorphic mutations in \textit{RAG1}/\textit{RAG2} cause reduced, but not absent, gene functions, and lead to Omenn syndrome. 
In Omenn syndrome patients, T cell numbers are normal, but repertoires of T cells are highly restricted, and B cells are absent. 
Deficiency of genes for repairing DNA double-strand breaks including Artemis, DNA protein-kinase catalytic subunit (DNA-PKcs) and DNA ligase IV cause abnormal sensitivity to ionizing radiation and radiation-sensitive SCID (RS-SCID). 
These patients produce very few B and T cells. 
Only rare VJ or VDJ joints are seen, and most are abnormal. 

\par

Deficiencies in signaling from T-cell receptors (TCRs) can cause SCID, as they block T cell activation early in thymic development. 
Deficiencies in CD3$\delta$, CD3$\epsilon$ or CD3$\zeta$ chains of CD3 complex causes defective pre-T-cell signaling and fail to process to double-positive stage. 
Inactivation of tyrosine phosphatase CD45 also causes SCID, leading to T cell reduction and abnormal B cell maturation. 
Cytosolic protein tyrosine kinase ZAP-70 transmits signals from T-cell receptor. 
Its deficiency causes SCID, where CD8 T cells are absent, while CD4 T cells are normal in number but not functioning. 
Inactivation of WAS protein (WASp) causes Wiskott-Aldrich syndrome (WAS). 
WASp is at the downstream of several TCR signaling pathways. 
It activates Arp2/3 complex thus, induces reorganization of cytoskeleton, which is essential for immune synapse formation and release of effector molecules by T cells. 

\par

SCID can be due to genetic defects in thymic function that block T cell development. 
Mutations in transcription factor FOXN1 cause failure of thymic epithelium differentiation and formation of thymus. 
The absent of thymic function prevents T cell development. 
B cell development is normal, but B cell functions are absent due to lack of T cells. 
DiGeorge syndrome is caused by deletion of transcription factor gene \textit{TBX1}. 
Patients are haploinsufficient for TBX1, which causes abnormal development of thymic epithelium. 

\par

Deficiencies in MHC molecule expression can cause SCID by disturbing positive selection in T cell development. 
MHC class II deficiency is caused by mutations in genes required for transcriptional activation of MHC class II genes, including \textit{CIITA}, \textit{RFXANK}, \textit{RFX5} and \textit{RFXAP}. 
As a result, positive selection of CD4 T cells is disturbed. 
MHC class I deficiency is caused by mutations in \textit{TAP1}, \textit{TAP2} or \textit{TAPBP}. 
\textit{TAP1/2} and \textit{TAPBP} encode components of peptide transporter complex that transports peptide synthesized in cytosol into endoplasmic reticulum. 
As a result of deficiency in \textit{TAP1}, \textit{TAP2} or \textit{TAPBP}, MHC class I fails to locate on plasma membrane, disturbing CD8 T cell development. 
However, there is evidence that pathways dependent on \textit{TAP1}, \textit{TAP2} or \textit{TAPBP} are compensated to allow sufficient development of cytotoxic CD8 T cells to control viruses. 

\par

Defects in B cell development result in deficiencies in antibody production, and cause inablity to clear extracellular bacteria and some viruses. 
In Bruton's X-linked agammaglobulinemia (XLA), tyrosine kinase BTK is inactivated. 
BTK transduces signals of pre-B cell receptor (BCR) and triggers proliferation and differentiation of pre-B cells. 
Autosomal recessive deficiencies of other components of pre-BCR also block B cell development. 
These genes include that encodes $\mu$ heavy chain (\textit{IGHM}), Ig$\alpha$ (\textit{CD79A}) and Ig$\beta$ (\textit{CD79B}). 
Mutations in BCR signaling adaptor BLNK also disturb B cell development. 

\subsection{Immunodeficiency due to disturbance of T and B cell activation and function}

\subsection{Immunodeficiency due to disturbance of dendritic cell development}

\subsection{Immunodeficiency due to disturbance of complement system}

\subsection{Immunodeficiency due to disturbance of phagocytic cells}

\subsection{Autoinflammation due to mutations in regulators of inflammation}

\subsection{Hematopoietic stem cell transplantation or gene therapy corrects defects in lymphocyte development}
Defects in lymphocyte development can be corrected by replacing the defect component, often by hematopoietic stem cell (HSC) transplantation. 
The main difficulties in HSC transplantation is human leukocyte antigen (HLA) polymorphism. 
HLA allele is expressed by thymic epithelium determine which T cells can be positively selected. 
In HSC transplantation, T cells and antigen presenting cells are derived from graft, while thymic epithelium is derived from host. 
Therefore, unless the host and the graft share some similar HLA, T cells selected by host thymic epithelium cannot be activated by graft-derived antigen-presenting cells. 
Another danger in HSC transplantation is that post-thymic T cells may contaminate donor HSCs and attack host, causing graft-versus-host disease (GVHD). 
Also, in some SCID patients, T cells are reduced but not absent, and can attack graft HSCs, causing host-versus-graft disease (HVGD). 
This can be minimized by deconstructing bone marrow with cytotoxic drugs before HSC transplantation. 

\par

An alternative therapy for immunodeficiencies is somatic gene therapy. 
It includes isolation of HSCs from patients, introduction of a normal copy of the defective gene by a viral vector, and reinfusion of the corrected stem cells. 
Initially, retroviral vectors were used. 
However, the sit where retrovirus inserts into are not under control. 
Insertion into a proto-oncogene induces tumor. 
Retrovirus with a strong promoter can transactivate genes close to insertion site and causes problems. 
More recently, self-inactivating retroviral and lentiviral vectors shows promise as a means of avoiding these problems. 

\subsection{Secondary immunodeficiency}

\section{Evasion and subversion of immune defenses}
Opportunistic pathogens refer to microbes that are normally defeated by a healthy immune system, but cause infection when there are immunodeficiencies. 
True pathogens are microbes than can avoid immune destruction and infect individuals with normal immune defenses, at least long enough to replicate in the infected host and spread to new hosts. 
At one end, pathogens establish an acute infection, replicate quickly and spread to new hosts before being elimination by a successful immune response. 
At the other end, pathogens establish chronic infections and persisting long term in host. 
Pathogens have evolved different strategies to achieve these ends. 

\subsection{Extracellular bacterial pathogens}

\subsection{Intracellular bacterial pathogens}

\subsection{Protozoan parasites}

\subsection{RNA viruses}
Viral genetic materials are recognized by intracellular patern recognition receptors (PRRs) and provoke cytolytic responses by NK cells and CD8 T cells. 
Type I interferon responses are also induced to activate cell-intrinsic mechanisms to limit viral replication. 
Plasmacytoid dendritic cells are specialized for high levels of type I interferon production in early stage of viral infection, and along with NK cells, play a central role in antiviral responses before adaptive immune responses. 
Antiviral adaptive immunity involves induction of T$_H1$ cells that help production of opsonizing and complement-fixing virus specific antibodies that block viral entry into uninfected cells and activate complement to destroy enveloped viruses. 
It also includes cytolytic CD8 T cells that destroy infected cells and produce interferon-$gamma$. 

\par

Antigenic type of RNA viruses evolve rapidly due to unstable genome. 
RNA virus replication requires RNA polymerase that lack the proofreading ability of DNA polymerase. 
Therefore, RNA viruses have a greater mutation rate than DNA viruses, with the practical consequence that RNA viruses cannot support large genome. 
Further, some RNA viruses have segmental genomes enabling reassortment during replication. 

\par

Antigenic drift are caused by point mutations in viral genes. 
Mutations in viral surface proteins can allow viruses to be free from neutralization by antibodies present in host populations. 
Other mutations can affect epitopes that are recognized by T cells, causing cells infected by mutated viruses escape from destruction. 
However, such point mutations change viral proteins in a mirror way, and there is still some cross-reaction with antibodies and memory T cells produced against previous variant. 
Therefore, most of host population still has some level of immunity to new variants. 

\par

Antigenic shift are caused by reassortment of segmental genome of RNA viruses. 
When two variants infect the same cell, genetic material of two variants can be wrapped into a signal viral particle, and thus a new variant is generated. 
Antigenic shift changes viral genome rapidly, causing poor cross-reaction with antibodies and T cells against previous variants. 

\subsection{DNA viruses}
DNA viruses have larger and stabler genomes than RNA viruses, and thus, less able to evade immunity by antigenic drift/shift, but contain a remarkable number of immune evasion genes than enables chronic infection. 

\par

Viral proteins can disturb immune responses by inhibiting antigen presentation. 
Antiviral immunity requires viral peptide:MHC class I molecule complex present at infected cell surface, and thus signals CD8 T cells to kill infected cells. 
Viral protein immunoevasins can prevent viral peptide:MHC class I molecule complex presenting on infected cells. 
Some immunoevasions target at TAP transporter, thus block peptide entry into the endoplasmic reticulum and prevent formation of viral peptide:MHC class I molecule complex. 
Viral immunoevasins can also retain MHC class I molecule in endoplasmic reticulum and thus, prevent viral peptide:MHC class I molecule complex reaching cell surface. 
Several viral proteins catalyze degradation of newly synthesized MHC class I molecules by dislocation, which triggers pathway used to degrade misfolded endoplasmic reticulum proteins by directing them back to cytosol. 
There are also viral proteins targeting at MHC class II pathway and ultimately targeting at CD4 T cells. 

\par

NK cells recognize and cytolysize cells with donw-regulated MHC class I molecules, which is suggestive of attempts to evade detection by cytotoxic T cells. 
Therefore, viruses targeting at MHC class I molecule also repress NK cell activity. 
Strategies include expression of viral homology of MHC class I that engage killer inhibitory receptors (KIRs) and leokocyte inhibitory receptors (LIRs), and thus inhibit NK-cell cytolysis. 
Viral products can also inhibit activating receptors on NK cells and inhibit NK cell effector pathways. 

\par

DNA viruses can enable evasion by targeting at cytokines or chemokines and their receptors. 
Type I and II interferon are major effector cytokines in antiviral defenses, and thus, are central targets for viral evasion. 
Strategies include encoding viral homology receptors or inhibitory proteins, repressing JAK/STAT signaling by interferon receptors, ,inhibiting cytokine transcription, or disturbing transcription factors induced by interferon. 
Some viruses produce antagonists of pro-inflammatory cytokines IL-1, IL-18 and THF-$\alpha$. 
Viral homologs of immunosuppressive cytokines are also produced. 

\par

Some viruses can interfere chemokine responses by encoding chemokine receptors or chemokine homology. 
There are also viruses mediate inhibition of CD8 T cells. 
Under this settings, CD8 T cells express an inhibitory receptor of CD28 superfamily, the programmed death 1 (PD-1) receptor, which is activated by ligand PD-L1 and inhibit CD8 T cell effector function. 

\par

Some latent viruses can establish chronic infection by ceasing to replicate. 
They maintain their genome within the nucleus of infected cells without replication, but establish lysogenic phase by expressing latency associated transcripts (LATs). 
LATs suppress the transcription of the remaining viral genome. 
It also interfere apoptosis of host cell by disturbing immune mechanisms that can clear host cell and expand cell life span. 
Herpes simplex virus (HSV) keeps latent in neurons than carry low levels of MHC class I and are hard to be recognized by CD8 T cells. 
Epstein-Barr virus (EBV) infects B cells and becomes latent in memory B lymphocytes by expressing EBNA1, which interacts with proteasome to prevent its own degradation into peptide that would trigger T cell responses. 
Infected B cells occasionally undergo malignant transformation, giving rise to B cell lymphoma, in which TAP1/2 are down-regulated, inhibiting antigen presentation by MHC class I molecule. 

\section{Acquired immune deficiency syndrome}
Acquired immune deficiency syndrome (AIDS) caused by human immunodeficiency virus (HIV) is the most extreme example of immune subversion by pathogens. 
The syndrome is characterized by progressively loss of CD4 T cells, leading to susceptibility to opportunistic infections and malignancies. 

\subsection{Life history of HIV}
HIV is an enveloped RNA virus infecting immune cells, targeting at CD4 T cells, macrophages and dendritic cells. 
Entry of HIV into cells require trimer complex composed of viral gp120 and gp41. 
gp120 binds with CD4 with high affinity, or with other molecules on surface of macrophage and dendritic cell with less affinity. 
Then gp120 confirmation changes, exposing binding site to co-receptors, often chemokine receptor CCR5 or CXCR4. 
Then gp41 unfolds and inserts into plasma membrane, inducing fusion of viral envelop with plasma membrane and entry of viral genome and proteins. 
After entry, viral reverse transcriptase transcribes viral RNA into cDNA, which further integrated into host chromosomes by viral integrase and thus, form provirus. 
Viral integrase recognizes long terminal repeats (LTRs). 
Transcripts from integrated viral DNA serve as both mRNAs to direct synthesis of viral proteins and as RNA genome of new viral particles that escape from host cell by budding. 

\par

HIV genome contains three major genes. 
\textit{gag} encodes structural proteins of viral nucleocapsid core; 
\textit{pol} encodes reverse transcriptase, integrase and protease; 
\textit{env} encodes gp160, which is cleaved into gp120 and gp41 by host protease. 
mRNAs of \textit{gag} and \textit{pol} are translated into polypeptide chains which are cleaved into individual functional proteins by viral protease. 
There are other six regulatory genes in HIv. 
\textit{Tat} and \textit{Rev} regulate early viral replication cycle, while \textit{Nef}, \textit{Vif}, \textit{Vpr} and \textit{Vpu} are essential for efficient viral production. 

\par

After infection, HIV can complete replication or become latent as provirus. 
Virus infecting a dormant cell tend to be latent. 
In case of long-living CD4 T cells, HIV tends to be latent until T cells are activated. 
While in short-living macrophages and dendritic cells, HIV tends to complete relication. 
Furthermore, HIV provirus requires host cell activation to complete replication, as transactivation of provirus requires host transcription factors, often NF-$\kappa$B and NFAT. 
NF-$\kappa$B is expressed in all immune cells with HIV, while NFAT is primarily activated in CD4 T cells. 
Also, CD4 cells are long-lived and abundant. 
Taking together, CD4 T cells are the major source of HIV replication.

\par

Activation of T cells by antigen induces NF-$\kappa$B and NFAT activation; 
activation of memory T cells by cytokines induces NF-$\kappa$B. 
Thus, provirus can be activated. 
Viral protein Tat binds to transcriptional activation region (TAR) in 5'LTR, recruiting cellular cyclin T1 and cycline dependent kinase 9 (CDK9) to form a complex that phosphorylates RNA polymerase to enhance provirus transcription. 
Viral protein Rev binds to Rev response element (RRE) on unspliced viral transcripts and shttling it out of nucleus. 
Nef, Vif, Vpu and Vpr defeats host immune mechanisms of viral clearance. 
Nef enhances T cell activation by lowering threshod for TCR signaling and down-regulating inhibitory co-simulatory receptor CTLA4. 
Nef down-regulates MHC I/II molecules to avoid infected cells triggering antiviral responses or being killed by cytotoxic T cells. 
It also promotes clearance of CD4 than binds to budding viruses. 
Vif targets at cytidine deaminase APOBEC, which catalyzes deoxycytidine to deoxyuridine in viral cDNA and destroys protein encoding. 
Vpu targets at tetherin, which inserts into plasma membrane and viral envelop, and blocks viral release. 
Vpr targets at SAMHD1, which inhibits HIV infection in myeloid cells and silences CD4 T cells by limiting intracellular pool of deoxynucleotides available for cDNA synthesis. 

\subsection{Routes for HIV transmission and establishment of infection}
HIV is transmitted by exchange of body fluid. 
After entry into a new host, HIV must make contact with CD4 expressing immune cells. 
Before HIV can contact with target cells in genital and intestinal mucosae, it must transverse epithelium of these tissues. 
Molecules expressed by epithelium cells, including CCR5 and gp120-binding glycosphingolipids, foster transcytosis of virus across the epithelium. 
Dendritic cells ramifying across epithelium provide another route for transcytosis. 
HIV can attach to dendritic cells by gp120 binding to CD207, CD206 and DC-SIGN. 
The bound viruses are taken up into vacuoles, and remain stable until encountering susceptible CD4 T cells, whether in the local mucosal environment or after being carried to lumphoid tissue. 
Furthermore, CCR5 expressing CD4 T cells resident in sone mucosal sites and are sites of early HIV replication. 

\par

During acute phase of infection, there is a rapid replication of HIV in CCR5 expressing CD4 T cells. 
As a result, HIV abundance increases, while CD4 T cells drop rapidly. 
After establishment of adaptive immune responses including cytolic CD8 T cells killing infected cells and HIV-specific antibodies, HIV abundance drops dramatically, accomapnied with a slight increase in CD4 T cells. 
Under the strong selection pressure of adaptive immunity, HIV secape mutants that are no longer detected by immune cells are selected, giving rise to many variants in a single infected individuals. 
This causes loss of CD4 T cells and progress to AIDS. 
When CD4 T cells declines below a critical level, cell-mediated immunity is lost, and various infections from opportunistic pathogens appear. 

\subsection{Therapy strategies for AIDS}
Drugs targeting at viral reverse transcriptase, viral integrase and viral protease can block HIV replication. 
Combination of therapy with inhibitors of viral protease and reverse transcriptase, known as highly active antiretroviral therapy (HAART) can rapidly reduce load of HIV. 
HAART is accompanied with a slow but stable increase in CD4 T cells, but many other components of immunity are still compromised. 
There are three complementary mechanisms: 
redistribution of CD4 T memory cells from lymphoid into circulation, 
reduction of immune activation as HIV infection is controlled, 
and emergence of new naive T cells in thymus. 

\par

HAART inhibits HIV replication, and thus, prevents progression of AIDS effectively. 
However, it cannot eliminate all viral stores. 
Cessation of HAART leads to a rapid rebound of viral multiplication, so patients require therapy indefinitely. 
Also, HARRT is of high cost with side effects. 

\par

New drugs prevent AIDS through other targets. 
Viral entry inhibitors block gp120 binding to CCR5 or block viral envelop fusion with plasma membrane by inhibiting gp41. 
Viral integrase inhibitor block insertion of viral cDNA into host genome. 
There are also strategies enhancing HIV restriction factors including APOBEC and TRIM 5$\alpha$. 
APOBEC introduces mutations into viral cDNA and blocks its expression. 
TRIM 5$\alpha$ preventing assembly and release of viral particles. 
To purge latently infected cells, potential strategies induce viral replication in combination with enhancement of immune clearance of viruses and infected cells. 
Administration of cytokines like IL-2, IL-6 and TNF-$\alpha$ can activate provirus. 
Agents that target epigenetic modifiers like histone deacetylase inhibitors also activate latent viruses. 

\par

Due to the rapid evolution of RNA viruses, a singal patient is loaded with multiple vairants of HIV. 
When drugs are administrated, resistant variants are selected and multiply rapidly. 
Vaccination against HIV is an attractive solution. 
Ideally, an effective vaccine would elicit broadly neutralization antibodies that block viral entry into cells and cytotoxic T cell responses. 
However, development of vaccination is difficult as HIV directly undermines central component of adaptive immunity and escapes from strong immune responses by rapidly mutating. 

\end{sloppypar}
\end{document}