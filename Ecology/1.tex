\documentclass[11pt]{article}

\usepackage{setspace}
\usepackage[colorlinks,linkcolor=blue,anchorcolor=red,citecolor=black]{hyperref}
\usepackage{lineno}
\usepackage{booktabs}
\usepackage{graphicx}
\usepackage{float}
\usepackage{floatrow}
\usepackage{subfigure}
\usepackage{caption}
\usepackage{subcaption}
\usepackage{geometry}
\usepackage{multirow}
\usepackage{longtable}
\usepackage{lscape}
\usepackage{booktabs}
\usepackage{natbib}
\usepackage{natbibspacing}
\usepackage[toc,page]{appendix}
\usepackage{makecell}
\usepackage{amsfonts}
\usepackage{amsmath}

\title{Measuring Biodiversity by Hill Numbers}
\date{}

\linespread{1.5}
\geometry{left=2cm,right=2cm,top=2cm,bottom=2cm}

\begin{document}

  \maketitle

  \linenumbers

\section{Diversity of species, phylogeny and function}
For an assemblage of individuals, its diversity can be measured in ways different in incorporation of species difference. 
In species diversity, all species are assumed to be equally distinct. 
In phylogenetic diversity, evolutionary or phylogenetic differences, \textit{e.g.} taxonomic classification and well-supported phylogenetic tree, is taken into consideration. 
All else being equal, an assemblage of closely related species is less diverse in terms of phylogeny than an assemblage composed of highly divergent species. 
In functional diversity, species are described by a set of traits and species difference can be measured by dissimilarities of trait profiles. 
There are three main approach to functional diversity measures: distance based, dendrogram based and trait-value based.

\section{Classic measures of species diversity}
\label{SpeciesRichness}
Species richness is a simple count of number of species present in an assemblage. 
It is an intuitive and frequently used diversity index and is a key metric in conservation biology. 
However, it does not incorporate information about species abundances and is difficult to estimate accurately from small samples. 

\newline

Shannon index is another measure of diversity, defined as 
\begin{equation}
    H_{Sh} = -\sum\limits_{i=1}^{S}p_i \ln p_i
    \label{ShannonIndex}
\end{equation}
where $S$ is species richness and $p_i$ is the relative abundance of $i$th species.

\newline

Gini-Simpson index is another popular measure of diversity, defined as 
\begin{equation}
    H_{GS}=1-\sum\limits_{i=1}^{S}p_i^2
    \label{GiniSimpsonIndex}
\end{equation}
It gives the probability that two randomly drawn individuals belong to different species.

\newline

Species richness, Shannon index and Gini-Simpson index can be united into a family of generalized indexes, parametrized by a variable $q \ge 0$. 
The generalized index is defined as 
\begin{equation}
    ^{q}H = \frac{1-\sum\limits_{i=1}^{S} p_i^q}{q-1}
    \label{GeneralizedIndex}
\end{equation}
Thus,
\begin{equation}
    ^{0}H = S-1
\end{equation}
\begin{equation}
    ^{1}H = \lim\limits_{q\rightarrow 1}{^{q}H} = H_{Sh}
\end{equation}
\begin{equation}
    ^{2}H = H_{GS}
\end{equation}

\section{Classic measures of phylogenetic diversity}
\label{FaithPD}
Faith's phylogenetic diversity (PD) (Faith 1992) is a widely used measure of phylogenetic diversity. 
It is defined as the sum of all branch lengths of a phylogenetic tree connecting all species present in an assemblage. 
Like species richness, Faith's PD does not incorporate information on species abundance.

\newline

Rao's quadratic entropy (Rao 1982) incorporates both phylogeny and species abundances, defined as 
\begin{equation}
    Q = \sum\limits_{i=1}^{S}\sum\limits_{i=1}^{S} d_{ij}p_{i}p_{j}
    \label{RaoPD}
\end{equation}
where $d_{ij}$ is the phylogenetic distance between $i$th and $j$th species. 

\newline

Allen \textit{et al.} (2009) proposed another measure of phylogenetic measure: 
\begin{equation}
    H_p = -\sum\limits_{i} L_ia_i \ln a_i
    \label{AllenPD}
\end{equation}
where $L_i$ is the length of branch $i$, $a_i$ is the summed relative abundance of all species descended from branch $i$.

\newline

For an ultrametric tree in which all branch tips are the same distance (denote by depth $T$) to the root, Faith's PD, Rao's $Q$ and $H_p$ of Allen \textit{et al.} can be generalized as (Pavoine \textit{et al.} 2009)
\begin{equation}
    ^{q}I = \frac{T-\sum\limits_{i}L_ia_i^q}{q-1}
    \label{GeneralizedPD}
\end{equation}
Thus,
\begin{equation}
    ^{0}I = \sum\limits_{i}L_i - T
\end{equation}
\begin{equation}
    ^{1}I = \lim_{q\rightarrow 1} {^{q}I} = -\sum\limits_{i}L_ia_i\ln a_i
\end{equation}
\begin{equation}
    ^{2}I = T-\sum\limits_{i}L_i a_i^2
\end{equation}

\section{Classic measures of functional diversity (distance-based)}
Functional attribute diversity (FAD) (Walker \textit{et al.} 1999) is the sum of the pairwise distances between species: 
\begin{equation}
    FAD = \sum\limits_{i=1}^{S}\sum\limits_{j=1}^{S}d_{ij}
    \label{FAD}
\end{equation}
where $d_{ij}$ is the functional distance of species $i$ and $j$.

\newline

Rao's $Q$ (Equation \ref{RaoPD}) can be used as a measure of functional diversity, replacing $d_{ij}$ by functional distance. 
Guiasu (2011, 2012) proposed a weighted Gini-Simpson index for pairs of species as follows:
\begin{equation}
    GS_D = \sum\limits_{i=1}^{S}\sum\limits_{j=1}^{S}d_{ij}p_ip_j = Q-\sum\limits_{i=1}^{S}\sum\limits_{j=1}^{S}d_{ij}(p_ip_j)^2
    \label{Guiasu}
\end{equation}

\newline

Ricota and Szeidl (2009) and de Bello \textit{et al.} (2010) transformed Rao's $Q$ for functional diversity to the effective number of species with a maximum species pairwise distance $d_{max}$. 
It is the theoretical species richness of a perfectly even assemblage with the same Rao's $Q$ as the original assemblage:
\begin{equation}
    Q_e = \frac{1}{1-Q/d_{max}}
    \label{EffectiveRichnessQ}
\end{equation}

\section{Measure diversity by Hill numbers}
Measures of species diversity, phylogenetic diversity and functional diversity can be incorporated into a unified framework based on Hill numbers (Chao and Chiu \textit{et al.} 2014)
Denote by $C$ an assemblage of entities (species or phylogenetic tree branch segments or species pairs), and by $u$ elements in $C$. 
For each $u$, denote by $v_u$ its attribute value and by $a_u$ its weight (\textit{e.g.} biomass, cover area, abundance). 
The weighted mean attribute value is
\begin{equation}
    \bar{V} = \sum\limits_{u\in C}a_uv_u
    \label{WeightedMeanAttributeValue}
\end{equation}
The Hill number of order $q$ is defined as 
\begin{equation}
    ^{q}AD(\bar{V}) = 
    [\sum\limits_{u\in C} v_u \times (\frac{a_u}{\bar{V}})^q]^{\frac{1}{1-q}} = [\sum\limits_{u\in C} v_u \times (\frac{a_u}{\sum\limits_{u\in C}a_uv_u})^q]^{\frac{1}{1-q}}
    \label{HillNumber}
\end{equation}
AD is short for attribute diversity. 
Further, 
\begin{equation}
    ^{0}AD(\bar{V}) = \sum\limits_{u\in C}v_u
\end{equation}
\begin{equation}
    ^{1}AD(\bar{V}) = \lim\limits_{q\rightarrow 1}{^{1}AD(\bar{V})} = exp\{-\sum\limits_{u\in C}v_u\frac{a_u}{\bar{V}}\ln\frac{a_u}{\bar{V}}\} 
\end{equation}
\begin{equation}
    ^{2}AD(\bar{V}) = \frac{1}{\sum\limits_{u\in C}v_u (\frac{a_u}{\bar{V}})^2}
\end{equation}

\newline

For species diversity, $C$ is an assemblage of species indexed by $i = 1,2,3,\dots,S$. 
Denote by $p_i$ the relative abundance of $i$th species and let it be its weight. 
Since all species are equally distinct, they have same attribute value 1. 
Thus, Equation \ref{HillNumber} reduces to ordinary Hill numbers (Hill 1973):
\begin{equation}
    ^{q}D = (\sum\limits_{i=1}^{S}p_i^q)^\frac{1}{1-q}
    \label{HillNUmber_Species}
\end{equation}

\newline

For phylogenetic diversity, $C$ is the assemblage of all branch segments in a phylogenetic tree. 
Index each branch by $i = 1,2,3,\dots,B$, and denote by $L_i$ the length of branch $i$. 
Let $a_i$ be the branch abundance, \textit{i.e.} the summed relative abundance of all species descended from the branch $i$. 
The weighted mean attribute value is 
\begin{equation}
    \bar{T} = \sum\limits_{i=1}^{B}L_ia_i
    \label{WeightedMeanAttributeValue_Phylogenetic}
\end{equation}
The Hill number for phylogenetic diversity is (Chao \textit{et al.} 2010)
\begin{equation}
    ^{q}\bar{D}(\bar{T}) = [\sum\limits_{i=1}^{B}L_i \times (\frac{a_i}{\sum\limits_{i=1}^{B}L_ia_i})^q]^{(\frac{1}{1-q})}
    \label{HillNumber_Phylogenetic}
\end{equation}

\newline

For functional diversity, $C$ is the assemblage of pairs of species. 
Denote by $S$ species richness and by $p_i$ relative abundance of $i$th species. 
For species pair $u=(i,j)$, its attribute value $v_u$ equals functional distance of species $i$ and $j$, \textit{i.e.} $v_u = d_{ij}$. 
Its abundance $a_u = p_ip_j$. 
The weighted mean attribute value is 
\begin{equation}
    Q = \sum\limits_{i,j=1}^{S}d_{ij}p_ip_j
    \label{WeightedMeanAttributeValue_Functional}
\end{equation}
The Hill number for functional diversity is (Chiu and Chao 2014)
\begin{equation}
    ^{q}FD(Q) = [\sum\limits_{i,j=1}^{S}d_{ij} \times (\frac{p_ip_j}{\sum\limits_{i,j=1}^{S}d_{ij}p_ip_j})^q]^{\frac{1}{1-q}}
\end{equation}

\section{Replication principle}
Assume there is a total of $N$ assemblages of entities (species or phylogenetic tree branch segments or species pairs), denoted by $C_1, C_2,\dots,C_N$. 
These assemblages are completely distinct, \textit{i.e.} $\forall i,j=1,2,\dots,N, C_iC_j = \emptyset$. 
For assemblage $C_i$, denote by by $v_{ui}$ the attribute value of element $u \in C_i$ and by $a_{ui}$ its weight. 
Assume all assemblages are identical in weighted mean attribute value and Hill number of same order $q$, \textit{i.e.}
\begin{equation}
    \bar{V} = \sum\limits_{u\in C_i}a_{ui}v_{ui}, i = 1,2,3,\dots,N
\end{equation}
and 
\begin{equation}
    ^{q}AD(\bar{V}) = [\sum\limits_{u\in C_i} v_{ui} \times (\frac{a_{ui}}{\bar{V}})^q]^{\frac{1}{1-q}}, i = 1,2,3,\dots,N
\end{equation}
For pooled assemblage $\sum\limits_{i=1}^{N}C_i$, its weighted mean attribute value 
\begin{equation}
    \bar{V}_p = \sum\limits_{i=1}^{N} \sum\limits_{u\in C_i}a_{ui}v_{ui} = \sum\limits_{i=1}^{N} \bar{V} = N \times \bar{V}
\end{equation}
The Hill number of pooled assemblage is 
\begin{equation}
    \begin{aligned}
    ^{q}AD(\bar{V}_p) & = 
    [\sum\limits_{i=1}^{N} \sum\limits_{u\in C_i} v_{ui} \times (\frac{a_{ui}}{\bar{V}_p})^q]^{\frac{1}{1-q}} \\ & =
    [\sum\limits_{i=1}^{N} \sum\limits_{u\in C_i} v_{ui} \times (\frac{a_{ui}}{\bar{V}})^q \times \frac{1}{N^q}]^{\frac{1}{1-q}} \\ & = 
    \{\sum\limits_{i=1}^{N} \frac{1}{N_q} \times [^{q}AD(\bar{V})]^{1-q}\}^{\frac{1}{1-q}} \\ &=
    N \times ^{q}AD(\bar{V})
    \end{aligned}
    \label{ReplicationPrinciple}
\end{equation}

\section{Decomposition of Hill numbers}
Consider a pooled assemblage composed of $N$ assemblages. 
Denote by $z_{ik}$ the weight of $i$th entities (species or phylogenetic tree branch segments or species pairs) in $k$th assemblage, where $i=1,2,3,\dots,S$ and $k=1,2,3,\dots,N$. 
The total weight in $k$th assemblage is $z_{+k}=\sum_{i=1}^{S}z_{ik}$, and the total weight of $i$th entity in the pooled assemblage is $z_{i+}=\sum_{k=1}^{N}z_{ik}$. 
The total weight of all entities in the pooled assemblage is $z_{++}=\sum_{k=1}^{N}\sum_{i=1}^{S}z_{ik}$. 
The attribute value of $i$th entity is $v_i$. 

\newline

The gamma diversity of the pooled assemblage given by Equation \ref{HillNumber} is 
\begin{equation}
    ^qAD_{\gamma}(\bar{V}) = (\sum_{i=1}^{S}v_i \times (\frac{z_{i+}}{\sum_{i=1}^{S}z_{i+}v_i})^q )^{\frac{1}{1-q}}
    \label{Gamma}
\end{equation}
The alpha diversity is (Chiu \textit{et al.}) 
\begin{equation}
    ^qAD_{\alpha}(\bar{V}) = ( \sum_{i=1}^{S}v_i \times \sum_{k=1}^{N} (\frac{z_{ik}}{\sum_{i=1}^{S}z_{i+}v_i} )^q )^{\frac{1}{1-q}}
    \label{Alpha}
\end{equation}
The beta diversity is
\begin{equation}
    ^qAD_{\beta}(\bar{V}) = \frac{1}{N}\frac{^qAD_{\gamma}(\bar{V})}{^qAD_{\alpha}(\bar{V})}
\end{equation}

\newline

When the $N$ assemblages are completely identical, \textit{i.e.} $z_{ik}=z$ for all $i=1,2,3,\dots,S$ and $k=1,2,3,\dots,N$, $^qAD_{\beta}(\bar{V})$ equals 1. 
When the $N$ assemblages are completely distinct, \textit{i.e.} any two assemblages do not have shared entity, $^qAD_{\beta}(\bar{V})$ equals $N$.

\newpage
\textbf{References} %APA
\newline
[1] Chao, A. , Chiu, C. H. , & Jost, L. . (2014). Unifying species diversity, phylogenetic diversity, functional diversity, and related similarity and differentiation measures through hill numbers. Annual Review of Ecology Evolution & Systematics, 45(1), 297-324.
\end{document}