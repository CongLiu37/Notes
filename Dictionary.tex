\documentclass[11pt]{article}

% \usepackage[UTF8]{ctex} % for Chinese 

\usepackage{setspace}
\usepackage[colorlinks,linkcolor=blue,anchorcolor=red,citecolor=black]{hyperref}
\usepackage{lineno}
\usepackage{booktabs}
\usepackage{graphicx}
\usepackage{float}
\usepackage{floatrow}
\usepackage{subfigure}
\usepackage{caption}
\usepackage{subcaption}
\usepackage{geometry}
\usepackage{multirow}
\usepackage{longtable}
\usepackage{lscape}
\usepackage{booktabs}
\usepackage{natbib}
\usepackage{natbibspacing}
\usepackage[toc,page]{appendix}
\usepackage{makecell}
\usepackage{amsfonts}
 \usepackage{amsmath}

\title{Dictionary}
\author{}
\date{}

\linespread{1.5}
\geometry{left=2cm,right=2cm,top=2cm,bottom=2cm}

\begin{document}
\begin{sloppypar}
  \maketitle

  \linenumbers

\section{GC skew}
GC skew is when the nucleotides G and C are over- or under-abundant in a particular region of DNA or RNA.
In equilibrium conditions (without mutational or selective pressure and with nucleotides randomly distributed within the genome) there is an equal frequency of the four DNA bases on both single strands of a DNA molecule. 
However, in most bacteria and some archaea, nucleotide compositions are asymmetric between the leading strand and the lagging strand: 
the leading strand contains more G and T, whereas the lagging strand contains more A and C. 
This phenomenon is referred to as GC and AT skew and the corresponding statistics are defined as:
\newline
GC skew = (C - G)/(G + C)
\newline
AT skew = (A − T)/(A + T) 

\section{Synteny/Collinearity}
Preservation of the precise order of genes on a chromosome passed down from a common ancestor. 
Shared synteny is one of the most reliable criteria for establishing the orthology of genomic regions in different species. 
Additionally, exceptional conservation of synteny can reflect important functional relationships between genes.   

\end{sloppypar}
\end{document}