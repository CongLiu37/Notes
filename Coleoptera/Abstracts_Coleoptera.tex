\documentclass[11pt]{article}

% \usepackage[UTF8]{ctex} % for Chinese 

\usepackage{setspace}
\usepackage[colorlinks,linkcolor=blue,anchorcolor=red,citecolor=black]{hyperref}
\usepackage{lineno}
\usepackage{booktabs}
\usepackage{graphicx}
\usepackage{float}
\usepackage{floatrow}
\usepackage{subfigure}
\usepackage{caption}
\usepackage{subcaption}
\usepackage{geometry}
\usepackage{multirow}
\usepackage{longtable}
\usepackage{lscape}
\usepackage{booktabs}
\usepackage{natbibspacing}
\usepackage[toc,page]{appendix}
\usepackage{makecell}
\usepackage{amsfonts}
\usepackage{amsmath}

\usepackage[backend=bibtex,style=authoryear,sorting=nyt,maxnames=1]{biblatex}
\bibliography{} # Reference bib

\title{Abstracts_Coleoptera}
\author{}
\date{}

\linespread{1.5}
\geometry{left=2cm,right=2cm,top=2cm,bottom=2cm}

\setlength\bibitemsep{0pt}
\setcitestyle{braket}

\begin{document}
\begin{sloppypar}
\maketitle

\linenumbers

\textbf{Maire \textit{et al.}, 2018, An IMD-like pathway mediates both endosymbiont control and host immunity in the cereal weevil \textit{Sitophilus spp.}.} \newline
In the cereal weevil \textit{Sitophilus spp.}, which houses \textit{Sodalis pierantonius}, endosymbionts are secluded in specialized host cells, the bacteriocytes that group together as an organ, the bacteriome. 
At standard conditions, the bacteriome highly expresses the coleoptericin A (colA) antimicrobial peptide (AMP), which was shown to prevent endosymbiont escape from the bacteriocytes. 
However, following the insect systemic infection by pathogens, the bacteriome upregulates a cocktail of AMP encoding genes, including colA. 
The regulations that allow these contrasted immune responses remain unknown. 
Here, evidence shows that an IMD-like pathway is conserved in two sibling species of cereal weevils, \textit{Sitophilus oryzae} and \textit{Sitophilus zeamais}. 
RNA interference (RNAi) experiments showed that \textit{imd} and \textit{relish} genes are essential for 
(i) colA expression in the bacteriome under standard conditions, 
(ii) AMP up-regulation in the bacteriome following a systemic immune challenge, 
(iii) AMP systemic induction following an immune challenge. 
Histological analyses also showed that \textit{relish} inhibition by RNAi resulted in endosymbiont escape from the bacteriome, strengthening the involvement of an IMD-like pathway in endosymbiont control. 
It is concluded that \textit{Sitophilus}’ IMD-like pathway mediates both the bacteriome immune program involved in endosymbiont seclusion within the bacteriocytes and the systemic and local immune responses to exogenous challenges. 
This work provides an example of how a conserved immune pathway, initially described as essential in pathogen clearance, also functions in the control of mutualistic associations.
  
\end{sloppypar}
\end{document}