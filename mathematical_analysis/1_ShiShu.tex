\documentclass[11pt]{article}

\usepackage[UTF8]{ctex} % for Chinese 

\usepackage{setspace}
\usepackage[colorlinks,linkcolor=blue,anchorcolor=red,citecolor=black]{hyperref}
\usepackage{lineno}
\usepackage{booktabs}
\usepackage{graphicx}
\usepackage{float}
\usepackage{floatrow}
\usepackage{subfigure}
\usepackage{caption}
\usepackage{subcaption}
\usepackage{geometry}
\usepackage{multirow}
\usepackage{longtable}
\usepackage{lscape}
\usepackage{booktabs}
\usepackage{natbib}
\usepackage{natbibspacing}
\usepackage[toc,page]{appendix}
\usepackage{makecell}
\usepackage{amsfonts}

\title{实数}
\date{}

\linespread{1.5}
\geometry{left=2cm,right=2cm,top=2cm,bottom=2cm}

\begin{document}

  \maketitle

  \linenumbers
\section{实数公理}
集合$\mathbb{R}$为实数集,若I-VI条件成立。

\newline

(I)\textbf{加法公理}
\newline
定义加法运算
\begin{equation}
    +:\mathbb{R} \times \mathbb{R} \rightarrow \mathbb{R}
\end{equation}
使$\forall x \in \mathbb{R}, \forall y \in \mathbb{R} \Rightarrow \exists x+y \in \mathbb{R}$,且有:
\newline
(1)中性元素0:
\newline
$\exists 0 \in \mathbb{R}, \forall x \in \mathbb{R} \Rightarrow x+0=0+x=x$
\newline
(2)逆元素$-x$:
\newline
$\forall x \in \mathbb{R}, \exists -x \in \mathbb{R} \Rightarrow x+(-x)=(-x)+x=0$
\newline
(3)结合律:
\newline
$\forall x \in \mathbb{R}, \forall y \in \mathbb{R}, \forall z \in \mathbb{R} \Rightarrow x+(y+z)=(x+y)+z$
\newline
(4)交换律:
\newline
$\forall x \in \mathbb{R}, \forall y \in \mathbb{R} \Rightarrow x+y=y+x$
\newline

(II)\textbf{乘法公理}
\newline
定义乘法运算
\begin{equation}
    \bullet:\mathbb{R} \times \mathbb{R} \rightarrow \mathbb{R}
\end{equation}
使$\forall x \in \mathbb{R}, \forall y \in \mathbb{R} \Rightarrow \exists x \bullet y\in\mathbb{R}$,且有:
\newline
(1)中性元素1:
\newline
$\exists 1 \in \mathbb{R}, \forall x \in \mathbb{R} \Rightarrow x \bullet 1=1 \bullet x=x$
\newline
(2)逆元素$x^{-1}$:
\newline
$\forall x \in \mathbb{R}, \exists x^{-1} \in \mathbb{R} \Rightarrow x \bullet x^{-1}=x^{-1} \bullet x=x$
\newline
(3)结合律:
\newline
$\forall x \in \mathbb{R}, \forall y \in \mathbb{R}, \forall z \in \mathbb{R} \Rightarrow x \bullet (y \bullet z)=(x \bullet y) \bullet z$
\newline
(4)交换律:
\newline
$\forall x \in \mathbb{R}, \forall y \in \mathbb{R} \Rightarrow x\bullet y=y\bullet x$
\newline

(I,II)\textbf{加法公理与乘法公理}
\newline
$\forall x \in \mathbb{R}, \forall y \in \mathbb{R}, \forall z \in\mathbb{R} \Rightarrow (x+y) \bullet z=x \bullet z+y \bullet z$
\newline

(III)\textbf{序公理}
\newline
$\mathbb{R}$中元素存在关系$\le$。对$\forall x \in \mathbb{R}, \forall y \in \mathbb{R}$,可判断关系$x \le y$是否成立。且有:
\newline
(1)$\forall x \in \mathbb{R} \Rightarrow x \le x$
\newline
(2)$(x \le y) \land (y \le x) \Rightarrow x=y$
\newline
(3)$(x \le y) \land (y \le z) \Rightarrow x \le z$
\newline
(4)$\forall x \in \mathbb{R}, \forall y \in \mathbb{R} \Rightarrow x \le y \lor y \le x$
\newline

(I,III)\textbf{加法公理与序公理}
\newline
$\forall x \in \mathbb{R}, \forall y \in \mathbb{R}, \forall z \in \mathbb{R}, x \le y \Rightarrow x+z \le y+z$
\newline

(II,III)\textbf{乘法公理与序公理}
\newline
$\forall x \in \mathbb{R}, \forall y \in \mathbb{R}, (x \ge 0) \land (y \ge 0) \Rightarrow x \bullet y \ge 0$
\newline

(VI)\textbf{连续性(完备性)公理}
\newline
$X, Y$为$\mathbb{R}$的非空子集,且$\forall x \in X, \forall y \in Y \Rightarrow x \le y$
则$\exists c \in \mathbb{R}$,使
\newline
$\forall x \in X, \forall y \in Y \Rightarrow x \le c \le y$
\newline

\section{实数公理之推论}
\textbf{加法公理(I)之推论}
\newline
(1)中性元素$0$唯一
\newline
\textit{假设有$0_1, 0_2$为$\mathbb{R}$中两个不同的零元素,则
$0_1=0_1+0_2=0_2+0_1=0_2$}
\newline
(2)任意实数$x$有且仅有一个逆元素$-x$
\newline
\textit{假设$x_1, x_2$为$x \in \mathbb{R}$的两个不同的逆元素,则
$x_1=x_1+0=x_1+x+x_2=0+x_2=x_2$}
\newline
(3)关于$x$的方程$a+x=b$在$\mathbb{R}$中有唯一解$x=b+(-a)$
\newline
\textit{$a$有唯一逆元素$-a$,使得$a+(-a)=0$,故$b+(-a)$唯一。且有
$x=x+0=a+x+(-a)=b+(-a)$}
\newline

\textbf{乘法公理(II)之推论}
\newline
(1)中性元素$1$唯一
\newline
(2)任意实数$x$有且仅有一个逆元素$x^{-1}$
\newline
(3)$a \not = 0$时,关于$x$的方程$a \bullet x=b$在$\mathbb{R}$中有唯一解$x=b\bullet a^{-1}$
\newline

\textbf{(I,II)之推论}
\newline
(1) $\forall x \in \mathbb{R} \Rightarrow x \bullet 0 = 0 \bullet x = 0$
\newline
\textit{$x \bullet 0=x \bullet (0+0) = x \bullet 0+x \bullet 0 \Rightarrow -(x \bullet 0)+x \bullet 0 = x \bullet 0 = 0}$}
\newline
(2) $x \bullet y=0 \Rightarrow x=0 \lor y=0$
\newline
\textit{假设$x\not =0$,方程$x\bullet y=0$有唯一解$y=0\bullet x^{-1}=0$;假设$y\not =0$同理}
\newline
(3) $\forall x \in \mathbb{R} \Rightarrow (-1) \bullet x=-x$
\newline
\textit{给定$x \in \mathbb{R}$,$-x$唯一,且$x+(-1)\bullet x=x\bullet(1+(-1))=x\bullet 0=0$}
\newline
(4)$\forall x \in \mathbb{R} \Rightarrow (-1)\bullet(-x)=x$
\newline
(5)$\forall x \in \mathbb{R} \Rightarrow (-x) \bullet (-x)=x\bullet x$
\newline

\textbf{序公理之推论}
\newline
(1)$\forall x \in \mathbb{R}, \forall y \in \mathbb{R} \Rightarrow x<y \lor x=y \lor x>y$
\newline
(2)$\forall x \in \mathbb{R}, \forall y \in \mathbb{R}, \forall z \in \mathbb{R}$
\newline
$x<y \land y \le z \Rightarrow x<z$
\newline
$x \le y \land y<z \Rightarrow x<z$
\newline

\textbf{(I,III), (II,III)之推论}
\newline
(1)$\forall x \in \mathbb{R}, \forall y \in \mathbb{R}, \forall z \in \mathbb{R}, \forall w \in \mathbb{R}$
\newline
$x<y \Rightarrow x+z<y+z$
\newline
$x>0 \Rightarrow -x<0$
\newline
$x \le y \land z \le w \Rightarrow x+z \le y+w$
\newline
$x \le y \land z<w \Rightarrow x+z<y+w$
\newline
(2)$\forall x \in \mathbb{R}, \forall y \in \mathbb{R}, \forall z \in \mathbb{R}$
\newline
$x>0 \land y>0 \Rightarrow xy>0$
\newline
$x<0 \land y<0 \Rightarrow xy>0$
\newline
$x>0 \land y<0 \Rightarrow xy<0$
\newline
$x<y \land z>0 \Rightarrow xz>yz$
\newline
$x<y \land z<0 \Rightarrow xz<yz$
\newline
(3)$0<1$
\newline
(4)$\forall x \in \mathbb{R}, x>0 \Rightarrow x^{-1}>0$
\newline
(5)$\forall x \in \mathbb{R}, \forall y \in \mathbb{R}, x>0 \land x<y \Rightarrow 0<y^{-1}<x^{-1}$
\newline

\textbf{界}
\newline
集合$X \subset \mathbb{R}$,若$\exists c \in \mathbb{R}$,使得$\forall x \in X, x \le c (x \ge c)$,则$c$为$X$之上(下)界
\newline

\textbf{有界集}
\newline
既有上确界又有下确界的集合为有界集
\newline

\textbf{最值}
\newline
集合$X \subset \mathbb{R}$,数$a \in X$,若$\forall x \in X, x \le a (x \ge a)$,则$a$为$X$之最大值(最小值),记作$\max\limits_{x\in X}x (\min\limits_{x\in X}x)$
\newline

\textbf{确界}
\newline
集合$X \subset \mathbb{R}$,其最小上界为上确界,记作$\sup\limits_{x \in X}x$;其最大下界为下确界,记作$\inf\limits_{x \in X}x$
\newline

\textbf{确界定理}
\newline
实数集任何有上(下)界的子集有唯一上(下)确界
\newline
\textit{记上有界集$X\in\mathbb{R}$,其上界集合$Y=\{y\in\mathbb{R}}:\forall x \in X, x\le y\}$
\newline
则根据连续性公理,$\exists c, \forall x \in X, \forall y \in Y, x \le c \le y$
\newline
所以 $c$为$X$之上界,亦为$Y$之下界,且$c \in Y$
\newline
所以 $c=\min Y$
\newline
所以 $c$唯一,且$c=\sup X$}

\section{实数类}
\textbf{自然数$\mathbb{N}$}
\newline
$\mathbb{N}={1,2,3,\dots}$
\newline

\textbf{归纳集}
\newline
集$X\subset\mathbb{R}$为归纳集,若$\forall x \in X \Rightarrow x+1 \in X$
\newline

\textbf{自然数集}
\newline
包含1的一切归纳集之交集为自然数集$\mathbb{N}$
\newline

\textbf{数学归纳原理}
\newline
$E \subset \mathbb{N}, 1 \in E$,且有$x \in E \Rightarrow x+1 \in E$,则$E=\mathbb{N}$
\newline

\textbf{数学归纳原理之推论}
\newline
(1)两自然数之和与积为自然数
\newline
\textit{任取$n \in \mathbb{N}$,则$n+1 \in \mathbb{N}$
\newline
假设$n+m \in \mathbb{N}, m \in \mathbb{N}$,则$n+m+1 \in \mathbb{N}$
\newline
所以 两自然数之和与积为自然数
\newline
对自然数$n$,有$n \bullet 1 \in \mathbb{N}$
\newline
假设$n \bullet m \in \mathbb{N}, m \in \mathbb{N}$,则$n\bullet(m+1)=n\bullet m+n \in \mathbb{N}$
\newline
所以 两自然数之积为自然数}
\newline
(2)$n \in \mathbb{N}-\{1\} \Rightarrow n-1 \in \mathbb{N}$
\newline
\textit{记$E=\{n-1:n\in \mathbb{N}-{1}\}$
\newline
所以 $1=2-1 \in E$
\newline
任取$m \in E, m=n-1$,则$m+1=n=(n+1)-1 \in E$
\newline
所以 $E=\mathbb{N}$}
\newline
(3)$\forall n \in \mathbb{N}, \{x\in\mathbb{N}:x>n\}$有最小元素$n+1$
\newline
\textit{$n=1$时
\newline
记 $M=\{x\in\mathbb{N}: x=1 \lor x\ge2\}$
\newline
假设 $x\in M$,则$x=1\in M\Rightarrow x+1=2\in M$或$x\ge2\Rightarrow x+1\ge2\Rightarrow x+1\in M$
\newline
所以 $M=\mathbb{N}$
\newline
所以 $x\not = 1\land x\in\mathbb{N}\Rightarrow x\ge2$
\newline
所以 $n=1$时,$\min\{x\in\mathbb{N}:n<x\}=n+1=2$
\newline
假设 自然数$n$使得命题成立,即$\min\{y\in\mathbb{N}:n<y\}=n+1$
\newline
取 $x\in\{x\in\mathbb{N}:x>n+1\}$
\newline
有 $x-1>n$
\newline
所以 $x-1\ge n+1\Rightarrow x\ge n+2$
\newline
所以 $n+1$使得命题成立}
\newline
(4)$m\in\mathbb{N} \land m\in\mathbb{N} \land n<m \Rightarrow n+1\le m$
\newline
(5)自然数$n$,不存在自然数$x\in (n,n+1)$
\newline
(6)自然数$n$,$n\not = 1$,不存在自然数$x\in (n-1,n)$
\newline
(7)自然数集之非空子集有最小元素
\newline
\textit{记$M\subset\mathbb{N}, M\not = \emptyset$
\newline
若 $1\in M$,则$\min M=1$
\newline
若 $1\not\in M$,则$1\in\mathbb{N}-M$
\newline
则有自然数$n$,使得所有小于等与$n$的自然数属于$\mathbb{N}-M$,且$n+1\in M$
\newline
否则 $1\in\mathbb{N}-M$ 且 $\forall n\in\mathbb{N}, n\in M\land n+1\in M \Rightarrow M=\mathbb{N}$
\newline
所以 $\min M=n+1$}
\newline

\textbf{整数集}
\newline
自然数集、自然数的相反数集、0的并集为整数集$\mathbb{Z}$
\newline

\textbf{算术基本定理}
\newline
自然数可唯一地(不计相乘顺序)表示为有限个素数之积
\newline

\textbf{有理数}
\newline
形如$m\bullet n^{-1}, m\in \mathbb{Z}, n\in \mathbb{Z}$的数为有理数
\newline
有理数集记为$\mathbb{Q}$
\newline

\textbf{无理数}
\newline
不属于有理数集的实数为无理数

\section{Archimedes原理}
(1)自然数集的任何非空上有界子集有最大值
\newline
\textit{任意$E\subset\mathbb{N}, E\not =\emptyset$有唯一上确界$s=\sup E$
\newline
所以 $\exists n\in E, s-1<n\le s$
\newline
所以 $n+1>s\Rightarrow n+1\notin E$
\newline
所以 $n=\max E$}
\newline
(2)自然数集无上界
\newline
(3)整数集的任何非空上有界子集有最大值
\newline
(4)整数集的任何非空下有界子集有最小值
\newline
(5)整数集无上下界
\newline

\textbf{Archimedes原理}
\newline
任意固定正数$h$,则$\forall x\in\mathbb{R}, \exists! k\in\mathbb{Z}, (k-1)h\le x<kh$
\newline
\textit{考虑下有界集合$\{n\in\mathbb{Z}:n>\frac{x}{h}\}$,其唯一最小值记为$k$
\newline
所以 $k-1\le\frac{x}{h}<k$
\newline
所以 $(k-1)h\le x<kh$}
\newline

(1)$\forall\epsilon>0, \exists n\in\mathbb{N}, 0<\frac{1}{n}<\epsilon$
\newline
\textit{任意固定正数$\epsilon$,则$x=1$时,$\exists n\in\mathbb{N}, 0<1<n \epsilon$
\newline
所以 $0<\frac{1}{n}<\epsilon$}
\newline
(2)非负实数$x$,且$\forall n\in\mathbb{N}, x<\frac{1}{n}$,则$x=0$
\newline
\textit{若$x>0$,则$\exists n\in\mathbb{N}, x>\frac{1}{n}$}
\newline
(3)$\forall a\in\mathbb{R}, \forall a\in\mathbb{R}, a<b \Rightarrow \exists r\in\mathbb{Q}, a<r<b$
\newline
\textit{取$n\in\mathbb{N}$,使得$b-a>\frac{1}{n}$
\newline
另取$m\in\mathbb{Z}$,使得$(m-1)\frac{1}{n} \le a<m\frac{1}{n}$
\newline
假设$b\le \frac{m}{n}$,则$(m-1)\frac{1}{n}\le a<b \le\frac{m}{n}\Rightarrow b-a \le \frac{m}{n}-(m-1)\frac{1}{n}=\frac{1}{n}$
\newline
所以 $b>\frac{m}{n}$
\newline
所以 $\exists r=\frac{m}{n} \in\mathbb{Q},a<r<b$}
\newline
(4)$\forall x\in\mathbb{R}, \exists!k\in\mathbb{Z}, k\le x<k+1$
\newline

\section{误差}
\textbf{绝对误差和相对误差}
\newline
某量有精确值$x$,近似值$\tilde{x}$
\newline
近似值之绝对误差$\Delta(\tilde{x})=|x-\tilde{x}|$
\newline
相对误差$\delta(\tilde{x})=\frac{\Delta(\tilde{x})}{|\tilde{x}|}$
\newline

\textbf{绝对误差的估计}
\newline
精确值$x,y$分别有估计值$\tilde{x},\tilde{y}$,绝对误差$\Delta(\tilde{x})=|x-\tilde{x}|,\Delta(\tilde{y})=|y-\tilde{y}|$
\newline
(1)$\Delta(\tilde{x}+\tilde{y})=|(x+y)-(\tilde{x}+\tilde{y})|\le\Delta(\tilde{x})+\Delta(\tilde{y})$
\newline
(2)$\Delta(\tilde{x}\tilde{y})=|xy-\tilde{x}\tilde{y}|\le |\tilde{x}|\Delta(\tilde{y})+|\tilde{y}|\Delta(\tilde{x})+\Delta(\tilde{x})\Delta(\tilde{y})$
\newline
(3)若$y\not=0,\tilde{y}\not=0,\delta(\tilde{y})=\frac{\Delta(\tilde{y})}{|\tilde{y}|}<1$,则
\newline
$\Delta(\frac{\tilde{x}}{\tilde{y}})=|\frac{x}{y}-\frac{\tilde{x}}{\tilde{y}}|\le \frac{|\tilde{x}|\Delta(\tilde{y})+ |\tilde{y}|\Delta(\tilde{x})}{\tilde{y}^2} \frac{1}{1-\delta(\tilde{y})}$
\newline
\textit{记$x=\tilde{x}+\alpha, y=\tilde{y}+\beta$
\newline
(1)$\Delta(\tilde{x}+\tilde{y})=|(x+y)-(\tilde{x}+\tilde{y})|=|(x-\tilde{x})+(y-\tilde{y})|\le |x-\tilde{x}|+|y-\tilde{y}|=\Delta(\tilde{x})+\Delta(\tilde{y})$
\newline
(2)$\Delta(\tilde{x}\tilde{y})=|(\tilde{x}+\alpha)(\tilde{y}+\beta)+\alpha\beta|=|\tilde{x}\beta+\tilde{y}\alpha+\alpha\beta|\le |\tilde{x}|\Delta(\tilde{y})+|\tilde{y}|\Delta(\tilde{x})+\Delta(\tilde{x})\Delta(\tilde{y})$
\newline
(3)$\Delta(\frac{\tilde{x}}{\tilde{y}})=|\frac{x}{y}-\frac{\tilde{x}}{\tilde{y}}|=|\frac{x\tilde{y}-y\tilde{x}}{y\tilde{y}}|=|\frac{(\tilde{x}+\alpha)\tilde{y}-(\tilde{y}+\beta)\tilde{x}}{\tilde{y}^2}||\frac{1}{1+\frac{\beta}{\tilde{y}}}|\le \frac{|\tilde{x}||\beta|+|\tilde{y}||\alpha|}{\tilde{y}^2} \frac{1}{1-\delta(\tilde{y})}=\frac{|\tilde{x}|\Delta(\tilde{y})+ |\tilde{y}|\Delta(\tilde{x})}{\tilde{y}^2} \frac{1}{1-\delta(\tilde{y})}$
}
\newline

\textbf{相对误差的估计}
\newline
(1)$\delta(\tilde{x}+\tilde{y}) \le \frac{\Delta(\tilde{x})+\Delta(\tilde{y})}{|\tilde{x}+\tilde{y}|}$
\newline
(2)$\delta(\tilde{x}\tilde{y}) \le \delta(\tilde{x})+\delta(\tilde{y})+\delta(\tilde{x})\delta(\tilde{y})$
\newline
(3)$\delta(\frac{\tilde{x}}{\tilde{y}}) \le \frac{\delta(\tilde{x})+\delta(\tilde{y})}{1-\delta(\tilde{y})}$
\newline
近似值足够精确时,有
\newline
$\Delta(\tilde{x})\Delta(\tilde{y})\approx 0, \delta(\tilde{x})\delta(\tilde{y})\approx 0, 1-\delta(\tilde{y})\approx 1$
\newline
所以
\newline
$\Delta(\tilde{x}\tilde{y}) \le |\tilde{x}|\Delta(\tilde{y})+|\tilde{y}|\Delta(\tilde{x})$
\newline
$\Delta(\frac{\tilde{x}}{\tilde{y}}) \le \frac{|\tilde{x}|\Delta(\tilde{y})+ |\tilde{y}|\Delta(\tilde{x})}{\tilde{y}^2}$
\newline
$\delta(\tilde{x}\tilde{y}) \le \delta(\tilde{x})+\delta(\tilde{y})$
\newline
$\delta(\frac{\tilde{x}}{\tilde{y}}) \le \delta(\tilde{x})+\delta(\tilde{y})$
\newline

\section{几个定理}
\textbf{区间套引理}
\newline
$\mathbb{R}$中闭区间套$I_1\supset I_2\supset I_3\supset\dots\supset I_n\supset\dots$,必有点$c$属于所有这些闭区间
\newline
若$\forall\epsilon>0 \exists I_k, |I_k|<\epsilon$,则$c$为闭区间套唯一公共点
\newline
\textit{$\forall m \in\mathbb{N}, \forall n\in\mathbb{N}, I_m=[a_m,b_m],I_n=[a_n,b_n] \Rightarrow a_m<b_n$
\newline
$\forall a_m\in\{a_m:m\in\mathbb{N}\}, \forall b_n\in\{b_n:n\in\mathbb{N}\}, \exists c\in\mathbb{R}, a_m<c<b_n$
\newline
所以$\forall n\in\mathbb{N}, \exists c, a_n<c<b_n \Rightarrow c\in I_n$
\newline
若$\forall\epsilon>0 \exists I_k, |I_k|<\epsilon$,假设$c_1,c_2$为闭区间套的两个半不同公共点且$c_1<c_2$
\newline
所以 $\forall n\in\mathbb{N}, c_1\in I_n, c_2\in I_n \Rightarrow a_n\le c_1<c_2\le b_n$
\newline
所以 $|I_n|=b_n-a_n>c_2-c_1$,与$I_n<\epsilon$矛盾}
\newline

\textbf{覆盖}
\newline
集合族$S=\{X\}$,若$Y\subset\sum\limits_{X\in S}X$,则集合族$S$覆盖$Y$
\newline

\textbf{有限覆盖引理}
\newline
覆盖闭区间的开区间族总有覆盖该闭区间的有限子族
\newline
\textit{记开区间族$S=\{U\}$覆盖闭区间$I_1=[a,b]$
\newline
假设$S$无覆盖$I_1$的有限子族
\newline
所以 将$I_1$等分为两个闭区间,其中至少有一个不能被任何$S$的有限子族覆盖,记为$I_2$
\newline
继续均分$I_2$,取其中不能被任何$S$的有限子族覆盖者,记为$I_3$
\newline
重复此操作,得闭区间套$I_1\supset I_2\supset I_3\supset\dots$
\newline
且$|I_n|=\frac{|I_1|}{2^{n-1}}$
\newline
所以 闭区间套$\{I_n,n\in\mathbb{N}\}$有唯一公共点$c$
\newline
所以$\exists U=(\alpha,\beta)\in S, c\in U$
\newline
取$I_n$,使得$|I_n|<\min\{c-\alpha,\beta-c\}$
\newline
又因为$ c\in I_n$
\newline
所以$I_n\subset U$,与$I_n$不能被任何$S$的有限子族覆盖矛盾}
\newline

\textbf{极限点}
\newline
(1)若点$p\in\mathbb{R}$的任何邻域与集$X$的交集均有不等于$p$的元素,则$p$为$X$之极限点
\newline
(2)若点$p\in\mathbb{R}$的任何邻域与集$X$的交集为无穷集,则$p$为$X$之极限点
\newline
1$\longleftrightarrow$2
\newline
\textit{若点$p\in\mathbb{R}$的任何邻域与集$X$的交集为无穷集,则此交集中必有不等于$p$的元素(2$\rightarrow$1)
\newline
若点$p\in\mathbb{R}$的任何邻域与集$X$的交集均有不等于$p$的元素
\newline
任取$p$的$\Delta$邻域$U(p,\Delta)=(p-\Delta,p+\Delta), \Delta>0$
\newline
假设$U(p,\Delta)X=\{a_1,a_2,a_3,\dots,a_n\}$为有限集
\newline
则集合$\{|p-a_1|,|p-a_2|,|p-a_3|,\dots,|p-a_n|\}$有最小值$m$
\newline
则取$\delta<m$,邻域$U(p,\delta)$与集$X$的交集没有不等于$p$的元素
\newline
所以 1$\rightarrow$2}
\newline

\textbf{极限点引理}
\newline
无穷有界数集至少有一个极限点
\newline
\textit{记$X\subset\mathbb{R}$为无穷有界集
\newline
取闭区间$I=[a,b]$覆盖$X$
\newline
假设$I$中无$X$的极限点
\newline
对每个$x\in I$,取其邻域$U(x)$,使$U(x)$与$X$之交集为空集或非空有限集
\newline
由此得开区间族$\{U(x)\}$覆盖$I$
\newline
所以可选出有限个开区间$U(x_1),U(x_2),U(x_3),\dots,U(x_n)$覆盖$I$,亦覆盖$X\subset I$
\newline
所以 $\sum\limits_{i=1}^{n}U(x_i)$为空集或有限集,与包含无穷集$X$矛盾}

\end{document}