\documentclass[11pt]{article}

\usepackage{setspace}
\usepackage[colorlinks,linkcolor=blue,anchorcolor=red,citecolor=black]{hyperref}
\usepackage{lineno}
\usepackage{booktabs}
\usepackage{graphicx}
\usepackage{float}
\usepackage{floatrow}
\usepackage{subfigure}
\usepackage{caption}
\usepackage{subcaption}
\usepackage{geometry}
\usepackage{multirow}
\usepackage{longtable}
\usepackage{lscape}
\usepackage{booktabs}
\usepackage{natbib}
\usepackage{natbibspacing}
\usepackage[toc,page]{appendix}
\usepackage{makecell}

\title{Atomism}
\date{}

\linespread{1.5}
\geometry{left=2cm,right=2cm,top=2cm,bottom=2cm}

\begin{document}
\begin{sloppypar}
  \maketitle

  \linenumbers
Atomism is often attributed to Democritus (460 BC-370BC). 
It is said that Democritus is the pupil of Leucippus, about whom we almost know nothing. 
Some scholars even double the existence of Leucippus. 
Although we have enough material to gain a comprehensive knowledge about atomism, the authorship shall be doubled. 
It is said that the works of Democritus and Leucippus has intermingled, and it is nearly impossible now to tell which part belongs to the former or the later.

\section{Metaphysics} 
Atomists agree with Eleatic school that absolute change is impossible, but relative change caused by the motion of permanent particles, or in other words, atoms, do exist. 
To make motion possible, a clear concept of empty space, or non-being is needed. 
Without empty space, movement would be unthinkable. 
The non-being is real. 
It is the empty space in which atoms move. 
Space is not corporeal, but is as real as objects and possesses its own properties. 
The atoms and the space in which they move are the only two realities. 
Being is not one and continuous as Parmenides thinks, but plural and separated into different parts by space. 

\newline

As for atoms, they are underived, indestructible, unchangeable and indivisible particles of objects. 
An atom is not a mathematical point of no volume, no area and no length, but a particle which cannot be divided into parts physically. 
All atoms are the same in qualities, but different in quantitative levels like shape, weight, size and position. 
The motion of atoms is responsible for the change and transformation of things: 
origin means union; 
destruction means separation; 
and transformation means rearrangement. 
Things differ because they are composed of different atoms and the ways that atoms distribute in them differ.

\newline

Then, what makes atoms move? 
Atomists insist on mechanical law and refuse to introduce spiritual concept. 
Space cannot make atoms move, hence motion should be regarded as a inherent property of each atom. 
It is a quality of atoms and is underived like atoms themselves.
  
\section{Cosmology}
Atoms are heavy and fall downward, but the larger ones fall faster. 
So the lighter atoms are forced upward. 
This causes a whirling motion that keeps extending and atoms of similar size and weight are collected. 
The heavy atoms at the centre form solid earth, water and air. 
The light atoms form the heavenly fire and the ether. 
Multiple worlds are produced in this way. 
Each system has a centre and forms a sphere. 
Worlds can be different from each other. 
Some may have no moon, while others may have more or less planets compared with our world. 
  
\section{Psychology}
The soul is composed of the finest, roundest, most nimble and fiery atoms, which are scattered over the whole body and are the cause of movements. 
Certain organs are the seat of particular mental functions: 
the brain for thought, the heart for anger, the liver for desire. 
We inhale and exhale soul-atoms and life exists only when this process continues. 
At death, the soul-atoms are scattered.

\newline

Sense perception is explained as a change in the soul-atoms caused by emanation. 
Objects modify the arrangement of atoms nearby and throw off emanations. 
The emanations spread by influencing the arrange of atoms, reach the organs of sense and influence the soul-atoms. 
Thus, we sense things and reliable information about what we sense is produced. 
However, the process of spreading of emanations may be interfered by each other, then illusions result.
  
\section{Theory of Knowledge}
Atomists’ view on knowledge is pretty modern and scientific. 
Sense do provide reliable information about objects, like the colour, taste, smell and so on, but this information about the qualities of objects is gross and obscure knowledge. 
This is because since atoms of which everything is composed of are qualitatively the same, there should be no qualitative difference between things, and senses can be interfered. 
Rational thought, which is based on sensible information and transcends it, is the only way to transfer gross knowledge and appearance to genius knowledge, the atoms, which we cannot see, but we can think. 
  
\section{Theology}
Atomists do not deny the existence of gods, but make them unlike gods by making them subjective to the impersonal law of the motion of atoms. 
There are gods, and they are also composed of atoms. 
They are mortal like us, though gods are longer-lived, more powerful and possess reason of a high order. 
We, mortals, can know about gods, while gods do not interfere affairs of men. 
So they need not to be feared or propitiated.
  
\section{Ethics}
The end of all conduct is well-being, the satisfaction accompanies with the exercise of the rational faculties, rather than pleasures of senses brought by material wealth. 
The true end of life is happiness, which is described as an inner state of satisfaction or pleasure, depending on the tranquillity, harmony, fearlessness of the soul. 
It does not depend on material goods nor on the pleasure of the body, but on moderation on pleasure and symmetry of life. 
The less we desire, the less apt we are to be disappointed. 
The best way to achieve the goal is to exercise one’s mental powers, through rational reflection and contemplation of beautiful acts. 

\newline

All virtues are valuable, because they contribute to happiness. 
We should do right, not from fear of punishment, but from a sense of duty. 
To be good, one must not merely refrain from doing wrong, he must not even desire to do wrong. 
And we ought to serve the state, because "when the state is in a healthy condition, all things prosper; when it is corrupt, all things go to ruin". 

\newline

Atomists know that individual person is a part of the society and he cannot live alone:
"A well-administrate state is our greatest safeguard". 
\end{sloppypar}
\end{document}