\documentclass[11pt]{article}

\usepackage{setspace}
\usepackage[colorlinks,linkcolor=blue,anchorcolor=red,citecolor=black]{hyperref}
\usepackage{lineno}
\usepackage{booktabs}
\usepackage{graphicx}
\usepackage{float}
\usepackage{floatrow}
\usepackage{subfigure}
\usepackage{caption}
\usepackage{subcaption}
\usepackage{geometry}
\usepackage{multirow}
\usepackage{longtable}
\usepackage{lscape}
\usepackage{booktabs}
\usepackage{natbib}
\usepackage{natbibspacing}
\usepackage[toc,page]{appendix}
\usepackage{makecell}

\title{Sophists}
\date{}

\linespread{1.5}
\geometry{left=2cm,right=2cm,top=2cm,bottom=2cm}

\begin{document}
\begin{sloppypar}
  \maketitle

  \linenumbers
From philosophers of Milesian school to atomists, Greek philosophy has made great progress. 
They has focused on the problems of the world, resulting in the profound change in related views and conceptions. 
This is plainly showed by the transformation from a universe filled with gods and mythical beings to a world of mechanical doctrines proposed by atomists. 
However, the spirit of free inquiry, independent thinking, reflecting and criticizing is not confined in philosophy, but permeates to other fields. 
In literature, writers’ views of life are deepened and broadened by criticism, which would be obvious by a comparison between the dramatic poems of Aeschylus (525 BC-456 BC), Sophocles (496 BC-405 BC) and Euripides (480 BC-406 BC). 
In historical research, legends and other forms of superstitions are discredited, although they used to be acceptable to some extend. 
Herodotus (484 BC-420 BC) paves the way for a critical study of history and Thucydides (460 bc-400 BC) is one of the most outstanding representatives in this field. 
In medicine, old ideas are abandoned and a need of new knowledge of man and nature is felt. 
Some philosophical theories, like the four elements of Empedocles, are applied in the art of healing and lead to the direction of scientific research of medicine, showing the importance of observation and experiments. 
The zeal for investigations is intense and extends to all sorts of problems.

\par

The progress of thought is accompanied with the improvement of the whole society. 
The Persian Wars (500 BC-449 BC) makes Athens the center of commerce, art and academe. 
Different ideas intermingle with each other and give an impetus to thought and action. 
Old views and traditions are examined and in many cases, lose their credit, resulting in the demand for new thoughts, views and conceptions. 
Individual is wondering the meaning of life instead of merely living it. 
He is pondering it and is no longer content to give voice to traditional views of his nation, but is prompted to set forth his own personal thoughts and theories. 
This inevitably leads to the rise of skepticism and individualism, and Sophists are the representatives of this new movement. 

\section{Theory of Knowledge}
The term "sophist" originally means a wise and skillful man, and it gradually becomes the name of a professional teacher who travels around, giving instructions in the art of thinking and speaking for payment, and preparing young men for political life. 
Protagoras, the representative of Sophists, once said to a young man:
\newline
\textit{If you associate with me, on the very day you will return a better man than you came.}
\newline
When Socrates asked him how he would bring this about, he answered:
\newline
\textit{If he comes to me, he will learn that which he comes to learn. And this is prudence in affairs, private as well as public; he will learn to order his house in the best manner, and he will be able to speak and act for the best in affairs of the state.}
\newline
As it is shown here, the main aim of Sophists is practical, but this has no influence on the fact that they start the investigations in new fields. 
Attention is paid to moral and political questions, leading to a systemic and thorough treatment of theory of knowledge, ethics and politics. 
For the first time, philosophers’ line of sight is moved from the objective world to human knowledge and conduct. 

\par

Sophists represent the new time, but the name Sophist gradually becomes a term of reproach. 
This is partly attributed to the fact that Sophists take pay, partly to the degeneration to subjectivism and relativism, resulting from their anxiety to make their pupils efficient and the growth of individualism. 
Individualism means that individuals have their own ideas,and it would be nature that these ideas are highly diversified. 
Then who is right? 
What makes one’s views better than others’? 
The growth of subjectivism and relativism makes the answer become “Nothing”. 
So what one happens to think true is true, what one happens to think right is right. 
One’s ideas and acts are just as good as others’. 
Everyone is right, and there is no truth, only different views. 
Such a tendency and the anxiety to make students efficient make the instructions of Sophists become to teach their pupils how to vanquish an opponent by fair means, to win a debate with all sorts of logical fallacies. 
A reference from Thucydides may help understand the chaos caused by the corruption of Sophists:
\newline
\textit{The common meaning of words was turned about at men’s pleasure; the most reckless bravado was deemed the most desirable friend; a man of prudence and moderation was styled a coward; a man who listened to reason was a good-for-nothing simpleton.}
\newline
Thus, the arguments of Sophists corrupt into the art of making the worse appear the better, making the black appear the white. 
Until today, the word “sophist” is still can be understood as “a person who reasons with clever but fallacious arguments”.
  
\textit{The Theory of Knowledge}
The Sophistic movement is not primarily a moral and religious reformation, although it does contain such aspects. 
The aim of Sophists, in the first instance, is a protest against the paradoxical conclusions drawn by nature philosophers. 
Earlier philosophers focused on the problem of substance and change, and their answers are diversified. 
The substance, or substances by which all objects are composed of, could be fire, air, water, or all of them together. 
The answer could even be more weird: the \textit{Infinite}, numbers, atoms and so on. 
As for the problem of change, things are not better. 
One says that change is impossible, and another one says there is nothing but change. 
Still, there are some people trying to muddle through this by distinguishing change into absolute level and relative level. 
Facing such a mess, Sophists turn their attention on human himself. 
They realize that human mind is an important factor in the process of knowing and reach a conclusion that all the paradoxical theories of their predecessors are caused by the limitation of human mind. 
Human’s intelligence is not qualified to solve the problem of the universe and gain the absolute truth. 
As a result, all the knowledge we get is dependent on the people who propose it. 
So what seems to be true to an individual is true for him, and nobody has ever gotten absolute truth. 
What thinkers has proposed are just subjective opinions. 

\par

Then, what we can do to deal with these conflictual opinions?
Protagoras (485 BC-420 BC) teaches that they are all true, which lies in his saying: 
\newline
\textit{Man is the measure of all things.} 
\newline
The word “man” is not man generically, but the individual man. 
The individual is the law of himself in matters of knowledge. 
Opinions from different individuals are different, even opposed to one another, but they can be all true. 
Then, the task of thinkers is not to demonstrate the truth, but to persuade people to embrace one of those opposing statements rather than others. 
Even two opposing theories can be both true, Protagoras still believes that one can be better than the other. 
Now here comes a paradox: Protagoras holds the view that two conflictual ideas are both true, then what can make one better than the other, resulting in the conclusion that we should embrace one of them? 
In other words, Protagoras is guilty of reinstating a standard of truth after having denied all standards.

\par

Facing the great mess of conflictual views, Protagoras goes to one extreme by asserting that they are all true, while some philosophers, like Gorgias (483 BC-375 BC), goes to another extreme, declaring that they are all fake. 
In Gorgias’ works, he forms a completely negativistic philosophy in three statements: 
first, there is nothing; 
second, even if there were something, we could not know it; 
third, even if it existed and we could know it, we could not communicate this knowledge to others. 

\par

Sophistic theory of knowledge is negative and skeptical. 
It reflects and criticizes knowledge itself, but it still has positive sides. 
The dialectical arguments used by Sophists, although the primary aim is to confute their adversaries, paves the way for the dialectic of Plato and the logic of Aristotle. 
Sophists also recognize the pragmatic aspects of knowledge. 
Absolute theoretic truth maybe unattainable, but relative knowledge achieved by individuals can still be put into practice and used in affairs of life. 
Thus, the original form of pragmatism is offered.

\section{Ethics}
The negativistic trend appears in the theory of knowledge should have had its place in Sophistic ethics. 
If knowledge is impossible, then knowledge of right and wrong is impossible, so there is no universal right and wrong, and morality is just a subjective concept. 
Or in another line of argument, if people’s diversified and conflictual opinions can be all true, then their opinions on the knowledge of right and wrong can be all true, too. 
So there is no universal right and wrong, and morality is just a subjective concept. 
Such an attitude can originate from the diversity of customs, morals and traditions of different nations, similar to the growth of skepticism in knowledge caused by the conflictual theories of nature philosophers.

\par

However, such a nihilistic attitude to ethics is not drawn by Sophists, attributed to the spirit of pragmatism, which also appears in Sophistic theory of knowledge. 
Earlier Sophists, like Protagoras and Gorgias, tend to insist that all established morality and laws, are merely conventional, but emphasis their necessity at the same time. 
Certain moral and legal rules must be employed, if there is to be any social and moral order at all. 
The moral and social conventions raise man above the level of brute and transform him into a social animal. 
Later Sophists does not go to the extreme of nihilistic ethics, neither. 
They set a distinction between nature and convention, and attribute moral standards and rules of conduct to the latter. 
So, morality is not grounded in the very nature and constitution of things, but the product of convention and arbitrary agreement among men. 
According to some, laws are made by the weak, the majority, in order to restrain the strong, the best, to hinder the fittest from getting their due. 
Hence, human laws violate the justice of nature, which is the right of the stronger. 
While according to others, laws are made by the few, the strong, in order to promote their own interests. 
The two versions of Sophistic doctrines are conflictual, but share a common core that laws and morality are conventional device for the promotion of group interests, emphasizing their pragmatic significance.
\end{sloppypar}
\end{document}