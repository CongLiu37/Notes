\documentclass[11pt]{article}

\usepackage{setspace}
\usepackage[colorlinks,linkcolor=blue,anchorcolor=red,citecolor=black]{hyperref}
\usepackage{lineno}
\usepackage{booktabs}
\usepackage{graphicx}
\usepackage{float}
\usepackage{floatrow}
\usepackage{subfigure}
\usepackage{caption}
\usepackage{subcaption}
\usepackage{geometry}
\usepackage{multirow}
\usepackage{longtable}
\usepackage{lscape}
\usepackage{booktabs}
\usepackage{natbib}
\usepackage{natbibspacing}
\usepackage[toc,page]{appendix}
\usepackage{makecell}

\title{Early Christian Theology}
\date{}

\linespread{1.5}
\geometry{left=2cm,right=2cm,top=2cm,bottom=2cm}

\begin{document}

  \maketitle

  \linenumbers
The period of ancient philosophy officially ends at 529, with Justinian's edict closing the school at Athens. 
The intellectual life of next era is dominated by Christianity, whose history traces back to first century and begins with patristic period. 

\par

The philosophy of patristic period is treated as a prelude to medieval philosophy. 
On the narrowest interpretation, patristic period ranges from the time of Christ to the death of St. Augustine in 430. 
Interpreted most widely, it also includes further development of Christian dogmas until the Council of Trullo in 692.
It resulted from the fusion of early Christian religion with Hellenistic philosophy, and is much richer in theology than in philosophy. 

\section{Early Christianity}
In the last period of antiquity, Christianity, a new religion appeared in Palestine with the soil of Judalism, was making converts in the Roman world. 
It taught the gospel of a Father-God who is merciful and just and loves all his children alike, and promised the redemption of mankind through his son, Jesus Christ. 
It taught that there was hope for all: no man was too lowly to be saved, and Christ would come again to establish his kingdom of righteousness and love. 
On the judgment day, the pure in heart, however poor and lowly, would enter into glory, and the wicked, however rich and powerful, would be confounded. 
The deliverance from the sinful world and future life of blessedness are conditioned on a virtuous life, repentance, and love of God and man. 
With such inner purity, love and forgiveness take the place of hate and revenge, and would bring human being to the ultrametric salvation. 

\par

In its spread among the great Roman Empire, Christianity could not ignore the philosophical conceptions rooted in this Graeco-Roman civilization. 
This new world-religion arrived when the times were ripe for its appearance: 
the existance of a universal empire; 
the spirit of cosmopolitanism and brotherhood boosted by Stoicism; 
the prevalence of the concept of a spiritual deity taught by late-antiquity philosophers; 
the acceptance of doctrines of immortality in Greek mysteries and Oriental religions; 
the success of Jewish personal God over the abstract notions of metaphysicians. 
Christianity was, in a large measure, a child of Judalism and Hellenic-Roman civilization. 

\section{Early Theology}
In the process Christianity getting its influence among people, it had to justify its faith, to defend itself against attacks of publicists and philosophers, to define its doctrines declaring its attitude towards prevailing Jewish religion and Hellenistic philosophy. 
Jewish-Greek philosophy serves as the best adapted system for this immediate purpose. 
Allegorical explanation of \textit{Old Testament} and doctrine of logos forms the centre of the new religion. 
Judaism is put under the light of Hellenistic philosophy to reconcile thoughts of Greek metaphysicians and Jewish teachers. 

\par

The beginning of Christian theology is found in writings of St. Paul. 
The person Jesus Christ is exalted as the incarnate and unique Son of God, and is interpreted in terms of philosophic concepts originated from Hellenistic world. 
He is identified with God's power and wisdom. 
He is the Logos. 
He pre-existed as the archetypal man, but was created by God.
Thus, religious and philosophic elements begin to be welded together with emphasis on religious part. 
The Logos is interpreted as a person, the Son of a living Father, not a cold philosophical abstraction.

\section{Gnostics}
It is not surprising that in the growth of the union of Greek philosophy and Jewish religion, some thinkers sought to interpret this new religion with the light of philosophic thinking, to rationalize it, to transform faith into knowledge. 
This work started by Gnostics in second century. 
They speculated upon faith to provide a Christian philosophy, a harmony of faith and knowledge. 

\par

Gnosticism is an embryonic scholasticism, though close to mythology it is. 
They asserted that their secret or esoteric doctrines had been transmitted by Jesus Christ to his followers who were able to receive them. 
Christianity is a divine doctrine, while Judaism is a corrupted paganism. 
The Judaism God is false, opposite to the true God. 
Christ, one of the highest spirit, entered a human body in order to free spirits imprisoned in matter by Judaism God. 
Those able to comprehend true teaching of Christ become gnostics and eventually escape from their material bondage via asceticism. 
Those who fail to understand teaching of Christ perish with their matter bodies. 
The material world is a result of fall towards matter the evil. 

\par

Gnostics failed to provide a philosophy of Christianity, and were in conflict with prevailing concepts of Jesus' teaching by reputing \textit{Old Testament} and conceiving Jesus as a man whose body is used by a heavenly spirit. 
At the same time, gnostics exercised considerable influence on the new religion, and provided an impetus to philosophical formulation of Christian theology.

\section{Apologists}
Apologists were active among late first and second centuries. 
They appealed to philosophy in their defense of the faith against the heathen as well as against the fantastic interpretations of Gnosticism. 
They claimed that Christianity was both philosophy and revelation. 
The truths of Christianity are of supernatural origin and absolute certain, but they are rational truths, even though they can only be comprehended by a divinely inspired mind. 

\par

The fundamental thought in the writtings of the Apologists is: 
The world, though perishable, exhibits traces of reaseon and order, and points to one eternal, unchangable, good and just First Cause, the source of all life and being. 
Reason, or the Logos, must be a part of the inner nature of the God, due to the order and purpose in the universe. 

\par

God emits the Logos, which proceeds from his as the light proceeds from the sun. 
As the light emitted from the sun does not separate from the sun, so the divine reason does not separate from God. 
The Logos remains with the Creator, but is conceived as another personality at the same time, a personality identical with God in essence, Jesus Christ. 

\par

The personification of divine reason has been seen in the Greek philosophy of religion. 
Reason is the instrument by which God acts on the world. 
The transcendency of God is proclaimed, yet there is attempt to maintain the independence of the Logos. 
The Logos is conceived to be of same essence with the God, but originated from the God. 
It is a creature subordinated to the God, and becomes the person Jesus Christ by God's will. 

\par

The creation of the world is interpreted in accordance with Greek ideas of materials and forms. 
The Creator fashioned the world from formless matter, which he created out of nothing, in conformity with the Logos, the divine intelligence, the pattern of all created beings. 

\par

Creation is an expression of God's love and goodness and is for the benefit of man, but the goal of man resides in the hereafter rather than this world. 
The highest good of man is the withdrawl of the soul from the world of sense to God. 

\par

God created souls with the capacity to distinguish between good and evil, and the freedom to choose between them. 
Some souls turned toward the flesh and away from the God. 
As a punishment for this sin, they fall into a lower life in carnal bodies. 
Man, by leading a Christian life, may regain his lost estate through divine grace and the truth of the Logos. 
On the day of judgment, the soul of just will enter eternal life while the unjust will be rejected forever. 

\par

The first man or heavenly soul brought sin into the world, for which mankind is suffering. 
However, there is hope for the ultimate redemption if man will turn away from sense and seek to be reunited with God. 
This redemption is Jesus Christ, the son of God who came to deliver man from sin. 
This proposition gave rise to a series of problems on the relationship of God, the Logos, Jesus Christ and man, on which Christian theologians debated for centuries. 

\section{The Logos and the God}
\section{Jesus and the God}
\section{Free will and original sin}


\end{document}