\documentclass[11pt]{article}

\usepackage{setspace}
\usepackage[colorlinks,linkcolor=blue,anchorcolor=red,citecolor=black]{hyperref}
\usepackage{lineno}
\usepackage{booktabs}
\usepackage{graphicx}
\usepackage{float}
\usepackage{floatrow}
\usepackage{subfigure}
\usepackage{caption}
\usepackage{subcaption}
\usepackage{geometry}
\usepackage{multirow}
\usepackage{longtable}
\usepackage{lscape}
\usepackage{booktabs}
\usepackage{natbib}
\usepackage{natbibspacing}
\usepackage[toc,page]{appendix}
\usepackage{makecell}

\title{Socrates}
\date{}

\linespread{1.5}
\geometry{left=2cm,right=2cm,top=2cm,bottom=2cm}

\begin{document}
\begin{sloppypar}
  \maketitle

  \linenumbers
Sophists attach too much importance to the diversity of human and fail to recognize the universal elements in man. 
They exaggerate the differences in human views but ignore the agreements, resulting in skepticism and criticism, which do lead thinkers to a new area of human knowledge and conduct and bring philosophy from heaven to dwellings of men. 
By employing all sorts of logical fallacies, they make the study of the correct laws of thinking necessary, leading to the birth of logic. 
Sophists are dukes in the kingdom of philosophy, and the division of the kingdom leads to intellectual and moral chaos. 
Now an emperor is needed to end such a mess and unify the kingdom of philosophy, defending knowledge against skepticism and showing how truth may be reached by a logical method. 
This emperor appears in Socrates (469 BC-399 BC), one of the greatest figures in the history of thought, the intellectual father of a line of philosophers whose ideas and ideals play a dominant role in western civilization for two thousand years, and still have an important place in today’s speculations.

\section{The Socratic Problem}
The skepticism and relativism of Sophists mainly appear in the theory of knowledge. 
Although these philosophers hold a conservative attitude to ethics for the sake of pragmatism, morality and the state would finally corrupt from their foundations if skepticism was the last word of the age. 
Socrates sees clearly that the problem of knowledge is the core of the whole situation. 
With an optimistic faith in the power of human reason, he enters his mission to fight against fallacies, to reach the truth and to defend knowledge.

\par

In order to reach the truth, we must not believe opinions that enter our minds by chance. 
With these undigested opinions, a lot of prejudices would fulfill our minds and for the majority of them we do not understand their meaning. 
In this case, we do not have genuine knowledge. 
We have built our intellectual house on sand and it would inevitably collapse unless we reconstruct it on solid foundations. 
To make it, we must check all the terms and conceptions we use, understand their real meaning and define them correctly. 
Furthermore, we should find reasons for our views, think about them and prove them instead of guessing. 
Our views must be verified by facts, and be modified and corrected accordingly. 
Facing the diversity of opinions, it is our duty to figure out their exact meanings, discover whether there are fundamental agreements or principles that can at least make some opinions all stand. 
To reach universal judgement is the main aim of Socrates and in order to make it, he has developed a unique method, the Socratic method with an ingenious form of cross-examination.
  
\section{The Socratic Method}
The Socratic method is dialogic. 
Generally, in discussion of a subject, Socrates starts with popular or hastily formed opinions. 
Then he tests them by illustrations from daily life, showing that these opinions are not well-funded and need to be defined more precisely. 
Then by suggesting instances of relevance, Socrates helps others who take part in the dialogue form correct opinions and dose not stop until the truth has been developed step by step.
Hence, definitions are evolved in a way of induction. 
They are further examined by more examples, and broadened or narrowed, until all essential characters of the subject are found and satisfactory definitions are finally reached. 
At times, Socrates uses deductive method, testing the statement by criticizing the principles from which it origins or deducing what the statement would result in.

\par

Thus, knowledge is possible, as long as proper method is employed. 
In this method, all terms are defined clearly and properly, and the first principle, on which all reasoning is based, is constructed on solid foundations to make the whole argument make sense. 
It shall be learned that negative instances that make a statement invalid play an important role in Socratic method. 
For example, we may say that cheating is wrong. 
Then Socrates would ask: 
how about cheating a general of hostile country to win a victory? 
Is it right or wrong? 
It is right. 
Then the original statement that “cheating is wrong” is invalid and it need to be modified. 
The usage of negative instances indicated that Socrates knows that knowledge concerns the general and the typical, which is not learned by Sophists. 
For Sophists, they would emphasize that cheating enemies is right, while cheating friends is wrong. 
The two statements are both true, but provide opposite opinions on whether cheating is right or wrong. 
Thus, cheating is both right and wrong, and there would be no knowledge about it. 
But for Socrates, facing the opposite but all true opinions, he goes back to the original statement that “cheating is wrong” and tries to solve the contradiction by defining the core conceptions of the problem, which is true knowledge, and by modifying the original statement to make it general and typical, instead of denying knowledge. 
Hence, Socrates defends knowledge from nihilism using his method.

\par

With emphasis on the importance of Socratic method, it must be cleared that Socrates has nothing to do with methodology. 
All scholars have their own methods of speculation, but many of them have not developed theories of method, or methodology. 
Socrates has put his unique method into practice, but not developed the theory of the method he uses. 
Today, we may study Socratic method in a way of methology. 
It is based on the interpretation of the works of Socrates’ students, especially Plato and Xenophon.

\par

Overall, some characteristics of Socratic method can be distinguished. 
First, it is skeptical. 
The Socrates method starts with checking popular opinions and pointing out that they are not well-funded. 
Socrates share the skepticism with Sophists, although their philosophies develop towards different directions. 
Second, it is conversational. 
To Socrates, dialogue is a technique for the discovery of truth. 
Soon it would show its influence in Plato’s works. 
Third, it is definitional. 
Socrates’ final goal is gaining correct definitions or conceptions of subjects, which are tacitly assumed as knowledge. 
However, it shall be noted that precise definitions are indispensable to knowledge, but definitions alone cannot constitute knowledge. 
Besides, Socrates’ research area is matters of human, namely morality, ethics and politics. 
He does not see the value of cosmology concerned by pre-Sophistic philosophers. 
Forth, it is inductive. 
In Socratic method, definitions are criticized and corrected by reference to particular instances. 
Fifth, it is also deductive, for in checking definitions, their original statements or consequences are examined.
  
\section{Ethics} 
Socrates focuses on human problems, especially the problems of morality. 
Before him, the radical Sophists have seen ethical ideas and practice as merely conventions; while other conservatives have regarded them as self-evident. 
To conservatives, morality is not something that one can reason, but one has to obey. 
As for Socrates, he endeavours to understand the true meaning of morality, to find the criterion of right and wrong, to learn how human beings should act. 
For these problems, there must be universal answers that all rational beings would recognize and accept as long as they think them through. 
All these problems are summarized as a final question: what is the highest good, for the sake of which all else is good?

\par

Socrates tries to find the source of value, the highest good, and his answer is knowledge. 
The central thesis of Socratic ethics lies in his saying: 
“Knowledge is virtue.” 
To act rightly, one must think rightly and have relevant knowledge. 
One must know what virtue is and understand its meaning, so he could be virtuous. 
Without the knowledge of virtue, one cannot be virtuous, but knowing what virtue is, he will be virtuous. 
Thus, knowledge is both the necessary and sufficient condition of being virtuous. 
Being good is dependent on knowing what good is, and once he knows the meaning of good, he cannot refuse it and turn to be evil. 
No man voluntarily pursue evil or that which he thinks to be evil. 
To prefer evil to good is not in human nature; 
and when a man is compelled to choose between two evils, no one will choose the greater when he may have the less.
Thus, to Socrates, the knowledge of right and wrong is also a firm practical conviction, a matter not only of intellect, but of will.

\par

Since a parallel can be drawn between knowledge and virtue, Socrates makes some deductive implications. 
Knowledge is unity, an organized system of truths with a central statement, a start point of the whole system. 
So virtue is one, containing different forms, and there is something central to make all these aspects belong to virtue. 
Furthermore, virtue is not only good in itself, but also in human interest. 
The effect of being virtuous is to make life painless and pleasant. 
Hence, virtue and true happiness are identical. 
No one can be happy without being virtuous.

\end{sloppypar}
\end{document}