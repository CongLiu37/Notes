\documentclass[11pt]{article}

\usepackage[UTF8]{ctex} % for Chinese 

\usepackage{setspace}
\usepackage[colorlinks,linkcolor=blue,anchorcolor=red,citecolor=black]{hyperref}
\usepackage{lineno}
\usepackage{booktabs}
\usepackage{graphicx}
\usepackage{float}
\usepackage{floatrow}
\usepackage{subfigure}
\usepackage{caption}
\usepackage{subcaption}
\usepackage{geometry}
\usepackage{multirow}
\usepackage{longtable}
\usepackage{lscape}
\usepackage{booktabs}
\usepackage{natbib}
\usepackage{natbibspacing}
\usepackage[toc,page]{appendix}
\usepackage{makecell}

\title{玄学的兴起}
\date{}

\linespread{1.5}
\geometry{left=2cm,right=2cm,top=2cm,bottom=2cm}

\begin{document}

\maketitle

\linenumbers

自东汉末年至两晋,“清谈”盛行。
清谈所涉及之话题有一大致范围,而在此范围内提出的意见主张,亦大致表现出相近的思想倾向。
对此类言论加以命名,即为玄学。

\section{玄学与儒道}
先秦儒道二家由于价值观念的区别,本是对立关系。
盖二家虽均重视人之自觉,但孔孟据此言德性之价值;
老庄则谓自觉心应当超越万物,观赏世界之流变。
经汉代思想大混乱,学者不解儒道本旨,反随意摘取诸家学说,任意发挥,实未能继承任何一家的学说,发展更是无从谈起。

\newline

在这种历史条件下,学者逐渐混杂儒家和道家理论。
扬雄已兼有儒道色彩,但在立说时仍不曾将孔老混为一谈。
清谈名士则直接以为孔子和老子的学说为同一派。
例如,《世说新语·文学》载:

\textit{王辅嗣弱冠诣裴徽,徽问曰:“夫无者,诚万物之所资,圣人莫肯致言,而老子申之无已,何邪?”弼曰:“圣人体无,无又不可以训,故言必及有;老、庄未免于有,恒训其所不足。”}

为什么孔子不言“无”而老子总是说“无”,这个问题实在是无聊至极。
在现代社会问这种问题的一般被称为“标题党”。
没事的时候刷刷手机,便能找到一大堆这种货色。
值得注意的是,无论是裴徽(?-?)还是王弼(字辅嗣,226-249),均默认孔老为同一学派的不同境界的代表,王弼甚至认为孔子高于老子,足见此二人不知孔老之学。
由此可见清谈之流混乱儒道的风气。

\newline

尽管如此,魏晋名士的思想总体上还是倾向于道家,其旨趣在于形而上学和放荡生活两方面。
前者欲追老子“道”的观念,但相关理论浅薄;
后者欲追老庄“观赏世界”之逍遥境界,实际上误执形躯,未解老庄所论之真正自我。

\newline

总之,清谈之流大致持道家理论立场,尽管对老庄真意存在误解。
此辈混杂儒道,故在立说时往往援引儒家典籍,随意发挥甚至曲解,甚至将《道德经》、《南华经》和《易经》并列,谓之“三玄”,实在是有够“悬”的。

\section{才性派和名理派}
清谈之辈持道家立场,可分为才性和名理两派。

\newline

才性派以“人”本身为对象,评论其天赋气禀。
对此需先进行说明。倘若以人为研究对象,可关注人之自觉心,或者说主观能动性,所涉及的问题为“应该如此”;
亦可以人的存在为一客观事实,探究其构造和运行,所涉及的问题为“实际如此”和“必然如此”。
前者为伦理学、政治学或宗教的领域,即所谓“心性论”;
后者则一般为自然科学之领域。
清谈之流言“才性”,将其视为一客观事实,排除自觉心的作用,却又不持自然科学立场,仅将人视为一整体并加以判断。
才性派之流的态度更接近观赏或者品鉴,其议论类似艺术评论,仅阐述议论者的直观感受。
总之,才性派持观赏态度,显为道家精神的衍生。
但此辈于自觉心全无立场,不解老庄逍遥物外、观赏世界之真正自我,反执着于作为客观事实的才性,实误坠形躯,不明“薄形骸”之意。

\newline

名理派侧重形而上学观念的描摹和发挥,亦时有涉及逻辑学和知识论。
可见,名理一派实源于老庄之形而上学旨趣。
但此辈仅持观赏态度,非真有意于理论研究,故其议论浅薄,仅能代表一种思想倾向,实难以称之为一“学”。

\end{document}