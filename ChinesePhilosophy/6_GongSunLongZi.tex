\documentclass[11pt]{article}

\usepackage[UTF8]{ctex} % for Chinese 

\usepackage{setspace}
\usepackage[colorlinks,linkcolor=blue,anchorcolor=red,citecolor=black]{hyperref}
\usepackage{lineno}
\usepackage{booktabs}
\usepackage{graphicx}
\usepackage{float}
\usepackage{floatrow}
\usepackage{subfigure}
\usepackage{caption}
\usepackage{subcaption}
\usepackage{geometry}
\usepackage{multirow}
\usepackage{longtable}
\usepackage{lscape}
\usepackage{booktabs}
\usepackage{natbib}
\usepackage{natbibspacing}
\usepackage[toc,page]{appendix}
\usepackage{makecell}

\title{公孙龙子}
\date{}

\linespread{1.5}
\geometry{left=2cm,right=2cm,top=2cm,bottom=2cm}

\begin{document}

  \maketitle
  
  \linenumbers

先秦名家,以惠施和公孙龙为代表,散见诸古籍者另有邓析、桓团等。
名家者流重认知活动,致力于逻辑问题和形而上学思辨,有作认知探究之立场,倾向于纯粹思考。
名家之书,仅《公孙龙子》六篇传世,而其中《迹府》系后人伪作。
故研究名家,当以此五篇《公孙龙子》为主要资料。
此外,惠子之学,于《南华经·天下》有所载,亦于文末作一讨论,以为名学之补充。

\newline
公孙龙为赵人。
《南华经·天下》称其为“辩者之徒”,又谓辩者皆出惠子。
可推测公孙龙当少于惠子,而立说于《天下》时代之前。
史传多载公孙龙在平原君赵胜(?-251 BC)处与人辩论事,可知其时代当与平原君同时,而其余事迹皆难考矣。

\section{指物}
“指”即意义或概念,“物”即具体事物或概念所包含之元素。
指物所论,即概念与个体事物之性质和关系,其主旨在于“物莫非指,而指非指”。

\newline

“物莫非指”,即任一具体事物必属于某些概念之范围。
盖事物总有某些性质,而每一性质皆使得该事物属于某一概念。
例如,一切有“白”之性质的事物,皆属于“白”这一概念。
某“白”物亦可有“坚”、“石”等性质,则此物不仅属于“白”概念,亦属于“坚”概念和“石”概念。

\newline

“指非指”则指出概念不能属于某些概念,即任一概念皆不能作为另一概念所包含之元素。
此论点强调“指”非一“物”,二者是不同等级之存在。

\newline

“指”系与“物”不同等级之存在,但一“指”之反面则无实在性。
例如,“白”作为一“指”,具有实在性;而其反面“非白”表“白”性质之缺乏,不具有实在性。
此即公孙龙谓“非有非指”之意。
“非有”即无有,“非指”即否定性概念,系“指”之反面。

\section{白马}
“白马非马”之辩为公孙龙理论中最为人所熟知者。
汉以前言及公孙龙者,皆以白马之论为其代表。
白马之辩,实以指物理论为基础,现详述之。

\newline

首先,“白”和“马”各为一概念,二者皆为具有实在性的“指”,为同等级之存在。
此处需注意的是,“白”为一属性概念,而“马”为一实体概念,二者在认知活动中有先后之分。
盖认知“马”之概念时,仅依实体之“马”;而在认知“白”之概念时,“白”必须表现为“某实体之白”,才能为人所知。
由此,“白”之属性概念似较“马”之实体概念缺少实在性。
更进一步,遂有属性为实体之附庸而不与实体同等级之观点。
故常识独言“白马”或者“白色的马”,而不言“马白”或者“马的白色”。
此为一般性观点,而公孙龙坚持一切概念皆为同等级之实在。
故“马”与“白”为同等级之概念。
“白马”或者“马白”,所表为“马”和“白”二概念之交叠,自身亦为一概念。
由此,“白”、“马”、“白马”为三个同等级的实在性概念,显然彼此不相等。
故有“白马非马”之结论。
其中“非”意为“不相等”。
若依今日之观点,“白马”概念包含于“白”之概念,亦包含于“马”之概念。
但今传《公孙龙子》仅论二概念之不相等,而无包含关系之论述,公孙龙于此之立场亦无从考证矣。

\section{坚白}
白马之论,谓属性概念和实体概念为同一等级,进而从具体事物中抽离其性质,言属性之独立存在。
此点于坚白之论中尤详。

\newline

公孙龙论坚白,从认知历程入手。
《公孙龙子·坚白论》谓:

\textit{视不得其所坚,而得其所白者,无坚也;拊不得其所白,而得其所坚,得其坚,无白也。}

石对于视觉,仅表现其白之属性而不表现其坚;对于触觉,仅表现其坚之属性而不表现其白。
不视则不得其白,不拊则不得其坚。
由此,坚与白二属性通过不同的认知过程显现,遂各为独立之存在而互不相依。

\newline

如此,属性与属性之间的独立性借由对具体事物之感知过程证立,则属性自不能脱离具体事物。
故属性之依于实体得明。
若由此推演,则属性与实体难为同等级之存在,与前述“一切概念均为同等级之存在”相悖。
对此,公孙龙通过区分作为概念之属性和具体事物之属性来解决。
《公孙龙子·坚白论》云:

\textit{物白焉,不定其所白;物坚焉,不定其所坚。不定者兼。恶乎其石也?}

若无“所白”之物,则“白”为一普遍性概念;若无“所坚”之物,则“坚”为一普遍性概念。
文中所谓“兼”,即指普遍性。
作为概念之属性具有普遍性,与实体概念之“石”为同一等级。
而言石之坚白时,“坚”“白”均非普遍性概念,而是某一个体之“石”所具有的性质,故依附于实体。

\section{通变}
《公孙龙子·通变论》讨论概念之间的关系。
兹举其要旨。

\newline

第一, 概念皆为独立之实在,彼此之间同等级,无高下之分。
此说前文已有论述。

\newline

第二, 就“通”和“变”立论。
“通”即一般性,”变“即特殊性。
一切概念皆为一“概念”,彼此平等独立,此系概念之共有性质,即“通”。
而每一概念又必有一组定义条件。
根据此种条件,万物分属于不同的概念。
一切事物皆为物,有物之共性,故可同属于一“物”之概念。
根据各个概念之定义条件,“物”之概念可划分为各种特定概念,此一历程以“变”言之。
各个概念由此遂有特殊性,故彼此不相等。
如以白马之论为例,一切事物皆属于“物”之概念。
对此添加限定条件,进行划分,可得不同概念:以“白”为条件,得“白”之概念;以“马”为概念,得“马”之概念;以“白”和“马”之并列为条件,得“白马”之概念。
此三个概念彼此独立、为同等级之存在且互不相等,故“白马非马”。

\section{名实}
《公孙龙子·名实论》,旨在说明其学说所处理之根本问题。
首以“物”指一切对象,以“实”指每一物之属性。
“位”则指“物”之“实”得到正当定义、描述、决定之状态,而将产生此种正当状态之思考称为“正”。
可见公孙龙学说之纯粹思辨旨趣。

\section{惠子之学}
惠子名施,传为庄子好友,其学仅有《南华经》之材料为凭。
惠子所关注者,为物之同异问题,以为物之一切性质和标准,皆仅仅表相对关系,无绝对性。
盖一切事物皆为物,自然有其共同属性。
据此,万物彼此相同。
然而,各个事物之间又有不同之处,甚至同一物于两个时间点,亦彼此不同。
据此,万物彼此皆不同。
故万物彼此相同而又不同,即“万物毕同毕异”。
  
\end{document}