\documentclass[11pt]{article}

\usepackage[UTF8]{ctex} % for Chinese 

\usepackage{setspace}
\usepackage[colorlinks,linkcolor=blue,anchorcolor=red,citecolor=black]{hyperref}
\usepackage{lineno}
\usepackage{booktabs}
\usepackage{graphicx}
\usepackage{float}
\usepackage{floatrow}
\usepackage{subfigure}
\usepackage{caption}
\usepackage{subcaption}
\usepackage{geometry}
\usepackage{multirow}
\usepackage{longtable}
\usepackage{lscape}
\usepackage{booktabs}
\usepackage{natbib}
\usepackage{natbibspacing}
\usepackage[toc,page]{appendix}
\usepackage{makecell}

\title{《礼记·大学》}
\date{}

\linespread{1.5}
\geometry{left=2cm,right=2cm,top=2cm,bottom=2cm}

\begin{document}

  \maketitle

  \linenumbers
《礼记》系汉代儒生编辑所成。
初,汉河间献王刘德(171 BC-130 BC)得《礼记》百三十一篇。
后刘向(77 BC-6 BC)整理典籍,得河间献王之《礼记》存百三十篇,另外搜集八十篇资料,共二百四十篇。
汉元帝时(43 BC-33 BC),戴德(?-?)整理刘向所传之《礼记》,去除重复,有大戴《礼记》八十五篇。
戴德之侄戴圣(?-?)复删大戴之书,有今传《礼记》四十九篇。
可见,《礼记》为编篡之书,其中资料来源驳杂,考其时代,不早于战国末期。
其内容杂乱,有理论意义者,以《大学》《中庸》为主。
本篇论《大学》之旨。

\newline

《大学》之主旨,在于建立一套理论,说明政治生活完全决定于道德,进而将政治视为德性之延伸。
其文似论“治国”“平天下”,实未能触及政治理论。
此外,《大学》论道德,紧扣道德之实践,于其根源则全无所见。
  
\section{本末}
《大学》开篇云:
  
\textit{大学之道,在明明德,在亲民,在止于至善。知止而后有定,定而后能静,静而后能安,安而后能虑,虑而后能得。物有本末,事有终始。知所先后,则近道矣。}

盖《大学》之目的在于“至善”,此一境界包括“明明德”和“亲民”两个方面。
前者系个人德性,后者则为政治生活。

\newline

由此,如何达到至善之境界?
此一实践问题之理论始于“物有本末,事有终始。知所先后,则近道矣。”
下文又强调“本末”之重要性,谓:

\textit{其本乱而末治者否矣。其所厚者薄,而其所薄者厚,未之有也。此谓知本,此谓知之至也。}
  
盖本末之分决定实践的顺序。
德性实践的各个阶段必须按照顺序逐个进行,其中在前者为本,在后者为末。
倘本尚未完成,则末便无从谈起。
故区分本末为实践之先决条件,即为“知之至”。

\newline

基于本末之分,《大学》提出由本至末的八个实践阶段,即“八条目”:格物、致知、诚意、正心、修身、齐家、治国、平天下。
八条目中,修身以前皆是成德功夫,齐家之后则涉及政治生活,为德性之延伸。
各个阶段本应各有其目的或完成,以“止”字标明。
但今传《大学》并未点出全部八个阶段之目的,而是仅就部分阶段进行论述。
  
\section{八条目}
《大学》对格物、至知两条目并未做进一步论述。
但就字面意义而言,格物似表一种经验主义认知论。
而前文有“此谓知本,此谓知之至也”之语,可据此将至知解为知本末之分。
倘仿《大学》之句式,可总结为:格物致知,止于本末之辨。

\newline

《大学》论诚意,云:

\textit{所谓诚其意者,毋自欺也。如恶恶臭,如好好色,此之谓自谦。故君子必慎其独也。小人闲居为不善,无所不至,见君子而后厌然,掩其不善而著其善。人之视己,如见其肺肝然,则何益矣。此谓诚于中,形于外,故君子必慎其独也。 曾子曰:“十目所视,十手所指,其严乎!”富润屋,德润身,心广体胖,故君子必诚其意。}

诚意之旨在于意志和行为之统一。
善行皆为个人意志之体现,与外在条件无关。
人之所以要行善,是由于自觉心之要求,而非外在条件所致。
人前积德行善,人后恣意妄为,是小人也。
此说实持孟子之理论立场,以为德性之根源在于人之自觉。
若欲行善,只需遵循道德自觉,无需假于外物,即“毋自欺”也。

\newline

《大学》论正心,云:
  
\textit{所谓修身在正其心者,身有所忿懥,则不得其正,有所恐惧,则不得其正,有所好乐,则不得其正,有所忧患,则不得其正。心不在焉,视而不见,听而不闻,食而不知其味。此谓修身在正其心。}

所谓“正心”,即为不受情绪冲动之影响。即避免因情绪冲动破坏是非标准。

\newline

论修身,云:
  
\textit{所谓齐其家在修其身者,人之其所亲爱而辟焉,之其所贱恶而辟焉,之其所畏敬而辟焉,之其所哀矜而辟焉,之其所敖惰而辟焉。故好而知其恶,恶而知其美者,天下鲜矣。故谚有之曰:“人莫知其子之恶,莫知其苗之硕。”此谓身不修,不可以齐其家。}
  
此谓避免因私情破坏是非标准。

\newline

《大学》论“齐家”“治国”“平天下”者,系将政治生活视为个人德性之延伸或附庸,其要旨不过谓个人德性若可为一家、一国或者天下之表率,则众人相从,一切政治秩序皆可成立。
日后儒者言政治,多大谈“圣君贤相”之理想人格,不注目于社会制度,即是此意。 
\end{document}