\documentclass[11pt]{article}

\usepackage[UTF8]{ctex} % for Chinese 

\usepackage{setspace}
\usepackage[colorlinks,linkcolor=blue,anchorcolor=red,citecolor=black]{hyperref}
\usepackage{lineno}
\usepackage{booktabs}
\usepackage{graphicx}
\usepackage{float}
\usepackage{floatrow}
\usepackage{subfigure}
\usepackage{caption}
\usepackage{subcaption}
\usepackage{geometry}
\usepackage{multirow}
\usepackage{longtable}
\usepackage{lscape}
\usepackage{booktabs}
\usepackage{natbib}
\usepackage{natbibspacing}
\usepackage[toc,page]{appendix}
\usepackage{makecell}

\title{《易传》}
\date{}

\linespread{1.5}
\geometry{left=2cm,right=2cm,top=2cm,bottom=2cm}

\begin{document}

  \maketitle

  \linenumbers
《易传》指今传《易》中除卦爻辞外的所有解说卦爻的文字,包括《彖上》《彖下》《象上》《象下》《系上》《系下》《文言》《说卦》《序卦》《杂卦》十篇,又称十翼。
《易传》系汉代儒生杂取战国至秦汉的资料编辑而成,其内容驳杂至极,甚至互相冲突,可取者不多。
此种资料(包括《礼记》)本不足为学者所重视,然宋儒尤其推崇,其历史原因待论述宋代哲学时再加以研究。
总之,欲理解中国哲学之流变,不能不对这些资料之基本立场有所把握。
现论述《易传》之理论。

\section{道}
《易》本是占卜之书,其卦爻组织与人生历程相对应。
而卦爻的命名又与宇宙历程暗合,由此暗示宇宙历程和人生历程的相对应。
这一对应关系在《易传》中得到明确。
《易传》将自然历程和人类行为归于一超越性规律的支配之下。
此规律即“道”。
《易传·系上》云:

\textit{是故形而上者谓之道,形而下者谓之器。化而裁之谓之变,推而行之谓之通,举而错之天下之民谓之事业。}

可见,道为最高根源,贯穿宇宙和人事两个领域。

\newline

道之内容为阴和阳。
《易传·系上》云:
    
\textit{一阴一阳之谓道。继之者善也,成之者性也。}

阴为被动因素,阳为主动因素。
二者相互作用,生成万物。
后两句涉及价值理论,谓符合道即为善。
此处混淆了“应该如此”和“必然如此”两个领域。
道既为掌控万象之超越性规律,则焉能有不继道者?
此一问题《易传》作者未能察觉。

\section{汉代儒学总结}
至此,汉代儒学之主要资料已经全部论述完毕。
现对其做一总结。

\newline

秦燹之后,古学失传。
汉儒不解孔孟之心性论,反而杂取阴阳家之宇宙论及形而上学,以此为基础构建心性理论和价值规范。
就理论之发展而言,董仲舒代表此历程的早期。
董氏将宇宙之运行和人类之行为相对应,谓人之自觉活动应该与宇宙之运转相一致,遂兴“天人一体”之说。
《中庸》《易传》则代表此一历程的晚期。
董氏之学中的宇宙论被形而上学所取代。
一形而上的存在被视为万象之超越主宰和价值根源。
符合此主宰即为善。
但由此产生一个问题:
倘万象皆受一至高的支配,则万象皆符合其要求,故万象皆有价值,皆为善,焉能有恶?
此问题之根源在于对“应该如此”和“必然如此”两个彼此不同的领域的混淆,而《中庸》《易传》皆未能涉及这一方面。

\newline

此外,以道德化成世界为儒者之一贯立场。
但先秦儒学所欲化成之世界,大多仅限于人类社会。
故其学重视政治生活,不关注客观世界。
汉儒基于宇宙论或形而上学立场,其欲化成者明确为整个宇宙,包括人类社会和物质世界。
观《中庸》言圣人“赞天地之化育”诸语,此点甚明。
这种演变进一步加强了对德性的重视和对认知的轻视,削弱了人类对客观世界的掌控。
德性对认知的压缩为儒家之主要弊病,其始发于先秦,加重于汉,恶果则现于明清乃至近代。
 
\end{document}