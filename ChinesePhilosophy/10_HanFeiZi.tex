\documentclass[11pt]{article}

\usepackage[UTF8]{ctex} % for Chinese 

\usepackage{setspace}
\usepackage[colorlinks,linkcolor=blue,anchorcolor=red,citecolor=black]{hyperref}
\usepackage{lineno}
\usepackage{booktabs}
\usepackage{graphicx}
\usepackage{float}
\usepackage{floatrow}
\usepackage{subfigure}
\usepackage{caption}
\usepackage{subcaption}
\usepackage{geometry}
\usepackage{multirow}
\usepackage{longtable}
\usepackage{lscape}
\usepackage{booktabs}
\usepackage{natbib}
\usepackage{natbibspacing}
\usepackage[toc,page]{appendix}
\usepackage{makecell}

\title{韩非子}
\date{}

\linespread{1.5}
\geometry{left=2cm,right=2cm,top=2cm,bottom=2cm}

\begin{document}

  \maketitle
  
  \linenumbers

凡言法家,多举管子、申不害、商君、韩非等。
然申子之学不传,管子、商君之书亦伪。
今欲观法家之说,唯韩非子(280 BC-233 BC)之书可为资料。

\newline

韩非之学,其根本问题在于如何建立一强权统治。
故韩非否定一切价值标准,唯肯定人主之权力。
依韩非,欲避免混乱,必须将一切权力和是非标准归于人主。
为维护人主之权威,亦需有一套驭人之术。
故其学论阴谋权术者颇多。

\newline

韩非之说,以人主之权威为中心,提倡极权主义,对精神价值取纯否定态度,系文化精神之幻灭。
秦制承韩非之学,遂有兵燹之祸。
可以说,先秦时代的诸子百家,最终在韩非子手中被判了死刑,由秦始皇加以执行。
  
\section{韩非子与儒家}
韩非之理论,于儒、道、墨诸家皆有所取,又有所发展和歪曲。
此可反映韩非哲学之基本立场和态度,有必要作一论述。
兹先论韩非思想与儒学的关系。
韩非为荀门弟子,其思想中的儒学成分基本来自荀子,可概括为性恶、人之改造、权威主义三点。

\newline

荀子言性恶,谓人生而具有求利之动物性本能。
韩非子言性恶,则较荀子更为推进。
盖荀子虽言性恶,但尤谓人有学习善恶之能力,可受文化之熏陶改造。
韩非子则谓人只知争利害,毫无道德可言。
《韩非子·六反》云:

\textit{且父母之于子也,产男则相贺,产女则杀之。此俱出父母之怀衽,然男子受贺,女子杀之者,虑其后便,计之长利也。故父母之于子也,犹用计算之心以相待也,而况无父子之泽乎?}
  
可见韩非对人性持完全负面看法,将性恶论推向极致。

\newline

基于性恶论,荀韩皆强调人的改造。
但荀子所言强调礼仪文化之熏陶、道德品质之教化、个人治学之功夫,其基本立场仍是儒者之道德化成。
而韩非所言之改造,实为控制。
《韩非子·显学》谓:

\textit{夫圣人之治国,不恃人之为吾善也,而用其不得为非也。恃人之为吾善也,境内不什数;用人不得为非,一国可使齐。为治者用众而舍寡,故不务德而务法。夫必恃自直之箭,百世无矢;恃自圜之木,千世无轮矣。自直之箭,自圜之木,百世无有一,然而世皆乘车射禽者何也?隐栝之道用也。虽有不恃隐栝而有自直之箭、自圜之术,良工弗贵也。何则?乘者非一人,射者非一发也。不恃赏罚而恃自善之民,明主弗贵也。何则?国法不可失,而所治非一人也。故有术之君,不随适然之善,而行必然之道。}

此处以箭轮喻民众,以工匠喻人主。
言工匠不可指望箭自直、轮自圆,犹谓治国不可寄希望于通过教化使民众自发向善。
良工必须根据自身意志加工箭轮,方可使箭直轮圆,犹言明主必须以一定的手段驾驭、控制民众,使之遵循其命令。
总之,儒者言教化,多少依赖于自觉心之功效;韩非所言控制,则完全否定人之自觉。

\newline
最后,韩非发展荀子之权威主义,处处强调人主之权力。
此为韩非学说中唯一肯定者,可见其权威主义色彩远比荀子浓重。
  
\section{韩非子与道家}
道家思想,其主旨在于观赏之智慧,谓人之自觉不应执着于流变不息的物,而应守虚静以把握“道”,得以朗照万物,观赏物之流变。
“道”为万物流变之根本规律,知“道”自然生出对万物的支配力量。
此为道家思想之自然推论而不是其主旨。
韩非子则抓住此种由静观之智慧生出的支配力,将其用于阴谋,由此生出驭人之术,全无“观赏世界”之逍遥。
依照道家观点,韩非已然陷溺于物质世界,断非至人。
由此可知,韩非与道家之基本立场相去甚远。
古人有谓法家出自道家者,诚不解二家之本旨也。
  
\section{韩非子与墨家}
墨家之学,以兼爱为中心,以功利主义和权威主义为主脉。
韩非之学,有其强调人主利益的功利主义成分,但终究与墨家“兴天下之利”殊途。
韩非主张一切权力收归人主,使被统治者悉数遵循人主之价值标准,与墨子视“一人一义,十人十义”为天下之患同理。
但墨子之权威主义中,价值根源最终归于“天志”,带有宗教色彩。
而韩非不立超越主宰,仅以现实之统治者为最高规范。
故韩非之学,人主取代神权,统一思想,控制人民,凌然不可犯矣。
  
\section{权归人主}
至此,韩非思想之源流既明。
此种否定一切价值标准的立场不惟异于诸子,于世界历史亦不多见。
现论述韩非子之主要理论,就治乱问题始。

\newline

韩非子谓国欲强大,首在奉“法”。
《韩非子·有度》云:

\textit{故当今之时,能去私曲就公法者,民安而国治;能去私行,行公法者,则兵强而敌弱。故审得失有法度之制者,加以群臣之上,则主不可欺以诈伪;审得失有权衡之称者,以听远事,则主不可欺以天下之轻重。}

此处强调法于治国之重要性。
需要指出的是,韩非所言之“法”,是维护人主权威,改造、驾驭民众之工具,而非现代社会之法律。
《韩非子·说疑》云:

\textit{今世皆曰:“尊主安国者,必以仁义智能”,而不知卑主危国者之必以仁义智能也。故有道之主,远仁义,去智能,服之以法。是以誉广而名威,民治而国安,知用民之法也。}

此言“有道之主”以法令控制臣民,而不是以道德感化或以智慧管理。
故国家能否得到治理,在于人主能否控制臣民,而不在于人主之仁义智能。

\newline

人主对臣民的控制,既在于物质层面,亦包括精神层面。
《韩非子·诡使》云:
  
\textit{圣人之所以为治道者三:一曰“利”,二曰“威”,三曰“名”。夫利者,所以得民也;威者,所以行令也;名者,上下之所同道也。}

此谓治国,需以“利”收买人心,以“威”执行政令;并制定共同标准,即“名”。
“利”“威”皆系物质层面的控制,“名”则系精神方面的控制。
韩非又谓,国家混乱,统治失败,其主要原因是容许人民另立价值标准。
《诡使》又云:

\textit{夫立名号,所以为尊也;今有贱名轻实者,世谓“高”。设爵位,所以为贱贵基也;而简上不求见者,谓之“贤”。威利,所以行令也;而无利轻威者,世谓之“重”。法令,所以为治也;而不从法令为私善者,世谓之“忠”。官爵,所以劝民也;而好名义不进仕者,世谓之“烈士”。刑罚,所以擅威也;而轻法不避刑戮死亡之罪者,世谓之“勇夫”。民之急名也,甚其求利也;如此,则士之饥饿乏绝者,焉得无岩居苦身以争名于天下哉?故世之所以不治者,非下之罪,上失其道也。常贵其所以乱,而贱其所以治,是故下之所欲,常与上之所以为治相诡也。}

统治者不能使全体臣民完全接受统治者之价值标准,故不能致治。

\newline

因此,韩非主张统治者需统一言论。
《韩非子·问辩》云:

\textit{或问曰:“辩安生乎?”对曰:“生于上之不明也。”问者曰:“上之不明因生辩也,何哉?”对曰:“明主之国,令者,言最贵者也;法者,事最适者也。言无二贵,法不两适,故言行而不轨于法令者必禁。若其无法令而可以接诈、应变、生利、揣事者,上必采其言而责其实。言当,则有大利;不当,则有重罪。是以愚者畏罪而不敢言,智者无以讼。此所以无辩之故也。乱世则不然:主有令,而民以文学非之;官府有法,民以私行矫之。人主顾渐其法令而尊学者之智行,此世之所以多文学也。}

由此,一切言论行为皆统一于法令,统一于人主之意志。
如此则辩议不生,人主之权威大立,国家得治。

\newline

总之,韩非子强调人主之权利与权威,重视人主对臣民的控制,并以此为治理国家之根本。
《韩非子·诡使》云:

\textit{今下而听其上,上之所争也。}
  
\section{驭下之道}
韩非重视人主对臣下的控制,亦有一套理论和实践方法使统治者驾驭臣民。
韩非子论驾驭臣民之原则,歪曲道家“无为而无不为”之论,谓人主需“无为”,作为置身事外的观察者,朗照臣民言行,遂生对臣民的支配力量。
此系将道家之观照智慧歪曲以用于阴谋,以加强君权、控制臣民。

\section{驭下之术}
韩非论驾驭臣民,不仅有“无为”之原则,亦有实践方法。
此为韩非阴谋之具体操作步骤,现论述之。

\newline

《韩非子·外储说右上》载:

\textit{君所以治臣者有三:一,势不足以化则除之。······二,人主者,利害之轺毂也,射者众,故人主共矣。是以好恶见则下有因,而人主惑矣;辞言通则臣难言,而主不神矣。······三,术之不行,有故。不杀其狗则酒酸。}

此直言君王驭臣之数。第一,臣子难以控制,那就除掉。
原文后以师旷、晏子谏齐景公之故事加以说明。
盖二人谏景公施恩惠,以同公子尾、公子夏、田成氏争夺民众之拥护。
韩非子讥之曰:

\textit{景公不知用势之主也,而师旷、晏子不知除患之臣也。}
      
第二, 人主应使臣民不能测见己意。
盖臣子若能揣测到人主之意,便易于玩弄君王于股掌,有害于人主之权威。
第三,人主要避免被臣子所蒙蔽、挟持。
“不杀其狗则酒酸”对应原文的寓言,言有卖酒者因其狗凶恶,众人不敢靠近购买,直至酒酸也未能售出。

\newline

此外,韩非子尚有二柄之说,供人主驭下。
《韩非子·二柄》谓:

\textit{明主之所导制其臣者,二柄而已矣。二柄者,刑德也。何谓刑德?曰:杀戮之谓刑,庆赏之谓德。为人臣者畏诛罚而利庆赏,故人主自用其刑德,则群臣畏其威而归其利矣。}

所谓刑德,即罚赏之权。
此为人主意志得以落实的途径,为权力之实质所在。
人主必须牢牢抓住赏罚大权,方可保持对臣民的控制。
若失去赏罚之柄,便失去人主之权。
故《二柄》又谓:

\textit{夫虎之所以能服狗者,爪牙也。使虎释其爪牙而使狗用之,则虎反服于狗矣。人主者,以刑德制臣者也。今君人者释其刑德而使臣用之,则君反制于臣矣。}

在此推荐一下电视剧《雍正王朝》。
这个连续剧几乎每一集都有韩非驭人术的教科书级示范。
稍微看几集,则韩非子的理论可明白十之八九。
人多谓《雍正王朝》为职场厚黑学之教科书,殊不知此种法家精神内核带来的是文化的幻灭和体制的僵化。
当君王的权威比正确错误更加重要的时候,这个国家是不会有未来的,即使它可以凭借君王个人的雄才大略兴盛一时。

\newline
  
\section{贤势之辨}
最后一个问题,韩非强调人主对臣下的控制,并提出诸多阴谋。
欲执行此类阴谋,按理说对君王之才能要求颇高。
由此引出韩非所论之“势”与“贤”。
所谓“势”,即外在条件或机遇;所谓“贤”,即个人之才能。
韩非所欲论者,即于人主而言,二者究竟哪个更重要。

\newline
  
为此,《韩非子·难势》中反复辩驳,先引慎子之言,谓势重于贤:

\textit{慎子曰:飞龙乘云,腾蛇游雾,云罢雾霁,而龙蛇与蚓蚁同矣,则失其所乘也。贤人而诎于不肖者,则权轻位卑也;不肖而能服于贤者,则权重位尊也。尧为匹夫,不能治三人;而桀为天子,能乱天下:吾以此知势位之足恃而贤智之不足慕也。}

又反驳之,言必贤而后方能乘势,否则虽有势,亦不能加以利用:

\textit{应慎子曰:飞龙乘云,腾蛇游雾,吾不以龙蛇为不托于云雾之势也。虽然,夫择贤而专任势,足以为治乎?则吾未得见也。夫有云雾之势而能乘游之者,龙蛇之材美之也;今云盛而蚓弗能乘也,雾而蚁不能游也,夫有盛云雾之势而不能乘游者,蚓蚁之材薄也。今桀、纣南面而王天下,以天子之威为之云雾,而天下不免乎大乱者,桀、纣之材薄也。}

而后又指出,势为中立性存在。
贤者以势求治,不肖者以势为乱。
而人之贤者少,不肖者多,故势之利用,其结果必然是不肖者以势为乱:

\textit{夫势者,非能必使贤者用之,而不肖者不用之也。贤者用之则天下治,不肖者用之则天下乱。人之情性,贤者寡而不肖者众,而以威势之利济乱世之不肖人,则是以势乱天下者多矣,以势治天下者寡矣。}

至此,似有贤重于势之结论,而下文又为势辩护:

\textit{复应之曰:其人以势为足恃以治官;客曰“必待贤乃治”,则不然矣。夫势者,名一而变无数者也。势必于自然,则无为言于势矣。吾所为言势者,言人之所设也。}

此处韩非强调,其所言之“势”乃人为之势,系人经主观努力所营造出的事物的发展趋势,而不是事物自然产生的趋势。
亦表达重势轻贤的观点。

\newline

几番纠结下来,韩非指出问题所在:

\textit{客曰:“人有鬻矛与盾者,誉其盾之坚,‘物莫能陷也',俄而又誉其矛曰:‘吾矛之利,物无不陷也。'人应之曰:‘以子之矛,陷子之盾,何如?'其人弗能应也。”以为不可陷之盾,与无不陷之矛,为名不可两立也。夫贤之为势不可禁,而势之为道也无不禁,以不可禁之势,此矛盾之说也。夫贤势之不相容亦明矣。}

此谓,说“贤”时,每以“贤”不受任何形势之限制;言“势”时,又以“势”可限制一切。
这两种观念互相矛盾,不能两立。
如此则“贤”与“势”究竟如何取舍?
韩非子最后提出,人以中等才能者为多,就一般情况,中材之人主治理国家,得势则治,失势则乱。
故就常态,求治需重势。其言曰:

\textit{且夫尧、舜、桀、纣千世而一出,是比肩随踵而生也。世之治者不绝于中,吾所以为言势者,中也。中者,上不及尧、舜,而下亦不为桀、纣。抱法处势则治,背法去势则乱。今废势背法而待尧、舜,尧、舜至乃治,是千世乱而一治也。抱法处势而待桀、纣,桀、纣至乃乱,是千世治而一乱也。}

总之,只要“抱法处势”,即能将权力和价值标准收归于人主,能有意识的营造和利用事物之发展趋势,只需中等才干的君王便可使国家得治,无需等待千世一出的圣王雄主来创造奇迹。
  
\end{document}