\documentclass[11pt]{article}

\usepackage[UTF8]{ctex} % for Chinese 

\usepackage{setspace}
\usepackage[colorlinks,linkcolor=blue,anchorcolor=red,citecolor=black]{hyperref}
\usepackage{lineno}
\usepackage{booktabs}
\usepackage{graphicx}
\usepackage{float}
\usepackage{floatrow}
\usepackage{subfigure}
\usepackage{caption}
\usepackage{subcaption}
\usepackage{geometry}
\usepackage{multirow}
\usepackage{longtable}
\usepackage{lscape}
\usepackage{booktabs}
\usepackage{natbib}
\usepackage{natbibspacing}
\usepackage[toc,page]{appendix}
\usepackage{makecell}

\title{董仲舒}
\date{}

\linespread{1.5}
\geometry{left=2cm,right=2cm,top=2cm,bottom=2cm}

\begin{document}

  \maketitle

  \linenumbers 
汉儒喜言阴阳五行,以宇宙论取代孔孟之心性论,实为哲学之堕落。
这其中又以董仲舒(179 BC-104 BC)最有代表性。
董氏之学,将宇宙运行和人类活动相对应,谓人事与天象相似,且应当相似。
由此,董氏遂以顺应天象为人类活动之最高价值标准,实将“天”立为超越主宰和价值根源,弗承孔孟心性论之精华,反坠原始信仰之阴霾。

\section{宇宙论}
董仲舒宇宙论的基本框架与阴阳家无异。
宇宙由天、地、阴、阳、木、火、土、金、水、人组成。
木、火、金、水分别与方向和季节相对应。
木对应东方和春季,火对应南方和夏季,金对应西方和秋季,水对应北方和冬季。
土则居中。
四季交替系阴阳运行之结果。
春季阳初盛,位于东方。
夏季阳达到全盛,位于南方。
根据物极必反之原则,秋季阴初盛而阳始衰。
此时阴位于东方。
冬季阴全盛,位于北方。
而后阳复初盛而阴始衰,四季复始。
此类古代宇宙论,看看就好。
写玄幻或者穿越小说的作家不妨多研究研究,应该能获得不少灵感。

\section{天人一体}
董仲舒进一步将宇宙和人相对应,以宇宙之运行诠释人类生活的方方面面,从人性到伦理道德,从政治活动到朝代更替,皆有涉及。
其基本逻辑是因为天如此,所以人亦如此。
此处需先解释“天”这一术语。
中国哲学中,“天”可指一君临宇宙之主宰,类似基督教之人格化上帝或亚里士多德哲学中的“第一推动者”;
又可指整个宇宙,即自然。
董氏论天,往往兼有这两种含义,但侧重可能不同。

\newline

董氏言天人一体,即人类活动在各个方面均与宇宙运行类似。
宇宙之运作由阴阳推动,人类行为亦由两种驱动力,即“性”与“情”。
性与阳对应,顺之则有仁德;
情与阴对应,顺之则有贪欲。
《春秋繁露·深察名号》云:

\textit{天两有阴阳之施,身亦两有贪仁之性。}

更进一步,社会伦理亦与阴阳对应。
《春秋繁露·基义》云:

\textit{君臣、父子、夫妇之义,皆与诸阴阳之道。君为阳,臣为阴;父为阳,子为阴;夫为阳,妻为阴。}

由此,董仲舒将阴阳与君臣、父子、夫妇等三种社会伦理关系相对应,谓之为“三纲”。
汉代其他儒者又将仁义礼智信五种品德与五行对应,即“五常”,谓仁为木,义为金,礼为火,智为水,信为土。

\newline

董仲舒的政治哲学也是建立在宇宙论的基础之上。
政府的责任在于帮助民众发挥其“性”。
《春秋繁露·深察名号》云:

\textit{天生民性有善质,而未能善,于是为之立王以善之,此天意也。}

更加具体的治理措施则为庆、赏、罚、刑四政,分别对应四季。
《春秋繁露·四时之副》载:

\textit{庆赏刑罚与春夏秋冬,以类相应也,如合符。故曰王者配天,谓其道。天有四时,王有四政,四政若四时,通类也,天人所共有也。}

统治方法与四季对应,政府的组织形式亦是如此。
一年分四季,故官员分四等。
每一季三个月,故每个官员手下有三个助手。
官员的考核分为四等,因为人的能力、品格天然分为四等。

\newline

更进一步,董仲舒以天解释朝代变更,认为朝代的更替依循“三统”,即黑统、白统、赤统的顺序循环往复。
例如,夏为黑统,殷为白统,周为赤统。
在三统循环的过程中,天赋予新君权力,令其建立新朝代并通过一系列举措承袭天命。
这些举措包括迁都、改国号、改纪元、改服色等。
然而,《春秋繁露·楚庄王》云:

\textit{若夫大纲、人伦、道德、政治、教化、习俗、文义,尽如故,亦何改哉?故王者有改制之名,无改制之实。}

可见,三统并无本质区别。

\section{顺天为义}
对董仲舒来说,天和人的密切相关不仅是客观事实,亦是价值论的依据。
依董仲舒,因为天是如此,所以人亦是如此,且应当如此。
天“任阳不任阴”。故人类活动亦应当重阳而轻阴。
就人性而言,“性”为阳而“情”为阴,故当重“性”而轻“情”,更进一步即为重仁德而轻贪欲。
就社会伦理而言,君阳臣阴,父阳子阴,夫阳妻阴,故君为臣纲,父为子纲,夫为妻纲。

\newline

四季的运行也是由于阴阳运转。
四季之中,春夏秋成生而冬丧死。
此是天“好德不好刑”之体现。
治理国家亦当遵守此天道,重德性教化而轻酷法严刑。
此为儒家之一贯主张,但董仲舒将其建立在宇宙论的基础之上,非孔孟之传。

\newline

总之,人类与天相对应,故人类活动应当遵循天道。
由此,董仲舒哲学的价值根源在于天而不是人的自觉。
人类活动应当遵守“任阳不任阴”、“好德不好刑”的天道,反之则会受到惩处。
这一点在政治活动中尤为重要。
为政而人事不臧,必招灾异,如地震、日食、洪水、干旱等。
对此,董仲舒提出两种解释。
第一种是目的论的,谓政府的过程导致天怒,故降灾殃。
此处天侧重主宰义。
另一种解释是机械论的,谓人作为天的一部分,其不正常活动势必影响到天的运行,造成灾异。
此处天侧重自然义。

\end{document}