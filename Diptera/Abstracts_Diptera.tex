\documentclass[11pt]{article}

% \usepackage[UTF8]{ctex} % for Chinese 

\usepackage{setspace}
\usepackage[colorlinks,linkcolor=blue,anchorcolor=red,citecolor=black]{hyperref}
\usepackage{lineno}
\usepackage{booktabs}
\usepackage{graphicx}
\usepackage{float}
\usepackage{floatrow}
\usepackage{subfigure}
\usepackage{caption}
\usepackage{subcaption}
\usepackage{geometry}
\usepackage{multirow}
\usepackage{longtable}
\usepackage{lscape}
\usepackage{booktabs}
\usepackage{natbibspacing}
\usepackage[toc,page]{appendix}
\usepackage{makecell}
\usepackage{amsfonts}
 \usepackage{amsmath}

\usepackage[backend=bibtex,style=authoryear,sorting=nyt,maxnames=1]{biblatex}
\bibliography{} # Reference bib

\title{}
\author{}
\date{}

\linespread{1.5}
\geometry{left=2cm,right=2cm,top=2cm,bottom=2cm}

\setlength\bibitemsep{0pt}
\setcitestyle{braket}

\begin{document}
\begin{sloppypar}
  \maketitle

  \linenumbers
\textbf{1. Lee \textit{et al.}, 2017, Microbiota, gut physiology, and insect immunity.}
\par
\textbf{2. Husink \textit{et al.}, 2020, Insect-symbiont gene expression in the midgut bacteriocytes of a blood-sucking parasite.}

\par

\textbf{Waterhouse \textit{et al.}, 2007, Immune-related genes and pathways in disease-vector mosquitoes.} \newline
Immunity-related genes:\newline 
285 \textit{Drosophila melanogaster} (Dm), 338 \textit{Anopheles gambiae} (Ag), and 353 \textit{Aedes aegypti} (Aa) genes from 31 gene families and functional groups implicated in classical innate immunity or defense functions such as apoptosis and response to oxidative stress. \newline
Orthology groups (whole genome): \newline
4951 orthologous trios (1:1:1 orthologs in the three species) and 886 mosquito-specific orthologous pairs (absent from Dm). \newline
Orthology groups (immune-related):\newline
91 trios and 57 pairs, plus a combined total of 589 paralogous genes in the three species. \newline
Immune-related orthology trios are more divergent than that of whole genome: \newline
Phylogenetic distances of genes in each trio is measured by amino acid substitutions. 
With Dm as reference, immune-related trios of Ag and Am are more divergent (on average) compared with trios of whole genome, and several Ag immunity genes are considerably more divergent than their Aa orthologs.\newline
Large variation exists in different immune families in their proportions of orthologous trios,
mosquito-specific pairs and species-specific genes: \newline
(1) Predominantly trio orthologs: 
apoptosis inhibitors (IAPs), 
oxidative defense enzymes [
    superoxide dismutases (SODs), 
    glutathione peroxidases (GPXs), 
    thioredoxin peroxidases (TPXs), 
    heme-containing peroxidases (HPXs)],
class A and B scavenger receptors (SCRs). \newline
(2) Rarely trio orthologs: 
immune effectors, including three antimicrobial peptides.\newline
(3) Intermediately: 
C-type lectins.\newline
Strong divergent evolution of immune recognition genes: \newline
Fruit fly and mosquito recognition proteins mostly form distinct clades within each gene family.
\par
\textbf{Bosco-Drayon \textit{et al.}, 2012, Peptidoglycan sensing by the receptor PGRP-LE in the \textit{Drosophila} gut induces immune responses to infectious bacteria and tolerance to microbiota.} \newline
In \textit{Drosophila}, peptidoglycan recognition protein (PGRP)-LE senses peptidoglycan, and induces NF-kappaB dependent responses to infectious bacteria, but also tolerance to symbionts via up-regulation of pirk and PGRP-LB, which inhibits IMD signaling. 
Loss of PGRP-LE-mediated detection of bacteria in the gut results in systemic immune activation, which can be rescued by overexpressing PGRP-LB in the gut.
\par
\textbf{Wang \textit{et al.}, 2009, Interactions between mutualist \textit{Wigglesworthia} and tsetse peptidoglycan recognition protein (PGRP-LB) influence trypanosome transmission.} \newline
Tsetse flies have coevolved with mutualistic endosymbiont \textit{Wigglesworthia glossinidiae}. 
A tsetse peptidoglycan recognition protein (PGRP-LB) is crucial for symbiotic tolerance and trypanosome infection processes. 
Tsetse \textit{pgrp-lb} is expressed in the \textit{Wigglesworthia}-harboring organ (bacteriome) in the midgut, and its level of expression correlates with symbiont numbers. 
Adult tsetse cured of \textit{Wigglesworthia} infections have significantly lower \textit{pgrp-lb} levels than corresponding normal adults. 
RNA interference (RNAi)-mediated depletion of \textit{pgrp-lb} results in the activation of the immune deficiency (IMD) signaling pathway and leads to the synthesis of antimicrobial peptides (AMPs), which decrease \textit{Wigglesworthia} density. 
Depletion of \textit{pgrp-lb} also increases the host's susceptibility to trypanosome infections. 
Finally, parasitized adults have significantly lower \textitP{pgrp-lb} levels than flies, which have successfully eliminated trypanosome infections. 
When both PGRP-LB and IMD immunity pathway functions are blocked, flies become unusually susceptible to parasitism. 
Based on the presence of conserved amidase domains, tsetse PGRP-LB may scavenge the peptidoglycan (PGN) released by \textit{Wigglesworthia} and prevent the activation of symbiont-damaging host immune responses. 
In addition, tsetse PGRP-LB may have an anti-protozoal activity that confers parasite resistance. 
\par
\textbf{Martinez \textit{et al.}, 2016, Addicted? Reduced host resistance in populations with defensive symbionts.} \newline
Heritable symbionts that protect their hosts from pathogens have been described in a wide range of insect species. 
By reducing the incidence or severity of infection, these symbionts have the potential to reduce the strength of selection on genes in the insect genome that increase resistance. 
Therefore, the presence of such symbionts may slow down the evolution of resistance. 
Here we investigated this idea by exposing \textit{Drosophila melanogaster} populations to infection with the pathogenic Drosophila C virus (DCV) in the presence or absence of \textit{Wolbachia}, a heritable symbiont of arthropods that confers protection against viruses. 
After nine generations of selection, we found that resistance to DCV had increased in all populations. 
However, in the presence of \textit{Wolbachia} the resistant allele of \textit{pastrel}—a gene that has a major effect on resistance to DCV—was at a lower frequency than in the symbiont-free populations. 
This finding suggests that defensive symbionts have the potential to hamper the evolution of insect resistance genes, potentially leading to a state of evolutionary addiction where the genetically susceptible insect host mostly relies on its symbiont to fight pathogens.  
\par
\textbf{You \textit{et al.}, 2014, Homeostasis between gut-associated microorganisms and the immune system in Drosophila.} \newline

\end{sloppypar}
\end{document}