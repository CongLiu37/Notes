\documentclass[11pt]{article}

% \usepackage[UTF8]{ctex} % for Chinese 

\usepackage{setspace}
\usepackage[colorlinks,linkcolor=blue,anchorcolor=red,citecolor=black]{hyperref}
\usepackage{lineno}
\usepackage{booktabs}
\usepackage{graphicx}
\usepackage{float}
\usepackage{floatrow}
\usepackage{subfigure}
\usepackage{caption}
\usepackage{subcaption}
\usepackage{geometry}
\usepackage{multirow}
\usepackage{longtable}
\usepackage{lscape}
\usepackage{booktabs}
\usepackage{natbib}
\usepackage{natbibspacing}
\usepackage[toc,page]{appendix}
\usepackage{makecell}
\usepackage{amsfonts}
 \usepackage{amsmath}

\title{Insect hemocytes and their role in immunity}
\author{}
\date{}

\linespread{1.5}
\geometry{left=2cm,right=2cm,top=2cm,bottom=2cm}

\begin{document}
  \maketitle

  \linenumbers
\section{Abbreviations}
AMP: antimicrobial peptide.\newline
PO: phenoloxidase.\newline
PPO1: proPO 1.\newline
JNK: Jun kinase.\newline
PSC: posterior signaling center.\newline
Srp: Serpent.\newline
JAK: Janus kinase.\newline
STAT: signal transducers and activators of transcription.\newline
gcm: glial cell missing.\newline
PRR: pattern recognition receptor.\newline
LPS: lipopolysaccharide.\newline
PGN: peptidoglycan.\newline
LPSBP: LPS-binding protein.\newline
GNBP: Gram-negative binding protein.\newline
PGRP: PGN recognition protein.\newline
GRP: glucan recognition protein.\newline
Dscam: Down's syndrome cell adhesion molecule.\newline
SR: scavenger receptor.\newline
SPZ: Spaetzle.\newline
NF-$\kappa$B: nuclear factor $\kappa$B.\newline

\section{Introduction}
The innate immune system of insects consists of humoral and cellular defense response. 
Humoral defenses refer to soluble molecules including antimicrobial peptides (AMPs), complement-like proteins and products from protealytic cascades such as phenoloxidase (PO) pathway. 
Cellular defenses refer to responses like phagocytosis, encapsulation and clotting that are directly mediated by hemocytes.

\section{Hemocyte types}
Hemocytes have similar function in immunity across insects, but naming of hemocyte types varies among taxa. 
\textit{Drosophila} larvae contain three terminally differentiated hemocyte types: plasmatocytes, crystal cells and lamellocytes. 
Plasmatocytes represent 90-95\% of mature hemocytes, are strongly adhesive \textit{in vitro}. and function as professional phagocytes that engulf pathogens and dead cells. 
Molecular markers for plasmatocytes include extracellular matrix protein peroxidasin and a surface factor P1 antigen. 
Crystal cells represent about 5\% of mature hemocytes. 
They are non-adhesive rounded cells that express PO cascade components such as proPO 1 (PPO1). 
Lamellocytes are absent in healthy \textit{Drosophila} larvae, but rapidly differentiated from prohemocytes after being attacked by parasitoid wasps and during metamorphosis. 
They are large, flat, adhesive cells that express reporters related to Jun kinase (JNK) signaling and L1 antigen. 
The main function of lamellocytes is encapsulation of parasitoids and other large foreign targets. 
Each of these hemocyte types differentiate from precursor prohemocytes that originate from pre-prohemocytes, which mainly reside in hematopoietic organs, and a small number in circulation.

\newline

In Lepidoptera, main differentiated hemocytes in circulation are granulocytes, plasmatocyte, spherule cells and oenocytoids. 
Granulocytes are the most abundant and characterized by the granules in their cytoplasm, the ability to adhere and spread on foreign surface in primary culture, and the tendency to spread systemically. 
They function as professional phagocytes. 
Plasmatocytes are usually larger than granulocytes, spread asymmetrically on foreign surfaces, and are the main capsule-forming hemocytes. 
Non-adhesive hemocytes in larval stage Lepidoptera include oenocytoids that contain PO cascade components, and spherule cells that are potential sources of cuticular components. 

\newline

In mosquitoes, hemocyte types include granulocytes, oenocytoids and prohemocytes. 
Granulocytes are strongly adhesive, phagocytic, and the most abundant cell types. 
They express PO activity induced by immune challenge. 
Oenocytoids are non-adhesive and constitutively express PO activity. 
Prohemocytes are characterized by uniform size, rounded morphology and large nuclear. 
It is unknown whether they differentiate into granulocytes/oenocytoids.

\section{Hematopoiesis}
Hemocytes arise during two stages of development. 
The first population of hemocytes arises during embryogenesis from head or dorsal mesoderm, and the second is produced during the larval or nymphal stages in mesodermally derived hematopoietic organs. 
The hematopoietic organs of \textit{Drosophila} are lymph glands that form bilaterally along the anterior part of the dorsal vessel during embryogenesis. 
By the third instar, each lymph gland consists of an anterior primary lobe and several posterior secondary lobes separated by pericardial cells. 
The primary lobe has three zones: 
(1) a posterior signaling center (PSC) that contains cells marked by the expression of transcription factor Collier and Notch ligand Serrate; 
(2) a medullary zone that contains quiescent prohemocytes; 
(3) a cortical zone that contains plasmatocytes, crystal cells and following parasitoid attack, lamellocytes. 
Secondary lobes contain pre-prohemocytes, prohemocytes and some plasmatocytes. 

\newline

Earliest lymph gland cells, hemocyte precursor cells, are identified by expression of GATA transcription factor homolog Serpent (Srp). 
As transition to pre-prohemocytes, they initiate expression of receptor tyrosine kinase Pvr followed by expressing JAK/STAT (Janus kinase/signal transducers and activators of transcription) signaling pathway receptor Dome, which characterizes maturation of prohemocytes. 
In differentiation of prohemocytes into hemocyte types, Dome is down-regulated. 
Specification of plasmatocytes requires expression of transcription factor glial cell missing (gcm) and gcm2, while crystal cell specification requires Runt-domain protein Lozenge (Lz) and Serrate signaling through Notch. 
The PSC along with JAK/STAT and JNK signaling have been implicated in differentiation of lamellocytes.

\newline

The maintenance of hemocytes in circulation involves two aspects: production and release of cells from lymph glands, and proliferation of hemocytes already in circulation. 
Furthermore, the number of circulating hemocytes increases rapidly in response to stress, wounding or infection. 

\section{Hemocyte-mediated defense responses}
Immune responses mediated by hemocytes are phagocytosis, encapsulation and clotting. 
Phagocytosis is a conserved defense response in which individual cells internalize and destroy targets. 
It depends on receptor-mediated recognition and binding of the target to a hemocyte followed by formation of a phagosome and engulfment of the target via actin polymerization-dependent mechanisms. 
The phagosome then matures to a phagolysosome by a series of fissin and fusion events with endosomes and lysosomes. 
Insect hemocytes phagocytize bacteria, yeast, fungi, protozoans, apoptotic bodies and inanimate materials like synthetic beads and ink particles. 

\newline

Encapsulation refers to the envelopment of large targets by multiple hemocytes. 
In \textit{Drosophila}, the capsules formed around invaders are mainly comprised of lamellocytes. 
In Lepidoptera, formation of capsules is mainly conducted by plasmatocytes, while cooperation of granulocytes are sometimes required for recognition and encapsulation of targets. 
Besides, melanin is often deposited within and around the capsules.

\newline

Coagulation of insect hemocytes occurs at sites of external wounding. 
Soft clots initially consist of fibrous matrix embedded with hemocytes, mainly granulocytes (Lepidoptera) or plasmatocytes (\textit{Drosophila}). 
This is followed by clot hardening due to cross-linking of proteins and melanization.

\section{Receptors and signaling pathways mediating hemocyte function}
Defense responses including phagocytosis and encapsulation are dependent on recognition of targets as foreign, followed by activation of downstream signaling and effector responses. 
Some foreign invaders are recognized by humoral pattern recognition receptors (PRRs), which bind to targets to enhance recognition by other receptors on hemocyte surface. 
This process is opsonization. 
Other targets are recognized directly by hemocyte surface receptors.

\subsection{Opsonin-dependent and -independent recognition}
Humoral PRRs can opsonize microorganisms by binding to lipopolysaccharides (LPSs), peptidoglycans (PGNs) and glucans. 
These include 
hemolin, 
LPS-binding proteins (LPSBPs), 
Gram-negative binding protein (GNBPs), 
soluble PGN recognition proteins (PGRP-SA and PGRP-SD), 
glucan recognition proteins (GRPs), 
soluble Down's syndrome cell adhesion molecule (Dscam) 
and complement-like TEP proteins. 
Another group of PGRPs (PGRP-SB1, -SC1a, -SC1b, -SC2) enzymatically degrade PGN. 
This activity kill some bacteria and releases PGN fragments triggering hemocyte effector responses. 
Other humoral molecules implicated in pathogen recognition and opsonization include 
leucine-rich repeat proteins, 
glutamine-rich protein 
and immunolectins. 
The sources of humoral PRRs include hemocytes and other immune tissues, \textit{e.g.} the fat body. 

\newline

Cell surface receptors involved in opsonin-independent immunity include 
Peste, a class B scavenger receptor (SR) (or CD36 family member); 
dSR-CI, a class C SR; 
transmembrane protein Eater; 
membrane bound PGRPs (PGRP-LC and its co-receptor PGRP-LE); 
transmembrane form of Dscam; 
class B SR Croquemort; 
low-density lipoprotein (LDL) receptor-related protein LRP1. 
A long version of PGRP-LE can act as intracellular receptor recognizing bacteria. 
Other proteins implicated to be cellular receptors include 
integrins, 
tetraspanin proteins, 
neuroglian (an immunoglobulin superfamily member).

\subsection{Cytokines and signaling pathways}
Cytokines are extracellular molecules that regulate hemocyte function. 
These include cysteine-knot-like growth factor Spaetzle (SPZ) that is activated by a protealytic cascade and interacts with Toll receptors located on cell membrane. 
This leads to activation of nuclear factor $\kappa$B (NF-$\kappa$B) transcription factors, which initiate a number of immune genes including several AMPs. 
Upstream PRRs involved in initiating protealytic cascade that lead to SPZ activation include PGRP-SA and soluble GNRPs. 
Cytokine PSP is also processed from a precursor protein by a protealytic cascade. 
After binding to its membrane receptor, PSP simulates plasmatocytes to adhere and spread on foreign surfaces.

\newline

In addition to Toll signaling, other pathways also are also activated in hemocytes by cytokine and/or binding of foreign to surface receptors. 
These include Imd pathway activated by PGRP-LC binding with Gram-negative bacteria. 
Imd signaling induces expression of immune effector genes. 
TEP proteins are involved in activation of JAK/STAT signaling, while JNK signaling is associated with phagocytosis and adhesion. 

\end{document}