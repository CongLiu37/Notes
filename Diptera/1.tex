\documentclass[11pt]{article}

\usepackage[UTF8]{ctex} % for Chinese 

\usepackage{setspace}
\usepackage[colorlinks,linkcolor=blue,anchorcolor=red,citecolor=black]{hyperref}
\usepackage{lineno}
\usepackage{booktabs}
\usepackage{graphicx}
\usepackage{float}
\usepackage{floatrow}
\usepackage{subfigure}
\usepackage{caption}
\usepackage{subcaption}
\usepackage{geometry}
\usepackage{multirow}
\usepackage{longtable}
\usepackage{lscape}
\usepackage{booktabs}
\usepackage{natbib}
\usepackage{natbibspacing}
\usepackage[toc,page]{appendix}
\usepackage{makecell}
\usepackage{amsfonts}
 \usepackage{amsmath}

\title{Insect and vertebrate immunity: key similarities versus differences}
\author{}
\date{}

\linespread{1.5}
\geometry{left=2cm,right=2cm,top=2cm,bottom=2cm}

\begin{document}
  \maketitle

  \linenumbers
\section{Abbreviations}
PAMP: pathogen-associated molecular pattern.\newline
PRR: pattern-recognition receptors.\newline
LPS: lipopolysaccharide.\newline
PGN: peptidoglycan.\newline
LTA: lipoteichoic acid.\newline

\section{Similarities}
\subsection{Sensing mechanims}
\subsubsection{Recognition of pathogen-associated molecular patterns}
Distinction between self and non-self relies on pattern-recognition receptors (PRRs) that bind to diagnostic sites for potential pathogens, or pathogen-associated molecular patterns (PAMPs). 
One precondition for sensing non-self by PAMP recognition is that these molecular patterns are conserved enough to allow the host to evolve binding proteins before the pathogen is able to eliminate or modify the target site. 
Common PAMPs include bacterial lipopolysaccharide (LPS), peptidoglycan (PGN), lipoteichoic acid (LTA) and fungal beta-1, 3-glucans.

\subsubsection{Extracellular sensor particles}
Extracellular lipid particles are involved in systemic immune response to pathogens. 
Apolipoprotein III is sensitive to both particle lipid composition and immune elicitors. 
Moreover, lipid particles are associated with typical immune proteins including prophenoloxidase and its upstream proteins, such as LPS- and PGN-binding proteins. 

\subsubsection{Recognition of self (histocompatibility and self-incompatibility) and altered-self (apoptotic and tumor cells)}

\subsection{Effector mechanisms}
\subsubsection{Antimicrobial peptide response}
Antimicrobial peptides are defense molecules against microbes by permeation and disruption of target membranes. 
They often kill microorganism via non-receptor-mediated mechanisms, although some bind to bacterial cell wall components (\textit{e.g.} nisin Z). 

\subsubsection{Phagocytosis (clearance of damaging objects)}
Phagocytosis is the cellular uptake of particular substrate. 
It is a fundamental cellular process in eukaryotes and essential for the clearance of damaging objects in multicellular organisms. 
In many animals, specialized cells engage in phagocytosis, such as phagocytes in vertebrates and macrophage-like hemocytes in insects.

\subsubsection{Endocytosis}

\section{Differences}
\subsection{Adaptive immune system in higher vertebrates and immunological memory involving clonally selected antibody-producing cells}
Adaptive immunity of vertebrates is fundamentally different from innate immunity. 
In adaptive immunity, the anticipatory nature of antibody repertoires is capable of binding epitopes never encountered by the organism or its predecessors using direct antibody-epitope specific binding. 
Self-recognizing antibody-producing cells are removed by clonal selection during ontogeny. 
The specific propagation of antibody-producing immune cells provides the basis for an immunological memory. 
Instead, in innate immunity, PRRs are acquired through evolutionary processes resulting from exposure to pathogens over generations. 
Retaining pathogen-binding proteins and removing self-recognizing proteins are facilitated at population level. 

\subsection{Inducible tolerance and memory in invertebrates}
Although lack of adaptive immunity, insects are able to induce immune activity after sub-lethal encounters with pathogens. 
Exposure to sub-lethal concentration of damaging objects enables latter survival under lethal level. 
This immune induction and protection comes with fitness cost, which is often expressed as a delay in development. 
Moreover, the induction of immune defense can be maternally transmitted to subsequent generations, occurring by potential epigenetic mechanisms or the incorporation of female-derived immune-inducible material into oocytes.

\end{document}