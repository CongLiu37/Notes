\documentclass[11pt]{article}

% \usepackage[UTF8]{ctex} % for Chinese 

\usepackage{setspace}
\usepackage[colorlinks,linkcolor=blue,anchorcolor=red,citecolor=black]{hyperref}
\usepackage{lineno}
\usepackage{booktabs}
\usepackage{graphicx}
\usepackage{float}
\usepackage{floatrow}
\usepackage{subfigure}
\usepackage{caption}
\usepackage{subcaption}
\usepackage{geometry}
\usepackage{multirow}
\usepackage{longtable}
\usepackage{lscape}
\usepackage{booktabs}
\usepackage{natbib}
\usepackage{natbibspacing}
\usepackage[toc,page]{appendix}
\usepackage{makecell}
\usepackage{amsfonts}
 \usepackage{amsmath}

\title{Wounding-mediated gene expression and accelerated viviparous reproduction of the pea aphid \textit{Acyrthosiphon pisum}}
\author{}
\date{}

\linespread{1.5}
\geometry{left=2cm,right=2cm,top=2cm,bottom=2cm}

\begin{document}
  \maketitle

  \linenumbers
Piercing of the pea aphid \textit{Acyrthosiphon pisum} with a bacteria-contaminated needle elicits lysozyme-like activity in the haemolymph but no detectable activities against live bacteria. 
No homologues of known antimicrobial peptides were found in cDNA library generated by using the suppression subtractive hybridization method or in over 90 000 public expressed sequence tag (EST) sequences, but lysozyme genes have recently been described in pea aphid. 
Production of viviparous offspring was significantly accelerated upon wounding.

\newline

Pea aphid showed weakened immune system. 
No homologues of known antimicrobial peptides were found. 
The presence of insect defensins in other Hemiptera and in the basal apterygote insect \textit{Thermobia domestica} (Altincicek & Vilcinskas, 2007) suggests that at least this type of antimicrobial peptides may have been lost during aphid evolution. 
Interestingly, the observation that pierced aphids showed a limited capacity to seal their wound by haemolymph coagulation and melanization agrees with the finding that an encapsulation response of pea aphid to the parasitoid wasp \textit{Aphidius ervi} is either very weak or non-existent (Oliver et al., 2005).

\newline

Regards to weakened immunity of pea aphids: 
(1) Aphids and relatives of Hemiptera share the unique ability to exploit exclusively phloem sap as diet, which is usually sterile (Douglas, 2006). Thus, the risk of encountering pathogens in their diet is limited. 
(2) Aphids harbour primary symbionts that are vertically transmitted and located intracellularly, as well as secondary symbionts that are both vertically and horizontally transmitted and also survive extracellularly in the insect haemolymph where they face the host's antimicrobial defences (Moran & Dunbar, 2006; Haine, 2008). 
It is possible that the symbionts provide protection, \textit{e.g.} pea aphid has been reported to be protected against fungal pathogens by the facultative symbiotic Gram-negative bacterium \textit{Regiella insecticola} (Scarborough et al., 2005) and also against the parasitoid wasp \textit{Aphidius ervi} by the facultative symbiotic Gram-negative bacterium \textit{Hamiltonella defensa} (Oliver et al., 2005). 
This may further explain why only lysozyme-like activity is present in the haemolymph, as lysozymes target mainly Gram-positive bacteria, whereas aphid symbionts belong to Gram-negative bacteria. 
(3) As immune responses are costly because they require investment of resources which are shared with other fitness-relevant traits (Rolff & Siva-Jothy, 2003; Schmidt-Hempel, 2005; Freitak et al., 2007), it is reasonable that aphids increase terminal reproductive investment in response to a putative survival threat such as an immune challenge. 

\end{document}