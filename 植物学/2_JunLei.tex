\documentclass[11pt]{article}

\usepackage[UTF8]{ctex} % for Chinese 

\usepackage{setspace}
\usepackage[colorlinks,linkcolor=blue,anchorcolor=red,citecolor=black]{hyperref}
\usepackage{lineno}
\usepackage{booktabs}
\usepackage{graphicx}
\usepackage{float}
\usepackage{floatrow}
\usepackage{subfigure}
\usepackage{caption}
\usepackage{subcaption}
\usepackage{geometry}
\usepackage{multirow}
\usepackage{longtable}
\usepackage{lscape}
\usepackage{booktabs}
\usepackage{natbib}
\usepackage{natbibspacing}
\usepackage[toc,page]{appendix}
\usepackage{makecell}
\usepackage{amsfonts}
 \usepackage{amsmath}

\title{菌类(Fungi)}
\date{}

\linespread{1.5}
\geometry{left=2cm,right=2cm,top=2cm,bottom=2cm}

\begin{document}

  \maketitle

  \linenumbers
菌类真核,异养,孢子繁殖,多由丝状多细胞结构构成。

\section{黏菌门(Myxomycota)}
在生长期为无细胞壁的原生质团,多核,类似变形虫,称为变形体(plasmodium),多外被胶质鞘。
子实体为产孢子组织,外有包被。
多腐生,少数寄生植物。

\subsection{发网菌目(Stemonitales)}
孢子紫褐色。
原生质团细小,质地均一,而后伸长分枝为网状,无胶质鞘。


\subsubsection{发网菌(\textit{Stemonitis})}
变形体为不规则网状。
无性生殖时,变形体生发状突起。
每个突起发育为具柄孢子囊,呈长桶形,紫色。
孢子囊内有胞丝(capillitium)。
胞子核由原生质体核减数分裂形成。
孢子有细胞壁,后萌发为有两条不等长鞭毛的游动细胞。
游动细胞鞭毛收缩,形成变形菌胞。

\par

有性生殖时,单倍体孢子发育而来的游动细胞或变形菌胞两两结合为合子。
合子核有丝分裂为原生质团。

\section{根肿菌门(Plasmodiophoromycota)}
仅根肿菌科(Plasmodiophoraceae)一科。
胞内寄生高等植物、藻类或真菌。
休眠孢子在寄主细胞内游离。
常引起寄主细胞过度增大、分裂,造成组织增生。
如芸薹根肿菌(\textit{Plasmodiophora brassicae})。

\section{卵菌门(Oomycota)}
多水生,腐生或寄生。
细胞壁不含几丁质(chitin)。
无性生殖产游动孢子,有等长双鞭毛。
营养体细胞为二倍体,减数分裂形成配子。
有性生殖产厚壁卵孢子。
仅卵菌纲(Oomycetes)。

\subsection{水霉目(Saprolegniales)}
多腐生。
孢子囊层出。
藏卵器内有卵一至多枚。

\subsubsection{水霉(\textit{Saprolegnia})}
常寄生于淡水鱼或淡水动物尸体。
菌丝体白色,绒毛状,细长分枝。
以根状菌丝插入寄主组织。

\par

无性生殖时,菌丝顶部膨大,细胞核进入顶端,生隔形成长桶形游动孢子囊。
孢子囊顶端开口,释放游动孢子。
旧孢子囊基部生第二个孢子囊,伸入旧孢子囊空壳,即孢子囊层出。
游动孢子顶生两条鞭毛,条件适宜时直接萌发。
条件不适时,游动孢子鞭毛消失,形成球形静孢子。
静孢子再生出一条鞭毛,形成次生孢子。
次生孢子鞭毛消失,形成静孢子后萌发。

\par

有性生殖时,菌丝顶端形成精囊和卵囊。
精囊紧靠卵囊,生出丝状输精管。
精核经输精管于卵核结合,形成卵孢子。
卵孢子休眠后萌发,
形成菌丝体。

\subsection{霜霉目(Peronosporales)}
多寄生于维管植物。

\subsubsection{白锈菌(\textit{Peronosporales})}
高等维管植物专性寄生菌。
无性生殖时,菌丝在寄主表皮下长出短、不分枝、密集成排的棍棒状孢囊梗。
自孢囊梗顶端到基部,多核孢子囊不断形成,造成宿主表皮突起、破裂,释放孢子囊。
孢子囊萌发,产生游动胞子,萌发后侵入寄主。

\par

有性生殖时,菌丝顶端生横壁,形成精囊和卵囊。
精核由受精管进入卵囊,形成卵孢子。
卵孢子壁上有纹饰,萌发形成游动孢子,侵入寄主。

\subsubsection{霜霉(\textit{Peronospora})}
高等维管植物专性寄生菌。
常侵染叶。
菌丝体无色,在寄主细胞间隙生长。
无性生殖为主。
孢囊梗自寄主气孔伸出分叉。

\section{真菌(True fungi)}
除少数单细胞品种外,多为分枝丝状体,即菌丝(hyphae)。
菌丝集合为菌丝体(mycedium)。
菌丝有隔或无隔。
隔上有小孔。
细胞壁含几丁质。

\par

生活史的某个阶段,菌丝体交织形成结构紧密的菌组织,包括拟薄壁组织(pseudoparenchyma)和疏丝组织(prosenchyma)。
拟薄壁组织由紧密排列的菌丝细胞组成。
疏丝组织菌丝细胞长,多平行排列或互相交错,结构疏松。

\par

菌组织形成不同结构的营养或繁殖结构,包括根状菌索(rhizomorph)、子座(stroma)、菌核(sclerotium)。
根状菌索外层颜色深,为拟薄壁组织组成的皮层;
内层为疏丝组织组成的髓层。
顶端有一生长点。
子座为容纳子实体的基座,是从营养阶段到繁殖阶段的过渡,亦由拟薄壁组织和疏丝组织组成。
菌核质地坚硬,颜色深,抗逆性强,为度过不良环境的休眠体。

\par

营养繁殖通过细胞分裂或菌丝断裂进行。
无性生殖产生孢子。
游动孢子(zoospore)在游动孢子囊(zoosporangium)内形成,有单鞭,无细胞壁;
孢囊胞子是在孢子囊(sporangium)内形成的不动孢子;
分生孢子(conidiospore)是菌丝或分生孢子梗从生长点出芽形成的芽殖型(blastic)孢子,或菌丝顶端生横隔后断裂形成的菌丝型(thallic)孢子。
有性生殖产生有性孢子,如接合孢子、子囊孢子、担孢子等。

\par

少数品种营准性生殖(parasexuality)。
异宗菌丝接触、融合、质配,使得同一细胞内有两个异宗核,称为异核体(heterokaryon)。
在生殖结构中,异核发生核配并减数分裂,形成孢子。
少数品种的非生殖结构中,异核偶尔能融合为杂合核。
杂合核有丝分裂时发生染色体交换、重组,产生新的杂合核、非整倍核或单倍核。
非整倍核不稳定,染色体随机分配,形成单倍体。

\subsection{壶菌门(Chytridiomycota)}
多单细胞品种,单细胞体兼司营养和繁殖,为整体产果式(holocarpic)。
少数有无隔分枝菌丝。
菌丝有营养和繁殖之分工。
繁殖时部分菌丝形成繁殖结构,通过隔与其他营养菌丝各开,为分体产果式(eucarpic)。
无性游动孢子有一后生尾鞭毛。
有性生殖为同配、异配或卵配,产休眠孢子或卵孢子。

\subsubsection{壶菌目(Chytridiales)}
单细胞。
游动孢子囊薄壁。
休眠孢子囊厚壁,为有性生殖形成的。

\par

节壶菌(\textit{Physoderma})寄生高等植物,不引起组织膨大。

\subsection{接合菌门(Zygomycota)}
菌丝无隔多核。
无性生殖产生不动的孢囊孢子。
有性生殖时配子囊接合,产生二倍体接合孢子。

\subsubsection{毛霉目(Mucorales)}
无性生殖产孢囊孢子或分生孢子。

\par

根霉(\textit{Rhizopus})腐生,常生于富淀粉基质。
菌丝体棉絮状,在基质表面铺开大量匍匐枝,有假根伸入基质。
无性生殖时,基部向上生出直立孢囊梗(sporangiophore),其顶端膨大为孢子囊。
孢子囊中央有半圆形囊轴(columella),基部有囊托。
孢子囊中原生质分裂成块,形成多核静孢子。
成熟后孢子囊破裂,释放孢囊孢子。
有性生殖时,两个不同宗的菌丝上生出有柄配子囊。
配子囊顶端接触,囊壁融解,细胞核融合为二倍体接合孢子。
接合孢子减数分裂后产生单倍体孢子并释放。

\par

毛霉(\textit{Mucor})类似根霉,但无匍匐枝,无假根,孢子囊基部无囊托。

\subsubsection{虫霉目(Entomophthorales)}
常寄生昆虫。
无性生殖产分生孢子。

\par

虫霉(\textit{Entomophthom})寄生昆虫或腐生于粪便。
菌丝多有隔,常自隔处断裂形成虫菌体。
无性生殖发达,产光滑分生孢子。
有性生殖产接合孢子,或营孤雌生殖。

\subsection{子囊菌门(Ascomycota)}
除酵母外均为多细胞。
菌丝有隔。
营养繁殖方式为出芽(酵母)或菌丝断裂(多细胞品种)。
无性生殖多产分生孢子。
有性生殖多同宗配合,形成子囊(ascus)。
合子在子囊内进行减数分裂。
单细胞品种子囊裸露。
多细胞品种菌丝形成疏丝组织和拟薄壁组织,形成子囊果(ascocarp),包裹子实体。

\par

有性生殖时,菌丝体顶端生出短小、直立、二叉分枝的菌丝,其顶端发育为精囊或卵囊。
卵囊又称产囊体(ascogonium)。
产囊体顶端生出受精丝(trichogyne),连通精囊。
质配后,产囊体上半部生产囊丝(ascogenous hypha)。
雌雄核成对进入产囊丝,而后产囊丝基部生隔,形成双核细胞。
产囊丝顶部伸长弯曲为产囊丝钩(crosier)。
双核分裂为四核。
产囊丝钩生横隔,形成钩尖、钩头、钩柄、三个细胞。
钩尖、钩柄单核,钩头双核。
钩头细胞为子囊母细胞,核配后减数分裂,再有丝分裂,形成一行八个子囊孢子。
钩尖、钩柄细胞相连通后两个核再分裂并形成新的钩尖、钩头、钩柄细胞,进而形成新一行子囊孢子。
如此多次,一根产囊丝上形成一丛平行排列的子囊。
产囊体下方生不育菌丝,在子囊间形成侧丝(paraphysis);
在子囊外侧交织为包被(peridium),形成子囊果外壳。
子囊果内,侧丝核子囊整齐排列成一层,即子实层。

\par

子囊果分为三种类型。
子囊盘(apothecium)盘状、杯状或碗状,子实层在盘中且暴露向外。
闭囊壳(cleistothecium)球形,完全闭合,破裂后释放孢子。
子囊壳(perithecium)成瓶状,顶端开口。

\subsubsection{半子囊菌纲(Hemiascomycetes)}
子囊裸露,无产囊丝和子囊果。

\par

酵母(\textit{Saccharomyces})为单细胞,成圆形或椭圆形。
出芽生殖。
有性生殖时两营养细胞结合为合子。
合子出芽,形成单细胞子囊。
子囊减数分裂形成子囊孢子。

\subsubsection{不整囊菌纲(Plectomycetes)}
闭囊壳子囊果。
子囊近球形,不规则散生,不形成子实层。
仅散囊菌目(Eurotiales)。

\par

曲霉(\textit{Aspergillus})菌丝体发达。
分生孢子梗多,顶端膨大为泡囊(vesicle)。
泡囊表面布满放射状排列的瓶形结构,依次为梗基、初生小梗、次生小梗。
小梗顶端形成一串球形分生孢子。
不同品种分生孢子颜色不一。
有性生殖少,闭囊壳子囊果。
无子实层,子囊不规则分散在子囊果中。
子囊壁常融解,在子囊果中释放子囊孢子。

\par

青霉(\textit{Penicillium})菌丝产生长而直立的无性分生孢子梗,其顶端数次分枝,成扫帚状。
最末小枝为小梗,上生一串绿色分生孢子。
鲜有有性生殖。

\subsubsection{核菌纲(Pyrenomycetes)}
子囊果为子囊壳和闭囊壳。
子囊棍状,排列整齐,顶端开裂。

\par

赤霉菌(\textit{Gibberella})子囊壳蓝色或紫色。
孢子梭形,有横隔。

\par

麦角菌(\textit{Claviceps})寄生于禾本科植物子房。
菌核萌发出子实体,称为子座。
子座有一长柄,头部膨大球形,其内生子囊壳。
子囊壳椭圆形,孔口突出于子座表面。
子囊壳内有长圆柱形子囊,子囊内有线状子囊孢子。
子囊孢子侵入寄主,发育为菌丝体。
菌丝体生出分生孢子梗并释放分生孢子后,菌丝体发育为菌核。

\par

白粉菌(\textit{Erysiphe})寄生于植物。
闭囊壳,内有子囊排列成子实层。
子囊果表面有柔软丝状结构。

\par

虫草(\textit{Cordyceps})
子座从昆虫宿主虫体发出,肉质,多为棒状,直立。

\subsubsection{盘菌纲(Pezizomycetes)}
子囊盘。
子囊顶端不加厚,排列成子实层,分布于子囊盘内表面。

\par

羊肚菌(\textit{Morchella})腐生。
子实体有菌盖和菌柄。
菌盖近球形或圆锥形,边缘全部和菌柄相连,表面有网状棱纹。
菌柄平整或有凹槽。
子实层分布于菌盖凹陷处。
子囊之间有长侧丝。

\par

盘菌(\textit{Peziza})子囊盘成盘状,菌柄不发达。
子囊圆柱状,有侧丝。
子囊孢子椭圆形,无色,在子囊内排列成一行。

\par

核盘菌(\textit{Sclerotinia})营寄生。
子囊盘有柄。
子囊棍棒状,顶端加厚,中央有孔。

\subsection{担子菌门(Basidiomycota)}
陆生高等真菌,多细胞,菌丝有隔。
单倍体担孢子萌发的菌丝初期无隔多核,后生横隔,形成单核菌丝,称初生菌丝(primary hyphae)。
初生菌丝的两个细胞质配为双核细胞,后分裂形成双核菌丝,称为次生菌丝体(secondary hyphae)。
次生菌丝特化为双核三生菌丝体(tertiary hyphae),形成担子果(basidiocarp)。

\par

次生菌丝体和三生菌丝体细胞分裂时产生锁状联合(clamp connection)。
细胞中央侧生出喙状突起,向下弯曲。
一细胞核移入突起基部,另一核在其附近。
两核分裂为四个子核,其中两个在细胞上部,一个在基部,一个在突起中。
而后生隔,母细胞分为三个子细胞。
上方子细胞双核,基部和喙状突起形成的子细胞单核。
喙突向下弯曲的部位连通基部子细胞,形成双核细胞。
两个双核子细胞之间残留喙突。

\par

有性生殖可通过初生菌丝结合、形成担孢子或形成受精丝,常不形成特殊结构。
初生菌丝结合时仅质配,不核配,形成双核菌丝。双核菌丝有两种方式发育为次生菌丝。
一是生出分枝,双核移入分枝后分裂为四核,而后细胞生隔,形成两个双核细胞。
如此不断分裂,形成双核菌丝。
二是双核菌丝先进行核分裂,子核移如亲配型相反的初生菌丝体,即“+”核移入“-”菌丝。
而后子核迅速分裂并在细胞间迁移,直到母菌丝全部由异源双核细胞组成。

\par

形成担孢子(basidiospore)时,双核菌丝顶端膨大形成担子(basidium)。
担子核配后减数分裂为四个单倍体核。
担子顶端生四个小梗,小梗顶端膨大,单倍体核进入,形成担孢子。
单倍体担孢子萌发形成初生菌丝,或两两结合后形成双核菌丝。

\par

受精丝为起雌性器官作用的营养菌丝。
性孢子和受精丝接触并质配,形成双核菌丝。

\par

亦营无性生殖,产生分生孢子或粉孢子。

\subsubsection{冬孢菌纲(Teliomycetes)}
多寄生高等植物。
不形成子实体。
产自冬孢子(teliospore)。

\par

黑粉菌目(Ustilaginales)担孢子顶生或侧生。
每个担子产生担孢子数目不定,孢子不强力散射。
产生煤黑色粉状孢子。
如玉米黑粉菌(\textit{Ustilago maydis})。

\par

锈菌目(Uredinales)多寄生。
初生菌丝产生性孢子(pycniospore),次生菌丝产春孢子(aeciospore)、夏孢子(urediospore)、冬孢子。
冬孢子经核配和减数分裂,萌发为有四个细胞的担子,每个细胞发育为一个担孢子,借弹力散射。
如禾柄锈菌(\textit{Puccin graminis})。

\subsubsection{层菌纲(Hymenomycetes)}
担子果发达,分为三类。
裸果型(gymnocarpous type)担子果开放,担子整齐排列形成的子实层始终裸露。
半被果型(hemiangiocarpous type)担子果发育初期有内、外菌幕包裹子实层;
而后菌幕破裂,子实层裸露。
假被果型(pseudoangiocarpous type)最初子实层裸露;
后菌盖边缘向内弯曲,封闭子实层;
最后子实层再次裸露。

\par

担子整齐排列为子实层,分布在菌髓两侧。
子实层和菌髓形成子实层体(hymenophore)。

\par

银耳目(Tremellales)腐生,裸果型。
担子果胶质、软骨质或膜质。
子实层生于担子果两侧。
担子球形,纵分为四个细胞,每个细胞顶端有一小梗,梗上生一担孢子。
如银耳(\textit{Tremella fuciformis})。

\par

木耳目(Auriculariales)担子果裸果型,胶质。
子实层分布于担子果表面,或大部分包埋于其内。
担子圆柱形,纵分为四个细胞,每个细胞顶端有一小梗,梗上生一担孢子。
如木耳(\textit{Auricularia auricula})。

\par

伞菌目(Agaricales)担子果多肉质,有伞状的菌盖(pileus),下方为菌柄(stipe)。
菌柄多中生,菌盖腹面有放射排列的菌褶(gills)。
菌褶内部为菌髓。
子实层位于菌褶两面,主要为担孢子和侧丝。
担子单细胞,棒状,上生四个小梗,梗顶端有一个担孢子,可弹射。
子实体幼时常有内菌幕(partial veil)遮盖菌褶。
菌盖展开,内菌幕破裂,在菌柄残留环状菌环(annulus)。
部分品种子实体幼时有外菌幕(universal veil),包围子实体。
菌柄伸长,外菌幕破裂,在菌柄基部残留菌托(volva),在菌盖顶部可能残留鳞片(scale)。
多腐生。
如蘑菇(\textit{Agaricus spp.})、香菇(\textit{Lentinus spp.})、口蘑(\textit{Tricholoma spp.})、鹅膏菌(\textit{Amanita spp.})。

\par

多孔菌目(Polyporales)担子果裸果型,可为多年生,木质或肉质。
子实层位于菌管内或菌刺上,无菌褶。
担子单细胞,棒状。
如灵芝(\textit{Ganoderma spp.})、猴头(\textit{Hericium spp.})。

\subsubsection{腹菌纲(Gasteromycetes)}
被果型,担子果发达,常生于地下,成熟时露出地面。
担子果外有多层包被(peridium),内为产孢体(gleba)。
产孢体多腔,担子沿腔边缘生出。

\par

鬼笔目(Phallales)担子果生于地下,卵形、圆形或梨形。
成熟时包被裂开,包托伸长,露出地面。
包被残留形成菌托。
产孢组织成熟时有黏性,恶臭。
担孢子卵圆,表面光滑。
如鬼笔(\textit{Phallus spp.})、竹荪(\textit{Dictyophora spp.})。

\par

马勃目(Lycoperdales)担子果球形或近球形,基部有白色根状菌索。
包被多层,可能不裂开。
担子果成熟后产孢组织分解,留下粉末状孢子。
担子球形,上生四至八个小梗,各生一担孢子。
担孢子球形,常有疣刺。
如马勃(\textit{Lycoperdon spp.})、秃马勃(\textit{Calvatia spp.})、地星(\textit{Geastrum spp.})。

\subsection{半知菌类(Fungi imperfecti)}
未发现有性生殖的真菌。
单倍体,菌丝有隔。
菌丝体发达。
营养繁殖为主。
无性生殖形成分生孢子。
亦营准性生殖。

\subsubsection{丛孢目(Moniliales)}
分生孢子梗分散丛生于基质表面。
如稻瘟病菌(\textit{Piriculaxia oryzae})。

\subsubsection{黑盘孢目(Melanconiales)}
分生孢子梗紧密排列于分生孢子盘(acervulus)上。
如棉花炭疫病菌(\textit{Colletotrichum gossypii})。

\subsubsection{球壳孢目(Sphaeropsidales)}
分生孢子梗位于分生孢子器(pycnidium)壁上。
如茄褐纹病菌(\textit{Phomopsis vexans})。

\end{document}